\documentclass[12pt]{extarticle}
\usepackage[utf8]{inputenc}
\usepackage[english]{babel}
\usepackage[a4paper, total={6in, 9in}]{geometry}
\usepackage{tikz-cd}
 
\usepackage{amsthm, amssymb, amsmath, centernot}

\newcommand{\notimplies}{%
  \mathrel{{\ooalign{\hidewidth$\not\phantom{=}$\hidewidth\cr$\implies$}}}}

\renewcommand\qedsymbol{$\square$}
\newcommand{\cont}{$\boxtimes$}
\newcommand{\divides}{\mid}
\newcommand{\ndivides}{\centernot \mid}
\newcommand{\Z}{\mathbb{Z}}
\newcommand{\N}{\mathbb{N}}
\newcommand{\C}{\mathbb{C}}
\newcommand{\Zplus}{\mathbb{Z}^{+}}
\newcommand{\Primes}{\mathbb{P}}
\newcommand{\ball}[2]{B_{#1} \! \left(#2 \right)}
\newcommand{\Q}{\mathbb{Q}}
\newcommand{\R}{\mathbb{R}}
\newcommand{\Rplus}{\mathbb{R}^+}
\newcommand{\invI}[2]{#1^{-1} \left( #2 \right)}
\newcommand{\End}[1]{\text{End}\left( A \right)}
\newcommand{\legsym}[2]{\left(\frac{#1}{#2} \right)}
\renewcommand{\mod}[3]{\: #1 \equiv #2 \: \mathrm{mod} \: #3 \:}
\newcommand{\nmod}[3]{\: #1 \centernot \equiv #2 \: \mathrm{mod} \: #3 \:}
\newcommand{\ndiv}{\hspace{-4pt}\not \divides \hspace{2pt}}
\newcommand{\finfield}[1]{\mathbb{F}_{#1}}
\newcommand{\finunits}[1]{\mathbb{F}_{#1}^{\times}}
\newcommand{\ord}[1]{\mathrm{ord}\! \left(#1 \right)}
\newcommand{\quadfield}[1]{\Q \small(\sqrt{#1} \small)}
\newcommand{\vspan}[1]{\mathrm{span}\! \left\{#1 \right\}}
\newcommand{\galgroup}[1]{Gal \small(#1 \small)}
\newcommand{\aut}[1]{\mathrm{Aut} \small(#1 \small)}
\newcommand{\ints}[1]{\mathcal{O}_{#1}}
\newcommand{\sm}{\! \setminus \!}
\newcommand{\norm}[3]{\mathrm{N}^{#1}_{#2}\left(#3\right)}
\newcommand{\qnorm}[2]{\mathrm{N}^{#1}_{\Q}\left(#2\right)}
\newcommand{\quadint}[3]{#1 + #2 \sqrt{#3}}
\newcommand{\varpideal}{\mathfrak{p}}
\newcommand{\inorm}[1]{\mathrm{N}(#1)}
\newcommand{\tr}[1]{\mathrm{Tr} \! \left(#1\right)}
\newcommand{\delt}{\frac{1 + \sqrt{d}}{2}}
\newcommand{\ch}[1]{\mathrm{char} \: #1}
\renewcommand{\Im}[1]{\mathrm{Im}(#1)}
\newcommand{\minimal}[2]{\mathrm{Min}(#1;#2)}
\newcommand{\fix}[2]{\mathrm{Fix}_{#1} (#2)}
\newcommand{\id}{\mathrm{id}}
\newcommand{\Disc}[1]{\mathrm{Disc}(#1)}
\newcommand{\sgn}[1]{\mathrm{sgn}(#1)}


\theoremstyle{definition}
\newtheorem{theorem}{Theorem}[section]
\newtheorem{lemma}[theorem]{Lemma}
\newtheorem{proposition}[theorem]{Proposition}
\newtheorem{corollary}[theorem]{Corollary}


\newenvironment{definition}[1][Definition:]{\begin{trivlist}
\item[\hskip \labelsep {\bfseries #1}]}{\end{trivlist}}


\newenvironment{lproof}{\begin{proof} \renewcommand{\qedsymbol}{}}{\end{proof}}


\begin{document}

\section{Localization}

\begin{definition}
Let $R$ be an integral domain, then $S \subset R$ is multiplicative if $\forall s, s' \in S : s s' \in S$ and $1 \in S$ but $0 \neq S$. 
\end{definition}

\begin{definition}
Let $R$ be an integral domain and the subset $S \subset R$ be multiplicative, then the \textit{localization} of $R$ at $S$, denoted by $S^{-1} R \subset Q_R$ is the ring,
\[ S^{-1} R = \left\{ \frac{r}{s} \; \middle| \; r \in R \text{ and } s \in S \right \} \]  
\end{definition}

\begin{definition}
A \textit{discrete valuation ring (DVR)} is a Dedekind domain with a unique maximal ideal. 
\end{definition}

\begin{lemma} \label{idealsurj}
The map $I \mapsto S^{-1} I$ is a surjection from ideals of $R$ to ideals of $S^{-1} R$. 
\end{lemma}

\begin{proof}
Let $D$ be the map from ideals of $R$ to ideals of $S^{-1}R$ given by $D : I \mapsto S^{-1} I$. Now if $J \subset S^{-1}R$ is an ideal then consider $R \cap J \subset R$. This is an ideal of $R$ because if $x,y \in R \cap J$ then $xy \in R$ and $xy \in J$ so $xy \in R \cap J$ and for $r \in R$, $r = \frac{r}{1} \in S^{-1} R$ so $rx \in J$ so $rx \in R \cap J$. \\ \\
Take $x \in D(R \cap J)$ then $x = \frac{r}{s}$ with $r \in J$ and since $\frac{1}{s} \in S^{-1}R$, by absorption, $\frac{r}{s} = x \in J$. Take $\frac{r}{s} \in J$ with $r \in R$ then $r = s \frac{r}{s} \in J$ by absorption so $r \in R \cap J$ thus $\frac{r}{s} \in D(R \cap J)$. Therefore, $D(R \cap J) = J$ so $D$ is surjective.  \\ \\
\end{proof}


\begin{lemma} \label{idealbij}
The map $\mathfrak{p} \mapsto S^{-1} \mathfrak{p}$ is a bijection between the prime ideals of $R$ which do not intersect $S$ and prime ideals of $S^{-1} R$. 
\end{lemma}

\begin{proof}
Restrict $D$ to the set of prime ideals of $R$ which do not intersect $S$. Let $P$ be a prime ideal of $R$ and $P \cap S = \varnothing$. Take $\frac{r_1}{s_1} \frac{r_2}{s_2} = \frac{r}{s} \in S^{-1} P$ for $r_1, r_2 \in P$. Then $r_1 r_2 s = s_1 s_2 r \in P$. $P$ is prime so either $r_1 \in P$ or $r_2 s \in P$. If $r_2 s \in P$ then $r_2 \in P$ because $s \notin P$. Therefore, $r_1 \in P$ or $r_2 \in P$ so $\frac{r_1}{s_1} \in S^{-1}P$ or $\frac{r_2}{s_2} \in S^{-1} P$ and therefore $S^{-1} R$ is prime. Thus, $\Im{D}$ is contained within the set of prime ideals of $S^{-1} R$. \\ \\
Let $P$ and $Q$ be prime ideals of $R$ s.t. $P \cap S = Q \cap S = \varnothing$. Then suppose that $D(P) = D(Q)$ i.e. $S^{-1}P = S^{-1}Q$. Then $\frac{p}{s_1} = \frac{q}{s_2}$ for any $p \in P$ and $q \in Q$. Thus, $s_2 p = s_1 q$ so $s_2 p \in Q$ and $s_1 q \in P$ by absorption. The ideals are prime so $p \in Q$ and $q \in P$ since $s_2 \notin Q$ and $s_1 \notin P$. Therefore, $P \subset Q$ and $P \supset Q$ so $P = Q$. Therefore, $D$ is injective. \\ \\
Let $J \in S^{-1}R$ be prime then take $xy \in R \cap J$ with $x,y \in R$. Now $xy \in J$ so $x \in J$ or $y \in J$. Therefore, since both $x,y \in R$ then $x \in R \cap J$ or $y \in R \cap J$ so $R \cap J$ is a prime ideal in $R$. Suppose that $\exists s \in S \cap (R \cap J)$ then $s \in J$ so by absorption, $\frac{1}{s} s \in J$ since $\frac{1}{s} \in S^{-1} R$ thus $1 \in J$ so $J = S^{-1}R$ which contradicts $J$ being a prime ideal. Thus, $(R \cap J) \cap S  = \varnothing$ so $D$ is surjective in the set of prime ideals of $S^{-1}R$. \\ \\
Therefore, $D$ is a bijection from the set of prime ideals of $R$ which are disjoint with $S$ and the prime ideals of $S^{-1}R$. 
\end{proof}

\begin{theorem}
If $R$ is a Dedekind domain with $S \subset R$ then $S^{-1} R$ is Dedekind.
\end{theorem}

\begin{proof}
Let $R$ be a Dedekind domain. By Lemma \ref{idealsurj}, the map $I \mapsto S^{-1} I$ is a surjection. If $J_1 \subset J_2 \subset \cdots $ is an increasing chain of ideals of $S^{-1}R$ then $J_i = S^{-1}I_i$. Suppose that $I_i \supset I_{i+1}$, then $S^{-1} I_i \supset S^{-1}I_{i+1}$ so if $J_i \subsetneq J_{i+1}$ then $I_i \subsetneq I_{i+1}$. Therefore, $I_1 \subset I_2 \subset \cdots$ is an increasing chain of ideals of $R$. Since $R$ is Noetherian, the chain of $I_i$ terminates i.e. after some $n$, $I_n = I_{n+1} = \cdots$ so $I_n \supset I_{n+1} \supset \cdots$ and therefore, $J_n \supset J_{n+1} \supset \cdots$. Thus, the chain of $J_i$ also terminates at $n$ so $S^{-1} R$ is Noetherian. 
\\ \\
Suppose that $\alpha$ is integral over $S^{-1}R$. Then, for some monic polynomial $Q \in S^{-1}R[x]$, $Q(\alpha) = \alpha^n + c_{n-1} \alpha^{n-1} + \cdots + c_0 = 0$. But each $c_i \in S^{-1}R$ so $c_i = \frac{r_i}{s_i}$ for $r_i \in R$ and $s_i \in S$. Multiply through by $s^n = (s_{n-1} s_{n-2} \dots s_0)^n$, \[Q(\alpha)s^n = (s \alpha)^n + r_{n-1} (s_{n-2} \dots s_0)(s \alpha)^{n-1} + \cdots + s^{n-1}( s_{n-1} s_{n-2} \cdots s_1) r_0 = 0\]
Thus, $s\alpha$ is integral over $R$. However, $R$ is Dedekind and thus integrally closed so $s \alpha \in R$. Since $s \alpha \in R$ and $s \in S$ then $\frac{s \alpha}{s} = \alpha \in S^{-1}R$ so $S^{-1}R$ is integrally closed. 
\\ \\
Let $J \subset S^{-1} R$ be a non-zero prime ideal of $S^{-1}R$. By Lemma \ref{idealbij}, $J = S^{-1}I$ where $I$ is a non-zero prime ideal which is disjoint with $S$. Since $I$ is a non-zero prime ideal of $R$ and $R$ is Dedekind, then $I$ is maximal. Suppose that $J \subsetneq L \subset S^{-1}R$. Then $L = S^{-1} F$ for an ideal $F$. Then $I \subsetneq F$ so $F = R$ and thus $L = S^{-1}F = S^{-1}R$ so $J$ is maximal. Thus, $S^{-1}R$ is Dedekind.
\end{proof}

\begin{definition}
Let $R$ be a Dedekind domain and $\mathfrak{p} \subset R$ be a prime ideal then the \textit{loclization} of $R$ at $\mathfrak{p}$ is $R_{\mathfrak{p}} = S_\mathfrak{p}^{-1}R$ for $S_\mathfrak{p}^{-1} = R \sm \mathfrak{p}$.
\end{definition}

\begin{theorem}
Let $R$ be a Dedekind domain and $\mathfrak{p} \subset R$ be a prime ideal then $S_\mathfrak{p}^{-1} = R \sm \mathfrak{p}$ is multiplicative and $R_{\mathfrak{p}} = S_\mathfrak{p}^{-1}R$ is a DVR. 
\end{theorem}

\begin{proof}
Let $R$ be a Dedekind domain and $\mathfrak{p}$ be a prime ideal of $R$. Define $S_\mathfrak{p} = R \sm \mathfrak{p}$ and $R_\mathfrak{p} = S_\mathfrak{p}^{-1}R$. If $s, s' \in S_p$ then if $ss' \in \mathfrak{p}$ then either $s \in \mathfrak{p}$ or $s' \in \mathfrak{p}$ because $\mathfrak{p}$ is a prime ideal. However, $s, s' \in S_p$ so neither are in $\mathfrak{p}$. Thus, $ss' \notin \mathfrak{p}$ so $ss' \in S_\mathfrak{p}$. Also, $1 \notin \mathfrak{p}$ because a prime ideal cannot be the entire ring thus $1 \in S_\mathfrak{p}$. By Lemma \ref{idealbij}, there is a bijection between the prime ideals of $R$ which do not intersect with $S_\mathfrak{p}$ and the prime ideals of $R_\mathfrak{p}$. If $\mathfrak{q} \subset R$ is a non-zero prime ideal and $\mathfrak{q} \cap S_\mathfrak{p} = \mathfrak{q} \cap (R \sm \mathfrak{p}) = \varnothing$ then $\mathfrak{q} \subset \mathfrak{p}$ but $R$ is a Dedekind domain so every non-zero prime ideal is maximal and thus $\mathfrak{q} = \mathfrak{p}$ since $\mathfrak{p} \neq R$. Thus $\mathfrak{p}$ is the unique non-zero prime ideal of $R$ that is disjoint with $S_\mathfrak{p}$. Using the bijection, $S_\mathfrak{p}^{-1} \mathfrak{p}$ is the unique prime ideal of $R_\mathfrak{p}$. Furthermore, because $R$ is Dedekind the ring $R_\mathfrak{p} = S_\mathfrak{p}^{-1} R$ is also Dedekind. Thus, $R_\mathfrak{p}$ has a unique maximal ideal $S_\mathfrak{p}^{-1} \mathfrak{p}$ since an ideal in a Dedekind domain is maximal if and only if it is prime.  
\end{proof}

\begin{theorem} \label{locfield}
Let $R$ be a Dedekind domain and $\mathfrak{p} \subset R$ a prime ideal and $S \subset R$ a multiplicative set such that $S \cap \mathfrak{p} = \varnothing$ then $S^{-1} R / S^{-1} \mathfrak{p}^k \cong R / \mathfrak{p}^k$.
\end{theorem}

\begin{proof}
Consider the map $\pi : S^{-1} R \to R/ \mathfrak{p}^k$ given by $\pi\left(\frac{r}{s}\right) = (r + \mathfrak{p}^k)(s + \mathfrak{p}^k)^{-1}$. Iff $s \in S$ then $s \notin \mathfrak{p}$ but $(s) + \mathfrak{p}^k \subset \mathfrak{p}^k$ so by the uniqueness of Dedekind factorization, $(s) + \mathfrak{p}^k = \mathfrak{p}^r$ but $s \notin \mathfrak{p}^r$ for $r > 0$ so $(s) + \mathfrak{p}^k = R$. Therefore, there exists $r \in R$ such that $rs - 1 \in \mathfrak{p}^k$ and thus $(s + \mathfrak{p}^k)$ is invertible. Furthermore, if $\frac{r}{s} = \frac{r'}{s'}$ then $rs' = r's$ so $rs' + R = (r + \mathfrak{p}^k)(s' + \mathfrak{p}^k) = r's = (r' + \mathfrak{p}^k)(s + \mathfrak{p}^k)$ and thus $\pi\left(\frac{r}{s}\right) = \pi\left(\frac{r'}{s'}\right)$. Clearly, $\pi$ is a surjective homomorphism. Now, $\frac{r}{s} \in \ker{\pi}$ if and only if $r + \mathfrak{p} = 0$ or equivalently $\frac{r}{s} \in S^{-1} \mathfrak{p}^k$ thus $\ker{\pi} = S^{-1} \mathfrak{p}^k$. Therefore, $S^{-1} R / S^{-1} \mathfrak{p}^k \cong R / \mathfrak{p}^k$.
\end{proof}

\section{Properties of Discrete Valuation Rings}

\begin{theorem}
Any discrete valuation ring is a principal ideal domain which admits unique factorization of the form $a = u \varpi^k$ where $u$ is a unit and $\varpi$ is a uniformizer i.e. $(\varpi) = \mathfrak{m}$ the unique maximal ideal. 
\end{theorem}

\begin{proof}
Let $R$ be a DVR with maximal ideal $\mathfrak{m}$. Since $R$ is Noetherian and $\mathfrak{m}$ is an ideal of $R$ then $\mathfrak{m}$ is an $R$-submodule of finite type. Let $\mathfrak{m} = c_1 R + \dots + c_n R$. Then $(c_i)$ is an ideal of $R$ which is a Dedekind domain so it has a unique prime factorization. Since there is only one prime ideal, $(c_1) = \mathfrak{m}^{k_i}$. Take $\varpi$ to be the $c_i$ with the least $k_i$ then $(c_i) = \mathfrak{m}^{k_i} \subset \mathfrak{m}^{k_\varpi} = (\varpi)$ so $c_i \in (\varpi)$. Therefore, $c_i = r c$ so $\mathfrak{m} = \varpi R = (\varpi)$. 
\\ \\
For any $a \in R$, the ideal $(a)$ has a prime factorization because $R$ is a Dedekind domain. Thus, $(a) = \mathfrak{m}^k = (\varpi)^k = (\varpi^k)$. Thus, $a = u \varpi^k$ where $u$ is a unit.
\end{proof}


\section{Silverman and Tate Exercises}

\textbf{Excercise 2.7}\\
For a prime $p$, define,
\[ R = \{ r \in \Q \mid v_p(r) \ge 0 \} \]
Therefore, an arbitrary element of $R$ can be written as $R = \frac{a}{b}$ where $p \ndivides b$ which is equivalent to $R = S_{(p)}^{-1} \Z = \Z_{(p)}$ the localization of $\Z$ at the prime ideal $(p)$. Therefore, $R$ is a subring of $Q_{\Z} = \Q$ and $R$ is a DVR with uniformizer $p$ i.e. $p R$ is the unique maximal ideal. We have shown that any DVR is a PID and thus a UFD. Since $S_{(p)} \cap pR = \varnothing$, by Theorem \ref{locfield}, the field $R / p Z = \Z_{(p)}/ p \Z_{(p)} \cong \Z / p \Z \cong \finfield{p}$. By Dedekind prime factorization, every ideal can be factored into prime ideals, $I = \mathfrak{p}^k$ since there is a unique prime ideal. Lastly, let $\frac{a}{b} \in R^\times$ be a unit then there exists $\frac{x}{y} \in R$ such that $\frac{a}{b} \cdot \frac{x}{y} = \frac{ax}{by} = 1$ then $ax = by$. However, $p \ndivides by$ so $p \ndivides ax$ and thus $p \ndivides a$. Therefore, $p \ndivides ab$.   
\bigskip \\
\textbf{Excercise 2.8}\\ 
Consider the map $\phi : R \to p^\nu R / p^\sigma R$ given by $\phi(r) = p^\nu r + p^\sigma R$ which is clearly surjective. Now, $r \in \ker{\phi}$ if and only if $p^\nu r \in p^{\sigma} R$ or equivalently $r \in p^{\sigma - \nu} R$. Thus, $\ker{\phi} = p^{\sigma - \nu} R$. Therefore, $p^\nu R / p^\sigma R \cong R / p^{\sigma - \nu} R$. Since $R = S_{(p)}^{-1} \Z$ and $S_{(p)} \cap (p) = \varnothing$, by Theorem \ref{locfield}, $R / p^{\sigma - \nu} R = S_{(p)}^{-1} \Z / S_{(p)}^{-1} p^{\sigma - \nu} \Z \cong \Z / p^{\sigma - \nu} \Z$. Therefore, $p^\nu R / p^\sigma R \cong \Z / p^{\sigma - \nu} \Z$.
\bigskip \\
\textbf{Excercise 2.9}\\ 
Let $S_p = \{p^k \mid k \in \N \}$ then take $R = S_p^{-1} \Z \subset Q_{\Z} \cong \Q$ which is a Dedekind domain. Any element of $S^{-1} \Z$ can be written as $a p^\nu$ for $p \ndivides a$ and $\nu \in \Z$. Thus, if $a p^\nu \in R$ is a unit then there must exist $b p^\sigma \in R$ such that $ab p^{\nu - \sigma} = 1$. Therefore, $ab = \pm$ so $a, b = \pm 1$. Therefore, $a p^\nu = \pm p^\nu$. Finally, we know that there is a one-to-one correspondence between the prime ideals of $\Z$ which do not intersect $S_p$, i.e. $(q)$ for $q \neq p$, and the ideals of $R = S_p^{-1} \Z$. For any prime $q \in \Z$ we have that $S_p^{-1} (q) = q R$ is a prime ideal of $R$ and every prime ideal of $R$ is of this form. Furthermore, since $R$ and $\Z$ are Dedekind, the maximal and prime ideals are equivalent. Finally, since $(q) \cap S_p = \varnothing$, by Theorem \ref{locfield}, $R / q R \cong S_p^{-1} \Z / S_p^{-1} (q) \cong \Z / (q) = \finfield{q}$.    
\end{document}


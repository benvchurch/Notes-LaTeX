\documentclass[12pt]{extarticle}
\usepackage{import}
\import{./}{Includes}


\begin{document}
\atitle{7}

\section*{Page 193.} 
\subsection*{Problem 1.}
Over $\C$, the polynomial $P(X) = X^3 - 2 = \left(\frac{X}{\sqrt[3]{2}}\right)^3 - 1$ thus, any solution to $P(X) = 0$ must be a $3^{\mathrm{rd}}$ root of unity times $\sqrt[3]{2}$. The third roots of unity are generated by $\zeta_3 = \frac{-1 + \sqrt{3}}{2}$ so every root can be written in the form $\zeta_3^{n} \sqrt[3]{2}$. Also, $(\zeta_3^2 \sqrt[3]{2})/2 = \zeta_3$ so we know that both $\zeta_3$ and $\sqrt[3]{2}$ must be in the splitting field and clearly every root is generated by these elements. Thus, the splitting field is $\Q(\zeta_3, \sqrt[3]{2})$.   

\subsection*{Problem 2.}
Consider the polynomial $P(X) = X^4 + 5X^2 + 6$. Over $\Q$ this polynomial can be factored as $(X^2 + 3)(X^2 + 2)$ so the roots in $\C$ are $\pm i \sqrt{3}$ and $\pm i \sqrt{2}$. The splitting field must contain both $i \sqrt{3}$ and $i \sqrt{2}$ and these two elements plus $\Q$ generate all the roots. Thus, the splitting field is $\Q(i\sqrt{3}, i\sqrt{2})$.  

\subsection*{Problem 3.}
The polynomial $f = X^2 + X + 1$ has no roots in $\finfield{2}$. This is easily checked because $\finfield{2}$ is finite: $f(0) = \mod{0^2 + 0 + 1}{1}{2}$ and $f(1) = \mod{1^2 + 1 + 1}{1}{2}$. From problem \# 9 on assignment \# 3, we know that any degree two polynomial over $\finfield{2}$ is irreducible iff it has no roots in $\finfield{2}$. Thus, $f$ is irreducible over $\finfield{2}$ and therefore, $E = \finfield{2}[X]/(X^2 + X + 1)$ is a field. We know that $[E : F] = 2$ because $\deg{f} = 2$ with $\{1, X\}$ forming a basis of $E$ over $F$. Since $F$ contains $2$ elements, $E$ contains 4 elements, namely, $0, 1, X, 1 + X$. By the classification of finite fields, there is a unique field extension of $\finfield{2}$ of degree $2$ or equivalently of order $4$. Thus, $E \cong \finfield{2^2}$ which is the splitting field of $X^4 - X = X(X-1)(X^2 + X + 1)$ over $\finfield{2}$. This can be seen explicitly because $\finfield{2}$ already contains every root of $X$ and $X-1$ so we need only extend by the roots of $X^2 + X + 1$. We explicitly exhibit the addition and multiplication tables below.
\begin{center}
\begin{tabular}{ c | c c c c}
 + & $0$ & $1$ & $X$ & $1 + X$ \\
\hline
 $0$ & $0$ & $1$ & $X$ & $1 + X$ \\ 
 $1$ & $1$ & $0$ & $1 + X$ & $X$ \\  
 $X$ & $X$. & $1 + X$ & $0$ & $1$ \\
 $1 + X$ & $1 + X$ & $X$ & $1$ & $0$ 
\end{tabular}
\quad
\begin{tabular}{ c | c c c c}
$\cdot$ & $0$ & $1$ & $X$ & $1 + X$ \\
\hline
 $0$ & $0$ & $0$ & $0$ & $0$ \\ 
 $1$ & $0$ & $1$ & $X$ & $1 + X$ \\  
 $X$ & $0$ & $X$ & $1 + X$ & $1$ \\
 $1 + X$ & $0$ & $1 + X$ & $1$ & $X$ 
\end{tabular}
\end{center}
$(\finfield{2^2}, +) \cong \Z/2\Z \times \Z/2\Z$ and $(\finfield{2^2}^\times, \cdot) \cong \Z/3\Z$.

\subsection*{Problem 5.}
Let $K$ be a field of characteristic $p > 0$. Then suppose that $K$ contains a subfield, $F$, of order $p^n$. Then $F^\times = F \sm \{0 \}$ is a finite subgroup of $K^\times$ and thus cyclic of order $p^n - 1$. By Lagrange, $\forall r \in F^\times : r^{p^n - 1} = 1$. Thus, every $r \in F$ satisfies $r^{p^n} - r = 0$ (zero also satisfies this polynomial). Therefore, $K$ contains $p^n$ distinct roots of $X^{p^n} - X$ which has order $p^n$ so the polynomial $X^{p^n} - X$ must split over $K$. Conversely, suppose that $X^{p^n} - X$ splits over $K$. Then, $\exists \alpha_1, \dots, \alpha_{p^n} \in K$ such that,
\[ X^{p^n} - X = (X - \alpha_1) \cdots (X - \alpha_{p^n})\]  
$K$ has characteristic $p$ so $K$ contains a prime subfield isomorphic to $\finfield{p}$. The subfield $\finfield{p}(\alpha_1, \dots, \alpha_{p^n})$ is the splitting field of $X^{p^n} - X$ over $\finfield{p}$. By the classification of finite fields, $\finfield{p}(\alpha_1, \dots, \alpha_{p^n}) \cong \finfield{p^n}$. Thus, $K$ contains an isomorphic copy of $\finfield{p^n}$. This is the unique subfield of order $p^n$ because every element of $F \subset K$ with order $p^n$ must be a root of $X^{p^n} - X$ but all $p^n$ roots are contained in $\finfield{p}(\alpha_1, \dots, \alpha_{p^n})$ so $F \subset \finfield{p}(\alpha_1, \dots, \alpha_{p^n})$ however, they both have order $p^n$ so $F = \finfield{p}(\alpha_1, \dots, \alpha_{p^n})$.   

\subsection*{Problem 7.}
Let $K$ be a field and let $L, M$ be finite subfields of order $p^l$ and $p^m$ respectively.  Now, $L \cap M$ is a finite field and it is a subfield of both $L$ and $M$ therefore, it is a subgroup of both so by Lagrange, its order divides $p^l$ and $p^m$ so $|L \cap M| = p^d$ for some $d \le \max\{l, m\}$. By Lemma \ref{prob6}, since $|L \cap M|$ is a subfield of both $L$ and $M$ with order $p^d$ we must have that $d \divides l$ and $d \divides m$. Suppose that $c \divides l$ and $c \divides m$ then by Lemma \ref{prob6}, there exist subfields of $L$ and of $M$ with order $p^c$. However, by problem 5, $K$ contains at most one subfield of order $p^c$ so there is a single subfield, $F$ contained in both $L$ and $M$ with order $p^c$. Thus, $F \subset L \cap M$ so, by Lemma \ref{prob6}, $c \divides d$. Therefore, $d = \gcd{(l, m)}$.       

\section*{Lemmas}

\begin{lemma} \label{prob6}
There exists a subfield of order $p^m$ in $\finfield{p^n}$ if and only if $m \divides n$.
\end{lemma}
\begin{proof}
$\finfield{p^n}$ has characteristic $p$ and therefore contains an isomorphic copy of $\finfield{p}$. Suppose that $K$ is a subfield of $\finfield{p^n}$ then $[\finfield{p^n} : K][K : \finfield{p}] = [\finfield{p^n} : \finfield{p}] = n$ thus $[K : \finfield{p}] \divides n$ so $K = p^m$ with $m \divides n$. Suppose that $m \divides n$ then let $n = mr$,
\[P(X) = X^{p^n} - X = (X^{p^m} - X) (X^{p^{n - m}} + X^{p^{n - 2m} + 1} + \dots + X^{p^m + r - 1} + X^{r})\]
We can show that this division must be possible by modular arguments. Because $m \divides n$ we have $p^m - 1 \divides p^n - 1$ by Lemma \ref{divis} so,
\[ \mod{X^{p^m - 1}}{1}{(X^{p^m - 1} - 1)} \implies \mod{X^{p^n - 1}}{1}{(X^{p^m - 1} - 1)} \implies X^{p^m - 1} - 1 \divides X^{p^n - 1} - 1\]
However, $X^{p^n} - X$ splits over $\finfield{p^n}$ and thus, $X^{p^m} - X$ splits over $\finfield{p^n}$. Since $\finfield{p^n}$ has characteristic $p$, by problem 5, there exists a unique subfield of order $p^m$. 
\end{proof}

\begin{lemma} \label{divis}
$\gcd{(a^r - 1, a^s - 1)} = a^{\gcd{(r,s)}} - 1$ and in particular, $a^r - 1 \divides a^s - 1 \iff r \divides s$
\end{lemma}

\begin{proof}
Let $d = \gcd(r,s)$ so there exist integers $x,y$ s.t. $ax + by = d$. Now, let $g = a^d - 1$ then,
\[\mod{a^d}{1}{g} \implies \mod{a^r}{1}{g} \text{ and } \mod{a^s}{1}{g}\]
Therefore, $g \divides a^r -1$ and $g \divides a^s - 1$. Suppose that $c \divides a^r - 1$ and $c \divides a^s - 1$ then, 
\[\mod{a^s}{1}{c} \text{ and } \mod{a^r}{1}{c} \implies \mod{a^{ax + by} = a^d}{1}{c} \implies c \divides a^d - 1 = g \]
Thus, $g = \gcd{(a^r - 1, a^s - 1)}$. We need not worry about taking $a^r$ or $a^s$ to negative powers because they are invertable modulo $c$.  
\end{proof}

\end{document}
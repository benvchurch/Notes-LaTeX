\documentclass[12pt]{extarticle}
\usepackage[utf8]{inputenc}
\usepackage[english]{babel}
\usepackage[a4paper, total={7in, 9.5in}]{geometry}
 
\usepackage{amsthm, amssymb, amsmath, centernot}

\newcommand{\notimplies}{%
  \mathrel{{\ooalign{\hidewidth$\not\phantom{=}$\hidewidth\cr$\implies$}}}}
 
\renewcommand\qedsymbol{$\square$}
\newcommand{\cont}{$\boxtimes$}
\newcommand{\divides}{\mid}
\newcommand{\ndivides}{\centernot \mid}
\newcommand{\Z}{\mathbb{Z}}
\newcommand{\N}{\mathbb{N}}
\newcommand{\C}{\mathbb{C}}
\newcommand{\Zplus}{\mathbb{Z}^{+}}
\newcommand{\Primes}{\mathbb{P}}
\newcommand{\ball}[2]{B_{#1} \! \left(#2 \right)}
\newcommand{\Q}{\mathbb{Q}}
\newcommand{\R}{\mathbb{R}}
\newcommand{\Rplus}{\mathbb{R}^+}
\newcommand{\invI}[2]{#1^{-1} \left( #2 \right)}
\newcommand{\End}[1]{\text{End}\left( A \right)}
\newcommand{\legsym}[2]{\left(\frac{#1}{#2} \right)}
\renewcommand{\mod}[3]{\: #1 \equiv #2 \: \mathrm{mod} \: #3 \:}
\newcommand{\nmod}[3]{\: #1 \centernot \equiv #2 \: \mathrm{mod} \: #3 \:}
\newcommand{\ndiv}{\hspace{-4pt}\not \divides \hspace{2pt}}
\newcommand{\finfield}[1]{\mathbb{F}_{#1}}
\newcommand{\finunits}[1]{\mathbb{F}_{#1}^{\times}}
\newcommand{\ord}[1]{\mathrm{ord}\! \left(#1 \right)}
\newcommand{\quadfield}[1]{\Q \small(\sqrt{#1} \small)}
\newcommand{\vspan}[1]{\mathrm{span}\! \left\{#1 \right\}}
\newcommand{\galgroup}[1]{Gal \small(#1 \small)}
\newcommand{\ints}[1]{\mathcal{O}_{#1}}
\newcommand{\sm}{\! \setminus \!}
\renewcommand{\Im}[1]{\mathrm{Im} \: #1}
\newcommand{\ch}[1]{\mathrm{char} \: #1}
\renewcommand{\Im}[1]{\mathrm{Im}(#1)}
\newcommand{\minimal}[2]{\mathrm{Min}(#1;#2)}
\newcommand{\fix}[2]{\mathrm{Fix}_{#1} (#2)}
\newcommand{\id}{\mathrm{id}}
\newcommand{\Disc}[1]{\mathrm{Disc}(#1)}
\newcommand{\sgn}[1]{\mathrm{sgn}(#1)}
\newcommand{\tr}[1]{\mathrm{Tr} \: #1}
\newcommand{\Hom}[2]{\mathrm{Hom}\left( #1, #2 \right)}
\newcommand{\aut}[1]{\mathrm{Aut}\left( #1 \right)}
\newcommand{\repHom}[3]{\mathrm{Hom}^{#1} \left( #2, #3 \right)}
\newcommand{\Sym}[1]{\mathrm{Sym}(#1)}
\newcommand{\GL}[2]{\mathrm{GL}(#1, #2)}
\newcommand{\inner}[2]{\left< #1, #2 \right>}

\newcommand{\Res}[3]{\mathrm{Res}^{#1}_{#2} \, #3}

\newcommand{\Ind}[3]{\mathrm{Ind}^{#1}_{#2} \, #3}

\renewcommand{\theenumi}{(\alph{enumi})}

\newcommand{\atitle}[1]{\title{% 
	\large \textbf{Mathematics GU4044 Representations of Finite Groups 
	\\ Assignment \# #1} \vspace{-2ex}}
\author{Benjamin Church }
\maketitle}

 
\newtheorem{theorem}{Theorem}[section]
\newtheorem{lemma}[theorem]{Lemma}
\newtheorem{proposition}[theorem]{Proposition}
\newtheorem{corollary}[theorem]{Corollary}

\begin{document}
\atitle{11}

\section*{Problem 1.}

Let $f(X) = a_n X^n + \cdots + a_1 X + a_0$ be an irreducible polynomial over $K$. Now, \[f'(X) = n \cdot a_n X^{n-1} + \cdots a_1\] If $f' \neq 0$ then $f$ is $f$ is seperable by above. Otherwise, because $K$ has characteristic zero, the unique homomorpihsm $\Z \to K$ given by repeated addition is injective so $f' = 0$ implies that $a_n = \cdots= a_1 = 0$. Therefore, $f(X) = a_0$ which is already split in $K$ and has no roots. Thus, $f$ is vacuously seperable.  

\section*{Page 210.}
\subsection*{Problem 2.}

Let $f \in \R[X]$ be a degree $3$ polynomial. Because $f$ is of odd degree, it must take on both positive and negative values over $\R$. Therefore, by IVT, $f$ has at least one real root. Furthermore, if $f(\alpha) = 0$ then $\overline{f(\alpha)} = f(\overline{\alpha}) = 0$ so $\bar{\alpha}$ is also a root of $f$. The complex conjugation automorphism can be brought inside $f$ because its coeficients are all real. Thus, if $f$ has a complex root then it has a pair of complex roots. Thus, there are two possibilities, either $f$ has three real roots or $f$ has one real root and two conjugate complex roots. Now, consider the discriminant,
\[\Disc{f} = \Delta = \prod\limits_{i < j} (\alpha_i - \alpha_j)^2\]
If all three roots of $f$ are real then all the differences are real. Thus, $(\alpha_i - \alpha_j)^2 \ge 0$ so $\Delta \ge 0$. Therefore, if $\Delta < 0$ then $f$ has one real root and two complex roots. If $\Delta = 0$ then $f$ has a double root. However, if $f$ had only a single real root then it could not be equal to the other two complex roots which also could not equal eachother because they are complex conjugates and not real. Therefore, if $\Delta = 0$ then $f$ has three real roots. Finally, suppose that $\Delta > 0$. Then, $\Delta$ is a square in $\R$ so the Galois group of $E$, the splitting field of $f$ over $\R$, contains no odd permutations and is thus a subgroup of $A_3 \cong \Z/3\Z$. However, no element of $\Z/3\Z$ has order $2$ so complex conjugation cannot be a nontrivial automorphism of $E$. However, complex conjugation takes a root of $f$ to another root of $f$ and fixes the base field $\R$ so it preserves $E = \R(\alpha_1, \alpha_2, \alpha_3)$ and thus complex conjugation is an automorphism of $E$. Therefore, $E$ cannot be have complex elements because complex conjugation acts trivially so $f$ must have only real roots. Because $\Delta \neq 0$ we know that $f$ is seperable so it has exactly 3 real roots.  


\section*{Page 220.}
\subsection*{Problem 9.}

\section*{Lemmas}

\begin{lemma} \label{complimentlem}

\end{lemma}
\begin{proof}

\end{proof}

\end{document}
\documentclass[12pt]{extarticle}
\usepackage[utf8]{inputenc}
\usepackage[english]{babel}
\usepackage[a4paper, total={6in, 9in}]{geometry}
\usepackage{tikz-cd}
 
\usepackage{amsthm, amssymb, amsmath, centernot}
\usepackage{mathrsfs} 

\newcommand{\notimplies}{%
  \mathrel{{\ooalign{\hidewidth$\not\phantom{=}$\hidewidth\cr$\implies$}}}}
 
\renewcommand\qedsymbol{$\square$}
\newcommand{\cont}{$\boxtimes$}
\newcommand{\divides}{\mid}
\newcommand{\ndivides}{\centernot \mid}
\newcommand{\Z}{\mathbb{Z}}
\newcommand{\R}{\mathbb{R}}
\newcommand{\C}{\mathbb{C}}
\newcommand{\N}{\mathbb{N}}
\newcommand{\Zplus}{\mathbb{Z}^{+}}
\newcommand{\Primes}{\mathbb{P}}
\newcommand{\colim}[1]{\mathrm{colim}(#1)}
\newcommand{\Ob}[1]{\mathrm{Ob}(#1)}
\newcommand{\cat}[1]{\mathcal{#1}}
\newcommand{\id}{\mathrm{id}}
\newcommand{\Hom}[2]{\mathrm{Hom}\left( #1, #2 \right)}
\newcommand{\catHom}[3]{\mathrm{Hom}_{#1}\left( #2, #3 \right)}
\newcommand{\Top}{\mathbf{Top}}
\newcommand{\pTop}{\mathbf{Top}_{\bullet}}
\newcommand{\Set}{\mathbf{Set}}
\newcommand{\pSet}{\mathbf{Set}_\bullet}
\newcommand{\hTop}{\mathbf{hTop}}
\newcommand{\phTop}{\mathbf{hTop}_{\bullet}}
\renewcommand{\Im}[1]{\mathrm{Im}(#1)}
\newcommand{\homspace}[2]{\left< #1, #2 \right>}
\newcommand{\rp}{\mathbb{RP}}
\newcommand{\coker}[1]{\mathrm{coker}\: #1}

\theoremstyle{definition}
\newtheorem{theorem}{Theorem}[section]
\newtheorem{lemma}[theorem]{Lemma}
\newtheorem{proposition}[theorem]{Proposition}
\newtheorem{example}[theorem]{Example}
\newtheorem{corollary}[theorem]{Corollary}
\newtheorem{remark}{Remark}

\newenvironment{definition}[1][Definition:]{\begin{trivlist}
\item[\hskip \labelsep {\bfseries #1}]}{\end{trivlist}}


\newenvironment{lproof}{\begin{proof} \renewcommand{\qedsymbol}{}}{\end{proof}}
\renewcommand{\mod}[3]{\: #1 \equiv #2 \: mod \: #3 \:}
\newcommand{\nmod}[3]{\: #1 \centernot \equiv #2 \: mod \: #3 \:}
\newcommand{\ndiv}{\hspace{-4pt}\not \divides \hspace{2pt}}
\newcommand{\gen}[1]{\langle #1 \rangle}
\newcommand{\hook}{\hookrightarrow}
\newcommand{\Tor}[4]{\mathrm{Tor}^{#1}_{#2} \left( #3, #4 \right)}
\newcommand{\Ext}[4]{\mathrm{Ext}^{#1}_{#2} \left( #3, #4 \right)}

\tikzset{
    labl/.style={anchor=south, rotate=90, inner sep=.5mm}
}

\renewcommand{\bf}[1]{\mathbf{#1}}
\newcommand{\res}{\mathrm{res}}
\newcommand{\F}{\mathcal{F}}
\newcommand{\G}{\mathcal{G}}
\renewcommand{\O}{\mathcal{O}}
\renewcommand{\d}[1]{\mathrm{d} #1}
\newcommand{\deriv}[2]{\frac{\d{#1}}{\d{#2}}}
\newcommand{\Aut}[1]{\mathrm{Aut}\left( #1 \right)}
\newcommand{\End}[1]{\mathrm{End}\left( #1 \right)}


\begin{document}

\section{Lie Groups} 

\begin{definition}
A Lie Group $X$ is a smooth manifold with a smooth group stucture.
\end{definition}

\begin{definition}
Let $G$ be a Lie group and $X$ a manifold. A smooth action of $G$ on $X$ is a smooth map $A : G \times X \to X$ where we write $g \cdot x = A(g, x)$ such that $g_1 \cdot (g_2 \cdot x) = (g_1 g_2) \cdot x$ and $1 \cdot x = x$. This is equivalent to a smooth map $G \to \mathrm{Diffeo}(X)$. 
\end{definition}

\begin{definition}
The action of a Lie group $G$ on $X$ is proper if the map, $\pi : G \times X \to X \times X$ given by $(g, x) \mapsto (g \cdot x, x)$ is a proper map. In particular, \[ \mathrm{Stab}(x) \times \{x\} = \pi^{-1}(\left\{(x,x)\right\})\]
is compact. 
\end{definition}

\begin{lemma}
If $G$ is compact then any action of $G$ on $X$ is proper.
\end{lemma}

\begin{proof}
Let $D \subset X \times X$ be compact and thus closed because $D$ is compact in a Hausdorff manifold. Thus, $\pi^{-1}(D) = \{ (g, x) \mid (g \cdot x, x) \in D \}$ is closed in $G \times X$ and thus closed in $G \times \pi_2(D)$ which is compact. Therefore $\pi^{-1}(D)$ is compact. 
\end{proof}

\begin{proposition} \label{proper_self_actions}
The left and right actions of any Lie group on itself are proper.
\end{proposition}

\begin{proof}
Let $D \subset G \times G$ be compact and consider,
\[ \pi^{-1}(D) = \{ (g, h) \mid (g h, h) \in D \} = \{ (h' h^{-1}, h) \mid (h', h) \in D \}\]
However, this set is diffeomorphic to $D$ and is thus compact. The same argument works for a right action.
\end{proof}

\begin{proposition}
Let $G$ be a Lie group. The adjoint action of $G$ on $G$ given by $g \cdot x = gxg^{-1}$ is proper if and only if $G$ is compact.
\end{proposition}

\begin{proof}
If the action is proper then $\mathrm{Stab}(1) = G$ must be compact but if $G$ is compact then every action is proper. 
\end{proof}

\begin{proposition}
The orbits of a proper action of a Lie group $G$ on a manifold $X$ are submanifolds of $X$.
\end{proposition}

\begin{proof}
Take $x \in X$, consider $f : G \to X$ by $g \mapsto g \cdot x$. Furthermore the differential gives $\mathrm{d}f : T_g G \to T_{g\cdot x} X$ but $T_g G \cong T_1 G$ so we have a map $T_1 G \to T_{g \cdot x} X$. 
\end{proof}

\begin{lemma}
Let $R \subset X \times X$ be an equivalence relation on a topological space $X$ then $X / R$ is Hausdorff if and only if $R$ is closed. 
\end{lemma}

\begin{proof}
Consider the diagonal $\Delta \subset (X / R) \times (X / R)$ which is the set of equivalence classes $([x], [y]) \in \Delta \iff [x] = [y] \iff x \sim y \iff (x, y) \in R$. Consider the map $\pi : X \to X / R$ thus $R = \pi^{-1}(\Delta)$ so $R$ is closed if and only if $\Delta$ is closed if and only if $X/R$ is Hausdorff. 
\end{proof}


\begin{theorem}
Let a Lie group $G$ act on a manifold $X$ with an action that is,
\begin{enumerate}
\item proper, meaning that $G \times X \to X \times X$ is a proper map
\item free, meaning that $\forall x : G_x = \left\{ \id_G \right\}$ i.e. if $g \cdot x = x$ then $g = \id$
\end{enumerate}
then $X / G$ is a manifold. 
\end{theorem}

\begin{corollary}
If $H \subset G$ is a Lie subgroup then $G / H$ is a manifold.
\end{corollary}

\begin{proof}
The action of $G$ on $H$ is free because if $g \cdot h = gh = h$ then $g = \id_G$. Furthermore, the action of $G$ on $H$ is proper by Proposition \ref{proper_self_actions}.
\end{proof}

\section{Lie Algebras}

\newcommand{\g}{\mathfrak{g}}

\begin{definition}
A Lie Algebra $\g$ over a field $K$ is a algebra over $K$ with multiplication written $[- , - ] : \g \otimes \g \to \g$ satisfying,
\begin{enumerate}
\item $[x,y] = -[y,x]$

\item $[x, [y, z]] + [y, [z, x]] + [z, [x, y]] = 0$. 
\end{enumerate}
\end{definition}

\begin{definition}
Let $G$ be a Lie group. There is a canonical Lie group structure on $T_{1} G$.
\end{definition}

\begin{proof}
For $\xi, \eta \in T_{1} G$ we will define a bracket $[\xi, \eta]$. Consider the map $f_g : G \to G$ given by $x \mapsto g x g^{-1}$ then $\d{f}_g : T_{1} G \to T_{1} G$. Suppose we have a path, $\gamma : I \to G$ such that the unit tangent vector is mapped to $d \gamma(e_1) = \xi$. Then we write,
\[ [\xi, \eta] = \deriv{}{t} \Big( \d{f}_{\gamma(t)}(\eta) \Big) \Big|_{t = 0} \]  
\end{proof}

\newcommand{\h}{\mathfrak{h}}

\begin{proposition}
Let $f : G \to H$ be a Lie group homomorphism. Then $\d{f} : \g \to \h$ \footnote{All differentials in this section will be applied at the identity of the group unless explicilty stated otherwise.} is a morphism of Lie algebras i.e. $f([\xi, \eta]_G) = [f(\xi), f(\eta)]_H$. 
\end{proposition}

\begin{corollary}
Let $H \subset G$ be a Lie subgroup then there is a natural embedding of the Lie algebras $\h \subset \g$. 
\end{corollary}

\begin{definition}
A Lie Group representation of $G$ on $V$ is a Lie Group homomorphism $G \to \Aut{V}$. 
\end{definition}

\begin{definition}
Let $\rho_V : G \to \Aut{V}$ be a Lie Group representation. Then we can construct the \textit{dual} representation $\rho_V^* : G \to \Aut{V}$ via,
\[ \rho_V^*(g) = (\rho_V(g^{-1}))^* \]
which is a representation because,
\[ \rho_V^*(gh) = \left( \rho_V(h^{-1} g^{-1}) \right)^* = \left( \rho_V(h^{-1}) \rho_V(g^{-1}) \right)^* = \rho_V(g^{-1})^* \rho_V(h^{-1})^* = \rho_V^*(g) \rho_V^*(h) \]
\end{definition}

\newcommand{\Ad}{\mathrm{Ad}}
\newcommand{\ad}{\mathrm{ad}}
\newcommand{\gl}[1]{\mathfrak{gl}\left( #1 \right)}

\begin{definition}
The adjoint action $a : G \to \Aut{G}$ is given by $g \mapsto a_g : G \to G$ which acts via $x \mapsto g x g^{-1}$. Then, the differential gives, $\Ad(g) = \d{a_g} : \g \to \g$ and the map $\Ad : G \to \Aut{\g}$ is a $G$-representation. Then the differential gives a Lie algebra representation,
\[ \ad : \g \to \gl{\g} \]
where $\ad_\xi = \d{(\Ad)_\xi}$. 
\end{definition}

\begin{theorem}
For any $\xi, \in \g$ and $X \in \g$ we have,
\[ \ad_\xi(X) = [\xi, X ] \]
\end{theorem}

\begin{proof}
(DO THIS)
We may check that $\ad : \g \to \gl{\g}$ is in fact a Lie algebra representation by using the Jacobi identity. Recall that,
\[ [x, [y, z]] + [y, [z, x]] + [z, [x, y]] = 0 \]
which we may reagrange as,
\[ [x, [y, z]] - [y, [x, z]] = [[x, y], z] \]
and then rewrite as,
\[ (\ad_x \circ \ad_y - \ad_y \circ \ad_x)(z) = \ad_{[x, y]}(z) \]
where the left hand side is the bracket for $\gl{\g}$ impling that,
\[ [ \ad_x, \ad_y ] = \ad_{[x, y]} \]
so the map $\ad : \g \to \gl{\g}$ is a Lie algebra representation.
\end{proof}

\newcommand{\Lie}[1]{\mathrm{Lie}\left( #1 \right)}


\begin{theorem}[Lie]
For any Lie algebra $\g$ over $\R$ or $\C$ there exists a unique simply-connected real or complex Lie group $G$ with $\Lie{G} = \g$.
\end{theorem}

\section{The Exponential Map}

\begin{definition}
The multiplication map $m : G \times G \to G$ is smooth. Thus, $m(-,g)$ and $m(g,-)$ are smooth diffeomorphism $G \to G$. Thus, denote the action of $\d{m(g,-)} : T_e G \to T_g G$ on $\xi \in \g$ by $g \cdot \xi = \d{m(g,-)}(\xi) \in T_g G$ and, likewise, the action of $\d{m(-,g)} : T_e G \to T_g G$ on $\xi \in \g$ by $\xi \cdot g = \d{m(-,g)}(\xi) \in T_g G$. 
\end{definition}

\begin{definition}
The exponetial map $\exp : \g \to G$ is defined as follows. For $\xi \in \g$ we can define a smooth vector field $X^{\xi} \in \mathscr{X}(G)$ by $X^{\xi}_g = \xi \cdot g$. Let $\gamma : I \to G$ be an integral curve of $X$ such that $I(0) = e$. Then the exponential map is defined as $\exp{\xi} = \gamma(1)$. 
\end{definition}

\begin{proposition}
Let $f : G \to H$ be a Lie group homomorphism. Then the exponential diagram,
\begin{center}
\begin{tikzcd}[column sep = large, row sep = large]
\g \arrow[d, "\exp"'] \arrow[r, "f_*"] & \h \arrow[d, "\exp"]
\\
G \arrow[r, "f"'] & H
\end{tikzcd}
\end{center}
commutes where $f_* = \d{f}_e$. 
\end{proposition}

\begin{proof}
Let $\gamma$ be the interval curve of $X^\xi$. That is,
\[ \deriv{\gamma}{t} = X^{\xi}(\gamma(t)) = \xi \cdot \gamma(t) \]
Then consider the smooth path $f \circ \gamma : I \to H$ and its derivative,
\[ \deriv{(f \circ \gamma)}{t} = \d{(f \circ \gamma)}_t\left( \deriv{}{t} \right) = \d{f}_{\gamma(t)} \circ \d{\gamma}_t \left( \deriv{}{t} \right) = \d{f}_{\gamma(t)} \left( \deriv{\gamma}{t} \right) = \d{f}_{\gamma(t)} (\xi \cdot \gamma(t)) \]
We can reqrite this result using $\xi \cdot g = \d{m(-,g)}(\xi)$,
\begin{align*}
\deriv{(f \circ \gamma)}{t} = \d{f}_{\gamma(t)} \d{m(-,g)}(\xi) = \d{(f \circ m(-,g))}(\xi) 
\end{align*}
However, $f \circ m(-,g)(x) = f(xg) = f(x) f(g) = m(-,f(g)) \circ f(x)$ and thus $f \circ m(-,g) = m(-,f(g)) \circ f$. Therefore,
\[ \d{f}_g \circ \d{m(-,g)} = \d{(f \circ m(-,g))} = \d{(m(-,f(g)) \circ f)} = \d{m(-,f(g))}_e \circ \d{f}_e \]
Let $g = \gamma(t)$ then,
\begin{align*}
\deriv{(f \circ \gamma)}{t} = \d{(f \circ m(-, \gamma))}(\xi) = \d{m(-,f(\gamma))} \circ f_*(\xi) = f_*(\xi) \cdot (f \circ \gamma)(t)
\end{align*}
Thus, $f \circ \gamma$ is the integral curve starting at $f \circ \gamma(0) = f(e) = e$ of the vector field $X^{f_*(\xi)}$ given by $h \mapsto f_*(\xi) \cdot h$. Therefore,
\[ \exp(f_*(\xi)) = (f \circ \gamma)(1) = f(\gamma(1)) = f(\exp(\xi)) \]
\end{proof}

\begin{lemma}
Let $G$ be a Lie group and let $f_1 : M \to G$ and $f_2 : M \to G$ be smooth maps. Then, $F = f_1 \cdot f_2 = m \circ (f_1, f_2)$ is a smooth map with,
\[ \d{F}(\xi) = \d{f_1}(\xi) \cdot f_2 + f_1 \cdot \d{f_2}(\xi) \] 
\end{lemma}

\begin{proof}
We have,
\begin{align*}
\d{F}_p = \d{m}_{f_1(p), f_2(p)} \circ \d{(f_1, f_2)} = \d{m}_{f_1(p), f_2(p)} \circ ((\d{f_1})_p \oplus (\d{f_2})_p)
\end{align*}
Furthermore,
\[ \d{m} = \d{(m \circ \iota_1^{f_2(p)})} + \d{(m \circ \iota_1^{f_1(p)})} = \d{m(-,f_2(p))} + \d{m(f_1(p),-)} \]
and thus,
\begin{align*}
\d{F}_p = \d{m(-,f_2(p))} \circ (\d{f_1})_p + \d{m(f_1(p),-)} \circ (\d{f_2})_p
\end{align*}
Therefore, for $\xi \in T_p M$ we have,
\begin{align*}
\d{F}_p(\xi) & = \d{m(-,f_2(p))} \circ (\d{f_1})_p(\xi) + \d{m(f_1(p),-)} \circ (\d{f_2})_p(\xi)
\\
& = (\d{f_1})_p(\xi) \cdot f_2(p) + f_1(p) \cdot (\d{f_2})_p(\xi)
\end{align*}
\end{proof}

\begin{corollary}
For any $\xi \in \g$ we have $\Ad(\exp{\xi}) = \exp \circ (\ad_\xi)$. Therefore, on the lie algebra, for any $X \in \g$ we have,
\[ (\exp{\xi}) \cdot X \cdot (\exp{\xi})^{-1} = \Ad(\exp{\xi}) \cdot X = (\exp{(\ad_\xi)})(X) = (\exp{[\xi, - ]}) \cdot X \]  
\end{corollary}

\begin{proposition}
The left and right-invariant vector fields, $X^{\xi}_L, X^{\xi}_R \in \mathscr{X}(G)$ associated with $\xi \in \g$ i.e. $X^{\xi}_L(g) = g \cdot \xi$ and $X^{\xi}_R(g) = \xi \cdot g$ have the same integral curves at the identiy. Thus, either can be used to define the exponential map.
\end{proposition}

\begin{proof}
Let $\gamma_1, \gamma_2 : I \to G$ be smooth curves satisfying,
\[ \deriv{\gamma_1}{t} = X^{\xi}_L(\gamma_1(t)) = \gamma_1(t) \cdot \xi \quad \text{and} \quad \deriv{\gamma_2}{t} = X^{\xi}_R(\gamma_2(t)) = \xi \cdot \gamma_2(t) \]
First consider,
\[ \deriv{}{t} \left( \gamma \cdot \gamma^{-1} \right) = \deriv{\gamma}{t} \cdot \gamma^{-1} + \gamma \cdot \deriv{\gamma^{-1}}{t} \]
But $\gamma \cdot \gamma^{-1} = e$ so the differential is zero. Thus, 
\[ \deriv{\gamma^{-1}}{t} = - \gamma^{-1} \cdot \deriv{\gamma}{t} \cdot \gamma^{-1} \]
Therefore, consider,
\begin{align*}
\deriv{}{t} \left( \gamma_1 \cdot \gamma_2^{-1} \right) & = \deriv{\gamma_1}{t} \cdot \gamma_2^{-1} + \gamma_1 \cdot \deriv{\gamma_2^{-1}}{t} = \deriv{\gamma_1}{t} \cdot \gamma_2^{-1} - \gamma_1 \gamma_2^{-1} \cdot \deriv{\gamma_2}{t} \cdot \gamma_2^{-1} 
\\
& = \gamma_1 \cdot \xi \cdot \gamma_2^{-1} - \gamma_1 \gamma_2^{-1} \cdot \xi \cdot \gamma_2^{-1} \gamma_2 = \gamma_1 \cdot \xi \cdot \gamma_2^{-1} - \gamma_1 \gamma_2^{-1} \cdot \xi
\end{align*}
At $t = 0$ we have $\gamma_1(0) = \gamma_2(0) = e$ and thus,
\[ \deriv{}{t} \left( \gamma_1 \cdot \gamma_2^{-1} \right) \Big|_{t = 0} = \xi - \xi = 0 \]
Therefore, $\gamma_1 \cdot \gamma_2^{-1} = e$ is constant and thus $\gamma_1 = \gamma_2$. 
\end{proof}


\section{Lie Algebras}

\begin{definition}
A Lie Algebra $\g$ over a commutative ring $R$ is an $R$-module with a bilinear bracket $[-,-] : \g \otimes \g \to \g$ which satisfies,
\begin{enumerate}
\item $\forall x \in \g : [x,x] = 0$
\item $\forall x,y,z \in \g : [x, [y,z]] + [y, [z,x]] + [z, [x, y]] = 0$
\end{enumerate}
\end{definition}

\begin{definition}
The \textit{universal enveloping algebra} of a Lie algebra $\g$ over a ring $R$ is the unital associative $R$-algebra,
\[ U \g = T_R(\g) / I \]
where $I$ is the ideal generated by $\{ x \otimes y - y \otimes x - [x, y] \mid x,y \in \g \}$. Note that,
\[ x \otimes y - y \otimes x - [x, y] \in \g \oplus (\g \otimes \g) \subset T(\g) \]
The universal enveloping algebra defines a functor $U : \mathbf{LieAlg}_R \to \mathbf{Mod}_R$ 
\end{definition}

\begin{definition}
A representation of a  Lie Algebra $\g$ over $R$ is an $R$-module $M$ and a Lie algebra homomorphism $\g \to \gl{M}$. That is a linear map $\rho : \g \to \End{V}$ which preserves the bracket i.e.
\[ \rho([X, Y]) = \rho(X) \rho(Y) - \rho(Y) \rho(X) \]
\end{definition}


\begin{proposition}
The category of representations of a Lie algebra $\g$ is equivalent to  the category of $U \g$-modules. 
\end{proposition}

\begin{proof}
Any Lie algebra representation $\rho : \g \to \gl{M}$ may be extended to a ring map $U \g \to \End{M}$ by sending $\rho(m) = m \cdot \id$ and $\rho(x \otimes y) = \rho(x) \rho(y)$. Then we have,
\[ \rho(x \otimes y - y \otimes x) = \rho(x) \rho(y) - \rho(y) \rho(x) = \rho([x, y]) \]
so this extension is well-defined on the quotient. Likewise any map $U \g \to \End{M}$ restricts to $\g \to \End{M}$ and sends the bracket to the commutator thus giving a Lie algebra homomorphism $\g \to \gl{M}$.
\end{proof}

\begin{lemma}
Let $R$ be a ring and $M, N$ be simple $R$-modules. Then any $R$-module morphism $f : M \to N$ is zero or an isomorphism.
\end{lemma}

\begin{proof}
Let $f : V \to W$ be $A$-linear (i.e. a morphism of $A$-representations). Then $\ker{f} \subset V$ is a submodule so $\ker{f} = 0$ or $\ker{f} = V$ by simplicity. Thus either $f = 0$ or injective. Furthermore, $\Im{f} \subset W$ is a submodule so either $\Im{f} = 0$ or $\Im{f} = W$ thus either $f = 0$ or surjective. Therefore, either $f = 0$ or $f$ is an isomorphism.
\end{proof}

\begin{lemma}[Schur]
Let $A$ be a unital associative $K$-algebra over an algebraically closed field $K$ and $V$ and $W$ simple $A$-modules. Then,
\[ \catHom{A}{V}{W} = 
\begin{cases}
K & V \cong W
\\
0 & V \not\cong W
\end{cases} \]
\end{lemma}

\begin{proof}
By above, any nonzero map is an isomorphism.
In the case, $V \cong W$, fix an isomorphism $f : V \to W$. Consider any $g : V \to W$ then $f^{-1} \circ g : V \to V$ is an endomorphism over vectorspaces over an algebraically closed field so $f^{-1} \circ g$ has an eigenvector $v \in V$ with eigenvalue $\lambda$. Thus $f^{-1} \circ g - \lambda \cdot \id_V$ is not injective but is a morphism of representations so, by above, $f^{-1} \circ g - \lambda \cdot \id_V = 0$. Thus, $g = \lambda \cdot f$. 
\end{proof}


\begin{remark}
For the case $A = \C[G]$ for some group $G$ a simple $\C[G]$-module is the same as irreducible complex $G$-representation giving the standard form of the lemma. 
\end{remark}

\begin{corollary}
Let $A$ be a unital associative $K$-algebra over an algebraically closed field and $V$  a semisimple $A$-modules. Then there is a canoical isomorphism,
s\[ \bigoplus_X \catHom{A}{X}{V} \otimes_\C X \xrightarrow{\sim} V \] 
where $X$ runs over the simple $A$-modules.
\end{corollary}

\begin{proof}
The canonical map sends $f \otimes x \mapsto f(x)$. We need to show that this map is an isomorphism. Decompose,
\[ V = \bigoplus_{X} X^{n_X} \]
Then, by Schur,
\[ \catHom{X}{V} \cong \C^{n_X} \] 
which gives,
\[ \bigoplus_X \catHom{A}{X}{V} \otimes_\C X = \bigoplus_X \C^{n_X} \otimes_\C X = \bigoplus_X X^{n_X}  = V \]
by the evauluation map. 
\end{proof}

\begin{definition}
A Casimir element of a Lie algebra $\g$ is an element of $Z(U \g)$ i.e. an element of $U \g$ commuting with everything in $\g$ and thus all of $U \g$. 
\end{definition}

\begin{proposition}
Let $\g$ be a Lie algebra over an algebraically closed field $K$ and $\omega \in U \g$ a Casimir. Suppose that $\rho : \g \to \gl{V}$ is an irreducble $\g$-representation then $\rho(\omega) = \lambda \cdot \id_V$ for some $\lambda \in K$ where $\rho : U \g \to \End{V}$ is the induced map.
\end{proposition}

\begin{proof}
Let $\omega$ be a Casimir. I claim that $\rho(\omega)$ is a $\g$-morphism $V \to V$. This is because $\forall x \in U \g : x \otimes \omega = \omega \otimes x$ in $U \g$ meaning that $\rho(x) \circ \rho(\omega) = \rho(\omega) \circ \rho(x)$. Thus the map $\rho(\omega)$ is $U \g$-linear. Since $V$ is irreducible and $K$ is algebraically closed, by Schur's lemma, $\rho(\omega) = \lambda \cdot \id_V$. 
\end{proof}

\begin{remark}
In the previous case, we call $\lambda$ the Casimir invariant of the irreducible representation $V$ associated to the Casimir element $\omega$.
\end{remark}

\section{Misc}

\begin{theorem}[Poincare-Hopf]
Let $M$ be a compact smooth manifold and $X$ a smooth vector field on $M$ with isolated zeros. Then,
\[ \sum_{x \in X} \mathrm{index}_x(X) = \chi(M) \]
\end{theorem}

\begin{theorem}
A vector bundle of rank $r$ is trivial iff it admits $r$ pointwise linearly independent sections.
\end{theorem}

\begin{proof}

\end{proof}

\begin{theorem}
Let $G$ be a Lie group, then $TG \cong G \times \g$ i.e. the tangent bundle is trivial.
\end{theorem}

\begin{proof}

\end{proof}

\begin{theorem}
Let $G$ be a compact Lie group (of positive dimension) then $\chi(G) = 0$.
\end{theorem}

\begin{proof}
Since $\pi : T G \to G$ is a trivial bundle it admits $n = \dim{G}$ pointwise linearly independent sections (i.e. vector fields) which thus must be nonvanising everywhere (since $n > 0$). Thus, by Poincare-Hopf, $\chi(G) = 0$.
\end{proof}

\begin{theorem}
For $n$ even, $S^n$ admits no nonvanishing vector fields.
\end{theorem}

\begin{proof}
Such a vector field would give a homotopy $\id \simeq - \id$ and thus the degrees of these maps must be equal i.e. $(-1)^{n+1} = 1$ so $n$ must be odd. Alternativly, $\chi(S^n) = 1 + (-1)^n$ and therefore, in the case $n$ is even $\chi(S^n) = 2$. In that case, a nonvanishing vector field would contradict the Poincare-Hopf theorem.
\end{proof}

\begin{theorem}
Let $G$ be a compact Lie group then $\pi_2(G) = 0$. If $G$ is nonabelian then $\pi_3(G) \neq 0$.
\end{theorem}

\begin{corollary}
$S^n$ admits a Lie group structure exactly when $n = 0,1,3$. 
\end{corollary}

\begin{proof}
The case $S^0$ is a zero-dimensional Lie group is clear. Assume $n \ge 1$ so $S^n$ is connected.
If $G$ is an abelian Lie group then its Lie algebra is trivial. By the Lie group Lie algebra correspondence, its universal cover must be $\R^n$. However, $S^n$ is simply connected for $n > 1$ so $S^1$ is the only abelian sphere group. If $G$ is nonabelian then $\pi_3(G) \neq 0$  but $\pi_3(S^n) = 0$ for $n > 3$. Thus we have shown that $n \le 3$. The case $n = 2$ is excluded by noting that even dimensional spheres have nontrivial tangent bundles and thus cannot be Lie groups. 
\end{proof}


\end{document}


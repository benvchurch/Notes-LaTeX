\documentclass[12pt]{extarticle}
\usepackage[utf8]{inputenc}
\usepackage[english]{babel}
\usepackage[a4paper, total={7in, 9.5in}]{geometry}
 
\usepackage{amsthm, amssymb, amsmath, centernot}

\newcommand{\notimplies}{%
  \mathrel{{\ooalign{\hidewidth$\not\phantom{=}$\hidewidth\cr$\implies$}}}}
 
\renewcommand\qedsymbol{$\square$}
\newcommand{\cont}{$\boxtimes$}
\newcommand{\divides}{\mid}
\newcommand{\ndivides}{\centernot \mid}
\newcommand{\Z}{\mathbb{Z}}
\newcommand{\N}{\mathbb{N}}
\newcommand{\C}{\mathbb{C}}
\newcommand{\Zplus}{\mathbb{Z}^{+}}
\newcommand{\Primes}{\mathbb{P}}
\newcommand{\ball}[2]{B_{#1} \! \left(#2 \right)}
\newcommand{\Q}{\mathbb{Q}}
\newcommand{\R}{\mathbb{R}}
\newcommand{\Rplus}{\mathbb{R}^+}
\newcommand{\invI}[2]{#1^{-1} \left( #2 \right)}
\newcommand{\End}[1]{\text{End}\left( A \right)}
\newcommand{\legsym}[2]{\left(\frac{#1}{#2} \right)}
\renewcommand{\mod}[3]{\: #1 \equiv #2 \: \mathrm{mod} \: #3 \:}
\newcommand{\nmod}[3]{\: #1 \centernot \equiv #2 \: \mathrm{mod} \: #3 \:}
\newcommand{\ndiv}{\hspace{-4pt}\not \divides \hspace{2pt}}
\newcommand{\finfield}[1]{\mathbb{F}_{#1}}
\newcommand{\finunits}[1]{\mathbb{F}_{#1}^{\times}}
\newcommand{\ord}[1]{\mathrm{ord}\! \left(#1 \right)}
\newcommand{\quadfield}[1]{\Q \small(\sqrt{#1} \small)}
\newcommand{\vspan}[1]{\mathrm{span}\! \left\{#1 \right\}}
\newcommand{\galgroup}[1]{Gal \small(#1 \small)}
\newcommand{\ints}[1]{\mathcal{O}_{#1}}
\newcommand{\sm}{\! \setminus \!}
\renewcommand{\Im}[1]{\mathrm{Im} \: #1}
\newcommand{\ch}[1]{\mathrm{char} \: #1}
\renewcommand{\Im}[1]{\mathrm{Im}(#1)}
\newcommand{\minimal}[2]{\mathrm{Min}(#1;#2)}
\newcommand{\fix}[2]{\mathrm{Fix}_{#1} (#2)}
\newcommand{\id}{\mathrm{id}}
\newcommand{\Disc}[1]{\mathrm{Disc}(#1)}
\newcommand{\sgn}[1]{\mathrm{sgn}(#1)}
\newcommand{\tr}[1]{\mathrm{Tr} \: #1}
\newcommand{\Hom}[2]{\mathrm{Hom}\left( #1, #2 \right)}
\newcommand{\aut}[1]{\mathrm{Aut}\left( #1 \right)}
\newcommand{\repHom}[3]{\mathrm{Hom}^{#1} \left( #2, #3 \right)}
\newcommand{\Sym}[1]{\mathrm{Sym}(#1)}
\newcommand{\GL}[2]{\mathrm{GL}(#1, #2)}
\newcommand{\inner}[2]{\left< #1, #2 \right>}

\newcommand{\Res}[3]{\mathrm{Res}^{#1}_{#2} \, #3}

\newcommand{\Ind}[3]{\mathrm{Ind}^{#1}_{#2} \, #3}

\renewcommand{\theenumi}{(\alph{enumi})}

\newcommand{\atitle}[1]{\title{% 
	\large \textbf{Mathematics GU4044 Representations of Finite Groups 
	\\ Assignment \# #1} \vspace{-2ex}}
\author{Benjamin Church }
\maketitle}

 
\newtheorem{theorem}{Theorem}[section]
\newtheorem{lemma}[theorem]{Lemma}
\newtheorem{proposition}[theorem]{Proposition}
\newtheorem{corollary}[theorem]{Corollary}

\begin{document}
\atitle{6}

\section*{Problem 1.}

We know that the character of the dual representation satisfies $\chi_{V*} = \overline{\chi_V}$. Therefore,
\[ \inner{\chi_{V*}}{\chi_{V*}} = \inner{\overline{\chi_V}}{\overline{\chi_V}} = \inner{\chi_V}{\chi_V}\]
However, a representation is irreducible if and only if $\inner{\chi_W}{\chi_W} = 1$. Therefore, $V^*$ is irreducible if and only if $V$ is irreducible. 

\section*{Problem 2.}
Suppose $G$ acts on $X$ doubly transitively with $\#(X) = n$. Now consider the action of $G$ on $P$, the set of ordered distinct pairs of elements in $X$. By definition, this action must be transitive and thus there is one orbit of size $\#(P) = n^2 - n$. By orbit-stabilizer, $\#(G) = \#(\mathrm{Orb}(x)) \#(\mathrm{Stab}(x))$ and thus $\#(P) \divides \#(G)$. Therefore, $n^2 - n \divides \#(G)$. However, the order of $G$ must be positive so $\#(G) \ge n^2 - n$.      

\section*{Problem 3.}
Let $G$ be a nonabelian group of order $6$. We know that the number of conjugacy classes is equal to the number of irreducible representation of $G$. Furthermore, $\sum\limits_{i = 1}^h d_i^2 = \#(G) = 6$ where $d_i$ is the dimension of the $i^\mathrm{th}$ irreducible representation. Since $G$ is nonabelian, we cannot have $d_i = 1$ for all $i$. Therefore, at least one $d_i > 1$. However, $3^2 > 6$ so there must be exactly one 2-dimensional representation. Thus, up to order,$d_1 = 2$ which forces $d_2 = 1$ and $d_3 = 1$ so that $d_1^2 + d_2^2 + d_3^2 = 6$. Thus, there are three irreducible representations and thus three conjugacy classes. 
\bigskip \\
\bigskip \\
Let $G$ be a nonabelian group of order $8$. We know that the number of conjugacy classes is equal to the number of irreducible representation of $G$. Furthermore, $\sum\limits_{i = 1}^h d_i^2 = \#(G) = 8$ where $d_i$ is the dimension of the $i^\mathrm{th}$ irreducible representation. Since $G$ is nonabelian, we cannot have $d_i = 1$ for all $i$. Therefore, at least one $d_i > 1$. However, $3^2 > 8$ so there must be exactly one 2-dimensional representation. Thus, up to order,$d_1 = 2$. However, the trivial representation is always irredcible so take $d_2 = 1$. Since $8 - d_1^2 - d_2^2 = 3$ the rest of the sum is forced to be $d_3 = 1$, $d_4 = 1$ and $d_5 = 1$ so that $d_1^2 + d_2^2 + d_3^2 + d_4^2 + d_5^2 = 8$. Thus, there are five irreducible representations and thus five conjugacy classes. 

\section*{Problem 4.}
Last week we showed that $A_4$ has exactly $4$ irreducible representations. Therefore, $A_4$ has exactly $4$ conjugacy classes. Using the results from last week, we quote the character table,

\begin{center}
\begin{tabular}{ c | c c c c }
& e & (1 \: 2 \: 3) & (1 \: 3 \: 2) & (1 \: 2)\cdot( 3 \: 4) \\
\hline
$\chi_0$ & 1 & 1 & 1 & 1 \\
$\chi_1$ & 1 & $\zeta_3$ & $\zeta_3^2$ & 1 \\
$\chi_2$ & 1 & $\zeta_3^2$ & $\zeta_3$ & 1 \\
$\chi_W$ & 3 & 0 & 0 & -1 \\
\end{tabular}
\end{center}
The columns are all clearly orthogonal since $1 + \zeta_3 + \zeta_3^2 = 0$ and thus, $1 + \zeta_3 \overline{(\zeta_3)^2} + \zeta_3^2 \overline{\zeta_3} = 1 + \zeta_3^2 + \zeta_3$. 


\section*{Problem 5.}
\begin{enumerate}
\item As given, $\Hom{W_2}{W_2} \cong W_2 \otimes W_2$ and thus $\chi_{\Hom{W_2}{W_2}} = \chi_{W_2} \cdot \chi_{W_2}$. We can find the character $\chi_{W_2}$ by looking at the number of fixed points of each element in $S_4$. Thus, 
\begin{align*}
& \chi_{W_2}^2(e) = (4 - 1)^3 = 9 \quad \chi_{W_2}^2((1 \: 2)) = 1 \quad \chi_{W_2}((1 \: 2 \: 3)) = 0
\\
& \chi_{W_2}((1 \: 2)(3 \: 4))  = (-1)^2 \quad \chi_{W_2}^2((1 \: 2 \: 3 \: 4)) = (-1)^2 = 1
\end{align*}
This exhausts all cycle types and therefore all the conjugacy classes. Therefore,
\begin{align*}
\inner{\chi_{W^2}}{1} = \frac{1}{24} \left[ 1 \cdot 9 + 6 \cdot 1 + 8 \cdot 0 + 3 \cdot 1 + 6 \cdot 1 \right] = 1
\end{align*} \bigskip\\
Similarly, we know that $\inner{\chi_V}{1} = \dim{V^G}$ and thus,
\[\inner{\chi_{\Hom{W_2}{W_2}}}{1} = \dim{(\Hom{W_2}{W_2})^G} = \dim \repHom{G}{W_2}{W_2}\] 
Since $W_2$ is irreducible, by Schur's Lemma, $\dim \repHom{G}{W_2}{W_2} = 1$. Therefore,
 \[\inner{\chi_{W_2}}{\chi_{W_2}} = \inner{\chi_{\Hom{W_2}{W_2}}}{1} = 1\] 

\item Now,
\begin{align*}
\inner{\chi_{W^2}}{\chi_{W^2}} = \frac{1}{24} \left[ 1 \cdot 9^2 + 6 \cdot 1^2 + 8 \cdot 0^2 + 3 \cdot 1^2 + 6 \cdot 1^2 \right] = 4
\end{align*} 
There are five cycle types of $S_4$ and therefore five irreducible representations of $S_4$.  From part (a), we know that $\inner{\chi_{W_2}^2}{1} = 1$ which implies that the trivial representation is a summand of $\Hom{W_2}{W_2}$ with multiplicity $1$. Furthermore, $\inner{\chi_{W_2}^2}{\chi_{W_2}^2} = 4 = \sum_{i = 1}^r m_i^2$. But we know that $m_1 = 1$ for the trivial representation so $m_1 = m_2 = m_3 = m_4 = 1$. Therefore, $\Hom{W_2}{W_2}$ is the sum of exactly four distinct irreducible representations each with multiplicity $1$. We know that for the trivial representation $d_1 = 1$. There cannot be any other one-dimensional representation in the sum. If there were another one-dimemsional representation $V$ in the decomposition, then,
\[ \inner{\chi_{W_2}^2}{\chi_V}  \ge 1 \]
However, $\chi_{W_2}^2$ is positive and the real part of $\chi_{V}$ must be less than 1 for some values of $g$ for $\chi_V$ to not be the trivial homomorphism. Therefore, 
\[\inner{\chi_{W_2}^2}{\chi_V} < \inner{\chi_{W_2}^2}{1} = 1\] which is a contradiction. Therefore, the is a unique one-dimensional representation. By dimension counting, the remaining three representations mut give $3^2 - 1 = 8$ dimensions. The only way this is possible using irreducible representations of $S_4$ which are at most three dimensional is to have one two-dimensional representation and two three-dimensional representations in the decomposition of $\Hom{W_2}{W_2}$. 


\item First compute the inner product,
\begin{align*}
\inner{\chi_{W_2}^2}{\chi_{W_2}} = \frac{1}{24} \left[ 3^3 + 6 \cdot 1^3 + 8 \cdot 0^3 + 3 \cdot (-1) + 6 \cdot (-1)^3 \right] = 1
\end{align*}
Therefore, $W_2$ appears with multiplicity $1$ in the decomposition of $\Hom{W_2}{W_2}$. Similarly, consider the character of $\epsilon \otimes W_2$,
\begin{align*}
\inner{\chi_{W_2}^2}{\epsilon \otimes \chi_{W_2}} = \frac{1}{24} \left[ 3^3 - 6 \cdot 1^3 + 8 \cdot 0^3 + 3 \cdot (-1) - 6 \cdot (-1)^3 \right] = 1
\end{align*}
Therefore, $\epsilon \otimes W_2$ appears with multiplicity $1$ in the decomposition of $\Hom{W_2}{W_2}$. Because we know there are four irreducible representations each with multiplicity one in the decomposition of $\Hom{W_2}{W_2}$, we may write,
\[ \Hom{W_2}{W_2} = \C(1) \oplus W \oplus W_2 \oplus (\epsilon \otimes W_2) \]
where $W$ is a yet to be determined $S_4$-representation. By dimension counting, 
\[\dim{W} = \dim{\Hom{W_2}{W_2}} - \dim{W_2} - \dim{\epsilon \otimes W_2} - \dim{\C(1)} = 3^2 - 3 - 3 - 1 = 2\]
From general theory, we know there is a unique $S_4$-representation with dimension $2$ which therefore must be $W$. 

\end{enumerate}

\section*{Problem 6.}
\begin{enumerate}
\item 
Let $S_2$ act on $V^{\otimes 2}$ by premuting tensor products. The character of any permutation action is given by the number of fixed points. $e$ fixes everything so $\chi_{V^{\otimes 2}}(e) = \dim{V^{\otimes 2}} = d^2$. However, the flip, $(1 \: 2)$ only fixes elements of the form $v \otimes v$. This subspace is canonically isomorphic to $V$ so $\chi_{V^{\otimes 2}}((1 \: 2)) = \dim{V} = d$. \bigskip \\
If we write $\chi_{V^{\otimes 2}} = A \cdot 1 + B \cdot \varepsilon$ then $\chi_{V^{\otimes 2}}(e) = A + B = d^2$ and $\chi_{V^{\otimes 2}}((1 \: 2)) = A - B = d$. Therefore, $A = \tfrac{1}{2}(d^2 + d)$ and $B = \tfrac{1}{2} (d^2 - d)$. These are the multiplicities of the one-dimensional representations, $\C(1)$ and $\C(\epsilon)$ when we write,
\[ V^{\otimes 2} \cong \C(1)^A \oplus \C(\lambda)^B \] 

\item
Now let $S_3$ act on $V^{\otimes 3}$ by premuting tensor products. The character of any permutation action is given by the number of fixed points. $e$ fixes everything so $\chi_{V^{\otimes 3}}(e) = \dim{V^{\otimes 3}} = d^3$. However, the flip, $(1 \: 2)$ only fixes elements of the form $v \otimes v \otimes w$. This subspace is canonically isomorphic to $V \otimes W$ so $\chi_{V^{\otimes 3}}((1 \: 2)) = \dim{V \otimes W} = d^2$. Furthermore, the thee-cycle $(1 \: 2 \: 3)$ only fixes elements of the form $v \otimes v \otimes v$ which is a subspace canonically isomorphic to $V$. Therefore, $\chi_{V^{\otimes 2}}((1 \: 2 \: 3)) = \dim{V} = d$. This fixes the character on all conjugacy classes. \bigskip \\
Write $\chi_{V^{\otimes 3}} = A \cdot 1 + B \cdot \varepsilon + C \cdot \chi_{W_2}$ where $W_2$ is the irreducible two-dimensional permutation representation of $S_3$. Then, we know,
\begin{align*}
\chi_{V^{\otimes 3}}(e) &= d^3 = A + B + 2 C \\
\chi_{V^{\otimes 3}}((1 \: 2)) &= d^2 = A - B  \\
\chi_{V^{\otimes 3}}((1 \: 2 \: 3)) &= d = A + B - C 
\end{align*} 
Therefore,
\[ A = \tfrac{1}{6}(d^3 + 3d^2 + 2d) \quad B = \tfrac{1}{6}(d^3 - 3 d^2 + 2d) \quad C = \tfrac{1}{3} (d^3 - d)\]
Therefore, if we write,
\[ V^{\otimes 3} = \C(1)^A \oplus \C(\epsilon)^B \oplus W_2^C\] 
Then we have found the multiplicities,
\[ A = \tfrac{1}{6}(d^3 + 3d^2 + 2d) \quad B = \tfrac{1}{6}(d^3 - 3 d^2 + 2d) \quad C = \tfrac{1}{3} (d^3 - d)\]
\item
Now, let $S_n$ act on $V^{\otimes n}$. We have seen that the character of an $\sigma \in S_n$ element of $S_n$ is $d^{f}$ where $f$ is the dimension of the subspace fixed by $g$. Let $g = \gamma_1 \cdot \gamma_2 \cdot \dots \cdot \gamma_t$ be the product of disjoint cycles. The number of fixed points of a cycle of length is $\ell_i$ is $n - \ell_i$ and thus the dimension of the fixed subspace is $n + 1 - \ell_i$. To be a fixed point of $g$, a vector must be fixed by every cycle $\gamma_i$. Each cycle subtracts a factor of $\ell_i - 1$. Therefore, the character of $\sigma$ is, 
\[ \chi_{V^{\otimes n}}(\sigma) = d^{n + t - \sum_{i = 1}^t g_i}\] 
\end{enumerate}


\end{document}
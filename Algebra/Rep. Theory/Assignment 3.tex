\documentclass[12pt]{extarticle}
\usepackage[utf8]{inputenc}
\usepackage[english]{babel}
\usepackage[a4paper, total={7in, 9.5in}]{geometry}
 
\usepackage{amsthm, amssymb, amsmath, centernot}

\newcommand{\notimplies}{%
  \mathrel{{\ooalign{\hidewidth$\not\phantom{=}$\hidewidth\cr$\implies$}}}}
 
\renewcommand\qedsymbol{$\square$}
\newcommand{\cont}{$\boxtimes$}
\newcommand{\divides}{\mid}
\newcommand{\ndivides}{\centernot \mid}
\newcommand{\Z}{\mathbb{Z}}
\newcommand{\N}{\mathbb{N}}
\newcommand{\C}{\mathbb{C}}
\newcommand{\Zplus}{\mathbb{Z}^{+}}
\newcommand{\Primes}{\mathbb{P}}
\newcommand{\ball}[2]{B_{#1} \! \left(#2 \right)}
\newcommand{\Q}{\mathbb{Q}}
\newcommand{\R}{\mathbb{R}}
\newcommand{\Rplus}{\mathbb{R}^+}
\newcommand{\invI}[2]{#1^{-1} \left( #2 \right)}
\newcommand{\End}[1]{\text{End}\left( A \right)}
\newcommand{\legsym}[2]{\left(\frac{#1}{#2} \right)}
\renewcommand{\mod}[3]{\: #1 \equiv #2 \: \mathrm{mod} \: #3 \:}
\newcommand{\nmod}[3]{\: #1 \centernot \equiv #2 \: \mathrm{mod} \: #3 \:}
\newcommand{\ndiv}{\hspace{-4pt}\not \divides \hspace{2pt}}
\newcommand{\finfield}[1]{\mathbb{F}_{#1}}
\newcommand{\finunits}[1]{\mathbb{F}_{#1}^{\times}}
\newcommand{\ord}[1]{\mathrm{ord}\! \left(#1 \right)}
\newcommand{\quadfield}[1]{\Q \small(\sqrt{#1} \small)}
\newcommand{\vspan}[1]{\mathrm{span}\! \left\{#1 \right\}}
\newcommand{\galgroup}[1]{Gal \small(#1 \small)}
\newcommand{\ints}[1]{\mathcal{O}_{#1}}
\newcommand{\sm}{\! \setminus \!}
\renewcommand{\Im}[1]{\mathrm{Im} \: #1}
\newcommand{\ch}[1]{\mathrm{char} \: #1}
\renewcommand{\Im}[1]{\mathrm{Im}(#1)}
\newcommand{\minimal}[2]{\mathrm{Min}(#1;#2)}
\newcommand{\fix}[2]{\mathrm{Fix}_{#1} (#2)}
\newcommand{\id}{\mathrm{id}}
\newcommand{\Disc}[1]{\mathrm{Disc}(#1)}
\newcommand{\sgn}[1]{\mathrm{sgn}(#1)}
\newcommand{\tr}[1]{\mathrm{Tr} \: #1}
\newcommand{\Hom}[2]{\mathrm{Hom}\left( #1, #2 \right)}
\newcommand{\aut}[1]{\mathrm{Aut}\left( #1 \right)}
\newcommand{\repHom}[3]{\mathrm{Hom}^{#1} \left( #2, #3 \right)}
\newcommand{\Sym}[1]{\mathrm{Sym}(#1)}
\newcommand{\GL}[2]{\mathrm{GL}(#1, #2)}
\newcommand{\inner}[2]{\left< #1, #2 \right>}

\newcommand{\Res}[3]{\mathrm{Res}^{#1}_{#2} \, #3}

\newcommand{\Ind}[3]{\mathrm{Ind}^{#1}_{#2} \, #3}

\renewcommand{\theenumi}{(\alph{enumi})}

\newcommand{\atitle}[1]{\title{% 
	\large \textbf{Mathematics GU4044 Representations of Finite Groups 
	\\ Assignment \# #1} \vspace{-2ex}}
\author{Benjamin Church }
\maketitle}

 
\newtheorem{theorem}{Theorem}[section]
\newtheorem{lemma}[theorem]{Lemma}
\newtheorem{proposition}[theorem]{Proposition}
\newtheorem{corollary}[theorem]{Corollary}

\begin{document}
\atitle{3}

\section*{Problem 1.}

\begin{enumerate}
\item[(i)]
Take an element of $O(2)$ given in matrix form by,
\[ M =  
\begin{pmatrix}
a & b\\
c & d
\end{pmatrix}\]
Then, we know that $\det{M} = \pm 1$ so $ad - cb = \pm 1$. Furthermore, $MM^\top = I$ so, 
\[ MM^\top =  
\begin{pmatrix}
a & b\\
c & d
\end{pmatrix}
\begin{pmatrix}
a & c\\
b & d
\end{pmatrix}
=
\begin{pmatrix}
a^2 + b^2 & ac + bd \\
ac + bd & c^2 + d^2
\end{pmatrix}
=
\begin{pmatrix}
1 & 0 \\
0 & 1
\end{pmatrix}
\]
Therefore, $a^2 + b^2 = 1$ and $c^2 + d^2$ so $(a, b)$ and $(c, d)$ are points on the unit circle. Therefore, $a = \cos{\theta}$ and $b = \sin{\theta}$ and $c = \cos{\theta'}$ and $d = \sin{\theta'}$  for some $\theta, \theta' \in [0, 2 \pi)$. However, $a d - bc = \pm 1$ so $\cos{\theta} \sin{\theta'} - \sin{\theta} \cos{\theta'} = \sin{(\theta' - \theta)} = \pm 1$. Thus, $\theta' - \theta = (2n + 1) \pi$ for $n \in \Z$. Then, $\sin{\theta'} = \sin{(\theta + (2n + 1) \pi)} = \pm \sin{\theta}$ and $\cos{\theta'} = \cos{(\theta + (2n + 1) \pi)} = \mp 1$. Therefore,
\[ M =  
\begin{pmatrix}
\cos{\theta} & \pm \sin{\theta}\\
\sin{\theta} & \mp \cos{\theta}
\end{pmatrix} 
= B(\theta) \text{ or } A(\theta) \]

\item[(ii)]
First, 
\begin{align*}
A_{\theta_1} A_{\theta_2}  & = 
\begin{pmatrix}
\cos{\theta_1} & - \sin{\theta_1} \\
\sin{\theta_1} & \cos{\theta_1}
\end{pmatrix}
\begin{pmatrix}
\cos{\theta_2} & - \sin{\theta_2} \\
\sin{\theta_2} & \cos{\theta_2}
\end{pmatrix} 
\\
& = 
\begin{pmatrix}
\cos{\theta_1} \cos{\theta_2} - \sin{\theta_1} \sin{\theta_2} & - \cos{\theta_1} \sin{\theta_2} - \sin{\theta_1} \cos{\theta_2}
\\
\cos{\theta_1} \sin{\theta_2} + \sin{\theta_1} \cos{\theta_2} & - \sin{\theta_1} \sin{\theta_2} + \cos{\theta_1} \cos{\theta_2} 
\end{pmatrix}
\\
& = 
\begin{pmatrix}
\cos{(\theta_1 + \theta_2)} & - \sin{(\theta_1 + \theta_2)} \\
\sin{(\theta_1 + \theta_2)} & \cos{(\theta_1 + \theta_2)}
\end{pmatrix}
= A_{\theta_1 + \theta_2}
\end{align*}
Therefore, $A_\theta^2 = A_{2 \theta}$ and $A_{\theta} A_{-\theta} = A_{0} = I$ so $A_{\theta}^{-1} = A_{- \theta}$. Furthermore, define the matrix,
\[ R = 
\begin{pmatrix}
1 & 0 \\
0 & -1
\end{pmatrix} \]
then clearly $R^2 = I$ and,
\[ A_\theta R = 
\begin{pmatrix}
\cos{\theta} & - \sin{\theta}\\
\sin{\theta} &  \cos{\theta}
\end{pmatrix} 
\begin{pmatrix}
1 & 0 \\
0 & -1
\end{pmatrix} 
=
\begin{pmatrix}
\cos{\theta} &  \sin{\theta}\\
\sin{\theta} &  - \cos{\theta}
\end{pmatrix} 
= B_\theta
\]
Next,
\begin{align*}
B_{\theta_1} B_{\theta_2}  & = 
\begin{pmatrix}
\cos{\theta_1} & \sin{\theta_1} \\
\sin{\theta_1} & -\cos{\theta_1}
\end{pmatrix}
\begin{pmatrix}
\cos{\theta_2} & \sin{\theta_2} \\
\sin{\theta_2} & -\cos{\theta_2}
\end{pmatrix} 
\\
& = 
\begin{pmatrix}
\cos{\theta_1} \cos{\theta_2} + \sin{\theta_1} \sin{\theta_2} &  \cos{\theta_1} \sin{\theta_2} - \sin{\theta_1} \cos{\theta_2}
\\
\sin{\theta_1} \cos{\theta_2} - \cos{\theta_1} \sin{\theta_2} &  \sin{\theta_1} \sin{\theta_2} + \cos{\theta_1} \cos{\theta_2} 
\end{pmatrix}
\\
& = 
\begin{pmatrix}
\cos{(\theta_1 - \theta_2)} & - \sin{(\theta_1 - \theta_2)} \\
\sin{(\theta_1 - \theta_2)} & \cos{(\theta_1 - \theta_2)}
\end{pmatrix}
= A_{\theta_1 - \theta_2}
\end{align*}
so $B_{\theta}^2 = A_{0} = I$ and thus $B_{\theta}^{-1} = B_{\theta}$. Now, $R^{-1} A_{\theta} R = R A_{\theta} R = R B_{\theta}$ and,
\[ R B_\theta = 
\begin{pmatrix}
1 & 0 \\
0 & -1
\end{pmatrix} 
\begin{pmatrix}
\cos{\theta} & \sin{\theta}\\
\sin{\theta} & -\cos{\theta}
\end{pmatrix} 
=
\begin{pmatrix}
\cos{\theta} &  \sin{\theta}\\
-\sin{\theta} &  \cos{\theta}
\end{pmatrix} 
= A_{-\theta}
\] 
Therefore, $R^{-1} A_{\theta} R = R B_{\theta} = A_{-\theta}$. Thus,
$B_{\theta} B_{\theta} = A_{\theta} R A_{\theta} R = A_{\theta} A_{- \theta} = I$ since $R = R^{-1}$.  
\item[(iii)]
As calculated above, $A_{\theta_1} A_{\theta_2} = A_{\theta_1 + \theta_2}$ and $B_{\theta_1} B_{\theta_2} = A_{\theta_1 - \theta_2}$. Then,
\[A_{\theta_1} B_{\theta_2} = A_{\theta_1} A_{\theta_2} R = A_{\theta_1 + \theta_2} R = B_{\theta_1 + \theta_2}\]
Likewise,
\[B_{\theta_1} A_{\theta_2} = A_{\theta_1} R A_{\theta_2} = A_{ \theta_1} A_{-\theta_2} R =  A_{\theta_1 - \theta_2} R = B_{\theta_1 - \theta_2}\]
Finally, 
\[A_{\theta} R A_{\theta}^{-1} = A_{\theta} R A_{-\theta} = A_{\theta} A_{\theta} R = A_{2 \theta} R = B_{2 \theta}\] 

\item[(iv)]
\renewcommand{\u}{\mathbf{u}}
\newcommand{\e}{\mathbf{e}}
\newcommand{\Ahalf}{A_{\theta/2}}

Let $\u_1 = \Ahalf \e_1$ and $\u_2 = \Ahalf \e_2$. Since $\{\e_1, \e_2\}$ is an orthonormal basis and $A_\theta^{-1} = A_{-\theta} = A_{\theta}^\top$ so $A$ is an orthogonal matrix and thus $\{\u_1, \u_2\}$ is an orthonormal basis. Then, \[B_\theta \u_1 = \Ahalf R \Ahalf^{-1} \u_1 = \Ahalf R \e_1 = \Ahalf e_1 = \u_1\] Likewise,    
\[B_\theta \u_2 = \Ahalf R \Ahalf^{-1} \u_2 = \Ahalf R \e_2 = - \Ahalf e_2 =  - \u_2\]
Therefore, $B_{\theta}$ represents a reflection. Consider the eigenvalues of $A_\theta$ which must satisfy,
\[ \det{(I \lambda - A_\theta)} = \det{
\begin{pmatrix}
\lambda - \cos{\theta} & \sin{\theta} \\
- \sin{\theta} & \lambda - \cos{\theta}
\end{pmatrix}}
= (\lambda - \cos{\theta})^2 + \sin^2{\theta} = \lambda^2 - (2 \cos{\theta}) \lambda + 1 = 0 \] 
Therefore,
\[ \lambda = \cos{\theta} \pm \sqrt{\cos^2{\theta} - 1} \]
If $\lambda$ is to be real then $\cos^2{\theta} - 1 \ge 0$ so $\cos{\theta} = \pm 1$. Therefore, only rotations by $\theta = 0, \pi$ have real eigenvectors. This corresponds to either the identity transformation which fixes every vector or the transformation which reflects through the origin. In the second case, any vector $v$ is taken to $-v$ so every vector is an eigenvector.  
\end{enumerate}

\section*{Problem 2.}

Let $A \in \mathbb{M}_n(\C)$. Then $(A^*)_{ij} = \bar{A}_{ji}$. Therefore,
\[ \tr{A^*} = \sum_{i = 1}^n (A^*)_{ii} = \sum_{i = 1}^n \bar{A}_{ii} = \overline{\sum_{i = 1}^n A_{ii}} = \overline{\tr{A}} \]

\section*{Problem 3.}

Given any $\alpha \in \C$ such that $\alpha \bar{\alpha} = 1$. Then take $z = \alpha^{1/n}$. In particular, since $\alpha \in S^1$ we can take $\alpha = e^{i \theta}$ then take $z = e^{i \theta/n}$. Now, consider the diagonal matrix, $U = \mathrm{diag}(z, \cdots, z)$. This matrix is unitary because, \[U^* = \mathrm{diag}(\bar{z}, \cdots, \bar{z}) = \mathrm{diag}(\tfrac{1}{z}, \cdots, \tfrac{1}{z}) = U^{-1}\]
However,  $\det{U} = z^n = (e^{i \theta / n} )^n = e^{i \theta} = z$. 

\section*{Problem 4.}

Take any $U \in SU(2)$. For $\alpha, \beta, \gamma, \delta \in \C$ we can write,
\[ 
U =
\begin{pmatrix}
\alpha & \beta \\
\gamma & \delta 
\end{pmatrix}
\quad \text{ and thus } \quad 
U^* = 
\begin{pmatrix}
\bar{\alpha} & \bar{\gamma} \\
\bar{\beta} & \bar{\delta} 
\end{pmatrix}
\]
Using the identity, $U U^{*} = I$, we have,
\[ 
UU^* =
\begin{pmatrix}
\alpha & \beta \\
\gamma & \delta 
\end{pmatrix}
\begin{pmatrix}
\bar{\alpha} & \bar{\gamma} \\
\bar{\beta} & \bar{\delta} 
\end{pmatrix}
= 
\begin{pmatrix}
\alpha \bar{\alpha} + \beta \bar{\beta} & \alpha \bar{\gamma} + \beta \bar{\delta} \\
\gamma \bar{\alpha} + \delta \bar{\beta} & \gamma \bar{\gamma} + \delta \bar{\delta} 
\end{pmatrix}
= 
\begin{pmatrix}
1 & 0 \\
0 & 1
\end{pmatrix}
\]
Therefore, $\alpha \bar{\alpha} + \beta \bar{\beta} = 1$ and $\gamma \bar{\gamma} + \delta \bar{\delta} = 1$ and $ \alpha \bar{\gamma} + \beta \bar{\delta} = 0$ and the last equality follows from the conugate of the previous. If $\alpha \neq 0$ then write $\delta = \bar{\alpha} r$ so $\alpha \bar{\delta} + \beta \bar{\alpha} \bar{r} = 0$ then $\gamma = - \bar{\beta} r$. However, $U \in SU(2)$ so $\det{U} = 1$ and therefore, $\alpha \delta - \beta \gamma = |\alpha|^2 r + |\beta|^2 r = 1$ but $|\alpha|^2 + |\beta|^2 = 1$ so $r = 1$. Thus, $\delta = \bar{\alpha}$ and $\gamma = - \bar{\beta}$. \bigskip \\ If $\alpha = 0$ then $|\beta| = 1$ and $\beta \bar{\delta} = 0$ so $\delta = 0$. However, $\det{U} = 1$ so $-\beta \gamma = 1$ and therefore $\gamma = -\bar{\beta}$. In either case,
\[
U = 
\begin{pmatrix}
z & - \bar{w} \\
w & \bar{z} 
\end{pmatrix} \]
where $|z|^2 + |w|^2 = 1$. 
\section*{Problem 5.}

\begin{enumerate}
\item[(i)] Define the matrix,
\[ A =
\begin{pmatrix}
\cos{(2 \pi /n)} & - \sin{(2 \pi / n)} \\
\sin{(2 \pi / n)} & \cos{(2 \pi / n)}
\end{pmatrix}\]
The characteristic polynomial of $A$ is given by,
\begin{align*}
\det{(\lambda I - A)} & = \det{
\begin{pmatrix}
\lambda - \cos{(2 \pi /n)} &  \sin{(2 \pi / n)} \\
-\sin{(2 \pi / n)} & \lambda - \cos{(2 \pi / n)}
\end{pmatrix} }
\\
& = (\lambda - \cos{(2 \pi / n)})^2 + \sin^2{(2 \pi / n)} = \lambda^2 - (2 \cos{(2 \pi / n)}) \lambda + 1
\end{align*}
Therefore, the eigenvalues of $A$ are,
\[ \lambda = \cos{(2 \pi / n)} \pm \sqrt{ \cos^2{(2 \pi /n)} - 1}  = \cos{(2 \pi / n)} \pm i \sin{(2 \pi / n)} = e^{\pm i 2 \pi / n} \]
For these eigenvalues, we can find eigenvectors which span the null spaces of $I \lambda - A$ i.e.
\begin{align*}
(I e^{i 2\pi /n} - A)v = 
\begin{pmatrix}
i \sin{(2 \pi / n)} & \sin{(2 \pi / n)} \\
-\sin{(2 \pi / n)} &  i \sin{(2 \pi / n)}
\end{pmatrix}
\begin{pmatrix}
a \\
b
\end{pmatrix}
= 0 
\end{align*}
so we can take $a = i$ and $b = 1$. Likewise,
\begin{align*}
(I e^{- i 2\pi /n} - A)v = 
\begin{pmatrix}
-i \sin{(2 \pi / n)} & \sin{(2 \pi / n)} \\
-\sin{(2 \pi / n)} &  -i \sin{(2 \pi / n)}
\end{pmatrix}
\begin{pmatrix}
a \\
b
\end{pmatrix}
= 0 
\end{align*}
so we can take $a = 1$ and $b = i$. Thus, we can take $v_1, v_2 \in \C^2$ given by,
\[ v_1 = 
\begin{pmatrix}
i \\
1
\end{pmatrix}
\quad \quad v_2 =
\begin{pmatrix}
1 \\
i
\end{pmatrix}
\]
which each span a $A$ eigenspace. Define two $\Z /n \Z$-representations, $\rho_0 : \Z /n \Z \to \aut{\C^2}$ given by $\rho(k) = A^k$ and $\rho_1 : \Z / n \Z \to \aut{V_1 \oplus V_2}$ given by $\rho_1(k)(v_1) = e^{2 \pi i k / n} v_1$ and $\rho_1(k)(v_2) = e^{ - 2 \pi i k / n} v_2$ where $V_1 = \C \cdot v_1$ and $V_2 =  \C \cdot v_2$. \bigskip \\
Now, define the $\C$-linear map $F : \C^2 \to V_1 \oplus V_2$ given by $F(v_1) = (v_1, 0)$ and $F(v_2) = (0, v_2)$. Then, for any $v \in \C^2$ we can write $v = c_1 v_1 + c_2 v_2$ in the basis $\{v_1, v_2\}$. Then, 
\begin{align*}
F(\rho_0(k) v) & = F(A (c_1 v_1 + c_2 v_2)) = c_1 F(A v_1) + c_2 F(A v_2) = c_1  F(e^{2 \pi i k / n} v_1) + v_2 F(e^{- 2 \pi i k / n} v_2)  
\\
& = c_1 e^{2 \pi i k / n} F(v_1) + c_2 e^{- 2 \pi i k / n} F(v_2) = (c_1 e^{2 \pi i k / n} v_1, 0) + (0, c_2 e^{- 2 \pi i k / n} v_2)
\\
& = (c_1 e^{2 \pi i k / n} v_1, c_2 e^{ - 2 \pi i k / n} v_2)
\end{align*}
Likewise,
\begin{align*}
\rho_1(k) F(v) & = \rho_1(k) (c_1 F(v_1) + c_2 F(v_2)) = \rho_1(k) (c_1 (v_1, 0) + c_2 (0, v_2)) 
\\
& = \rho_1(k) (c_1 v_2, c_2 v_2) = (c_1 e^{2 \pi i k / n} v_1, c_2 e^{ - 2 \pi i k / n} v_2)
\end{align*}
Therefore, $F(\rho_0(k) v) = \rho_1(k) F(v)$ but $F : \C^2 \to V_1 \oplus V_2$ is isomorphic because $V_1 \cap V_2 = \{ 0 \}$ and $V_1 + V_2 = \C^2$. Thus, $F$ is a $\Z / n \Z$-isomorphism.
\item[(ii)]
For $n > 2$ the eigenvalues $e^{\pm 2 \pi i / n}$ are not equal because if $2 \pi / n = - 2 \pi / n  + 2 \pi k$ for $k \in \Z$ then $k = 2 / n$ so $n < 2$. Therefore, the vectors $v_1$ and $v_2$ must lie in different eigenspaces (Since $A$ acting on them gives a different value). Furthermore, the eigenspaces must be one-dimensional (because each is nonzero and not the full space which is two-dimensional) so their spanning sets are unique up to a scalar. Therefore, the basis $\{ v_1, v_2 \}$ of $A$ eigenvectors is unique up to order and scaling. Furthermore, the vectors $v_1$ and $v_2$ are not eigenvectors of 
\[ R = 
\begin{pmatrix}
1 & 0 \\
0 & -1
\end{pmatrix} \]
because $R v_1 = i v_2$ and $R v_2 = - i v_1$ but $\{ v_1, v_2 \}$ are indpendent. Therefore, $A$ and $R$ have no common eigenvectors. However, any nontrivial invariant subspace of the representation of $D_n$ mapping to $A$ and $R$ must be one-dimensional simply because $\C^2$ is two-dimensional. A one-dimensional invariant subspace is exactly equivalent to a common eigenvector which we know $A$ and $R$ do not have. Thus, this representation of $D_n$ is irreducible for $n > 2$. \bigskip \\
For $n = 2$, the eigenvalues $e^{ \pm 2 \pi i / 2} = - 1$ are equal. For $n = 2$, we have $A = - I$ so any vector is an eigenvector of $A$. Thus, $A$ and $R$ have a two common eigenvectors $e_1$ and $e_2$ so the representation of $D_2$ is reducible.       
\end{enumerate}

\section*{Problem 6.}

Consider the representation of $D_3 \cong S_3$ generated by the matrices,
\[ A = 
\begin{pmatrix}
-1/2 & - \sqrt{3}/2 \\
\sqrt{3}/2 & -1/2
\end{pmatrix}
\quad \quad 
R = 
\begin{pmatrix}
1 & 0 \\
0 & -1
\end{pmatrix}\]
Furthermore, consider the permutation representation over the subspace $W \subset \C^3$
\[ W = \{(t_1, t_2, t_3) \mid t_1 + t_2 + t_3 = 0 \} \]
with the group $S_3$ generated by $\sigma = (1 \: 2 \: 3)$ and $\tau = (2 \: 3)$. Let $w_1 = e_1 - e_2 \in W$ and $w_2 = e_2 - e_3 \in W$ then $w_1' = e_1 - \tfrac{1}{2}e_2 - \tfrac{1}{2} e_3 \in W$ is an eigenvector of $\rho(\tau)$ because, 
\[\rho(\tau)( e_1 - \tfrac{1}{2}e_2 - \tfrac{1}{2} e_3 ) = e_1 - \tfrac{1}{2}e_3 - \tfrac{1}{2} e_2 = e_1 - \tfrac{1}{2}e_2 - \tfrac{1}{2} e_3\]
Furthermore, $e_1 - \tfrac{1}{2}e_2 - \tfrac{1}{2} e_3 \notin \vspan{e_2 - e_3}$ so the vectors $w_2$ and $w_1'$ are independend and therefore form a basis because $\dim{W} = 2$. However, $\rho(\tau)(w_2) = \rho(\tau)(e_2 - e_3) = e_3 - e_2 = - w_2$. Furthermore, 
\[\rho(\sigma) (w_1') = \rho(\sigma)(e_1 - \tfrac{1}{2}e_2 - \tfrac{1}{2} e_3) = e_2 - \tfrac{1}{2}e_3 - \tfrac{1}{2} e_1 = - \tfrac{1}{2} (e_1 - \tfrac{1}{2} e_2 - \tfrac{1}{2} e_3) + \tfrac{3}{4} (e_2 - e_3) = - \tfrac{1}{2} w_1' + \tfrac{3}{4} w_2 \]    
Similarly,
\[\rho(\sigma) (w_2) = \rho(\sigma)(e_2 - e_3) = e_3 - e_1 = - (e_1 - \tfrac{1}{2} e_2 - \tfrac{1}{2} e_3) - \tfrac{1}{2} (e_2 - e_3) = - w_1' - \tfrac{1}{2} w_2 \]
Let $w_2' = \frac{\sqrt{3}}{2} w_2$ then $\rho(\sigma)(w_1') = -\tfrac{1}{2} w_1' + \frac{\sqrt{3}}{2} w_2'$ and $\rho(\sigma)(w_2') = - \frac{\sqrt{3}}{2} w_1' - \tfrac{1}{2} w_2'$. Therefore, $\rho(\sigma)$ in the basis $\{w_1', w_2'\}$ is given by the matrix,
\[
A = 
\begin{pmatrix}
- 1/2 & \sqrt{3}/2 \\
-\sqrt{3}{2} & - 1/2
\end{pmatrix}\]
and likewise, $\rho(\tau)(w_1') = w_1'$ and $\rho(\tau)(w_2') = - w_2'$ so $\rho(\tau)$ is given by the matrix,
\[
R = 
\begin{pmatrix}
1 & 0 \\
0 & - 1
\end{pmatrix}\]
\section*{Problem 7.}

Let $V$ and $W$ be $G$-representations and let $F : V \to W$ be a $G$-morphism. Take any $g \in G$. Take, $v \in \ker{F}$. Then, $F(v) = 0$ and thus, $\rho_W(g)(F(v)) = F(\rho_V(g)(v)) = 0$ so $\rho_V(g)(v) \in \ker{F}$. Therefore, $\ker{F}$ is invariant under the action of $\rho_V(g)$ for any $g \in G$. Therefore, $\ker{K}$ is a $G$-invariant subspace of $V$. Similarly, take $w \in \Im{F}$. Then there exists $v \in V$ such that $F(v) = w$. Therefore, $\rho_W(g)(w) = \rho_W(g)(F(v)) = F(\rho_V(g)(v)) \in \Im{F}$. Therefore, $\rho_V(g)(\Im{K}) \subset \Im{F}$ so $\Im{F}$ is a $G$-invariant subspace of $W$. 


\end{document}
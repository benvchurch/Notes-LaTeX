\documentclass[12pt]{extarticle}
\usepackage[utf8]{inputenc}
\usepackage[english]{babel}
\usepackage[a4paper, total={7in, 9.5in}]{geometry}
 
\usepackage{amsthm, amssymb, amsmath, centernot}

\newcommand{\notimplies}{%
  \mathrel{{\ooalign{\hidewidth$\not\phantom{=}$\hidewidth\cr$\implies$}}}}
 
\renewcommand\qedsymbol{$\square$}
\newcommand{\cont}{$\boxtimes$}
\newcommand{\divides}{\mid}
\newcommand{\ndivides}{\centernot \mid}
\newcommand{\Z}{\mathbb{Z}}
\newcommand{\N}{\mathbb{N}}
\newcommand{\C}{\mathbb{C}}
\newcommand{\Zplus}{\mathbb{Z}^{+}}
\newcommand{\Primes}{\mathbb{P}}
\newcommand{\ball}[2]{B_{#1} \! \left(#2 \right)}
\newcommand{\Q}{\mathbb{Q}}
\newcommand{\R}{\mathbb{R}}
\newcommand{\Rplus}{\mathbb{R}^+}
\newcommand{\invI}[2]{#1^{-1} \left( #2 \right)}
\newcommand{\End}[1]{\text{End}\left( A \right)}
\newcommand{\legsym}[2]{\left(\frac{#1}{#2} \right)}
\renewcommand{\mod}[3]{\: #1 \equiv #2 \: \mathrm{mod} \: #3 \:}
\newcommand{\nmod}[3]{\: #1 \centernot \equiv #2 \: \mathrm{mod} \: #3 \:}
\newcommand{\ndiv}{\hspace{-4pt}\not \divides \hspace{2pt}}
\newcommand{\finfield}[1]{\mathbb{F}_{#1}}
\newcommand{\finunits}[1]{\mathbb{F}_{#1}^{\times}}
\newcommand{\ord}[1]{\mathrm{ord}\! \left(#1 \right)}
\newcommand{\quadfield}[1]{\Q \small(\sqrt{#1} \small)}
\newcommand{\vspan}[1]{\mathrm{span}\! \left\{#1 \right\}}
\newcommand{\galgroup}[1]{Gal \small(#1 \small)}
\newcommand{\ints}[1]{\mathcal{O}_{#1}}
\newcommand{\sm}{\! \setminus \!}
\renewcommand{\Im}[1]{\mathrm{Im} \: #1}
\newcommand{\ch}[1]{\mathrm{char} \: #1}
\renewcommand{\Im}[1]{\mathrm{Im}(#1)}
\newcommand{\minimal}[2]{\mathrm{Min}(#1;#2)}
\newcommand{\fix}[2]{\mathrm{Fix}_{#1} (#2)}
\newcommand{\id}{\mathrm{id}}
\newcommand{\Disc}[1]{\mathrm{Disc}(#1)}
\newcommand{\sgn}[1]{\mathrm{sgn}(#1)}
\newcommand{\tr}[1]{\mathrm{Tr} \: #1}
\newcommand{\Hom}[2]{\mathrm{Hom}\left( #1, #2 \right)}
\newcommand{\aut}[1]{\mathrm{Aut}\left( #1 \right)}
\newcommand{\repHom}[3]{\mathrm{Hom}^{#1} \left( #2, #3 \right)}
\newcommand{\Sym}[1]{\mathrm{Sym}(#1)}
\newcommand{\GL}[2]{\mathrm{GL}(#1, #2)}
\newcommand{\inner}[2]{\left< #1, #2 \right>}

\newcommand{\Res}[3]{\mathrm{Res}^{#1}_{#2} \, #3}

\newcommand{\Ind}[3]{\mathrm{Ind}^{#1}_{#2} \, #3}

\renewcommand{\theenumi}{(\alph{enumi})}

\newcommand{\atitle}[1]{\title{% 
	\large \textbf{Mathematics GU4044 Representations of Finite Groups 
	\\ Assignment \# #1} \vspace{-2ex}}
\author{Benjamin Church }
\maketitle}

 
\newtheorem{theorem}{Theorem}[section]
\newtheorem{lemma}[theorem]{Lemma}
\newtheorem{proposition}[theorem]{Proposition}
\newtheorem{corollary}[theorem]{Corollary}

\begin{document}
\atitle{11}

\section*{Problem 1.}
\begin{enumerate}
\item[(i)]
Let $V$ be the 3-dimensional $A_4$-representation. 
$A_4$ has four conjugacy classes,
\[ [e] \quad [(1 \: 2 \: 3)] \quad [(1 \: 3 \: 2)] \quad [(1 \: 2) (3 \: 4)] \]
Therefore, using the fact that $\chi_V = \chi_{st.}  - 1$.
\[ \inner{\psi_2(\chi_V)}{1} = \frac{1}{\#(A_4)} \sum_{g \in A_4} \chi_V(g^2) = \frac{1}{12} \left[ 1 \cdot 3 + 4 \cdot 0 + 4 \cdot 0 + 3 \cdot 3 \right] = 1 \]
\item[(ii)]
Let $G = D_n = \left< r, s \mid r^n = s^2 = e, rs = sr^{-1} \right>$. Any element in $D_n$ can be written in the form $r^i s^j$ for $0 \le i < n$ and $0 \le j \le 1$. Any element of the form $r^i s$ satisfies $(r^i s)^2 = r^i s r^i s = r^i r^{-i} s^2 = e$. Furthermore, $(r^i)^2 = r^{2i}$. Let $V$ be a $2$-dimensional irreducible $D_n$-representation. Then, $\chi_V(r^i s) = \chi_V(e) = 2$. However, 
\[ | \chi_V(g^2) | \le \dim{V} = 2 \]
Therefore, 
\[ \inner{\psi_2(\chi_V)}{1} = \frac{1}{|D_n|} \sum_{g \in D_n} \chi_V(g^2) = \frac{1}{2n} \left[ 2 \cdot (n + 1) + \sum_{i = 0}^{n-1} \chi_V(r^{2 i}) \right] > 0 \]
because,
\[ \left| \sum_{i = 0}^{n-1} \chi_V(r^{2 i}) \right| < 2n \]
This implies that $V$ is defined over $\R$.

\end{enumerate}

\section*{Problem 2.}

\begin{enumerate}
\item[(i)] Let $V$ be a complex vector space and let $\psi : V \to V^*$ be a conjugate linear map which is an isomorphism as a real map. Define the map,
\[ \gamma : V \oplus V^* \to V \oplus V^* \]
via $\gamma(v, \ell) = (\psi^{-1}(\ell), \psi(v))$. I claim that $\gamma$ is a conjugate-linear involution. First,
\begin{align*}
\gamma(\alpha v_1 + \beta v_2, \alpha \ell_1 + \beta \ell_2) & = (\psi^{-1}(\alpha \ell_1 + \beta \ell_2), \psi(\alpha v_1 + \beta v_2)) 
\\
& = (\bar{\alpha} \psi^{-1}(\ell_1) + \bar{\beta} \psi^{-1}(\ell_2), \bar{\alpha} \psi(v_1) + \bar{\beta} \psi(v_2) ) 
\\
& = \bar{\alpha} (\psi^{-1}(\ell_1), \psi(v_1)) + \bar{\beta} (\psi^{-1}(\ell_2), \psi(v_2)) 
\end{align*}
Furthermore,
\[ \gamma^2(v, \ell) = \gamma(\psi^{-1}(\ell), \psi(v)) = (\psi^{-1}(\psi(v)), \psi(\psi^{-1}(\ell))) = (v, \ell) 
\]
since $\psi$ is a bijection. Thus, $\gamma$ is a conjugate-linear involution which we have shown defines a complex structure on $V \oplus V^*$.
 
\item[(ii)]
Let $V$ be a $G$-representation and $H$ a $G$-invariant positive definite Hermitian form. Define the map $\psi : V \to V^*$ by $\psi(v)(w) = H(w, v)$. Then, $\psi(\alpha v + \beta u)(w) = H(w, \alpha v + \beta u) = \bar{\alpha} H(w, v) + \bar{\beta} H(w, u)$. Furthermore, $V$ and $V^*$ have the same dimension since $V$ is finite dimensional. Therefore, it suffices to show that $\psi$ is an injection. Suppose that $\psi(v) = 0$ then $H(w, v) = 0$ for all $w$. However, $H(v, v) = 0 \iff v = 0$ so $v = 0$. Thus, $\phi$ is a bijection and thus an isomorphism as a real-linear map. Therefore, by $(i)$ the map $\psi$ gives rise to a conjugate-linear $\gamma$ involution $\gamma$ which puts a real structure on $V \oplus V^*$. However, $H$ is $G$-invariant so $\psi(g \cdot v)(g \cdot w) = H(g \cdot w)(g \cdot v) = H(w)(v) = \psi(v)(w)$. Thus, $\rho_{V^*}(g)^{-1} \circ \psi \circ \rho_{V}(g) = \psi$ so $\psi$ is a $G$-morphism between $V$ and $V^*$. Therefore,
\begin{align*}
\rho_{(V \oplus V^*)}(g)^{-1} \circ \gamma \circ \rho_{(V \oplus V^*)}(g) (v, \ell) & = (\rho_V(g)^{-1} \circ \psi^{-1}(\rho_{V^*}(g) \cdot \ell), \rho_{V^*}(g)^{-1} \circ \psi(\rho_V(g) \cdot v))
\\
& = (\psi^{-1}(\ell), \psi(v)) 
\end{align*}
Therefore, $\gamma$ is a $G$-morphism i.e. $\gamma$ commutes with the $G$-representation on $V \oplus V^*$. Thus, $V \oplus V^*$ is defined over $\R$ as a $G$-representation.  

\end{enumerate}

\section*{Problem 3.}

Let $p$ and $q$ be odd primes with $p < q$ and $\mod{q}{1}{p}$. Let $G$ be a nonabelian group of order $pq$. From HW. 8 problem 2 we know that $G$ has $h = \tfrac{1}{p}(q - 1) + p$ conjugacy classes. Write $q = pr + 1$ for some integer $r$. Thus, $h = r + p$. Now,
\[ pq - p = p(q-1) = p^2 r \]
however, since $p$ is odd, $\mod{p^2}{1}{8}$. However, $r = \tfrac{1}{p} ( q - 1 )$ is even because $p$ is odd and $q - 1$ is even. Therefore, $r = 2 s$. But $\mod{2 p^2}{2}{16}$ since $8 \divides p^2 - 1$ and thus $16 \divides 2 p^2 - 2$. Thus, $\mod{p^2 (2s)}{2s}{16}$. Therefore, $\mod{p^2 r}{r}{16}$ so $\mod{pq - p}{r}{16}$ and thus,
\[ \mod{pq}{r + p = h}{16} \]


\end{document}
\documentclass[12pt]{article}
\usepackage{import}
\import{./}{AlgebraCommands}

\newcommand{\A}{\mathcal{A}}
\newcommand{\B}{\mathcal{B}}
\newcommand{\C}{\mathcal{C}}
\newcommand{\D}{\mathcal{D}}

\renewcommand{\a}{\mathfrak{a}}
\newcommand{\m}{\mathfrak{m}}

\newcommand{\Set}{\mathbf{Set}}
\newcommand{\Ab}{\mathbf{Ab}}

\begin{document}

\section{The Yoneda Embedding}

\begin{lemma}
Let $\eta : \Hom{}{A}{-} \to \Hom{}{B}{-}$ be a natural transformation. Then $\eta$ is uniquely determined by $\eta_A(\id_A)$ via $\eta_X(f) = f \circ \eta_A(\id_A)$ for any $f \in \Hom{}{A}{X}$.  
\end{lemma}

\begin{proof}
Let $f : A \to X$ be some map.
Consider the naturality diagram,
\begin{center}
\begin{tikzcd}
\Hom{}{A}{A} \arrow[r, "f_*"] \arrow[d, "\eta_A"] & \Hom{}{A}{X} \arrow[d, "\eta_X"] 
\\
\Hom{}{B}{A} \arrow[r, "f_*"] & \Hom{}{B}{X} 
\end{tikzcd}
\end{center}
Consider the element $\id_A \in \Hom{}{A}{A}$ which, under the upper path, maps to $\eta_X (f_*(\id_A)) = \eta_X(f \circ \id_A) = \eta_X(f)$ and, under the lower path, $f_*(\eta_A(\id_A)) = f \circ \eta_A(\id_A)$. Therefore,
\[ \eta_X(f) = f \circ \eta_A(\id_A) \]
\end{proof}

\begin{corollary}
Natural transformations $\eta : \Hom{}{A}{-} \to \Hom{}{B}{-}$ are in one-to-one correspondence with functions $\Hom{}{B}{A}$. We say $f^*$ is the natural transformation $f^*_X(g) = g \circ f$ for any $g \in \Hom{}{A}{X}$.  
\end{corollary}

\begin{theorem}
Let $\C$ be any category. The functor $Y : \C^{\op} \to \Set^{\C}$ sending $A \mapsto h^A$ where $h^A = \Hom{}{A}{-}$ and $f \mapsto f^*$ described above is fully faithful.
\end{theorem}

\begin{proof}
Clearly $(\id_A)^* = \id_{h^A}$ since $(\id_A)^*(f) = f \circ \id_A = f$ and for $f \in \Hom{}{B}{A}$ and $g \in \Hom{}{C}{B}$ then $(f \circ g)^* = g^* \circ f^*$ since for any $q \in \Hom{}{A}{X}$ we send,
\[ (f \circ g)^*(q) = q \circ (f \circ g) = (q \circ f) \circ g = g^*(f^*(q)) \]
The above corollary proves that $Y$ is fully faithful.    
\end{proof}

\begin{lemma}
Let $F : \C \to \D$ be fully faithful then $X \cong Y \iff F(X) \cong F(Y)$.
\end{lemma}

\begin{proof}
If $F(X) \cong F(Y)$ then there are morphisms $f \in \Hom{}{F(X)}{F(Y)}$ and $g \in \Hom{}{F(Y)}{F(X)}$ which are inverses. However, since $F$ is full there exist morphisms $\tilde{f} : \Hom{}{X}{Y}$ and $g \in \Hom{}{Y}{X}$ such that $F(\tilde{f}) = f$ and $F(\tilde{g}) = g$. Then,
\[ F(\tilde{f} \circ \tilde{g}) = F(\tilde{f}) \circ F(\tilde{g}) = f \circ g = \id_{F(Y)} \quad \text{and} \quad F(\tilde{g} \circ \tilde{f}) = F(\tilde{g}) \circ F(\tilde{f}) = g \circ f = \id_{F(X)} \]
However, since $F$ is faithful then,
\[ \tilde{f} \circ \tilde{g} = \id_Y \quad \text{and} \quad \tilde{g} \circ \tilde{f} = \id_X \]
proving that $X \cong Y$. 
\end{proof}

\begin{definition}
We say a functor $F : \C \to \Set$ is \textit{representable} if $F \cong h^A$ for some $A \in \C$. 
\end{definition}

\section{Additive Categories}

\begin{definition}
A category $\mathcal{C}$ is pre-additive if its hom sets have the structure of an abelian group and composition of maps distributes over addition. Explicitly, for $X, Y, Z \in \mathcal{C}$, there exits a binary operation,
\[ + :  \Hom{}{X}{Y} \times \Hom{}{X}{Y} \to \Hom{}{X}{Y}\]
such that $(\Hom{}{X}{Y}, +)$ is an abelian group and, for $f, g : X \to Y$ and $h, k : Y \to Z$ we have $h \circ (f + g) = h \circ f + h \circ g$ and $(h + k) \circ f = h \circ f + k \circ f$. This is equvalent to the requirement that hom is a functor,
\[ \Hom{}{-}{-} : \mathcal{C}^{\mathrm{op}} \times \mathcal{C} \to \Ab \]
\end{definition}

\begin{lemma}
In a pre-additive cateogory, there exists an identity element $0 \in \Hom{}{X}{Y}$ such that $0 + f = f + 0 = f$ for $f \in \Hom{}{X}{Y}$ and $f \circ 0 = 0$ for $f \in \Hom{}{Y}{Z}$ and $0 \circ f = 0$ for $f \in \Hom{}{Z}{X}$.  
\end{lemma}

\begin{proof}
The hom sets are abelian groups by definiton and thus must have unique identiy elements satisfying $f + 0 = 0 + f = f$ for all $f \in \Hom{}{X}{Y}$. Furthermore, for $f \in \Hom{}{Y}{Z}$ we have $f \circ 0 = f \circ (0 + 0) = f \circ 0 + f \circ 0$ and thus $f \circ 0 = 0_{XZ}$. Furthermore for $f \in \Hom{}{Z}{X}$ we know that $0 \circ f = (0 + 0) \circ f = 0 \circ f + 0 \circ f$ so $0 \circ f = 0_{ZY}$.  
\end{proof}

\begin{definition}
A biproduct of an indexed set $\{X_i\}_I$ is an object $X = \bigoplus_I X_i$ along with projection maps $\pi_i : X \to X_i$ and inclusion maps $\iota_i : X_i \to X$ such that $(X, \{ \pi_i \}_I)$ is the product of $\{X_i\}_I$ and $(X, \{ \iota_i \}_I )$ is the coproduct of $\{ X_i \}_I$.  
\end{definition}



\begin{proposition}
Let $\mathcal{C}$ be a pre-additive category. Every finite product and finite coproduct is a biproduct. In particular, finite products and coproducts are equal. 
\end{proposition}

\begin{proof}
Let $X \times Y$ be the product of $X$ and $Y$. Consider the diagram,
\begin{center}
\begin{tikzcd}
X \arrow[rr, "\id_X"] \arrow[rd, "\iota_X"] & & X
\\
& X \times Y \arrow[ru, "\pi_X"] \arrow[rd, "\pi_Y"] &
\\
Y \arrow[rr, "\id_Y"] \arrow[ru, "\iota_Y"] & & Y 
\end{tikzcd}
\end{center}
where the maps $\iota_X : X \to X \times Y$ and $\iota_Y : Y \to X \times Y$ are defined via the universal property of the product applied to $(\id_X, 0)$ and $(0, \id_Y)$ respectivly where $0 \in \Hom{}{X}{Y}$ is the identiy element of the abelian group. The universal property gives,
\begin{align*}
\pi_X \circ \iota_X = \id_X &\quad \pi_Y \circ \iota_X = 0
\\
\pi_X \circ \iota_Y = 0 &\quad \pi_Y \circ \iota_Y = \id_Y 
\end{align*}
so the diagram commutes. We need to show that $X \times Y$ is universal with respect to the maps $\iota_X$ and $\iota_Y$. Suppose we have maps $f_X : Z \to X$ and $f_Y : Z \to Y$ then define $\tilde{f} = f_X \circ \pi_X + f_Y \circ \pi_Y$.
\begin{center}
\begin{tikzcd}
& X \arrow[ld, "f_X"'] \arrow[rr, "\id_X"] \arrow[rd, "\iota_X"] & & X
\\
Z & & X \times Y  \arrow[ll, dashed, "\tilde{f}"'] \arrow[ru, "\pi_X"] \arrow[rd, "\pi_Y"] &
\\
& Y \arrow[lu, "f_Y"'] \arrow[rr, "\id_Y"] \arrow[ru, "\iota_Y"] & & Y 
\end{tikzcd}
\end{center}
This map satisfies the required universal property because,
\[ \tilde{f} \circ \iota_X = (f_X \circ \pi_X + f_Y \circ \pi_Y) \circ \iota_X = f_X \circ \pi_X \circ \iota_X + f_Y \circ \pi_Y \circ \iota_X = f_X + 0 = f_X \]
and likewse,
\[ \tilde{f} \circ \iota_Y = (f_X \circ \pi_X + f_Y \circ \pi_Y) \circ \iota_Y = f_X \circ \pi_X \circ \iota_Y + f_Y \circ \pi_Y \circ \iota_Y = 0 + f_Y = f_Y \]
Lastly, we must show that $\tilde{f}$ is unique. Suppose there exits a map $\tilde{f} : X \times Y \to Z$ such that $\tilde{f} \circ \iota_X = f_X$ and $\tilde{f} \circ \iota_Y = f_Y$. Consider the map $I : X \times Y \to X \times Y$ given by,
\[ I = \iota_X \circ \pi_X + \iota_Y \circ \pi_Y \]
Therefore, 
\[ \pi_X \circ I = \pi_X \circ \iota_X \circ \pi_X + \pi_X \circ \iota_Y \circ \pi_Y = \pi_X + 0 = \pi_X \]
and furthermore,
\[ \pi_Y \circ I = \pi_Y \circ \iota_X \circ \pi_X + \pi_Y \circ \iota_Y \circ \pi_Y = 0 + \pi_Y = \pi_Y \]
However, by the universal property of the product, there exists a unique map, namely $\id_{X \times Y}$, satisfying these properties. Thus, $I = \id_{X \times Y}$. Thus,
\[ \tilde{f} = \tilde{f} \circ \id_{X \times Y} = \tilde{f} \circ I = \tilde{f} \circ \iota_X \circ \pi_X + \tilde{f} \circ \iota_Y \circ \pi_Y  = f_X \circ \pi_X + f_Y \circ \pi_Y \]
so the map we constructed earlier is unique. 
\bigskip\\
Similarly, let $X \coprod Y$ be the coproduct of $X$ and $Y$. A similar argument will hold reversing all arrows. 
\end{proof}

\begin{rmk}
Additionally, we see that a terminal object $T$ (empty product) is also initial (empty coproduct) because $\Hom{}{T}{X}$ must have a zero element $0 : T \to X$ and for any map $f : T \to X$ we know that $f \circ 0_{TT} = 0_{TX}$ but $0_{TT} = \id_T$ is the unique map $T \to T$ so $f = 0_{TX}$ and thus $T$ is also initial. 
\end{rmk}


\begin{definition}
A category is additive if it is pre-additive, has a zero object, and has all finite biproducts. The preceding dicussion implies that it is enough to check that either all finite products or all finite coproducts exit.  
\end{definition}

\begin{proposition}
In an additive category, the zero map is the indentity obeject of the $\Ab$-enriched hom-sets.
\end{proposition}

\begin{proof}

\end{proof}

\begin{definition}
A functor $T : \C \to \D$ is \textit{additive} if it preserves finite biproducts.  
\end{definition}

\begin{proposition}
A functor $T : \C \to \D$ is additive iff the map on enriched hom-sets,
\[ T_{X,Y} : \Hom{\C}{X}{Y} \to \Hom{\D}{T(X)}{T(Y)} \]
is a homomorphism in the category of abelian groups.
\end{proposition}

\begin{proof}
A biproduct $X \oplus Y$ with its projections and inclusions is completely characterized by the property $\id_{X \oplus Y} = \iota_X \circ \pi_X + \iota_Y \circ \pi_Y$. Thus $T$ preserves the biproduct structure iff it preserves addition i.e. iff,
\[ \id_{T(X \oplus Y)} = T(\id_{X \oplus Y}) = T(\iota_X \circ \pi_X + \iota_Y \circ \pi_Y) = T(\iota_X) \circ T(\pi_X) + T(\iota_Y) \circ T(\pi_Y) \]
\end{proof}

DEF-COMPLEX
PROP ADD-FUNC PRESERVE COMPLEXES

\section{Normality}

\begin{defn}
A morphism $f : X \to Y$ is called,
\begin{enumerate}
\item split on the left (admits a left inverse) if there exists a map $g : Y \to X$ such that $g \circ f = \id_X$
\item split of the right (admits a right inverse) if there exists a map $g : Y \to X$ such that $f \circ g = \id_Y$
\item an isomorphism if there exists a map $g : Y \to X$ such that $g \circ f = \id_X$ and $f \circ g = \id_Y$.
\end{enumerate}
\end{defn}

\begin{lemma}
The following hold in any category:
\begin{enumerate}
\item a morphism split on the left is monic
\item a morphism split of the right is epic
\item a morphism is split both on the left and on the right if and only if it is an isomorphism.
\end{enumerate}
\end{lemma}

\begin{proof}
First, if $f : X \to Y$ admits a left inverse $g : Y \to X$ then for any $a, b : Z \to X$ such that $f \circ a = f \circ b$ we have $g \circ f \circ a = g \circ f \circ b$ and thus $a = b$ since $g \circ f = \id_X$. Dually, if $f : X \to Y$ admits a right inverse $g : Y \to X$ then for any $a, b : Y \to Z$ such that $a \circ f = b \circ f$ we have $a \circ f \circ g = b \circ f \circ g$ and thus $a = b$ since $f \circ g = \id_Y$.
\bigskip\\
Finally, an isomorphism is clearly split on the left and right. To prove the converse, it suffices to show that the left and right inverses agree. Indeed if $f$ has a left inverse $g_L : B \to A$ and right inverse $g_R : B \to A$ such that $g_L \circ f = \id_A$ and $f \circ g_R = \id_B$ then,
\[ g_L = g_L \circ \id_B = g_L \circ (f \circ g_R) = (g_L \circ f) \circ g_R = \id_A \circ g_R = g_R \]
\end{proof}

\begin{rmk}
Due to the previous result, we alternatively call a morphisms split on the left a ``split mono'' and a morphism split on the right a ``split epi''. 
\end{rmk}

\begin{rmk}
Split monos and epis are important because every functor preserves them unlike usual monos and epis.
\end{rmk}

\begin{lemma}
Every equalizer is a monomorphism. Dually, every coequalizer is an epimorphism.
\end{lemma}

\begin{proof}
Let $f, g : X \to Y$ be morphisms and $e : E \to X$ be the equalizer. Given two maps $a,b : E \to X$ such that $q = e \circ a = e \circ b$ clearly we have $f \circ q = g \circ q$ because $f \circ e = g \circ e$. Therefore, $q$ factors \textit{uniquely} through $E$ meaning that $a = b$.
\end{proof}

\begin{corollary}
Every kernel is a monomorphism. Dually, every cokernel is a epimorphism.
\end{corollary}

\begin{defn}
Let $\A$ be a category with zero maps.
\begin{enumerate}
\item a monomorphism is \textit{normal} if it is the kernel of some morphism
\item an epimorphism is \textit{conormal} if it is the cokernel of some morphism
\item $\C$ is \textit{normal} if every monomorphism is the kernel of some morphism
\item $\C$ is \textit{conormal} if every epimorphism is the cokernel of some morphism
\item $\C$ is \textit{binormal} if it is normal and conormal.
\end{enumerate}
\end{defn}

\begin{prop}
Let $\A$ be a category with zero maps and $f : A \to B$ a morphism. Then,
\begin{enumerate}
\item if $f$ is a monomorphism and also a conormal epimorphism then $f$ splits uniquely on the left
\item if $f$ is an epimorphism and also a normal monomorphism then $f$ splits uniquely on the right.
\item if $f$ is a normal mono and also a conormal epi then $f$ is an isomorphism.
\end{enumerate}
\end{prop}

(HERE I THINK ITS ALREADY AN ISOMORPHISM)

\begin{proof}
If $f$ is a conormal epi, it is the cokernel of $a : K \to A$. Thus, $f \circ a = f \circ 0 = 0$ but $f$ is monic so $a = 0$. Therefore, we have a diagram,
\begin{center}
\begin{tikzcd}
K \arrow[r, "0"] & A \arrow[r, "f"] \arrow[d, "\id_A"'] & B \arrow[ld, dashed, "g"]
\\
& A
\end{tikzcd}
\end{center}
where $\id_A \circ a = 0$ because $a = 0$ and thus $\id_A$ factors through the cokernel $f : A \to B$ as $g \circ f$ for a unique $g : B \to A$.
\bigskip\\
The second is exactly the dual statement and thus is an application of the first in $\A^{\op}$. The third follows directly by applying both the previous statements. However, we will spell it out for clarity.
\bigskip\\
Let $f : A \to B$ be a normal mono and a conormal epi meaning $f$ must be a kernel and a cokernel of some maps,
\begin{center}
\begin{tikzcd}[column sep = large, row sep = large]
K \arrow[r, "a"] & A \arrow[r, "f"] & B \arrow[r, "b"] & C
\end{tikzcd}
\end{center}
where $f : A \to B$ is the cokernel of $a : K \to A$ and the kernel of $b : B \to C$. Then $f \circ a = 0 = f \circ 0$ so, since $f$ is monic, $a = 0$. Furthermore, $b \circ f = 0 = 0 \circ f$ so, since $f$ is an epic, $b = 0$. Therefore, consider the diagram,
\begin{center}
\begin{tikzcd}[column sep = large, row sep = large]
K \arrow[r, "0"] & A \arrow[d, "\id_A"'] \arrow[r, "f"] & B \arrow[dl, dashed] \arrow[r, "0"] & C
\\
& A & B \arrow[u, "\id_B"'] \arrow[ul, dashed] & 
\end{tikzcd}
\end{center} 
Where $\id_A \circ a = 0$ and $b \circ \id_B = 0$ (since $a = 0$ and $b = 0$) which implies that $\id_A$ factors through the cokernel $f :A \to B$ and $\id_B$ lifts over the kernel $f : A \to B$. Thus $f$ has a left inverse $g_L : B \to A$ and right inverse $g_R : B \to A$ such that $g_L \circ f = \id_A$ and $f \circ g_R = \id_B$. Thus $f$ is is both left and right split and thus is an isomorphism. Alternatively, the splittings are unique but notice that $g_R \circ f = (g_L \circ f) \circ g_R \circ f = g_L \circ f = \id_A$ so $g_R = g_L$ by uniqueness of the factorization.
\end{proof}

\begin{corollary}
Let $\A$ be a binormal category. Then any morphism in $\A$ which is both monic and epic is an isomorphism. 
\end{corollary}

\begin{prop}
Let $\A$ be a category with zero maps and $f : A \to B$ a morphism. Then,
\begin{enumerate}
\item if $f$ is a normal monomorphism then $f = \ker{\coker{f}}$
\item if $f$ is a conormal epimorphism then $f = \coker{\ker{f}}$
\end{enumerate}
\end{prop}

\begin{proof}
If $f$ is a normal monomorphism then $f = \ker{g}$ for some $g : B \to C$. Consider the diagram,
\begin{center}
\begin{tikzcd}
\\
& Z \arrow[d, "t"] \arrow[ld, dashed]
\\
A \arrow[r, "f"] & B \arrow[d, "k"] \arrow[r, "g"] & C
\\
& K \arrow[ru, "\tilde{g}"', dashed]
\end{tikzcd}
\end{center}
Because $f = \ker{g}$ we know that $g \circ f = 0$ and thus $g$ factors through the cokernel of $f : A \to B$ which is $K$. Then if $k \circ t = 0$ we see that $g \circ t = \tilde{g} \circ k \circ t = 0$ meaning that $t$ factors uniquely through $f : A \to B$ because $f = \ker{g}$ showing that $f = \ker{k}$. The second statement is exactly dual.
\end{proof}

WHEN DOES KER=0 IFF MONO?

\section{Images and Coimages (DOOO!!!! COMPARE WITH MAGIC SQUARE)}

\begin{defn}
The image $\im{f}$ of a morphism $f : X \to Y$ is the smallest subobject of $Y$ such that $f$ factors through $\im{f} \to Y$. Explicitly, this is a factorization $f = m \circ e$ with $m$ monic such that for any other factorization $f = m' \circ e'$ with $m$ monic as in the diagram,
\begin{center}
\begin{tikzcd}
X \arrow[rd, "e"] \arrow[rdd, "e'"'] \arrow[rr, "f"] & & Y 
\\
& \im{f} \arrow[ru, "m", hook] \arrow[d, dashed, "v"']
\\
& Z \arrow[ruu, "m'"', hook]
\end{tikzcd}
\end{center}
there is a unique arrow $v : \im{f} \to Z$ making the diagram commute.
\end{defn}

\begin{defn}
The coimage $\coim{f}$ of a morphism $f : X \to Y$ is the image of $f$ in the opposite category or equivalently the largest quotient of $X$ such that $f$ factors through $X \to \coim{f}$. Explicitly, this is a factorization $f = m \circ e$ with $e$ epic such that for any other factorization $f = m' \circ e'$ with $e$ epic as in the diagram,
\begin{center}
\begin{tikzcd}
X \arrow[rd, "e", two heads] \arrow[rdd, "e'"', two heads] \arrow[rr, "f"] & & Y 
\\
& \coim{f} \arrow[ru, "m"]
\\
& Z \arrow[ruu, "m'"']  \arrow[u, dashed, "v"']
\end{tikzcd}
\end{center}
there is a unique arrow $v : Z \to \coim{f}$ making the diagram commute.
\end{defn}


RELATE TO EQUALIZERS, RELATE TO KER OF COKER

\section{Abelian Categories (DO!!!)} 


DEFINITON OF AB-CAT

DEF OF IM AND COIM

IM = COIM

\begin{definition}
We say that a sequence,
\begin{center}
\begin{tikzcd}
X \arrow[r, "f"] & Y \arrow[r, "g"] & Z
\end{tikzcd}
\end{center}
is a complex if $g \circ f = 0$ giving a monomorphism $\Im{f} \to \ker{g}$. We say the sequence is \textit{exact} if this morphism is also epic i.e. an isomorphism by the above lemma. 
\end{definition}

\begin{proposition}
\begin{center}
\begin{tikzcd}
0 \arrow[r] & X \arrow[r, "f"] & Y \arrow[r, "g"] & Z
\end{tikzcd}
\end{center}
is exact iff $(X \xrightarrow{f} Y) = \ker{g}$ and, 
\begin{center}
\begin{tikzcd}
X \arrow[r, "f"] & Y \arrow[r, "g"] & Z \arrow[r] & 0
\end{tikzcd}
\end{center}
is exact iff $(Y \xrightarrow{g} Y) = \coker{f}$. 
\end{proposition}

\begin{proof}
DO THIS PROOF
\end{proof}

\begin{definition}
ABELIAN FUNCTOR
\end{definition}

\begin{definition}
Let $F : \A \to \B$ be an additive functor between abelian categories. Then we say that,
\begin{enumerate}
\item $F$ is \textit{left-exact} if $F$ preserves kernels
\item $F$ is \textit{right-exact} if $F$ preserves cokernels
\item $F$ is \textit{exact} if $F$ preserves exact sequences
\end{enumerate}
\end{definition}

\begin{proposition}
$F$ is exact iff $F$ is left and right-exact. 
\end{proposition}

\begin{proof}
DO THIS!!
\end{proof}

\begin{proposition}
Let $F : \A \to \B$ and $G : \A \to \B$ be an adjoint pair of additive functors between abelian categories. Then $F$ is right-exact and $G$ is left-exact.
\end{proposition}

\begin{proof}
Left-adjoints preserve colimits and right-adjionts preserve limits.
\end{proof}

\section{Homology}

\end{document}

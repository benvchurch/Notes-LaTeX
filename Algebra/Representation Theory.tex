\documentclass[12pt]{extarticle}
\usepackage[utf8]{inputenc}
\usepackage[english]{babel}
\usepackage[a4paper, total={6in, 9in}]{geometry}
\usepackage{tikz-cd}
 
\usepackage{amsthm, amssymb, amsmath, centernot}

\newcommand{\notimplies}{%
  \mathrel{{\ooalign{\hidewidth$\not\phantom{=}$\hidewidth\cr$\implies$}}}}

\renewcommand\qedsymbol{$\square$}
\newcommand{\cont}{$\boxtimes$}
\newcommand{\divides}{\mid}
\newcommand{\ndivides}{\centernot \mid}
\newcommand{\Z}{\mathbb{Z}}
\newcommand{\N}{\mathbb{N}}
\newcommand{\C}{\mathbb{C}}
\newcommand{\Zplus}{\mathbb{Z}^{+}}
\newcommand{\Primes}{\mathbb{P}}
\newcommand{\ball}[2]{B_{#1} \! \left(#2 \right)}
\newcommand{\Q}{\mathbb{Q}}
\newcommand{\R}{\mathbb{R}}
\newcommand{\Rplus}{\mathbb{R}^+}
\newcommand{\invI}[2]{#1^{-1} \left( #2 \right)}
\newcommand{\End}[1]{\text{End}\left( #1 \right)}
\newcommand{\legsym}[2]{\left(\frac{#1}{#2} \right)}
\renewcommand{\mod}[3]{\: #1 \equiv #2 \: \mathrm{mod} \: #3 \:}
\newcommand{\nmod}[3]{\: #1 \centernot \equiv #2 \: \mathrm{mod} \: #3 \:}
\newcommand{\ndiv}{\hspace{-4pt}\not \divides \hspace{2pt}}
\newcommand{\finfield}[1]{\mathbb{F}_{#1}}
\newcommand{\finunits}[1]{\mathbb{F}_{#1}^{\times}}
\newcommand{\ord}[1]{\mathrm{ord}\! \left(#1 \right)}
\newcommand{\quadfield}[1]{\Q \small(\sqrt{#1} \small)}
\newcommand{\vspan}[1]{\mathrm{span}\! \left\{#1 \right\}}
\newcommand{\galgroup}[1]{Gal \small(#1 \small)}
\newcommand{\ints}[1]{\mathcal{O}_{#1}}
\newcommand{\sm}{\! \setminus \!}
\newcommand{\norm}[3]{\mathrm{N}^{#1}_{#2}\left(#3\right)}
\newcommand{\qnorm}[2]{\mathrm{N}^{#1}_{\Q}\left(#2\right)}
\newcommand{\quadint}[3]{#1 + #2 \sqrt{#3}}
\newcommand{\pideal}{\mathfrak{p}}
\newcommand{\inorm}[1]{\mathrm{N}(#1)}
\newcommand{\tr}[1]{\mathrm{Tr} \! \left(#1\right)}
\newcommand{\delt}{\frac{1 + \sqrt{d}}{2}}
\newcommand{\ch}[1]{\mathrm{char} \: #1}
\renewcommand{\Im}[1]{\mathrm{Im}(#1)}
\newcommand{\minimal}[2]{\mathrm{Min}(#1;#2)}
\newcommand{\fix}[2]{\mathrm{Fix}_{#1} (#2)}
\newcommand{\id}{\mathrm{id}}
\newcommand{\Disc}[1]{\mathrm{Disc}(#1)}
\newcommand{\sgn}[1]{\mathrm{sgn}(#1)}
\newcommand{\Sym}[1]{\mathrm{Sym}(#1)}
\newcommand{\GL}[2]{\mathrm{GL}(#1, #2)}
\newcommand{\Hom}[2]{\mathrm{Hom}\left( #1, #2 \right)}
\newcommand{\aut}[1]{\mathrm{Aut}\left( #1 \right)}
\newcommand{\repHom}[3]{\mathrm{Hom}^{#1} \left( #2, #3 \right)}
\newcommand{\inner}[2]{\left<#1, #2 \right>}
\newcommand{\Ind}[3]{\mathrm{Ind}_{#2}^{#1}\left(#3\right)}
\newcommand{\Res}[3]{\mathrm{Res}_{#2}^{#1}\left(#3\right)}
\newcommand{\Homover}[3]{\mathrm{Hom}_{#1}\left(#2, #3\right)}

\theoremstyle{definition}
\newtheorem{theorem}{Theorem}[section]
\newtheorem{lemma}[theorem]{Lemma}
\newtheorem{proposition}[theorem]{Proposition}
\newtheorem{corollary}[theorem]{Corollary}
\newtheorem{remark}{Remark}


\newenvironment{definition}[1][Definition:]{\begin{trivlist}
\item[\hskip \labelsep {\bfseries #1}]}{\end{trivlist}}


\newenvironment{lproof}{\begin{proof} \renewcommand{\qedsymbol}{}}{\end{proof}}


\begin{document}

\section{Group Actions}
\begin{definition}
Let $G$ be a group acting on a set $X$, call $X$ a $G$-set, then there eixsts a homomorphism $\phi : G \to \Sym{X}$ the group of bijections of $X$ to itself. 
\end{definition}
For example, $\GL{n}{k}$ acts on $k^n$ for a field $k$. However, $\GL{n}{k}$ also action on $(k^n)^*$ by the action $A \cdot f = f \circ A^{-1}$. Furthermore, $\GL{n}{k}$ acts on $\Hom{k^n}{k^n}$ by $A \cdot F = A \circ F \circ A^{-1}$.  


\section{Group Representations}
\begin{definition}
A $G$-representation $(V, \rho_V)$ is a group action on a vector space $V$ with a homomorphism $\rho_V : G \to \aut{V}$ 
\end{definition}

\begin{definition}
A $G$-morphism between $G$-representations $\rho_V$ and $\rho_W$ is a linear map $F : V \to W$ satisfying $F \circ \rho_V(g) = \rho_W(g) \circ F$ for all $g \in G$. The set of all such $G$-morphisms is denoted $\repHom{G}{V}{W}$.  
\end{definition}

\begin{definition}
Let $\rho_V : G \to \aut{V}$ be a $G$-representation, then $W \subset V$ is a $G$-invariant subspace if $\rho(g)(W) \subset W$ for all $g \in G$.
\end{definition}

\begin{definition}
A $G$-representation $(V, \rho_V)$ is irreducible if $V \neq \{0\}$ and the only invariant subspaces are $\{0\}$ and $V$. 
\end{definition}

\begin{definition}
Given $G$-representations $(V, \rho_V)$ and $(W, \rho_W)$, we can form the following additional $G$-representations,
\begin{enumerate}
\item $(V^*, \rho_{V*})$ given by $\rho_{V^*}(g) \cdot \varphi = \varphi \circ \rho_V(g)^{-1}$

\item $(V \oplus W, \rho_V \oplus \rho_W)$ given by,
\[(\rho_V \oplus \rho_W) (g) \cdot (v \oplus w) = (\rho_V(g) \cdot v) \oplus (\rho_W(g) \cdot w)\]

\item $(\Hom{V}{W}, \rho_{\Hom{V}{W}})$ given by, $\rho_{\Hom{V}{W}} \cdot F = \rho_W(g) \circ F \circ \rho_V(g)^{-1}$. Note, the fixed points, $(\Hom{V}{W})^G = \repHom{G}{V}{W}$ because $\rho_W(g) \circ F \circ \rho_V(g)^{-1} = F$ for every $g \in G$ if and only if $F$ is a $G$-morphism. 

\item $(V \otimes W, \rho_V \otimes \rho_W)$ given by,
 \[(\rho_V \otimes \rho_W) (g) \cdot \left( \sum_{i = 1}^n v_i \otimes w_i \right) = \sum_{i = 1}^n (\rho_V(g) \cdot v_i) \otimes (\rho_W(g) \cdot w_i)\]

\end{enumerate}

\begin{lemma}
If $V$ is a $G$-representation such that $V \neq \{0\}$ then there exists a $G$-invariant subspace $W$ which is an irreducible $G$-representation. 
\end{lemma}

\begin{lemma}
Let $F : V \to W$ be a $G$-morphism then $\ker{F}$ and $\Im{F}$ are invariant subspaces.
\end{lemma}

\begin{proof}
Let $V$ and $W$ be $G$-representations and let $F : V \to W$ be a $G$-morphism. Take any $g \in G$. Take, $v \in \ker{F}$. Then, $F(v) = 0$ and thus, $\rho_W(g)(F(v)) = F(\rho_V(g)(v)) = 0$ so $\rho_V(g)(v) \in \ker{F}$. Therefore, $\ker{F}$ is invariant under the action of $\rho_V(g)$ for any $g \in G$. Therefore, $\ker{K}$ is a $G$-invariant subspace of $V$. Similarly, take $w \in \Im{F}$. Then there exists $v \in V$ such that $F(v) = w$. Therefore, $\rho_W(g)(w) = \rho_W(g)(F(v)) = F(\rho_V(g)(v)) \in \Im{F}$. Therefore, $\rho_V(g)(\Im{K}) \subset \Im{F}$ so $\Im{F}$ is a $G$-invariant subspace of $W$. 
\end{proof}

\begin{lemma}
Let $F : V \to W$ be a $G$-morphism then,
\begin{enumerate}
\item if $V$ is irreducible then $F$ is either $0$ or injective.
\item if $W$ if irreducible then $F$ is either $0$ or surjective.
\item if $V$ and $W$ are both irreducible then $F$ is either $0$ or an isomorphism.
\end{enumerate}
\end{lemma}
\begin{proof}
Let $V$ be irreducible. Since $\ker{F}$ is an invariant subspace, then $\ker{F} = \{0\}$ or $\ker{F} = V$ so $F$ is either injective or the zero map. Likewise, let $W$ be irreducible. Since $\Im{F}$ is an invariant subspace, then $\Im{F} = \{0\}$ or $\Im{F} = W$ so $F$ is either the zero map or surjective.
\end{proof}
\end{definition}

\begin{definition}
The notation $(V, \rho_V) \cong (W, \rho_W)$ with shorthand $V \cong W$ mean that there exists a $G$-isomorphism $F : V \to W$ i.e. a bijective $G$-morphism.
\end{definition}

\begin{theorem}[Schur's Lemma]
If $V$ is irreducible then $\repHom{G}{V}{V} \cong \C \cdot \id$. Also, 
if $V$ and $W$ are both irreducible then either $V \not\cong W$ and $\repHom{G}{V}{W} = \{0\}$ or $V \cong W$ and $\dim{\repHom{G}{V}{W}} = 1$. 
\end{theorem}

\begin{proof}
Let $F : V \to V$ be a $G$-morphism then $F$ is either zero or an isomorphism because $V$ is irreducible. Then $F$ has an eigenvalue $\lambda$ so consider the $G$-morphism $F - \lambda \id$. However, $\exists v \in V$ such that $F(v) = \lambda v$ so $(F - \lambda \id)(v) = 0$ and therefore, $F - \lambda v$ is not injective. However, $V$ is irreducible so $F$ must be the zero map. Thus, $F = \lambda \id$. Furthermore, if every $G$-morphism $F \in \repHom{G}{V}{W}$ is not an isomorphism then because $V$ and $W$ are irreducible we must have $F = 0$. Thus, if $\repHom{G}{V}{W} \neq \{0\}$ then there must exist a $G$-isomorphism $F$. In particular, $V \cong W$. Therefore, $\repHom{G}{V}{W} \cong \repHom{G}{V}{V} \cong \C \cdot \id$ so $\dim{\repHom{G}{V}{W}} = 1$. 
\end{proof}

\begin{corollary}
$F \in \repHom{G}{V}{W}$ is either zero or an isomorphism and therefore invertible. Therefore, $\repHom{G}{V}{W}$ is a division ring. 
\end{corollary}

\begin{definition}
A $G$-representation $(V, \rho_V$) is decomposable if $V \cong W_1 \oplus W_2$ where $W_i \neq \{0\}$
\end{definition}

\begin{definition}
A $G$-representation is completely reducible if $V \cong W_1 \oplus \cdots \oplus  W_n$ where $W_i$ is irreducible. 
\end{definition}

\begin{lemma}
Let $G$ be a finite group and $V$ a $G$-representation, the map $p : V \to V$ given by, 
\[ p(v) = \frac{1}{|G|} \sum_{g \in G} \rho_V(g)(v)\]
is a $G$-invariant projection $p : V \to V^G$. 
\end{lemma}

\begin{proof}
If $v \in V^G$ then,
\[p(v) = \frac{1}{|G|} \sum\limits_{g \in G} \rho_V(g)(v) = \frac{1}{|G|} \sum\limits_{g \in G} v = v\]
Furthermore, for any $v \in V$ consider,
\[ \rho_V(h) \circ p (v) = \frac{1}{|G|} \sum\limits_{g \in G} \rho_V(h) \circ \rho_V(g)(v) = \frac{1}{|G|} \sum\limits_{g \in G} \rho_V(hg)(v) = \frac{1}{|G|} \sum\limits_{g \in G} \rho_V(g)(v) = p(v)\]
so $p(v) \in V^G$. Therefore, $\Im{p} = V^G$. Furthermore,
\[ p \circ \rho_V(g) (v) = \frac{1}{|G|} \sum\limits_{g \in G}  \rho_V(g) \circ  \rho_V(h) (v) = \frac{1}{|G|} \sum\limits_{g \in G} \rho_V(gh)(v) = \frac{1}{|G|} \sum\limits_{g \in G} \rho_V(g)(v) = p(v)\]
Thus, $p \circ \rho_V(g) = \rho_V(g) \circ p$ for all $g \in G$.
\end{proof}

\begin{theorem}[Maschke]
If $G$ is a finite group and $W \subset V$ are $G$-representations then there exists a $G$-invariant complement $W' \subset V$ of $W$ and thus $V = W \oplus W'$.
\end{theorem}

\begin{proof}
Let $p_0 : V \to V$ be a projection onto $W$. Then, $p_0 \in \Hom{V}{V}$ so by the above lemma applied to the $G$-representation $(\Hom{V}{V}, \rho_{\Hom{V}{V}})$, the map,
\[p_0 \mapsto p = \frac{1}{|G|} \sum\limits_{g \in G} \rho_{\Hom{V}{V}}(g) \cdot p_0 = \frac{1}{|G|} \sum\limits_{g \in G} \rho_V \circ p \circ \rho_V^{-1}\]
is a projection map $\Hom{V}{V} \to (\Hom{V}{V})^G = \repHom{G}{V}{V}$. Thus,
$p$ is a $G$-invariant projection from $V$ to $W$ since $p(w) = w$. Therefore, $V \cong W \oplus \ker{p}$.   
\end{proof}

\begin{corollary}
If $G$ is a finite group then every nonzero $G$-representation is completely reducible. 
\end{corollary}

\begin{corollary}
If $G$ is a finite abelian group then any $G$-representation is a sum of $1$-dimensional representations.
\end{corollary}

\begin{proof}
It suffices to prove that every irreducible $G$-representation is $1$-dimensional. Let $W$ be an irreducible $G$-representation. However, since $G$ is abelian, $\rho_W(g)$ is a $G$-morphism in $\repHom{G}{V}{V} \cong \C$ so $\rho_W(g) = \lambda(g) \in \C$. Then, $\rho_W(g)(w) = \lambda(g) w$ so $\vspan{w}$ is a nonempty $G$-invariant subspace. However $W$ is irreducible so $W = \vspan{w}$ which has dimension $1$.
\end{proof}

\begin{corollary}
Let $A \in \GL{n}{\C}$ and suppose that $A$ has finite order then $A$ is diagonalizable. 
\end{corollary}

\begin{proof}
$A$ defines a representation of $\Z / N \Z$ where $N$ is the order of $A$. Therefore, $\C^n$ is the sum of $1$-dimensional $G$-invariant subspaces which are eigenspaces. Therefore, the eigenvectors of $A$ span $\C^n$. 
\end{proof}

\begin{corollary}
Let $\rho_V$ be a $G$-representation of a finite group $G$ then $\forall g \in G$ we can diagonalize $\rho_V(g)$ and its eigenvalues are roots of unity of order dividing $|G|$.
\end{corollary}

\begin{proof}
Because $G$ is finite, and $g \in G$ has finite order and $\ord{g} \divides |G| = $ so $\rho_V(g)$ has order dividing $n$ and is thus diagonalizable. Furthermore if $v$ is an eigenvector, $\rho_V(g) \cdot v = \lambda v$ then $\rho_V(g)^n \cdot v = \lambda^n v$ but $\rho_V(g^n) = \rho_V(e) = \id$ so $\lambda^n v = v$ and thus $\lambda^n = 1$ since $v \neq 0$ so $\lambda$ is a root of unity. 
\end{proof}

\section{Group Characters}

\begin{definition}
If $(V, \rho_V)$ is a $G$-representation, the character is the map $\chi : G \to \C$ defined by $\chi(g) = \tr{\rho_V(g)}$.  
\end{definition}

\begin{lemma}
Let $(V, \rho_V)$ be a $G$-representation with character $\chi$ then,
\begin{enumerate}
\item $\chi(e) = \tr{\id} = \dim{V}$

\item $\chi(hgh^{-1}) = \tr{\rho_V(h) \rho_V(g) \rho_V(h)^{-1}} = \tr{\rho_V(h)} = \chi(g)$. Thus, $\chi$ is a function on conjugacy classes.

\item $\chi(g^{-1}) = \overline{\chi(g)}$ because $\rho(g)$ is diagonalizable with norm-$1$ eigenvalues. 
\end{enumerate}
\end{lemma}

\begin{lemma}
Let $(V, \rho_V)$ and $(W, \rho_W)$ be $G$-representations with character $\chi_V$ and $\chi_W$ then,
\begin{enumerate}
\item $\chi_{V \oplus W} = \chi_{V} + \chi_{W}$
\item $\chi_{V^*} = \overline{\chi_{V}}$
\item $\chi_{V \otimes W} = \chi_V \cdot \chi_W$
\item $\chi_{\Hom{V}{W}} = \overline{\chi_V} \cdot \chi_W$
\end{enumerate}
\end{lemma}

\begin{lemma}
\[ \dim{V^G} = \frac{1}{|G|} \sum_{g \in G} \chi_V(g) \]
\end{lemma}

\begin{proof}
The map,
\[ p(v) = \frac{1}{|G|} \sum_{g \in G} \rho_V(g)(v)\]
is a $G$-invariant projection $p : V \to V^G$ so $\tr{p} = \dim{V^G}$. However,
\[ \tr{p} = \frac{1}{|G|} \sum_{g \in G} \tr{\rho_V(g)} = \frac{1}{|G|} \sum_{g \in G} \chi_V(g) \]
\end{proof}

\begin{corollary}
Applying this fact to $\Hom{V}{W}$, then,
\[ \dim{(\Hom{V}{W}^G)} = \dim{\repHom{G}{V}{W}} = \frac{1}{|G|} \sum_{g \in G} \chi_{\Hom{V}{W}} (g) = \frac{1}{|G|} \sum_{g \in G} \overline{\chi}_V(g) \chi_W(g) \]
\end{corollary}

\begin{corollary}
By Schur's lemma, 
\[ \dim{\repHom{G}{V}{W}} = \begin{cases}
1 & V \cong W \\
0 & \text{else}
\end{cases}\]
Therefore,
\[ \frac{1}{|G|} \sum_{g \in G} \overline{\chi}_V(g) \chi_W(g) = \frac{1}{|G|} \sum_{g \in G} {\chi}_V(g) \overline{\chi_W(g)} =
\begin{cases}
1 & V \cong W \\
0 & \text{else}
\end{cases}\]
where I have used the fact that the sum is real because it is equal to an integer.
\end{corollary}

\begin{definition}
For $f_1, f_2 \in \C[G]$ define the Hermitian inner product,
\[ \inner{f_1}{f_2} = \frac{1}{|G|} \sum_{g \in G} {f_1(g)} \overline{f_2(g)} \]
\end{definition}

\begin{proposition}
Therefore, for irreducible representations $(V, \rho_V)$ and $(W, \rho_W)$ with characters $\chi_V$ and $\chi_W$ then,
\[\inner{\chi_V}{\chi_W} = 
\begin{cases}
1 & V \cong W \\
0 & \text{else}
\end{cases}\]
\end{proposition}

\begin{corollary}
Let $V$ be a completely reducible representation, $V = \bigoplus\limits_{i = 1}^n V_i^{m_i}$ with $V_i \cong V_j$ only if $i = j$ then, 
\[\inner{\chi_V}{\chi_V} = \sum_{i,j} m_i m_j \inner{\chi_{V_i}}{\chi_{V_j}} = \sum_{i =  1}^n m_i^2\]
\end{corollary}

\begin{corollary} \label{chargivesisotofactor}
Let $V$ be a completely reducible $G$-representation, $V = \bigoplus\limits_{i = 1}^n V_i^{m_i}$ with $V_i \cong V_j$ only if $i = j$ and $W$ an irreducible $G$-representation then,
\[ \inner{\chi_W}{\chi_V} = 
\begin{cases}
m_i & W \cong V_i \\
0 & \text{else}
\end{cases}\]
\end{corollary}

\begin{proof}
We have, $\chi_V = \sum_{i = 1}^n m_i \chi_{V_i}$. Thus,
\[ \inner{\chi_W}{\chi_V} = \sum_{i = 1}^n m_i \inner{\chi_W}{\chi_{V_i}} =
\begin{cases}
m_i & W \cong V_i \\
0 & \text{else}
\end{cases}\]
since by hypothesis $i \neq j \implies V_i \not\cong V_j$. 
\end{proof}

\begin{corollary}
$V$ is irrediucible if and only if $\inner{\chi_V}{\chi_V} = 1$.
\end{corollary}

\begin{theorem}
Let $G$ be finite, then a $G$-representation $V$ is determined up to isomorphism by $\chi_V$. That is, $V \cong W \iff \chi_V = \chi_W$. 
\end{theorem}

\begin{proof}
If $V \cong W$ then there exists an isomorphism $F : V \to W$ such that $F \circ \rho_V(g) = \rho_W(g) \circ F$ and thus $\rho_V(g) = F^{-1} \circ \rho \circ F$. Thus, 
\[\chi_V = \tr{\rho_V(g)} = \tr{F^{-1} \circ \rho \circ F} = \tr{\rho_W(g)} = \chi_W(g)\]
Conversely, suppose that $\chi_V = \chi_W$. Then, because $G$ is finite, we can write any $G$-representations as,
\[V = \bigoplus\limits_{i = 1}^n V_i^{m_i} \quad \quad W = \bigoplus\limits_{i = 1}^n W_i^{k_i}\]
Therefore, $\chi_V = \sum\limits_{i = 1}^n m_i \: \chi_{V_i}$. Consider
\[ \inner{\chi_{V_i}}{\chi_W} = \inner{\chi_{V_i}}{\chi_{V}} = \inner{\chi_{V_i}}{\chi_{V}} = m_i\]
but $V_i$ is irreducible so $\inner{\chi_{V_i}}{\chi_W} = m_i$ implies that some factor $W_j^{k_j}$ is isomorphic to $V_i$ and $m_i = k_j$. Therefore, up to order, the expansions of $V$ and $W$ are equal. Thus, $V \cong W$. 
\end{proof}

\begin{definition}
The regular representation is $\rho_{reg} : G \to \C[G]$ given by $\rho(g) v = g \cdot v$. Call the character of this representation $\chi_{reg} = \chi_{\C[G]}$.
\end{definition}

\begin{lemma}
Let $G$ act on $X$ and let $(\C[X], \rho)$ be the regular $G$-representation. Then,
\[\chi_{\C[X]}(g) = \#(X^g)\]
\end{lemma}

\begin{proof}
We know that $\rho(g) \cdot x = g \cdot x$ so
\[\tr{\rho(\sigma)} = \sum_{i = 1}^{|X|} \mathbf{1}(g \cdot x = x) = \#(X^g)\]
\end{proof}

\begin{corollary}
\[\chi_{reg}(g) =
\begin{cases}
|G| & g = e \\
0 & g \neq e
\end{cases}\] 
\end{corollary}

\begin{proof}
A group acts freely on itself ($g h = h \implies g = e$) so there cannot be any fixed points of $G$ for any map except $\rho(e)$ which fixes every element. 
\end{proof}

\begin{lemma}
$\inner{\chi_V}{\chi_{reg}} = \dim{V}$
\end{lemma}
\begin{proof}
\[\inner{\chi_V}{\chi_{reg}} = \frac{\chi_V(e) |G|}{|G|} = \chi_V(e)=  \dim{V}\]
\end{proof}

\begin{theorem}
Write, 
\[\C[G] \cong \bigoplus_{i = 1}^n V_i^{d_i}\]
If $W$ is an irreducible $G$-representation then $W \cong V_i$ for some $i$. Furthermore, $\dim{V_i} = d_i$. 
\end{theorem}

\begin{proof}
Let $W$ be irreducible, then $\inner{\chi_W}{\chi_{reg}} = \dim{W} > 0$ and therefore by corollary \ref{chargivesisotofactor}, $W \cong V_i$ for a unique $i$. However, $\dim{V_i} = \inner{\chi_{V_i}}{ \chi_{reg}} = d_i$.
\end{proof}

\begin{corollary}
\[ \dim{\C[G]} = |G| = \sum_{i = 1}^n (d_i)^2 \]
\end{corollary}


\begin{corollary}
For any $g \in G$,
\[ \sum_{i = 1}^n d_i \cdot \chi_{V_i}(g) = \begin{cases}
|G| & g = e\\
0 & g \neq e
\end{cases} \]
\end{corollary}

\begin{proof}
Because,
\[\C[G] \cong \bigoplus_{i = 1}^n V_i^{d_i}\]
the character factors as,
\[\chi_{reg}(g) = \sum_{i = 1}^n d_i \cdot \chi_{V_i}(g) = \begin{cases}
|G| & g = e\\
0 & g \neq e
\end{cases} \]
\end{proof}

\begin{theorem}
If $G$ is a finite group, then there are finitely many irreducible $G$-representations.
\end{theorem}

\begin{proof}
Every irreducible $G$-representation must be isomorphic so a factor of the regular representation. Equivalently, the sum of the squares of the dimensions of all irreducible $G$-representations is $|G|$ which is, in particular, finite. 
\end{proof}

\begin{proposition}
Let $G$ be abelian, then every representation is one-dimensional so $d_i = 1$. Thus, $\sum\limits_{i = 1}^n d_i^2 = n = |G|$. So there are exactly $|G|$ irreducible $G$-representations. 
\end{proposition}

\section{The Permutation Representation}


\section{Class Functions}

\begin{definition}
$f : G \to \C$ is a class function if $f$ is constant on conjugay classes or equivalently, $\forall g,h\in G : f(hgh^{-1}) = f(g)$.
\end{definition}

\begin{definition}
$Z \subset \C[G]$ is the vectorspace of class functions.
\end{definition}

\begin{proposition}
$f_{Cl(x)}$ is the characteristic function of $[x]$ which is,
\[
f_{Cl(x)}(g) = \begin{cases}
1 & g \in Cl(x) \\
0 & g \notin Cl(x)
\end{cases}\]
form a basis of $Z$.
\end{proposition}

\begin{proposition}
\[\inner{f_{Cl(x)}}{f_{Cl(y)}} = 
\begin{cases}
\frac{|Cl(x)|}{|G|} & Cl(x) = Cl(y) \\
0 & \text{else}
\end{cases}\]
\end{proposition}

\begin{definition}
For $f \in \C[G]$ the map,
$F_{V, f} : V \to V$ is defined by,
\[F_{V,f} = \sum_{g \in G} f(g) \rho_V(g)\]
\end{definition}

\begin{lemma}
If $f$ is a class function, $F_{V, f}$ is a $G$-morphism. If in addition, $V$ is irreducible, then $F_{V, f}  = t \cdot \id$ where,
 \[t = \frac{|G| \cdot \inner{f}{\overline{\chi}_V}}{\dim{V}}\]
\end{lemma}

\begin{proof}
$F_{V, f}$ if a $G$-morphism if and only if $\forall h \in G$ we have $\rho_V(h) \circ F_{V,f} \circ \rho_V(h)^{-1} = F_{v,f}$. Expanding,
\[ \rho_V(h) \circ F_{V,f} \circ \rho_V(h)^{-1} = \sum_{g \in G} f(g) \rho_V(h) \circ \rho_{g} \circ \rho_V(h)^{-1} = \sum_{g \in G} f(g) \rho_V(hgh^{-1}) = \sum_{g \in G} f(h^{-1}gh) \rho_V(g) = F_{V, f}\]
because $f$ is a class function. \bigskip \\
Using Schur's Lemma, if $V$ is irreducible then because $F_{V, f}$ is a $G$-morphism we know that $F_{V, f} = t \cdot \id$. Thus, $\tr{F_{V, f}} = \tr{t \cdot \id} = t \dim{V}$. However,
\[ \tr{F_{V,f}} = \sum_{g \in G} f(g) \tr{\rho_V(g)} = \sum_{g \in G} f(g) \chi_V(g) = |G| \inner{f}{\overline{\chi_V}}\]
Therefore, $t \dim{V} = |G| \inner{f}{\overline{\chi_V}}$. 
\end{proof}

\begin{proposition}
If $f$ is a class function then $\inner{f}{\chi_V} = 0$ for all irreducible $V$ implies that $f = 0$. Furthermore, if $V_1, \cdots, V_n$ are the irreducible $G$-representations up to isomorphism then $\chi_{V_1}, \cdots, \chi_{V_n}$ are a basis for $Z$. Finally, $n$ is the number of conugacy classes of $G$. 
\end{proposition}

\begin{proof}
If $V$ is irreducible then $V^*$ is irreducible so $\inner{f}{\chi_{\overline{V}}} = 0$ and thus $F_{V, f} = 0 \cdot \id = 0$ for all irreducible $V$. However, $F_{V_1 \oplus V_2, f} = F_{V_1, f} + F_{V_2, f} = 0$ so by induction $F_{W, f} = 0$ for all $G$-representations.
In particular, $F_{\C[G], f} = 0$ that is,
\[ F_{\C[G], f} = \sum_{g \in G} f(g) \rho_{reg}(g) = 0\]
so applied to $1$,
\[ F_{\C[G], f} = \sum_{g \in G} f(g) \rho_{reg}(g)(1) = \sum_{g \in G} f(g) \cdot g = 0\]
and therefore $f = 0$ because $\C[G]$ is a free vectorspace over $G$. 
\bigskip \\
By orthogonality conditions, $\inner{\chi_{V_i}}{\chi_{V_j}} = \delta_{ij}$ and thus these characters are linearly independent. Consider the subspace of $Z$ orthogonal to all $\chi_{V_i}$. However, we have shown that if $\inner{f}{\chi_{V_i}} = 0$ for all irreducible representations $V_i$ then $f = 0$. Thus, the orthogonal complement is empty so the set $\{ \chi_{V_1}, \dots, \chi_{V_n} \}$ spans $Z$ and thus $\dim{V} = n$.
\bigskip\\
However, the functions $f_{Cl(x)}$ form a basis of $Z$. Therefore, $\dim{Z} = n$ is the number of conjugacy classes of $G$. 
\end{proof}

\begin{proposition}
$G$ is abelian if and only if every irreducible $G$-representation is one-dimensional.
\end{proposition}

\begin{proof}
If $d_i = 1$ then $\sum_{i = 1}^n d_i^2 = n = |G|$ so there are $|G|$ conjugacy classes and thus $G$ is abelian. We have already proved the converse. 
\end{proof}

\begin{proposition}
We having the following orthogonality relationship on $G$ over the set of irreducible characters,
\begin{itemize}
\item \[ \forall x \in G : \quad \sum_{i = 1}^h | \chi_{V_i} (x) |^2 = \frac{|G|}{|Cl(x)|} \]
\item \[ \forall x, y \in G : y \notin Cl(x) : \quad \sum_{i = 1}^h  \chi_{V_i}(x) \overline{\chi_{V_j}}(y) = 0 \]
\end{itemize}
\end{proposition}

\section{Fourier Inversion on Groups}

\subsection{The Structure of $\C[G]$}

\begin{definition}
A $K$-algebra is a $K$-vectorspace $A$ together with a $K$-bilinear map donoted by $B : A \times A \to A$ where $B(a, b) \mapsto ab$.
\end{definition}

\begin{proposition}
If $A$ is an \textit{associative} \textit{unital} $K$-algebra, then $A$ has a ring structure. 
\end{proposition}

\begin{proof}
$(a_1 + a_2)b = B(a_1 + a_2, b) = B(a_1, b) + B(a_2, b) = a_1 b + a_2 b$. The other properties are similar.
\end{proof}

\begin{definition}
A homomorphism of $K$-algebras is a $K$-linear map $F : A \to A'$ such that $F(B(a, b)) = B'(F(a), F(b))$. In particular, if $A$ is an associative unital algebra then $F$ is a linear ring homomorphism. 
\end{definition}

\begin{proposition}
A $G$-representation $(V, \rho_V)$ induces a homomorphism of $\C$-algebras $\rho_V : \C[G] \to \End{V} = \Hom{V}{V}$ given by,
\[ \rho_V \left( \sum_{g \in G} t_g \cdot g \right) = \sum_{g \in G} t_g \cdot \rho_V(g) \]
or alternatively given a map $f : G \to \C$ define,
\[ \rho_V(f) = \sum_{g \in G} f(g) \rho_V(g)\]
\end{proposition}

\begin{proposition}
Let $V = \C[G]$ then the regular representation induces a homomorphism $\rho_{\C[G]} : \C[G] \to \End{\C[G]}$. This map is given by $\rho_{\C[G]}(\alpha)(\beta) = \alpha \beta$. 
\end{proposition}

\begin{theorem}[Weddenburn]
Define $\rho : \C[G] \to \End{V_1} \times \cdots \times \End{V_h}$ where $V_1, \cdots, V_h$ enumerates all the irreducible $G$-representations by the map,
 \[\rho(\alpha) = (\rho_{V_1}(\alpha), \cdots, \rho_{V_h}(\alpha))\] where $\rho_{V_i}(\alpha) = \sum\limits_{g \in G} \alpha(g) \rho_V(g)$ for $\alpha \in \C[G]$. Then, $\rho$ is an isomorphism of $\C$-algebras. 
\end{theorem}


\begin{proof}
$\dim{\C[G]} = |G|$ and $\dim{(\End{V_1} \times \cdots \times \End{V_h})} = \dim{\End{V_1}} + \cdots + \dim{\End{V_h}} = (\dim{V_1})^2 + \cdots + (\dim{V_h})^2 = d_1^2 + \cdots + d_h^2 = |G|$. Therefore, to prove that $\rho$ is an isomorphism of $\C$-algebras it suffices to prove that $\rho$ is an injective $\C$-algebra homomorphism. Suppose that $\rho(\alpha) = 0$ then $\rho_{V_i}(\alpha) = 0$ for all $i$. Therefore, $\rho_V(\alpha) = 0$ for every representation because we have shown this for every irreducible component. In particular, $\rho_{\C[G]}(\alpha) = 0$ and in particular $\rho_{\C[G]}(\alpha)(1) = \alpha = 0$ so $\alpha = 0$. Therefore $\rho$ is injective and thus an isomorphism.  
\end{proof}

\begin{theorem}[Hard]
Suppose $K$ is a field of characteristic zero then,
\[K[G] \cong \End{D_1} \times \cdots \times \End{D_h}\]
where $D_i$ is not necessarily a field but a division ring. 
\end{theorem}

\begin{lemma}
The center $Z(\C[G]) \cong Z$ the set of class functions.
\end{lemma}

\begin{proof}
Suppose $g \in Z(\C[G])$ if and only if $\forall g \in \C[G]$ we have $f * g = g * f$. Thus,
\[ f \in Z(\C[G]) \iff \delta_x * f = f * \delta_x \iff f(x^{-1} y) = f(y x^{-1}) \iff f(h) = f(x h x^{-1}) \iff f \in Z\]
\end{proof}

\begin{remark}
We will sometimes refer to $\rho : \C[G] \to \End{V_1} \times \cdots \times \End{V_h}$ as the Fourier transform. 
\end{remark}

\begin{proposition}
For $(A_1, \cdots, A_n) \in \End{V_1} \times \cdots \times \End{V_h}$ we have,
\[ \rho^{-1}(A_1, \cdots, A_n) = \sum_{g \in g} t_g \cdot g \]
where 
\[ t_g = \frac{1}{|G|} \sum_{i = 1}^h d_i \tr{\rho_{V_i} (g^{-1}) \cdot A_i } \]
\end{proposition}

\begin{proof}
We know that $\rho$ is an isomorphism so $\rho$ takes any basis of $\C[G]$ to an basis of $End{V_1} \times \cdots \times \End{V_h}$.
\end{proof}


\subsection*{Classical Finite Fourier Analysis}

Let $G$ be an abelian group.

\begin{definition}
The dual group is $\hat{G} = \{\lambda : G \to \C^\times \mid \lambda \text{ is a homo.} \}$ with pointwise multiplication. 
\end{definition}

\begin{proposition}
$|\hat{G}| = |G|$
\end{proposition}

\begin{proof}
Suppose the group $G$ is cyclic, all its irreducible representations are finite. Therefore, there is a one-to-one correspondence between irreducible representations and homomorphisms $\lambda : G \to \C^\times$. However, there are exactly $|G|$ irreducible representations because in an abelian group every element defines a distinct conjugacy class.
\end{proof}

\begin{proposition}
For a finite group $G \cong \hat{G}$ (but not naturally) and $G \cong \hat{\hat{G}}$ naturally.
\end{proposition}

\begin{definition}
The Fourier transform is a map $\C[G] \to \C[\hat{G}]$ given by $f \mapsto \hat{f}$ where,
\[ \hat{f}(\lambda) = |G| \inner{f}{\lambda} = \sum_{g \in G} f(g) \lambda(g) \] 
\end{definition}

\begin{proposition}
The Fourier transform satisfies,
\begin{itemize}
\item $\widehat{f_1 * f_2} = \hat{f_1} \cdot \hat{f_2}$

\item Inversion: $f = \frac{1}{|G|} \sum\limits_{\lambda \in G} \hat{f}(\lambda) \cdot \lambda$ such that $f = \hat{\hat{f}}$ up to normalization.

\item $\left< f_1, f_2 \right> = \frac{1}{|G|} \left< \hat{f_1}, \hat{f_2} \right> $
\end{itemize}
\end{proposition}

\begin{proof}
Because $\lambda$ forms a unitary basis,
\[ f  = \sum_{\lambda \in \hat{G}} \inner{f}{\lambda} \cdot \lambda  = \frac{1}{|G|} \sum_{\lambda} \hat{f}(\lambda) \cdot \lambda \]
Furthermore,
\[ \inner{f_1}{f_2} = \sum_{\lambda \in \hat{G}} \inner{f_1}{\lambda} \inner{\lambda}{f_2} = \sum_{\lambda \in \hat{G}} \inner{f_1}{\lambda} \inner{\lambda}{f_2} = \frac{1}{|G|^2} \sum_{\lambda \in \hat{G}} \hat{f}_1(\lambda) \overline{\hat{f}_2}(\lambda) = \frac{1}{|G|} \inner{\hat{f_1}}{\hat{f_2}} \] 
\end{proof}

\begin{theorem}
Let $G$ be a finite abelian group then the map,
\[ ev : G \to \hat{\hat{G}} \]
is an isomorphism and $ev : f \mapsto \hat{\hat{f}} = |G| f(g^{-1})$.  
\end{theorem}

\section{One-Dimensional Representations}

\begin{theorem}
Let $G$ be finite. The number of one-dimensional representations of $G$ is the order of $G^{ab}$. 
\end{theorem}

\begin{proof}
Any one-dimensional representation is given by a homomorphism $\lambda : G \to \C^\times$. However, $\C^\times$ is abelian so such homomorphisms are in one-to-one correspondence with homomorphisms $G^{ab} \to \C^\times$ i.e. to the group $\widehat{G^{ab}}$. Therefore, the number of one-dimensional representations is $|G^{ab}|$ and thus this number divides $|G|$.  
\end{proof}

\begin{lemma}
A subgroup $N \triangleleft G$ such that $N \subset G'$ and $G / N$ is abelian then $N = G'$
\end{lemma}

\begin{proof}
We know that $G/N$ is abelian and $\pi : G \to G/N$ is a homomorphism so $G' \subset \ker{\pi} = N$. Thus, $N = G'$. 
\end{proof}

\section{Product Groups}

\begin{theorem}
Let $\rho_{V_1}$ be an irreducible $G_1$-representation and $\rho_{V_2}$ be an irreducible $G_2$-representation then $\rho_{V_1 \otimes V_2} : G_1 \times G_2 \to \aut{V_1 \otimes V_2}$ given by,
 \[\rho_{V_1 \otimes V_2}(g_1, g_2) = \rho_{V_1}(g_1) \otimes \rho_{V_2}(g_2) \]
is an irreducible $G_1 \times G_2$ representation and every irreducible $G_1 \times G_2$ representation is of this form.
\end{theorem}

\begin{proof}
The chracter is given by, 
\[\chi_{V_1 \otimes V_2}(g_1, g_2) = \tr{\rho_{V_1 \otimes V_2}(g_1, g_2))} = \tr{\rho_{V_1}(g_1)} \cdot \tr{\rho_{V_2}(g_2)} = \chi_{V_1}(g_1) \cdot \chi_{V_2}(g_2)\]
Therefore,
\begin{align*}
\inner{\chi_{V_1 \otimes V_2}}{\chi_{V_1 \otimes V_2}} & = \frac{1}{|G_1 \times G_2|} \sum_{(g_1, g_2) \in G_1 \times G_2} | \chi_{V_1 \otimes V_2}(g_1, g_2) |^2 
\\
& = \frac{1}{|G_1||G_2|} \sum_{g_1 \in G} |\chi_{V_1}(g_1)|^2 \sum_{g_2 \in G} |\chi_{V_2} (g_2) |^2 = \inner{\chi_{V_1}}{\chi_{V_1}} \cdot \inner{\chi_{V_2}}{\chi_{V_2}} = 1
\end{align*}
and therefore $\rho_{V_1 \otimes V_2}$ is irreducible. \bigskip\\
Furthermore, (WIP)
\end{proof}

\section{Burnside's Theorem}

\begin{definition}
$c(x) = |Cl(x)|$ is the size of the conjugacy class of $x$.
\end{definition}

\begin{lemma}
If $G$ is finite and $\rho_V$ is a $G$-representation, then $\chi_V(g)$ is an algebraic integer.
\end{lemma}

\begin{proof}
We know that $\rho_V(g)$ is diagonalizable and each eigenvalue is a root of unity because $\rho_V(g)^n = \rho_V(g^n) = \rho_V(e) = \id$. Therefore, $\chi_V(g) = \tr{\rho_V(g)}$ is the sum of roots of unity which is an algebraic integer.  
\end{proof}

\begin{theorem}
Let $V$ be an irreducible $G$-representation with $\dim{V} = d_V$ then for all $g \in G$ the number $\frac{c(g)}{d_V} \chi_V(g)$ is an algebraic integer.
\end{theorem}

\begin{proof}
Define the map $\rho_V : \C[G] \to \End{V}$ by,
\[ \rho_V(f) = \sum_{g \in G} f(g) \rho_V(g) \]
We know that since $V$ is irreducible if $f$ is a class function then,
\[\rho_V(g) = \frac{|G| \inner{f}{\overline{\chi_V}}}{\dim{V}} \cdot \id\]
Since $\delta_{Cl(x)}$ is a class function,
\[ \rho_V(\delta_{Cl(x)}) = \frac{|G| \inner{\delta_{Cl(x)}}{\overline{\chi_{V}}}}{d_V} \cdot \id \] 
but we know that,
\[ \inner{\delta_{Cl(x)}}{\overline{\chi_V}} = \frac{1}{|G|} \sum_{g \in G} \delta_{Cl(x)}(g) \chi_V(g) = \frac{1}{|G|} \sum_{g \in Cl(x)} \chi_V(g) = \frac{c(x)}{|G|} \chi_V(x)\]
since $\chi_V$ is a class function. 
Therefore,
\[\rho_V(\delta_{Cl(x)}) = \frac{c(x)}{d_V} \chi_V(x) \cdot \id \]
Therefore, 
\[\frac{c(x)}{d_V} \chi_V(x)\]
is the eigenvalue of the map $\rho_V(\delta_{Cl(x)}$ which must be an algebraic integer. 

\end{proof}

\begin{theorem}[Frobenius]
If $V$ is irreducible then $d_V \divides |G|$.
\end{theorem}

\begin{proof}
$\inner{\chi_V}{\chi_V} = 1$ so $|G| = \sum_{g_\in G} \chi_V(g) \overline{\chi_V(g)}$. We wrtite $G$ as the disjoint union over conjugacy classes. Thus,
\[ |G| = \sum_{i = 1}^n \sum_{g \in Cl(x_i)} \chi_V(g) \overline{\chi_V(g)} = \sum_{i = 1}^h c(x_i) \chi_V(x_i) \overline{\chi_V(x_i)} \]
Therefore,
\[ \frac{|G|}{d_V} = \sum_{i = 1}^h \left( \frac{c(x_i) \chi_V(x_i)}{d_V} \right) \overline{\chi_V(x_i)} \]
is the sum of products of algebraic integers and thus an algebraic integer. Therefore, $|G|/d_V$ is an algebraic integer but also rational. therefore $|G|/d_V \in \Z$ so $d_V \divides |G|$. 
\end{proof}

\begin{lemma}
Let $\lambda_1, \cdots, \lambda_d$ be roots of unity. Then,
\begin{enumerate}
\item $|\lambda_1 + \cdots + \lambda_d| \le d$ with equality iff $\lambda_1 = \cdots = \lambda_d$.
\item $\alpha = \frac{1}{d} (\lambda_1 + \cdots + \lambda_d)$ is an algebraic integer if and only if $\alpha = 0$ or $\lambda_1 = \cdots = \lambda_d$.
\end{enumerate}
\end{lemma}

\begin{proof}

\end{proof}

\begin{lemma}
Let $G$ be finite and $V$ any $G$-representation of dimension $d = d_V$ then,
\begin{enumerate}
\item $\forall g \in G : |\chi_V(g)| \le d_V$ with equality iff $\rho_V(g) = \frac{\chi_V(g)}{d_V} \id$
\item $\forall g \in G : \chi_V(g) = d_V \iff \rho_V(g) = \id \iff g \in \ker{\rho_V}$.
\end{enumerate}
\end{lemma}

\begin{proof}
We know that $\rho_V(g)$ is diagonalizable with eigenvalues which are roots of unity. Therefore $\chi_V(g) = \lambda_1 + \cdots + \lambda_d$. Thus, $|\chi_V(g)| \le d_V$ with equality iff $\lambda_1 = \cdots = \lambda_d = \frac{\chi_V(g)}{d_V}$ so $\rho_V(g) = \frac{\chi_V(g)}{d_V} \id$. Furthermore, 
\[\chi_V(g) = d_V \implies |\chi_V(g)| = d_V \implies \rho_V(g) = \frac{\chi_V(g)}{d_v} \id = \id\] 
And clearly if $\rho_V(g) = \id$ then $\chi_V(g) = \tr{\id} = d_V$. 
\end{proof}

\begin{corollary}
A finite group $G$ is not simple iff there exists a nontrivial irreducible $G$-representation $V$ such that $\exists g \in G \sm \{e\} : \chi_V(g) = \chi_V(e) = d_V$. 
\end{corollary}

\begin{proof}
$G$ is not simple if there exists $N \triangleleft G$ such that $N$ is nontrivial and proper. Therefore, $G/N$ is not isomorphic to $G$ or $\{e\}$. Therefore, there must exist a nontrivial representation $\rho_V : G/N \to \aut{V}$ of $G/N$ which lifts under $\pi : G \to G/N$ to a representation $\pi^* \rho_V = \rho_V \circ \pi : G \to \aut{V}$. \bigskip\\
Converseley, choose $\rho_V$ which is a nontrivial irreducible $G$-representation such that $\exists g \in G \sm \{e\} : \chi_V(g) = \chi_V(e) = d_V$. Then, $\ker{\rho_V} \triangleleft G$ but $\ker{\rho_V} \neq G$ since $\rho_V$ is nontrivial. However, there exists $g \in G \sm \{e\}$ such that $\chi_V(g) = d_V$ which implies that $g \in \ker{\rho_V}$ so $\ker{\rho_V}$ is nontrivial. Thus, $G$ is not simple becuase $\ker{\rho_V}$ is a nontrivial proper subgroup. 
\end{proof}

\begin{proposition}
Let $G$ be a finite group, let $V$ be an irreducible $G$-representation suppose that $\gcd{(c(g), d_V)} = 1$ then $\chi_V(g) = 0$ or $\rho_V(g) = \lambda \cdot \id$.
\end{proposition}

\begin{proof}
Since $gcd{(c(x), d_V)} = 1$ we know that $\exists a, b \in \Z$ such that $a c(x) + b d_V = 1$ but,
\[\frac{\chi_V(g)}{d_V} = (a c(x) + b d_V) \frac{\chi_V(g)}{d_V} = a \left( \frac{c(x) \chi_V(g)}{d_V} \right) + b \chi_V(g)\]
which is the sum of algebraic integers. Thus, $\frac{\chi_V(g)}{d_V}$ is an algebraic integer. However, $\chi_V(g) = \lambda_1 + \cdots + \lambda_d$ is a sum of roots of unity. Therefore, since $\frac{1}{d}(\lambda_1 + \cdots + \lambda_d)$ is an algebraic integer, we know that $\lambda_1 + \cdots + \lambda_d = 0$ so $\chi_V(g) = 0$ or $\lambda_1 = \cdots = \lambda_d$ so $\chi_V(g) = \lambda \cdot \id$. 
\end{proof}

\begin{corollary}
Let $G$ be a finite simple nonabelian group and $V$ a nontrivial irreducible $G$-representation then $\gcd{(c(g), d_V)} = 1 \implies \chi_V(g) =0$.
\end{corollary}

\begin{proof}
$G$ is simple so $\rho_V$ is injective since $\ker{\rho_V}$ is normal and $\rho_V$ is nontrivial. Therefore, take $g$ as in the condition, if $\chi_V(g) \neq 0$ then $\rho_V(g) = \lambda \cdot \id$. Therefore, $\rho_V(g) \in Z(\aut{V})$ so $\Im{\rho_V}$ is abelian so $G' \subset \ker{\rho_V} = \{e\}$. Therefore $G' = \{e\}$ which implies that $G/G' \cong G$ is abelian which contradicts the assumption that $G$ is nonabelian. Thus, $\chi_V(g) = 0$. 
\end{proof}

\begin{theorem}
Let $G$ be a nonabelian finite simple group let $g \in G \sm \{e\}$ then $c(g)$ is not a prime power.
\end{theorem}

\begin{proof}
Suppose that $|Cl(g)| = p^a$ for some prime $p$. If $a = 0$ then $a \in Z(G)$ but $Z(G) \neq G$ because $G$ is nonabelian so $Z(G)$ is a nontrivial proper normal subgroup contradicting simplicity. Let $V$ be an irreducible $G$-representation. If $\gcd{(c(x), d_V)} = 1$ then $\chi_V(g) = 0$. Therefore, if $p \ndivides d_V$ then $\chi_V(g) = 0$ so either $p \divides d_V$ or $\chi_V(g) = 0$. Consider,
\[\chi_{\mathrm{reg}}(g) = 0 = \sum_{i = 1}^h d_i \chi_{V_i}(g) = 1 + \sum_{i = 2}^h d_i \chi_{V_i}(g)\]
However, $\chi_V(g) = 0$ or $p \divides d_i$ so $\frac{d_i \chi_{V_i}(g)}{p}$ is an algebraic integer. Therefore,
\[ \frac{1}{p} \sum_{i = 2}^h d_i \chi_{V_i}(g) = - \frac{1}{p} \]
is an algebraic integer but $\frac{-1}{p}$ is rational so it would need to be in $\Z$ which is clearly false. Thus, $|Cl(g)| = p^a$ is false. 
\end{proof}

\begin{theorem}[Burnside]
If $|G| = p^a q^b$ for primes $p, q$ and $a, b \ge 1$ then $G$ is not simple. 
\end{theorem}
\begin{proof}
Assume that $G$ is simple. We know that $G$ cannot be abelian because $G$ does not have prime order. However, for all $g \in G$ we know that $c(g)$ is not a prime power. However, 
\[ |G| = p^a q^b = \sum_{i = 1}^h |Cl(x_i)| = 1 + \sum_{i \ge 2}^h |Cl(x_i)| \]
However, the nontrivial conjugacy classes divide $p^a q^b$ and cannot be prime powers so they each must be divisible by $pq$. Thus,
\[ \mod{p^a q^b = 1 + \sum_{i \ge 2}^h |Cl(x_i)|}{1}{p} \quad \text{and} \quad \mod{p^a q^b = 1 + \sum_{i \ge 2}^h |Cl(x_i)|}{1}{q} \]
which are clearly contradictions.
\end{proof}

\section{Induced Representations}

\begin{definition}
Let $G$ be a finite group and $H \subset G$ a subgroup then the induced representation,
\[ \Ind{G}{H}{W} = \C[G] \otimes_{\C[H]} W \]
as a left $\C[G]$ module thus a $G$-representation. Alternatively,
\[ \Ind{G}{H}{W} = \Homover{\C[H]}{\C[G]}{W} = \{ f : G \to W \mid f(hg) = \rho_W(h) f(g) \} \] 
\end{definition}

\begin{proposition}
Properties of the induced representation. 
\begin{enumerate}
\item \[ \Ind{G}{H}{\C} \cong \C[G/H] \]

\item \[ \Ind{G}{G}{V} \cong V \]
\end{enumerate}
\end{proposition}

\begin{remark}[Notation]
Let $x_1, \cdots, x_n$ be representatives for $G/H$. Then, $g x_i \in g x_i H = x_{j(i,g)} H$ so $g x_i = x_{j(i,g)} h_i(g)$ 
\end{remark}

We want to determine the structure $\Ind{G}{H}{W}$.

\begin{definition}
For $w \in W$, let $F_{i, w} : G \to W$ be given by,
\[ F_{i, w} (g) = \rho_W(h)^{-1}(w)\]
where $g = x_i h \in x_i H$ and zero otherwise. 
\end{definition}

\begin{proposition}
Properties of $F_{i,w}$,
\begin{enumerate}
\item $F_{i, w} \in \Ind{G}{H}{W}$

\item $F_{i, w_1 + w_2} = F_{i, w_1} + F_{i, w_2}$

\item $F_{i, t \cdot w} = t \cdot F_{i, w}$

\item $W^{(i)} = \{F_{i, w} \mid w \in W\}$ is a vector subspace of $\Ind{G}{H}{W}$ and, 
\[W^{(i)} = \{ F \in \Ind{G}{H}{W} \mid F(g) = 0 \text{ if } g \notin x_i H \}\]

\item $\forall F \in \Ind{G}{H}{W}$ we have $F = \sum\limits_{i = 1}^k F_{i, w_i}$ where $w_i = F(x_i)$.

\item We have the isomorphism of vectorspaces,
\[\Ind{G}{H}{W} \cong \bigoplus\limits_{i = 1}^k W^{(i)}\]
Therefore, 
\[\dim{\Ind{G}{H}{W}} = k \dim{W} = [G : H] \dim{W}\]
\end{enumerate}
\end{proposition}

\begin{proposition}
\[ \rho_{\Ind{G}{H}{W}}(g) \cdot F_{i, w} = F_{j(i, g), \rho_W(h_i(g)) \cdot w} \]
\end{proposition}

\begin{proof}
Consider, $\rho(g) \cdot F_{i, w}(x_\ell) = F_{i, w}(g^{-1} x_\ell)$. Now,
$g^{-1} x_\ell \in x_i H$ so $x_\ell \in g x_i H = x_j H$ therefore zero unless $\ell = j$. Assume that $\ell = j$ then $\rho(g) \cdot F_{i, w}(x) = F_{i, w}(g^{-1} x$ but $x \in x_j H$ so $x = x_j h$  
\end{proof}


\begin{theorem}[Frobenius Reciprocity]
\[ \repHom{H}{W}{\Res{G}{H}{U}} \cong \repHom{G}{\Ind{G}{H}{W}}{U} \]
\end{theorem}

\begin{theorem}
For any class functions $f_1 : H \to \C$ and $f_2 : G \to \C$ we have,
\[ \inner{f_1}{\Res{G}{H}{f_2}}_H = \inner{\Ind{G}{H}{f_1}}{f_2}_G \]
\end{theorem}

\begin{proof}
and the right hand side is,
\[ \inner{\Ind{G}{H}{f_1}}{f_2}_G = \frac{1}{|G|} \sum_{g \in G} \Ind{G}{H}{f_1}(g) \overline{f_2(g)} = \frac{1}{|G|} \sum_{g \in G} \frac{1}{|H|} \sum_{x \in G} \tilde{f}_1(x^{-1} g x) \overline{f_2(g)} \]
Rewriting,
\begin{align*}
\inner{\Ind{G}{H}{f_1}}{f_2}_G & = \frac{1}{|G| \cdot |H|} \sum_{g, x \in G} \tilde{f}_1(x^{-1} g x) \overline{f}_2(g)
\\
& = \frac{1}{|G| \cdot |H|} \sum_{g, x \in G} \tilde{f}_1(g) \overline{f}_2(x g x^{-1}) = \frac{1}{|H|} \sum_{g \in G} \tilde{f}_1(g) \overline{f}_2(g)
\end{align*}
where I have used the fact that $f_2$ is a $G$-class function. However, $\tilde{f}(g) = 0$  unless $g \in h$ so the left hand side becomes,
\[ \inner{\Ind{G}{H}{f_1}}{f_2}_G = \frac{1}{|H|} \sum_{h \in H} f_1(h) \overline{f_2(h)} = \inner{f_1}{\Res{G}{H}{f_2}}_H \]
\end{proof}

\begin{corollary}
\[ \inner{\chi_W}{\chi_{\Res{G}{H}{U}}}_H = \inner{\chi_{\Ind{G}{H}{W}}}{\chi_U}_G \]
\end{corollary}

\begin{theorem}[Projection Formula]
\[ \Ind{G}{H}{W \otimes \Res{G}{H}{U}} = (\Ind{G}{H}{W}) \otimes U \]
\end{theorem}

\begin{corollary}
\[ \Ind{G}{H}{\Res{G}{H}{V}} = \Ind{G}{H}{\Res{G}{H}{\C \otimes V}} = \C[G/H] \otimes V \]
\end{corollary}

\begin{definition}

\end{definition}

\begin{theorem}
Suppose that $W$ is irreducible then $\Ind{G}{H}{W}$ is irreducible if and only if $\forall x \in G \setminus H$ the representations $W$ and $W_x$ are not isomorphic $G$-representations.
\end{theorem}

\begin{proof}
\[ \inner{\chi_{\Ind{G}{H}{W}}}{\chi_{\Ind{G}{H}{W}}}_G = \inner{\chi_W}{\chi_{\Res{G}{H}{\Ind{G}{H}{W}}}}_H \]

\end{proof}

\begin{definition}
Let $H \subset G$ and $[G : H] = 2$ then define the homomorphism $\epsilon : G \to \{\pm 1\} \subset \C^\times$ by,
\[ \epsilon(g) = 
\begin{cases}
1 & g \in H \\
0 & g \notin H
\end{cases}\]
\end{definition}

\begin{theorem}
Let $V$ be an irreducible $G$-representation, $W = \Res{G}{H}{V}$ and let $V \otimes \epsilon$ correspond to $\epsilon \rho_V$. Then, exactly one of the following holds,
\begin{enumerate}
\item $V \cong V \otimes \epsilon$ and $W \cong W' \oplus W'_x$ where $W'$ is irreducible and $W' \not\cong W'_x$ and $V \cong \Ind{G}{H}{W'} \cong \Ind{G}{H}{W'_x}$.

\item $V \not\cong V \otimes \epsilon$ and $W \cong W_x$ is irreducible and $\Ind{G}{H}{W} \cong V \otimes (V \otimes \epsilon)$.
\end{enumerate}
\end{theorem}

\section{Real Representations}

\begin{definition}
A $G$-representation $\rho_V : G \to \aut{V}$ is real if $V$ is an $\R$-vectorspace.
\end{definition}

\begin{proposition}
If $\rho_V$ is a real representation then $V \cong V^{*}$ as a $G$-reprsentation.
\end{proposition}

\begin{proof}
If $\rho_V$ is real then $\chi_V$ is real so $\chi_V = \overline{\chi_V}$ and thus $V \cong V^*$.
\end{proof}

\begin{remark}
The condition $V \cong V^*$ is not sufficient to show that $\rho_V$ is the complexification of a real representation. 
\end{remark}

\begin{theorem}
Let $V$ be an irreducible $G$-representation then,
\begin{enumerate}
\item $V \not\cong V^*$ and $V$ cannot be defined over $\R$ if and only if $\left(\mathrm{Bil} \: V \right)^G = 0$.

\item $V \cong V^*$ and $V$ cannot be defined over $\R$ if and only if $\dim{\left(\bigwedge^2 V^* \right)^G} = 1$.

\item $V \cong V^*$ and $V$ can be defined over $\R$ if and only if $\dim{\left( \mathrm{Sym} \: V \right)^G} = 1$.
\end{enumerate}
\end{theorem}

\begin{proof}
We know that $\mathrm{Bil} \: V \cong \Hom{V}{V^*}$ so $\left( \mathrm{Bil} \: V  \right)^G = \repHom{G}{V}{V^*} = 0$ if and only if $V \not\cong V^*$. \bigskip\\
Furthermore, 
\end{proof}

\section{Representations of the Symmetric Group}

\begin{remark}
For any $n$ we always have the $1$-dimensional (irreducible) representations $\C$ and $\C(\epsilon)$ and the $n$-dimensional permutation representation $\C^n \cong \C \oplus V$ where $V$ is an $(n-1)$-dimensional irreducible $S_n$-representation. 
\end{remark}

\begin{lemma}
Any $\sigma \in S_n$ can be written as a unique product of disjoint nontrivial cycles $\gamma_1 \cdots \gamma_k$ ordered by length. The cycle type of $\sigma$ is $(n_1, \cdots, n_k)$ where $n_i$ is the length of $\gamma_i$. Futhermore, there is a one-to-one correspondence between cycle types and conjugacy classes. 
\end{lemma}

\begin{definition}
$\lambda$ is a partition of $n$ written as $\lambda \vdash n$ is a weakly decreasing sequence of positive integers $\lambda_1 \ge \lambda_2 \ge \cdots \ge \lambda_{\ell}$ such that,
\[ \sum_{i = 1}^{\ell} \lambda_i = n \] 
\end{definition}

\begin{proposition}
Every $\sigma \in S_n$ determines a partition of $n$. Furthermore, the action of $\left< \sigma \right>$ on $S_n$ by partition $S_n$ into orbits of sizes $\lambda_i, \dots, \lambda_{\ell}$. 
\end{proposition}

\begin{proposition}
Conjugacy classes of $S_n$ are indexed by partitions $\lambda \vdash n$.
\end{proposition}

\begin{definition}
The Young Subgroup of a partition $\lambda \vdash n$ is the group $S_{\lambda} = S_{\lambda_1} \times \cdots \times S_{\lambda_{\ell}}$ where $\sigma \in S_{\lambda}$ means that $\sigma$ preserves the partition $\lambda$ of the set $\{1, \cdots, n \}$. 
\end{definition}

\begin{definition}
For each $\lambda \vdash n$ we get an $S_n$-representation,
\[ M^\lambda = \C[S_n/S_\lambda] = \Ind{S_n}{S_{\lambda}}{\C} \]
For example, for the extreme partitions $\lambda = (n)$ we have $S_\lambda = S_n$ so $M^{(n)} = \C[S_n/S_n] = \C$. Furthermore, if $\lambda = (1, \cdots, 1)$ then $S_{\lambda} = \{e\}$ so $M^{(1, \cdots, 1)} = \C[S_n]$ the regular representation. 
\end{definition}

\newcommand{\dom}{\trianglerighteq}

\begin{definition}
Given two partitions $\lambda, \mu \vdash n$ then $\lambda$ dominates $\mu$ writen as $\lambda \dom \mu$ if,
\[ \forall i : \lambda_1 + \cdots + \lambda_i \ge \mu_1 + \cdots + \mu_i \]
\end{definition}

\begin{proposition}
Domination is a partial order on the set of paritions of $n$ and for any $\lambda \vdash n$ we have
$(n) \dom \lambda \dom (1, \cdots, 1)$ 
\end{proposition}

\end{document}


\documentclass[12pt]{article}
\usepackage{import}
\import{./}{LogicCommands}

\begin{document}

\newcommand{\rec}{\mathrm{rec}}
\newcommand{\ind}{\mathrm{ind}}
\renewcommand{\succ}{\mathrm{succ}}
\newcommand{\pr}{\mathrm{pr}}

\section{Chapter 1}

\subsection{Exercises}

\subsubsection{Exercise 1.9}

\begin{exercise}
Define the type family $\rm{Fin} : \N \to \U$ and dependent function $\rm{fmax} : \prod\limits_{n : \N} \rm{Fin}(\succ(n))$.
\end{exercise}

We use the recursive type constructor,
\[ \rm{Fin} = \rec_{\N}(\U, \bf{0}, \lambda n. \lambda A. (A + \bf{1})) \]
This satisfies,
\[ \rm{Fin}(0) :\equiv \bf{0} \quad \rm{Fin}(\succ(n)) :\equiv \rm{Fin}(n) + \bf{1} \] 
then the max function also has an inductive construction using the type family $\rm{Fin}$,
\[ \rm{fmax} = \ind_{\N}(\rm{Fin} \circ \succ, \star, c_s) \]
where,
\[ c_s : \prod_{n : \N} \rm{Fin}(\succ(n)) \to \rm{Fin}(\succ(\succ(n))) \]
is the function,
\[ c_s(n, q) :\equiv \rm{inr}(\star) \]
Which satisfies the properties,
\[ \rm{fmax}(0) \equiv \star : \bf{1} \quad \rm{fmax}(\succ(n)) \equiv \rm{inr}(\star) : \rm{Fin}(\succ(n)) + \bf{1} \]

\subsubsection{Exercise 1.10}

\newcommand{\ack}{\mathrm{ack}}

\begin{exercise}
Show that the Ackermann function $\ack : \N \to \N \to \N$ is definable using only $\rec_{\N}$ satisfying the following equations:
\begin{align*}
\ack(0,n) & \equiv \succ(n)
\\
\ack(\succ(m), 0) & \equiv \ack(m, 1)
\\
\ack(\succ(m), \succ(n)) & \equiv \ack(m, \ack(\succ(m), n))
\end{align*}
\end{exercise}

We define the following,
\[ \ack :\equiv \rec_{\N}(\N \to \N, \succ, \lambda (m : \N) . \lambda (g : \N \to \N) . \lambda (n : \N) . c_s(n, g, m)) \]
where,
\[ c_s :\equiv \rec_{\N}((\N \to \N) \to (\N \to \N), \lambda (g : \N \to \N) . \lambda (m : \N) . g(1), u) \]
where,
\[ u :\equiv \lambda (n : \N) . \lambda (c : (\N \to \N) \to (\N \to \N)) . \lambda (g : \N \to \N) . \lambda (m : \N) . g (c(g,m)) \]
Then we have the following,
\[ c_s(0) \equiv \lambda (g : \N \to \N) . \lambda (m : \N) . g(1) \]
and 
\[ c_s(\succ(n)) \equiv u(n, c_s(n)) \equiv \lambda (g : \N \to \N) . \lambda (m : \N) . g(c_s(n, g, m)) \]
Therefore,
\[ \ack(0) \equiv \succ \quad \text{ meaning } \ack(0, n) \equiv \succ(n) \]
and also,
\[ \ack(\succ(m)) \equiv [\lambda (m : \N) . \lambda (g : \N \to \N) . \lambda (n : \N) . c_s(n, g, m)](m, \ack(m)) \equiv \lambda (n : \N) . c_s(n, \ack(m), m) \]
This means that,
\[ \ack(\succ(m), n) \equiv c_s(n, \ack(m), m) \]
so in particular,
\[ \ack(\succ(m), 0) \equiv c_s(0, \ack(m), m) \equiv \ack(m, 1) \]
and also,
\[ \ack(\succ(m), \succ(n)) \equiv c_s(\succ(n), \ack(m), m) \equiv \ack(m, c_s(n, \ack(m), m)) \equiv \ack(m, \ack(\succ(m), n)) \]


\subsubsection{Exercise 1.11}

\begin{exercise}
Show that for any type $A$ we have $\neg \neg \neg A \to \neg A$. 
\end{exercise}

We need to show that the type $(((A \to \bf{0}) \to \bf{0}) \to \bf{0}) \to (A \to \bf{0})$ is inhabited. Indeed consider,
\[ \lambda (f : \neg \neg \neg A) . \lambda (a : A) . f( \lambda (g : \neg A) . g(a) ) \]

\subsubsection{Exercise 1.12}

\begin{exercise}
Using the propositions as types interpretation, derive the following tautologies,
\begin{enumerate}
\item If $A$ then (if $B$ then $A$).
\item If $A$ then not (not $A$).
\item If (not $A$ or not $B$) then not ($A$ and $B$).
\end{enumerate}
\end{exercise}

\begin{enumerate}
\item $(\lambda (a : A). \lambda (b : B). a) : (A \to B \to A)$
\item $(\lambda (a : A) . \lambda (p : \neg A) . p(a)) : (A \to (A \to \bf{0}) \to \bf{0})$
\item $\rec_{\neg A + \neg B}(\neg (A \times B), [\lambda (p : \neg A) . \lambda (f : A \times B) . p \circ \pr_1(f)], [\lambda (p : \neg B) . \lambda (f : A \times B) . p \circ \pr_2(f)]) : (\neg A + \neg B) \to \neg (A \times B)$
\end{enumerate}

\subsubsection{Exercise 1.15}

\begin{exercise}
Show that the indiscernibility of identicals follows from path induction.
\end{exercise}

We want to show show that for any family of types $C : A \to \U$ there is,
\[ f : \prod_{x,y : A} \prod_{p : x =_A y} C(x) \to C(y) \]
such that,
\[ f(x, x, \refl_x) :\equiv \id_{C(x)} \]
We apply path induction to the family of types,
\[ D(x,y,p) : \equiv C(x) \to C(y) \]
and our base case is,
\[ \prod_{x : A} D(x,x,\refl_x) \equiv \prod_{x : A} C(x) \to C(x) \]
which is inhabited by $\id_C$. Then we set,
\[ f :\equiv \ind_{=_A}(D, \id_C) \]
which satisfies,
\[ f(x,x,\refl_x) :\equiv \id_{C(x)} \]

\end{document}

\documentclass[12pt]{article}
\usepackage{import}
\import{./}{Includes}


\begin{document}

\atitle{3}

\section{Problem 1}

Let $X = S^2 \times Y$ where $Y = (\C / \Z \oplus \Z)^{n-1}$ is the $2n-2$-dimensional torus with the standard symplectic structure. For every compatible almost complex structure $J$, consider the moduli space $\Mbar_{A,0,n}(X;J)$ of stable $J$-holomorphic maps from a genus zero curve with $n$ marked points representing $A = S^2 \times * \in H_2(X; \Z)$.

\subsection{(a)}

Consider $\Mbar_{A,0,1}$. Each component of $C$ contracted by $f : C \to X$ must have at least three marked points. However, because the genus is zero (so the graph has no cycles) this means there cannot be any contracted components. Furthermore, each component maps to a positive class in $H_2(X; A)$ such that the sum is $A$ but this can only happen if there is one component and the map $f : C \to X \to \CP^1$ is degree $1$ meaning an isomorphism by Hurwitz and hence $f$ is simple.


\subsection{(b)}

$\Mbar_{A,0,1}$ consists of maps $\CP^1 \to X$ which are comprised of $\CP^1 \to X \to \CP^1$ given by a degree-$1$ algebraic map meaning a $\PGL{2}$-automorphism and $\CP^1 \to X \to Y$ constant. The second projection $\CP^1 \to X \to Y$ must be constant because it represents zero in $H_2(Y;\Z)$ and thus is nullhomotopic (in fact $\pi_2(Y) = 0$ so it is automatically nullhomotopic). Because the map is $J$-holomorphic this implies it is actually constant. Indeed, projecting $Y \to (\C / \Z \oplus \Z)$ on each factor we get a $J$-holomorphic map $\CP^1 \to (\C / \Z \oplus \Z)$ with the standard symplectic and (almost) complex structure which is integrable and thus defines an elliptic curve. Therefore this map is actually holomorphic and by Riemann-Hurwitz must be constant. Thus $\CP^1 \to X \to Y$ is constant since it is constant on each factor of $Y$. 

\subsection{(c)}

Using the previous description, the map,
\[ \ev : \Mbar_{A,0,1} \to X \]
is a diffeomorphism because such a map $\CP^1 \to X$ is determined by the image of the marked point. Thus it has degree $1$. If we assume that the cokernel of the linearization vanishes at all points in $\Mbar_{A,0,1}(X)_{J_0}$ then we conclude that,
\[ \GW_{A,0,1}(X, J_0, \alpha) = 1 \]
for $\alpha \in H^{2m}(X)$ Poincare dual to a fixed point and this does not depend on the complex structure as long as we are at a regular value of the fibration $\Mbar_{A,0,1}(X) \to \mathcal{J}$ (which we assumed that $J_0$ is thus allowing us to compute the GW invariant from the above calculations). 

\subsection{(d)}

Now let $J$ be any compatible almost complex structure on $X$ and chose a point $p \in X$ and $\alpha \in H^{2m}(X)$ be the corresponding Poincare dual class. To show that there exists a $J$-holomorphic curve $f : \CP^1 \to X$ representing $A$ whose image passes through $p$ it suffices to show that,
\[ \Mbar_{A,0,1}(X; J; \alpha) = \ev^{-1}(\alpha) \neq \empty \]
However, if we assume that $\Mbar_{A,0,1}(X, J, \alpha) = \empty$ then $J$ would be a regular value (vacuously) and thus we would immediately see that $\GW_{A,0,1}(X, J, \alpha) = 0$ contradicting our above calculation since the GW invariants are independent of $J$ in the regular locus. 

\section{Problem 2}

Let $X = \CP^2$ with the standard symplectic structure and $A \in H_2(X, \Z)$ the positive generator.

\subsection{(a)}

Let $J$ be a compatible almost complex structure and consider $\Mbar_{A,0,0}(x; J)$. By the same argument as before, since each contracted component has at least three marked points there cannot be contracted components of any $f : C \to X$. Thus the images in $H_2(X; \Z)$ of each component is positive but $f_* [C] = A$ so there can only be one component since $A$ is a generator. Thus $C = \CP^1$ and the map $f : C \to X$ is simple since its degree is given by $\deg{A} = 1$. 

\subsection{(b)}

Let $J_0$ be the standard complex structure on $X$. Maps $f : C \to X$ are degree $1$ holomorphic maps and hence by GAGA algebraic maps. These are exactly lines so $\Mbar_{A,0,0}(X; J_0) = \mathrm{Gr}(3,2) = \CP^2$. 
\bigskip\\
Alternatively, we can argue from positivity of the intersection number. A degree $1$ map by definition intersects the line class $L$ in exactly one point since $C \cdot L = 1$ implies that $\# (C \cap L) = 1$ by positivity of intersection numbers. Thus if we take the tangent line at a point and deform it slightly we get a transverse intersection and thus the line intersects $C$ at exactly this point. Therefore, the curvature of $C$ in $\CP^2$ must be zero at every point so it is a line.

\subsection{(c)}

Consider $\M_{A,0,2}(X; J_0)$ which corresponds to lines $f : \CP^1 \to \CP^2$ with two marked points. Therefore, the map $\M_{A,0,2}(X) \to \M_{A,0,0}(X)$ is a $(\CP^1 \times \CP^1) \setminus \Delta$ bundle where the fiber is exactly the space of distinct pairs of points on $\CP^1$. 
\bigskip\\
Consider the compactified space $\Mbar_{A,0,2}(X; J_0)$. First notice that because $A$ is not the sum of positive classes all but one components of any $f : C \to X$ in this space must be contracted. Therefore, all but one component must satisfy the stability condition of having at least three special points. Since the component graph has genus zero (no cycles) the only possibility is that $C$ is a nodal curve $C = C_1 \cup C_2$ where $C_1 \cong \CP^1$ and $f : C_1 \to \CP^2$ is a line and $f(C_2) = p$ and $C_2$ contains both marked points. Notice that any two such curves with fixed map $f : C_1 \to X$ and fixed image of the node $p \in X$ are isomorphic because $C_2$ has three special points and $f(C_2) = p$ so there is an automorphism of $C_2$ sending these points to any other three points. Therefore, for each line $f : C_1 \to X$ there is a $\CP^1$ of choices for $f : C \to X$ given by the choice of the node $x \in C_1$ or alternatively the choice of point $p \in \im{f}$ where the contracted bubble lands. 
\bigskip\\
Therefore, we should expect that $\Mbar_{A,0,2}(X,J_0) \to \Mbar_{A,0,0} = \CP^2$ is a $\CP^1 \times \CP^1$ bundle where $\CP^1 \times \CP^1$ parametrized the choices of marked points on the given line $f : \CP^1 \to X$ when they are distinct and the image of the node on the diagonal locus $\Delta \subset (\CP^1 \times \CP^1)$.

\subsection{(d)}

Consider the evaluation map,
\[ \ev : \Mbar_{A,0,2}(X; J_0) \to X \times X \]
which we assume is a smooth morphism of manifolds (in fact I think it is a holomorphic map of complex manifolds). Furthermore, assume that the cokernel of the linearization vanishes at all the points in $\Mbar_{A,0,2}(X; J_0)$ meaning $J_0$ is a regular value of the fibration $\Mbar_{A,0,2}(X) \to \cJ$.
\bigskip\\
Then let $\alpha_p, \alpha_q \in H^4(X, \Z)$ be Poincare dual classes to two distinct points $p,q \in X$. There is a unique line through $p,q$ in $X$ and therefore we see that the map,
\[ \ev : \Mbar_{A,0,2}(X; J_0) \to X \times X \]
has generic fibers (except over the diagonal locus $\Delta \subset X \times X$) which are given by lines $f : \CP^1 \to X$ passing through $p,q$ with marked points sent to $p,q$ (note on the boundary stratum of $\Mbar_{A,0,2}(X; J_0)$ the two marked points are mapped to the node and thus are not distinct so we need not worry about this extra locus in our computation of the generic degree) of which there is exactly one curve up to parametrization. Thus $\ev$ has generic degree $1$. This is equivalent to computing,
\[ \GW_{A,0,2}(X; J_0; \alpha_1, \alpha_2) = \int_{[\Mbar_{A,0,2}(X; J_0)]} \ev^{-1}(\alpha_p \times \alpha_q) = 1 \]
which computes the ``correct'' GW invariant because we assumed that $J_0$ is a regular value of the fibration $\Mbar_{A,0,2}(X) \to \cJ$. 

\subsection{(e)}

Let $J$ be any almost complex structure on $X$ and choose distinct points $p, q \in X$. To show that there exists a $J$-holomorphic curve $f : \CP^1 \to X$ whose image passes through $p$ and $q$ and which represents $A \in H_2(X, \Z)$ it suffices to show that,
\[ \Mbar_{A,0,2}(X; J; \alpha_p, \alpha_q) = \ev^{-1}(\alpha_p \times \alpha_q) \neq \empty \]
However, if we assume that $\Mbar_{A,0,2}(X; J; \alpha_p, \alpha_q) = \empty$ then $J$ would be a regular value (vacuously) and thus we would immediately see that $\GW_{A,0,2}(X; J; \alpha_p, \alpha_q) = 0$ contradicting our above calculation since the GW invariants are independent of $J$ in the regular locus. 
\bigskip\\
Now we know there exists a $J$-holomorphic curve $f : \CP^1 \to X$ passing through $p,q \in X$. To show uniqueness we can argue via positivity of intersection. Suppose that $C_1$ and $C_2$ are the images of two such holomorphic curves. Since both represent $A \in H_2(X; \Z)$ we see that the intersection number is $C_1 \cdot C_2 = 1$. By positivity of intersection, as long as $C_1$ and $C_2$ intersect properly (meaning at a finite set of points), each intersection counts positively so there is exactly one intersection point. However both $C_1$ and $C_2$ pass through $p$ and $q$ which are distinct points which would imply that $C_1 \cdot C_2 \ge 2$ if they intersected properly. Therefore we cannot have $C_1$ and $C_2$ intersecting properly. By the connectedness of $C_1$ and $C_2$ and the analytic continuation property of $J$-holomorphic curves this implies that $C_1 = C_2$ as submanifolds. Furthermore, the maps $f : \CP^1 \to X$ are then isomorphisms onto their images (this is not completely clear to me when we don't have the standard symplectic structure) so there is a reparametrization morphism $g : \CP^1\ to \CP^1$ making $f_1 : \CP^1 \to X$ and $f_2 : \CP^1 \to X$ represent the same holomorphic curves.

\end{document}
\documentclass[12pt]{article}
\usepackage{import}
\import{../}{GeometryCommands}

\newcommand{\dbar}{\bar{\partial}}
\DeclareMathOperator{\ind}{\mathrm{ind}}

\begin{document}

\section{Floer Homology}

Problems in dynamics: periodic orbits of Hamiltonian flows are critical points of a functional. Given a Hamiltonian $H_t : M \to \RR$ with $t \in S^1$. Induces a vector field $X_t$ which is $\omega$-dual to $\d{H_t}$. Periodic orbits are tranjectories $\gamma : S^1 \to M$ such that $\deriv{\gamma}{t} = X_t$. On $\Omega M$ there is a ``function'' (its more like a $1$-form) $\cA : L(M) = \{ \gamma : S^1 \to M$ \} \to \RR$,
\[ \cA(\gamma) = \int_{S^1} H_t + \int_{D^2} \omega \]
where $\partial D^2 = \gamma$. There is some multivaluedness in the choice of the disk. There are ways of making this actually work. We would like to use Morse theory to determine how many critical points (hence periodic orbits) there are. However, the gradient flow of $\cA$ is much worse than other studied functional (e.g. Geodesic flow). 
\bigskip\\
Floer: gradient flow lines in $\L(M)$ a map $\RR \to \L(M)$ which are the same as maps $\RR \times S^1 \to M$ so we translate an ODE on $\L(M)$ to a PDE on $M$. Gromov had already observed that on $M$ there is a natural class of PDEs that one can impose on maps form a Riemann surface $\Sigma \to M$ which are the pseudo-holomorphic curves. 

\subsection{Pseudo-Holomorphic Curves}

Pick an almost complex structure on $M$ which is \texitt{compatible} with the symplectic form. The symplectic structure is,
\begin{enumerate}
\item linear algebra data: $\omega$ a nondegenerate $2$-form
\item which varies in a constrained way: $\d{\omega} = 0$.
\end{enumerate}
The associated group to such a structure is $\Sp(2n, \RR)$ meaning that a symplectic structure is an integrable reduction of structure on the tangent bundle to $\Sp(2n, \RR)$. Furthermore, there are inclusions,
\[ \GL(n, \CC) \supset U(n) \subset \Sp(2n, \RR) \]
which are maximal compact subgroups and hence homotopy equivalences. Therefore, there is a contractible choice of almost complex structures (which are restrictions of the structure group to $\GL(n, \CC)$). Therefore, we want $J : TM \to TM$ such that $J^2 = -\id$ and $\omega(-, J-)$ is a Hermitian metric on $TM$. Such a $J$ exists and is unique up to contractible choices however often none will be integrable. 
\bigskip\\
Gromov showed that there is a natural compactification of the space of maps $\Sigma \to M$ which are $J$-holomorphic. Explciitly let $(\Sigma, j)$ be a complex structure then $f : \Sigma \to M$ is $J$-holomorphic if,
\[ \d{u} \circ j = J \circ \d{u} \] 
It is obtained by considering \textit{stable maps} (which is due to Kontsevich). 

\subsection{Floer's Solution}

In the presense of a Hamiltonain, we get a PDE for a function $u : \RR \times S^1 \to M$,
\[ (\d{u} - X_{H_t} \ot \d{t})^{0,1} = 0 \]
All the same bubbling operations apply. For a node, we get (after conformal diffeo) a very small region so the $\d{t}$ term vanishes so the bubbles are really $J$-holomorphic curves. The bubbling is why one cannot study the Morse theory of $\cA$ in $L(M)$ because there is a sequences of gradient trajectories in $L(M)$ which does not converge in $L(M)$ indeed it bubbles. This is why we need Floer theory NOT Morse theory. 

\subsection{Floer-Hofer}

Run this story for a discontinuous function. Floer homology for the characteristic function of a subset $L \subset M$ (e.g. open manifolds). For us, it is convenient to have a characteristic function which is zero on $K$ and $\infty$ off $K$. This is important because of Gromov non-squeezing. 
\bigskip\\
Local Floer homology is supposed to assign to compact subsets $K \subset M$ a chain complex,
\[ SK_K^\bullet(M) \]
read as ``symplectic cochains of $M$ with support in $K$''. This complex has the property that an embedding $K' \subset K$ induces a map,
\[ SC^\bullet_K(M) \to SC^\bullet_{K'}(M) \]
This does not give an obstruction (how do you obstruct having a map! it could be zero) However, there is additional structure (which is a filtration and a canonical closed element $e_K$ of degree $0$ which are functorial with respect to restriction). Suppose that $K' \subset K$ then $e_{K} \mapsto e_{K'}$. If both complexes are acyclic then there is some $\alpha$ with $e_{K} = \d{\alpha}$ so $\iota^* \alpha$ has to kill $e_{K'}$. Then you show this is impossible. 

\subsection{Topics}

\begin{enumerate}
\item algebraic structure on Floer homology: 
\begin{enumerate}
\item pair of pants (with marked points) product

\item circle action coming from moving the marked points
\end{enumerate}
these together satisfies the BV relation which gives a Lie bracket. Then $e_K$ is the unit.  

\item The ring structure is commutative. What sort of commutative rings appear? Interesting examples where it is not acyclic e.g. annulus of area $A$ in which case get a completion of Laurent Polynomials (completed via the filtration). Has an interpretation in rigid analytic geometry. It is the ring of analytic functions on the Larent domain in rigid analytic geometry. Something to do with Mirror symmetry. 

\item $\P^1 \sm \{ 0, 1\}$ is an annulus so this gives an example of: $X$ smooth projective variety $D$ normal crossings divisor then $X \sm D$ is an interesting symplectic manifold. The Local Floer homology of $X \sm D$ is related to GW theory of curves in $X$ with special marknigs along $D$. 

\item Introduce Lagrangians in this:
\[ L \subset X \supset D \]
study $HF^r_K(L, L)$ which gives the endomorphisms of $L$ inside the local Fukaya category.

\item Varalgures descent: there exists a class of covers of $M$ with the Mayer-Vietoris property: if $\cup K_i = M$ is a finite cover and $K_I = K_{i_1} \cap \cdots \cap K_{i_d}$. Now, $SC^\bullet_M(M) \iso QC^\bullet(M)$ is quasi-isomorphic to quantum-cohomologly. Then there is a diagram,
\begin{center}
\begin{tikzcd}
SC^\bullet_M(M) \arrow[r] \arrow[d] & \bigoplus_{i} SC_{K_i}^r(M) \arrow[r] & \bigoplus_{i < j} SC^r_{K_{ij}}(M) \arrow[r] & \cdots 
\\
QC^\bullet(M)
\end{tikzcd}
\end{center}
Then the top complex is acyclic. Therefore, $SC^\bullet_M(M) \iso \check{C}^\bullet(A, SC^\bullet_{K_I})$ is a quasi-isomorphism. For example the hemisphere cover of $\CP^1$ satisfies the Mayer-Vietoris property. 

\item Combining 4 + 5 gives a new method for computing the Fukaya category. With 3 gives a method for proving Homological mirror symmetry. 
\end{enumerate}

\begin{rmk}
$SC^\bullet_M(M) \iso QC^\bullet(M)$ is quasi-isomorphic to quantum-cohomologly. Then there is a diagram,
\begin{center}
\begin{tikzzcd}
\end{center}
\end{rmk}

\section{Hamiltonian $HF$}

Let $M$ be a closed manfold. And $H : M \times S^1 \to \R$ be a periodic Hamiltonian such that the flow $\varphi_H$ generated by $X_{H_t}$ has nondegenerate fixed points. Non-degeneracy, is the statement that,
\[ \Gamma_{\varphi_H} = \{ (x, \varphi_H(x)) \mid x \in M \} \subset M \times M \]
and $\Delta$ are transverse. We will often want to weaken this in applications. 
 
\end{document}




\documentclass[12pt]{extarticle}
\usepackage[utf8]{inputenc}
\usepackage[english]{babel}
\usepackage[a4paper, total={6in, 9in}]{geometry}
\usepackage{tikz-cd}
 
\usepackage{amsthm, amssymb, amsmath, centernot}

\newcommand{\notimplies}{%
  \mathrel{{\ooalign{\hidewidth$\not\phantom{=}$\hidewidth\cr$\implies$}}}}
 
\renewcommand\qedsymbol{$\square$}
\newcommand{\cont}{$\boxtimes$}
\newcommand{\divides}{\mid}
\newcommand{\ndivides}{\centernot \mid}
\newcommand{\Z}{\mathbb{Z}}
\newcommand{\R}{\mathbb{R}}
\newcommand{\N}{\mathbb{N}}
\newcommand{\C}{\mathbb{C}}
\newcommand{\Zplus}{\mathbb{Z}^{+}}
\newcommand{\Primes}{\mathbb{P}}
\newcommand{\colim}[1]{\mathrm{colim}(#1)}
\newcommand{\Ob}[1]{\mathrm{Ob}(#1)}
\newcommand{\cat}[1]{\mathcal{#1}}
\newcommand{\id}{\mathrm{id}}
\newcommand{\Hom}[2]{\mathrm{Hom}\left( #1, #2 \right)}
\newcommand{\catHom}[3]{\mathrm{Hom}_{#1}\left( #2, #3 \right)}
\newcommand{\Top}{\mathbf{Top}}
\newcommand{\pTop}{\mathbf{Top}_{\bullet}}
\newcommand{\Set}{\mathbf{Set}}
\newcommand{\pSet}{\mathbf{Set}_\bullet}
\newcommand{\hTop}{\mathbf{hTop}}
\newcommand{\phTop}{\mathbf{hTop}_{\bullet}}
\renewcommand{\Im}[1]{\mathrm{Im}(#1)}
\newcommand{\homspace}[2]{\left< #1, #2 \right>}
\newcommand{\rp}{\mathbb{RP}}
\newcommand{\coker}[1]{\mathrm{coker}\: #1}
\newcommand{\Tr}[1]{\mathrm{Tr}\left( #1 \right)}

\renewcommand{\d}[1]{\: \mathrm{d}#1 \:}
\newcommand{\dn}[2]{\: \mathrm{d}^{#1} #2 \:}
\newcommand{\deriv}[2]{\frac{\d{#1}}{\d{#2}}}
\newcommand{\nderiv}[3]{\frac{\dn{#1}{#2}}{\d{#3^2}}}
\newcommand{\pderiv}[2]{\frac{\partial{#1}}{\partial{#2}}}
\newcommand{\parsq}[2]{\frac{\partial^2{#1}}{\partial{#2}^2}}

\theoremstyle{definition}
\newtheorem{theorem}{Theorem}[section]
\newtheorem{lemma}[theorem]{Lemma}
\newtheorem{proposition}[theorem]{Proposition}
\newtheorem{example}[theorem]{Example}
\newtheorem{corollary}[theorem]{Corollary}
\newtheorem{remark}{Remark}

\newenvironment{definition}[1][Definition:]{\begin{trivlist}
\item[\hskip \labelsep {\bfseries #1}]}{\end{trivlist}}


\newenvironment{lproof}{\begin{proof} \renewcommand{\qedsymbol}{}}{\end{proof}}
\renewcommand{\mod}[3]{\: #1 \equiv #2 \: mod \: #3 \:}
\newcommand{\nmod}[3]{\: #1 \centernot \equiv #2 \: mod \: #3 \:}
\newcommand{\ndiv}{\hspace{-4pt}\not \divides \hspace{2pt}}
\newcommand{\gen}[1]{\langle #1 \rangle}
\newcommand{\hook}{\hookrightarrow}
\newcommand{\Tor}[4]{\mathrm{Tor}^{#1}_{#2} \left( #3, #4 \right)}
\newcommand{\Ext}[4]{\mathrm{Ext}^{#1}_{#2} \left( #3, #4 \right)}

\tikzset{
    labl/.style={anchor=south, rotate=90, inner sep=.5mm}
}

\renewcommand{\bf}[1]{\mathbf{#1}}
\newcommand{\Class}[2]{\mathcal{C}^{#1} \left( #2 \right)}
\newcommand{\Res}[2]{\mathrm{Res}_{#1} \: #2}
\newcommand{\F}{\mathcal{F}}
\newcommand{\G}{\mathcal{G}}
\renewcommand{\O}{\mathcal{O}}

\newcommand{\Xcut}{X_{\text{cut}}}

\begin{document}

\title{Introduction to Complex Analysis and Riemann Surfaces}
\author{Ben Church}
\maketitle
\tableofcontents
\newpage

\section{Holomorphic Maps}

\begin{definition}
A subset $\Omega \subset \C$ is a domain if $\Omega$ is open and connected.
\end{definition}

\begin{definition}
A map $f : \Omega \to \C$ is holomorphic at $z \in \Omega$ if the limit,
\[ f'(z) = \lim_{h \to 0} \frac{ f(z + h) - f(z) }{h} \]
exists. The map $f$ is holomorphic on $\Omega$ if it is holomorphic at each $z \in \Omega$. 
\end{definition}

\begin{proposition}
Let $f : \Omega \to \C$ be holomorphic at $z \in \Omega$. Then we may write $f$ as a function of two real variables as, $f(x, y) = f(x + i y)$. This done,
\[ f'(z) = \pderiv{f}{x} = \frac{1}{i} \pderiv{f}{y} \]
and thus,
\[ \pderiv{f}{x} + i \pderiv{f}{y} = 0 \] 
\end{proposition}


\begin{proposition}
\[ \pderiv{f}{z} = \frac{1}{2} \left[ \pderiv{f}{x} - i \pderiv{f}{y} \right] \quad \text{and} \quad \pderiv{f}{\bar{z}} = \frac{1}{2} \left[ \pderiv{f}{x} + i \pderiv{f}{y} \right] \]
Therefore, if $f$ is holomorphic then 
\[ \pderiv{f}{z} = f'(z) \quad \text{and} \quad \pderiv{f}{\bar{z}} = 0 \]
\end{proposition}


\begin{definition}
Let $U \subset \R^m$ then denote the vectorspace of continuous functions $U \to \C$ by $\Class{0}{U}$ and for $n > 0$ define, 
\[ \Class{n}{U} = \{ f : U \to \R^m \mid \forall p \in U : f'_p \text{ exists and } \forall \bf{v} \in \R^n : f'(\bf{v}) \in \Class{n-1}{U} \} \] 
where $f' \cdot \bf{v}$ is the map $p \mapsto f'_p(\bf{v})$. Furthermore, the space of smooth functions is,
\[ \Class{\infty}{U} = \bigcap_{k} \Class{k}{U} \] 
\end{definition}


\begin{theorem}
Let $\Omega$ be a domain and $f : \Omega \to \C$. Then the following are equivalent,
\begin{enumerate}
\item $f : \Omega \to \C$ is holomorphic.
\item $f \in \Class{1}{\Omega}$ and 
\[ \pderiv{f}{\bar{z}} = 0 \]
\item $f \in \Class{1}{\Omega}$ and for $D \subseteq \Omega$ with piecewise $\Class{1}{\Omega}$ boundary we have \[ \oint_{\partial D} f(z) \d{z} = 0 \]
\item $\forall B_{r}(w) \subsetneq \Omega$ we have,
\[ f(z) = \frac{1}{2 \pi i} \oint_{\partial B_{r}(w)} \frac{f(\zeta)}{\zeta - z} \d{\zeta} \] 
for all $z \in B_r(w)$. 
\item $\forall w \in \Omega \exists r > 0$ such that whenever $|z - w| < r$ we have,
\[ f(z) = \sum_{n = 0}^\infty a_n(x - w)^n \]
\end{enumerate}
\end{theorem}

\begin{proof}
We will show that,
\[ (2) \iff (3) \implies (4) \implies (5) \implies (1) \implies (2) \]
\begin{enumerate}
\item[$(4) \implies (5) $]
We assume that,
\[ f(z) = \frac{1}{2 \pi i} \oint\displaylimits_{|\zeta - w| = r} \frac{f(\zeta)}{\zeta - z} 
\d{\zeta} \]
We express the function,
\[ \frac{1}{\zeta - z} = \frac{1}{\zeta - w - (z - w)} = \frac{1}{\zeta - w} \frac{1}{1 - \left( \frac{z -w}{\zeta - w} \right)} = \frac{1}{\zeta - w} \sum_{n = 0}^\infty \left( \frac{z - w}{\zeta - w} \right)^n = \sum_{n = 0}^{\infty} \frac{(z - w)^n}{(\zeta - w)^{n+1}} \]
Then, formally,
\[ f(z) = \frac{1}{2 \pi i} \oint\displaylimits_{|\zeta - w| = r} f(\zeta) \left( \sum_{n = 0}^{\infty} \frac{(z - w)^n}{(\zeta - w)^{n+1}} \right) \d{\zeta} = \sum_{n = 0}^\infty \left( \frac{1}{2 \pi i} \oint\displaylimits_{|\zeta - w| = r} f(\zeta) \frac{\d{\zeta}}{(\zeta - w)^{n+1}} \right) (z - w)^n \]
However, to interchange the sum and integral we must establish uniform and absolute convergence. We know that $| \zeta - w | = r$ and $z \in B_r(w)$ so $|z - w| < r$ and thus the sum,
\[ \sum_{n = 0}^\infty \left| \frac{z - w}{\zeta - w} \right|^n  \]
converges. Furthermore, 
\[ \left| \left( \frac{z - w}{\zeta - w} \right)^n \right| = \left| \frac{z - w}{\zeta - w} \right|^n < \left| \frac{z - w}{\zeta - w} \right| = M < 1 \]
so the functions are bounded by $M^n$ whose sum coverges and thus by the Weierstrass $M$-test the series converges absolutly and uniformly. Therefore,
take,
\[ a_n = \frac{1}{2 \pi i} \oint\displaylimits_{|\zeta - w| = r} \frac{f(\zeta) \d{\zeta}}{(\zeta - w)^{n+1}} \]
\item[$(5) \implies (1)$]
It is clear that if,
\[ f(z) = \sum_{n = 0}^\infty a_n(x - w)^n \]
then,
\[ f'(z) = \sum_{n = 1}^\infty n a_n(x - w)^{n-1} \]
exists. 
\item[$(1) \implies (2)$]
Suppose that $\Omega = B_{\delta}(w)$. For each $z \in \Omega$, let $\ell_z$ be the segment joining $w$ to $z$ and define,
\[ F(z) = \int_{\ell_z} f(\zeta) \d{\zeta} \] 
Now compute the ratio,
\[ \frac{F(z + h) - F(z)}{h} = \frac{1}{h} \left[ \int_{\ell_z} f(\zeta) \d{\zeta} - \int_{\ell_{z + h}} f(\zeta) \d{\zeta} \right] \]
(PROGRESS)
Because the integral over the tringle is zero, we have,
\[ \frac{1}{h} \left[ \int_{\ell_z} f(\zeta) \d{\zeta} - \int_{\ell_{z + h}} f(\zeta) \d{\zeta} \right] = \frac{1}{h} \int_{z}^{z + h} f(\zeta) \d{\zeta} = \int_0^1 f(z + th) \d{t} \to f(z) \]
where we have parametrized the path $z$ to $z + h$ by $z + th$ for $0 \le t \le 1$. 
Thus, $F'(z) = f(z)$ which implies that $F$ is $\Class{1}{\Omega}$ and holomorphic so,
\[ \partial{f}{\bar{z}} = 0 \]
and thus satisfies (2). Therefore, by $(2) \implies (5)$ we have that $F$ is a power seies and thus $f = F'$ is a power series so $f$ is $\Class{1}{\Omega}$. Furthrermore, $f$ is holomorphic which implies that 
\[ \pderiv{f}{\bar{z}} = 0 \].
Therefore, we have (2). 
\end{enumerate}
\end{proof}

\begin{theorem}
For any $z_0 \in \Omega$, either $f \equiv 0$ in a neighborhood of $z_0$ or we can express $f = (z - z_0)^n u(z)$ for $u(z)$ holomorphic and $u(z) \neq 0$.
\end{theorem}

\begin{proof}
In a neighborhood of $z_0$, we can write,
\[ f(z) = \sum_{n = 0}^\infty n_n(z - z_0)^n\]
Either $c_n = 0$ for each $n$ so $f = 0$ or $c_N \neq 0$ for some $n$ and $c_n - 0$ for $n < N$. Therefore,
\[ f(z) = \sum_{n \ge N}^\infty c_n(z - z_0)^n = (z - z_0)^N \left( \sum_{m = 0}^\infty c_{N + m} (z - z_0)^m \right) = (z - z_0)^N u(z) \]
Furthermore, $u(z_0) = c_N \neq 0$ so, by continuity, there exists a neighborhood of $z_0$ on which $n(z) \neq 0$.  
\end{proof}


\begin{theorem}
Let $f$ be holomorphic on a domain $\Omega$. If $f \equiv 0$ on some open set inside $\Omega$ then $f \equiv 0$ on all of $\Omega$. 
\end{theorem}

\begin{proof}
Define,
\[ \Omega' = \{ z \in \Omega \mid f \equiv 0 \text{ on an open neighborhood of } z \} \]
Clearly $\Omega'$ is open in $\Omega$ because each $z \in \Omega'$ in inside an open neighborhood of $\Omega$ on which $f$ vanishes so is contained in an open neighborhood of $\Omega'$.
\\
Take $z_1 \notin \Omega'$. Thus, $f$ does not vanish identically on every neighborhood of $z$ so there exists a neighborhood $U$ such that $f(z) = (z - z_1)^N u(z)$ for $u(z) \neq 0$. Then $f(z) \neq 0$ on $U \setminus \{z_1\}$. Therefore, $U \subset (\Omega')^C$ because $f$ is nonzero on $U \setminus \{z\}$ and thus cannot be identically zero on any neighborhood of any point of $U$. Thus, $(\Omega')^C$ is open so $\Omega'$ is clopen. However, $\Omega$ is connected and thus $\Omega' = \Omega$.  
\end{proof}


\begin{example}
Consider the solution to the equation $w^2 = z$. First take the open domain $U = \C \setminus \{ z \in \R \mid z \ge 0 \}$ and for $z = r e^{i \theta}$ with $0 < \theta < 2 \pi$ define $w = r^{1/2} e^{i \theta / 2} = \sqrt{z}$. The function $f(z) = w$ is perfectly holomorphic on $U$. However, the line we choose to remove is artificial, any cut will work with a redefinition of the angular interval. We solve this problem by taking two copies of $U$ called (I) and (II) and then constructing a surface $X$ by gluing (I) and (II) along the cuts such that moving across the cut in $\C$ corresponds to changing sheets. We can define $w$ on all of $X$ by $w(p) = w(z) = \sqrt{z}$ if $p$ is on sheet (I) at position $z$ and otherwise $w(p) = - w(z) = - \sqrt{z}$ if $p$ is on sheet (II) at position $z$.  
\bigskip\\
Topologically, $X$ is a sphere minus two points. We call $\hat{X}$ the compactified version of $X$ constrcuted by adding back the two points such that $\hat{X} \cong S^2$. 
\end{example}


\section{Meromorphic Functions}

\begin{definition}
A function $f : \Omega \to \C$ is meromorphic if, near any $z_0 \in \Omega$, it can be written as,
\[ f(z) = \sum_{n \ge - N} c_n (z - z_0)^n \] 
We call $N$ the order of the pole (assuming that $c_n \neq 0$) and $c_{-1}$ the residue at $z_0$. 
\end{definition}


\begin{theorem}[Residue]
Let $f : \Omega \to \C$ be meromorphic and $D \subset \overline{D} \subset \Omega$ be a domain in $\Omega$ with piecewise smooth boundary $\partial D$ such that no poles of $g$ lie on $\partial D$. Then,
\[ \oint_{\partial D} f(z) \d{z} = 2 \pi i  \sum_{p \in D} \Res{f(p)} \]
\end{theorem}

\begin{proof}
We can deform the path $\partial D$ to a sum of small circles of radius $r$ surrounding each pole. Since $f$ is holomorphic on the region $D$ minus these circles the two integrals along these paths (whose difference is the integral over the boundary) are equal. Thus,
\begin{align*}
\oint_{\partial D} f(z) \d{z} - 2 \pi i \sum_{p \in D} \Res{p}{f} & = \sum_{p \in D} \left[ \oint_{\partial B_r(p)}  f(p + z) \d{z}  - 2 \pi i \Res{p}{g}   \right]
\\
& = \sum_{p \in D} \left[ \int_0^{2\pi} i \bigg( f(p + r e^{i\theta}) r e^{i \theta}  - \Res{p}{g} \bigg) \d{\theta}   \right]
\end{align*}
However,
\[ \Res{p}{f} = \lim_{z \to p} (z - p) f(z) = \lim_{h \to 0} f(p + h) h \]
and thus, for each $\epsilon > 0$ we can choose some $\delta$ such that $r < \delta$ implies that,
\[ \left| f(z + r r^{i \theta}) r e^{i \theta} - \Res{p}{f} \right| < \epsilon \]
Therefore,
\begin{align*}
\left| \oint_{\partial D} f(z) \d{z} - 2 \pi i \sum_{p \in D} \Res{p}{f} \right| & \le \sum_{p \in D} \left[ \int_0^{2\pi} \Big| f(p + r e^{i\theta}) r e^{i \theta}  - \Res{p}{g} \Big| \d{\theta}   \right]
\\
\le \sum_{p \in D} \int_0^{2 \pi} \epsilon = 2 \pi N \epsilon 
\end{align*}
where $N$ is the number of poles. Since $\epsilon$ is arbitrary,
\[ \oint_{\partial D} f(z) \d{z} = 2 \pi i \sum_{p \in D} \Res{p}{f} \]
\end{proof}

\begin{theorem}
Let $f : \Omega \to \C$ be meromorphic and $D \subset \overline{D} \subset \Omega$ be a domain in $\Omega$ with piecewise $\mathcal{C}^1$ boundary $\partial D$ such that no poles of $g$ lie on $\partial D$. Then,
\[ 
\frac{1}{2 \pi i} \oint_{\partial D} \frac{f'(z)}{f(z)} \d{z} = \text{(\# of zeros)} - \text{(\# of poles)}
\]
\end{theorem}

\begin{theorem}
At each point $p \in D$ we can expand,
\[ f(z) = (z - p)^N u(z) \]
where $u$ is holomorphic and nonvanishing. Therefore,
\[ \frac{f'(z)}{f(z)} = \deriv{}{z} \log{f(z)} = \deriv{}{z} \left[ (z - p)^N u(z) \right] = \frac{N}{x - p} + \frac{u'(z)}{u(z)} \]
Thus when $f$ has either a zero ($N > 0$) or a pole ($N < 0$) the logarithmic derivative has residue,
\[ \Res{p}{\left(\frac{f'}{f}\right)} = N \]
Therefore the result holds by the residue theorem. 
\end{theorem}

\begin{corollary}
Let $f : \Omega \to \C$ be holomorphic take $w \in \C$, then the number of solutions in $D$ to the equation $f(z) - w = 0$ is equal to,
\[ \#\{ z \in D \mid f(z) = w \} = \oint_{\partial D} \frac{f'(z)}{f(z) - w} \d{z}  \]
\end{corollary}

\begin{proof}
Since $f - w$ is holomorphic on $\Omega$ is has no poles. Therefore, the only residues are from roots of $f - w$ i.e. solutions to $f(z) - w = 0$. As above, the integral of the logarithmic derivative counts the number of such poles.  
\end{proof}


\section{Taylor's Theorem}

\begin{theorem}
A function $f : U \to \C$ of a real variable defined on open $U \subset \R$ is the restriction of some holomorphic function $f : \Omega \to \C$ on a domain $\Omega$ containing $U \subset \R$ iff $f$ is (real) analytic.
\end{theorem}

\begin{proof}

\end{proof}

\begin{theorem}[Cauchy]
Let $f : \Omega \to \C$ be holomorphic, for any subset $D \subset \Omega$ homeomorphic to a disc with $\Class{1}{I}$ boundary and $w \in D^\circ$ we have,
\[ f^{(n)}(w) = \frac{n !}{2 \pi i} \oint_{\partial D} \frac{f(z)}{(z - w)^{n+1}} \d{z} \]
\end{theorem}

\begin{corollary}[Cauchy Estimate]
Let $f : \Omega \to \C$ be holomorphic and $w \in \Omega$. Let $r > 0$ be such that $B_r(w) \subset \Omega$ then,
\[ \frac{|f^{(n)}(w)|}{n!} \le \frac{\sup \{ |f(z)| \mid z \in \partial B_r(w) \}}{r^n} \]
\end{corollary}

\begin{proof}
Via the Cauchy derivative formula,
\[ f^{(n)}(w) = \frac{n !}{2 \pi i} \oint_{\partial B_r(w)} \frac{f(z)}{(z - w)^{n+1}} \d{z}  \]
Taking the norm of both sides,
\begin{align*}
\frac{|f^{(n)}(w)|}{n!} & = \frac{1}{2 \pi} \left| \oint_{\partial B_r(w)} \frac{f(z)}{(z - w)^{n+1}} \d{z} \right| 
\\
& = \frac{1}{2 \pi} \left| \int_0^{2\pi} \frac{f(z)}{(r e^{i \theta})^{n+1}} i r e^{i \theta} \d{t} \right| \le \frac{1}{2 \pi} \int_0^{2 \pi} \frac{|f(z)|}{r^n} \d{t} \le \frac{\sup \{ |f(z)| \mid z \in \partial B_r(w) \}}{r^n}
\end{align*}
\end{proof}

\begin{corollary}
Let $f : \Omega \to \C$ be holomorphic and $w \in \Omega$. If $B_r(w) \subset \Omega$ then the talor series about $w$ has radius of convergence at least $r$.
\end{corollary}

\begin{proof}
Consider,
\[ T_{N,w}(z) = \sum_{n = 0}^N \frac{f^{(n)}(w)}{n!} (z - w)^n \] 
For $z \in B_r(w)$ consider,
\[ \sum_{n = 0}^N \left| \frac{f^{(n)}(w)}{n!} (z - w)^n \right| = \sum_{n = 0}^N \frac{|f^{(n)}(w)| r^n}{n!} \left( \frac{|z - w|}{r} \right)^n \le \sum_{n = 0}^N M_r x^n \]
where $M_r = \sup \{ |f(z)| \mid z \in \partial B_r(w) \}$ and $x = |z - w|/r < 1$ since $z \in B_r(w)$. Then, 
\[ \sum_{n = 0}^N M_r x^n = M_r \sum_{n = 0}^N x^n \]
converges for $N \to \infty$ so $\lim\limits_{N \to \infty} T_{N,w}(z)$ converges absolutly on $B_r(w)$. 
\end{proof}

\begin{theorem}
An entire function $f : \C \to \C$ is globally a power series. In particular, $f$ is everywhere equal to its taylor series about any point. 
\end{theorem}

\begin{proof}
Since for any $r > 0$ and $w \in \C$ we have $B_r(w) \subset \C$ the above argument shows that the radius of convergence of $T_w(z)$ is infinite. 
\end{proof}

\begin{remark}
This is completely false for everywhere real analytic functions. For example,
\[ f(x) = \frac{1}{1 + x^2} \]
is analytic but has finite radius of covergence about each point. This is because its extension to the complex plane has poles at $x = \pm i$ and thus is not entire. 
\end{remark}

\begin{theorem}
Let $f : \Omega \to \C$ be holomorphic and take $w \in \Omega$ and $r > 0$ s.t. $B_r(w) \subset \Omega$. Then, for all $z \in B_r(w)$,
\[ T_w(z) = \sum_{n = 0}^\infty \frac{f^{(n)}(w)}{n!} (z - w)^n = f(z) \]
Furthermore, the error term,
\[ R_{N, w}(z) = f(z) - T_{N, w}(z) = \frac{}{} \]
\end{theorem}

\begin{proof}
We know that the sum,
\[ T_w(z) = \sum_{n = 0}^\infty \frac{f^{(n)}(w)}{n!} (z - w)^n \]
is absolutly and uniformly convergent. By the Cauchy integral formula,
\[ \frac{f^{(n)}(w)}{n!} = \frac{1}{2 \pi i} \oint_{\partial B_r(w)} \frac{f(\zeta)}{(\zeta - w)^{n+1}} \d{\zeta} \]
and thus,
\begin{align*}
T_w(z) & = \sum_{n = 0}^\infty \frac{f^{(n)}(w)}{n!} (z - w)^n = \sum_{n = 0}^\infty  \left( \frac{1}{2 \pi i} \oint_{\partial B_r(w)} \frac{f(\zeta)}{(\zeta - w)^{n+1}} \d{\zeta} \right) (z - w)^n
\\
& = \frac{1}{2 \pi i } \left( \oint_{\partial B_r(w)} f(\zeta) \sum_{n = 0}^\infty \left[ \frac{(z - w)^n}{(\zeta - w)^{n+1}}   \right] \d{\zeta} \right)
\\
& = \frac{1}{2 \pi i} \oint_{\partial B_r(w)}  \frac{f(\zeta)}{(\zeta - w) - (z - w)} \d{\zeta} = \frac{1}{2 \pi i} \oint_{\partial B_r(w)} \frac{f(\zeta)}{\zeta - z} \d{\zeta} = f(z) 
\end{align*}
where I may interchange the integrals and sums by uniform convergence. Furthermore, I have used the series,
\[  \frac{(z - w)^n}{(\zeta - w)^{n+1}} = \frac{1}{\zeta - w} \frac{(z - w)^n}{(\zeta - w)^{n}} =  \frac{1}{\zeta - w} \cdot \frac{1}{1 - \frac{z - w}{\zeta - w}} = \frac{1}{(\zeta - w) - (z - w)} \]
which converges because $\zeta \in \partial B_r(w)$ so $|\zeta - w| = r$ and $z \in B_r(w)$ so $|z - w| < r$ and thus,
\[ \left| \frac{z - w}{\zeta - w} \right| < 1 \]
Now we may compute the error term as follows,
\begin{align*}
R_{w, N}(z) & = f(z) - T_{w, N}(z) = \sum_{n = 0}^\infty \frac{f^{(n)}(w)}{n!} (z - w)^n  - \sum_{n = 0}^N \frac{f^{(n)}(w)}{n!} (z - w)^n 
\\
& = \sum_{n = {N+1}}^\infty \frac{f^{(n)}(w)}{n!} (z - w)^n  = \sum_{n = {N+1}}^\infty  \left( \frac{1}{2 \pi i} \oint_{\partial B_r(w)} \frac{f(\zeta)}{(\zeta - w)^{n+1}} \d{\zeta} \right) (z - w)^n
\\
& = \frac{1}{2 \pi i } \left( \oint_{\partial B_r(w)} f(\zeta) \sum_{n = {N+1}}^\infty \left[ \frac{(z - w)^n}{(\zeta - w)^{n+1}}   \right] \d{\zeta} \right)
\\
& = \frac{(z - w)^{N+1}}{2 \pi i} \left( \oint_{\partial B_r(w)}  \frac{f(\zeta)}{(\zeta - w)^{N+1} (\zeta - z)}  \d{\zeta} \right) 
\end{align*}
where,
\begin{align*}
\sum_{n = {N+1}}^\infty \left[ \frac{(z - w)^n}{(\zeta - w)^{n+1}}   \right] & = \frac{(z - w)^{N+1}}{(\zeta - w)^{N+2}} \sum_{n = 0}^\infty \left[ \frac{z - w}{\zeta - w} \right]^n 
\\
& = \frac{(z - w)^{N+1}}{(\zeta - w)^{N+2}} \cdot \frac{1}{1 - \frac{z - w}{\zeta - w}} = \frac{(z - w)^{N+1}}{(\zeta - w)^{N+1} (\zeta - z)} 
\end{align*}
\end{proof}

\begin{lemma}
Let $f : \R \to \R$ be $(n+1)$-differentiable on $[a, b]$ and $f^{(k)}(a) = 0$ for each $n \le k$. Then there exists some $\xi \in [a,b]$ such that,
\[ f(b) = \frac{f^{(n+1)}(\xi)}{(n+1)!} (b - a)^{n + 1} \] 
\end{lemma}

\begin{proof}
Suppose that $f$ is $(n+1)$-differentiable on $[a,b]$ and $f^{(n)}(a) = 0$. Consider the function,
\[ g(x) = f(x) - \frac{f(b)}{(b - a)^{n+1}} (x - a)^{n+1} \]
Then $g$ satisfies the same condition that $g^{(k)}(a) = 0$ for $k \le n$ and $g$ is $(n+1)$-differentiable on $[a,b]$ but also $g(b) = 0$. Now, for each $k \le n + 1$ I claim that there exists $\xi_k \in [a, b]$ such that $g^{(k)}(\xi_k) = 0$. For $k = 0$, by the mean value theorem, there exists $\xi_0 \in [a, b]$ such that,
\[ g'(\xi_0) = \frac{g(b) - g(a)}{b - a} = 0 \]
Now we proceed by induction. Suppose we have $\xi_k \in [a,b]$ such that $g^{(k)}(\xi_k) = 0$. Then for $k \le n$ we also know that $g^{(k)}(a) = 0$. Then, since $g^{(k)}$ is differentiable for $k \le n$, by the mean value theorem, there exists $\xi_{k + 1} \in [a, \xi_k] \subset [a, b]$ such that,
\[ g^{(k+1)}(\xi_{k+1}) = \frac{g^{(k)}(\xi_k) - g^{(k)}(b)}{b - a} = 0 \]
Proving the claim by induction. Finally,
\[ g^{(n+1)}(\xi_{n+1}) = f^{(n+1)}(\xi_{n+1}) - f(b) \frac{(n+1)!}{(b - a)^{n+1}}  = 0 \] 
which implies that,
\[ f(b) = \frac{f^{(n+1)}(\xi_{n+1})}{(n+1)!} (b - a)^{n+1} \]
\end{proof}

\begin{theorem}[Lagrange Error Form]
Let $f : \R \to \R$ be $(n+1)$-differentiable on $[a, b]$. Then the remainder term,
\[ R_{a,n}(b) = f(b) - T_{a,n}(b) = \frac{f^{(n+1)}(\xi)}{(n + 1)!} (b - a)^{n + 1} \]
for some $\xi \in (a, b)$. 
\end{theorem}

\begin{proof}
Consider the function,
\[ R_{a,n}(x) = f(x) - T_{a,n}(x) \]
which is $(n+1)$-differentiable on $[a,b]$ and satisfies $R^{(k)}_{a,n}(a) = 0$ for each $k \le n$ so by the lemma, there exists $\xi \in [a,b]$ such that,
\[ R_{a,n}(b) = \frac{f^{(n+1)}(\xi)}{(n + 1)!} (b - a)^{n + 1}  \]
Because $R_{a,n}^{(n+1)}(b) = f^{(n+1)}(b)$ since the Taylor partial sum has order $n$.
\end{proof}

\end{document}


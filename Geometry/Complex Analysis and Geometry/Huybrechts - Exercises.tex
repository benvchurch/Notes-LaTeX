\documentclass[12pt]{article}

\usepackage{import}
\import{../}{GeometryCommands}


\begin{document}



\section{Chapter 2}

\section{2.2}

Given an exact sequence of vector bundles,
\begin{center}
\begin{tikzcd}
0 \arrow[r] & L \arrow[r] & E \arrow[r] & F \arrow[r] & 0
\end{tikzcd}
\end{center}
consider the exact sequence,
\begin{center}
\begin{tikzcd}

\end{tikzcd}
\end{center}

\section{2.5}


Let $L, L^*$ be holomorphic line bundles on a compact complex manifold $X$. Suppose that $L$ and $L^*$ admit nonzero global holomorphic sections $s, s'$. Then consider $s \otimes s'$ a global section of $L \otimes L^* \cong \struct{X}$. However, all nonzero sections of $\struct{X}$ are nonvanishing because $X$ is compact and thus $H^0(X, \struct{X}) = \C$. Therefore, $s$ and $s'$ are nonvanishing meaning that $L \cong L^* \cong \struct{X}$.

\section{2.6}

\subsubsection{1}

I think $f$ is holomorphic iff $\d{f}(Iv) = i \d{f} (v)$

\subsubsection{2}

\subsubsection{3}

\subsubsection{4}


Let $f : X \to Y$ be a surjective holomorphic map between connected xomplex manifolds. We want to look at the smooth locus of $f$.

I claim the following is true: for a morphism of vector budles (not necessarily constant rank) $\phi : \E_1 \to \E_2$ then $\phi$ has full rank $k = \min{\{m, n \}}$ iff the morphism $\phi' : \bigwedge^k \E_1 \to \bigwedge^k \E_2$ is nonzero (is this true).
\bigskip\\
Therefore, the locus where $\phi$ is not full rank is the vanishing the section
\[ \phi' \in \mathcal{HOM}_{\struct{X}}\left(\bigwedge^k \E_, \bigwedge^k \E_2\right) \]
Now apply this to the map $f^* \Omega_Y \to \Omega_X$ to get the nonsmooth 
locus.

\subsubsection{6}

The cousins' problem has a solution because $H^1(X, \struct{X}) = 0$. Question: why is every hypersurface defined by a $H^0(K^\times / \struct{X}^\times)$. Question: how are we supposed to use the poincare lemma.

\subsubsection{7}

\renewcommand{\A}{\mathcal{A}}

We define,
\[ H^{p,q}_{\mathrm{BC}}(X) = \frac{\{ \alpha \in \A^{p,q}(X) \mid \d{\alpha} = 0 \}}{ \partial \bar{\partial} \A^{p-1,q-1}(X)} \] 
This makes sense because if $\alpha = \partial \bar{\partial} \gamma$ then \[ \d{\alpha} = \partial^2 \bar{\partial} \gamma - \bar{\partial}^2 \partial \gamma = 0 \]
Now, the inclusion of $\d{}$-closed forms into $\bar{\partial}$-closed forms induces a map,
\[ H^{p,q}_{\mathrm{BC}}(X) \to H^{p,q}(X) \]
which is well-defined because if $\alpha = \partial \bar{\partial} \gamma$ then $\alpha = - \bar{\partial} \partial \gamma$ and is thus $\bar{\partial}$-exact. If $X$ is furthermore compact Kahler then by the $\partial \bar{\partial}$-lemma we see if $\alpha$ maps to zero i.e. $\alpha = \partial \bar{\partial} \beta$ and $\d{\alpha} = 0$ then $\alpha = \d{\gamma}$ so the map is injective. Furthermore, by the Hodge decomposition, $H^{p,q}(X)$ can be represented by Harmonic forms which are $\d$-closed and thus this map is surjective as well.

\subsubsection{8}

Is this just because we can take $M \to M$ via complex conjugation.
 
 
\subsection{3.2}

\subsubsection{3.2.4}
What does this really mean?? Ask Ron.

\subsubsection{3.2.6}

Let $X$ be a compact \kahler manifold. Then,
\[ H^{n}(X, \C) = \bigoplus_{p + q = n} H^{p,q}(X) \]
Furthermore, $H^{q,p} = \overline{H^{p,q}}$. Therefore,
\[ b_{2k + 1} = \sum_{p + q = 2k + 1} h^{p, q} = \sum_{i = 0}^k (h^{2 k + 1 - i, i} + h^{i, 2k + 1 - i}) = 2 \sum_{i = 0}^k h^{2k + 1 - i, i} \]
is even.

\subsubsection{3.2.7}

No! (PROVE IT)

\subsubsection{3.2.8}

Let $X$ be a compact \kahler manifold. Let $\omega \in H^0(X, \Omega^p_X)$. Clearly, $\bar{\partial} \omega = 0$ since $\omega$ is a holomorphic $(p, 0)$-form. Furthermore, 
\[ \bar{\partial}^* \omega = - (\bar{\star} \circ \bar{\partial} \circ \bar{\star}) \, \omega \]
but $\bar{\star} \omega$ is a $(n - p, n)$-form and thus $\bar{\partial} \bar{\star} \omega = 0$. Therefore, $\bar{\partial} \omega = 0$ and $\bar{\partial}^* \omega = 0$ and thus $\Delta_{\bar{\partial}} \, \omega = 0$.

\subsubsection{3.2.13}

Let $X$ be a complex \kahler manifold and $\alpha \in \A^k(X)$ which is $\d$-closed and $\dc$-exact where $\dc = i (\bar{\partial} - \partial)$. Notice that $\d \dc = 2 i \partial \bar{\partial}$. Write $\alpha = \alpha^{k,0} + \cdots + \alpha^{0, k}$. (FINISH!!)

\subsubsection{3.2.14}

DO! 

\subsubsection{3.2.15}

DO!

\subsubsection{3.2.16}

Let $X$ be a compact \kahler manifold. Let $\omega$ and $\omega'$ be \kahler forms such that $[\omega] = [\omega'] \in H^2(X, \R)$. Then $\eta = \omega - \omega' = \d{\alpha}$ for some real $1$-form $\alpha$. Thus $\eta$ is a closed real $(1,1)$-form which is $\d$-exact and thus by the $\partial \bar{\partial}$-lemma $\eta = i \partial \bar{\partial} f$ for some $f \in \A^{0,0}$. Notice,
\[ \bar{\eta} = - i \bar{\partial} \partial \bar{f} = i \partial \bar{\partial} \bar{f} \]
however $\eta$ is real so $\bar{\eta} = \eta$ and thus $\bar{f} = f$ so $f \in \A^0_\R$ is a real function and,
\[ \omega = \omega' + i \partial \bar{\partial} f \]

\section{Extra Questions for Ron}

\subsubsection{1}

Kodaira embedding says that every positive line bundle is ample in the sense of having some power very ample. Does the algebraic geometry definition work here? I.e. $L$ is ample iff for each bundle $Q$ we have $Q \otimes L^n$ generated by global sections for $n \gg 0$. Do we need $Q$ to be arbitrary coherent sheaf.
\\
Yes, in fact we only need this for vector bundles because it then follows by resolution for all coherent sheaves.

\subsubsection{2}

If we have a big line bundle $H^0(X, L^{\otimes m}) \sim m^n$ then does it follow there is an ample line bundle i.e. $X$ is projective. I am guessing not. This is similar to asking if there are non algebraic examples of compact Moishezon manifolds $a(X) = \dim{X}$.

\subsubsection{3.3.1}

\subsubsection{3.3.2}



\subsubsection{3.3.3}


\section{Chapter 4}

\subsection{Section 4.3}

\subsubsection{4.3.1}

\subsubsection{4.3.2}

\subsubsection{4.3.3}

\subsubsection{4.3.4}

\subsubsection{4.3.5}

Let $X$ be complex manifold. Let $L$ be a holomorphic line bundle with a hermitian structure $h$ whose Chern connection has positive curvature. Then $F_\nabla \in \A^{1,1}(X)$ is an imaginary $(1,1)$-form. Furthermore, note that $F_\nabla = \bar{\partial} \partial \log{h}$ and thus,
\[ \d{F_\nabla} = (\partial + \bar{\partial}) \bar{\partial} \partial \log{h} = 0 \]
because $\bar{\partial}^2 = 0$ and $\partial \bar{\partial} \partial = - \partial^2 \bar{\partial} = 0$. Since $\omega = i F_\nabla$ is positive, it is a \kahler form. Furthermore if $X$ is compact then,
\[ \int_X A(L)^n = \int_X F_\nabla^n = \int_X \omega^n = n! \int_X \mathrm{vol}_\omega > 0 \]
(CHECK THIS! FACTORS OF I)

\subsubsection{4.3.6}

\subsubsection{4.3.7}

\subsubsection{4.3.8}

\subsubsection{4.3.9}

Let $X$ be a compact \kahler manifold with $b_1(X) = 0$. Suppose that $\nabla$ is a flat connection on $\struct{X}$ with $\nabla^{0,1} = \bar{\partial}$. Then $\nabla = \d + \omega$ where $\omega : \A^0(X) \to \A^1(X)$ is $\A^0(X)$-linear and thus $\omega \in \A^1(X)$. Futhermore, $\nabla^{0,1} = \bar{\partial}$ so $\omega$ is a smooth $(1,0)$-form. Now consider the curvature,
\[ F_\nabla = \nabla \circ \nabla(1) = \nabla (\omega \otimes 1) = \d{\omega} \otimes 1 - \omega \wedge \nabla(1) = \d{\omega} \otimes 1 - \omega \wedge \omega \otimes 1 = \d{\omega} \]
Since $\nabla$ is flat we must have $\d{\omega} = 0$. Thus $\omega$ defines a de Rham cohomology class $[\omega] \in H^1(X, \C)$ but $b_1(X) = 0$ so $\omega$ is exact. Take $\omega = \d{f}$ for some smooth function $f$. However, $\omega$ is a $(1,0)$-form so $f$ is holomorphic. But $X$ is compact so $f$ is constant and thus $\omega = 0$ showing that $\nabla = \d$.
\bigskip\\
Now suppose that $L$ is a line bundle on $X$ with $c_1(L) = 0$. From the exponential sequence,
\begin{center}
\begin{tikzcd}
H^1(X, \struct{X}) \arrow[r] & \Pic{X} \arrow[r, "c_1"] & H^2(X, \Z) 
\end{tikzcd}
\end{center}
and thus $\ker{c_1} = \Im{H^1(X, \struct{X}) \to \Pic{X}}$. However, $b_1(X) = 0$ so by the \kahler decomposition, $H^1(X, \struct{X}) = 0$. Therefore, $\ker{c_1}$ is trivial so $L = \struct{X}$.

\subsubsection{4.3.10}

Let $\nabla$ be a connection on a complex vector bundle $E$. We want to show that $E$ locally has parallel frames iff $F_\nabla = 0$.
\bigskip\\
Suppose that $E$ has a local frame $e_1, \dots, e_n$ of parallel sections over $U$ i.e. $\nabla e_i = 0$ and these are independent on each fiber. Since the curvature form $\omega_\nabla(s) = \nabla_1 \circ \nabla(s)$ is $\struct{X}$-linear, writing $s = f_i e_i$ we get, 
\[ \omega_\nabla(f_i e_i) = f_i \omega_\nabla(e_i) = f_i \nabla_1 \circ \nabla e_i = 0 \]
Therefore, $\omega_\nabla = 0$ so $\nabla$ must be flat.
\bigskip\\
Locally write $E|_U \cong \struct{U}^{\oplus n}$ write $e_i$ for a local frame of $E|_U$. Now write $\nabla e_j = \omega_{ij} \otimes e_i$ thus we see,
\[ \nabla (f_j e_j) = \d{f_j} \otimes e_j + \omega_{ij} f_j \otimes e_i = ( \d{f_i} + \omega_{ij} f_j) \otimes e_i \] 
Now, applying $\nabla_1 : \Omega^1_X \otimes E \to \Omega^2_X \otimes E$ we get,
\begin{align*}
\nabla_1 \circ \nabla (f_j e_j) & = \nabla_1 (\d{f_i} + \omega_{ij} f_j) \otimes e_i = \d{\d{f_i}} \otimes e_i + \d{(\omega_{ij} f_j)} \otimes e_i - (\d{f_i} + \omega_{ij} f_j) \wedge \nabla e_i 
\\
& = (\d{\omega_{ij}} \, f_j - \omega_{ij} \wedge \d{f_j}) \otimes e_i - (\d{f_i} + \omega_{ij} f_j) \wedge \omega_{ki} \otimes e_k
\\
& = \d{\omega_{ij}} f_j \otimes e_i + \d{f_j} \wedge \omega_{ij} \otimes e_i - \d{f_i} \wedge \omega_{ki} \otimes e_k + \omega_{ki} \wedge \omega_{ij} f_j \otimes e_k
\\
& = (\d{\omega_{ij}} + \omega_{ik} \wedge \omega_{kj}) f_j \otimes e_i
\end{align*}
Therefore,
\[ \omega_\nabla(f_j e_j) = (\d{\omega_{ij}} + \omega_{ik} \wedge \omega_{kj}) f_j \otimes e_i \]
is linear as it should be. Now assume $\nabla$ is flat i.e. $\omega_\nabla = 0$. Thus,
\[ \d{\omega_{ij}} + \omega_{ik} \wedge \omega_{kj} = 0 \] 
First, in the case $n = 1$ the connection is given by a $1$-form $\omega$. Then $\omega_\nabla = 0 \iff \d{\omega} = 0$ in which case locally $\omega = - \d{f}$ and thus $\nabla (fe) = \d{f} \otimes e + \omega \otimes e = 0$ so we get a frame of parallel sections.
\bigskip\\
Now we proceed by induction for the general case. First, using a $\GL(n, \C)$ transformation we can 

  Assume we can find a frame $e_1, \dots, e_{n-1}, s$ such that $\nabla e_i = 0$. 

\subsection{Section 4.4}

\subsubsection{4.4.2}

Let $X$ be a compact complex manifold and $L$ a basepoint-free line bundle. Then $L$ defines a map $f : X \to \P^N$ such that $f^* \struct{\P^N}(1) = L$. Let $h$ be the standard hermitian structure on $\struct{\P^N}(1)$ so $f^* h$ gives a hermitian structure on $L$. Taking the Chern connections $\nabla_{f^* h} = f^* \nabla_h$ and thus,
\[ F(L, f^* h) = F(f^* \struct{\P^N}(1), f^* h) = f^* F(\struct{\P^N}(1), h) = f^* \omega_{\text{FS}} \]
which is a positive form.
Therefore,
\[ c_1(L) = f^* [\omega_{\text{FS}}] \] so we see that,
\[ \int_{X} c_1(L)^{n} = \int_X (f^* \omega_{\text{FS}})^{n} = \int_X f^* \omega_{\text{FS}}^n \ge 0 \]

\subsubsection{4.4.4}

Ask Ron about interpretation!!

\subsubsection{4.4.9}

Note that $\End{E} \cong E^* \otimes E$ then,
\[ c_k(\End{E}) = \sum_{i + j = k} c_i(E^*) \cdot c_j(E) = \sum_{i + j} (-1)^i c_i(E) \cdot c_j(E) \]
In particular,
\[ c_1(\End{E}) = c_0(E) \cdot c_1(E) - c_1(E) \cdot c_0(E) = 0 \]
and likewise,
\[ c_2(\End{E}) = c_0(E) \cdot c_2(E) - c_1(E) \cdot c_1(E) + c_2(E) \cdot c_0(E) = 2 c_2(E) - c_1(E)^2 \]
Then if $E = L \oplus L$ where $L$ is a line bundle we have,
\[ c(L) = 1 + c_1(L) \]
and thus,
\[ c_1(E) = 2 c_1(L) \quad \text{and} \quad c_2(E) = c_1(L)^2 \]
Therefore, we see that,
\[ (4 c_2 - c_1^2)(E) = 4 c_1(E)^2 - (4 c_1(E))^2 = 0 \]
Furthermore, if $E \cong E^*$ then $c_{2k+1}(E) = c_{2k+1}(E^*) = (-1)^{2k + 1} c_{2k+1}(E) = - c_{2k + 1}(E)$ and thus $c_{2k + 1}(E) = 0$.

\subsubsection{4.4.10}

Let $L$ be a holomorphic line bundle on $X$ a compact K\"{a}hler manifold. Suppose that $c_1(L) = [\alpha]$ where $\alpha$ is closed a real $(1,1)$-form. Let $h_0$ be a Hermitian structure on $L$ then,
\[ c_1(L, h_0) = \frac{i}{2 \pi} \bar{\partial} \partial \log{h_0} \]
Now consider,
\[ \eta = \alpha - c_1(L, h_0) \]
is a real $(1,1)$-form and since $[\alpha] = [c_1(L, h_0)]$ also $\eta$ is $\d$-exact. Thus, by the $\partial \bar{\partial}$-lemma, we know,
\[ \eta = - \frac{i}{2 \pi} \partial \bar{\partial} f \]
for $f \in \A^{0, 0}_\R(X)$ i.e. $f$ is a real smooth function. Therefore,
\[ \alpha = \frac{i}{2 \pi} \bar{\partial} \partial \left[ f + \log{h_0} \right] = \frac{i}{2 \pi} \bar{\partial} \partial \log{e^f h_0} \]
Therefore, let $h = e^f h_0$ be annother Hermitian structure (since $f$ is real) then we see $c_1(L, h) = \alpha$.

\subsubsection{4.4.11}

Let $X$ be compact \kahler and $E$ a vector bundle with a Chern connection $\nabla$. If we let,
\[ \sum_{i = 0}^r \tilde{P}_i(B) = \tr{e^B} = \sum_{n = 0}^\infty \frac{1}{n!} \tr{B^n}  \]
so,
\[ \tilde{P}_k(B_1, \dots, B_k) = \frac{1}{k!} \tr{B_1 \cdots B_k} \] 
and then define,
\[ \ch_k(E,\nabla) := \tilde{P}_k \left( \frac{i}{2 \pi} F_\nabla \right) \in \A_\C^{2k}(M) \]
where $\tilde{P}_k$ acts on $\End{E}$-valued $2$-forms via,
\[ \tilde{P}_k(\alpha_1 \otimes \varphi_1, \dots, \alpha_k \otimes \varphi_k) = (\alpha_1 \wedge \cdots \wedge \alpha_k) \, \tilde{P}_k(\varphi_1, \dots, \varphi
_k) = (\alpha_1 \wedge \cdots \wedge \alpha_k) \, \frac{1}{k!} \tr{\varphi_1 \cdots \varphi_k} \]
This is the composition of $(\Omega_X^2)^{\otimes k} \to \Omega_X^{2k}$ via exterior product and $\End{E}^{\otimes k} \to \End{E}$ via composition and finally taking trace. We see that,
\[ \ch_k(E, \nabla) = \frac{1}{k!} \left( \frac{i}{2 \pi} \right)^k \tr{F_\nabla^{\otimes k}} \]
where $F_\nabla^{\otimes k}$ is the image under $(\Omega_X^2 \otimes \End{E})^{\otimes k} \to \Omega_X^{2k} \otimes \End{E}$. Now taking Dolbeault  cohomology classes via $\A^{k,k}_\C(\End{E}) \to H^k(X, \Omega^k \otimes \End{E})$,
\[ \ch_k(E) = \frac{1}{k!} \left( \frac{i}{2 \pi} \right)^k \tr{[F_\nabla]^{\otimes k}} \]
where $[F_\nabla]^{\otimes k}$ is the image under the map,
\[ H^1(X, \Omega^1_X \otimes \End{E}) \times \cdots \times H^1(X, \Omega^1_X \otimes \End{E}) \to H^{k}(X, \Omega^k \otimes \End{E}) \]
Furthermore $[F_\nabla] = A(E)$ so we get,
\[ \ch_k(E) = \frac{1}{k!} \left( \frac{i}{2 \pi} \right)^k \tr{A(E)^{\otimes k}} \]
as a class under the map $H^k(X, \Omega^k \otimes \End{E}) \xrightarrow{\mathrm{tr}} H^k(X, \Omega^k_X) \subset H^{2k}(X, \C)$.

\subsubsection{4.4.12}

Let $X$ be compact \kahler and $E$ a holomorphic vector bundle admitting a holomorphic connection. Then $A(E) = 0$ and therefore $c_k(E) = 0$. 

\section{Chapter 5}

\subsection{Section 5.1}

\subsubsection{5.1.1}

\subsection{Section 5.2}

\subsubsection{5.2.1}

\subsection{Section 5.3}

\subsubsection{5.3.1}

\section{Chapter 6}

\subsection{Section 6.1}

\subsection{6.1.1}

\subsection{6.1.2}

\subsection{6.1.3}

\end{document}



\documentclass[12pt]{article}
\usepackage[english]{babel}
\include{AstroCommands}

\begin{document}

\atitle{1}

\section*{Problem 1}

Consider an ideal blackbody star with specific intensity $B_\nu$ constant on its surface and independent of angle in the ourtwards facing hemisphere. Now we may compute,
\begin{align*}
J_\nu & = \frac{1}{4 \pi} \int B_\nu \d{\Omega} = \frac{1}{4 \pi} \int_0^1 B_\nu \d{\cos{\theta}} \int_0^{2 \pi} \d{\phi} = \tfrac{1}{2} B_\nu
\\
F_\nu & = \int B_\nu \cos{\theta} \d{\Omega} = \int_0^1 B_\nu \cos{\theta} \d{\cos{\theta}} \int_0^{2\pi} \d{\phi} = \pi B_\nu
\\
P_\nu & = \frac{1}{c} \int B_\nu \cos^2{\theta} \d{\Omega} = \frac{1}{c} \int_0^1 B_\nu \cos^2{\theta} \d{\cos{\theta}} \int_0^{2\pi} \d{\phi} = \tfrac{2 \pi}{3 c} B_\nu
\\
u_\nu & = \tfrac{4 \pi}{c} J_\nu = \tfrac{2 \pi}{c}  B_\nu 
\end{align*}

\section*{Problem 2}

Consider a radiation field of the form,
\[ I_\nu(\mu) = a_\nu + b_\nu \mu \]
where $\mu = \cos{\theta}$. 
Then we may compute,
\begin{align*}
J_\nu & = \frac{1}{4 \pi} \int B_\nu \d{\Omega} = \frac{1}{4 \pi} \int_{-1}^1 [a_\nu + b_\nu \mu ] \d{\mu} \int_0^{2 \pi} \d{\phi} = a_\nu
\\
F_\nu & = \int B_\nu \cos{\theta} \d{\Omega} = \int_{-1}^1 [a_\nu + b_\nu \mu ] \mu \d{\mu} \int_0^{2\pi} \d{\phi} = \tfrac{4}{3} \pi b_\nu
\\
P_\nu & = \frac{1}{c} \int B_\nu \cos^2{\theta} \d{\Omega} = \frac{1}{c} \int_{-1}^1 [a_\nu + b_\nu \mu ] \mu^2 \d{\mu} \int_0^{2\pi} \d{\phi} = \tfrac{4 \pi}{3c} a_\nu 
\\
u_\nu & = \tfrac{4 \pi}{c} J_\nu = \tfrac{4 \pi}{c}  a_\nu 
\end{align*}
Therefore this radiation field has the same energy density and pressure as an isotropic one however has nonzero flux unlike the purely isotropic case. 
\end{document}
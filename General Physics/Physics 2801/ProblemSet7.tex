\documentclass[11pt]{amsart}
\usepackage{geometry} % see geometry.pdf on how to lay out the page. There's lots.
\geometry{a4paper} % or letter or a5paper or ... etc
% \geometry{landscape} % rotated page geometry
\usepackage{amsmath}
\usepackage{graphicx}
\usepackage{breqn}

\setcounter{MaxMatrixCols}{10}

\flushbottom
\chardef\atcode=\catcode`\@
\makeatletter
\@addtoreset{figure}{section}
\@addtoreset{table}{section}
\renewcommand{\figurename}{Figure}
\renewcommand{\tablename}{Table}
\setcounter{topnumber}{3}               % orig: 2
\setcounter{totalnumber}{4}             % orig: 3
\renewcommand{\textfraction}{0}         
\renewcommand{\bottomfraction}{0.65}    
\renewcommand{\topfraction}{0.75}       
\renewcommand{\floatpagefraction}{0.75} 
\catcode`\@=\atcode 
\newcommand{\grad}{$^\circ$}
\newcommand{\gradm}{^\circ}
\newcommand{\bqn}{ \begin{eqnarray} }
\newcommand{\eqn}{ \end{eqnarray} }
\newcommand{\beq}{ \begin{equation} }
\newcommand{\eeq}{ \end{equation} }
\setlength{\baselineskip}{2.1ex}
\renewcommand{\baselinestretch}{1.06}
\setlength{\parskip}{1.5ex plus 0.8ex minus 0.6ex}
\setlength{\evensidemargin}{0.3cm}
\setlength{\oddsidemargin}{-0.3cm}
\setlength{\topmargin}{-1cm}
\setlength{\textwidth}{17 cm}
\setlength{\textheight}{26cm}
\newcommand{\mat}[1]{\mbox{$\underline{\underline{#1}}$}}
\newcommand{\etal}{\mbox{\sl et al.}}
\renewcommand{\refname}{}
\newcommand{\vol}[1]{{\bf{#1}}}
\newcommand{\dg}{$^\circ\;$}
\def\D{\displaystyle}
\newcommand{\lapprox}{\ensuremath{<\atop{\mbox{\raisebox{0.5ex}{$\sim$}}}}}
\parindent 0cm
\input{tcilatex}

% See the ``Article customise'' template for come common customisations

\title{Section II: Problem 3}
\author{Laura Havener laura.havener17@gmail.com}
\date{Nov } % delete this line to display the current date

%%% BEGIN DOCUMENT
\begin{document}


\maketitle

Problem 3. Hall Effect \\
In this problem we have a strip of copper with a width of 1.5 cm and a thickness of 0.10 cm. Assume some length l since it will hopefully cancel out in the final answer. A magnetic field of 1.2 T is applied perpindicular to the slab and a current of 5.0 A is running along the length of the slab. First let's find the magnetic force. \\
\begin{eqnarray*}
F_{m} &=& IlXB = IlB \mbox{ (l and B are perpindicular)} 
\end{eqnarray*} \\
The magnetic force induces an electric field along the width of the slab so we can set the electric force and magnetic force equal to each other to determine the electric field. \\
\begin{eqnarray*}
IlB &=& Eq \\
E &=& \frac{IlB}{q} 
\end{eqnarray*} \\
The electric field will create a voltage along the width of the slab which is the Hall Voltage that one would measure. \\
\begin{eqnarray*}
V_{h} &=& Ew = \frac{IBlw}{q} 
\end{eqnarray*} \\
Now the charge needs to be determined. Each copper atom contributes one free electron to the body of the material. Therefore, $q=eN$, where N is the number of copper atoms and e is the charge of an electron. \\
\begin{eqnarray*}
N &=& nV \mbox{ (n is the number density)} \\
n &=& \frac{\rho}{m} \\
m &=& \frac{m_{molar}}{N_{A}} \\
V &=& lwt \\
V_{h} &=& \frac{IBm}{\rho{}teN_{A}} = \frac{(5A)(1.2T)(63.5g)}{(8.95*10^{6}g/m^{3})(1*10^{-3}m)(1.6*10^{-19}C)(6*10^{23})} = 4.43*10^{-7} V 
\end{eqnarray*} 


\end{document}
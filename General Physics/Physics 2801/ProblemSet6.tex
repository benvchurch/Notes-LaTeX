\documentclass[11pt]{amsart}
\usepackage{geometry} % see geometry.pdf on how to lay out the page. There's lots.
\geometry{a4paper} % or letter or a5paper or ... etc
% \geometry{landscape} % rotated page geometry
\usepackage{amsmath}
\usepackage{graphicx}
\usepackage{breqn}

\setcounter{MaxMatrixCols}{10}

\flushbottom
\chardef\atcode=\catcode`\@
\makeatletter
\@addtoreset{figure}{section}
\@addtoreset{table}{section}
\renewcommand{\figurename}{Figure}
\renewcommand{\tablename}{Table}
\setcounter{topnumber}{3}               % orig: 2
\setcounter{totalnumber}{4}             % orig: 3
\renewcommand{\textfraction}{0}         
\renewcommand{\bottomfraction}{0.65}    
\renewcommand{\topfraction}{0.75}       
\renewcommand{\floatpagefraction}{0.75} 
\catcode`\@=\atcode 
\newcommand{\grad}{$^\circ$}
\newcommand{\gradm}{^\circ}
\newcommand{\bqn}{ \begin{eqnarray} }
\newcommand{\eqn}{ \end{eqnarray} }
\newcommand{\beq}{ \begin{equation} }
\newcommand{\eeq}{ \end{equation} }
\setlength{\baselineskip}{2.1ex}
\renewcommand{\baselinestretch}{1.06}
\setlength{\parskip}{1.5ex plus 0.8ex minus 0.6ex}
\setlength{\evensidemargin}{0.3cm}
\setlength{\oddsidemargin}{0.2cm}
\setlength{\topmargin}{-1cm}
\setlength{\textwidth}{16.5cm}
\setlength{\textheight}{26cm}
\newcommand{\mat}[1]{\mbox{$\underline{\underline{#1}}$}}
\newcommand{\etal}{\mbox{\sl et al.}}
\renewcommand{\refname}{}
\newcommand{\vol}[1]{{\bf{#1}}}
\newcommand{\dg}{$^\circ\;$}
\def\D{\displaystyle}
\newcommand{\lapprox}{\ensuremath{<\atop{\mbox{\raisebox{0.5ex}{$\sim$}}}}}
\parindent 0cm
\input{tcilatex}

% See the ``Article customise'' template for come common customisations

\title{Physics C2801 Fall 2013 Problem Set 6}
\author{Laura Havener}
\date{Oct 24} % delete this line to display the current date

%%% BEGIN DOCUMENT
\begin{document}


\maketitle

Problem 1. Kleppner and Kolenkow 6.9 \\ \\
We want to analyze the motion of a bar on 2 rollers with kinetic friction between the rollers and the bar. Start by looking at all the forces on the bar. You have the weight ($W=mg$) acting from the center of mass. There is also a normal force from each roller acting up on the bar ($N_{1}$ and $N_{2}$). Then finally the friction force acts towards the center from each roller on the ball ($f_{1}$ and $f_{2}$). In this problem there is forces as well as torque, but lets start with Newton's Laws for forces in the x and y direction. \\
\begin{eqnarray*}
\sum{F_{y}} &=& N_{1}+N_{2}-mg = m\ddot{y} = 0 \\
\sum{F_{y}} &=& f_{1}-f_{2} = \mu{}N_{1}-\mu{}N_{2} = m\ddot{x} 
\end{eqnarray*} \\
At this point, we have 2 equations and 3 unknowns so we know that we need another equation. Therefore, look at the equation of motion for the torques which is just summing all the torques acting on the object. Keep in mind the center of mass, which is the turning point, so the contribution from the weight is 0 for torque. Also, remember that for there to be torque the force has to act perpindicular to the pivot arm. The 2 frictional forces are completely parallel to the arm so they also have no contribution. The bar is not rotating so $\alpha$ will be 0. Set the positive direction to be clockwise. \\
\begin{eqnarray*}
\sum{\tau} &=& N_{1}(L/2+x(t))-N_{2}(L/2-x(t)) = I\alpha = 0 
\end{eqnarray*} \\
Now we can manipulate these equations to get an equation of motion for the bar in the x-direction. \\
\begin{eqnarray*}
N_{2} &=& \frac{L/2+x}{L/2-x}N_{1} \\
mg &=& N_{1}+\frac{L/2+x}{L/2-x}N_{1} \\
mg &=& \frac{N_{1}L}{L/2-x} \\
N_{1} &=& \frac{mg}{L}(L/2-x) \\
N_{2} &=& \frac{mg}{L}(L/2+x) \\
 m\ddot{x} &=& \mu(\frac{mg}{L}(L/2-x)-\frac{mg}{L}(L/2+x) \\
-2mg\mu/L &=& m\ddot{x} \\
\ddot{x} &=& -\frac{2g\mu}{L}x \\
\ddot{x} &=& -\omega^{2}x \\
\omega^{2}  &=&\frac{2g\mu}{L} 
\end{eqnarray*} \\
This is the equation for the harmonic oscillator, which we know has the solutions $x(t)=Acos(\omega{t})+Bsin(\omega{t})$, which initial conditions of $x(0)=x_{0}$ and $\dot{x}(0)=0$. \\
\begin{eqnarray*}
x_{0} &=& A \\
\dot{x}(0) &=& 0 = -\omega{B} \\
B &=& 0 \\
x(t) &=& x_{0}cos(\omega{t}) 
\end{eqnarray*} \\

Problem 2. Kleppner and Kolenkow 6.13 \\ \\
a.) There is an external force to pull the rope down, so work must be done on the rope making kinetic energy not conserved. There are no external torques though so angular momentum is conserved. \\
\begin{eqnarray*}
mv_{0}r &=& mv_{f}R \\
v_{f} &=& \frac{r}{R}v_{0} 
\end{eqnarray*} \\
b.) The torque is not equal to 0 so the angular momentum isn't conserved. There is no work done though so energy will be conserved. \\
\begin{eqnarray*}
\frac{1}{2}mv_{0}^{2} &=& \frac{1}{2}mv_{f}^{2} \\
v_{f} &=& v_{0} 
\end{eqnarray*} \\

Problem 3. Kleppner and Kolenkow 6.18 \\ \\
We want to find the period of a physical pendulum. The pendulum consits of a disk on the end of the rod. We want to start with the equation of motion of the system by looking at the torques. The only torque is gravity acting on the center on mass of the total system. Do this for small angles, so $sin(\theta)\approx\theta$.\\
\begin{eqnarray*}
I\ddot{\theta} &=& -(m+M)gr_{cm}sin(\theta) \approx -(m+M)gr_{cm}\theta \\
r_{cm} &=& \frac{ml/2+Mla}{m+M} \\
\ddot{\theta} &\approx& -\frac{gl}{I}(m/2+M)\theta \\
\omega &=& \sqrt{\frac{gl(m/2+M)}{I}} \\
T &=& 2\pi/\omega = 2\pi\sqrt{\frac{I}{gl(m/2+M)}} 
\end{eqnarray*} \\
This is true for both the caes discussed in the problem. The difference between the fixed and free rotating disk is in the moment of inertia. In the case where the disk is fixed, then the moment of inertia of the disk contributes to the rotation of the whole system about the pivot point. In the case where it is free to rotate it doesn't contribute because the disk doesn't have to rotate about the pivot point, it can just rotate freely seperate from the rod. Therefore, for this first case, the moment of intertia will be the sum of the moment of inertia for the rod (rod rotation about end) and the moment of inertia of the disk rotating about a pivot point a distance l away from the disk center.  Then the period is found from this moment of inertia. \\
\begin{eqnarray*}
I &=& I_{rod}+I_{disk} \\
I_{rod} &=& \frac{1}{3}ml^{2} \\
I_{disk} &=& \frac{1}{2}MR^{2}+Ml^{2} \mbox{ (parallel axis theorem)} \\
I &=& \frac{1}{2}MR^{2}+(m/3+M)l^{2}  \\
T &=& 2\pi\sqrt{\frac{\frac{1}{2}MR^{2}+(m/3+M)l^{2}}{gl(m/2+M)}} 
\end{eqnarray*} \\
Then for the second case, it has already been explained that the the moment of inertia of the disk doesn't matter since it is free to rotate. Therefore, we should just treat it is a point mass. The moment of inertial and period for this case are given below. \\
\begin{eqnarray*}
I &=& \frac{1}{3}ml^{2}+Ml^{2} \\
T &=& 2\pi\sqrt{\frac{\frac{1}{3}ml^{2}+Ml^{2}}{gl(m/2+M)}} 
\end{eqnarray*} \\

Problem 4. Kleppner and Kolenkow 6.20 \\ \\
To begin, let's think of things that are conserved in this problem. Energy is conserved as the rod falls. Angular momentum and linear momentum are not because there are external forces and torques. Therefore, lets start with energy conservation. Set the reference point to be 0 potential when the angle is at 90 degrees, thus there is no final potential energy. The initial kinetic energy is 0 because we start from rest.  \\
\begin{eqnarray*}
\frac{1}{2}mv_{f}^{2}+\frac{1}{2}I\omega_{f}^{2} &=& mgl\frac{l}{2}cos(60) \\
I &=& \frac{1}{3}ml^{2} \\
\frac{2}{3}mv_{f}^{2} &=& mgl\frac{1}{4} \\
v_{f}^{2} &=& \frac{3}{8}gl 
\end{eqnarray*} \\
We are looking for the force  that the pivot exerts on the hinge which will be equal and opposite to the force from the hinge on the pivot, which has a horizontal ($F_{h}$) and a vertical force ($F_{v}$). Let's look at the equations of motion in the $\hat{r}$ and $\hat{\theta}$ direction at the point where we want to evaluate the forces (angle at 90 degrees). \\
\begin{eqnarray*}
\sum{F_{r}} &=& -F_{h} = ma_{c} = mv_{f}^{2}/(l/2) \\
\sum{F_{\theta}} &=& F_{v}-mg = ma_{\theta} 
\end{eqnarray*} \\
We can immediately find the horizontal force because we have the final velocity from energy conservation and thus the centripetal acceleration. What we need is the acceleration in the $\hat{\theta}$ direction at the point where the rod is vertical. We can find this by looking at the kinematic equations in the y-direction. We know that at the point where the rod is vertical, the acceleration in the y-direction is equal to the acceleration in the $\hat{\theta}$ direction, which is what we need. \\
\begin{eqnarray*}
v_{f}^{2} &=& v_{0}^{2}+2a\Delta{y} \\
\Delta{y} &=&0-\frac{l}{2}cos(60) = -\frac{l}{4} \\
\frac{3}{8}gl &=& -2\frac{l}{4}a \\
a &=& -\frac{3}{4}g \\
F_{v}-mg &=& -\frac{3}{4}mg \\
-F_{v} &=& \frac{1}{4}mg \\
F_{h}&=& -\frac{2m}{l}\frac{3}{8}gl =-\frac{3}{4}mg 
\end{eqnarray*} \\
The pivot exterts a force that is directly equal and opposite of this force. Then we need to find the the magnitude and the direction of the force. \\
\begin{eqnarray*} 
F(pivot)_{h} &=& \frac{3}{4}mg \\
F(pivot)_{v} &=& -\frac{1}{4}mg \\
|F| &=& mg\sqrt{(9/16+1/16)} = \frac{\sqrt{10}}{4}mg \\
tan(\theta) &=& -\frac{1}{4}\frac{4}{3} = -\frac{1}{3} \\
\theta &=& -18 \mbox{ (from the horizonal)} 
\end{eqnarray*} \\
The other way to do this is to use the equations for  angular impulse to find the acceleration then proceed the same way as before. \\
\begin{eqnarray*} 
L_{f}-L_{i} &=& \int{\tau\,dt} = -mg(l/4)\Delta{t} = I\omega_{f} = \frac{1}{3}lmv_{f} \\
v_{f} &=& v_{i}+a\Delta{t} \\
\Delta{t} &=& \frac{v_{f}}{a} \\
\frac{1}{3}v_{f} &=&- (g/4)\frac{v_{f}}{a} \\
a &=& -\frac{3}{4}g
\end{eqnarray*} \\

Problem 5. Kleppner and Kolenkow 6.23 \\ \\
a. We want to relate the acceleration of the block and spool to the anglular acceleration of the tape. The total length of the of the tape will be L to start, but then will change as the tape unwinds by $R\theta$. \\
\begin{eqnarray*} 
x+X &=& L+R\theta 
\end{eqnarray*} \\
Then take the derivative twice to get the constraint equation for the accelerations. \\
\begin{eqnarray*} 
\ddot{x}+\ddot{X} &=& R\ddot{\theta} \\
a+A &=& R\alpha 
\end{eqnarray*} \\
Check the hint in the book: \\
\begin{eqnarray*}
A &=& 2a \\
a+2a &=& R\alpha \\
\alpha &=& \frac{3a}{R} 
\end{eqnarray*} \\
b. Now we want to solve for the accelations. This can be done by looking at the sum of the forces on the block, the sum of the forces on the spool, and the sum of the torques on the spool. \\
\begin{eqnarray*}
\sum{F_{m}} &=& mg-T =ma \\
\sum{F_{M}} &=& Mg-T = MA \\
\sum{\tau} &=& TR = I\alpha = \frac{1}{2}MR^{2}\alpha \\
\end{eqnarray*} \\
Now we need to solve these 3 equations along with the constraint equation for the accelerations. \\
\begin{eqnarray*}
T &=& m(g-a) = M(g-A) \\
m(g-a)R &=& \frac{1}{2}MR^{2}\alpha \\
a &=& g- \frac{M}{2m}R\alpha \\
M(g-A)R &=& \frac{1}{2}MR^{2}\alpha \\
A &=& g-\frac{1}{2}R\alpha \\
R\alpha &=&   g- \frac{M}{2m}R\alpha +g-\frac{1}{2}R\alpha \\
\alpha(R+\frac{M}{2m}R+\frac{1}{2}R) &=& 2g \\
\alpha &=& \frac{2g}{R(\frac{3}{2}+\frac{M}{2m})} = \frac{4g}{R(3+\frac{M}{m})} \\
a &=& g-\frac{M}{2m}R\frac{4g}{R(3+\frac{M}{m})} =g-\frac{2gM/m}{3+M/m} \\
a &=& \frac{g(3+M/m)-2gM/m}{3+M/m)} = \frac{g(3-M/m)}{3+M/m} \\
A &=& g-\frac{1}{2}R\frac{4g}{R(3+\frac{M}{m})} = \frac{g(3+M/m)-2g}{3+\frac{M}{m}} \\
A &=& \frac{g(1+M/m)}{3+M/m} 
\end{eqnarray*} \\

Problem 6. Kleppner and Kolenkow 6.27 \\ \\
We want to figure out what the largest value of the force exerted on the yo-yo can be before it starts to slip. The condition for rolling without of slipping is that $a=\alpha{R}$. Use the equation of motion for the torque on the yo-yo since the force and friction will exert torques on it. The yo-yo will move in the direction of the force. \\
\begin{eqnarray*}
\sum{\tau} &=& fR-Fb = I\ddot{\theta} = \frac{1}{2}mR^{2}\frac{a}{R} = \frac{1}{2}maR \\
\end{eqnarray*} \\
We can also look at the sum of the forces in the horizontal and vertical direction acting on the yo-yo. \\
\begin{eqnarray*}
\sum{F_{x}} &=& F-f= ma \\
\sum{F_{y}} &=& N=mg \\
N &=& mg \\
f &=& \mu{N} = \mu{mg} \\
\mu{mg}R-Fb &=& \frac{1}{2}mR(\frac{F}{m}-\mu{g}) \\
\frac{3}{2}\mu{mg}R &=& F(b+R/2) \\
F &=& \frac{3\mu{mgR}}{2b+R} \\
\end{eqnarray*} \\

Problem 7. Kleppner and Kolenkow 6.30 \\ \\
A bowling ball is launched with a velocity $v_{0}$ and begins sliding without rolling. We want to figure out what the velocity of the ball will be when the ball starts rolling instead of slipping. This can be done in a couple of ways. One way is by using the impulse equation for linear and angular momentum for the ball at $v_{0}$ to the point where it is rolling without slipping. This is because there is an external torque and force acting on the center of the ball while it is sliding that causes the angular and linear momentum not to be conserved about the center of mass. As the ball travels it begins to roll because of the external torque. Then eventually it will roll without slipping, which will be the case that $v=R\omega$. Keep in mind that the frictional force oposes the motion, but the frictional torque acts in the direction of rolling. \\
\begin{eqnarray*}
L_{f}-L_{i} &=& fR\Delta{t} \\
L_{i} &=& 0 \\
I\omega_{f} &=& fR\Delta{t} \\
p_{f}-p_{i} &=& -f\Delta{t} \\
 mv_{f}-mv_{0} &=& -f\Delta{t} \\
mv_{f}-mv_{0} &=& -I\omega_{f}/R \\
I &=& \frac{2}{5}mR^{2} \\
mv_{f}-mv_{0} &=& -\frac{2}{5}mRv_{f}/R \\
v_{f}-v_{0} &=& -\frac{2}{5}v_{f} \\
v_{f} &=& \frac{5}{7}v_{0} 
\end{eqnarray*} \\
Another way to do it is the find the work done by the frictional force and torque by using the kinematic equations for velocity and angular momentum to determine the critical time when the ball starts rolling and not slipping. Then use the work-kinetic energy theorem to find the final velocity.  \\
\begin{eqnarray*}
W &=& -\int{f\,dl+\tau\,d\theta} = \Delta{KE} \\
dl &=& vdt \\
d\theta &=& \omega{dt} \\
W &=& \int{fv\,dt}+\int{\tau\omega\,d\theta} = f\int{v\,dt}+fR\int{\omega\,d\theta} \\
v &=& v_{0}+at \\
\omega_{f} &=& \alpha{t} \\
\omega_{f} &=& v_{f}/R = \alpha{t} \\
t_{c} &=& v_{f}/R\alpha \\
ma &=& -f \\
I\alpha &=& fR \\
t_{c} &=& v_{f}/R\frac{I}{fR} = \frac{Iv_{f}}{fR^{2}} \\
W &=& -f(\int{(v_{0}-ft/m\,dt})+fR\frac{fR}{I}\int{t\,dt} \\
&=& -f(v_{0}t_{c} -\frac{ft_{c}^{2}}{2m})+\frac{f^{2}R^{2}}{2I}t_{c}^{2} \\
&=& -f(v_{0}\frac{Iv_{f}}{fR^{2}}+\frac{I^{2}fv_{f}^{2}}{2mf^{2}R^{4}})+\frac{I^{2}f^{2}R^{2}v_{f}^{2}}{2If^{2}R^{4}} \\
&=& -\frac{2}{5}mv_{f}^{2}+\frac{2}{5}v_{f}at_{c}+\frac{2mv_{f}^{2}}{25}+\frac{mv_{f}^{2}}{5} \\
&=& -\frac{1}{5}mv_{f}^{2}+\frac{2}{25}mv_{f}^{2}-\frac{4}{25}mv_{f}^{2}+\frac{mv_{f}^{2}}{5} \\
&=& -\frac{7}{25}mv_{f}^{2} =\Delta{KE} \\
&=& \frac{7}{10}mv_{f}^{2}-\frac{1}{2}mv_{0}^{2} \\
\frac{49}{50}v_{f}^{2} &=& \frac{1}{2}mv_{0}^{2} \\
v_{f} &=& \frac{5}{7}v_{0} \\
\end{eqnarray*} \\
Yet another way to do is to look at the motion about the ground instead of about the center of the ball. In this case the external torque is 0 and you can look at angular momentum conservation. \\
\begin{eqnarray*} 
L_{i} &=& L_{f} \\
mv_{0}R &=& (I_{ball}+mR^{2})\omega_{f} \\
mv_{0}R &=& (\frac{2}{5}mR^{2}+mR^{2})\frac{v_{f}}{R} \\
v_{0} &=& \frac{7}{5}v_{f} 
\end{eqnarray*} \\

Problem 8. Kleppner and Kolenkow 6.33 \\ \\
a. To find the angular velocity of the cone when the block reaches the bottom we can use conservation of angular momentum before and after the block slides down the cone. Before there is just angular momentum from the cone spinning because the block is located at the axis of rotation. Afterwards though, both have angular momentum. \\
\begin{eqnarray*} 
L_{i} &=& L_{f} \\
I_{0}\omega_{0} &=& I_{0}\omega_{f}+mv_{f} 
\end{eqnarray*} \\
Then since the ball is fixed on the cone it is rotating at angular velocity of $\omega_{f}=v_{f}/R$. \\
\begin{eqnarray*}
\omega_{f} &=& \frac{I_{0}\omega_{0}}{I_{0}+mR} 
\end{eqnarray*} \\
b. To find the speed of the block when it reaches the botton of the cone in inertial space (this means the direction along the track) we can use energy conservation of the system. Initially, the block only has gravitational potential energy and the cone has rotational kinetic energy. Then at the bottom, they both have only kinetic energy.  \\
\begin{eqnarray*}
PE_{i}+KE_{i} &=& PE_{f} +KE_{f} \\
mgh +\frac{1}{2}I_{0}\omega_{0}^{2} &=& \frac{1}{2}mv_{f}^{2} +\frac{1}{2}I_{0}\omega_{0}^{2} \\
v_{f} &=& \sqrt{2mgh+I_{0}(\omega_{0}^{2}-\omega_{f}^{2})} 
\end{eqnarray*} \\

\end{document}

\documentclass[11pt]{amsart}
\usepackage{geometry} % see geometry.pdf on how to lay out the page. There's lots.
\geometry{a4paper} % or letter or a5paper or ... etc
% \geometry{landscape} % rotated page geometry
\usepackage{amsmath}
\usepackage{graphicx}
\usepackage{breqn}

\setcounter{MaxMatrixCols}{10}

\flushbottom
\chardef\atcode=\catcode`\@
\makeatletter
\@addtoreset{figure}{section}
\@addtoreset{table}{section}
\renewcommand{\figurename}{Figure}
\renewcommand{\tablename}{Table}
\setcounter{topnumber}{3}               % orig: 2
\setcounter{totalnumber}{4}             % orig: 3
\renewcommand{\textfraction}{0}         
\renewcommand{\bottomfraction}{0.65}    
\renewcommand{\topfraction}{0.75}       
\renewcommand{\floatpagefraction}{0.75} 
\catcode`\@=\atcode 
\newcommand{\grad}{$^\circ$}
\newcommand{\gradm}{^\circ}
\newcommand{\bqn}{ \begin{eqnarray} }
\newcommand{\eqn}{ \end{eqnarray} }
\newcommand{\beq}{ \begin{equation} }
\newcommand{\eeq}{ \end{equation} }
\setlength{\baselineskip}{2.1ex}
\renewcommand{\baselinestretch}{1.06}
\setlength{\parskip}{1.5ex plus 0.8ex minus 0.6ex}
\setlength{\evensidemargin}{0.3cm}
\setlength{\oddsidemargin}{0.2cm}
\setlength{\topmargin}{-1cm}
\setlength{\textwidth}{16.5cm}
\setlength{\textheight}{26cm}
\newcommand{\mat}[1]{\mbox{$\underline{\underline{#1}}$}}
\newcommand{\etal}{\mbox{\sl et al.}}
\renewcommand{\refname}{}
\newcommand{\vol}[1]{{\bf{#1}}}
\newcommand{\dg}{$^\circ\;$}
\def\D{\displaystyle}
\newcommand{\lapprox}{\ensuremath{<\atop{\mbox{\raisebox{0.5ex}{$\sim$}}}}}
\parindent 0cm
\input{tcilatex}

% See the ``Article customise'' template for come common customisations

\title{Physics C2801 Fall 2013 Problem Set 3}
\author{Laura Havener}
\date{Sept 19} % delete this line to display the current date

%%% BEGIN DOCUMENT
\begin{document}


\maketitle

Problem 1. Kleppner and Kolenkow 2.11 \\ \\
a.) Force diagram. \\ \\
b.) To find the tensions sum up the forces in the x and y directions using Newton's laws. \\
\begin{eqnarray*}
\sum{F_{y}} &=& T_{up}cos(\theta)-T_{low}cos(\theta)-mg = ma_{y} = 0 \\
\sum{F_{x}} &=& -T_{up}sin(\theta)-T_{low}sin(\theta) = ma_{x} = -\frac{mv^{2}}{R} =-m\omega^{2}R\\
R &=& lsin(\theta) \\
m\omega^{2}l &=& T_{up}+T_{low} \\
T_{up} &=& -T_{low} +m\omega^{2}l \\
mg &=& (-T_{low} +m\omega^{2}l)cos(\theta) -T_{low}cos(\theta) \\
2T_{low}cos(\theta) &=& m(\omega^{2}lcos(\theta)-g) \\
T_{low} &=& m\frac{\omega^{2}l-\frac{g}{cos(\theta)}}{2} \\
T_{up} &=& - m\frac{\omega^{2}l-\frac{g}{cos(\theta)}}{2} +m\omega^{2}l \\
&=& m\frac{\omega^{2}l+\frac{g}{cos(\theta)}}{2} \\
cos(45) &=& \frac{\sqrt{2}}{2} \\
T_{low} &=& m\frac{\omega^{2}l-\frac{2g}{\sqrt{2}}}{2} \\
T_{up} &=& m\frac{\omega^{2}l+\frac{2g}{\sqrt{2}}}{2}
\end{eqnarray*} \\
Now try the hint in the problem: \\
\begin{eqnarray*}
l\omega^{2} &=& \sqrt{2}g \\
T_{up} &=&  m\frac{\sqrt{2}g+\frac{2g}{\sqrt{2}}}{2} = m\frac{\sqrt{2}g+\frac{2g\sqrt{2}}{\sqrt{2}\sqrt{2}}}{2} \\
&=& m\frac{\sqrt{2}g+g\sqrt{2}}{2} = mg\sqrt{2}
\end{eqnarray*} \\
Thus our solution makes sense. \\ \\
Probem 2. Kleppner and Kolenkow 2.15 \\ \\ 
a.) Force diagram. Make sure to draw the accelerations on the boxes as well as the forces. \\ \\
b.) We need to sum all the forces on each block. \\
\begin{eqnarray*}
f_{1} &=& \mu{m_{1}g} \\
f_{2} &=& \mu{m_{2}g} \\
\sum{F_{1}} &=&  -f_{1} + T =   -\mu{m_{1}g} +T = m_{1}a_{1} \\
\sum{F_{2}} &=&  f_{2} - T =   \mu{m_{2}g} -T = -m_{2}a_{2} \\
\sum{F_{3}} &=& -2T+m_{3}g = m_{3}a_{3} \\
l &=& x_{1} + x_{2} -2x_{3} \\
a_{3} &=& \frac{a_{1}+a_{2}}{2} \\
-2T+m_{3}g &=& \frac{m_{3}}{2}((T/m_{1}-\mu{g})+(\-mu{g} +T/m_{2})) \\
2T +\frac{Tm_{3}}{2m_{1}} +\frac{Tm_{3}}{2m_{2}} &=& m_{3}g(\mu+1) \\
Tm_{1}m_{2} +Tm_{3}m_{2}/4 +Tm_{3}m_{1}/4 &=& m_{3}m_{2}m_{1}g(\mu+1)/2 \\
T &=& \frac{\mu+1}{2/m_{3} + 1/2m_{1} + 1/2m_{2}} 
\end{eqnarray*} \\
Problem 3. Kleppner and Kolenkow 2.17 \\ \\
a.) The minimum value for the friction force happens just before it slides. Therefore, $\vec{f_{f}}=\mu\vec{N}$. Before it slides there is only static friction, thus the acceleration has to be 0 in both directions at the threshold. Therefore, we should solve Newton's laws in the x and y coordinates of the reference frame orientated with y being perpindicular to the ramp. \\
\begin{eqnarray*} 
\sum{F_{y}} &=& N-mgcos(\theta) = 0 \\
\sum{F_{x}} &=& -\mu{N}+mgsin(\theta) = 0 \\
\mu(mgcos(\theta))&=&mgsin(\theta) \\
\mu &=& tan(\theta) 
\end{eqnarray*}
b.) Now find the minimum acceleration of the wedge needed to keep the block from slipping. Now the block has a net acceleration. \\ 
\begin{eqnarray*} 
a_{x} &=& a_{xramp} + a_{xblock} \\
a_{xramp} &=& acos(\theta) \\
a_{y} &=& a_{yramp} + a_{yblock} \\
a_{yramp} &=& asin(\theta) 
\end{eqnarray*} \\
We still want the accelerations of the block to be 0 in the frame of the ramp. Therefore, Newton's equations look like the following. \\ 
\begin{eqnarray*}
N-mgcos(\theta) &=& masin(\theta) \\
-\mu{N}+mgsin(\theta) &=& macos(\theta) \\
N &=& masin(\theta)+mgcos(\theta) \\
-\mu(masin(\theta)+mgcos(\theta)) + mgsin(\theta) &=& macos(\theta) \\
a(cos(\theta)+\mu{sin(\theta)}) &=& g(-\mu{cos(\theta)}+sin(\theta)) \\
a &=& \frac{g(sin(\theta)-\mu{cos(\theta)})}{cos(\theta)+\mu{sin(\theta)}} 
\end{eqnarray*} \\
Check the hint. \\
\begin{eqnarray*}
\theta &=& \pi/4 \\
sin(\pi/4) &=& cos(\pi/4) = \sqrt{2}/2 \\
a &=& \frac{g(\sqrt{2}/2-\mu{\sqrt{2}/2})}{\sqrt{2}/2+\mu{\sqrt{2}/2}} = \frac{g(1-\mu)}{1+\mu} 
\end{eqnarray*} \\
Thus our solution makes sense. \\ \\
c.) This asks you to look for the maximum acceleration before sliding. This can be interpreted as the block moving up the ramp instead of down. Thus, friction force in the x direction changes sign. \\
\begin{eqnarray*} 
N-mgcos(\theta) &=& masin(\theta) \\
\mu{N}+mgsin(\theta) &=& macos(\theta) \\
N &=& masin(\theta)+mgcos(\theta) \\
\mu(masin(\theta)+mgcos(\theta)) + mgsin(\theta) &=& macos(\theta) \\
a(cos(\theta)-\mu{sin(\theta)}) &=& g(\mu{cos(\theta)}+sin(\theta)) \\
a &=& \frac{g(sin(\theta)+\mu{cos(\theta)})}{cos(\theta)-\mu{sin(\theta)}} \\
\theta &=& \pi/4 \\
a &=& \frac{g(\sqrt{2}/2+\mu{\sqrt{2}/2})}{\sqrt{2}/2-\mu{\sqrt{2}/2}} = \frac{g(1+\mu)}{1-\mu}
\end{eqnarray*} \\
Therefore, our solution makes sense with the hint in the problem. \\ \\
Problem 4. Kleppner and Kolenkow 2.19 \\ \\
This problem asks you to figure out what force is needed to keep the mass hanging from the pulley on the "Pedagogical Machine" from falling. Start by looking in the frame of mass 1 and analyze Newton's Laws on mass 3. If the mass is not rising or falling then its acceleration in the frame has to be 0. \\
\begin{eqnarray*}
\sum{F_{3}} &=& T-m_{3}g = 0 \\
T &=& m_{3}g
\end{eqnarray*} \\
Then you can analyze Newton's laws on mass 2 while still in the the frame of mass 1. \\ 
\begin{eqnarray*}
\sum{F_{2}} &=& T = m_{2}a_{2} \\
a_{2} &=& \frac{m_{3}}{m_{2}}g 
\end{eqnarray*} \\
Then if you move back to the frame where the mass 1 is accelerating you can see that if mass 3 is not rising or falling then mass 3 and 2 can not be accelerating with respect to mass 1. Thus $a_{2}=a_{3}=a$ and they are all accelerating horizontally with the same acceleration a. This accelerating can be found by looking at the sum of the forces in the x direction on the whole system. \\ 
\begin{eqnarray*}
\sum{F_{x}} &=& F = (m_{1}+m_{2}+m_{3})a \\
a &=& \frac{F}{m_{1}+m_{2}+m_{3}} \\
a &=& a_{2} = \frac{m_{3}}{m_{2}}g =\frac{F}{m_{1}+m_{2}+m_{3}} \\
F &=& \frac{m_{3}g}{m_{2}}(m_{1}+m_{2}+m_{3}) 
\end{eqnarray*} \\
To see if this solution makes sense, check the case of all the masses being equal which is given as a hint for the problem. \\
\begin{eqnarray*}
m &=& m_{1} = m_{2} = m_{3} \\
F &=& \frac{mg}{m}(m+m+m)  = 3mg 
\end{eqnarray*} \\
This is what the book gets for this case, thus our solution makes sense. \\ \\
Problem 5. Kleppner and Kolenkow 2.24 \\ \\
You have to look at this problem in the form of differentials. This means looking at some differential segment of the capstan and look at the forces on this point. This is at some location $Rd\theta$. There is a tension on one side T and one the other side T+dT. Then you have the static frictional force acting in the same direction of the smaller tension. You also have the normal force acting up on the rope from the capstan. Look at Newton's laws in the tangential and radial direction. \\
\begin{eqnarray*}
\sum{F_{r}} &=& -Tsin(d\theta/2)-Tsin(d\theta/2)-dTsin(d\theta/2) +N =0 \\
sin(d\theta) &\approx& d\theta/2 \\
N &=& 2Td\theta/2 +dTd\theta/2 \\
\sum{F_{\theta}} &=& (T+dT)cos(d\theta)-Tcos(d\theta) -f_{s} =0 \\
f_{s} &=& \mu{N} \\
\mu{N} &=& dTcos(d\theta) \\
(Td\theta +dTd\theta)\mu &=& dTcos(d\theta)
\end{eqnarray*} \\
Then term $dTd\theta$ is very small because it is of second order, while all the other terms are in first order, so drop it. \\ 
\begin{eqnarray*} 
Td\theta\mu &=& dTcos(d\theta) \\
cos(d\theta) &\approx& 1 \\
Td\theta\mu &=& dT \\
\frac{dT}{T} &=& \mu{d\theta} \\
\int_{T_{B}}^{T_{A}}\frac{1}{T'}\,dT' &=& \mu\int_{0}^{\theta}\,d\theta' \\
ln(T')_{T_{B}}^{T_{A}} &=& \mu\theta \\
ln(T_{A}/T_{B}) &=& \mu\theta \\
ln(T_{B}/T_{A}) &=& -\mu\theta \\
T_{B}/T_{A} &=& e^{-\mu\theta} \\
T_{B} &=& T_{A}e^{-\mu\theta} 
\end{eqnarray*} \\
This is what we were trying to show. \\ \\
Problem 6. Kleppner and Kolenkow 2.30 \\ \\
This problem asks you to figure out the acceleration of mass A immediately after the catch is released. To begin look at the sum of the forces acting on each mass immediately after release in the radial direction. \\
\begin{eqnarray*}
\sum{F_{A}} &=& T = m_{A}(\ddot{r_{A}}-\omega^{2}r_{A}) \\
\sum{F_{B}} &=& T = m_{B}(\ddot{r_{B}}-\omega^{2}r_{B}) \\
 m_{A}(\ddot{r_{A}}-\omega^{2}r_{A})  &=& m_{B}(\ddot{r_{B}}-\omega^{2}r_{B})
\end{eqnarray*} \\
Then we need to somehow related $r_{A}$ and $r_{B}$. This can be done since they are connected by a string of lenth l which will stay constant. \\
\begin{eqnarray*}
l &=& r_{A} +r_{B} \\
r_{B} &=& l-r_{A} \\
\ddot{r_{B}} &=& - \ddot{r_{A}} 
\end{eqnarray*} \\
Then substitute everything for mass B in terms of A and solve for the acceleration of mass A. \\
\begin{eqnarray*} 
 m_{A}(\ddot{r_{A}}-\omega^{2}r_{A})  &=& m_{B}(-\ddot{r_{A}}-\omega^{2}(l-r_{A}) \\
\ddot{r_{A}}(m_{A}+m_{B}) &=& \omega^{2}(m_{A}r_{A}-m_{B}l+m_{B}r_{A}) \\
\ddot{r_{A}} &=& \omega^{2}(r_{A} - \frac{m_{B}l}{m_{A}+m_{B}}) 
\end{eqnarray*} \\ 
Problem 7. Box on freely sliding ramp. \\ \\ 
a.) This part wants you to evaluate the the accelerations of the ramp and the block sliding on the ramp. The first step is similar to what is done in the book which is to find how the accelerations relate to each other. This is a constaint that remains throughout this entire problem, meaning it exists whether or not there is friction involved. If you look at the system, x will be the position of the block in the x direction, y will be the positive of the block in the y direction, and X will be the position ramp in the x direction, all with respect to a stationary origin as the picture in the book depicts. The positions are all related through the angle $\theta$. If you look at a small triange at the top with the sides x-X and h-y you can see this relation. \\
\begin{eqnarray*}
tan(\theta) &=& \frac{h-y}{x-X} \\
(x-X) &=& (h-y)cot(\theta) \\
\ddot{x}-\ddot{X} &=& -\ddot{y}cot(\theta) 
\end{eqnarray*} \\
To solve for the accelerations, the first step is to look at the sum of the forces in the x and y direction on the two different masses. Define right to be positve x and up to be positive y.\\ 
\begin{eqnarray*} 
\sum{F_{x1}} &=& Nsin(\theta) = m_{1}\ddot{x} \\
\sum{F_{y1}} &=& Ncos(\theta)-m_{1}g = m_{1}\ddot{y} \\
\sum{F_{x2}} &=& -Nsin(\theta) = m_{2}\ddot{X} \\
\sum{F_{y2}} &=& -Ncos(\theta)+N_{surface}-m_{2}g=0 
\end{eqnarray*} \\
Notice how the normal force N is an equal but opposite force between the ramp and the block so contributes in newton's law for both masses. The components of N were found from a geometric arguement through like triangles. The next step is to solve for the accelerations of the block and then use these to find the normal force from the equation with the acceleration of the ramp. \\
\begin{eqnarray*}
\ddot{x} &=& \frac{Nsin(\theta)}{m_{1}} \\
\ddot{y} &=& \frac{Ncos(\theta)}{m_{1}}-g \\
\ddot{X} &=& \ddot{x}+\ddot{y}cot(\theta) \\
&=& \frac{Nsin(\theta)}{m_{1}} + (\frac{Ncos(\theta)}{m_{1}}-g)\frac{cos(\theta)}{sin(\theta)} \\
&=& \frac{Nsin^{2}(\theta)+Ncos^{2}(\theta)}{m_{1}sin(\theta)}-gcot(\theta) \\
&=& \frac{N}{m_{1}sin(\theta)}-gcot(\theta) \\
-Nsin(\theta) &=& m_{2}(\frac{N}{m_{1}sin(\theta)}-gcot(\theta)) \\
N[\frac{m_{2}}{m_{1}sin(\theta)}+sin(\theta)] &=& m_{2}gcot(\theta) \\
N &=& \frac{m_{2}gcos(\theta)}{\frac{m_{2}}{m_{1}}+sin^{2}(\theta)} \\
&=& \frac{m_{1}m_{2}gcos(\theta)}{m_{2}+m_{1}sin^{2}(\theta)} \\
\ddot{x} &=& \frac{m_{2}gcos(\theta)sin(\theta)}{m_{2}+m_{1}sin^{2}(\theta)} \\
\ddot{y} &=& \frac{m_{2}gcos^{2}(\theta)}{m_{2}+m_{1}sin^{2}(\theta)} -g \\
&=& \frac{m_{2}gcos^{2}(\theta)-m_{2}g+m_{1}gsin^{2}(\theta)}{m_{2}+m_{1}sin^{2}(\theta)} \\
&=& \frac{(m_{1}-m_{2})gsin^{2}(\theta)}{m_{2}+m_{1}sin^{2}(\theta)} \\
\ddot{X} &=& \frac{m_{2}gcot(\theta)}{m_{2}+m_{1}sin^{2}(\theta)}-gcot(\theta) \\
&=& \frac{m_{2}gcot(\theta)-m_{2}gcot(\theta)-m_{1}gcos(\theta)}{m_{2}+m_{1}sin^{2}(\theta)} \\
&=& \frac{-m_{1}gcos(\theta)}{m_{2}+m_{1}sin^{2}(\theta)} 
\end{eqnarray*} \\
b.) The next part is to calculate the normal force from the surface on the ramp. This can be done by using the equation for sum of the forces in the y direction on the ramp. \\
\begin{eqnarray*}
0 &=& -Ncos(\theta)+N_{surface}-m_{2}g \\
N_{surface} &=& Ncos(\theta)+m_{2}g \\
N_{surface} &=& \frac{m_{1}m_{2}gcos^{2}(\theta)}{m_{2}+m_{1}sin^{2}(\theta)} + m_{2}g \\ 
&=& \frac{m_{1}m_{2}gcos^{2}(\theta)+m_{2}^{2}g +m_{1}m_{2}gsin^{2}(\theta)}{m_{2}+m_{1}sin^{2}(\theta)} \\
&=& \frac{m_{1}m_{2}g+m_{2}^{2}g}{m_{2}+m_{1}sin^{2}(\theta)} \\
&=& \frac{m_{2}g(m_{1}+m_{2})}{m_{2}+m_{1}sin^{2}(\theta)} 
\end{eqnarray*} \\
c.) Now there is friction between the box and the ramp but the coefficient of friction is large enough that the block doesn't slide on the ramp. This means that $\ddot{y}=0$ and $\ddot{x}=\ddot{X}$. The new equations for newton's laws are as follows.\\ 
\begin{eqnarray*}
\sum{F_{x1}} &=& Nsin(\theta) - f_{s}cos(\theta) = m_{1}\ddot{X} \\
\sum{F_{y1}} &=& Ncos(\theta)-m_{1}g+f_{s}sin(\theta) = 0 \\
\sum{F_{x2}} &=& -Nsin(\theta) +f_{s}cos(\theta) = m_{2}\ddot{X} \\
\sum{F_{y2}} &=& -Ncos(\theta)+N_{surface}-m_{2}g-f_{s}sin(\theta)=0 
\end{eqnarray*} \\
Now try to find the acceleration of the ramp by again solving for the normal force and frictional force. \\
\begin{eqnarray*} 
f_{s} &=& -Ncot(\theta)+m_{1}g/sin(\theta) \\
m_{1}\ddot{X} &=&  Nsin(\theta) -  (-Ncot(\theta)+m_{1}g/sin(\theta))cos(\theta)   \\
\frac{N}{sin(\theta)} &=& m_{1}\ddot{X} +m_{1}gcot(\theta) \\
N &=& m_{1}\ddot{X}sin(\theta) +m_{1}gcos(\theta) \\
 m_{2}\ddot{X} &=& -Nsin(\theta) +f_{s}cos(\theta) \\
 &=& -Nsin(\theta) + (-Ncot(\theta)+m_{1}g/sin(\theta))cos(\theta) \\
&=& -\frac{N}{sin(\theta)}+m_{1}gcot(\theta) \\
&=& -\frac{m_{1}\ddot{X}sin(\theta) +m_{1}gcos(\theta)}{sin(\theta)} +m_{1}gcot(\theta) \\
&=& -m_{1}\ddot{X} \\
\ddot{X} &=& \ddot{x} = 0 
\end{eqnarray*} \\
The student forget the introduction of the friction force that acts in both directions on both masses. \\ \\
d.) The point at which the box will start sliding down the ramp is when $f_{s} = \mu{N}$. Use the results from part c for the accelerations and the force equations to obtain this result. I chose to use the equation in the x direction on mass 1 but you could also find this from the equation in the x direction on mass 2. \\ 
\begin{eqnarray*}
\sum{F_{x1}} &=& Nsin(\theta) - \mu{N}cos(\theta) = m_{1}\ddot{X}  = 0 \\
\mu{N}cos(\theta) &=& Nsin(\theta) \\
\mu &=& tan(\theta) 
\end{eqnarray*} \\
This is no different from the conditions on $\mu$ for the stationary ramp, which we already found in Problem 3a. \\ \\
e.) Now we have kinetic friction but the mass is sliding down the ramp. Thus a new force is introduced and we should rewrite the equations of motions. \\
\begin{eqnarray*}
\sum{F_{x1}} &=& Nsin(\theta)-\mu{N}cos(\theta) = m_{1}\ddot{x} \\
\sum{F_{y1}} &=& Ncos(\theta)-m_{1}g +\mu{N}sin(\theta)= m_{1}\ddot{y} \\
\sum{F_{x2}} &=& -Nsin(\theta)+\mu{N}cos(\theta) = m_{2}\ddot{X} \\
\sum{F_{y2}} &=& -Ncos(\theta)+N_{surface}-m_{2}g -\mu{N}sin(\theta)=0 \\
\ddot{x}-\ddot{X} &=& -\ddot{y}cot(\theta)  
\end{eqnarray*} \\
Now follow the same procedure as part a and isolate the normal force in order to solve for the accelerations. \\
\begin{eqnarray*}
\ddot{x}  &=& \frac{N}{m_{1}}sin(\theta)-\frac{\mu{N}}{m_{1}}cos(\theta) \\
\ddot{y} &=&  \frac{N}{m_{1}}cos(\theta)-g +\frac{\mu{N}}{m_{1}}sin(\theta) \\
\ddot{X} &=& \ddot{x}+\ddot{y}cot(\theta) \\
&=& \frac{N}{m_{1}}sin(\theta)-\frac{\mu{N}}{m_{1}}cos(\theta)+(\frac{N}{m_{1}}cos(\theta)-g +\frac{\mu{N}}{m_{1}}sin(\theta))cot(\theta) \\
&=&  -\frac{N}{m_{2}}sin(\theta)+\frac{\mu{N}}{m_{2}}cos(\theta) \\
gcot(\theta) &=& N[\frac{1}{m_{1}}sin(\theta)-\frac{\mu}{m_{1}}cos(\theta)+\frac{1}{m_{1}}cos(\theta)cot(\theta)+\frac{\mu}{m_{1}}cos(\theta)+\frac{1}{m_{2}}sin(\theta)-\frac{\mu}{m_{2}}cos(\theta)]  \\
gcot(\theta) &=& N[\frac{1}{m_{1}sin(\theta)}+\frac{1}{m_{2}}sin(\theta)-\frac{\mu}{m_{2}}cos(\theta)] \\
gcos(\theta) &=& N[\frac{1}{m_{1}}+\frac{1}{m_{2}}sin^{2}(\theta)-\frac{\mu}{m_{2}}sin(\theta)cos(\theta)] \\
N &=& \frac{m_{1}m_{2}gcos(\theta)}{m_{2}+m_{1}sin^{2}(\theta)+\mu{sin(\theta)cos(\theta)}} \\
\ddot{x} &=& \frac{m_{2}gcos(\theta)(sin(\theta)-\mu{cos(\theta)})}{m_{2}+m_{1}sin^{2}(\theta)+\mu{sin(\theta)cos(\theta)}} \\
\ddot{y} &=& \frac{m_{2}gcos(\theta)(cos(\theta)+\mu{sin(\theta)})}{m_{2}+m_{1}sin^{2}(\theta)+\mu{sin(\theta)cos(\theta)}} -g \\
&=&\frac{m_{2}gcos(\theta)(cos(\theta)+\mu{sin(\theta)})-gm_{2}-gm_{1}sin^{2}(\theta)-g\mu{sin(\theta)cos(\theta)}}{m_{2}+m_{1}sin^{2}(\theta)+\mu{sin(\theta)cos(\theta)}} \\
&=& \frac{gsin^{2}(\theta)(m_{2}-m_{1})+\mu{g}cos(\theta)sin(\theta)(m_{2}-m_{1})}{m_{2}+m_{1}sin^{2}(\theta)+\mu{sin(\theta)cos(\theta)}} \\
\ddot{X} &=&- \frac{m_{1}gcos(\theta)(sin(\theta)+\mu{cos(\theta)})}{m_{2}+m_{1}sin^{2}(\theta)+\mu{sin(\theta)cos(\theta)} }
\end{eqnarray*} \\
f.) This last part asks you to consider the case where there is now friction between the ramp and the surface, but not between the block and the ramp. To do this we need to rewrite the equations of motion again, taking into account this new force. We are looking for the condition at which it will start sliding. This is the case where $F_{s}=\mu_{r}N$. This also means that the acceleration on the ramp in the x direction is 0. This also simplifies our constriant equation. \\
\begin{eqnarray*}
\sum{F_{x1}} &=& Nsin(\theta) = m_{1}\ddot{x} \\
\sum{F_{y1}} &=& Ncos(\theta)-m_{1}g = m_{1}\ddot{y} \\
\sum{F_{x2}} &=& -Nsin(\theta) +\mu_{r}N_{surface} = 0 \\
\sum{F_{y2}} &=& -Ncos(\theta)+N_{surface}-m_{2}g =0 \\
\ddot{x} &=& -\ddot{y}cot(\theta) 
\end{eqnarray*} \\
We then need to manipulate these equations to obtain a solution for the mass fraction in terms of the coefficient of friction. \\
\begin{eqnarray*} 
\frac{N}{m_{1}}sin(\theta) &=& (-\frac{N}{m_{1}}cos(\theta)+g)cot(\theta) \\
\frac{N}{m_{1}}[sin(\theta)+cos(\theta)cot(\theta)] &=& gcot(\theta) \\
\frac{N}{m_{1}sin(\theta)} &=& gcot(\theta) \\
N &=& m_{1}gcos(\theta) \\
m_{1}gcos(\theta)sin(\theta) &=& \mu_{r}N_{surface} \\
N_{surface} &=& \frac{m_{1}gcos(\theta)sin(\theta)}{\mu_{r}} \\
0 &=& -m_{1}gcos^{2}(\theta) +\frac{m_{1}gcos(\theta)sin(\theta)}{\mu_{r}} -m_{2}g \\
(m_{1}gcos^{2}(\theta)+m_{2}g)\mu_{r} &=& m_{1}gcos(\theta)sin(\theta) \\
m_{1}cos(\theta)(\mu_{r}cos(\theta)-sin(\theta)) &=& -m_{2}\mu_{r} \\
\frac{m_{1}}{m_{2}} &=& -\frac{\mu_{2}}{\mu_{r}cos^{2}(\theta)-sin(\theta)cos(\theta)} \\
\frac{m_{1}}{m_{2}} &=& \frac{\mu_{2}}{sin(\theta)cos(\theta)-\mu_{r}cos^{2}(\theta)} 
\end{eqnarray*} \\
Here we finally used the normal force on the ramp to find the accelerations. \\ \\
Problem 8. Stretchy pendulum \\ \\
a.) To get the equation of motion look at Newton's Laws on the mass. Let's set the cooridate system to be so that the 0 is at the top of the pendulum and that positive y is down. \\
\begin{eqnarray*} 
\sum{F} &=& -T+mg = m\ddot{y} \\
-k(y-y_{0})+mg &=& m\ddot{y} \\
y_{0} &=& L_{0} 
\end{eqnarray*} \\
Then to get the equilbium position, set the acceleration equal to 0. \\
\begin{eqnarray*}
k(y_{eq}-y_{0}) &=& mg \\
y_{eq} &=& \frac{mg}{k}+y_{0} 
\end{eqnarray*} \\
b.) Find a change of variables to get the harmonic oscillator equation. \\
\begin{eqnarray*}
-k(y-y_{0})+mg &=& m\ddot{y} \\
-ky +ky_{0}+mg &=& m\ddot{y} \\
-ky+ky_{eq} &=& m\ddot{y} \\
y' &=& y -y_{eq} \\
\Delta{y} &=& y_{eq} \\
\ddot{y}' &=& \ddot{y} \\
-ky' &=& m\ddot{y}' \\
\ddot{y}' &=& -\frac{k}{m}y' = -\omega^{2}y' \\
\omega &=& \sqrt{k/m} 
\end{eqnarray*} \\
c.) The place where the rope stops having the stretchy effect is at y greater than $y_{0}$. \\
\begin{eqnarray*}
y &>& y_{0} \\
y' &>& y_{0}-y_{eq} =y_{0}-\frac{mg}{k}-y_{0}  = -\frac{mg}{k} 
\end{eqnarray*} \\
d.) To get y(t) solve the equation from part b for values about the minimum y' value. Then solve for the motion less than this value for the equation of motion without the tension. \\
\begin{eqnarray*}
\ddot{y}' &=& -\omega^{2}y' \\
y'(t) &=& Acos(\omega{t})+Bsin(\omega{t}) \\
y'(0) &=& y_{0}-y_{eq} = -\frac{mg}{k} =A \\
\dot{y(0)}' &=& 0 = \omega(-B) \\
y'(t) &=& y(t)-y_{eq} = -\frac{mg}{k}cos(\omega{t}) \\
y(t) &=& y_{eq} -\frac{mg}{k}cos(\omega{t}) = y_{0} +\frac{mg}{k}(1-cos(\omega{t}) ) 
\end{eqnarray*} \\
e.) Now that the mass is swinging also, you have to take into account the contributions in $\theta$. Look at the solutions in the $\theta$ and r direction. \\
\begin{eqnarray*}
\sum{F_{r}} &=& T-mgcos(\theta) = m(\ddot{r}-r\dot{\theta}^{2}) \\
\sum{F_{r}} &=& -k(r-r_{0})+mgcos(\theta) = m(\ddot{r}-r\dot{\theta}^{2}) \\
r_{0} &=& y_{0} = L_{0} \\
\sum{F_{\theta}} &=& -mgsin(\theta) = m(r\ddot{\theta}+2\dot{r}\dot{\theta}) 
\end{eqnarray*} \\
These are differential equations that are not solvable so we must make approximations in order to solve them. The next few parts walks you through these approximations and then has you show why these approximations are valid. \\ \\
f.) The first step is to make the small angle approximation. \\
\begin{eqnarray*}
cos(\theta) &\approx& 1 \\
sin(\theta) &\approx& \theta \\
k(r-L_{0})-mg &=& m(\ddot{r}-r\dot{\theta}^{2}) \\
-mg\theta &=& m(r\ddot{\theta}+2\dot{r}\dot{\theta}) 
\end{eqnarray*} \\
Next a change of variables like we did early needs to be done to this equation. The problem hints that it will be the same as the one from before, but lets makes sure. \\
\begin{eqnarray*}
-k(r-L_{0})+mg &=& m(\ddot{r}-r\dot{\theta}^{2}) \\
-kr+kr_{0} +mg &=& m(\ddot{r}-r\dot{\theta}^{2}) \\
r' &=& r-r_{0}-mg/k \\
\Delta{r} &=& r_{0} +mg/k \\
-kr' &=& m\ddot{r}' - (r'-\Delta{r})\dot{\theta}^{2} = m\ddot{r}' - (r'-r_{0}-mg/k)\dot{\theta}^{2}  \\
\end{eqnarray*} \\
Now we need to drop the centripetal term to see that a harmonic oscillator equation arises similar to before. The centripetal term is the $r'\dot{\theta}^{2}$ term. We will show later why this is valid. \\
\begin{eqnarray*}
-kr' &=& m\ddot{r'} \\
\ddot{r}' &=& -\frac{k}{m}r' = -\omega^{2}r' \\
\omega &=& \sqrt{k/m} 
\end{eqnarray*} \\
g.) Now we want to look at the angular equation and make similar changes of variables and approximations. First do a change of varibles in r. \\
\begin{eqnarray*} 
-mg\theta &=& m((r'+r_{0}+mg/k)\ddot{\theta}+2\dot{r}'\dot{\theta}) 
\end{eqnarray*} \\
Now make an approximation that the angle and the radial displacement are small. This makes the changes in these values small. You want to keep the first order terms, but drop the second order terms in r', $\theta$, $\dot{r}'$, and $\dot{\theta}$. \\
\begin{eqnarray*}
-mg\theta &=& m(r_{0}+mg/k)\ddot{\theta} 
\end{eqnarray*} \\
The $r'\ddot{\theta}$ and $\dot{r}'\dot{\theta}$ terms were dropped because they are second order terms. This is now in the form of a simple harmonic oscillator in $\theta$. \\
\begin{eqnarray*}
\ddot{\theta} &=& -\frac{g}{r_{0}+mg/k}\theta = -\omega^{2}\theta \\
\omega &=& \sqrt{\frac{g}{r_{0}+mg/k}} 
\end{eqnarray*} \\
h.) This is where we will show that dropping terms is valid for small angle and small radial displacement approximations. Start with the solutions given in the problem and calculate what the centripetal term would be assuming these solutions. This should remind you of the techniques we used in the first problem set to show the the velocity equation can be approximated to first order by showing that the second order term is small compared to the first term when you assume the first order time. \\ 
\begin{eqnarray*}
r' &=& Acos(\omega_{r}t-\phi_{r}) \\
\theta &=& Bcos(\omega_{\theta}t-\phi_{\theta}) \\
m(r'-\Delta{r})\dot{\theta}^{2} &=& m( Acos(\omega_{r}t-\phi_{r})+\Delta{r})B^{2}\omega_{\theta}^{2}sin^{2}(\omega_{\theta}t-\phi_{\theta}) \\
&=& B^{2}\omega_{\theta}^{2}(mAcos(\omega_{r}t-\phi_{r}) + \Delta{r})(1-cos^{2}(\omega_{\theta}t-\phi_{\theta}) \\
\ddot{r}' &=& -\omega_{r}^{2}r' \\
\ddot{r} &=& -A\omega_{r}^{2}cos(\omega_{r}t-\phi_{r}) \\
\omega^{2}r' &=&\omega_{r}^{2}Acos(\omega_{r}t-\phi_{r}) 
\end{eqnarray*} \\
Then Brian only wants you to compare the cos and cos squared terms in the $\theta$ equation to one of the terms in the harmonic equation (harmonic terms are of the same order so you only have to use one of them). \\
\begin{eqnarray*} 
\frac{m\omega_{\theta}^{2}B^{2}(A+L_{0} +mg/k)}{Am\omega_{r}^{2}} << 1 \\
\frac{\frac{g}{L_{0}+mg/k}B^{2}(A+L_{0} +mg/k)}{Ak/m} &<<& 1 \\
\frac{B^{2}}{A}(\frac{Amg}{kL_{0}+mg}+mg/k) &<<& 1 
\end{eqnarray*}
i.) Now do the same for the terms dropped in the $\theta$ equation. \\
\begin{eqnarray*}
r'\ddot{\theta} &=& -AB\omega_{\theta}^{2}cos(\omega_{r}t-\phi_{r})cos(\omega_{\theta}t-\phi_{\theta}) \\
2\dot{r}'\dot{\theta} &=& 2AB\omega_{r}\omega_{\theta}sin(\omega_{r}t-\phi_{r})sin(\omega_{\theta}t-\phi_{\theta}) \\
\ddot{\theta} &=& -B\omega_{\theta}^{2}\omega_{\theta}t-\phi_{\theta}) \\
\omega_{\theta}^{2}\theta &=& -B\omega_{\theta}^{2}\omega_{\theta}t-\phi_{\theta}) \\
\frac{2AB\omega_{r}\omega_{\theta}-AB\omega_{\theta}^{2}}{B\omega_{\theta}^{2}} &<<& 1 \\
\frac{2A\omega_{r}\omega_{\theta}-A\omega_{\theta}^{2}}{\omega_{\theta}^{2}} &<<& 1 \\
\end{eqnarray*} \\
Problem 9. Whirling rope with mass at the end \\ \\
For this problem think about the fact the tension on the mass at the end gives you an idea of the value at L. Do the sum of the forces on a differential mass at some point r on the rope. Then interate from r to L along the rope. \\ \\
\begin{eqnarray*}
\lambda &=& m/L \\
dm &=& \lambda{dr} \\
\sum{F_{m}} &=& T(L) = m\omega^{2}L \\
\sum{dF} &=& T(r+dr)-T(r) =dT =dm\omega^{2}r =\lambda{dr}\omega^{2}r \\
\int_{T(r)}^{T(L)}\,dT' &=& \lambda\omega^{2}\int_{r}^{L}r'\,dr' \\
T(L)-T(r) &=&  \lambda\omega^{2}(1/2)(r')^{2}|_{r}^{L}  \\
m\omega^{2}L -T(r) &=& (1/2)\lambda\omega^{2}[L^{2}-r^{2}] \\
T(r) &=&m\omega^{2}L+\lambda\omega^{2}[r^{2}-L^{2}] 
\end{eqnarray*} \\
Then check to see if the tension at L is what you expect. \\
\begin{eqnarray*}
T(L) &=& m\omega^{2}L 
\end{eqnarray*} \\


\end{document}

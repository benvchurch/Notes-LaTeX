\documentclass[11pt]{amsart}
\usepackage{geometry} % see geometry.pdf on how to lay out the page. There's lots.
\geometry{a4paper} % or letter or a5paper or ... etc
% \geometry{landscape} % rotated page geometry
\usepackage{amsmath}
\usepackage{graphicx}
\usepackage{breqn}

\setcounter{MaxMatrixCols}{10}

\flushbottom
\chardef\atcode=\catcode`\@
\makeatletter
\@addtoreset{figure}{section}
\@addtoreset{table}{section}
\renewcommand{\figurename}{Figure}
\renewcommand{\tablename}{Table}
\setcounter{topnumber}{3}               % orig: 2
\setcounter{totalnumber}{4}             % orig: 3
\renewcommand{\textfraction}{0}         
\renewcommand{\bottomfraction}{0.65}    
\renewcommand{\topfraction}{0.75}       
\renewcommand{\floatpagefraction}{0.75} 
\catcode`\@=\atcode 
\newcommand{\grad}{$^\circ$}
\newcommand{\gradm}{^\circ}
\newcommand{\bqn}{ \begin{eqnarray} }
\newcommand{\eqn}{ \end{eqnarray} }
\newcommand{\beq}{ \begin{equation} }
\newcommand{\eeq}{ \end{equation} }
\setlength{\baselineskip}{2.1ex}
\renewcommand{\baselinestretch}{1.06}
\setlength{\parskip}{1.5ex plus 0.8ex minus 0.6ex}
\setlength{\evensidemargin}{0.3cm}
\setlength{\oddsidemargin}{-0.3cm}
\setlength{\topmargin}{-1cm}
\setlength{\textwidth}{17 cm}
\setlength{\textheight}{26cm}
\newcommand{\mat}[1]{\mbox{$\underline{\underline{#1}}$}}
\newcommand{\etal}{\mbox{\sl et al.}}
\renewcommand{\refname}{}
\newcommand{\vol}[1]{{\bf{#1}}}
\newcommand{\dg}{$^\circ\;$}
\def\D{\displaystyle}
\newcommand{\lapprox}{\ensuremath{<\atop{\mbox{\raisebox{0.5ex}{$\sim$}}}}}
\parindent 0cm
\input{tcilatex}

% See the ``Article customise'' template for come common customisations

\title{Physics C2801 Fall 2013 Problem Set 8}
\author{Laura Havener}
\date{Nov 20} % delete this line to display the current date

%%% BEGIN DOCUMENT
\begin{document}


\maketitle

Problem 1. $\vec{A}X(\vec{B}X\vec{C})$ \\ 
a. We want to show that: \\
\begin{eqnarray*}
\vec{A} &=& (\vec{A}\cdot{\hat{n}})\hat{n}+(\hat{n}X\vec{A})X\hat{n} 
\end{eqnarray*} 
We can break A up into 2 components along $\hat{n}$ and along a unit vector $\hat{m}$ that is perpindicular $\hat{n}$ but in the plane of $\vec{A}$ and $\hat{n}$. This would look like: $\vec{A} = A_{n}\hat{n}+A_{m}\hat{m}$. Then the first term on the right hand side of the equation defined above is just vector that represents the component of $\vec{A}$ in the direction of $\hat{n}$, $A_{n}\hat{n}$. To determine the second term, look at the cross product in parenthesis. This is just a vector pointing perpindular to the plane of $\vec{A}$ and $\hat{n}$. Then when this vector is crossed again with $\hat{n}$ we get the vector $\hat{m}$. Therefore, the second term is just the component of $\vec{A}$ along $\hat{m}$, $A_{m}\hat{m}$, which is what we expected. \\
b. We want to prove the triple cross product identity by writing the cross products out in cartisean coordinates. We simplify this greatly by just looking at the first cross product in the $\hat{z}$ direction then cyclically permutating to get the remaining components. \\
\begin{eqnarray*}
(\vec{B}X\vec{C})_{z} &=& (B_{x}C_{y}-B_{y}C_{x})\hat{z} 
\end{eqnarray*} \\
Then we can take the cross product with the vector A: \\
\begin{eqnarray*}
\vec{A}X(B_{x}C_{y}-B_{y}C_{x})\hat{z} &=& A_{y}(B_{x}C_{y}-B_{y}C_{x})\hat{x}-A_{x}(B_{x}C_{y}-B_{y}C_{x})\hat{y} \\
&=& A_{y}C_{y}B_{x}\hat{x}-A_{y}B_{y}C_{x}\hat{x}-A_{x}B_{x}C_{y}\hat{y}+A_{x}C_{x}B_{y}\hat{y} 
\end{eqnarray*} \\
Then we can cyclically permute to find all the other terms of the cross product. \\
\begin{eqnarray*} 
\vec{A}X(B_{y}C_{z}-B_{z}C_{y})\hat{x} &=& A_{z}C_{z}B_{y}\hat{y}-A_{z}B_{z}C_{y}\hat{y}-A_{y}B_{y}C_{z}\hat{z}+A_{y}C_{y}B_{z}\hat{z} \\
\vec{A}X(B_{z}C_{x}-B_{x}C_{z})\hat{y} &=& A_{x}C_{x}B_{z}\hat{z}-A_{x}B_{x}C_{z}\hat{z}-A_{z}B_{z}C_{x}\hat{x}+A_{z}C_{z}B_{x}\hat{x} 
\end{eqnarray*} \\
Then start by look at the $\hat{z}$ components only. \\
\begin{eqnarray*}
\vec{A}X(\vec{B}X\vec{C})_{z} &=& (A_{x}C_{x}+A_{y}C_{y})B_{z}\hat{z}-(A_{x}B_{x}+A_{y}B_{y})C_{z}\hat{z} 
\end{eqnarray*} \\
Then lets add and subtract $A_{z}B_{z}C_{z}$. \\
\begin{eqnarray*}
\vec{A}X(\vec{B}X\vec{C})_{z} &=& (A_{x}C_{x}+A_{y}C_{y}+A_{z}C_{z})B_{z}\hat{z}-(A_{x}B_{x}+A_{y}B_{y}+A_{z}B_{z})C_{z}\hat{z} \\
&=& (\vec{A}\cdot{\vec{C}})B_{z}\hat{z}-(\vec{A}\cdot{\vec{B}})C_{z}\hat{z}
\end{eqnarray*} \\
Then cyclically permute this to get the $\hat{x}$ and $\hat{y}$ components. \\
\begin{eqnarray*} 
\vec{A}X(\vec{B}X\vec{C})_{x} &=& (\vec{A}\cdot{\vec{C}})B_{x}\hat{x}-(\vec{A}\cdot{\vec{B}})C_{x}\hat{x} \\
\vec{A}X(\vec{B}X\vec{C})_{y} &=& (\vec{A}\cdot{\vec{C}})B_{y}\hat{y}-(\vec{A}\cdot{\vec{B}})C_{y}\hat{y} 
\end{eqnarray*} \\
Then we can add the three terms together to get the final expression for the triple cross product. \\
\begin{eqnarray*}
\vec{A}X(\vec{B}X\vec{C}) &=& (\vec{A}\cdot{\vec{C}})\vec{B}-(\vec{A}\cdot{\vec{B}})\vec{C} 
\end{eqnarray*} \\
c. Use part b to show part a. \\
\begin{eqnarray*}
\vec{A} &=& (\vec{A}\cdot{\hat{n}})\hat{n}+(\vec{n}X\vec{A})X\vec{n} \\
&=& (\vec{A}\cdot{\hat{n}})\hat{n}+(\vec{n}\cdot{\vec{n}})\vec{A}-(\vec{n}\cdot{\vec{A}})\hat{n} \\
&=& (\vec{n}\cdot{\vec{n}})\vec{A} = (1)\vec{A} \
\end{eqnarray*} \\
d. Now we want to consider a special case and argue it geometrically. \\
\begin{eqnarray*}
\vec{A}X(\vec{B}X\vec{A}) &=& (\vec{A}\cdot{\vec{A}})\vec{B}-(\vec{A}\cdot{\vec{B}})\vec{A} \\
&=& A^{2}B\hat{B}-A^{2}Bcos(\theta)\hat{A} \\
&=& A^{2}Bsin(\theta)\hat{C}  
\end{eqnarray*} \\
$\hat{C}$ is peridicular to $\hat{A}$ and in the plane of $\vec{A}$ and $\vec{B}$. We need to show that those 2 things are equal. We can define B has having a component in the $\hat{C}$ and $\hat{A}$ direction since these vectors are perpindicular. Vectors can be written in a basis of 2 perpindicular vectors. \\
\begin{eqnarray*}
A^{2}B\hat{B} &=& A^{2}B(cos(\theta)\hat{A}+sin(\theta)\hat{C}) \\
\vec{A}X(\vec{B}X\vec{A}) &=& A^{2}B(cos(\theta)\hat{A}+sin(\theta)\hat{C})-A^{2}Bcos(\theta)\hat{A} \\
&=& A^{2}Bsin(\theta)\hat{C} 
\end{eqnarray*} \\
e. Now we want to argue generally that the 3 vector cross product indentity makes sense geometrically. First, consider $\vec{B}X\vec{C}$. This will be a vector perpindicular to  the place of $\hat{A}$ and $\hat{C}$ (call it $\hat{D}$). Then when this is crossed with $\hat{A}$, the resultant will have to be in the plane of $\hat{B}$ and $\hat{C}$. Thus it makes sense for the triple cross product to be written in the bases of vectors B and C. \\
\begin{eqnarray*}
\vec{A}X(\vec{B}\cdot{\vec{C}}) &=& C_{1}\vec{B}+C_{2}\vec{C} 
\end{eqnarray*} \\
To determine the coefficients, first think of the general triple cross product. \\
\begin{eqnarray*}
\vec{A}X(\vec{B}X\vec{C}) &=& ABC(\hat{A}X(\hat{B}X\hat{C}) 
\end{eqnarray*} \\
This implies that the magnitude of the constants must be proportional to the magnitues of the 3 vectors. Now we need to determine the coefficients and the direction. Let's take the case from part d. \\
\begin{eqnarray*}
\vec{C} &=& \vec{A} \\
\vec{A}X(\vec{B}X\vec{A}) &=& C_{1}\vec{B}+C_{2}\vec{A} \\
&=& (\vec{A}\cdot{\vec{A}})\vec{B}-(\vec{A}\cdot{\vec{B}})\vec{A} \\
C_{1} &=& \vec{A}\cdot{\vec{A}} \\
C_{2} &=& -\vec{A}\cdot{\vec{B}} 
\end{eqnarray*} \\
Then we can generalize this back to to having 3 different vectors and we obtain the coefficients desired from the identity. \\

Problem 2. Kleppner and Kolenkow 8.2 \\ \\
a. You can use the work-kinetic energy theorem to determine the instantenous angular velocity of the door. The work is done by the "torque" from the acceleration on the car in the frame of the door. The door can be treated like a swinging rod from its end. \\
\begin{eqnarray*}
W &=& \int_{\pi/2}^{0}\tau{}\,d\theta \\
\tau &=& mA(w/2)cos(\theta) \\
W &=& \frac{1}{2}mAw(-sin(\theta))|_{\pi/2}^{0} \\
&=& \frac{1}{2}mAw \\
&=& \frac{1}{2}I\omega^{2} \\
I &=& \frac{1}{3}mw^{2} \\
\omega &=& \sqrt{\frac{3A}{w}} 
\end{eqnarray*} \\
b. To find the horizontal force on the door look at the forces in the frame of the door when it is at 90 degrees. Here you can treat the acceleration as one of the forces. \\
\begin{eqnarray*}
\sum{F_{r}} &=& F-F_{A} = m\omega^{2}(w/2) \\
F_{A} &=& mA \\
F &=& mA+\frac{3}{2}mA = \frac{5}{2}mA  
\end{eqnarray*} \\ 

Problem 3. Kleppner and Kolenkow 8.5	 \\ \\
To begin write out the equation for a gyroscope. Then the we can take the acceleration of the vehicle as a torque acting on the gyroscope in the frame of the gyroscope. \\
\begin{eqnarray*}
\tau &=& \Omega{}\omega_{s}I_{s} = mal 
\end{eqnarray*} \\
 To find the final velocity, we can use the kinematic equations for the linear and angular velocity of the gyroscope. Then we will want to look at the final velocity at a particular time so the time will be the same in both equations of motion. \\
\begin{eqnarray*}
v &=& v_{0}+at = at \\
\Omega &=& \theta{t} \\
\theta{t}I_{s}\omega_{s} &=& mvtl \\
v &=& \frac{\theta\omega_{s}I_{s}}{ml} 
\end{eqnarray*} \\

Problem 4. Kleppner and Kolenkow 8.6 \\ \\
a. To look at the motion of the top when the elevator is at rest we can just analyze the equation of motion for the torque and angular momentum. The torque will be the gravitation force along the axis of rotation in the body frame of gyroscope.\\
\begin{eqnarray*} 
\tau &=& \frac{dL}{dt} = \Omega{X}\vec{L} = \Omega{I_{0}}\omega_{s} \\
\tau &=& mglcos(\phi) \\
\Omega &=& \frac{mglcos(\phi)}{I_{0}\omega_{s}} 
\end{eqnarray*} \\
b. For the accelerating elevator, we can look at equations of motion to find the new effective force that causes the torque on the gyroscope. \\
\begin{eqnarray*} 
2mg &=& mg-F \\
F &=& -mg \\
\tau &=& -mglcos(\phi) \\
\Omega &=& -\frac{mglcos(\phi)}{I_{0}\omega_{s}} 
\end{eqnarray*} \\

Problem 5. Kleppner and Kolenkow 8.10 \\ \\
We can use the equation for acceleration relative to rotating coordinates. \\
\begin{eqnarray*}
\vec{a}_{in} &=& \vec{a}_{rot}+2\vec{\Omega}X\vec{v}_{rot}+\vec{\Omega}X(\vec{\Omega}X\vec{r}) 
\end{eqnarray*} \\
There is no rotation velocity, the rotational acceleration is what the problem defines as $\vec{g}_{0}$, and $\Omega$ is the rotation of the earth. If you do this problem in cartisean coordinates with the lattitude ($\lambda$) representing the angle from the pole ($\hat{z}$) to the x-y plane and the angle ($\theta$) as the angle in the x-y plane, you can write the quantities as: \\
\begin{eqnarray*}
\vec{g}_{0} &=& g_{0}(cos(\lambda)cos(\theta)\hat{x}+cos(\lambda)sin(\theta)\hat{y}+sin(\lambda)\hat{z}) \\
\vec{r} &=& R_{e}((cos(\lambda)cos(\theta)\hat{x}+cos(\lambda)sin(\theta)\hat{y}+sin(\lambda)\hat{z}) \\
\Omega &=& \Omega_{e}\hat{z} 
\end{eqnarray*} \\
Do the centrifugal term first. From problem 1, we can write the triple cross product in a simpler form. \\
\begin{eqnarray*} 
\vec{\Omega}X(\vec{\Omega}X\vec{r}) &=& (\vec{\Omega}\cdot{\vec{r}})\vec{\Omega}-(\vec{\Omega}\cdot{\vec{\Omega}})\vec{r} \\
&=& (\Omega_{e}R_{e}sin(\lambda))\Omega_{e}\hat{z}-(\Omega_{e}^{2})\vec{r} 
\end{eqnarray*} \\
Then the problem wants us the find the magnitude of the effective gravity. Therefore, we have to square and add the x, y, and z components of the full acceleration. \\
\begin{eqnarray*} 
\vec{g} &=& (g_{0}-\Omega_{e}^{2}R_{e})cos(\lambda)cos(\theta)\hat{x}+(g_{0}-\Omega_{e}^{2}R_{e})cos(\lambda)sin(\theta)\hat{y} +(g_{0}+\Omega_{e}^{2}R_{e}-\Omega_{e}^{2}R_{e})sin(\lambda)\hat{z} \\
g &=& (g_{0}^{2}sin^{2}(\lambda)+g_{0}^{2}cos^{2}(\lambda)cos^{2}(\theta)+g_{0}^{2}cos^{2}(\lambda)sin^{2}(\theta) +\Omega_{e}^{4}R_{e}^{2}cos^{2}(\lambda)cos^{2}(\theta)+\Omega_{e}^{4}R_{e}^{2}cos^{2}(\lambda)sin^{2}(\theta)\\  
& & -2g_{0}\Omega_{e}^{2}R_{e}cos^{2}(\lambda)cos^{2}(\theta)-2g_{0}\Omega_{e}^{2}R_{e}cos^{2}(\lambda)sin^{2}(\theta))^{1/2} \\
&=& \sqrt{g_{0}^{2}+\Omega_{e}^{4}R_{e}^{2}cos^{2}(\lambda)-2g_{0}\Omega_{e}^{2}R_{e}cos^{2}(\lambda)} \\
x &=& \frac{R_{e}\Omega_{e}^{2}}{g_{0}} \\
g &=& g_{0}\sqrt{1+x^{2}cos^{2}(\lambda)-2xcos^{2}(\lambda)} 
\end{eqnarray*} \\
Problem 6. Kleppner and Kolenkow 8.12 \\ \\
This problem can be analzed by looking at the equation for the forces on the pendulum along the motion of the pendulum ($\hat{\theta}$), but with a new force from the rotation of the table along with the force from gravity. Assume small oscillations.\\
\begin{eqnarray*}
\sum{F} &=& -mgsin(\theta)+m\Omega^{2}lsin(\theta) = ma = ml\ddot{\theta} \\
sin(\theta) &\approx& \theta \\
l\ddot{\theta} &=& -(g-\Omega^{2}l)\theta \\
\ddot{\theta} &=& -(g/l-\Omega^{2})\theta \\
\end{eqnarray*} \\
This is the equation for a simple harmonic oscillator thus we can pull out the frequency of oscillation for the pendulum. \\
\begin{eqnarray*}
\omega &=& \sqrt{g/l-\Omega^{2}} 
\end{eqnarray*} \\
To determine if this makes sense, take a look at the limits. In limit where the table is not rotating ($\Omega=0$), $\omega=\sqrt{g/l}$, which is the frequency of oscillation that you would expect for a pendulum. \\

\end{document}
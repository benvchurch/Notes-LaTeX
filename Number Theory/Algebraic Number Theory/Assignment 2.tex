\documentclass[12pt]{extarticle}
\usepackage[utf8]{inputenc}
\usepackage[english]{babel}
\usepackage[a4paper, total={7in, 9.5in}]{geometry}
 
\usepackage{amsthm, amssymb, amsmath, centernot}
\usepackage{mathtools}
\DeclarePairedDelimiter{\floor}{\lfloor}{\rfloor}

\newcommand{\notimplies}{%
  \mathrel{{\ooalign{\hidewidth$\not\phantom{=}$\hidewidth\cr$\implies$}}}}
 
\renewcommand\qedsymbol{$\square$}
\newcommand{\cont}{$\boxtimes$}
\newcommand{\divides}{\mid}
\newcommand{\ndivides}{\centernot \mid}
\newcommand{\Z}{\mathbb{Z}}
\newcommand{\N}{\mathbb{N}}
\newcommand{\C}{\mathbb{C}}
\newcommand{\Zplus}{\mathbb{Z}^{+}}
\newcommand{\Primes}{\mathbb{P}}
\newcommand{\ball}[2]{B_{#1} \! \left(#2 \right)}
\newcommand{\Q}{\mathbb{Q}}
\newcommand{\R}{\mathbb{R}}
\newcommand{\Rplus}{\mathbb{R}^+}
\newcommand{\invI}[2]{#1^{-1} \left( #2 \right)}
\newcommand{\End}[1]{\text{End}\left( A \right)}
\newcommand{\legsym}[2]{\left(\frac{#1}{#2} \right)}
\renewcommand{\mod}[3]{\: #1 \equiv #2 \: \mathrm{mod} \: #3 \:}
\newcommand{\nmod}[3]{\: #1 \centernot \equiv #2 \: mod \: #3 \:}
\newcommand{\ndiv}{\hspace{-4pt}\not \divides \hspace{2pt}}
\newcommand{\finfield}[1]{\mathbb{F}_{#1}}
\newcommand{\finunits}[1]{\mathbb{F}_{#1}^{\times}}
\newcommand{\ord}[1]{\mathrm{ord}\! \left(#1 \right)}
\newcommand{\quadfield}[1]{\Q \small(\sqrt{#1} \small)}
\newcommand{\vspan}[1]{\mathrm{span}\! \left\{#1 \right\}}
\newcommand{\galgroup}[1]{Gal \small(#1 \small)}
\newcommand{\ints}[1]{\mathcal{O}_{#1}}
\newcommand{\sm}{\! \setminus \!}
\newcommand{\norm}[3]{\mathrm{N}^{#1}_{#2}\left(#3\right)}
\newcommand{\qnorm}[2]{\mathrm{N}^{#1}_{\Q}\left(#2\right)}
\newcommand{\quadint}[3]{#1 + #2 \sqrt{#3}}
\newcommand{\pideal}{\mathfrak{p}}
\newcommand{\inorm}[1]{\mathrm{N}(#1)}
\newcommand{\tr}[1]{\mathrm{Tr} \! \left(#1\right)}
\renewcommand{\Im}[1]{\mathrm{Im}\left( #1 \right)}

\newcommand{\atitle}[1]{\title{% 
	\large \textbf{Mathematics W4043 Algebraic Number Theory
	\\ Assignment \# #1} \vspace{-2ex}}
\author{Benjamin Church \\ \textit{Worked With Matthew Lerner-Brecher} }
\maketitle}

 
\newtheorem{theorem}{Theorem}[section]
\newtheorem{lemma}[theorem]{Lemma}
\newtheorem{proposition}[theorem]{Proposition}
\newtheorem{corollary}[theorem]{Corollary}

\begin{document}
\atitle{2}
 
\begin{enumerate}
\item Let $\ints{K}$ be the ring of integers of a number field $K$. Then $M \subset K$ is a fractional ideal of $\ints{K}$ if $M$ is a $\ints{K}$-module of finite type. 
\begin{enumerate}
\item Let $M$ be a fractional ideal of $K$ then since $M$ has finite type, $M = m_1 \ints{K} + \dots + m_n \ints{K}$. Futhermore, by Lemma \ref{multintoints} for any $\alpha \in K$ there exists $z \in \Z$ s.t. $\alpha z \in \ints{K}$. Therefore, for each $1 \le i \le n$ take $z_i \in \Z$ s.t. $z_i m_i \in \ints{K}$. Define $r = \mathrm{lcm}(z_1, \dots, z_n)$. \\ \\ 

For any $m \in M$, by the finite type condition, $m = o_1 m_1 + \cdots o_n m_n$ with $o_i \in \ints{K}$. But $z_i \divides r$ so $r = k_i z_i$ for $k_i \in \Z$ so $rm_i = k_i (m_i z_i) \in \ints{K}$. Thus, $rm = o_1 r m_1 + \dots + o_n r m_1 = o_1 k_1 (m_1 z_1) + \dots o_n k_n (m_n z_n) \in \ints{K}$. Therefore, $\forall m \in M : rm \in \ints{K}$.

\item Let $M$ and $M'$ be fractional ideals of $\ints{K}$ then define \[M \cdot M' = \{o_1 m_1 m_1' + \dots + o_n m_n m_n' \mid m_i \in M \text{ and } m_i' \in M' \text{ and } o_i \in \ints{K} \}\]
Now $M$ and $M'$ are $\ints{K}$-modules of finite type so for any $m_i \in M$ write: \\ $m_i = d_{i1} w_{1} + \dots + d_{ik} w_{k}$ and $m_i' = d_{i1}' w_{1}' + \dots + d_{ik'}' w_{k'}'$ for $d_{ij}, d_{ij}' \in \ints{K}$ Therefore, \[o_1 m_1 m_1' + \dots + o_n m_n m_n' = \sum_{l = 1, i = 1, j = 1}^{n, k, k'} o_l d_{ij} d_{ij}' w_i w_j'\] 
Thus, $\{w_i w_j'\}$ generates $M \cdot M'$ so $M \cdot M'$ has finite type.  

\item Let $M$ be a fractional ideal of $\ints{K}$ then define: \[M^{-1} = \{ \alpha \in K \mid \forall m \in M : \alpha m \in \ints{K} \} \]

If $\alpha, \beta \in M^{-1}$ then $\forall m \in M : \alpha m, \beta m \in \ints{K}$ thus, $(\alpha + \beta)m = \alpha m + \beta m \in \ints{K}$. Also, if $o \in \ints{K}$ then $o \alpha m = (\alpha m) o \in \ints{K}$ because $\alpha m, o \in \ints{K}$. Thus, $M^{-1}$ is an $\ints{K}$-module. \\ \\
Now $M$ has finite type so $M = m_1 \ints{K} + \dots + m_n \ints{K}$. If $\alpha \in M^{-1}$ then $\alpha m_i \in \ints{K}$ so $\alpha \in \frac{1}{m_i} \ints{K}$ therefore, $\alpha \in \bigcap\limits_{i = 1}^n \frac{1}{m_i} \ints{K}$. \\ \\ Conversely, if $\alpha \in \bigcap\limits_{i = 1}^n \frac{1}{m_i} \ints{K}$ then for each $i$, $\alpha \in \frac{1}{m_i} \ints{K}$ so $\alpha m_i \in \ints{K}$ therefore, $\alpha (m_1 o_1 + \dots + m_n o_n) = (\alpha m_1)o_1 + \dots + (\alpha m_n) o_n \in \ints{K}$ so $\alpha \in M^{-1}$ and thus, \[M^{-1} = \bigcap\limits_{i = 1}^n \frac{1}{m_i} \ints{K} \subset \frac{1}{m_1} \ints{K}\] However, $\frac{1}{m_1} \ints{K}$ is an $\ints{k}$-module of manifestly finite type therefore, $M^{-1} \subset \frac{1}{m_1} \ints{K}$ is an $\ints{k}$-submodule which has finite type because $\ints{K}$ is Noetherian. 

\end{enumerate}
\item Let $\{\mathfrak{p}_i \mid i \in \N\}$ a sequence of distinct prime ideals of $\ints{K}$. Then take $I = \bigcap\limits_{i = 1}^\infty \mathfrak{p}_i$. Now since $\ints{K}$ is Dedekind, its ideals have prime factorizaion. In particular, $I = \prod\limits_{i = 1}^{k} \mathfrak{q}_i$ so take $\mathfrak{p}_{r+1}$ distinct from every $\mathfrak{q}_i$. I claim that $I + \mathfrak{p}_{r+1} = \ints{K}$. In that case, $I \mathfrak{p}_{r+1} = I \cap \mathfrak{p}_{r+1} = I$. Therefore, if $I \neq \{0\}$ then $\mathfrak{p}_{r+1} = \ints{K}$ which is a contradiction. Therefore, $I = \{0\}$.   \\ \\
To prove the claim, we show that for prime ideal $\mathfrak{p}$ distinct from all $\mathfrak{q}_i$ that $\prod\limits_{i = 1}^{n} \mathfrak{q}_i + \mathfrak{p} = \ints{K}$. Proceed by induction, for $k = 1$, $\mathfrak{q_1} \subset \mathfrak{q}_1 + \mathfrak{p}$ but $\mathfrak{q}_1$ is maximal and since the ideals are distinct, $\mathfrak{q}_1 \subsetneq \mathfrak{q}_1 + \mathfrak{p}$ therefore, $\mathfrak{q}_1 + \mathfrak{p} = \ints{K}$. \\ \\
Now suppose that $\prod\limits_{i = 1}^{k} \mathfrak{q}_i + \mathfrak{p} = \ints{K}$ then $\left( \prod\limits_{i = 1}^{k} \mathfrak{q}_i \right) \mathfrak{q}_{k+1} + \mathfrak{p} \mathfrak{q}_{k+1} = \mathfrak{q}_{k+1} $. Thus, \\
$\prod\limits_{i = 1}^{k+1} \mathfrak{q}_i + (\mathfrak{p} \mathfrak{q}_{k+1} + \mathfrak{p}) = \mathfrak{q}_{k+1} + \mathfrak{p}$ but $\mathfrak{q}_{k+1} + \mathfrak{p} = \ints{K}$ because both are maximal and also $\mathfrak{p} \mathfrak{q}_{k+1} + \mathfrak{p} = \mathfrak{p}$ because $\mathfrak{p} \mathfrak{q}_{k+1} \subset \mathfrak{p}$. Therefore, $\prod\limits_{i = 1}^{k+1} \mathfrak{q}_i + \mathfrak{p} = \ints{K}$ so the claim holds by induction. 

\item 
Let $R$ be an integral domain with fraction field $K$. And for a multiplicative subset $S$ let \[S^{-1} R = \left\{ \frac{r}{s} \: \big| \: r \in R \text{ and } s \in S \right\} \]
For any ideal $I \subset R$, \[S^{-1} I = \left\{ \frac{r}{s} \: \big| \: r \in I \text{ and } s \in S \right\} \] is an ideal of $S^{-1}R$. This holds because if $\frac{r_1}{s_1}, \frac{r_2}{s_2} \in S^{-1}I$ then $\frac{r_1}{s_1} + \frac{r_2}{s_2} = \frac{r_1 s_2 + r_2 s_1}{s_1 s_2} \in s^{-1}I$ and $\frac{r_1}{s_1} + \frac{r_2}{s_2} = \frac{r_1 r_2}{s_1 s_2} \in S^{-1} I$ and for $\frac{r}{s} \in S^{-1}R$ and $\frac{a}{s'} \in S^{-1}I$, then $\frac{r}{s} \frac{a}{s'} = \frac{ra}{ss'} \in S^{-1}I$ all by absorption of $I$ and multiplictive property of $S$.
\begin{enumerate}
\item Let $S = R \sm \{0\}$ then \[S^{-1} R = \left\{ \frac{p}{q} \: \big| \: p \in R \text{ and } q \in R \sm \{0\} \right\} = K\] by definition. 

\item Let $R$ be a Dedekind domain. Then by part $(c)$, the map $I \mapsto S^{-1} I$ is a surjection. If $J_1 \subset J_2 \subset \cdots $ is an increasing chain of ideals of $S^{-1}R$ then $J_i = S^{-1}I_i$. Suppose that $I_i \supset I_{i+1}$, then $S^{-1} I_i \supset S^{-1}I_{i+1}$ also if $J_i \subsetneq J_{i+1}$ then $I_i \subsetneq I_{i+1}$. Therefore, $I_1 \subset I_2 \subset \cdots$ is an increasing chain of ideals of $R$. Since $R$ is Noetherian, the chain of $I_i$ terminates i.e. after some $n$, $I_n = I_{n+1} = \cdots$ so $I_n \supset I_{n+1} \supset \cdots$ and therefore, $J_n \supset J_{n+1} \supset \cdots$. Thus, the chain of $J_i$ also terminates at $n$ so $S^{-1} R$ is Noetherian. 
\\ \\
Suppose that $\alpha$ is integral over $S^{-1}R$. Then, for some monic polynomial $Q \in S^{-1}R[x]$, $Q(\alpha) = \alpha^n + c_{n-1} \alpha^{n-1} + \cdots + c_0 = 0$. But each $c_i \in S^{-1}R$ so $c_i = \frac{r_i}{s_i}$ for $r_i \in R$ and $s_i \in S$. Multiply through by $s^n = (s_{n-1} s_{n-2} \dots s_0)^n$, \[Q(\alpha)s^n = (s \alpha)^n + r_{n-1} (s_{n-2} \dots s_0)(s \alpha)^{n-1} + \cdots + s^{n-1}( s_{n-1} s_{n-2} \cdots s_1) r_0 = 0\]
Thus, $s\alpha$ is integral over $R$. However, $R$ is Dedekind and thus integrally closed so $s \alpha \in R$. Since $s \alpha \in R$ and $s \in S$ then $\frac{s \alpha}{s} = \alpha \in S^{-1}R$ so $S^{-1}R$ is integrally closed. 
\\ \\
Let $J \subset S^{-1} R$ be a non-zero prime ideal of $S^{-1}R$. By the bijection derived in part $(c)$, $J = S^{-1}I$ where $I$ is a non-zero prime ideal which is disjoint with $S$. Since $I$ is a non-zero prime ideal of $R$ and $R$ is Dedekind, then $I$ is maximal. Suppose that $J \subsetneq L \subset S^{-1}R$. Then $L = S^{-1} F$ for an ideal $F$. Then $I \subsetneq F$ so $F = R$ and thus $L = S^{-1}F = S^{-1}R$ so $J$ is maximal. Thus, $S^{-1}R$ is Dedekind.   

\item Let $D$ be the map from ideals of $R$ to ideals of $S^{-1}R$ given by $D : I \mapsto S^{-1} I$. Now if $J \subset S^{-1}R$ is an ideal then consider $R \cap J \subset R$. This is an ideal of $R$ because if $x,y \in R \cap J$ then $xy \in R$ and $xy \in J$ so $xy \in R \cap J$ and for $r \in R$, $r = \frac{r}{1} \in S^{-1} R$ so $rx \in J$ so $rx \in R \cap J$. \\ \\
Take $x \in D(R \cap J)$ then $x = \frac{r}{s}$ with $r \in J$ and since $\frac{1}{s} \in S^{-1}R$, by absorption, $\frac{r}{s} = x \in J$. Take $\frac{r}{s} \in J$ with $r \in R$ then $r = s \frac{r}{s} \in J$ by absorption so $r \in R \cap J$ thus $\frac{r}{s} \in D(R \cap J)$. Therefore, $D(R \cap J) = J$ so $D$ is surjective.  \\ \\

Restrict $D$ to the set of prime ideals of $R$ which do not intersect $S$. Let $P$ be a prime ideal of $R$ and $P \cap S = \emptyset$. Take $\frac{r_1}{s_1} \frac{r_2}{s_2} = \frac{r}{s} \in S^{-1} P$ for $r_1, r_2 \in P$. Then $r_1 r_2 s = s_1 s_2 r \in P$. $P$ is prime so either $r_1 \in P$ or $r_2 s \in P$. If $r_2 s \in P$ then $r_2 \in P$ because $s \notin P$. Therefore, $r_1 \in P$ or $r_2 \in P$ so $\frac{r_1}{s_1} \in S^{-1}P$ or $\frac{r_2}{s_2} \in S^{-1} P$ and therefore $S^{-1} P$ is prime. Thus, $\Im{D}$ is contained within the set of prime ideals of $S^{-1} R$. \\ \\

Let $P$ and $Q$ be prime ideals of $R$ s.t. $P \cap S = Q \cap S = \emptyset$. Then suppose that $D(P) = D(Q)$ i.e. $S^{-1}P = S^{-1}Q$. Then $\frac{p}{s_1} = \frac{q}{s_2}$ for any $p \in P$ and $q \in Q$. Thus, $s_2 p = s_1 q$ so $s_2 p \in Q$ and $s_1 q \in P$ by absorption. The ideals are prime so $p \in Q$ and $q \in P$ since $s_2 \notin Q$ and $s_1 \notin P$. Therefore, $P \subset Q$ and $P \supset Q$ so $P = Q$. Therefore, $D$ is injective. \\ \\

Let $J \in S^{-1}R$ be prime then take $xy \in R \cap J$ with $x,y \in R$. Now $xy \in J$ so $x \in J$ or $y \in J$. Therefore, since both $x,y \in R$ then $x \in R \cap J$ or $y \in R \cap J$ so $R \cap J$ is a prime ideal in $R$. Suppose that $\exists s \in S \cap (R \cap J)$ then $s \in J$ so by absorption, $\frac{1}{s} s \in J$ since $\frac{1}{s} \in S^{-1} R$ thus $1 \in J$ so $J = S^{-1}R$ which contradicts $J$ being a prime ideal. Thus, $(R \cap J) \cap S  = \emptyset$ so $D$ is surjective in the set of prime ideals of $S^{-1}R$. \\ \\
Therefore, $D$ is a bijection from the set of prime ideals of $R$ which are disjoint with $S$ and the prime ideals of $S^{-1}R$. 

\item Let $R$ be a Dedekind domain and $\mathfrak{p}$ be a prime ideal of $R$. Define $S_\mathfrak{p} = R \sm \mathfrak{p}$ and $R_\mathfrak{p} = S_\mathfrak{p}^{-1}R$. If $s, s' \in S_p$ then if $ss' \in \mathfrak{p}$ then either $s \in \mathfrak{p}$ or $s' \in \mathfrak{p}$ because $\mathfrak{p}$ is a prime ideal. However, $s, s' \in S_p$ so neither are in $\mathfrak{p}$. Thus, $ss' \notin \mathfrak{p}$ so $ss' \in S_p$. Also, $1 \notin \mathfrak{p}$ because a prime ideal cannot be the entire ring thus $1 \in S_p$. Now there is a bijection between the prime ideals of $R$ which do not intersect with $S_\mathfrak{p}$ and the prime ideals of $R_\mathfrak{p}$. If some ideal $I \subset R$ is a non-zero prime ideal and $I \cap S_\mathfrak{p} = I \cap (R \sm \mathfrak{p}) = \emptyset$ then $I \subset \mathfrak{p}$ but $R$ is a Dedekind domain so every non-zero prime ideal is maximal so $I = \mathfrak{p}$. Thus $\mathfrak{p}$ is the unique non-zero prime ideal of $R$ that is disjoint with $S_\mathfrak{p}$. Using the bijection, $S_p^{-1} \mathfrak{p}$ is the unique prime ideal of $R_\mathfrak{p}$. Furthermore, because $R$ is Dedekind so is $R_\mathfrak{p} = S_\mathfrak{p}^{-1} R$. Thus, $R_\mathfrak{p}$ is a discrete valuation ring. \\ \\

$R_\mathfrak{p}$ is a Noetherian ring  and $R_\mathfrak{p}$ is a $R_\mathfrak{p}$-module of finite type. Therefore, since $\mathfrak{p}$ is an ideal of $R_\mathfrak{p}$ then $\mathfrak{p}$ is an $R_\mathfrak{p}$-submodule and thus has finite type. Let $ \mathfrak{p} = c_1 R_\mathfrak{p} + \dots + c_n R_\mathfrak{p}$. Then $(c_i)$ is an ideal of $R_\mathfrak{p}$ which is a Dedekind domain so it has a prime factorization. Since there is only one prime ideal, $(c_1) = \mathfrak{p}^{k_i}$. Take $c$ to be the $c_i$ with the least $k_i$ then $(c_i) = \mathfrak{p}^{k_i} \subset \mathfrak{p}^{k_c} = (c)$ so $c_i \in (c)$. Therefore, $c_i = r c$ so $\mathfrak{p} = c R_\mathfrak{p} = (c)$. \\ \\
For any $a \in R_\mathfrak{p}$, the ideal $(a)$ has a prime factorization because $R_\mathfrak{p}$ is a Dedekind domain. Thus, $(a) = \mathfrak{p}^k = (c)^k = (c^k)$. Thus, $a = r c^k$ and $c^r = s a$ thus $a = (rs) a$. Since $R_\mathfrak{p}$ is a domain, $rs = 1$ so $r$ is a unit. 

\end{enumerate}

\end{enumerate}

\section*{Lemmas}

\begin{lemma} \label{multintoints}
If $\ints{K}$ is the ring of algebraic integers of a number field $K/\Q$ and $\alpha \in K$ then $\exists z \in \Z$ s.t. $z \alpha \in \ints{K}$. 
\end{lemma}
\begin{proof}
Since $K/\Q$ is a finite field extension, $[K : \Q] = n$ so for any $\alpha \in K$, $\{1, \alpha, \alpha^2, \dots, \alpha^n\}$ is a dependent set. Therefore, $\exists c_i \in \Q :  Q(\alpha) = \alpha^n + c_{n-1} \alpha^{n-1} + \cdots + c_0 = 0$. For each $c_i$ there are integers $p_i, q_i \in \Z$ s.t. $c_i = \frac{p_i}{q_i}$. Multiply by $z = \text{lcm}(q_{n-1}, \dots, q_0)$ and let $k_i q_i = z$ so \[z^n Q(\alpha) = (z \alpha)^n + p_{n-1} k_{n-1} (z \alpha)^{n-1} + \dots + p_0 k_0 z^{n-1} = 0\]
So $z \alpha \in \ints{K}$.     
\end{proof}

\end{document}
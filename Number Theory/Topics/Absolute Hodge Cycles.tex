\documentclass[12pt]{article}
\usepackage{import}
\import{../}{NumberTheoryCommands}


\begin{document}

\title{Deligne's Theory of Absolute Hodge Cycles}
\maketitle

\section{Sept. 26 CM Abelian Varities}

\begin{prop}
Let $F$ be a number field and $c$ a field automorphism of $F$ with $c^2 = 1$. Then the following are equivalent,
\begin{enumerate}
\item $\forall \tau : F \embed \CC$ then $\tau \circ c = \bar{\tau}$
\item $\tr_{\F/\Q}{a c(a)} > 0$ for all $a \in F^\times$
\item $F^{+} = F^{\tau}$ is a totally real field and either $F = F^+$ or $F$ is totally imaginary.
\end{enumerate}
Then $(F, c)$ is a CM pair and for each $F$ there is at most one $c$ making $(F, c)$ a CM pair and if it exists then $F$ is CM. 
\end{prop}

\begin{example}
$\Q, \Q(i), \Q(\sqrt{\sqrt{2} - 2})$ are all CM. 
\end{example}

Let $L$ be a field of characteristic $0$. 

\begin{defn}
An abelian variety $A/L$ is a smooth, projective, geometrically connected, abelian group scheme $A/L$. 
\end{defn}

\begin{rmk}
Given $A/L$ we have the invariants,
\begin{enumerate}
\item The Lie algebra $\Lie(A) = T_e A$ is an $L$-vectorspace and we get,
\[ \Lie(A)^\vee = \Omega^1(A) \]
\item The Tate module,
\[ T A = \ilim_{N} A[N](\tau) \cong \hat{\Z}^{2 \dim{A}} \]
which is a $\Gal{\bar{L}/L}$-module. Then write,
\[ VA = TA \ot_{\hat{\Z}} \A^{\infty} \]
\item $\Hom{}{A}{B}$ is a fg free abelian group with a Galois action
\item $\Hom{\circ}{A}{B} = \Hom{}{A}{B} \ot_{\Z} \Q$. Then an element $f \in \Hom{\circ}{A}{B}$ is called \textit{quasi-isogeny} if there exists $f^{-1} \in \Hom{\circ}{A}{B}$ a two-sided rational inverse
\item morphisms which are quasi-isogenies are called isogenies and are exactly the finite flat group maps. Then $\deg{f} = T B / f_* T A$.
\item $\End{\circ}{A}$ is a semi-simple $\Q$-algebra and hence a sum of matrix algebras over division algebras $D_i$ with $Z(D_i) = F_i$ fields over $L$ so we write,
\[ \End{\circ}{A} \cong \bigoplus_i M_{n_i}(D_i) \]
then we get the bound,
\[ \sum n_i \rank_{F_i}(D_i) [F_i : \Q] \le 2 \dim{A} \] 
\end{enumerate}
\end{rmk}

Note: the Hom and End are taken over $\ol{L}$ and given a Galois action.

\begin{defn}
We call $A$ \textit{potentially CM} if the previous inequality is an equality. 
\end{defn}

\begin{prop}
The following are equivalent,
\begin{enumerate}
\item $A$ is potentially CM
\item there exists $F \subset \End{\circ}{A}$ which is a product of fields with $[F : \Q] = 2 \dim{A}$ 
\item there exists $F \subset \End{\circ}{A}$ a product of imaginary CM fields with $[F : \Q] = 2 \dim{A}$ 
\item the $F_i$ in the previous remark are CM fields.
\end{enumerate}
\end{prop}

\begin{defn}
If $F$ is an imaginary CM field then a $F$-CM abelian variety over $L$ is a pair $(A, \iota)$ where $A/L$ is an abelian variety of dimension $[F : \Q]/L$ and $\iota : F \embed \End{\circ}{A}^{\Gal{\ol{L}/L}}$. 
\end{defn}

\begin{example}
Let $y^2 z = x^3 - x z^2$ over $\Q(i)$ is a $\Q(i)$-CM abelian variety where,
\[ [i] \cdot [x : y : z] = (-x : i y : z] \]
In this case it is a coincidence that the extension of $L = \Q$ over which the CM is realized equals the CM field. 
\end{example}

We have $A^\vee$ is the moduli space of homologically trivial line bundles on $A$ with universal (Poincare bundle) $\cP$ on $A \times A^\vee$ and $A^{\vee \vee} = A$. For $f : A \to B$ then get $f^{\vee} : B^{\vee} \to A^{\vee}$. Then there is a pairing,
\[ T A \times T A^\vee \to T \Gm = \hat{\Z}(1) \]

\begin{defn}
A \textit{polarization} of $A$ is a map $\lambda : A \to A^\vee$ such that $\lambda^\vee = \lambda$ and $(1 \times \lambda)^* \cP$ is ample.
\end{defn}

\begin{defn}
A \textit{quasi-polarization} is a $\lambda \in \Hom{\circ}{A}{A^\vee}$ such that $n \lambda$ is a polarization for some positive integer $n$. Then $\left<-, \lambda -\right>_A : V A \times VA \to \A^{\infty}(1)$ is alternation. The Rosati involution is defined by,
\[ f \mapsto f^{* \lambda} = \lambda^{-1} \circ f^{\vee} \circ \lambda \in \End{\circ}{A} \]
\end{defn}

If $A$ is a polarized CM and $\lambda$ is a quasi-polarizarion there exists $F \subset \End{\circ}{A}$ a product of CM fields with $[F : \Q] = 2 \dim{A}$ and $F$ is preserved by $*_{\lambda}$ menaing $*_{\lambda|_F = c$.

\begin{defn}
By a \textit{polarized $F$-CM abelian variety} we mean $(A, \iota, \lambda)$ where $(A, \iota)$ is a $F$-CM abelian variety and $\lambda : A \to A^\vee $is a quasi-polarization such that $*_{\lambda} |_F = c$.
\end{defn}

\begin{defn}
Let,
\[ \Phi(A, \iota) = \{ (\sigma_i : F_i \embed \CC) \} \]
\end{defn}

\subsection{$L = \CC$}

Over $L = \CC$ we have,
\[ A(\CC) \cong \Lie(A) / H_1(A, \ZZ) \]
and $VA = H_1(A(\CC), \ZZ) \ot \AA^{\infty}$.

\begin{prop}
Polarizations on $A$ correspond to Riemann forms: 
\[ E : H_1(A(\CC), \ZZ)^2 \to \Z \]
alternating and non-degenerate where,
\[ E_{\RR}(ix, iy) = E_{\RR}(x,y) \]
and $E_{\RR}(ix, x) > 0$ for all $x \neq 0$.
\end{prop}

\begin{prop}
For $W$ a fd $\CC$-vectorspace and $\Z \subset W$ a $\Z$-lattice then $W/\Lambda$ is the complex manifold underlying an abelian variety if and only if $\Lambda$ has a Riemann form.
\end{prop}

\begin{lemma}
If $(A, \iota)/L$ is an $F$-CM abelian variety then,
\[ \Phi(A, \iota) \sqcup \Phi(A, \iota) \circ c = \Hom{}{F}{\ol{L}} \]
\end{lemma}

\begin{proof}
Reduce to $L = \CC$ then we check this is $\Lie(A) \ot_{\RR} \CC = H_1(A(\CC), \Q) \ot_{\QQ} \CC$. Then there is an action of $F$ on the RHS but these have the same $\Q$-dimension so the RHS equals $F$ since the only representation of a field which has the same dimension over the perfect field is just the field $F$. 
\end{proof}

Construction of CM-abelian varities:

\begin{defn}
Polarized $F$-CM data: $(F, \a, \Phi, \xi)$ where \begin{enumerate}
\item $F$ is an imaginary CM-field 
\item $\a \subset F$ is a $\Z$-lattice
\item $\xi \in F^\times$ with $\xi^2 \in F^{+}_{\ll 0}$
\item $\Phi = \{ \varphi : F \embed \CC \mid \im{\varphi(\xi)} > 0 \}$ which is determined by the previous data.
\end{enumerate}
Given this data, we can associate a polarized $F$-CM abelian variety,
\[ A(\sigma) = \C^{\Phi} / \Phi(\a) \]
where,
\[ \Phi(\a) = \{ (\varphi(a))_{\varphi \in \Phi} \mid a \in \a \} \]
The action of $F$ on $A(\sigma)$ is given by,
\[ \iota(a) \cdot (x_{\varphi})_{\varphi \in \Phi} = (\varphi(a) x_{\varphi})_{\varphi \in \Phi} \]
which makes sense for all $a \in \a$ preserving the lattice under this action which is an order. Therefore, all of $\a$ acts by quasi-isogenies. Furthermore, we get an $F$-linear isomorphism,
\[ \eta^{\text{can}} : \AA^{\infty}_F \iso V A \]
defined by,
\[ \phi : F / \a \to A(\sigma) \]
Furthermore,
\begin{enumerate}
\item any polarized $F$-CM abelian variety arises in this way
\item the maps between these examples are,
\[ \Hom{F}{\Ab(F, \a, \Phi, \xi)}{\Ab(F, \a', \Phi', \xi')} =
\begin{cases}
0 & \Phi \neq \Phi'
\\
\{ a \in F \mif a \a \subset \a' & \Phi = \Phi' 
\end{cases} \] 
\item therefore, $\Ab(F, \a, \Phi, \xi) \cong \Ab(F, \a', \Phi', \xi')$ as polarized $F$-CM abelian varities if and only if $\exists \alpha \in F^\times$ such that $\a' = \alpha \a$ and $\Phi' = \Phi$ and $\xi' = \xi / \alpha c(\alpha)$.
\end{enumerate}
\end{defn}

If $F$-CM data then $(F, \Phi, \a)$ where $F$ is a CM-field and $\Phi$ is a CM-type (so $\Phi \subset \Hom{}{F}{\CC} = \Phi \sqcup \Phi \circ c$) and $\a \subset F$ is a lattice. Then we get,
\[ \{ F\text{-CM data} \} / F^\times \iff \{ F\text{-CM abelian varities} \} \]
here without polarization.


\end{document}




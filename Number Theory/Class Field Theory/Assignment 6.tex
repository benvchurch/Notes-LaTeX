\documentclass[12pt]{extarticle}
\usepackage[utf8]{inputenc}
\usepackage[english]{babel}
\usepackage[a4paper, total={7in, 9.5in}]{geometry}
 
\usepackage{amsthm, amssymb, amsmath, centernot}
\usepackage{mathtools}
\DeclarePairedDelimiter{\floor}{\lfloor}{\rfloor}

\newcommand{\notimplies}{%
  \mathrel{{\ooalign{\hidewidth$\not\phantom{=}$\hidewidth\cr$\implies$}}}}
 
\renewcommand\qedsymbol{$\square$}
\newcommand{\cont}{$\boxtimes$}
\newcommand{\divides}{\mid}
\newcommand{\ndivides}{\centernot \mid}
\newcommand{\Z}{\mathbb{Z}}
\newcommand{\N}{\mathbb{N}}
\newcommand{\C}{\mathbb{C}}
\newcommand{\Zplus}{\mathbb{Z}^{+}}
\newcommand{\Primes}{\mathbb{P}}
\newcommand{\ball}[2]{B_{#1} \! \left(#2 \right)}
\newcommand{\Q}{\mathbb{Q}}
\newcommand{\R}{\mathbb{R}}
\newcommand{\Rplus}{\mathbb{R}^+}
\newcommand{\invI}[2]{#1^{-1} \left( #2 \right)}
\newcommand{\End}[1]{\text{End}\left( A \right)}
\newcommand{\legsym}[2]{\left(\frac{#1}{#2} \right)}
\renewcommand{\mod}[3]{\: #1 \equiv #2 \: (\mathrm{mod} \: #3) \:}
\newcommand{\nmod}[3]{\: #1 \centernot \equiv #2 \: (\mathrm{mod} \: #3) \:}
\newcommand{\ndiv}{\hspace{-4pt}\not \divides \hspace{2pt}}
\newcommand{\finfield}[1]{\mathbb{F}_{#1}}
\newcommand{\finunits}[1]{\mathbb{F}_{#1}^{\times}}
\newcommand{\ord}[1]{\mathrm{ord}\! \left(#1 \right)}
\newcommand{\quadfield}[1]{\Q \small(\sqrt{#1} \small)}
\newcommand{\vspan}[1]{\mathrm{span}\! \left\{#1 \right\}}
\newcommand{\galgroup}[1]{Gal \small(#1 \small)}
\newcommand{\ints}[1]{\mathcal{O}_{#1}}
\newcommand{\sm}{\! \setminus \!}
\newcommand{\norm}[3]{\mathrm{N}^{#1}_{#2}\left(#3\right)}
\newcommand{\qnorm}[2]{\mathrm{N}^{#1}_{\Q}\left(#2\right)}
\newcommand{\quadint}[3]{#1 + #2 \sqrt{#3}}
\newcommand{\pideal}{\mathfrak{p}}
\newcommand{\inorm}[1]{\mathrm{N}(#1)}
\newcommand{\tr}[1]{\mathrm{Tr} \! \left(#1\right)}
\newcommand{\delt}{\frac{1 + \sqrt{d}}{2}}
\renewcommand{\Im}[1]{\mathrm{Im}(#1)}
\newcommand{\modring}[1]{\Z / #1 \Z}
\newcommand{\modunits}[1]{(\modring{#1})^\times}
\renewcommand{\empty}{\varnothing}
\renewcommand{\d}[1]{\mathrm{d}#1}
\newcommand{\deriv}[2]{\frac{\d{#1}}{\d{#2}}}
\newcommand{\pderiv}[2]{\frac{\partial{#1}}{\partial{#2}}}
\newcommand{\parsq}[2]{\frac{\partial^2{#1}}{\partial{#2}^2}}

\newcommand{\atitle}[1]{\title{% 
	\large \textbf{Mathematics W4043 Algebraic Number Theory
	\\ Assignment \# #1} \vspace{-2ex}}
\author{Benjamin Church \\ \textit{Worked With Matthew Lerner-Brecher} }
\maketitle}

 
\newtheorem{theorem}{Theorem}[section]
\newtheorem{lemma}[theorem]{Lemma}
\newtheorem{proposition}[theorem]{Proposition}
\newtheorem{corollary}[theorem]{Corollary}


\begin{document}
\atitle{6}
 
\section{} 
\begin{theorem}
Let $L/K$ be galois with group $G = \galgroup{L/K}$. Given $\alpha \in K^\times$ write the Frobenius $s_\alpha = (\alpha, L/K) \in G^{ab}$. Let $\chi \in \Hom{G}{\Q/\Z} = H^2(G, \Z)$ be a continuous character of degree $1$ and let $\delta_\chi \in H^2(G, \Z)$ be the image of $\chi$ by the coboundary map $\delta : H^1(G, \Q/\Z) \to H^2(G, \Z)$. Let $\bar{\alpha} \in K^\times / N_{L/K}(L^\times) = \hat{H}^0(G, L^\times)$ be the image of $\alpha$. Then, 
\[ inv_K(\bar{\alpha} \smile \delta \chi) = \chi(s_\alpha) \]
\end{theorem}
\noindent
The proof of this theorem and some associated lemmata will require the following properties of the cup product:
\begin{enumerate}
\item Associativity: $x \smile (y \smile z) = (x \smile y) \smile z$. 
\item For $x \in H^r(G, M)$ and $y \in H^s(G, N)$ we have $x \smile y = (-1)^{rs} (y \smile x)$.
\item Given a short exact sequence of $G$-modules, $1 \to A \to B \to C \to 1$ and $N$ a flat module consider the short exact sequence, 
\begin{center}
\begin{tikzcd}
1 \arrow[r] & A \otimes N \arrow[r] & B \otimes N \arrow[r] & C \otimes N \arrow[r] & 1
\end{tikzcd}
\end{center} 
Each short exact sequence gives rise to a very long exact sequence of Tate cohomology. Let $\delta : \hat{H}^r(G, C) \to \hat{H}^{r+1}(G, A)$ and $\delta' : \hat{H}^r(G, C \otimes N) \to \hat{H}^{r+1}(G, A \otimes N)$ be the boundary maps for these two very long exact seqences. Then, for $x \in \hat{H}^r(G, C)$ and $y \in \hat{H}^s(G, N)$, the cup product satisfies,
\[ \delta(x) \smile y = \delta'(x \smile y) \]
\item The inflation map commutes with cup products,
\[ \inf(x \smile y) = \inf(x) \smile \inf(y) \]
\end{enumerate}
Now we need to prove three lemmata from the appendix to Serre's book. I will state the first two and prove lemma 3. 

\begin{remark}
Given $a \in A^G$ I will use the notation $a^0 \in \hat{H}^0(G,A) = A^G / Nm_G(A)$ for its image. Furthermore, for $a \i A$ if $Nm_G(a) = 0$ then write $a_0 \in \hat{H}^{-1}(G, A)$ for its image.  
\end{remark}

\begin{lemma}
Given $a \in A^G$ let $f_a : \Z \to A$ be the unique $G$-morphis such that $f_a(1) = a$. If $x \in \hat{H}^n(G, B)$ then,
\[ a^0 \smile x \in \hat{H}^n(G, A \otimes B) \]
is the image of $x$ under the map $f_a \otimes I : \Z \otimes B \to A \otimes B$. 
\end{lemma}

\begin{lemma} \label{second}
Given $a \in A$ such that $Nm_G(a) = 0$ and $f$ a $1$-cocycle of $G$ to $B$, take $\bar{f} \in H^1(G, B)$. Then in $\hat{H}^1(G, A \otimes B)$ we have,
\[ a_0 \smile \bar{f} = c^0 \]
where,
\[ c = - \sum_{t \in G} ta \otimes f(t) \] 
\end{lemma}

\begin{lemma} \label{third}
Let $B$ be a $G$-module and $f : G \to B^1$ a $1$-cocycle with image $\bar{f} \in H^1(G, B)$. Then for each $s \in G$ we have $\bar{s} \smile \bar{f} = \overline{f(s)_0}$ in $\hat{H}^{-1}(G, B)$. 
\end{lemma}

\begin{proof}
First, some notation. Let $I_G$ be the augmentation ideal and for any $s \in G$ let $i_s = s - 1 \in I_G$. Note that $f$ is a $1$-cocycle and therefore a crossed homomorphism $f(\sigma \tau) =  f(\sigma) + \sigma \cdot f(\tau)$. Thus, consider,
\[ \sum_{\sigma \in G} \sigma \cdot f(\tau) = \sum_{\sigma \in G} (f(\sigma \tau) - f(\sigma)) = \sum_{\sigma' \in G} f(\sigma') - \sum_{\sigma \in G} f(\sigma) = 0 \]
Thus, $Nm_G f(\tau) = 0$ for any $\tau$ so $f(\tau)_0 \in \hat{H}^{-1}(G, B)$ is well-defined. Furthermore, $Nm_G(i_s) = 0$ so $(i_s)_0 \in \hat{H}^{-1}(G, I_G)$ is also well defined. Let $\delta : \hat{H}^{-2}(G, \Z) \to \hat{H}^{-1}(G, I_G)$ be the boundary map induced by the exact sequence,
\begin{center}
\begin{tikzcd}
1 \arrow[r] & I_G \arrow[r] & \Z[G] \arrow[r] & \Z \arrow[r] & 0
\end{tikzcd}
\end{center}
However,
\[ \hat{H}^{-2}(G, \Z) \cong G^{ab} \cong I_G / I_G^2 \cong \ker{(Nm_G)}/I_G^2 \cong \hat{H}^{-1}(G, I_G) \]
so $\delta$ is an isomorphism and if $s, s'$ lie in the same coset of $G^{ab}$ then $(i_s)_0 = (i_{s'})_0$. 
Tensoring with $B$ we get a map,
\[ \delta' : \hat{H}^r(G, \Z \otimes B) \to \hat{H}^r(G, I_G \otimes B) \]
which is also an isomorphism because $\delta$ is.  
First, consider,
\[ d'(\bar{s} \smile \bar{f}) = i_s \smile f = (i_s)_0 \smile f \]
By lemma \ref{second},
\[ (i_s)_0 \smile f = \left[ - \sum_{t \in G} (ti_s) \otimes f(t) \right]^0 \]
However, using the fact that $f(ts) = f(t) + t\cdot f(s)$ because $f$ is a crossed homomorphism, 
we can rewrite
\begin{align*}
\sum_{t \in G} (ti_s) \otimes f(t) & = \sum_{t \in G} (t - ts) \otimes f(t) = \sum_{t \in G} t \otimes f(t)  ts \otimes f(t)
\\
& = \sum_{t \in G} t \otimes f(t) - ts \otimes (t(ts)  - t \cdot f(s))
\\
& = \sum_{t \in G} t \otimes t(f0 - ts \otimes f(ts) + ts \otimes t \cdot f(s)
\\
& = \sum_{t \in G} t \otimes f(t) - t \otimes f(t) + ts \otimes t \cdot f(s)
\\
& = \sum_{t \in G} ts \otimes t \cdot f(s)
\end{align*}
where I have reindexed the second sum. Thus,
\[ \sum_{t \in G} (t i_s) \otimes f(t) = \sum_{t \in G} ts \otimes t\ cdot f(s) \]
Now we consider how $d'$ acts,
\[ d'(f(s)_0) = \left[ \sum_{t \in G} (t \otimes t \cdot f(s)) \right]^0 \]
Comparing these results gives,
\begin{align*}
d'(f(s)_0) - d'(\bar{s} \smile \bar{f}) & = \left[ \sum_{t \in G} (t \otimes t \cdot f(s))  \sum_{t \in G} ts \otimes t\ cdot f(s) \right]^0
\\
& = \left[ \sum_{t \in G} t(s-1) \otimes t \cdot f(s) \right]^0
\\
& = \left[ \sum_{t \in G} t \cdot \left[ (s - 1) \otimes f(s) \right] \right]^0
\\
& = \left[ Nm_G[ (s - 1) \otimes f(s) ] \right]^0 = 0 
\end{align*}
The image of the norm map is zero in $\hat{H}^0(G, I \otimes B)$ because $\hat{H}^0(G, I \otimes B)$ is the cokernel of $Nm_G$ be definition. Therefore, $d'(f(s)_0) = d'(\bar{s} \smile \bar{f})$ proving the claim. 
\end{proof}
\noindent
Now, we give the proof of the main theorem.
\begin{proof}
Let $u_{L/K}$ be a generator of the cyclic group $\hat{H}^2(G, L^\times)$. The map $x \mapsto x \smile u_{L/K}$ induces an isomorphism $\hat{H}^{-2}(G, \Z) \to \hat{H}^0(G, L^\times)$. However, we know that $\hat{H}^{-2}(G,\Z) \cong G^{\text{ab}}$ and $(L^\times)^G = K^\times$. Thus,
\[ G^{\text{ab}} \cong \hat{H}^{-2}(G, \Z) \cong \hat{H}^0(G, L^\times) \cong K^\times / Nm_G(L^\times) \]
I will denote the inverse of this isomorphism by,
\[ \theta_{L/K} : K^\times / Nm_G(L^\times) \xrightarrow{\sim} \hat{H}^{-2}(G, \Z) \] 
There is an exact sequence of trivial $G$ modules,
\begin{center}
\begin{tikzcd}
1 \arrow[r] & \Z \arrow[r] & \Q \arrow[r] & \Q / \Z \arrow[r] & 1
\end{tikzcd}
\end{center}
Taking the long exact chomology sequence gives an isomorphism $\delta : \hat{H}^1(G, \Q / \Z) \cong \hat{H}^2(G, \Z)$ because $\hat{H}^1(G, \Q) = \hat{H}^2(G, \Q) = 0$. Because these are trivial $G$-modules, the first cohomolgy is identified with the set of homs. Thus, we have an isomorphism,
\[ \delta : \Hom{G}{\Q / \Z} \to \hat{H}^2(G, \Z) \]
Take a character $\chi \in Hom{G}{\Q / \Z}$ and define $s_\alpha = \theta_{L/K}(\alpha)$. Then,
\[ \alpha \smile u_{L/K} = \alpha \]
where $\alpha \in K^\times / Nm_G(L^\times)$.  
Furthermore,
\[ \alpha \smile \delta(\chi) = (s_{\alpha} \smile u_{L/K}) \smile \delta(\chi) \]
since the grading of these elements in the chomology ring is even, the cup product is commutative and (is always) associative. Thus,
\[ \alpha \smile \delta(\chi) = u_{L/K} \smile (s_{\alpha} \smile \delta(\chi)) = u_{L/K} \smile \delta'(s_\alpha \smile \chi) = u_{L/K} \smile \delta'(\chi(s_\alpha)) \]
by Lemma \ref{third}. If the degree of $L/K$ is $[L : K] = n$ then $\chi(s_\alpha) = b/n$ for some $b \in \Z$ and $\delta'(b/n) = b$. thus, $\alpha \smile \delta(\chi) = u_{L/K} \smile b$. Applying the invariant map,
\[ inv_{L/K}(\alpha \smile \delta(\chi)) = inv_{L/K}(u_{L/K} \smile b) = b / n = \chi(s_{\alpha}) \] 
\end{proof}


\section{} 

Suppose that $K \subset K' \subset L$ is a sequence of $p$-adic fields with $L/K$ abelian. Let,
\[ r_{L/K} : K^\times \to \galgroup{L/K} \quad r_{K'/K} : K^\times \to \galgroup{K'/K} \]
be the reciprocity maps. Take any $a \in K^\times$. We need to show that $r_{L/K}(a)|_{K'} = r_{K'/K}(a)$. 
\bigskip\\
I claim that the following diagram commutes,
\begin{center}
\begin{tikzcd}
H^1(G_{K'/K}, \Q / \Z) \arrow[r, "\delta"] \arrow[d, "\inf"] & H^2(G_{K'/K}, \Z) \arrow[r, "\bar{\alpha} \smile"] \arrow[d, "\inf"] & H^2(G_{K'/K}, (K')^\times) \arrow[d, "\inf"] \arrow[r, "inv_{K'/K}"] & \Q / \Z \arrow[d, "\id"]
\\
H^1(G_{L/K}, \Q / \Z) \arrow[r, "\delta"] & H^2(G_{L/K}, \Z) \arrow[r, "\inf{\bar{\alpha}} \smile"] & H^2(G_{L/K}, (K')^\times) \arrow[r, "inv_{L/K}"] & \Q / \Z
\end{tikzcd}
\end{center}
To show this we use the definition of the invariant map and the fact that the inflation map commutes with cup products. 
Take any $\chi \in \Hom{G_{K'/K}}{\Q/\Z}$ and its image $\chi' = \inf{\chi}$ under the inflation map. Using the theorem in problem $1$, for any $\alpha \in K^\times$,
\[ \chi(r_{K'/K}(\alpha)) = inv_{K'/K}(\bar{\alpha} \smile \delta \chi) \]
Using the fact that the above diagram commutes,
\[ inv_{K'/K}(\bar{\alpha} \smile \delta \chi) = inv_{L/K}(\inf \bar{\alpha} \smile \delta \inf{\chi}) = inv_{L/K}(\inf \bar{\alpha} \smile \delta \chi') = \chi'(r_{L/K}(\alpha)) = \chi(r_{L/K}(\alpha) |_{K'}) \]
Thus, it suffices to show that if all characters agree on $g_1$ and $g_2$ then $g_1 = g_2$. That is, 
\[ \left( \forall \chi \in \Hom{G_{K'/K}}{\Q/\Z} : \chi(g_1) = \chi(g_2) \right) \implies g_1 = g_2 \]
Suppose that $\chi(g_1) = \chi(g_2)$ and thus $\chi(g_1 g_2^{-1}) = 1$ for each character $\chi$. However, $G_{K'/K}$ is a locally compact hausdorff topological group and thus the canonical map $ev_G : G \to \hat{\hat{G}}$ is an isomorphism. However, $ev_G(g_1 g_2^{-1})(\chi) = \chi(g_1 g_2^{-1}) = 1$ and thus $g_1 g_2{-1} = 1$ since $ev_G$ is an injection. Since, all characters agree on $r_{K'/K}(\alpha)$ and $r_{L/K}(\alpha) |_{K'}$ we have the desired result that,
\[ r_{L/K}(\alpha) |_{K'} = r_{K'/K}(\alpha) \]

\section{} 
There is a one-to-one correspondence between index two subgroups and surjective homomorphisms to $\{ \pm 1 \}$. To see this, suppose $\phi : G \to \{ \pm 1 \}$ is a surjective homomorphism then $G/ \ker{\phi} \cong \{ \pm 1 \}$ so $\ker{\phi}$ has index $2$. Conversely, suppose that $[G : H] = 2$ then $H$ is normal so $G/H \cong \{ \pm 1 \}$ and thus $\pi : G \to G/H$ is a surjective homomorphism to $\{ \pm 1 \}$ with kernel $H$. We will use this fact to find all the index $2$ open subgroups of $C_\Q = \idele{\Q} / \Q^\times$.
\bigskip\\
We can write the idele class group as,
\[ C_\Q = \frac{\idele{\Q}}{\Q^\times} = \R^\times_{> 0} \times \prod_p \Z_p^\times \]
For each odd prime $p$ we can take the reciprocity map, 
\[ f_p : C_\Q \to \Z_p^\times \xrightarrow{\pi} \finfield{p}^\times \xrightarrow{\legsym{-}{p}} \{ \pm 1 \} \]
Furthermore, for $p = 2$ we need to consider the elements modulo $8$. Since $(\Z / 8 \Z)^\times \cong (\Z / 2 \Z)^3$ we can take three different nontrivial homomorphisms. For odd $\delta$,
\[ f_{2, \delta}(x) = 
\begin{cases}
1 & \mod{x}{1,\delta}{8}
\\
-1 & \text{else}
\end{cases}\]
These are clearly homomorphisms $\Z_2^\times \to \{ \pm 1 \}$. For $p \neq 2$ any quadratic residue lifts to a square in $\Z_p^\times$ by Hensel's Lemma. Furthermore any two nonresidues always differ by a quadratic residue so any homomorphism $f : \Z_p^\times \to \{ \pm 1 \}$ must take all residues to $1$ and must be constant on the set of nonresidues. However, this does not hold for $p = 2$ which is why we must consider the elements modulo $8$ which determines the class of lifts in $\Z_2^\times$ since only elements $1$ modulo $8$ lift to squares in $\Z_2^\times$. 
Furthermore, if $S$ is a finite set of primes then,
\[ f_S = \prod_{p \in S} f_p : C_\Q \to \{ \pm 1 \} \]
Clearly these maps are surjective so their kernels are index $2$ subgroups. The prime $2$ needs special attention in $S$. 
Take a quadratic extension $K / \Q$ and consider the global Artin map which extends to each local Artin map via,
\begin{center}
\begin{tikzcd}
\Q_v^\times \arrow[r, "\phi_v"] \arrow[d] & \galgroup{K_v/\Q_p} \arrow[d]
\\
\idele{\Q}/\Q^\times \arrow[r, "\phi_{K}"] & \galgroup{K/\Q}
\end{tikzcd}
\end{center}
By properties of the local Artin map, if $v$ is unramified then $\phi_v$ takes any unit $u_v \in \Z_v^\times$ to the identity. By the global existence theorem, any open index $2$ subgroup of $C_\Q$ the image under the norm map of some quadratic field $K / \Q$. Thus, each $f_S$ must have a norm subgroup as its kernel so it factors through the global Artin map for $K / \Q$. However, for any prime $p$ we know that $\Z_p^\times$ is not contained in the kernel of $f_S$ if and only if $p \in S$. Thus, $S$ must be the set of ramified primes in $K / \Q$. However, we have classified all quadratic extensions of $\Q$ which are of the form $K = \Q(\sqrt{ \pm p_1 \cdots p_r })$ for distinct primes $p_1, \dots, p_r$ which has discriminant,
\[ \Delta_K = 
\begin{cases}
\pm p_1 \cdots p_r & \mod{\pm p_1 \cdots p_r}{1}{4}
\\
\pm 4 p_1 \cdots p_r & \text{else}
\end{cases}\]
Therefore, the set of ramified primes in $K / \Q$ is, $p_1, \dots, p_r$ if $\mod{\pm p_1 \cdots p_r}{1}{4}$ and otherwise, $2, p_1, \dots, p_r$ when all these primes are odd. For $K = \Q(\sqrt{d})$ let $\mod{\alpha}{d}{8}$ be the reduction modulo $8$. Given a set of odd primes $p_1, \dots, p_r$, I claim that Artin reciprocity gives the following correspondence, 
\begin{align*}
S = \{ p_1, \dots, p_r \} & \iff K = \Q(\sqrt{ \pm p_1 \dots p_r}) \quad \text{for} \quad \alpha = 1,5
\\
S = \{ (2, 5), p_1, \dots, p_r \} & \iff K = \Q(\sqrt{\pm p_1 \dots p_r}) \quad \text{for} \quad \alpha = 3,7
\\
S = \{ (2, 1 - \alpha), p_1, \dots, p_r \} & \iff K = \Q( \sqrt{ \pm 2 p_1 \dots p_r })  
\end{align*}
We have already shown that $S$ must contain exactly the ramified primes of $K$ which are exactly the prime factors of $\Delta_K$ modulo annoyances at $p = 2$. In the first case, $\mod{d}{1}{4}$ so $2$ is unramified and we have $S = \{p_1, \dots, p_r\}$ as required. To establish which residues we will need in $\Z_2^\times$ we need to check the norm map explicitly. Using just elements in $\Z_2^\times$, the image of the norm map for the field $K = \Q(\sqrt{d})$ will contain the residues, 
\[ \mod{x^2 - d y^2}{x^2 - \alpha y^2}{8} \]
In the second case, it is easy to see that $5$ is in the image of $x^2 - \alpha y^2$ modulo $8$. However, $2$ is ramified so the kernel must be nontrivial and thus correspond to the map $f_{2, 5}$. Finally, in the last case , $\alpha$ is even but not a multiple of $4$ so for $(x, y) = (1, 1)$,
\[ \mod{x^2 - \alpha y^2}{1 - \alpha}{8} \]
and $1 - \alpha$ is an odd residue not equal to one modulo $8$ because $8 \ndivides \alpha$. Therefore, the norm map has elements with residue $1 - \alpha$ in its image and thus it corresponds to the kernel of $f_{2,1 - \alpha}$.  
\bigskip\\
Since we have classified all quadratic extensions $K / \Q$ by Artin reciprocity and the global existence theorem, we have also found all open subgroups of index $2$ of $C_\Q$. In summary, the open index $2$ subgroups of $C_\Q$ are exactly, $\ker{f_S}$ for any set of primes (remembering that $2$ comes with three options) with the correspondence between the set $S$ and the associated quadratic field whose norm image is $\ker{f_S}$. 

\section{} 

First, let $K$ be a number field such that $K / \Q$ is a finite Galois extension with Galois group $G = \galgroup{K / \Q}$. We will restrict to the case in which $K$ is a quadratic field. However, first we will consider some general background results. 
\bigskip\\
Consider the exact sequence,
\begin{center}
\begin{tikzcd}
1 \arrow[r] & \frac{\sidele{K}{S_\infty} \cdot K^\times}{K^\times} \arrow[r] & C_K \arrow[r] & Cl(K) \arrow[r] & 1
\end{tikzcd}
\end{center}
obtained by applying the third isomorphism theorem to lemma \ref{idele_ideal} where the subgroup,
\[ \frac{\sidele{K}{S_{\infty}} \cdot K^\times}{K^\times} \subset C_K = \frac{ \idele{K} \cdot K^\times }{K^\times} \]
is the norm subgroup of the idele class group corresponding to the Hilbert class field of $K$ under the Artin reciprocity map. Let $G = \galgroup{K/\Q}$. This short exact sequence gives rise to a long exact sequence of cohomology,
\begin{center}
\begin{tikzcd}
1 \arrow[r] & \left( \frac{\sidele{K}{S_\infty} \cdot K^\times}{K^\times} \right)^G \arrow[r] & C_K^G \arrow[r] & Cl(K)^G \arrow[r] & H^1(G, (\sidele{K}{S_\infty} \cdot K^\times)/K^\times) \arrow[r] & 1
\end{tikzcd}
\end{center}
Where I have used the fact that $H^1(G, C_K) = 1$ by Lemma \ref{trivial_idele_class_homology}. Using Lemmata \ref{fixed_idele} and \ref{fixed_idele_subgroup} this exact sequence becomes,
\begin{center}
\begin{tikzcd}
1 \arrow[r] & C_\Q \arrow[r, two heads] & C_\Q \arrow[r] & Cl(K)^G \arrow[r] & H^1(G, (\sidele{K}{S_\infty} \cdot K^\times)/K^\times) \arrow[r] & 1
\end{tikzcd}
\end{center}
where the map $C_\Q \rightarrow C_\Q$ is surjective because it is simply the restriction of the inclusion map,
\begin{center}
\begin{tikzcd}
1 \arrow[r] & \frac{\sidele{K}{S_\infty} \cdot K^\times}{K^\times} \arrow[r] & C_K
\end{tikzcd}
\end{center}
which clearly takes the subgroup $C_\Q \mapsto \C_\Q$. Therefore, the map $C_\Q \to Cl(K)^G$ in the previous exact sequence is the zero map since the map $C_\Q \rightarrow C_\Q$ has full image. Thus, we get an exact sequence,
\begin{center}
\begin{tikzcd}
1\arrow[r] & Cl(K)^G \arrow[r] & H^1(G, (\sidele{K}{S_\infty} \cdot K^\times)/K^\times) \arrow[r] & 1
\end{tikzcd}
\end{center}
which gives a canonical isomorphism $Cl(K)^G \cong H^1(G, (\sidele{K}{S_\infty} \cdot K^\times)/K^\times)$. However, by the second isomorphism theorem,
\[ \frac{\sidele{K}{S_\infty} \cdot K^\times}{K^\times} \cong \frac{\sidele{K}{S_\infty}}{\sidele{K}{S_\infty} \cap K^\times} = \frac{\sidele{K}{S_\infty}}{\ints{K}^\times}  \]
where $\sidele{K}{S_\infty} \cap K^\times$ are the elements of $K^\times$ which are units in every local field and thus factor into no primes i.e. elements of $\ints{K}^\times$.
Therefore,
\[ Cl(K)^G \cong  H^1\left(G, \frac{\sidele{K}{S_\infty} \cdot K^\times}{K^\times}\right) \cong H^1\left(G, \frac{\sidele{K}{S_\infty}}{\ints{K}^\times}\right) \]
\bigskip\\
Now, consider the short exact sequence,
\begin{center}
\begin{tikzcd}
1 \arrow[r] & \ints{K}^\times \arrow[r] & \sidele{K}{S_\infty} \arrow[r] & \frac{\sidele{K}{S_\infty}}{\ints{K}^\times} \arrow[r] & 1
\end{tikzcd}
\end{center}
which gives rise to a long exact sequence of cohomology,
\begin{center}
\begin{tikzcd}
1 \arrow[r] & (\ints{K}^\times)^G \arrow[r] & (\sidele{K}{S_\infty})^G \arrow[draw=none]{d}[name=Z, shape=coordinate]{} \arrow[r, two heads] & \left( \frac{\sidele{K}{S_\infty}}{\ints{K}^\times} \right)^G \arrow[r] 
\arrow[dll,
rounded corners, crossing over,
to path={ -- ([xshift=2ex]\tikztostart.east)
|- (Z) [near end]\tikztonodes
-| ([xshift=-2ex]\tikztotarget.west)
-- (\tikztotarget)}]
& 1
\\ 
1 \arrow[r] & H^1(G, \ints{K}^\times) \arrow[r] & H^1(G, \sidele{K}{S_\infty}) \arrow[r] & H^1(G, \sidele{K}{S_\infty} / \ints{K}^\times) \arrow[r] & H^2(G, \ints{K}^\times) \arrow[r] & H^2(G, \sidele{K}{S_\infty}) 
\end{tikzcd}
\end{center}
By Lemma \ref{fixed_idele_subgroup}, the top row becomes,
\begin{center}
\begin{tikzcd}
1 \arrow[r] & \ints{\Q}^\times \arrow[r] & \sidele{\Q}{S_\infty} \arrow[r, two heads] & \frac{\sidele{\Q}{S_\infty}}{\ints{\Q}^\times} \arrow[r] & 1 
\end{tikzcd}
\end{center}
which can be extended to $1$ because the map $\sidele{\Q}{S_\infty} \to \frac{\sidele{\Q}{S_\infty}}{\ints{\Q}^\times}$ is the restriction of the projection map to a subgroup and its corresponding sub-quotient which is still a  surjective map. \bigskip \\
Thus, if we can show that the map $H^2(G, \ints{K}^\times) \to H^2(G, \sidele{K}{S_\infty})$ is injective then we have a short exact sequence, 
\begin{center}
\begin{tikzcd}
1 \arrow[r] & H^1(G, \ints{K}^\times) \arrow[r] & H^1(G, \sidele{K}{S_\infty}) \arrow[r] & H^1(G, \sidele{K}{S_\infty} / \ints{K}^\times) \arrow[r] & 1
\end{tikzcd}
\end{center}
which implies that,
\[ Cl(K)^G \cong H^1\left(G, \sidele{K}{S_\infty} / \ints{K}^\times \right) \cong H^1(G, \sidele{K}{S_\infty}) / H^1(G, \ints{K}^\times) \]

\subsection*{(a)}
Now we restrict to the case of an imaginary quadratic extension $K / \Q$. Since $G = \galgroup{K/\Q} \cong \Z / 2 \Z$ is finite cyclic, there is a natural isomorphism $\hat{H}^0(G, M) \xrightarrow{\sim} \hat{H}^2(G, M) = H^2(G, M)$. In particular, consider the map,
\[ \hat{H}^0(G, \ints{K}^\times) \to \hat{H}^0(G, \sidele{K}{S_\infty}) \]
We will exclude the cases $K \neq \Q(i)$ and $K \neq \Q(\zeta_3)$ such that $\ints{K}^\times = \{ \pm 1 \}$\footnote{This is not much of a restriction since we know that the class numbers of the fields $\Q(i)$ and $\Q(\zeta_3)$ are both $1$.}. Therefore, $\ints{K}^\times$ is a trivial $G$-module. In particular,
\[ H^0(G, \ints{K}^\times) = (\ints{K}^\times)^G = \ints{K}^\times \text{ and } \hat{H}^0(G, \ints{K}^\times) = H^0(G, \ints{K}^\times) / \mathrm{Nm_G}(\ints{K}^\times) = \ints{K}^\times \] 
However, at the ramified places of $\sidele{K}{S_\infty}$, the image of the norm map cannot contain $-1$ so the image of $-1$ inside the group $\hat{H}^0(G, \sidele{K}{S_\infty})$ is nontrivial. The map,
$H^2(G, \ints{K}^\times) \to H^2(G, \sidele{K}{S_\infty})$
is nontrivial by naturality of the shift by two isomorphism. However, \[H^2(G, M) = \hat{H}^2(G, M)\cong \hat{H}^0(G, M) \cong \ints{K}\]
 which has size two. Thus, any nontrivial map is injective. By the theory above, we have that,
\[ Cl(K)^G \cong H^1\left(G, \sidele{K}{S_\infty} / \ints{K}^\times \right) \cong H^1(G, \sidele{K}{S_\infty}) / H^1(G, \ints{K}^\times) \] 
Furthermore, since $\ints{K}^\times$ is a trivial $G$-module,
\[ H^1(G, \ints{K}^\times) = \Hom{G}{\ints{K}^\times} \cong \Z / 2 \Z \]
and using Lemma \ref{ramified_primes_decomp} we find that,
\[ Cl(K)^G \cong \left( \prod_{p \text{ ram.}} (\Z / e_{p} \Z) \right) / (\Z / 2 \Z) \]
However, since $n = 2$ for a quadratic field and $efg = 2$ we know that if a prime $p$ is ramified then $e_p = 2$. Thus,
\[ Cl(K)^G \cong (\Z / 2 \Z)^{r - 1} \]
where $r$ is the number of ramified primes in $K$. In particular, this implies that the class number $h_K$ of the imaginary quadratic field $K = \Q(\sqrt{- p_1 \cdots p_k})$ for distinct primes $p_i$ has a fast growing lower bound, \footnote{The prime $2$ may ramify even if $p_i \neq 2$ for any $i$. This occurs when $\mod{p_1 \cdots p_k}{1}{4}$ and only increases the exponent by one thus not altering the result.} 
\[ h_K \ge 2^{k - 1} \] 
This gives an affirmative answer to Gauss' conjecture that the class number of the quadratic field $\Q(\sqrt{d})$ goes to infinity as $d$ goes to infinity.

\subsection*{(b)}

I give up. Please have mercy. 

\subsection*{(c)}

If $K/\Q$ is a real quadratic extension rather than an imaginary one then vital steps in our proof break down. First, the map $H^2(G, \ints{K}^\times) \to H^2(G, \sidele{K}{S_{\infty}})$ will not be generically be surjective. Since the group of units will be infinite by Dirichlet's theorem, $H^1(G, \ints{K}^\times)$ will be much more complicated. However, the identification,
\[ Cl(K)^G \cong  H^1\left(G, \frac{\sidele{K}{S_\infty} \cdot K^\times}{K^\times}\right) \cong H^1\left(G, \frac{\sidele{K}{S_\infty}}{\ints{K}^\times}\right) \]
does still hold. However, rather than moding by torsion, we are now moding this idele group by an infinite abelian group. The resulting cohomology is much more difficult to calculate. I certianlly don't know how to do it in general. 

\newpage

\section{Lemmata}

\begin{lemma} \label{idele_ideal}
There is an exact sequence,
\begin{center}
\begin{tikzcd}
1 \arrow[r] & \sidele{K}{S_\infty} \cdot K^\times \arrow[r] & \idele{K} \arrow[r] & Cl(K) \arrow[r] & 1 
\end{tikzcd}
\end{center}
In particular, if $Cl(K) = 1$ then the exact sequence reduces to,
\begin{center}
\begin{tikzcd}
1 \arrow[r] & \sidele{K}{S_\infty} \cdot K^\times \arrow[r] & \idele{K} \arrow[r] & 1
\end{tikzcd}
\end{center}
and thus $\idele{K} \cong \sidele{K}{S_\infty} \cdot K^\times$. 
\end{lemma}

\begin{proof}
Define a map $\Phi : \idele{K} \to Cl(K)$ via,
\[ (a_v) \mapsto \prod_{v \notin S_{\infty}} \mathfrak{p}_v^{\ord_{\mathfrak{p}}(a_v)} \]
which is clearly surjective.
The kernel of this map is exactly elements of the form $(a_v) \in \idele{K}$ such that $\Phi((a_v)) = k \ints{K}$ is a principal ideal. Then, at each non-archimedean place, by Dedekind factorization,
\[ \ord_{\mathfrak{p}_v}(k) = \ord_{\mathfrak{p}_v} \left( \prod_{v \notin S_{\infty}} \mathfrak{p}_v^{\ord_{\mathfrak{p}}(a_v)} \right) = \ord_{\mathfrak{p}_v}(a_v) \] 
Thus, $a_v = k u_v$ where $u_v \in \ints{v}^\times$ since $a_v$ and $k$ generate the same ideal in $\ints{v}$. Thus, $(a_v) \in \sidele{K}{S_\infty} \cdot K^\times$. Clearly, any element of $\sidele{K}{S_\infty} \cdot K^\times$ is principal and thus in the kernel of $\Phi$. Thus, $\ker{\Phi} = \sidele{K}{S_\infty} \cdot K^\times$ and the required exact sequence follows immediately.  
\end{proof}

\begin{lemma} \label{trivial_idele_class_homology}
Let $L/K$ be a galois extension of global fields with $G = \galgroup{L/K}$. Let $C_L$ is the idele class group of $L$, then $H^1(G, C_L) = 1$. 
\end{lemma}

\begin{proof}
See Milne Section VII, Theorem 5.1. 
\end{proof}

\begin{lemma} \label{fixed_idele}
Let $L/K$ be a finite galois extensions with $G = \galgroup{L/K}$. Then $C_L^G = C_K$.  
\end{lemma}

\begin{proof}
Consider the short exact sequence,
\begin{center}
\begin{tikzcd}
1 \arrow[r] & L^\times \arrow[r] & \idele{L} \arrow[r] & C_L \arrow[r] & 1
\end{tikzcd}
\end{center}
which gives rise to a long exact sequence of cohomology,
\begin{center}
\begin{tikzcd}
1 \arrow[r] & (L^\times)^G \arrow[r] & (\idele{L})^G \arrow[r] & (C_L)^G \arrow[r] & H^1(G, L^\times) = 1 \arrow[r] & \cdots
\end{tikzcd}
\end{center}
where $H^1(G, L^\times) = 1$ by Hilbert's theorem 90. However, $(L^\times)^G = K^\times$ and $(\idele{L})^G = \idele{K}$ by Galois theory. Therefore, we have a short exact sequence, 
\begin{center}
\begin{tikzcd}
1 \arrow[r] & K^\times \arrow[r] & \idele{K} \arrow[r] & (C_L)^G \arrow[r] & 1
\end{tikzcd}
\end{center}
Thus, under the natural inclusions,
\[ (C_L)^G = \frac{\idele{K}}{K^\times} = C_K \]
\end{proof}

\begin{lemma} \label{ramification}
Let $L/K$ be a finite galois extensions. Let $\mathfrak{p}$ be a finite prime in $K$ and $\mathfrak{P}$ a prime of $L$ lying above $v$ with ramification index $e_{\mathfrak{P} | \mathfrak{p}}$ and decomposition group $D(\mathfrak{P}) = \galgroup{L_{\mathfrak{P}}/K_{\mathfrak{p}}}$. Then, 
\[H^1(D(\mathfrak{P}), \ints{\mathfrak{P}}^\times) \cong \Z / e_{\mathfrak{P} | \mathfrak{p}} \Z\]
\end{lemma}

\begin{proof}
Let $D = \galgroup{L_{\mathfrak{P}}/K_{\mathfrak{p}}}$. Consider the short exact sequence associated to a local field $L_w$,
\begin{center}
\begin{tikzcd}
1 \arrow[r] & \ints{\mathfrak{P}}^\times \arrow[r] & L^\times_{\mathfrak{P}} \arrow[r, "\ord_{\mathfrak{P}}"] & \Z \arrow[r] & 1 
\end{tikzcd}
\end{center}
This short exact sequence gives rise to a long exact sequence of cohomology,
\begin{center}
\begin{tikzcd}
1 \arrow[r] & (\ints{\mathfrak{P}}^\times)^D \arrow[r] & (L^\times_{\mathfrak{P}})^D \arrow[r, "\ord_{\mathfrak{P}}"] & \Z^D \arrow[r] & H^1(D, \ints{\mathfrak{P}}^\times) \arrow[r] & H^1(D, L_{\mathfrak{P}}^\times) \arrow[r] & \cdots 
\end{tikzcd}
\end{center}
However, by Hilbert's Theorem 90, $H^1(D, \ints{\mathfrak{P}}^\times) = 1$ so we get he exact sequence,
\begin{center}
\begin{tikzcd}
1 \arrow[r] & \ints{\mathfrak{p}}^\times\arrow[r] & K^\times_{\mathfrak{p}} \arrow[r, "\ord_{\mathfrak{P}}"] & \Z \arrow[r, "\varphi"] & H^1(D, \ints{\mathfrak{P}}^\times) \arrow[r] & 1
\end{tikzcd}
\end{center}
However, the image of $\ord_{\mathfrak{P}}$ on $K^\times_{\mathfrak{p}}$ is determined by, 
\[\ord_{\mathfrak{P}} \left( \mathfrak{p} \right) = \ord_{\mathfrak{P}} \left( \prod_{\mathfrak{P'} \divides \mathfrak{p}} \mathfrak{P'}^e \right) = \ord_{\mathfrak{P}} \left( \mathfrak{P}^e \right) = e\]
By exactness, $\ker{\varphi} = \Im{\ord_{\mathfrak{P}}} = e \Z$ so by the first isomorphism theorem,
\[ H^1(D, \ints{\mathfrak{P}}^\times) = \Z / e \Z \]
\end{proof}

\begin{lemma}
Let $L/K$ be a finite galois extension with Galois group $G = \galgroup{L/K}$. Let $v$ be a prime of $K$ with a prime $w_0$ in $L$ such that $w_0 \divides v$. Then,
\[ H^r(G, \prod_{w \divides v} L_w^\times) \cong H^r(D(w_0), L_{w_0}^\times) \] 
and likewise,
\[ H^r(G, \prod_{w \divides v} \ints{w}^\times) \cong H^r(D(w_0), \ints{w_0}^\times) \] 
\end{lemma} 

\begin{proof}
We use the fact that,
\[ \prod_{w \divides v} L_w^\times = \Ind{G}{D(w_0)} L_{w_0}^\times \]
and similarly, that,
\[ \prod_{w \divides v} \ints{w}^\times = \Ind{G}{D(w_0)} \ints{w_0}^\times \]
Therefore, by Shapiro’s Lemma,
\[ H^r(G, \prod_{w \divides v} L_w^\times) = H^r(G, \Ind{G}{D(w_0)} L_{w_0}^\times) = H^r(D(w_0), L_{w_0}^\times) \]
and similarly,
\[ H^r(G, \prod_{w \divides v} \ints{w}^\times) = H^r(G, \Ind{G}{D(w_0)} \ints{w_0}^\times) = H^r(D(w_0), \ints{w_0}^\times) \]
\end{proof}

\begin{lemma} \label{ramified_primes_decomp}
Let $L/K$ be finite galois with $G = \galgroup{L/K}$. Let $S$ be a finite set of primes in $K$ with $T$ the set of primes in $L$ lying above some prime in $S$. Then,
\[ H^r(G, \sidele{L}{T}) = \prod_{v \notin S} H^r(D(w_0), \ints{w_0}^\times) \times \prod_{v \in S} H^r(D(w_0), L_{w_0}^\times) \] 
In particular, $(\idele{L})^G = H^0(G, \idele{L}) = \idele{K}$ and $H^1(G, \idele{L}) = 1$ and last but not least,
\[ H^1(G, \sidele{L}{T_\infty}) = \prod_{v \text{ ram.}} (\Z / e_{w_0 | v} \Z) \]   
\end{lemma}

\begin{proof}
By definition,
\[ \sidele{L}{T} = \prod_{w \notin T} \ints{w}^\times \times \prod_{w \in T} L_w^\times = \prod_{v \notin S} \prod_{w \divides v} \ints{w}^\times \times \prod_{v \in S} \prod_{w \divides v} L_w^\times \]
which is a decomposition as a product of $G$-modules. Therefore, by the fact that cohomology commutes with products,
\[ H^r(G, \sidele{L}{T}) = \prod_{v \notin S} H^r(G, \prod_{w \divides v} \ints{w}^\times) \times \prod_{v \in S} H^r(G, \prod_{w \divides v} L_w^\times) \]
Thus, by the previous lemma,
\[ H^r(G, \sidele{L}{T}) = \prod_{v \notin S} H^r(D(w_0), \ints{w_0}^\times) \times \prod_{v \in S} H^r(D(w_0), L_{w_0}^\times) \]
In particular, 
\[ \idele{L} = \varinjlim_{T_0 \subset T} \sidele{L}{T} \]
where if $T \subset T'$ then $\sidele{L}{T} \subset \sidele{L}{T'}$. Thus, we can choose $S_0$ to contain the set of ramified primes (since there are finitely many) and $T_0$ to be all such primes lying over $T_0$. Thus,
\[ H^r(G, \idele{L}) =  \varinjlim_{T_0 \subset T} H^r(G, \sidele{L}{T}) = \varinjlim_{S_0 \subset S} \prod_{v \notin S} H^r(D(w_0), \ints{w_0}^\times) \times \prod_{v \in S} H^r(D(w_0), L_{w_0}^\times) \]
However, by assumption, all the ramified primes are in $S$ so by a previous lemma,
\[ H^1(D(w_0), \ints{w_0}^\times) = 0 \]
Furthermore, by Hilbert's theorem 90,
\[ H^1(D(w_0), L_{w_0}^\times) = 0 \]
Therefore, 
\[ H^1(G, \idele{L}) = 0 \]
Furthermore, 
\begin{align*}
H^0(G, \idele{L}) & = \varinjlim_{T_0 \subset T} H^r(G, \sidele{L}{T}) = \varinjlim_{S_0 \subset S} \prod_{v \notin S} H^0(D(w_0), \ints{w_0}^\times) \times \prod_{v \in S} H^0(D(w_0), L_{w_0}^\times)
\\
& = \varinjlim_{S_0 \subset S} \prod_{v \notin S} (\ints{w_0}^\times)^{D(w_0)} \times \prod_{v \in S} (L_{w_0}^\times)^{D(w_0)} = \varinjlim_{S_0 \subset S} \prod_{v \notin S} \ints{v}^\times \times \prod_{v \in S} L_v^\times = \idele{K}
\end{align*}
\bigskip\\
Likewise, using Hilbert's Theorem 90 and Lemma \ref{ramification},
\begin{align*}
H^1(G, \sidele{L}{T_\infty}) & = \prod_{v \notin S_{\infty}} H^1(D(w_0), \ints{w_0}^\times) \times \prod_{v \in S_{\infty}} H^1(D(w_0), L_{w_0}^\times) 
\\
& = \prod_{v \notin S_{\infty}} (\Z / e_{w_0 | v} \Z) = \prod_{v \text{ ram.}} (\Z / e_{w_0 | v} \Z)
\end{align*}
\end{proof}

\begin{lemma} \label{fixed_idele_subgroup}
\[ \left(\frac{\sidele{K}{S_\infty} \cdot K^\times}{K^\times} \right)^G = \frac{\sidele{\Q}{S_\infty} \cdot \Q^\times}{\Q^\times}  = \frac{\idele{\Q}}{\Q^\times} = C_\Q \]
and similarly,
\[ \left(\frac{\sidele{K}{S_\infty} \cdot K^\times}{K^\times} \right)^G = \left( \frac{\sidele{K}{S_\infty}}{\ints{K}^\times} \right)^G = \frac{\sidele{\Q}{S_\infty}}{\ints{\Q}^\times} = C_\Q \]
\end{lemma}

\begin{proof}
Let $G = \galgroup{K/\Q}$ where $K/\Q$ is finite galois. Note that $\Q$ has class number $1$ so by Lemma \ref{fixed_idele} we know that $\idele{\Q} = \sidele{\Q}{S_\infty} \cdot \Q^\times$. Now, consider the exact sequence, 
\begin{center}
\begin{tikzcd}
1 \arrow[r] & K^\times \arrow[r] & \sidele{K}{S_\infty} \cdot K^\times \arrow[r] & \frac{\sidele{K}{S_\infty} \cdot K^\times}{K^\times} \arrow[r] & 1
\end{tikzcd}
\end{center}
which gives rise to a long exact sequence of cohomology,
\begin{center}
\begin{tikzcd}
1 \arrow[r] & (K^\times)^G \arrow[r] & (\sidele{K}{S_\infty} \cdot K^\times)^G \arrow[r] & \left( \frac{\sidele{K}{S_\infty} \cdot K^\times}{K^\times} \right)^G \arrow[r] & H^1(G, K^\times) = 1
\end{tikzcd}
\end{center}
where $H^1(G, K^\times) = 1$ by Hilbert's Theorem 90. However,
\[ \idele{\Q} = \sidele{\Q}{S_\infty} \cdot \Q^\times \subset ( \sidele{K}{S_\infty} \cdot K^\times)^G \subset (\idele{K})^G = \idele{\Q} \]
and thus, $( \sidele{K}{S_\infty} \cdot K^\times )^G = \idele{Q}$.
Therefore, the long exact sequence reduces to a short exact sequence,
\begin{center}
\begin{tikzcd}
1 \arrow[r] & \Q^\times \arrow[r] & \idele{\Q} \arrow[r] & \left( \frac{\sidele{K}{S_\infty} \cdot K^\times}{K^\times} \right)^G \arrow[r] & 1
\end{tikzcd}
\end{center}
Therefore, 
\[ \left( \frac{\sidele{K}{S_\infty} \cdot K^\times}{K^\times} \right)^G = \idele{\Q} / \Q^\times = C_{\Q} \]
Furthermore, by the second isomorphism theorem,
\[ \frac{\sidele{K}{S_\infty} \cdot K^\times}{K^\times} = \frac{\sidele{K}{S_\infty}}{K^\times \cap \sidele{K}{S_\infty}} = \frac{\sidele{K}{S_\infty}}{\ints{K}^\times} \]
And thus, again by the second isomorphism theorem,
\[ \left(\frac{\sidele{K}{S_\infty} \cdot K^\times}{K^\times} \right)^G = \left( \frac{\sidele{K}{S_\infty}}{\ints{K}^\times} \right)^G = C_\Q = \frac{\idele{\Q}}{\Q^\times} = \frac{\sidele{\Q}{S_\infty} \cdot \Q^\times}{\Q^\times} = \frac{\sidele{\Q}{S_\infty}}{\ints{\Q}^\times}  \]
\end{proof}

\end{document}
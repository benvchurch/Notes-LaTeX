\documentclass[12pt]{article}
\usepackage{import}
\import{./}{NumberTheoryCommands}


\begin{document}

\section{Early Developments}

\renewcommand{\C}{\mathbb{C}}
\newcommand{\wh}[1]{\widehat{#1}}
\renewcommand{\Re}[1]{\mathrm{Re}\left(#1\right)}

\begin{defn}
The Riemann zeta function is,
\[ \zeta_(s) = \sum_{n = 1}^\infty \frac{1}{n^s} \]
\end{defn}

\begin{prop}
The summation form of $\zeta$ converges absolutely and is analytic for $\Re{s} > 1$. 
\end{prop}

\begin{proof}
Let $s = a + i b$ then,
\[ \sum_{n = 1}^\infty \frac{1}{|n^s|} = \sum_{n = 1}^\infty \frac{1}{n^a} \]
converges when $a > 1$. Furthermore, 
\[ \deriv{}{s} \zeta(s) = \sum_{n = 1}^\infty \frac{\ln{n}}{n^s} \] 
and additionally,
\[ \sum_{n = 1}^\infty \frac{\ln{n}}{n^a} \] 
converges for $a > 1$.
\end{proof}

\begin{prop}
For all $\Re{s} > 1$ there is a convergent Euler product representation,
\[ \zeta(s) = \prod_{p} \frac{1}{1 - p^{-s}} \]
\end{prop}


\begin{cor}
There are infinitely many primes.
\end{cor}

\begin{proof}
The limit $\lim_{s \to 1^{+}} \zeta(s) = \infty$ and therefore,
\[ \prod_{p} \frac{1}{1 - p^{-s}} \]
cannot be a finite product else it would converge in the limt $s \to 1^{+1}$ to,
\[ \prod_{p} \frac{1}{1 - p^{-1}} \]
\end{proof}

\subsection{Linear Characters}

\begin{defn}
Let $G$ be an abelian group. Then $\wh{G} = \Hom{}{G}{\C^\times}$ is the character group where $(\varphi_1 \cdot \varphi_2)(g) = \varphi_1(g)\varphi_2(g)$.
\end{defn}

\begin{prop}
Let $\chi_1, \chi_2$ be characters,
\[ \sum_{g \in G} \overline{\chi_1(g)} \chi_2(g) = 
\begin{cases}
|G| & \chi_1 = \chi_2
\\
0 & \chi_1 \neq \chi_2
\end{cases} \]
Let $g,h \in G$ be group elements,
\[ \sum_{\chi \in \wh{G}} \overline{\chi(g)} \chi(h) = 
\begin{cases}
|\wh{G}| & g = h
\\
0 & g \neq h
\end{cases} \]
\end{prop}

\begin{proof}
We know,
\[ \overline{\chi_1(g)} \left( \sum_{h \in G} \overline{\chi_1(h)} \chi_2(h) \right) \chi_2(g) = \sum_{h \in G} \overline{\chi_1(gh)} \chi_2(gh) = \sum_{h' \in G} \overline{\chi_1(h')} \chi_2(h') \]
Therefore, either $\chi_1(g) = \chi_2(g)$ for all $g$ or the sum is zero. The second is similar.
\end{proof}

\subsection{Dirichlet $L$-Functions}

\begin{defn}
Let $\chi : (\Z / m \Z)^\times \to \C^\times$ be a character. This extends to a multiplicative function $\chi : \Z \to \C^\times$ by setting $\chi(a) = 0$ for $\gcd(a, m) > 1$ and $\chi_0(a) = 1$. Then the Dirichlet $L$-function,
\[ L(s, \chi) = \sum_{n = 1}^\infty \frac{\chi(n)}{n^s} \]
\end{defn}

\begin{rmk}
We have $L(s, \chi_0) = \zeta(s)$.
\end{rmk}

\begin{prop}
For $\chi \neq \chi_0$ the $L$-function $L(s, \chi)$ is well-defined in the limit $s \to 1^{+}$ and $L(1, \chi) \neq 0$.
\end{prop}

\begin{proof}

\end{proof}

\begin{prop}
The $L$ function has an Euler product for $\Re{s} > 1$,
\[ L(s, \chi) = \prod_{p} \frac{1}{1 - \chi(p) p^{-s}} \]
\end{prop}

\begin{prop}
The density of primes in the arithmetic progression $an + b$ is $\frac{1}{\varphi(a)}$.
\end{prop}

\begin{proof}
Consider 
\end{proof}

\end{document}


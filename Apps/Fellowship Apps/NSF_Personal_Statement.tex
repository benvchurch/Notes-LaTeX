\documentclass[11pt]{article}
\usepackage{import}
\usepackage[margin=1.01in]{geometry}

\usepackage{hyperref}
\usepackage{endnotes}
\let\footnote=\endnote

\usepackage{microtype}
\usepackage{amsmath, amssymb, amsfonts}


\usepackage{fontspec}
\setmainfont{Times New Roman}
\usepackage{unicode-math}
% If using the built-in font:
\setmathfont{[Cambria-Math.ttf]}

% Math Packages
 
\usepackage{amsthm, amssymb, amsmath, centernot, graphicx}
\usepackage{stmaryrd}
\usepackage{relsize}
\usepackage{mathtools}
\usepackage{tikz}
\usepackage{tikz-cd}
\usepackage{graphicx}

% needed for correctly formatting words

\usepackage{xspace}
\newcommand{\et}{\text{\'{e}t}}
\newcommand{\etale}{\'{e}tale\xspace}
\newcommand{\cech}{\v{C}ech\xspace}

% Shorthand

\newcommand{\ot}{\otimes}
\newcommand{\id}{\mathrm{id}}
\newcommand{\F}{\mathcal{F}}


% Number Theory Commands

\newcommand{\Z}{\mathbb{Z}}
\newcommand{\ZZ}{\mathbb{Z}}
\newcommand{\N}{\mathbb{N}}
\newcommand{\NN}{\mathbb{NN}}
\newcommand{\Zplus}{\mathbb{Z}^{+}}
\newcommand{\Primes}{\mathbb{P}}
\newcommand{\Q}{\mathbb{Q}}
\newcommand{\FF}{\mathbb{F}}
\newcommand{\RR}{\mathbb{R}}
\newcommand{\CC}{\mathbb{C}}
\newcommand{\divides}{\mid}
\newcommand{\ndivides}{\not \mid}
\newcommand{\modeq}[3]{#1 \equiv #2 \quad (\mathrm{mod} \: {#3})}
\DeclareMathOperator{\ch}{\mathrm{char}}

% Important Schemes 

\newcommand{\A}{\mathbb{A}}
\renewcommand{\P}{\mathbb{P}}
\newcommand{\Ga}{\mathbb{G}_a}
\newcommand{\Gm}{\mathbb{G}_m}

% Operations on Schemes

\newcommand{\Sing}[1]{\mathrm{Sing}\left( #1 \right)}
\newcommand{\NS}[1]{\mathrm{NS}\left( #1 \right)}


% Rational Maps

\newcommand*{\DashedArrow}[1][]{\mathbin{\tikz [baseline=-0.25ex,-latex, dashed,#1] \draw [#1] (0pt,0.5ex) -- (1.3em,0.5ex);}}%
\newcommand*{\DashedBiArrow}[1][]{\mathbin{\tikz [baseline=-0.25ex,-latex, dashed, #1] \draw [#1] (0pt,0.5ex) -- (1.3em,0.5ex) node[midway, above] {$\sim$} ;}}
\newcommand{\rat}{\DashedArrow[densely dashdotted]}
\newcommand{\birat}{\DashedBiArrow[densely dashdotted]}
\newcommand{\Dom}[1]{\mathrm{Dom}\left(#1 \right)}

% Theorem Formats

\newtheorem*{conj}{Conjecture}
\newtheorem*{prop}{Proposition}
\newtheorem*{defn}{Definition}
\newtheorem*{thm}{Theorem}

% Bib Formatting

\usepackage{blindtext}
\usepackage[style=science, backend=bibtex8,
            autocite=footnote, notetype=endonly, labeldateparts]{biblatex}

\addbibresource{bibliography.bib}

\begin{document}

\pagestyle{empty}
\fontsize{11.1}{12.1}\selectfont % weird lower limit set by grfp website.
\noindent
\textbf{Personal Background:} Although I enjoyed and excelled at math from an early age, my curiosity and attention were more focused on the natural world: using mathematics to uncover nature's mysteries. Through college, I fell in love with the startling depth and overwhelming beauty to be found within mathematics itself. Having completed a wide array of mathematics coursework and my first year of graduate school, I am excited to fully immerse myself in mathematical research and begin a career as an academic.
\newline
\noindent
\textbf{Intellectual Merit:} Due to my broad interests in science, I have explored diverse areas through
coursework, research, and independent study. This array of projects has equipped me with a wide set of skills, a flexible approach to problem-solving, and a unique breadth of perspective. These experiences have prepared me with the perseverance and the particular background to excel in mathematical research. 
\par
I began my first serious project under the direction of Prof Jessica Mar studying differential skewness in gene-expression data collected from cancer cells. During the project, she asked that I read foundational texts on probability theory and statistics to ensure I fully understood the mathematical
underpinnings of techniques such as correction for multiple testing and Bayesian clustering analysis. This showed me the value of rigorous pure mathematics even in applied settings, both as a tool and a guide, and influenced my interest in interdisciplinary mathematical work. Our work was published in BMC Bioinformatics\footfullcite{skew} and I presented it at the 2019 International Conference on Intelligent Biology and Medicine.
\par 
My next research project was in galaxy astrophysics and dark matter physics under Prof J. P. Ostriker. The fuzzy cold dark matter (FDM) model hypothesizes that a significant fraction of the dark sector is comprised of ultra-light axion-like bosons and predicts that quantum-mechanical effects, such as interference, should be manifest on galactic scales. During galaxy formation, FDM interference can set up
long-living standing waves that produce gravitational fluctuations that pump energy into the stars forming a galactic disc. We proposed this as a mechanism for causing the observed thickness of the Milky Way disc. I modeled the evolution of adiabatic invariants of the disc as it underwent gravitational interactions with the standing waves formed by the collision-less dark fluid which allowed me to derive the history and radial profile of this disc heating. My predictions closely agree with observations of the Milky Way and the observed evolution profiles of galactic discs. Fitting this model to observations, I was able to set a lower bound on the mass of the ultra-light particle hypothesized to constitute such dark matter\footfullcite{FDM}.
\par
I participated in the 2018 Columbia math REU studying the zeta functions of diagonal weighted-
projective surfaces over finite fields using computational methods. We aimed to generalize Shioda’s classification of supersingular Fermat varieties to weighted-projective diagonal hypersurfaces. However, the naive generalization given by applying Shioda’s classification to the minimal covering Fermat surface
provided a sufficient but not necessary condition for supersingularity. We approached the problem from an arithmetic perspective by using a result of Weil to compute the zeta functions of these diagonal hypersurfaces in terms of Gaussian sums. We then applied Stickelberger's theorem to determine the
factorization of ideals generated by Gaussian sums and thus determine the roots and poles of the zeta function corresponding to eigenvalues of the Frobenius action on the variety’s l-adic cohomology reducing the problem to a numerical condition on the exponents and characteristic. Using a computer search, I was able to identify patterns in certain new examples of supersingular surfaces. From this observation, I proved the existence of an infinite family of supersingular weighted-projective surfaces
such that the minimal Fermat surface parametrizing them fails to be supersingular. We then identified other infinite families with this same property.
\par 
In the summer of 2019, I had the wonderful opportunity to study toric geometry and inequalities in convex
geometry in Paris through a joint program between Columbia and Paris Diderot University. Under the
direction of Prof Huayi Chen, my group studied the relationship between intersection pairings of big nef divisors and inequalities in convex geometry and the relationships between these inequalities and
constructions on toric varieties. Associated to such divisors are compact convex sets called Okunkov
bodies whose volume reflects the intersection pairing and the asymptotic number of sections. Variants of the
Brunn-Minkowski, Alexandrov-Finchel, and isoperemetric inequalities applied to these Okunkov bodies
can be strengthened by introducing probabilistic techniques \footfullcite{isoperimetric}. Specifically, an important term arising in these inequalities is the correlation between convex bodies which has a form similar to a Kantorovich
optimal transport problem. We established an upper bound on the correlation between special cases of
convex pairs using the Brenier map.
\par
The Paris Diderot program gave me a solid background in toric geometry which prepared me for writing my undergraduate thesis. Under Prof Johan de Jong, I studied the problem of embedding smooth curves in toric
surfaces. I showed that very general curves cannot be
embedded in any toric surface and I gave examples with obstructions to embeddings being
Cartier. This project was inspired by a paper of Dokchitser \footfullcite{models_of_curves} which provides an algorithm to construct, for a given curve, the minimal regular normal crossings model defined over a DVR. Dokchitser's method
requires embedding the curve in a toric surface and constructs the model by gluing toric surfaces associated to subdivisions of the Newton polygon. I constructed a degeneration of a genus 5
curve with a nontrivial Galois action on the components of its special fiber and showed that such a regular
normal crossings model cannot result from Dokchitser's method. For my thesis, I was awarded the John Dash van Buren Jr. Prize in Mathematics. This work was an invaluable learning experience about how research is conducted in mathematics: what techniques to try, how to manage frustration, and how problems evolve. Its enjoyment banished my reservations about pursuing graduate studies in mathematics and fully convinced me that mathematical research is what I want to spend my life doing
\par 
Upon arriving at Stanford to start graduate school, I completed my qualifying exams giving me the flexibility to spend my first year focused on learning and research. I did independent reading with Prof Brian Conrad using his two courses on algebraic groups. Two other graduate students and I worked out the majority of Prof Conrad's exercises on algebraic groups. Working out these specific details turned out to be immensely rewarding in forcing me to carefully understand how to apply certain techniques for systematically reducing problems to special cases such as algebraically closed fields or small extensions of Artin rings. Concurrently, I finished carefully writing up solutions to nearly every exercise of Hartshorne's \textit{Algebraic Geometry} with advice from Prof Conrad and Prof Vakil.  
\par 
The student algebraic geometry seminar at Stanford has been a remarkably effective learning opportunity for me. Starting my first quarter, I began giving talks in the seminar and helping to organize the topics. So far, we have given talk series on: the method of Deligne-Illusie, Mori's bend and break, variations of Hodge structures, and Mazur's theorem on torsion of elliptic curves. The quarter on bend and break has been especially helpful as it directly relates to my current research and so I lead the seminar, writing detailed notes and planning the structure of the talks. 
\par
At the end of my first quarter, I began regularly meeting with Prof Ravi Vakil -- who has now agreed to be my advisor -- to discuss possible research topics. We first discussed the question: does a smooth proper map $f : X \to \P^1$ from a variety over $\overline{k}$ admit a section? In characteristic zero this is proven by techniques in symplectic geometry. We tried to answer it in positive characteristic but were unsuccessful. While taking Prof Vakil's courses on moduli problems and stacks, I wrote weekly notes attempting to answer extra questions Prof Vakil would raise in class including patching a flaw Prof Vakil found in the standard proof of the flattening stratification. These questions motivated me to systematically study the properties of \textit{isotrivial} families -- those whose fibers are all isomorphic -- in terms of the residual gerbes of moduli stacks. Motivated by the result in complex geometry that a proper submersion with isomorphic fibers is locally trivial, I asked if the same were true for smooth proper morphisms with the \etale topology. I was able to show that isotrivial families of smooth \textit{polarized} varieties with smooth automorphism groups are \etale-locally trivial but I am still unsure if the result holds for non-polarized families. 
\par 
Over the past summer, Prof Vakil and I worked on a project attempting to define a new version of the Bott periodicity homotopy equivalence between suitably defined algebraic classifying stacks of Clifford modules. Prof Vakil and Hannah Larson proved the existence of such a $2$-periodic homotopy equivalence for the ind-stack $\mathrm{BGL}$. I attempted to find similar maps between algebraic realizations of the spaces which appear in eight-fold real Bott periodicity. Topologically these are equivalent to certain classifying spaces of Clifford modules. My approach is to study the corresponding classifying stacks of algebraic Clifford modules realized as algebraic group quotients. I am currently helping to organize a seminar on $\A^1$-homotopy theory which I hope will help me understand better how the objects work when I return to the project in the future. I am spending the current semester at Columbia University where I am working with Nathan Chen and Prof Vakil on finding rational curves on supersingular surfaces. This work has been extremely rewarding as Nathan is both an invaluable resource -- teaching me about Koll\'{a}r's degeneration method and other positive characteristic tricks -- and a stimulating collaborator. 
\newline
\noindent
\textbf{Broader Impacts:} I am deeply indebted to the opportunities and people who made me see mathematics the way I do today. Therefore, I see it as my duty to be that person for a new generation of aspiring
mathematicians and scientists. I have volunteered to teach at least 20 Splash classes at Columbia, MIT, and Stanford
aimed at high school students curious about physics and math beyond the classroom, as well as a six-week course on elliptic curves for high school students through HSSP. I had highly-motivated students
who I led on a meandering journey through some of the most surprising and wonderful connections
between algebra and geometry. I will be very proud if I succeeded in instilling an appreciation for the
beautiful geometry hiding in high-school-level algebra. I additionally taught at the Ross Mathematics Program for high school students, mentoring a group of five students in the number theory coursework.
\par 
At Stanford, I have served as a course assistant for \textit{Proofs and Modern Mathematics} and \textit{Algebraic Geometry} for which I guided students during weekly office hours and graded homework assignments. It was rewarding to help students towards modern mathematics' insights, both those just beginning their studies and those learning advanced topics.  Hunter Spink, the instructor for \textit{Algebraic Geometry}, nominated me for a teaching award for, beyond the confines of my official duties, writing notes untangling the subtler points confusing my students that I felt I had not adequately explained during office hours.
\par
Teaching is not only a way to give back; I believe it is essential for my own intellectual growth. Teaching forces one to critically reexamine what one thinks they know and to consider alternative perspectives and mental models.
I was significantly involved with the math and physics communities at Columbia through two student-run organizations: the Columbia Undergraduate Mathematics Society (UMS) and the Columbia chapter of the
Society of Physics Students (SPS) which offer educational outreach and exposure to a broad array of
ideas in the field. Serving as president of SPS during my senior year, I organized weekly talks by
professors and graduate students to help undergraduates understand the field and possibly find research
topics that interest them. Furthermore, SPS provides outreach and demos for NYC middle schoolers.
Along with the Columbia Society for Women in Physics, SPS organizes mentorship, advising, study and help
sessions, as well as a relaxed environment for students to meet and learn from their peers. I also served on
the board of UMS which provided similar environment, outreach, and weekly talks for the math
community. I helped create and participated in an undergraduate speaker series for undergraduates to
practice their presentation skills and to promote teaching and learning between peers. Over the course of
my involvement with these clubs, I have given over a dozen talks and presentations on a wide variety of topics
from the whimsical, ``is it possible to start a fire with moonlight'' aimed at exciting new SPS members, to
more technical topics such as ``Galois representation attached to elliptic curves''. In collaboration
with the Columbia Association for Women in Mathematics, I helped deliver a series of introductory talks
and help sessions and wrote example-based introductory materials aimed at supporting freshmen who
were new to proof-based college-level mathematics courses. Recognizing that disparities in higher education are widened by disparities in access to advanced preparation in high school, we aimed to mitigate the attrition of underrepresented groups in mathematics by helping incoming students bridge those gaps through resources, support, and peer-to-peer mentoring.
\par
I have continued this commitment to community outreach in my first year at Stanford. I mentored an undergraduate reading Marker's \textit{Model Theory} in the directed reading program. I have also spoken at the undergraduate math club and participated in their outreach events including a panel on graduate school advice. I described the program we started at Columbia for supporting incoming students to the leadership of the Stanford undergraduate math society and hope to help organize a similar program in the future. 
\newline
\noindent
\textbf{Future Goals:} My interests have coalesced around algebraic geometry, specifically birational geometry
in positive characteristic. In the short term, I intend to continue studying the Shioda conjecture and rationality problems and pursue such topics in my thesis work. I also hope to return to questions about the homotopy theory of stacks once I have deepened my background understanding since it offers an intriguing intersection between algebraic geometry and classical homotopy theory. I hope that in my graduate studies, I will be able to unify my lifelong interests towards solving
research problems on the boundary between mathematics and physics. My ultimate goal is to become a professor. This goal goes beyond having a research position; I see teaching as a vital aspect of my career
both in order to repay the excellent instruction I was given and because I believe teaching others to be an
invaluable part of my own intellectual development.
\vspace{-2.5em}
\begingroup
\let\enotesize\normalsize
\renewcommand\notesname{\hrulefill \vspace{-0.9em}}
\theendnotes
\endgroup

\end{document}
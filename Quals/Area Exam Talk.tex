\documentclass[12pt]{article}
\usepackage{import}
\import{"../Algebraic Geometry/"}{AlgGeoCommands}

\newcommand{\Loc}[1]{\mathfrak{Loc}\left( #1 \right)}
\newcommand{\AbGrp}{\mathbf{AbGrp}}

\renewcommand{\K}{\mathbb{K}}

\newcommand{\inner}[2]{\left< #1, #2 \right>}

\newcommand{\B}{\mathcal{B}}
\newcommand{\R}{\mathbb{R}}

\newcommand\eqae{\mathrel{\stackrel{\makebox[0pt]{\mbox{\normalfont\tiny a.e.}}}{=}}}
\renewcommand{\F}{\mathcal{F}}
\renewcommand{\K}{\mathcal{K}}

\newcommand{\cO}{\mathcal{O}}
\newcommand{\cQ}{\mathcal{Q}}
\newcommand{\ad}{\mathrm{ad}}

\begin{document}

\tableofcontents

\section{Talk}

\subsection{Motivation: unirationality}

Unirationaltiy is the weakest notion of a polynomial system being ``solvable'' by polynomials. 

\begin{defn}
Let $X$ be a variety over $k$. We say that $X$ is \textit{unirational} (over $k$) if there is a dominant rational map $\P^n \rat X$.
\end{defn}

This says exactly that (most) points of $X$ can be parametrized by some rational functions. 

\begin{theorem}[Luroth]
If $X$ is a curve then $X$ is unirational iff $X$ has genus $0$.
\end{theorem}

\begin{theorem}[Castellunovo]
Let $X / \CC$ be a smooth proper surface. TFAE
\begin{enumerate}
\item $X$ is unirational
\item $H^0(X, \omega_X^{\ot 2}) = 0$ and $H^1(X, \CC) = 0$
\item $X \birat \P^2$
\end{enumerate}
\end{theorem}

Let's talk about how we prove varities are not unirational in characteristic zero.

\subsubsection{Characteristic Zero}

Proving that a variety is not unirational is often quite sublte. For example, even in characteristic zero there is no known example of a rationally connected variety where we can prove it is not unirational.
\par 
However, if the variety admits any sort of forms we can use these as an obstruction.

\begin{prop}
Let $k$ be of characteristic zero and $X$ a proper $k$-variety. Then if $\omega \in H^0(X, \Omega_X^{\ot m})$ is a nonzero form for $m > 0$ then $X$ is not unirational.
\end{prop}

\begin{proof}
Indeed, if $f : \P^n \rat X$ is a unirational parametrization with $n = \dim{X}$ then $f$ is generically \etale. Furthermore, since $X$ is proper $f$ is defined in codimension $1$ so $f^* \omega$ extends to a section of $H^0(\P^n, \Omega_{\P^n}^{\ot m})$ giving a contradiction. 
\end{proof}

This argument fails completely in positive characteristic because of the existence of inseparable maps for which $f^* \omega = 0$.

\begin{example}
Consider a surface $X / \FF_p$ defined as a smooth compactification of an equation of the form,
\[ z^p = f(x,y) \]
for some $f \in \FF_p[x,y]$. Then if we choose $f$ to have large degree $X$ may be of general type. However, notice that if we adjoin $p^{\text{th}}$-roots $s,t$ of $x$ and $y$ then the equation becomes
\[ z^p = f(s^p, t^p) = f(s,t)^p \]
therefore $k(X) \subset k(s,t)$ given by $z = f(s,t)$ and $x = s^p$ and $y = t^p$. This shows that $X$ is always uniratonal. 
\end{example}

This demonstrates that unirationality in positive characteristic is a phenmenon that persists even for quite ``complex'' varieties from the perspective of the usual classification of surfaces. 

\subsection{Shioda's Example}

Shioda used some very clever tricks and computations to compute the following.

\begin{theorem}[Shioda]
Let $F_n$ be the Fermat surface over $\FF_p$ ($p > 2$) given by the equation
\[ \{ X^n + Y^n + Z^n + W^n = 0 \} \subset \P^3 \]
Then $F_n$ is unirational if and only if $p^\nu \equiv -1 \mod n$ for some $\nu$.
\end{theorem}

\begin{example}
Let $G = \Z / 5 \Z$ act on $F_5$ by $\lambda \cdot [X : Y : Z : W] = [X : \lambda Y : \lambda^2 Z : \lambda^3 W]$. This action is free and hence $X = F_5 / G$ is a smooth projective surface with $\pi_1(X) = \Z / 5 \Z$ called the Godeaux surface. Shioda's result shows that this is supersingular over infinitely many primes. However, we saw this cannot happen over $\CC$, indeed $X$ has general type. 
\end{example}

How does Shioda prove this theorem: for the ``if'' part he uses an extremely clever change of variables to just write down the unirational parametrization. To show the ``only if'' he needs an obstruction, this comes from \etale cohomology. 

\begin{prop}
Let $X$ be a smooth projective unirational surface over $\FF_q$. Then $\Frob_q \acts H^2_{\et}(X, \QQ_\ell)$ has eigenvalues of the form $\zeta \cdot q$ for $\zeta$ a root of unity. We call this property \textit{supersingularity}. 
\end{prop}

\begin{proof}
Pass to an extension $\FF_{q^n}$ such that the unirational parametrization and all points of indeterminacy are defined over $\FF_{q^n}$. Resolving the indeterminacy locus, we get $\Bl_S \P^2 \to X$ where $S \subset \P^2$ is a finite set of points. Therefore $H^2_{\et}(X, \QQ_\ell) \embed H^2_{\et}(\Bl_S \P^2, \Q_\ell) = \Q_\ell(-1)^{1 + \# S}$. Thus $\Frob_{q^n} = \Frob_{q}^n$ acts via $q^n$.  
\end{proof}


\subsection{Jet Obstructions}


Here we describe one approach to obstructing unirationality. The idea is very simple: recall that in characteristic zero we may use forms to obstruct unirationality. Hover, the issue is that degree $p$ purely inseparable maps can kill first-order derivatives. The fundamental observation is that they cannot kill $p^{\text{th}}$-order derivatives. Therefore, if we can find higher-order forms on our varities then these will pullback nontrivially even along certain inseparable maps to produce obstructions. This idea can be formalized using various notions of jet bundles.

\begin{defn}
Let $X \to S$ be a smooth scheme. Then $J^n_{X/S} := \pi_{1*} \pi_2^* \struct{X}$ where the projections are along the $n^{\text{th}}$-order thickening of the diagonal $\Delta^n_{X/S} := V(\I_{\Delta}^{n+1}) \subset X \times_S X$. This satisfies the following universal property,
\[ \shHom{\struct{X}}{J^n_{X/S}}{\E} = \shDiff{\le n}{X}{\struct{X}}{\E} \]
\end{defn}

Now I define a notion of bigness for lifts of jets.

\newcommand{\jetamp}{\mathrm{jetamp}}

\begin{defn}
Let $\wt{J}^n_X := \ker{(J^n_X \to \struct{X})}$ using the natural projection $J^n_X \to J^0_X = \struct{X}$.
\end{defn}

\begin{defn}
Let $X$ be a smooth proper variety. Define the \textit{jet-amplitude} of $X$ to be
\[ \jetamp(X) := \sup \left \{ \frac{n}{r} : \exists \, \omega \in H^0(X, \wt{J}^{n}_X) \text{ such that } \omega|_{\wt{J}^{r}_X} \neq 0 \right\} \]
\end{defn}

\begin{theorem}[C]
Let $X$ be a smooth proper variety over a perfect field $k$. Suppose that $\log_p \jetamp(X) > \ell$ then there does not exist a unirational parametrization $\P^n \rat X$ of inseparability degree $\ell$.
\end{theorem}

\begin{proof}
There is a factorization of rational maps
\begin{center}
\begin{tikzcd}
X^{(1/q)} \arrow[r, dashed, "f'"] \arrow[rd, "F"'] & \P^n \arrow[d, dashed, "f"] 
\\
& X
\end{tikzcd}
\end{center}
where $F : X^{(1/q)} \to X$ is the $\ell^{\text{th}}$-relative Frobenius. 
Using extension in codimension $\ge 2$, there are well-defined pullback maps
\begin{center}
\begin{tikzcd}
H^0(X, \wt{J}^{s}_X) \arrow[r] \arrow[d, "f^*"] & H^0(X, \wt{J}^{r}_X) \arrow[d, "f^*"] 
\\
H^0(\P^n, \wt{J}^{s}_X) \arrow[d, "f'^*"] \arrow[r] & H^0(\P^n, \wt{J}^{r}) \arrow[d, "f'^*"] 
\\
H^0(X^{(1/q)}, \wt{J}^{s}_{X^{(1/q)}}) \arrow[r] & H^0(X^{(1/q)}, \wt{J}^{r}_{X^{(1/q)}}) 
\end{tikzcd}
\end{center} 
where the composition along column is $F^*$. Therefore, it suffices to show that $F^* \omega \neq 0$. To do this, we may work \etale-locally and reduce to the following computation: $J^r_{\A^n}$ is represented by the algebra,
\[ k[x_1, \dots, x_n][\d{x_1}, \dots, \d{x_n}] / (\d{x_1}, \dots, \d{x_n})^{r+1} \] and $F^* \d{x_i} = \d{x_i^{q}} = (\d{x_i})^{q}$. Therefore, this is nonzero if $s/r \ge q$.
\end{proof}

Unfortunately it is quite difficult to compute $\jetamp(X)$. For ``generic'' complete intersections of large degree, it is known that $\jetamp(X) = \infty$. However, for any particular example it is very hard to get ones hands on this thing. There is a trivial bound as follows.


\begin{prop}
Let $X$ be a smooth projective variety. Then,
\[ \jetamp(X) \ge \left(\frac{\ul{\hat{h}}^0(\Omega_X)}{\hat{h}^1(\Omega_X)}\right)^{\frac{1}{2 \dim{X}}} \]
where $\hat{h}^i$ these are so-called \textit{asymtotic cohomology}
\[ \hat{h}^i(X, \E) := \limsup_{m \to \infty} \frac{h^0(X, \nSym{m}{\E})}{m^{\dim{X} + \rank{\E} - 1}} \]
and likewise $\ul{\hat{h}}^i$ are the \textit{lower asymtotic cohomology}
\[ \ul{\hat{h}}^i(X, \E) := \liminf_{m \to \infty} \frac{h^0(X, \nSym{m}{\E})}{m^{\dim{X} + \rank{\E} - 1}} \]
\end{prop}


\begin{proof}
This arises from the exact sequence
\begin{center}
\begin{tikzcd}
0 \arrow[r] & \nSym{n}{\Omega_X} \arrow[r] & J^n_X \arrow[r] & J^{n-1}_X \arrow[r] & 0 
\end{tikzcd}
\end{center}
\end{proof}

\begin{example}
Suppose $h^0(X, \nSym{2}{\Omega_X}) \ge h^1(X, \nSym{3}{\Omega_X}) + h^1(X, \nSym{4}{\Omega_X})$ then $X$ is not a Zariski space in characteristic $2$.
\end{example}

\subsection{Hyperbolicity}

\begin{defn}
A complex variety $X$ is \textit{hyperbolic} if ever entre map $f : \CC \to X$ is constant.
\end{defn}

In particular, this says that $X$ does not contain any rational or elliptic curves. It is expected that general type varities are \textit{close} to being hyperbolic in the following precise sense.

\begin{conj}[Green-Griffiths-Lang]
Let $X$ be a smooth projective complex surface of general type. Then there exists a proper Zariski closed subset $Z \subsetneq X$ such that any entire curve $f : \CC \to X$ has image inside $Z$.
\end{conj}

The remarkable feature of this conjecture is that we expect every entire curve to be \textit{algebraically degenerate} meaning it satisfies a polynomial relation (i.e. it lives in $Z$). Moreover, we expect that these relations can be chosen independently of the map! 
\bigskip\\
The strategy developed by Green, Griffiths, Demailly and many others towards this conjecture is to produce algebraic differential equations that entire curves must satisfy. We can think of a first-order differential equation as a closed subspace of the tangent bundle of $X$. If we could produce enough differential equations so that the intersection of these loci in $T X$ does not dominate $X$ then we win. 

\subsection{Induced Foliations}

Suppose that $\omega \in H^0(P_k X, \struct{P_k X}(m))$ is a Demailly-Semple jet. Then the vanishing locus $V(\omega) \subset P_k X$ is pure codimension $1$. For each irreducible component $Z$, consider the following subbundle of the tangent bundle,
\[ \F := \ker{(\T_Z \oplus \struct{P_k X}(-1)|_Z \to \pi^* \T_{P_{k-1}X} |_Z)} \]
Since $X$ is a surface, $\F$ is generically rank $1$ and therefore automatically closed under Lie bracket i.e. it defines a foliation. {\color{red} CHECK THIS} Since every curve $f : C \to X$ lifts to a curve $f_k : C \to P_k X$ parallel to the directed structure (meaning its differential $\d{f_k} : \T_C \to \T_{P_k X}$ factors through $\E_k$)

\subsection{$p$-Curvature}

\begin{defn}
Let $k$ be a field of characteristic $p$ and $\partial : A \to A$ a derivation. Then there is a well-defined derivation $\partial^p$ given by 
\[ x \mapsto \underbrace{\partial \cdots \partial}_{p \text{ times}} x \]
This defines a nonlinear map $\T_X \to \T_X$. Given an involutive subsheaf $\F$ the $p$-\textit{curvature} map 
\[ \psi_p : \Frob^* \F \to \T_X / \F \]
is a \textit{linear} $\struct{X}$-module map measuring the failure of $\F$ to be $p$-closed.
\end{defn}

\begin{proof}
To show $\psi_p(\partial_1 + \partial_2)$ we use that
\[ (\partial_1 + \partial_2)^p = \partial_1^p + \partial_2^p + \sum_{i = 1}^{p-1} s_i(\partial_1, \partial_2) \]
but the $s_i$ are Lie polynomials so $s_i(\partial_1, \partial_2) \in \F$ because $\F$ is clsoed under Lie bracket.
We need to show that $\psi_p(f \partial) = f^p \psi_p(\partial)$. Indeed, 
\[ (f \partial)^p = f^p \partial^p + f \ad_{\partial}^{p-1}(f^{p-1}) \partial \]
and because $\ad_{\partial}$ is just the action of $\partial$ on a function so the second term lies in $\F$.
\end{proof}



\begin{prop}
Suppose that $f : C \to X$ is a map from a curve to a foliated variety $(X, \F)$ meaning that $\d{f} : \T_C \to f^* \F \to f^* \T_X$. Then $f^* \psi_p = 0$. Hence if $X$ is a surface and $\F$ is a foliation by curves then $\im{f} \subset \Delta_p := V(\psi_p)$.
\end{prop}


\begin{theorem}
Let $X$ be a smooth projective surface over $k$ a field of characteristic $p$. Suppose that $\omega \in H^0(X, \nSym{n}{\Omega_X})$ is a nonzero symmetric form and $\F_{\omega}$ is the foliation induced on $V(\omega)$. If $\F_{\omega}$ is \textit{not} $p$-closed (on each irreducible component) then $X$ has finitely many rational curves.
\end{theorem}

\begin{proof}
Indeed, $V(\omega) \subset \P(\Omega_X)$ is pure codimension $1$. The induced foliation $\F_\omega$ is assumed to be not $p$-closed on each irreducible component $Z \subset V(\omega)$ therefore $\psi_p : \Frob^* \F \to \T_Z/\F$ is a nonzero map of generically rank $1$-torsion-free sheaves. Hence its zero locus is not dense in $Z$ and hence is dimension $1$. Therefore, the union of these images in $X$ is a finite union of curves. However, we have seen that any rational curve on $X$ must lie inside this locus.
\end{proof}

\begin{example}
Consider two curves $C_1, C_2$ defined as $A_6$-covers of $\P^1$ branched over three points with monodromies
\begin{enumerate}
\item $(1 6)(3 4) \quad (25436) \quad (16452)$
\item $(123)(456) \quad (125) \quad (1465)(23)$
\end{enumerate}
Let $X$ be the minimal resolution of $(C_1 \times C_2)/A_6$. This is a minimal surface with $c_1(X)^2 = c_2(X) = 6$ and $\pi_1(X) = A_4 \times \Z / 5 \Z$ which is finite. Furthermore, $h^{2,0} = 0$ so the cohomology of $X_{\CC}$ is generated by algebraic cycles by the Lefschetz $(1,1)$ theorem. Therefore, $X$ is supersingular when reduced mod any prime of good reduction. From the main two invariants we cannot see that $X$ is not unirational except at $p = 2,3,5$. However, one can show that $\Omega_X$ is big. Choosing a symmetric form $\omega$ we get a foliation $\F_\omega$ and whenever $\F_{\omega}$ is not $p$-closed $X$ is not unirational. The Grothendieck $p$-curvature conjecture predicts this happens infinitely often. One can put this on a computer and start generating a list of primes where $X$ is not unirational.  
\end{example}

\section{Adendum}

\begin{prop}
Let $X / \CC$ be a smooth proper variety. If $X$ is unirational then $\pi_1(X) = 0$.
\end{prop} 


\begin{proof}
Let $f : \P^n \sm Z \to X$ be a unirational parametrization and we can choose $\codim{Z} \ge 2$. Thus $\pi_1(\P^n \sm Z) = 1$ so the map lifts to the universal cover $\wt{X} \to X$ as a holomorphic map. However $f$ is generically a finite cover so this forces $\wt{X} \to X$ to be a finite cover also. Hence $\wt{X}$ is also a smooth proper variety. Now consider the diagram,
\begin{center}
\begin{tikzcd}
Y \arrow[r] \arrow[d] \pullback & \wt{X} \arrow[d]
\\
\P^n \sm Z \arrow[r] & X 
\end{tikzcd}
\end{center}
since $\pi_1(\P^n \sm Z) = 1$ we see that $Y$ is a finite union of copies of $\P^n \sm Z$ and hence $\wt{X}$ is also unirational. Therefore we showed that $h^0(\wt{X}, \Omega_{\wt{X}}^p) = 0$ so by hodge symmetry $h^p(\wt{X}, \struct{\wt{X}}) = 0$ so $\chi(\wt{X}) = 1$. However, it is well-known that if $\wt{X} \to X$ has degree $n$ then $\chi(\struct{\wt{X}}) = n \chi(\struct{X})$ hence $n = 1$. 
\end{proof}


\begin{lemma}
Let $f : X \to Y$ be surjection of smooth proper varieties. Then $f^* : H^i_{\et}(Y, \Q_\ell) \embed H^i_{\et}(X, \Q_\ell)$ is injective.
\end{lemma}

\begin{proof}
Poincare duality defines a pushforward map $f_* = D^{-1} \circ (f_*)^{\vee} \circ D$ where $D(\alpha) = \tr{(\alpha \smile -)}$. If $f$ is generically finite, it suffices to show that $f_* f^* = \deg{f}$ since this is invertible on $\Q_\ell$. Indeed, 
\[ D(f_* f^* \alpha) = \tr{(f^* \alpha \smile f^* - )} = \tr{f^* (\alpha \smile -)} = (\deg{f}) D(\alpha) \]
because $f^* = \deg{f}$ on top cohomology. If $f$ is not generically finite then we may take a generic multisection $Z \subset X \to Y$ and then $(f \circ \iota)^*$ is injective so $f^*$ is injective.
\end{proof}

\subsection{Shioda's Example}

Shioda used some very clever tricks and computations to compute the following.

\begin{theorem}[Shioda]
Let $F_n$ be the Fermat surface over $\FF_p$ ($p > 2$) given by the equation
\[ \{ X^n + Y^n + Z^n + W^n = 0 \} \subset \P^3 \]
Then $F_n$ is unirational if and only if $p^\nu \equiv -1 \mod n$ for some $\nu$.
\end{theorem}

\begin{example}
Let $G = \Z / 5 \Z$ act on $F_5$ by $\lambda \cdot [X : Y : Z : W] = [X : \lambda Y : \lambda^2 Z : \lambda^3 W]$. This action is free and hence $X = F_5 / G$ is a smooth projective surface with $\pi_1(X) = \Z / 5 \Z$ called the Godeaux surface. Shioda's result shows that this is supersingular over infinitely many primes. However, we saw this cannot happen over $\CC$, indeed $X$ has general type. 
\end{example}

How does Shioda prove this theorem: for the ``if'' part he uses an extremely clever change of variables to just write down the unirational parametrization. To show the ``only if'' he needs an obstruction, this comes from \etale cohomology. 

\begin{prop}
Let $X$ be a smooth projective unirational surface over $\FF_q$. Then $\Frob_q \acts H^2_{\et}(X, \QQ_\ell)$ has eigenvalues of the form $\zeta \cdot q$ for $\zeta$ a root of unity. We call this property \textit{supersingularity}. 
\end{prop}

\begin{proof}
Pass to an extension $\FF_{q^n}$ such that the unirational parametrization and all points of indeterminacy are defined over $\FF_{q^n}$. Resolving the indeterminacy locus, we get $\Bl_S \P^2 \to X$ where $S \subset \P^2$ is a finite set of points. Therefore $H^2_{\et}(X, \QQ_\ell) \embed H^2_{\et}(\Bl_S \P^2, \Q_\ell) = \Q_\ell(-1)^{1 + \# S}$. Thus $\Frob_{q^n} = \Frob_{q}^n$ acts via $q^n$.  
\end{proof}

Shioda then computes that $F_n$ is supersingular exactly when $p^\nu \equiv - 1 \mod p$ for some $\nu$ and this covers all the cases his construction works to show $F_n$ is moreover unirational. This, and a handful of other examples, leads him to make the following conjecture.

\begin{conj}[Shioda]
Let $X$ be a smooth projective surface over $\FF_q$. Suppose that
\begin{enumerate}
\item $\pi_1(X) = 1$
\item $X$ is supersingular
\end{enumerate}
then $X$ is unirational.
\end{conj}

\begin{rmk}
One might ask why doesn't Shioda just require that $\pi_1(X)$ is finite as suggested by Serre's theorem and the Godeaux surface example: this conjecture would be false because it turns out the Godeaux surface is supersingular in every characteristic (of good reduction) but it unirational only in the characteristics that $F_5$ is unirational in.
\end{rmk}

The goal of this talk is to describe new obstructions to unirationality that could be used to test Shioda's conjecture.

\subsection{Proof of the Jet amplitdue Theorem}

\begin{theorem}[C]
Let $X$ be a smooth proper variety over a perfect field $k$. Suppose that $\log_p \jetamp(X) > \ell$ then there does not exist a unirational parametrization $\P^n \rat X$ of inseparability degree $\ell$.
\end{theorem}

\begin{proof}
Since $\jetamp$ only increases anlong separable maps we may assume that $X$ has a purely inseparable unirational parametrization $f : \P^n \rat X$ of degree $k$. Now suppose that $\jetamp \ge p^{\ell} := q$ then there are integers $r,s \ge 0$ such that $\omega \in H^0(X, \wt{J}_X^{s})$ such that $\omega|_{\wt{J}^{r}} \neq 0$ and $s \ge q r$. By decreasing $r$ we may assume that $\omega|_{\wt{J}^{r-1}} = 0$ so $\omega|_{\wt{J}^{r}} \in H^0(X, \nSym{r}{\Omega_X})$ then decreasing $s$ we may assume $s = q r$. There is a factorization of rational maps
\begin{center}
\begin{tikzcd}
X^{(1/q)} \arrow[r, dashed, "f'"] \arrow[rd, "F"'] & \P^n \arrow[d, dashed, "f"] 
\\
& X
\end{tikzcd}
\end{center}
where $F : X^{(1/q)} \to X$ is the $\ell^{\text{th}}$-relative Frobenius. 
Using extension in codimension $\ge 2$, there are well-defined pullback maps
\begin{center}
\begin{tikzcd}
H^0(X, \wt{J}^{s}_X) \arrow[r] \arrow[d, "f^*"] & H^0(X, \wt{J}^{r}_X) \arrow[d, "f^*"] 
\\
H^0(\P^n, \wt{J}^{s}_X) \arrow[d, "f'^*"] \arrow[r] & H^0(\P^n, \wt{J}^{r}) \arrow[d, "f'^*"] 
\\
H^0(X^{(1/q)}, \wt{J}^{s}_{X^{(1/q)}}) \arrow[r] & H^0(X^{(1/q)}, \wt{J}^{r}_{X^{(1/q)}}) 
\end{tikzcd}
\end{center} 
where the composition along column is $F^*$. Therefore, it suffices to show that $F^* \omega \neq 0$. To do this, we may work \etale-locally. Indeed, if $c : U \to X$ is an \etale chart then it suffices to show that $F_U^* \omega|_U \neq 0$ since the diagram
\begin{center}
\begin{tikzcd}
U^{(1/q)} \arrow[r, "F_U"] \arrow[d] & U \arrow[d]
\\
X^{(1/q)} \arrow[r, "F"] & X 
\end{tikzcd}
\end{center}
commutes. Since $X$ is smooth, there are \etale charts $U \to X$ with $g : U \to \A^n_k$ \etale. The diagram, 
\begin{center}
\begin{tikzcd}
\wt{J}^s_{\A^n} \arrow[d, "F^*"] \arrow[r] & \wt{J}^r_{\A^n} \arrow[d, "F^*"] \arrow[ld, "t", dashed]
\\
\wt{J}^s_{\A^n} \arrow[r] & \wt{J}^r_{\A^n}
\end{tikzcd}
\end{center}
has a commuting lift $t$. Indeed, $J^r_{\A^n}$ is represented by the algebra,
\[ k[x_1, \dots, x_n][\d{x_1}, \dots, \d{x_n}] / (\d{x_1}, \dots, \d{x_n})^{r+1} \] and $F^* \d{x_i} = \d{x_i^{q}} = (\d{x_i})^{q}$. Therefore, the kernel of $\wt{J}^s_{\A^n} \to \wt{J}^r_{\A^n}$ are polynomials $h(\d{x_1}, \dots, \d{x_n})$ with coefficients in $k[x_1, \dots, x_n]$ of minimal degree $\ge r + 1$ and hence $F^* h = h(\d{x_1^{q}}, \dots, \d{x_n^{q}}) = 0$ in $J^s_{\A^n}$ since it has minimal degree $\ge q (r + 1) \ge s + 1$. Furthermore, consider a symmetric form $\eta \in H^0(\A^n, \nSym{r}{\Omega})$ write
\[ \eta = \sum_{I} f_I(x_1, \dots, x_n) \d{x_{i_1}} \cdots \d{x_{i_r}} \]
then we see
\[ t(\eta) = \sum_I f_I(x_1^q, \dots, x_n^q) \d{x_{i_1}^q} \cdots \d{x_{i_r}^q} \in H^0(\A^n, \nSym{s}{\Omega}) \]
but for different $I \subset \{1, \dots, n\}$ of size $r$, these are distinct basis elements of $H^0(\A^n, \nSym{s}{\Omega})$. Therefore, $t(\eta) = 0$ if and only if all $f_I(x_1^q, \dots, x_n^q) = 0$ if and only if all $f_I = 0$. Hence $t$ is injective. Since $g$ is \etale, we get isomorphisms $g^* \wt{J}_{\A^n}^r \iso \wt{J}^r_{U}$ for all $r$ so the diagram pulls back to
\begin{center}
\begin{tikzcd}
\wt{J}^s_{U} \arrow[r] \arrow[d, "F^*_U"] & \wt{J}^r_{U} \arrow[d, "F^*_U"] \arrow[dl, "t_U"]
\\
\wt{J}^s_{U} \arrow[r] & \wt{J}^r_{U}
\end{tikzcd}
\end{center}
such that the restriction $t_U : \nSym{r}{\Omega_U} \to \nSym{s}{\Omega_U}$ is injective. Therefore $F^*_U \omega|_U = t(\omega|_{\wt{J}^r_U}) \neq 0$ because we assumed that $\omega|_{\wt{J}^r} \in H^0(X, \nSym{r}{\Omega_X})$ and is nonzero.
\end{proof}

\begin{prop}
Let $X$ be a smooth projective variety. Then,
\[ \jetamp(X) \ge \left(\frac{\ul{\hat{h}}^0(\Omega_X)}{\hat{h}^1(\Omega_X)}\right)^{\frac{1}{2 \dim{X}}} \]
where $\hat{h}^i$ these are so-called \textit{asymtotic cohomology}
\[ \hat{h}^i(X, \E) := \limsup_{m \to \infty} \frac{h^0(X, \nSym{m}{\E})}{m^{\dim{X} + \rank{\E} - 1}} \]
and likewise $\ul{\hat{h}}^i$ are the \textit{lower asymtotic cohomology}
\[ \ul{\hat{h}}^i(X, \E) := \liminf_{m \to \infty} \frac{h^0(X, \nSym{m}{\E})}{m^{\dim{X} + \rank{\E} - 1}} \]
\end{prop}


\begin{proof}
This arises from the exact sequence
\begin{center}
\begin{tikzcd}
0 \arrow[r] & \nSym{n}{\Omega_X} \arrow[r] & J^n_X \arrow[r] & J^{n-1}_X \arrow[r] & 0 
\end{tikzcd}
\end{center}
Therefore, 
\[ \dim \im{(H^0(X, J^{rs}_X) \to H^0(X, J^r_X))} \ge h^0(X, J^r_X) - \sum_{k = r+1}^{rs} \left[h^1(X, \nSym{k}{\Omega_X}) \right] \]
but also
\[ h^0(X, J^r_X) \ge \sum_{i = 1}^r \left[ h^0(X, \nSym{i}{\Omega_X}) - h^1(X, \nSym{i}{\Omega_X}) \right] \] 
therefore
\[ \dim \im{(H^0(X, J^{rs}_X) \to H^0(X, J^r_X))} \ge \sum_{i = 1}^r h^0(X, \nSym{i}{\Omega_X}) - \sum_{i = 1}^{rs} h^1(X, \nSym{i}{\Omega_X}) \]
Applying the lemma, for any $\epsilon > 0$ for $r \gg_{\epsilon} 0$ we have
\[ \sum_{i = 1}^r h^0(X, \nSym{i}{\Omega_X}) - \sum_{i = 1}^{rs} h^1(X, \nSym{i}{\Omega_X}) \ge \frac{1}{2 \dim{X}} \left[ (\ul{\hat{h}}^0 - \epsilon) r^{2 \dim{X}} - (\hat{h}^1 + \epsilon) (rs)^{2 \dim{X}} \right] \]
Thus,
\[ s^{2 \dim{X}} \le \frac{\ul{\hat{h}}^0 - \epsilon}{\hat{h}^1 + \epsilon} \implies \jetamp(X) \ge s  \]
\end{proof}

\begin{lemma}
Let $f : \N \to \RR^+$ be an integer function. Suppose that
\[ \liminf_{n \to \infty} \frac{f(n)}{n^k} = a \quad \text{ and } \quad \limsup_{n \to \infty} \frac{f(n)}{n^k} = b \]
then\footnote{Notice that the outer inequalities are not usually equalities. For example let 
\[ f(n) = \begin{cases} 0 & n \text{ even} \\ 1 & n \text{ odd} \end{cases} \]
then the hypotheses are satisfies with $k = 0$ and $a = 0$ and $b = 1$ but
\[ \sum_{i = 1}^n f(i) = \ceil{\frac{n}{2}} \] 
and therefore the limit
\[ \lim_{n \to \infty} \frac{1}{n} \sum_{i = 1}^n f(i) = \frac{1}{2} \]
exists so the inequalities in the conclusion of the lemma are strict. }
\[ \frac{b}{k+1} \ge \limsup_{n \to \infty} \frac{1}{n^{k+1}} \sum_{i = 1}^n f(i) \ge \liminf_{n \to \infty} \frac{1}{n^{k+1}} \sum_{i = 1}^n f(i) \ge \frac{a}{k+1}  \]
\end{lemma}

\begin{proof}
The assumptions give for all $\epsilon > 0$ there is $n_\epsilon$ such that $n \ge n_\epsilon$ implies 
\[ (b + \epsilon) n^k \ge f(n) \ge (a - \epsilon) n^k \]
summing over this we get,
\[ (b + \epsilon) \frac{1}{n^{k+1}} \sum_{i = 1}^n i^k \ge \frac{1}{n^{k+1}} \sum_{i = n_\epsilon}^n f(i) + \frac{1}{n^{k+1}} \sum_{i = 1}^{n_\epsilon} i^k \ge (a - \epsilon) \frac{1}{n^{k+1}} \sum_{i = 1}^n i^k \]
Therefore,
\[ \frac{b + \epsilon}{k + 1} (1 + O(n^{-1})) \ge \frac{1}{n^{k+1}} \sum_{i = 1}^n f(i) + \frac{1}{n^{k+1}} \sum_{i = 1}^{n_\epsilon} (i^k - f(i)) \ge \frac{a - \epsilon}{k+1} (1 + O(n^{-1}))  \]
Taking the limit $n \to \infty$ and then $\epsilon \to 0$ gives the bounds
\[ \frac{b}{k+1} \ge \limsup_{n \to \infty} \frac{1}{n^{k+1}} \sum_{i = 1}^n f(i) \ge \liminf_{n \to \infty} \frac{1}{n^{k+1}} \sum_{i = 1}^n f(i) \ge \frac{a}{k+1}  \]
Furthermore, for any $\epsilon > 0$ and $n_0$ there is $n \ge n_0$ such that $(a + \epsilon) n^k \ge f(n)$ 
\end{proof}

\section{p-curvature computations}

\newcommand{\Lie}{\mathrm{Lie}}
\newcommand{\g}{\mathfrak{g}}

\begin{lemma}
Let $R$ be a (possibily noncommutative) associative ring of characteristic $p$. Then,
\begin{enumerate}
\item there are universal Lie polynomials $s_i(x,y) \in R \left< x, y \right>$ such that
\[ (a + b)^p = a^p + b^p + \sum_{i = 1}^{p-1} s_i(a,b) \]
where $s_i$ is defined by the relation,
\[ (\ad_{a t + b})^{p-1}(a) = \sum_{i = 1}^{p-1} i s_i(a,b) t^{i-1} \]
\item for all $g, \theta \in R$ such that $\{ \ad_{\theta}^n(g^m) \}_{n,m \ge 0}$ commute with each other,
\[ (g \theta)^p = g^p \theta^p + g \cdot \ad_{\theta}^{p-1}(g^{p-1}) \theta \]
\end{enumerate}
\end{lemma}

\begin{rmk}
$s_i(x,y) \in \Lie(x,y)_{p-1}$ for the $(p-1)^{\text{th}}$-stage of the lower central series $\g_{i+1} = [\g_i, \g]$.
\end{rmk}


\begin{example}
For example, if $p = 2$ then $(a + b)^2 = a^2 + ab + ba + b^2$ so $s_1(a,b) = ab + ba = [a,b]$ because the characteristic is 2.
\end{example}


\begin{lemma}
Let $A,B$ be (possibily noncommutative) associative rings of characteristic $p$ and $M$ is an $(A,B)$-bimodule. Suppose that for some $a \in A$ and $m \in M$,
\[ a \cdot m = \sum_i f_i \cdot m \cdot b_i \]
such that the $f_i \in A$ commute with each other and $b_i \in B$ then
\[ a^p \cdot m - \sum_i f_i^p \cdot m \cdot b_i^p \in \left< r \cdot m \cdot s \mid r \in \ZZ[f_1, \dots, f_r, \Lie(a, f_1, \dots, f_r)_{p-1}] \text{ and } s \in \Lie(b_1, \dots, b_r)_{p-1} \right>  \] 
where the $1$ indicates the first stage of the lower central series.
\end{lemma}


\begin{proof}
$M$ is an $A \ot B^\op$-module and $(a \ot 1 - \sum_i f_i \ot b_i)$ acts as zero. Furthermore,
\[ (a \ot 1 - \sum_i f_i \ot b_i)^p = a^p \ot 1 + \left(- \sum_i f_i \ot b_i \right)^p + \sum_{j = 1}^{p-1} s_j(a \ot 1, -\sum_i f_i \ot b_i) \]
so the last term is in the span of $\Lie(a, f_1, \dots, f_r)_{p-1} \ot b_i$. By Jacobson again,
\[  \left(- \sum_i f_i \ot b_i \right)^p = - \sum_i f_i^p \ot b_i^p + g \]
where $g$ is in the span of $f_i \ot \Lie(b_1, \dots, b_r)_{p-1}$. Therefore,
\[ a^p \cdot m - \sum_i f_i^p \cdot m \cdot b_i^p = -\sum_{j = 1}^{p-1} s_j(a \ot 1, \sum_i f_i \ot b_i) \cdot m  -g \cdot m \]
\end{proof}

\begin{prop}
Suppose that $f : C \to X$ is a map from a curve to a foliated variety $(X, \F)$ meaning that $\d{f} : \T_C \to f^* \F \to f^* \T_X$. Then $f^* \psi_p = 0$. Hence if $X$ is a surface and $\F$ is a foliation by curves then $\im{f} \subset \Delta_p := V(\psi_p)$.
\end{prop}

\begin{proof}
Indeed, this follows from the following lemma.
\end{proof}

\begin{prop}
Let $f : (X, \F_X) \to (Y, \F_Y)$ be a map of foliated varities. Then the diagram
\begin{center}
\begin{tikzcd}
\Frob^* \F_X \arrow[r, "\psi_p"] \arrow[d] & \T_X / \F_X \arrow[d]
\\
f^* \Frob^* \F_Y \arrow[r, "f^* \psi_p"] & f^* (\T_Y / \F_Y) 
\end{tikzcd}
\end{center}
of $\struct{X}$-linear maps commutes.
\end{prop}

\begin{proof}
This is a local statement so we reduce to a ring map $\phi : A \to B$ with a diagram,
\begin{center}
\begin{tikzcd}
\Der[R]{B}{B} \arrow[r, "\partial \mapsto \partial \circ \phi"] & \Der[R]{A}{B} 
\\
M \arrow[u] \arrow[r, "\kappa"] & B \ot_A N \arrow[u] 
\end{tikzcd}
\end{center}
Suppose that $\partial \in M$ we need to show that $\psi_p(\partial) \circ \phi - \psi_p(\kappa(\partial)) \in B \ot_A N$. Indeed, let
\[ \kappa(\partial) = \sum_i b_i \ot \partial_i \]
for $\partial_i \in N$. Then 
\[ \psi_p(\Frob^* \kappa(\partial)) = \psi_p \left(\sum_i b_i^p \ot \partial_i \right) = \sum_i b_i^p \ot \partial_i^p \]
We apply the lemma to $B \ot_A \D_A$ as a $(\D_B, \D_A)$-bimodule and $\kappa$ shows that $\partial \cdot 1 = \sum_i b_i \cdot 1 \cdot \partial_i$. Therefore,
\[ \kappa(\psi_p(\partial)) - \psi_p(\Frob^* \kappa(\partial)) \]
is in the submodule spanned by elements of the form $r \ot s$ for $r$ generated by iterated application of $\partial$ to the $b_i$ and products of $b_i$ and $s$ is generated by commutators of $\partial_i$ which lie in $N$ so we see this is zero in $B \ot_A (\Der[R]{A}{A} / N)$.
\end{proof}

\subsection{Semple Jets}

\begin{defn}
A \textit{directed variety} $(X, \E)$ is a pair of a variety $X$ with a subbundle $\E \subset \T_X$. A morphism of directed varities $f : (X, \E) \to (Y, \E')$ is a morphism $f : X \to Y$ such that under $f_* \T_X \to \T_Y$ we have $f_* \E \to \E'$.
\end{defn}

\begin{rmk}
Demailly's philosophy is that it is usefull to study this ``relative notion'' even for the absolute case $\E = \T_X$ since it has better functoriality properties.
\end{rmk}

\begin{rmk}
Here our convention is that $\P(\E) := \rProj{X}{\Sym{}{\E^\vee}}$ so that $\cO(-1)$ is the universal subbundle. Hence $\cO(1)$ on $\P(\T_X)$ is what I usually call $\cO(1)$ on $\P(\Omega_X)$.
\end{rmk}

\begin{defn}
To a directed pair $(X, \E)$ we introduce the \textit{projectivization} to produce a new pair $\P(X, \E) := (\wt{X}, \wt{\E})$ where $\wt{X} := \P(\E)$ and $\wt{\E}$ is defined via the diagram,
\begin{center}
\begin{tikzcd}[row sep = small, column sep = large]
0 \arrow[r] & \T_{\wt{X}/X} \arrow[dd] \arrow[r] & \wt{\E} \arrow[dd] \pullback \arrow[r] & \cO(-1) \arrow[d, hook] \arrow[r] & 0
\\
& & & \pi^* \E \arrow[d, hook]
\\
0 \arrow[r] & \T_{\wt{X}/X} \arrow[r] & \T_{\wt{X}} \arrow[r] & \pi^* \T_{X} \arrow[r] & 0
\end{tikzcd}
\end{center}
Then we have,
\[ \dim{\wt{X}} = \dim{X} + \rank{\E} - 1 \quad \quad \rank{\wt{\E}} = \rank{\E} \] 
\end{defn}

\begin{rmk}
Note that the Euler exact sequence takes the form,
\begin{center}
\begin{tikzcd}
0 \arrow[r] & \cO \arrow[r] & \pi^* \E \ot \cO(1) \arrow[r] & \T_{\wt{X}/X} \arrow[r] & 0
\end{tikzcd}
\end{center}
\end{rmk}

\begin{prop}
Given a morphism of directed varities $f : (X, \E) \to (Y, \F)$ we get a rational map $\wt{f} : (\wt{X}, \wt{\E}) \rat (\wt{Y}, \wt{\F})$ such that the diagram,
\begin{center}
\begin{tikzcd}
(\wt{X}, \wt{\E}) \arrow[d,"\pi"] \arrow[r, "\wt{f}", dashed] & (\wt{Y}, \wt{\F}) \arrow[d, "\pi"]
\\
(X, \E) \arrow[r, "f"] & (Y, \F)
\end{tikzcd}
\end{center} 
commutes in the category of directed manifolds (with rational maps). Moreover, if $f$ is ``immersive along $\E$'', meaning $f_{\#} : \E \to f^* \F$ is injective, then $\wt{f}$ is a morphism.
\end{prop}

\begin{defn}
Let $(X, V)$ be a directed manifold. The \textit{projectivized Semple $k$-jet bundle} $P_k V = X_k$ is defined iteratively via,
\[ (X_0, V_0) := (X, V) \quad \quad (X_{k+1}, V_{k+1}) := (\wt{X_k}, \wt{V_k}) \]
and we have,
\[ \dim{P_k V} = \dim{X} + k (\rank{V} - 1) \quad \quad \rank{V_k} = \rank{V} \]
\end{defn}


The semple tower is defined so that the following holds. Suppose that $f : C \to X$ is an immersed curve such that $\d{f} : \T_C \to f^* \T_X$ factors through $f^* \E \subset f^* \T_X$. Since $\d{f}$ is a subbundle this gives a subbundle $\T_X \embed \pi^* \E$ and hence a lift $f' : C \to \wt{X}$ such $\d{f} : \T_C \to f^* \E \to f^* \T_X$ is $f'^* [\struct{\wt{X}}(-1) \to \pi^* \E \to \pi^* \T_X]$. Therefore, consider $\d{f'} : \T_C \to f'^* \T_{\wt{X}}$. Since this map lifts $\d{f}$ we see that $\d{f'} : \T_X \to f'^* \wt{\E}$.

Hence, if we start with an immersed curve $f : C \to X$ then there are lifts $f_k : C \to P_k$ for all $k$. The fundamental property is:

\begin{prop}
Let $f : \P^1 \to X$ be a rational curve. Then the lift $f_k : \P^1 \to P_k X$ lies in the base locus of $H^0(P_k X, \struct{P_k X}(m))$ for all $m > 0$.
\end{prop}

\begin{proof}
Analogous to the statement that for any symmetric form $\omega \in H^0(X, \nSym{n}{\Omega_X})$ we must have $f^* \omega = 0$ because there are no global pluri-forms on $\P^1$.
\end{proof}

The following result gives some hope of proving the GGL conjecture:

\begin{theorem}[Green-Griffiths, Demailly]
Let $(X, \E)$ be a smooth projective directed variety that lifts to characteristic zero. Suppose that $\det{\E^\vee}$ is big. Then
\[ H^0(P_k(X, \E), \struct{}(m)) \]
has (many) nonzero sections for $m \gg k \gg 1$.
\end{theorem}

\end{document}
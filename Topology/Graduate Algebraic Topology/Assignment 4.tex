\documentclass[12pt]{extarticle}
\usepackage{import}
\import{./}{Includes}
\newcommand{\C}{\mathbb{C}}
\newcommand{\F}{\mathbb{F}}
\newcommand{\Vect}{\mathrm{Vect}}

\begin{document}
\atitle{4}

\section{Maps of Hopf Invariant One}

Recall that the Hopf invariant is a integer $h(f) \in \Z$ defined for maps $f : S^{2n - 1} \to S^n$ as follows.

\begin{defn}
Let $f : S^{2n - 1} \to S^n$ be a continuous map. Then consider $C_f = D^{2n} \cup_f S^n$. Choosing generators we have $H^n(C_f ; \Z) = \alpha \Z$ and $H^{2n}(C_f; \Z) = \beta \Z$. Then,
\[ \alpha^2 \in H^{2n}(C_f ; \Z) \implies \alpha^2 = h(f) \beta \] 
\end{defn} 

\begin{rmk}
Notice that when $n$ is odd $\alpha^2 = \alpha \smile \alpha = 0$ since $\alpha$ has odd degree. Therefore, we may restrict ourself to considering maps $f : S^{4n - 1} \to S^{2n}$. 
\end{rmk}

\begin{prop}
The Hopf invariant gives a homomorphism $h : \pi_{2n - 1}(S^n) \to \Z$ with the following properties,
\begin{enumerate}
\item if $n$ is odd then $h = 0$ (since $\alpha \smile \alpha = 0$ in odd $n$).
\item for the Hopf fibration $H : S^3 \to S^2$ then $C_f = S^2 \cup_H D^4 = \CP^2$ and $H^*(\CP^2; \Z) = \Z[x]/(x^3)$ so the generator of $H^2(\CP^2; \Z)$ squares to the generator of $H^4(\CP^2; \Z)$ which implies that $h(H) = 1$. In particular, $h : \pi_3(S^2) \xrightarrow{\sim} \Z$ sending $H \mapsto 1$. 
\end{enumerate}
\end{prop}
\noindent\\
Our main result is the following.

\begin{theorem}
For all $n$, there exists a map $f : S^{4n - 1} \to S^{2n}$ with Hopf invariant: $h(f) = 2$. 
\end{theorem}
\noindent\\
To prove this theorem, we consider the following spaces.

\subsection{The James Restricted Product}

\begin{defn}
Let $(X, e)$ be a based topological space. Define the \textit{James restriced product} as the following quotient space,
\[ J_k(X) = X^k / \sim \]
where we identify $(x_1, \dots, x_i, e, \dots, x_k) \sim (x_1, \dots, e, x_i, \dots, x_k)$. Furthermore, we can define the total James space, $J(X) = \varinjlim J_m(X)$. 
\end{defn}

\begin{example}
We have $J_1(X) = X$ and $J_2(X) = X \times X / (x, e) \sim (e, x)$. 
\end{example}
\noindent\\
When $X$ is a CW complex, $J_m(X)$ inherents a CW complex structure from the product CW structure on $X$. Explicitly, we glue together the subcomplexes with one coordinate fixed at $e$. These James restricted products are especially interesting for us in the case of spheres in which case the cohomology is particularly easy to understand.

\begin{theorem}
Fix even $n > 0$. Then $H^p(J(S^n); \Z)$ is isomorphic to $\Z$ whenever $n \divides p$. Let $\alpha_k \in H^{nk}(J(S^n); \Z)$ be a generator. Then for each $k \ge 1$ we have $\alpha_1^k = k! \cdot \alpha_k$. 
\end{theorem} 

\begin{proof}
(GIVE PROOF)
\end{proof}

\subsection{The Proof}

We consider, explicitly, the space $J_2(S^n) = S^n \times S^n / (x, e) \sim (e, x)$. Consider the cell structure,
\[ S^n = \{ e \} \cup D^n \] 
Then we get a cell decomposition,
\[ J_2(S^n) = \{ e \} \cup D^n \cup D^{2n} = S^n \cup D^{2n} \]
since the product cells $\{ e \} \times D^n$ and $D^n \times \{ e \}$ are glued together. 

\section{K-Theory of Projective Space}

\subsection{K-Theory}

\begin{prop}
$K^*(X) \cong K(X \times S^1)$ 
\end{prop}

\subsection{$G$-Spaces}

\begin{defn}
Let $G$ be a topological group. A $G$-\textit{space} is a topological space along with a continous action $\rho : G \times X \to X$. A \textit{morphism} of $G$-spaces is a continous map $f : X \to Y$ which commutes with the $G$-action. We say a vector bundle $\pi : E \to X$ is a $G$-\textit{bundle} if $E$ is a $G$-space with a linear action and $\pi : E \to X$ is a morphism of $G$-spaces.
\end{defn}

\begin{prop}
Suppose that $G \acts X$ freely. Then there is an equivalence of categories between the category of $G$-vector bundles on $X$ and the category of vector bundles on $X / G$. 
\end{prop}

\begin{proof}

\end{proof}

\begin{defn}
Let $G$ be a finite discrete group and $X$ a $G$-space. Let $\Vect_G(X)$ denote the category of $G$-vector bundles on $X$. The set of isomorphism classes is a commutative monoid under $\oplus$. Then let $K_G(X)$ be the group completion which is a ring under $\otimes$.
\end{defn}

\begin{example}
If $G = 1$ then $K_G(X) = K(X)$. 
\end{example}

\begin{example}
If $X = *$ then $\Vect_G(X)$ is the category of finite dimensional $G$-representations. Then $K_G(X) = R(X)$ which is the Grothendieck group of $G$-representations. 
\end{example}

\subsection{Thom Isomorphism} 

\begin{defn}
Let $E \to X$ be a vector bundle. Then we define the unit sphere bundle $S(E)$ and the unit ball bundle $B(E)$. Then the \textit{Thom space} is $X^E = B(E) / S(E)$. Note that,
\[ K(B(E), S(E)) = \tilde{K}(X^E) \]
Furthermore, the exterior bundle $\Lambda^*(E)$ defines a vector bundle $\lambda_E \in \tilde{K}(X^E)$. 
\end{defn}

\begin{prop}
Let $E$ be a decomposable vector bundle over $X$. Then $\tilde{K}^*(X^E)$ is a free $K^*(X)$-module with $\lambda_E$ as generator.
\end{prop}

\begin{theorem}
Let $X$ be a $G$-space such that $K^1_G(X) = 0$ and $E$ be a decomposable $G$-vector bundle. Let $S(E)$ be the associated sphere bundle then there is an exact sequence,
\begin{center}
\begin{tikzcd}
0 \arrow[r] & K^1_G(S(E)) \arrow[r] & K^0_G(X) \arrow[r, "\varphi"] & K_G^0(X) \arrow[r] & K^0_G(S(E)) \arrow[r] & 0
\end{tikzcd}
\end{center}
where $\varphi$ is multiplication by,
\[ \lambda_{-1} [E] = \sum (-1)^i \lambda^i [E] \]
\end{theorem}

\begin{proof}
Consider the pair $(B(E), S(E))$ where $B(E)$ is the unit ball bundle. Then there is a long exact sequence in $K$-theory,
\begin{center}
\begin{tikzcd}
K^{-1}_G(B(E)) \arrow[r] & K^{-1}_G(S(E)) \arrow[r] & K^{0}_G(B(E), S(E)) \arrow[r] & K^{0}_G(B(E)) \arrow[r] & K^0_G(S(E)) \arrow[r] & K^{-1}_G(B(E), S(E)) \arrow[r] & K^{-1}_G(B(E)) 
\end{tikzcd}
\end{center}
but $B(E)$ is homotopy equivalent to $X$. Therefore, we get $K^1_G(B(E)) = K^1_G(X) = 0$ and $K^0_G(B(E)) = K^0_G(X)$ so we see,
\begin{center}
\begin{tikzcd}
0 \arrow[r] & K^{1}_G(S(E)) \arrow[r] & K^{0}_G(x) \arrow[r] & K^{0}_G(X) \arrow[r] & K^0_G(S(E)) \arrow[r] & K^{-1}_G(B(E), S(E)) \arrow[r] & 0
\end{tikzcd}
\end{center}
why $K_G^0(B(E), S(E)) = K_G^0(X)$ (USE PREVIOUS PROP)
\end{proof}

\begin{prop}
Let $X = *$ then $K^1_G(X) = 0$. 
\end{prop}

\begin{proof}
(SHOW THIS!)
\end{proof}

\begin{cor}
Let $G$ be a cyclic group and $E$ a $G$-module with $S(E)$ having a free $G$-action. Then there is an exact sequence,
\begin{center}
\begin{tikzcd}
0 \arrow[r] & K^1(S(E) / G) \arrow[r] & R(G) \arrow[r] & R(G) \arrow[r] & K^0(S(E)/G) \arrow[r] & 0 
\end{tikzcd}
\end{center}
\end{cor}

\subsection{Application to the Case of Projective Space}

\begin{rmk}
For $E = \C^n$ we have $S(E) = S^{2n - 1}$. Let $G = \Z/2\Z$ which acts freely on $E$ via $x \mapsto -x$. Then $G$ acts on $S(E)$ freely via $x \mapsto -x$, the antipodal action. Therefore, $S(E)/G = \RP^{2n - 1}$. This will alow us to apply the above sequence. First we need to understand the representation theory of $G$. First, recall that by Maschke's theorem, $G$-representations are semi-simple so need only understand irreducible representations. 
\end{rmk}

\begin{theorem}
Let $G$ be a finite abelian group. Then all irreducible $G$-representations are one-dimensional i.e. are characters. 
\end{theorem}

\begin{proof}
Let $\rho : G \to \Aut{V}$ be an irreducible $G$-representation. Then for any $g, h \in G$ we have,
\[ \rho(g) \circ \rho(h) = \rho(gh) = \rho(hg) = \rho(h) \circ \rho(g) \]
Therefore, $\rho(g) : V \to V$ is a $G$-morphism. Since $V$ is irreducible, by Shur's Lemma, $\rho(g) = \lambda_g \id$ and thus $\rho : G \to \C^\times$ is a character.   
\end{proof}


\begin{example}
Representations of $G = \Z / 2 \Z$ are thus direct sums of characters. The characters $\rho : G \to \C^\times$ are determined by the image of $1$. We must have $\rho(1) = \pm 1$. These options are $1$ the trivial character and $\rho$ the nontrivial character. Furthermore, $\rho \otimes \rho : G \to \C^\times$ is trivial since $(-1)^2 = 1$. Therefore, representations are sums,
\[ n + m \rho : = 1 \oplus \cdots 1 \oplus \rho \oplus \cdots \oplus \rho \]
for $n, m \ge 0$ with the relation $\rho^{\otimes 2} = 1$. Thus, taking the group completion we find,
\[ R(G) = \Z[\rho]/(\rho^2 - 1) \]
Furthermore, the map $R(G) \to R(G)$ is given by,
\[ \lambda_{-1} [E] = \sum (-1)^i \rho^i = (1 - \rho)^n \]
\end{example}

\begin{prop}
We have $\tilde{K}^0(\RP^{2n - 1}) = \Z / 2^{n-1} \Z$ and $K^1(\RP^{2n-1}) = \Z$.
\end{prop}

\begin{proof}
Applying the exact sequence,
\begin{center}
\begin{tikzcd}
0 \arrow[r] & K^1(\RP^{2n-1}) \arrow[r] & \Z[\rho]/(\rho^2 - 1) \arrow[r] & \Z[\rho]/(\rho^2 -  1) \arrow[r] & K^0(\RP^{2n-1}) \arrow[r] & 0 
\end{tikzcd}
\end{center}
We change variables $\rho = \rho - 1$ then $\sigma^2 = - 2 \sigma$ and the map sends $1 \mapsto \sigma^n = (-2)^{n-1} \sigma$. Then the kernel is,
\[ K^1(\RP^{2n - 1}) \cong \Z \]
Finally, the cokernel is,
\[ K^0(\RP^{2n - 1}) = \Z[\sigma]/(\sigma^2 + 2 \sigma, (-2)^{n-1} \sigma) = \Z \oplus \Z/2^{n-1} \Z \]
\end{proof}

\end{document}

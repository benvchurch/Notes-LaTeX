\documentclass[12pt]{extarticle}
\usepackage{import}
\import{../}{TopologyCommands}

\newcommand{\SU}[1]{\mathrm{SU}(#1)}
\newcommand{\C}{\mathbb{C}}
\newcommand{\Sp}{\mathrm{Sp}}

\begin{document}

\section{Jan. 14}

\subsection{Mapping Class Groups}

\newcommand{\Map}[1]{\mathrm{Map}\left( #1 \right)}

\begin{defn}
Let $S$ be a surface. Then,
\[ \Map{S} = \mathrm{Homeo}^+(S, \partial S) / \mathrm{Isotopy}(S, \partial S) \]
is orientation-preserving homeomorphisms fixing the boundary up to boundary-preserving isotopies. 
\end{defn}

\begin{example}
$\Map{D^2} = 0$. Given any map $f : D^2 \to D^2$ we define,
\[ F(x, t) = 
\begin{cases}
t f(x/t) & |x| \le t
\\
x & \text{else}
\end{cases} \]
is an isotopy to the constant map. 
\end{example}

\begin{rmk}
This works for $D^2$ with one puncture. This also works for $(D^n, \partial D^n)$. However, it is not known if this works in the differentiable category for $n \ge 4$. 
\end{rmk}

\subsection{Alexander Method}

Let $S$ be a surface. Find some curves $\{ \gamma_i \}$ on $S$ such that $S \setminus \bigcup r_i$ is a union of disks or once-punctured disks. If $f \in \Map{S}$ takes $f(\gamma_i) \sim \gamma_i$ then there exists an isotopy $I$ of $S$ such that $I(\bigcup f(\gamma_i)) = \bigcup \gamma_i$. If the action of $I(f)$ on the graph $\bigcup \gamma_i$ fixes each vertex edge and preserves the orientation of the edges then $f \sim \id$.

\newcommand{\SL}{\mathrm{SL}}

\begin{example}
Define $\Phi : \Map{T^2} \to \SL(2,\Z)$ via its action of $H_1(T^2 ; \Z) = \Z^2$ preserving orientation. Then we can use the Alexander method to show that this is an isomorphism. 
\end{example}

\begin{defn}
A curve is an essential, simple, closed curve (sometimes arcs) meaning it is not homotopic to a point or puncture and has no self-intersections.
\end{defn}


\begin{defn}
The geometric intersection number,
\[ i([\alpha], [\beta]) = \min \{ |a \cap b | \mid a \in [\alpha] \text{ and } b \in [\beta] \text{ intersect transversally} \} \]
where $[\alpha]$ and $[\beta]$ are homotopy classes. Then the algebraic intersection number,
\[ \hat{i}([\alpha], [\beta]) \]
is the signed intersection which is a homotopy invariant so we don't need to minimize. Therefore,
\[ i([\alpha], [\beta]) \equiv \hat{i}([\alpha], [\beta]) \mod 2 \]
\end{defn}

\begin{lemma}
$a,b$ transverse simple closed curves $a,b$ are in minimal position iff there is no bigon (embedded disk bounded by arcs contained in the two curves) between $a$ and $b$. 
\end{lemma}

\section{Jan. 21}

\newcommand{\MCG}[1]{\mathrm{MCG}\left( #1 \right)}

The mapping class group of the torus is $\SL{2, \Z}$. 

\begin{thm}[Degn-Licorice]
The mapping class group is generated by Dehn twists.
\end{thm}

\subsection{The Word Problem}

Given $(G, g_1, \dots, g_n)$ is there an algorithm to check if any given $w = \prod g_i$ is trivial? 

\begin{rmk}
The answer only depends on $G$ not on the generating set for $G$.
\end{rmk}

\begin{rmk}
There exists a group $G$ for which the word problem is unsolvable.
\end{rmk}

\begin{thm}
The word problems for mapping class groups of surfaces are solvable.
\end{thm}

\begin{proof}
Let $G = \MCG{\Sigma}$ be the mapping class group and $T_{a_1}, \dots, T_{_an}$ Dehn twists for curves $a_1, \dots, a_n$ generating $G$. Define $\Gamma$ to be the graph of $\cup a_i$. Given $\varphi \in \MCG{\Sigma}$ then $\varphi(\Gamma) \subset \Sigma$. 
\bigskip\\
Then $\varphi$ is the identity if and only if $\varphi(\Gamma) \sim \Gamma$ are isotopic by the Alexander method. This works if the $a_i$ are pairwise minmal (no bigons) and the $a_i$ fill $\Sigma$
\bigskip\\
How to prove finite generation? Quasi-stabilizer generation theorem. Suppose $G \acts X$ and $D \subset X$ which satisfies $G D = X$. Then the ``quasi-stabilizer'' of $D$ generates $G$. We will use this by the action of $G$ on a curve complex $X$.
\end{proof}

\subsection{The Quasi-Stabilizer Theorem}

\newcommand{\QStab}[1]{\mathrm{QStab}\left( #1 \right)}

\begin{rmk}
Recall that,
\[ \Stab{x} = \{ g \in G : g \cdot x = x \} \]
Then for a set we could define,
\[ \{ \Stab{S} = \{ g \mid g S \subset S \} \]
Now we define the quasi-stabilizer,
\[ \QStab{S} = \{ g \mid g S \cap S \neq \empty \} \]
\end{rmk}

\begin{defn}
Let $\Sigma$ be a surface, and $c(\Sigma)$ is the flag complex defined by the following $1$-skeleton (graph):
\begin{enumerate}
\item vertices are isomorphism class of essential simple closed curves
\item edges from $a$ to $b$ iff $i(a,b) = 0$
\end{enumerate}
then the flag complex adds all possible maximal simplices with edges defined by this graph. 
\end{defn}

\begin{thm}
Let $G$ act on a connected topological space $X$. Suppose that $D \subset X$ an open tralation domain (meaning $G D = X$ e.g. a fundamental domain) then $\left< \QStab{D} \right> = G$. 
\end{thm}

\begin{proof}
Let $g \in G$. assume that $g D \cap \left< \Stab{D} \right> D \neq \empty$. There exists $s \in \left< \QStab{D} \right> $ such that $g D \cap s D \neq \empty \iff s^{-1} g D \cap D \neq \empty$. Therefore $s^{-1} g \in \QStab{D}$ so $g \in s \QStab{D} \subset \left< \QStab{D} \right>$. Therefore $(G \setminus \left< \QStab{D} \right>)D$ is disjoint from $\left< \QStab{D} \right> D$ which is nonempty and therefore the first must be empty by connectedness proving the claim.
\end{proof}

\begin{thm}
Let $X$ be a connected simplicial complex, $G \acts X$ . If $D \subset X$ is a subcomplex translational domain, then $\QStab{D}$ generates $G$.
\end{thm}

\begin{proof}
Use induction.
\end{proof}

\subsection{Finite Generation of Mapping Class Groups}

\begin{thm}
For all $g \ge 0$ the mapping class group $\MCG{\Sigma_g}$ is finitely generated by the Dehn twisits by non-separating essential simple closed curves.
\end{thm}

\begin{proof}
For $g \le 1$ we knkow $\MCG{\Sigma_g}$ explicitly. Assume $g \ge 2$. Then $\MCG{\Sigma_g} \acts \hat{N}(\Sigma_g)$ the connected curve complex (I forgot which one). Suppose $\MCG{\Sigma_{g-1}}$ is finitely generated etc. Then let $a$ be a curve in $\Sigma_g$ and $b$ a curve with $i(a,b) = 1$. Then $T_a T_b(a) = b$ and therefore if $g \cdot a = b$ then $g$ is in the quasi-stabilizer for the segment $a$ to $b$. 
\end{proof}

\end{document}
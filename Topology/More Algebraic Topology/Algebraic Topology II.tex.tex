\documentclass[12pt]{extarticle}
\usepackage[utf8]{inputenc}
\usepackage[utf8]{inputenc}
\usepackage[english]{babel}
\usepackage[a4paper, total={7in, 9.5in}]{geometry}
\usepackage{tikz-cd}

 
\usepackage{amsthm, amssymb, amsmath, centernot, graphicx}
\usepackage{accents}
\DeclareMathAccent{\wtilde}{\mathord}{largesymbols}{"65}

\usepackage{graphicx}
\usepackage{bbm}

\newcommand{\acts}{ \;  \rotatebox[origin=c]{-90}{$\circlearrowright$} \;  }



\newcommand{\orb}[1]{\mathrm{Orb}(#1)}
\newcommand{\stab}[1]{\mathrm{Stab}(#1)}
\newcommand{\rp}{\mathbb{RP}}
\newcommand{\cp}{\mathbb{CP}}

\newcommand{\notimplies}{%
  \mathrel{{\ooalign{\hidewidth$\not\phantom{=}$\hidewidth\cr$\implies$}}}}
 
\renewcommand\qedsymbol{$\square$}
\newcommand{\cont}{$\boxtimes$}
\newcommand{\ndivides}{\centernot \mid}
\newcommand{\Z}{\mathbb{Z}}
\newcommand{\N}{\mathbb{N}}
\newcommand{\C}{\mathbb{C}}
\newcommand{\Zplus}{\mathbb{Z}^{+}}
\newcommand{\Primes}{\mathbb{P}}
\newcommand{\ball}[2]{B_{#1} \! \left(#2 \right)}
\newcommand{\Q}{\mathbb{Q}}
\newcommand{\R}{\mathbb{R}}
\newcommand{\Rplus}{\mathbb{R}^+}
\newcommand{\invI}[2]{#1^{-1} \left( #2 \right)}
\newcommand{\End}[1]{\text{End}\left( A \right)}
\newcommand{\legsym}[2]{\left(\frac{#1}{#2} \right)}
\renewcommand{\mod}[3]{\: #1 \equiv #2 \: \mathrm{mod} \: #3 \:}
\newcommand{\nmod}[3]{\: #1 \centernot \equiv #2 \: mod \: #3 \:}
\newcommand{\ndiv}{\hspace{-4pt}\not \divides \hspace{2pt}}
\newcommand{\finfield}[1]{\mathbb{F}_{#1}}
\newcommand{\finunits}[1]{\mathbb{F}_{#1}^{\times}}
\newcommand{\ord}[1]{\mathrm{ord}\! \left(#1 \right)}
\newcommand{\quadfield}[1]{\Q \small(\sqrt{#1} \small)}
\newcommand{\vspan}[1]{\mathrm{span}\! \left\{#1 \right\}}
\newcommand{\galgroup}[1]{Gal \small(#1 \small)}
\newcommand{\sm}{\! \setminus \!}
\newcommand{\topo}{\mathcal{T}}
\newcommand{\base}{\mathcal{B}}
\renewcommand{\bf}[1]{\mathbf{#1}}
\renewcommand{\Im}[1]{\mathrm{Im} \: #1}
\newcommand{\coker}[1]{\mathrm{coker} \: #1}
\renewcommand{\empty}{\varnothing}
\newcommand{\id}{\mathrm{id}}
\newcommand{\Hom}[3]{\mathrm{Hom}_{#1}\left( #2, #3 \right)}
\newcommand{\Tor}[4]{\mathrm{Tor}^{#1}_{#2} \left( #3, #4 \right)}
\newcommand{\Ext}[4]{\mathrm{Ext}^{#1}_{#2} \left( #3, #4 \right)}
\newcommand{\embed}{\hookrightarrow}
\newcommand{\RP}{\mathbb{RP}}

\newcommand{\catHom}[3]{\mathrm{Hom}_{#1}\left( #2, #3 \right)}
\newcommand{\Top}{\mathbf{Top}}
\newcommand{\pTop}{\mathbf{Top}_{\bullet}}
\newcommand{\Set}{\mathbf{Set}}
\newcommand{\pSet}{\mathbf{Set}_\bullet}
\newcommand{\hTop}{\mathbf{hTop}}
\newcommand{\phTop}{\mathbf{hTop}_{\bullet}}
\renewcommand{\Im}[1]{\mathrm{Im}(#1)}
\newcommand{\homspace}[2]{\left< #1, #2 \right>}

\theoremstyle{definition}
\newtheorem{theorem}{Theorem}[section]
\newtheorem{lemma}[theorem]{Lemma}
\newtheorem{proposition}[theorem]{Proposition}
\newtheorem{example}[theorem]{Example}
\newtheorem{corollary}[theorem]{Corollary}
\newtheorem{remark}{Remark}

\newenvironment{definition}[1][Definition:]{\begin{trivlist}
\item[\hskip \labelsep {\bfseries #1}]}{\end{trivlist}}


\newenvironment{lproof}{\begin{proof} \renewcommand{\qedsymbol}{}}{\end{proof}}
\renewcommand{\mod}[3]{\: #1 \equiv #2 \: mod \: #3 \:}
\newcommand{\gen}[1]{\langle #1 \rangle}
\newcommand{\hook}{\hookrightarrow}

\tikzset{
    labl/.style={anchor=south, rotate=90, inner sep=.5mm}
}

\renewcommand{\bf}[1]{\mathbf{#1}}
\newcommand{\res}{\mathrm{res}}
\newcommand{\F}{\mathcal{F}}
\newcommand{\G}{\mathcal{G}}
\renewcommand{\O}{\mathcal{O}}
\renewcommand{\d}[1]{\mathrm{d} #1}
\newcommand{\deriv}[2]{\frac{\d{#1}}{\d{#2}}}
\newcommand{\Aut}[1]{\mathrm{Aut}\left( #1 \right)}

\renewcommand{\Hom}[3]{\mathrm{Hom}_{#1}\left( #2, #3 \right)}



\renewcommand{\theenumi}{(\alph{enumi})}

\newcommand{\atitle}[1]{\title{% 
	\large \textbf{Mathematics GU6308 Algebraic Topology
	\\ Assignment \# #1} \vspace{-2ex}}
\author{Benjamin Church }
\maketitle}






\begin{document}

\section{Topics}

\begin{enumerate}
\item Basic homotopy theory
\item Obstruction theory
\item Characteristic Classes
\item The Serre spectral sequence
\item The Steenrod operations
\item K-theory
\end{enumerate}

References: Fuchs - Fomenko: homotopical topology, Hatcher's books

Six homeworks (one per topic)

\section{Homotopy Theory}

Basic Questions: 

\begin{enumerate}
\item given maps $f, g : X \to Y$ are they homotopy equivalent?
\item given spaces $X$ and $Y$ are they homotopy equivalent? 
\end{enumerate}

\begin{remark}
All spaces will be connected and locally connected. 
\end{remark}

\begin{definition}
The set $[X, Y] = \Hom{\mathbf{hTop}}{X}{Y}$. Given based spaces $X, Y$ we define $\left< X, Y \right> = \Hom{\mathbf{hTop}_{\bullet}}{X}{Y}$ where morphisms in $\mathbf{hTop}_\bullet$ are continuous maps preserving the basepoint up to homotopy. Note that homotopies in $\mathbf{Top}_\bullet$ are basepoint preserving.
\end{definition}

\begin{example}
Consider $S^n$. Given $f : S^n \to X$ we can construct, $X \sqcup_f D^{n+1}$ by gluing along $f$. This is the coproduct,
\begin{center}
\begin{tikzcd}
D^{n+1} \arrow[r] & X \sqcup_f D^{n+1}
\\
S^n \arrow[r, "f"] \arrow[u] & X \arrow[u]
\end{tikzcd}
\end{center}
Now if $f \sim f'$ then $X \sqcup_f D^{n+1} \sim X \sqcup_f D^{n+1}$. 
\end{example}

\begin{definition}
Given a based space $(X, x_0)$ we define the $n^{\mathrm{th}}$ homotopy group,
\[ \pi_n(X, x_0) = \left< (S^n, p_0), (X, x_0) \right> \]
The group structure is given by the equator squeezing map $s : S^n \to S^n \vee S^n$. Then we define $f * g = (f \vee g) \circ s$. 
\end{definition}

\begin{proposition}
$\pi_n(X, x_0)$ is abelian for $n \ge 2$. 
\end{proposition}


\begin{theorem}
$\pi_n(S^m) = 0$ if $n < m$.
\end{theorem}

\begin{theorem}
$\pi_n(S^n) = \Z$
\end{theorem}

\begin{theorem}
$\pi_3(S^2) = \Z$ generated by the Hopf fibration $\eta : S^3 \to S^2$. 
\end{theorem}

\begin{theorem}
For sufficiently large $n$,
\[ \pi_{n+1}(S^n) = \Z / 2 \Z \quad \quad \pi_{n + 2}(S^n) = \Z / 2 \Z \quad \quad \pi_{n+3}(S^3) = \Z / 2 4 \Z \]
\end{theorem}

\begin{remark}
Given $f : X \to Y$ we get $f_* : \pi_n(X) \to \pi_n(Y)$. 
\end{remark}

\begin{theorem}
Given a path $\gamma : x_1 \to x_2$ in $X$ we get a map,
\[ \gamma_{\#} : \pi_n(X, x_1) \to \pi_n(X, x_2) \]
depending only on the homotopy class of $\gamma$. 
In particular we have a $\pi_1(X, x_0)$-action on $\pi_n(X, x_0)$.
\end{theorem}

\begin{remark}
In the case $n = 1$ this is the conjugation action of $\pi_1(X, x_0)$ on itself.  
\end{remark}

\begin{proposition}
Given the previous proposition, we have,
\[ [S^n, X] = \pi_n(X, x_0) / \pi_1(X, x_0) \]
\end{proposition}

\begin{proposition}
If $p : \tilde{X} \to X$ is a covering map then for $n \ge 2$ the induced map,
\[ p_* : \pi_n(\tilde{X}) \to \pi_1(X) \]
is an isomorphism.
\end{proposition}

\begin{proof}
Injectivity is the homotopy lifting property. Furthermore given $f : S^n \to X$ we can lift it to $\tilde{f} : S^n \to \tilde{X}$ provided that $f_*(\pi_1(S^n)) \subset p_*(\pi_1(\tilde{X}))$. In the case $n \ge 2$, we have $\pi_1(S^n)$ thus such a lift always exists proving surjectivity. 
\end{proof}

\begin{example}
Let $\Sigma_g$ be a genus $g$ surface. For $g \ge 1$ then $\Sigma_g$ has universal cover $\R^2$ which is contractible and thus $\pi_n(\Sigma_g) = \pi_n(\R^2) = 0$ for $n \ge 2$. 
\end{example}

\begin{example}
For $n \ge 2$ we have $\pi_n(\RP^k) = \pi_n(S^k)$. 
\end{example}

\subsection{Basic Operations on Spaces}

\begin{definition}
The suspension of $X$ is $\Sigma X = X \vee S^1$.
\end{definition}

\begin{definition}
The loops space of $X$ is $\Omega X = \Hom{\mathbf{Top}_\bullet}{S^1}{X}$ with the compact-open topology. 
\end{definition}

\begin{theorem}[Adjunction]
\[ \left< \Sigma X, Y \right> = \left< X, \Omega Y \right> \]
\end{theorem}

\begin{example}
$\Sigma S^n = S^{n + 1}$
\end{example}

\begin{proposition}
$\pi_{n+1}(Y) = \left< S^{n+1}, Y \right> = \left< \Sigma S^n, Y \right> = \left< S^n, \Omega Y \right> = \pi_n(\Omega Y)$
\end{proposition}

\begin{proposition}
The space $\Omega X$ is a group object in the category $\mathbf{hTop}_\bullet$. 
\end{proposition}

\begin{remark}
The following definition is due to Hatcher. 
\end{remark}

\begin{definition}
A pointed space $(X, e, \mu)$ is an H-space is there is a map $\mu : X \times X \to X$ such that $\mu(-, e) \sim \id$ and $\mu(e, -) \sim \id$ as pointed maps (relative to the basepoint).  
\end{definition}

\begin{remark}
Any topological group (group object in $\Top$) is an H-space (pointed at the identity element). 
\end{remark}

\begin{remark}
Loop spaces are H-spaces since they are group objects in $\phTop$. 
\end{remark}

\begin{theorem}[Adams]
The spheres $S^n$ admitting an H-space structure are exactly $S^0, S^1, S^3, S^7$. 
\end{theorem}

\begin{corollary}
$\R^n$ has a unital division $\R$-algebra structure iff $n = 1,2,4,8$. 
\end{corollary} 

\begin{proof}
Consider the unit length elements $U = S^{n-1}$. Then a division algebra on $\R^n$ gives a multiplication $U \times U \to U$ (well defined since $xy = 0 \implies x = 0 \text{ or } y = 0$ and thus the result can be scalled to lie in $U$). 
\end{proof}

\section{Relative Groups}

\begin{definition}
Given a space $X$ a subspace $A \subset X$ and a point $x_0 \in A$ we denote the pointed pair as $(X, A, x_0)$. 
\end{definition}

\begin{definition}
For a pointed pair $(X, A, x_0)$ we define $\pi_n(X, A, x_0)$ as maps,
\[ f : (D^n, S^{n-1}, p_0) \to (X, A, x_0) \]
modulo homotopy through maps of this form. 
\end{definition}

\begin{remark}
Suppose $[f] \in \pi_n(X, A, x_0)$ is zero if it is homotopic to a map with image inside $A$. In fact if this is the case then $f$ may be homotoped relative to the boundary. Compression Lemma. 
\end{remark}

\begin{theorem}
There is a long exact sequence for the pointed pair $(X, A, x_0)$,
\begin{center}
\begin{tikzcd}
\cdots \arrow[r] & \pi_{n}(A, x_0) \arrow[r] \arrow[draw=none]{d}[name=Z, shape=coordinate]{} & \pi_{n}(X, x_0) \arrow[r] & \pi_{n}(X, A, x_0) 
\arrow[dll,
rounded corners, crossing over,
to path={ -- ([xshift=2ex]\tikztostart.east)
|- (Z) [near end]\tikztonodes
-| ([xshift=-2ex]\tikztotarget.west)
-- (\tikztotarget)}]
& &
\\ 
& \pi_{n-1}(A, x_0) \arrow[r] & \pi_{n-1}(X, x_0) \arrow[r] & \pi_{n-1}(X, A, x_0) \arrow[r] & \cdots
\end{tikzcd}
\end{center}
\end{theorem}

\section{Results on CW Complexes}

\begin{definition}
A CW pair is a CW complex $X$ with a subcomplex $A \subset X$ (a closed subset which is a cunion of cells e.g.$X^k$ the $k$-skelleton). 
\end{definition}

\begin{theorem}[homotopy extension]
Let $(X, A)$ be a CW pair. Then $(X, A)$ has the homotopy extension property i.e. $\iota : A \to X$ is a cofibration. 
\end{theorem}

\begin{proof}
Working cell-by-cell we can reduce to the case $(X, A) = (D^n, S^{n-1})$. In this case we are given a map on $D^n \times \{ 0 \} \cup S^{n-1} \times I$ which is a deformation retract of $D^n \times I$ so any map can be extended. 
\end{proof}

\begin{definition}
A map $f : X \to Y$ between CW complexes is \textit{cellular} if $f(X^k) \subset Y^k$.
\end{definition}

\begin{theorem}[cellular approximation]
Any map $f : X \to Y$ of CW complexes is homotopic to a cellular map.
\end{theorem}

\begin{corollary}
If $n < m$ then $\pi_n(S^m) = 0$. 
\end{corollary}

\begin{theorem}
If $\pi_i(X, x_0) = 0$ for $i \le n$ (i.e. $X$ is $n$-connected) then $X$ is homotopic to a CW complex with a single zero $0$-cell and no $i$-cells for $1 \le i \le n$. 
\end{theorem}

\begin{lemma}
If $(X, A)$ is a CW-pair and $A$ is contractible then $X \to X / A$ is a homotopy equivalence. 
\end{lemma}


\section{More Results on CW Complexes (01/29)}

\begin{theorem}[Whitehead]
Let $f : X \to Y$ be a map of CW complexes such that $f_* : \pi_n(X, x_0) \to \pi_n(Y, y_0)$ is an isomorphism for each $n$ then $f$ is a homotopy equivalence. 
\end{theorem}


\begin{example}
If $\pi_n(X, x_0) = 0$ for all $n \ge 0$ and $X$ is a CW complex then $X$ is contractible. To see this consider the constant map $X \to *$. 
\end{example}

\begin{example}
Consider $S^\infty = \varinjlim S^n$ where we consider $S^n \subset S^{n+1}$ as the equator. Then $\pi_n(S^\infty) = 0$ since any map $S^n \to S^{\infty}$ can be deformed to a point using the copy of $S^{n+1}$. Thus $S^\infty$ is contractible. 
\end{example}

\begin{remark}
In Whitehead's theorem, simply knowing $\pi_n(X) \cong \pi_n(Y)$ for each $n \ge 0$ does not imply $X \sim Y$ we need these isomorphisms to be induced by a single topological map $f : X \to Y$. 
\end{remark}

\begin{example}
Quotienting by the natural involution on $S^\infty$ we get a double cover $p : S^\infty \to \RP^{\infty}$. Using covering theory we find,
\[ \pi_n(\RP^\infty) 
= \begin{cases}
\Z / 2 \Z & n = 1
\\
0 & n > 1
\end{cases} \]
Furthermore, consider $X = S^2 \times \RP^\infty$ whose universal cover is $\tilde{X} = S^2 \times S^{\infty} \sim S^2$ and thus,
\[ \pi_n(X) 
= \begin{cases}
\Z / 2 \Z & n = 1
\\
\Z & n = 2
\\
0 & n > 1
\end{cases} \]
This has exactly the same homotopy groups as $Y = \RP^2$ whose universal vover is also $\tilde{X} = S^2$ and also has a two-fold cover. However, 
$H_*(\RP^2, \Z / 2 \Z)$ is finite dimensional and $H_*(S^2 \times \RP^\infty, \Z / 2 \Z)$ is infinite dimensional so they cannot be homotopy equivalent. 
\end{example}


\begin{definition}
The mapping cylinder of a morphism $f : X \to Y$ is the pushout,
\[ Mf = Y \coprod_f (X \times I) \]
There is a natural inclusion $\iota : X \embed Mf$ and a deformation retract $j : Mf \to Y$.  
\end{definition}

\begin{remark}
If $X$ and $Y$ are CW complexes then we may homotope $f : X \to Y$ to a cellular map in which case $Mf$ is a CW complex and $\iota : X \embed M(f)$ makes $(Mf, X)$ a CW pair. 
\end{remark}

\begin{definition}
If $X$ and $Y$ are any spaces $f : X \to Y$ is a \textit{weak homotopy equivalence} if $f_* : \pi_n(X) \to \pi_n(Y)$ is an isomorphism for all $n \ge 0$. 
\end{definition}

\begin{theorem}
Any space is weakly homotopy equivalent to a CW complex. 
\end{theorem}

\begin{remark}
Suspension is a functor: given $f : X \to Y$ we get $\Sigma f : \Sigma X \to \Sigma Y$ given by $\Sigma f(t, x) = (t, f(x))$. 
\end{remark}

\begin{remark}
The unit of the suspension-looping adjunction gives a map $X \to \Omega \Sigma X$ given by $x \mapsto (t \mapsto (t, x))$. Applying the functor $\pi_n$ gives the Freudenthal map $\sigma_n : \pi_n(X) \to \pi_{n+1}(\Sigma X)$. 
\end{remark}

\begin{theorem}[Freudenthal Suspension]
Let $X$ be an $n$-connected pointed space. Then the Freudenthal map $\Sigma_k : \pi_k(X) \to \pi_{k+1}(\Sigma X)$ is an isomorphism if $k \le 2n$ and an epimorphism if $k = 2 n + 1$. 
\end{theorem}

\begin{corollary}
$\pi_n(S^n) = \Z$.
\end{corollary}

\begin{proof}
We show this by induction. For $n = 1$ the result $\pi_1(S^1) = \Z$ is a simple application of covering space theory. Now we assume the result for $S^n$. Then since $S^n$ is $(n-1)$-connected, by the Fruedenthal suspension theorem we get an isomorphism $\pi_k(S^n) \xrightarrow{\sim} \pi_{k+1}(S^{n+1})$ for $k < 2n - 1$. Setting $k = n$ we see that $\pi_{n+1}(S^{n+1}) \cong \pi_{n}(S^n)$ for $n > 1$. However, for the case $n = 1$ we only get an epimorphism $\pi_1(S^1) \to \pi_2(S^2)$ since $1 = 2 - 1$. However, there is a surjective degree map $\pi_2(S^2) \to \Z$ and thus $\pi_2(S^2) = \Z$.  
\end{proof}

\section{Spectra}

\begin{definition}
A spectrum is a sequence $X_n$ of CW complexes along with structure maps $s_n : \Sigma X_n \to X_{n+1}$. 
\end{definition}

\begin{definition}
Let $X$ be a spectrum then we define the homotopy groups of $X$ via,
\[ \pi_k(X) = \varinjlim_n \pi_{k + n}(X_n) \]
where the maps $\Sigma X_n \to X_{n+1}$ induce $\pi_{k + n}(X_n) \to \pi_{k + n +1}(X_{n+1})$ by adjunction making the groups $\pi_{k + n}(X_n)$ a directed system.
\end{definition}

\begin{remark}
Spectra may have homotopy in negative dimension i.e. $\pi_k(X) \neq 0$ for $k \le 0$ in general. 
\end{remark}

\begin{definition}
We say a spectrum is stable if the structure maps are eventually all weak homotopy equivalences. 
\end{definition}

\begin{example}
Given a CW complex $X$ we can form the suspension specturm $X_n = \Sigma^n X = S^n \wedge X$ with identity maps $\Sigma X_n \to X_{n+1}$. This is clearly a stable spectrum.
\end{example}

\renewcommand{\S}{\mathbf{S}}

\begin{example}
The suspension spectrum of $S^0$ is the sphere spectrum $\S$ given by $\S_n = S^n$ with the natural homeomorphisms $\Sigma S^n \to S^{n+1}$. 
\end{example}

\begin{definition}
An $\Omega$-spectrum is a specturm $X$ such that the adjunction of the structue map $X_n \to \Omega X_{n+1}$ is a weak homotopy equivalence. 
\end{definition}

\section{Feb 12}

\begin{theorem}
Two CW complexes of type $K(G, n)$ are homotopy equivalent.
\end{theorem}

\begin{proof}
Let $X, Y$ be CW complexes. Assume that $X$ has no $1, \dots, (n-1)$-cells (since it is $(n-1)$-connected) and one $0$-cell (since it is connected). Then,
\[ X^n = \bigvee_{i \in I} S^n \]
each of these spheres represents an element $\pi_n(X) = G$. Construct $f_n : X^n \to Y$ by sending each $S^n$ to the corresponding element in $\pi_n(Y) = G$. Next construct $f_{n+1} : X^{n+1} \to Y$ so that each $\partial D^{n+1} = S^n \xrightarrow{f_n} Y$ represents $0 \in \pi_n(Y)$ (since the $(n+1)$-cells give the relations on $G$) then $\partial D^{n+2} = S^{n+1} \xrightarrow{f_{n+1}} Y$ is nullhomotopic because $\pi_{n+1}(Y) = 0$. Repeating, we can extend to all $X$. 
\end{proof}

\begin{remark}
Key point: $\pi_n(X)$ is generated by $n$-cells and has relations by $(n+1)$-cells. This is a first glimpse of obstruction theory. We ask the following questions:
\begin{enumerate}
\item[Q1] Given a CW pair $(X, A)$ and $f : A \to Y$ can we extend this to $\tilde{f} : X \to Y$?
\item[Q2] Given a giber bundle $p : E \to B$ and a map $f : X \to B$ can we lift it to $\tilde{f} : X \to E$?
\end{enumerate}
For Q1, assume that $\pi_1(Y) \acts \pi_n(Y)$ trivially (i.e. $Y$ is simple so we need not worry about basepoints!). Given $f : X^n \to Y$ can we extend it to $X^{n+1}$? Gluing a disk $D^{n+1}$ then $f$ extends to $D^{n+1}$ iff $f|_{S^n} : S^n \to Y$ is nullhomotopic i.e. is zero in $\pi_n(Y)$. In general, to each $(n + 1)$-cell $e$, $[f_e] \in \pi_n(Y)$ then we can construct $c_f \in C^{n+1}(X, \pi_n(Y))$ a cellular cochain called the obstruction cochain. Then $f$ extends to $X^{n+1} \iff c_f = 0$. 
\end{remark}

\begin{lemma}
$\delta c_f = 0$ i.e. $c_f$ is a cocycle. Therefore, $O_f := [c_f] \in H^{n+1}(X; \pi_n(Y))$ is the obstructuon class. 
\end{lemma}

\begin{theorem}
$f|_{X^{n-1}}$ extends to $X^{n+1}$ iff $O_f = 0$.  
\end{theorem}

\begin{proof}
First we prove the Lemma. Consider the diagram,
\begin{center}
\begin{tikzcd}
C_{n+2}(X) \arrow[dd, "\partial"] \arrow[r, equals] & H_{n+2}(X^{n+2}, X^{n+1}) \arrow[r, "h^{-1}"] & \pi_{n + 2}(X^{n+2}, X^{n+1}) \arrow[d, "\partial"]
\\
& & \pi_{n+1}(X^{n+1}) \arrow[d, "\iota"]
\\
C_{n+1}(X) \arrow[rrdd, "c_f"] \arrow[r, equals] & H_{n+1}(X^{n+1}, X^n) \arrow[r, "h^{-1}"] & \pi_{n+1}(X^{n+1}, X^n) \arrow[d, "\partial"] 
\\
& & \pi_n(X^n) \arrow[d, "f_*"]
\\
& & \pi_n(Y)
\end{tikzcd}
\end{center}
The piece of the LES,
\begin{center}
\begin{tikzcd}
\pi_{n+1}(X^{n+1}) \arrow[r] & \pi_{n+1}(X^{n+1}, X^n) \arrow[r] & \pi_n(X^n)
\end{tikzcd}
\end{center}
composes to zero so by the commutativity of the above diagram $c_f \circ \partial = 0$. 
\end{proof}

\begin{definition}
Suppose there are two maps $f, g : X^n \to Y$ that agree on $X^{n-1}$ then for each $n$-cell $D^n$ if we glue two $D^n$ along the boundary on which $f,g$ agree then we get a map $(f, g) : S^n \to Y$ and thus an element $\pi_n(Y)$ for each $n$-cell. This gives a difference cochain $d_{f,g} \in C^n(X ; \pi_n(Y))$ and $d_{f,g} = 0$ iff $f,g : X^n \to Y$ are homotopic relative to $X^{n+1}$.  
\end{definition}

\begin{lemma}
$\delta d_{f,g} = c_g - c_f$.
\end{lemma}

\begin{lemma}
Given $f : X^n \to Y$ for any $d \in C^n(X; \pi_n(Y))$ there is $g : X^n \to Y$ with $f|_{X^{n-1}} = g|_{X^{n-1}}$ s.t. $d_{f,g} = d$. 
\end{lemma}

\begin{proof}
For $d \in C^n(X; \pi_n(Y))$ then for an $n$-cell $e$ we have $d(e) \in \pi_n(Y)$ then consider the sum of maps $f$ and $d(e)$ using the sum structure on $e$ contracting the equator. 
\end{proof}

\begin{proof}
Now we prove the theorem. Suppose that $O_f = 0$ then $c_f = \delta d$ for some $d \in C^n(X ; \pi_n(Y))$. Now there exists $g : X^n \to Y$ with $f|_{X^{n-1}} = f|_{X^{n-1}}$ and $d_{f,g} = -d$. Also, $\delta d_{f,g} = c_g - c_f$ and thus $c_g = c_f + \delta d_{f,g} = c_f - \delta d = 0$ therefore $c_g = 0$ so $g$ can extend to $X^{n+1}$ and $f|_{X^{n-1}} = g|_{X^{n-1}}$. 
\end{proof}

\begin{theorem}
Let $f,g : X^n \to Y$ be maps with $f|_{X^{n-2}} = g|_{X^{n-2}}$. Then $[d_{f,g}] = 0$ iff they are homotopic relative to $X^{n-2}$. 
\end{theorem}

\subsection{Cohomology of $K(G, n)$}

Let $n \ge 2$ and $G$ abelian. Consider a map $f : X \to K(G, n)$. By Hurewicz, $H_n(K(G, n), \Z) = \pi_n(K(G, n)) = G$ and $H_{n-1}(K(G, n), \Z) = 0$. Now, by the universal coefficient theorem,
\[ H^n(K(G, n), G) = \Hom{}{H_n(K(G, n), \Z)}{G} = \Hom{}{G}{G} \]
Therefore, there is a canonical element $\mathbbm{1} \in H^n(K(G, n), G)$ which is the class of $\id : G \to G$. 
\bigskip\\
Also, via $f : X \to K(G, n)$, we also get $f^*(\mathbbm{1}) \in H^n(X ; G)$, which depends only on the homotopy class of $f$. 

\begin{theorem}
The map $[X, K(G, n)] \to H^n(X, G)$ sending $[f] \mapsto f^\times(\mathbbm{1})$ is an isomorphism. 
\end{theorem}

\begin{remark}
We say that $K(G, n)$ classifies $H^n(-,G)$ meaning that the functor,
\[ H^n(-,G) : \{ \text{CW-complexes} \} \to \mathbf{Set} \]
 is represented by $[-, K(G,n)]$.
\end{remark}

\begin{definition}
Given a contravariant functor $h : \{ \text{CW-complexes} \} \to \mathbf{Set}$ we say that $C$ classifies $h$ if there is a natural isomorphism $h \cong [-, C]$ in this case we say that $h$ is representable and the pair $(C, \id \in h(C))$ is a representation of $h$. 
\end{definition}

\end{document}
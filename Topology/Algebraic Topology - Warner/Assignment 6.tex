\documentclass[12pt]{extarticle}
\usepackage{import}
\import{./}{Includes}

\begin{document}
\atitle{6}

Note. My order of path concatenation follows Hatcher,
\[\gamma * \delta(x) = \begin{cases}
\gamma(2x) & x \le \tfrac{1}{2} \\
\delta(2x - 1) & x \ge \tfrac{1}{2}
\end{cases}\]
 
\section*{Problem 1.}

Suppose the following diagram of abelian groups commutes,
\begin{center}
\begin{tikzcd}[column sep = large, row sep = large]
A \arrow[r, "a"] \arrow[d, "f"] & B \arrow[r, "b"]\arrow[d, "g"] & C \arrow[r, "c"] \arrow[d, "h"] & D \arrow[r, "d"] \arrow[d, "i"] & E \arrow[d, "j"] \\
A' \arrow[r, "a'"] & B' \arrow[r, "b'"] & C' \arrow[r, "c'"] & D' \arrow[r, "d'"] & E' \\
\end{tikzcd}
\end{center}  
with exact rows and $f$, $g$, $i$, and $j$ are isomorphims. Suppose that $h(x) = 0$ then $c' \circ h(x) = 0$. By commutativity, $i \circ c(x) = 0$ but $i$ is an injection so $c(x) = 0$. Thus, $x \in \ker{c} = \Im{b}$ so there exists $y \in B$ such that $b(y) = x$ but $h(x) = 0$ so $h \circ b(y) = b' \circ g (y) = 0$ so $g(y) \in \ker{b'} = \Im{a'}$ so there exists $z \in A'$ such that $a'(z) = g(y)$. But $f$ is a surjection so there exists $q \in A$ such that $f(q) = z$. Then, $g \circ a(q) = a' \circ f(q) = a'(z) = g(y)$ but $g$ is an injection so $a(q) = y$. Then $b \circ a(q) = b(y) = x$. However, the top row is exact so $\ker{b} = \Im{a}$ but $a(q) \in \Im{a}$ so $a(q) \in \ker{b}$ so $b \circ a(q) = x = 0$. Thus, $h$ is injective. \bigskip\\
In this proof, we never used the maps $d$, $j$, and $d'$ so only the first four groups in the sequences are needed. Also, I only used the fact that $f$ is a surjection, $g$ is an injection, and $i$ is an injection.
  
\section*{Problem 2.}

WARNING: The following is wrong. It assumes that $\gamma_t(r) = \gamma(1 - (1 - r) t)$ satisfies $\gamma_t(0) = \pi(x)$ for all $t$ which is clearly false unless $\gamma$ is contained in the fiber. I should have know it was wrong because the fact that $p$ is a fibration is not actually used.
\bigskip\\
Let $p : (E, e_0) \to (B, b_0)$ be a pointed fibration. The fiber of $p$ is the subspace $F = p^{-1}(b_0)$. Then, define the map $\phi : F \to N_p$ by $\phi(x) = (x, e_{b_0})$ where $e_{b_0}$ is the constant loop at $b_0$. This map is well-defined because $x \in F = p^{-1}(b_0)$ so $p(x) = b_0 = e_{b_0}(0)$. The projection $\pi_1 : N_p \to E$ is given by $\pi_1(x, \gamma) = x$. Therefore, $\pi_1 \circ \phi(x) = \pi_1(x, e_{b_0}) = x$ so $\pi_1 \circ \phi = \id_F$. However, $\phi \circ \pi_1(x, \gamma) = \phi(x) = (x, e_{b_0})$. Define the homotopy $H : N_p \times I \to N_p$ by $H(x, \gamma, t) = (x, \gamma_t)$ where $\gamma_t(r) = \gamma(1 - (1 - r) t)$. Thus, $\gamma_0(r) = \gamma(1) = b_0$ and $\gamma_1(r) = \gamma(r)$. Therefore, $H(x, \gamma, 0) = (x, \gamma_0) = (x, e_{b_0}) = \phi \circ \pi_1(x, \gamma)$ and $H(x, \gamma, 1) = (x, \gamma_1) = (x, \gamma)$. Thus, $H$ is a homotopy between $\phi \circ \pi_1$ and $\id_{N_p}$ so $\phi$ is a homotopy equivalence. 
\bigskip\\
Also this doesn't work because $\pi_1(x, \gamma) = x$ is in $E$ but not necessarily $F$ because we only know that $p(x) = \gamma(0)$ not that $p(x) = b_0$. Here's how to actually do it.
\bigskip\\
Let $p : E \to B$ be a based fibration with fiber $F = p^{-1}(b_0)$. Define $\phi : F \to N_p$ by $\phi(x) = (x, e_{b_0})$ where $e_{b_0}$ is the constant loop at $b_0$ (this is well defined because $e_{b_0}(0) = b_0 = p(x)$ and $e_{b_0}(1) = b_0$).
Define a homotopy $g : N_p \times I \to B$ sending $(x, \gamma, t) \mapsto \gamma(t)$. Then $g_0(x, \gamma) = \gamma(0) = p(x)$ so setting $\tilde{g}_0(x, \gamma) = x$ we can apply the homotopy lifting property to the fibration $p : E \to B$,
\begin{center}
\begin{tikzcd}[column sep = large, row sep = large]
N_p \times \{ 0 \} \arrow[r, "\tilde{g}_0"] \arrow[d] & E \arrow[d, "p"]
\\
N_p \times I \arrow[r, "g"'] \arrow[ru, "\tilde{g}", dashed] & B
\end{tikzcd}
\end{center}
gives a homotopy $\tilde{g} : N_p \times I \to E$ satisfying $p \circ \tilde{g}(x, \gamma, t) = g(x, \gamma, t) = \gamma(t)$. Thus we may define,
\[ h : N_p \times I \to N_p \quad \text{via} \quad h(x, \gamma, t) = (\tilde{g}(x, \gamma, t), \gamma|_{[t, 1]}) \]
This is well-defined because $p \circ \tilde{g}(x, \gamma, t) = \gamma(t) = \gamma|_{[t, 1]}(0)$ so $h(x, \gamma, t) \in N_p$. Furthermore, 
\[ h_0(x, \gamma) = (\tilde{g}_0(x, \gamma), \gamma) = (x, \gamma) \implies h_0 = \id_{N_p} \]
Notice that $p \circ \tilde{g}_1 = g_1$ sends $(x, \gamma) \mapsto \gamma(1) = b_0$ giving a map $\tilde{g}_1 : N_p \to F$
Furthermore, $h_1(x, \gamma) = (\tilde{g}_1(x, \gamma), e_{b_0}) = \phi \circ \tilde{g}_1(x, \gamma)$. So we see that $h$ gives a homotopy between $\id_{N_p}$ and $\phi \circ \tilde{g}_1$. Finally, $\tilde{g}_1 \circ \phi(x) = \tilde{g}_1(x, e_{b_0})$ so consider $\tilde{g}(x, e_{b_0}, t)$ which satisfies $p \circ \tilde{g}(x, e_{b_0}, t) = g(x, e_{b_0}, t) = b_0$ so $\tilde{g}(x, e_{b_0}, t) \in F$. Therefore $\tilde{g}(-, e_{b_0}, -)$ is a homotopy $F \times I \to F$ from $\tilde{g}_0(-, e_{b_0}) = \id_F$ to $\tilde{g}_1(-, e_{b_0}) = \tilde{g}_1 \circ \phi$. Therefore, $\phi : F \to N_p$ is a homotopy equivalence.

\section*{Problem 3.}

Let $f : X \to Y$ be a map of pointed spaces. Consider the projection $\pi_1 : N_f \to X$ given by $\pi_1(x, \gamma) = x$. Take any space $Z$ and maps $g : Z \to N_f$ and $h : Z \times I \to X$ such that the following diagram commutes,
\begin{center}
\begin{tikzcd}[column sep = huge, row sep = huge]
Z \arrow[d, hook, "\iota"] \arrow[r, "\tilde{g}_0"] & N_f \arrow[d, "\pi_1"] \arrow[r, "\pi_2"] & PY \arrow[d, "\ev_0"] \\
Z \times I \arrow[r, "g"] \arrow[ru, dashed, "\tilde{g}"] & X \arrow[r, "f"] & Y
\end{tikzcd}
\end{center}
The outside rectangle is a lifting diagram for $\ev_0 : PY \to Y$.
I claim that $\ev_0$ is a fibration. It is the fibrant replacement of $* \to Y$ i.e. $PY = E_{* \to Y}$. Consider a diagram,
\begin{center}
\begin{tikzcd}[column sep = huge, row sep = huge]
Z \arrow[d, hook, "\iota"] \arrow[r, "\tilde{h}_0"] &  PY \arrow[d, "\ev_0"] \\
Z \times I \arrow[r, "h"] \arrow[ru, dashed, "\tilde{h}"] & Y
\end{tikzcd}
\end{center}
Let $\gamma_x = \tilde{h}_0(x)$ and note that $\gamma_x(1) = y_0$ and $\gamma_x(0) = \ev_0 \circ \tilde{h}(x) = h(x, 0)$. Then $h(x,-)$ is a path starting at $\gamma_x(0)$. Thus we can define $\tilde{h} : Z \times I \to PY$ via,
\[ \tilde{h}(x, t) = \gamma_x * (-h(x,-)|_{[0,t]}) \]
Notice that $\tilde{h}(x, t)(1) = \gamma_x(1) = y_0$ so this is a well-defined function $\tilde{h} : Z \times I \to PY$. Finally, $\ev_0 \circ \tilde{h}(x,t) = h(x,t)$ so $\ev_0 \circ \tilde{h} = h$ so this is a lift proving that $\ev_0$ is a fibration. See Hatcher 4.64 for more details. 
\bigskip\\
Now we prove that $\pi_1$ is a fibration by showing that the (strict) pullback of a fibration is a fibration. Indeed, returning to the original diagram, we get maps $\pi_2 \circ \tilde{g}_0 : Z \to PY$ and $f \circ g : Z \times I \to Y$ such that the outer rectangle commutes. By the homotopy lifting property of the fibration $\ev_0 : PY \to Y$ there is a lift $\tilde{g}' : Z \times I \to PY$. However, by the universal property of the pullback we get a map $\tilde{g} : Z \times I \to N_f$ from the pair $g : Z \times I \to X$ and $\tilde{g}' : Z \times I \to PY$ making the square commute. Now $\pi_1 \circ \tilde{g} = g$ and I claim that $\tilde{g} \circ \iota = \tilde{g}_0$. Indeed, $\pi_1 \circ \tilde{g} \circ \iota = g \circ \iota = \pi_1 \circ \tilde{g}_0$ and $\pi_2 \circ \tilde{g} \circ \iota = \tilde{g}' \circ \iota = \pi_2 \circ \tilde{g}_0$ so by the universal property of the pullback $\tilde{g} \circ \iota = \tilde{g}_0$. Therefore we get a lift in the leftmost square proving that $\pi_1 : N_f \to X$ is a fibration.
\bigskip \\
Let $\pi = \pi_1 : N_f \to X$ be the fibration considered above and take, $\phi : F \to N_\pi$, the natural inclusion on the fiber $F = \pi^{-1}(x_0)$ which is given by $\phi(x_0, \gamma) = (x_0, \gamma, e_{x_0})$ for $(x_0, \gamma) \in \pi^{-1}(x_0)$. Since $(x_0, \gamma) \in N_f$ we have $\gamma(0) = f(x_0) = y_0$ and $\gamma(1) = y_0$. Therefore, $\gamma$ is a loop so $F \cong \Omega Y$ via $(x_0, \gamma, e_{x_0}) \mapsto \gamma$. Thus, $\phi$ can be viewed as a map $\phi : \Omega Y \to N_\pi$. However, as proven in problem (2), $\phi : F \to N_\pi$ is a homotopy equivalence when $\pi$ is a fibration. Therefore, $\phi : \Omega Y \to N_\pi$ is a homotopy equivalence. 
 
\section*{Problem 4.}
Consider the covering map $p : S^n \to \rp^n$ given by the quotient map on antipodal points. We know from covering space theory that for $m \ge 2$, the map $p_* : \pi_m(S^n) \to \pi_m(\rp^n)$ is an isomorphism. However, since we have some fancy new long exact sequences it seems a shame not to use them! \bigskip\\
The covering map $p : S^n \to \rp^n$ is a fibration with fiber $S^0$. This fibration induces the long exact sequence,
\begin{center}
\begin{tikzcd}
\cdots \arrow[r] & \pi_4(S^0) \arrow[r] & \pi_4(S^n) \arrow[r] & \pi_4(\rp^n) \arrow[r] & \pi_3(S^0) \arrow[r] & \pi_3(S^n) \arrow[draw=none]{d}[name=Z, shape=coordinate]{} \arrow[r] & \pi_3(\rp^n)
\arrow[dlllll,
rounded corners, crossing over,
to path={ -- ([xshift=2ex]\tikztostart.east)
|- (Z) [near end]\tikztonodes
-| ([xshift=-2ex]\tikztotarget.west)
-- (\tikztotarget)}]
\\ 
& \pi_2(S^0) \arrow[r] & \pi_2(S^n) \arrow[r] & \pi_2(\rp^n) \arrow[r] & \pi_1(S^0) \arrow[r] & \pi_1(S^n) \arrow[r] & \pi_1(\rp^n) 
\end{tikzcd}
\end{center}
However, $\pi_m(S^0) = 0$ for any $m > 0$ because $S^0$ is a disjoint union of points. Therefore, for each $m \ge 2$, we can pick out the exact sequence,
\begin{center}
\begin{tikzcd}
0 \arrow[r] & \pi_m(S^n) \arrow[r, "f"] & \pi_m(\rp^m) \arrow[r] & 0
\end{tikzcd}
\end{center}
Because this sequence is exact, $\ker{f} = \Im{0} = 0$ and $\Im{f} = \ker{0} = \pi_m(\rp^m)$ so $f$ is an isomorphism. Therefore, $\pi_m(S^n) \cong \pi_m(\rp^n)$ for $m \ge 2$.

\section*{Problem 5.}

For $m, n \in \Z_{> 1} \cup \{ \infty \}$ let $X = \rp^m \times S^n$ and $Y = \rp^n \times S^m$. Using the previous problem, for $ i \ge 2$,
\[ \pi_i(X) = \pi_i(\rp^m) \times \pi_i(S^n) \cong \pi_i(S^m) \times \pi_i(S^n) \cong \pi_i(S^n) \times \pi_i(S^m)  \cong \pi_i(\rp)^n \times \pi_i(S)^m \cong \pi_i(\rp^n \times S^m) = \pi_i(Y) \]
For $i = 0$ this statement is trivial because both spaces are connected. For $i = 1$ we must check the formula explicitly,
\[ \pi_1(\rp^m \times S^n) \cong (\Z / 2 \Z) \times 1 \cong \Z / 2 \Z \quad \text{and} \quad \pi_1(\rp^n \times S^m) \cong (\Z/2 \Z) \times 1 \cong \Z\]
so $\pi_1(\rp^m \times S^n) \cong \pi_1(\rp^n \times S^m)$. I have used the formula $\pi_1(S^n) = 1$ for $n > 1$ and $\pi_1(\rp^n) \cong \Z / 2 \Z$ for $n > 1$ because $S^n$ is a double cover of $\rp^n$ which is the universal cover. \bigskip\\
An alternative proof of this fact using covering spaces goes as follows. Because the product of covering maps is a covering map, the product of simply connected spaces is simply connected, and th universal cover is unique up to isomorphism, we know that $\tilde{X} = S^m \times S^n$ and $\tilde{Y} = S^n \times S^m$ because $S^n$ is simply connected and the universal cover of $\rp^m$ is $S^m$. Therefore, $\tilde{X} \cong \tilde{Y}$. However, for $n \ge 2$ the covering map $p : \tilde{X} \to X$ induces an isomorphism, $p_* : \pi_i(\tilde{X}) \to \pi_i(X)$. Therefore,
\[ \pi_i(X) \cong \pi_i(\tilde{X}) \cong \pi_i(\tilde{Y}) \cong \pi_i(Y) \]

\section*{Problem 6.}

Consider the long exact sequence of abelian groups such that every third map $\iota_n$ is injective,
\begin{center}
\begin{tikzcd}
\cdots \arrow[r] & C_{n + 1} \arrow[r, "f_{n+1}"] & A_n \arrow[r, "\iota_n"] & B_n \arrow[r] & C_n \arrow[r, "f_n"] & A_{n-1} \arrow[r, "\iota_{n-1}"] & \arrow[r] B_{n_1} \arrow[r] & \cdots 
\end{tikzcd}
\end{center}
Since $\iota_n$ is injective, $\ker{\iota_n} = 0 = \Im{f_{n+1}}$ so $f_{n+1}$ is the zero map. Likewise, $\iota_{n-1}$ is injective and the sequence is exact so $\ker{\iota_{n - 1}} = \Im{f_n} = 0$ so $f_n$ is the zero map. Therefore, the sequence,  
\begin{center}
\begin{tikzcd}
0 \arrow[r] & A_n \arrow[r, "\iota_n"] & B_n  \arrow[r] & C_n \arrow[r] & 0
\end{tikzcd}
\end{center}
is short exact.
\section*{Problem 7.}
Suppose that the sequence of abelian groups,
\begin{center}
\begin{tikzcd}
0 \arrow[r] & A \arrow[r, "f"] & B  \arrow[l, bend left, "g"] \arrow[r, "h"] & C \arrow[r] & 0
\end{tikzcd}
\end{center}
is short exact and the map $g : B \to A$ satisfies $g \circ f = \id_A$. For define the homomorphism $F : B \to A \oplus C$ by $F(x) = (g(x), h(x))$. Because the kernel of the last zero map is $C$, the map $h$ is surjective. Also, $g$ is a left inverse so $g$ is surjective. Thus, $F$ is surjective. Furthermore, suppose that $(g(x), h(x)) = 0$ then $h(x) = 0$ so $x \in \ker{h} = \Im{f}$ so there exists $y \in B$ such that $f(y) = x$ but $g \circ f(y) = y$ so $g(x) = y = 0$. Thus, $y = 0$ so $f(y) = x = 0$ so $F$ is injective. Therefore, $F$ is an isomorphism. Thus, $B \cong A \oplus C$.  
\section*{Problem 8.}
Let $(X, A)$ be a pointed pair. We showed in class that the following sequence induced by the inclusion $\iota : A \to X$,
\begin{center}
\begin{tikzcd}
\cdots \arrow[r] & \pi_2(X, A) \arrow[r] & \pi_1(A) \arrow[r, "\iota_*"] & \pi_1(X) \arrow[r] & \pi_1(X, A) \arrow[r] & \pi_0(A) \arrow[r, "\iota_*"] & \pi_0(X)
\end{tikzcd}
\end{center} 
is long exact. Suppose that there exists a retraction $r : X \to A$. Then we know, $r \circ \iota = \id_A$. Therefore, $r_* \circ \iota_* = \id_{\pi_n(A)}$. Therefore, $\iota_*$ is an injection. Applying the result of problem $6$ to this long exact sequence, we have the following short exact sequence for each $n$,
\begin{center}
\begin{tikzcd}
0 \arrow[r] & \pi_n(A) \arrow[r, "\iota_*"] & \pi_n(X)  \arrow[r] & \pi_n(X, A) \arrow[r] & 0
\end{tikzcd}
\end{center}
However, $r_* : \pi_n(X) \to \pi_n(A)$ is a left inverse of $\iota_*$ so by problem $7$ this short exact sequence splits. Therefore, $\pi_n(X) \cong \pi_n(A) \oplus \pi_n(X, A)$. 

\end{document}

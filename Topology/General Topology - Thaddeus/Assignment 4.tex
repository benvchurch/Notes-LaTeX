\documentclass[12pt]{extarticle}
\usepackage{import}
\import{./}{Includes}

\begin{document}
\atitle{4}
 
\section*{Problem 1.}
Let $X, Y$ be topological spaces and $X \times Y$ have the product topology. Take $(x, y) \in \overline{A \times B}$ then for any open sets $U, V$ such that $x \in U \in \topo_X$ and $y \in V \in \topo_Y$. Then $(x, y) \in U \times V$ so because $(x,y)$ is in the closure, $A \times B \cap U \times V \neq \emptyset$. Then $A \cap U \neq \emptyset$ and $B \cap V \neq \emptyset$. Thus, $x \in U \implies A \cap U \neq \emptyset$ and $y \in V \implies B \cap V \neq \emptyset$ so $x \in \bar{A}$ and $y \in \bar{B}$ thus, $(x, y) \in \bar{A} \times \bar{B}$. Therefore, $\overline{A \times B} \subset \bar{A} \times \bar{B}$. \\ \\
Alternatively, by Lemma \ref{closedprods}, $\bar{A} \times \bar{B}$ is a closed subset of $X \times Y$ and $A \subset \bar{A}$ and $B \subset \bar{B}$. Therefore, $A \times B \subset \bar{A} \times \bar{B}$ which is closed so $\overline{A \times B} \subset \bar{A} \times \bar{B}$. 
\\ \\
Conversely, if $(x, y) \in \bar{A} \times \bar{B}$ and $(x, y) \in W \in \topo_{X \times Y}$ then by the definition of the product topology, $\exists U \in \topo_X, V \in \topo_Y$ s.t. $(x, y) \in U \times V \subset A \times B$ but $x \in \bar{A}$ so $U \cap A \neq \emptyset$ and similarly, $y \in \bar{B}$ so $V \cap B \neq \emptyset$. Thus, $U \times V \cap A \times B \neq \emptyset$ but $U \times V \subset W$ so $W \cap A \times B \neq \emptyset$ so $(x, y) \in \overline{A \times B}$. Thus, $\bar{A} \times \bar{B} \subset \overline{A \times B}$,

\section*{Problem 2.}
Take $A = \left(-2, 1 \right) \cup \{2 \} \subset \R$ and $B = \{-2\} \cup \left(-1, 2 \right) \subset \R$  \bigskip Then, $A \cap B = (-1, 1)$ and  $\bar{A} \cap B = \{-2 \} \cup (-1, 1]$ and $A \cap \bar{B} = [-1, 1) \cup \{2\}$ and $\overline{A \cap B} = [-1, 1]$ and $\bar{A} \cap \bar{B} = [-1,1] \cup \{-2, 2\}$. \bigskip No two of these are equal. 

\section*{Problem 3.}
Let $(X, \topo)$ be a Hausdorff space. Consider $x \in X$ and any $y \in X \sm \{x\}$. Now, since $x \neq y$, by the Haudorff property, there exist $U_y, V_y \in \topo$ s.t. $x \in U_y$ and $y \in V_y$ and $U_y \cap V_y = \emptyset$. Thus, since $x \in U_y$ then $x \notin V_y$. Now take \[V = \bigcup\limits_{y \in X \sm \{x\}} V_y\]
Because $x \notin V_y$ we have $x \notin V$ so $V \subset X \sm \{x\}$. However, for any $y \in X \sm \{x \}$ we have $y \in V_y$ thus $y \in V$ so $V = X \sm \{x\}$. But each $V_y$ is open thus $V = X \sm \{x\}$ is open so $\{x\}$ is closed. 

\section*{Problem 4.}
Let the \textit{diagonal} of $X$ be the set $\Delta = \{(x, x) \in X \times X \mid x \in X\}$. Let $\Delta$ be closed in the product topology $X \times X$. Then $\Delta^C = (X \times X) \sm \Delta$ is open. Take $x \neq y$ then $(x, y) \in \Delta^C$ so by openness, $\exists : U, V \in \topo$ s.t. $(x, y) \in U \times V \subset \Delta^C$. For any $z \in U$ if $z \in V$ then $(z, z) \in U \times V \subset \Delta^C$ but $(z, z) \in \Delta$ which is a contradiction. Thus, $U \cap V = \emptyset$ which gives the Hausdorff condition. \\ \\
Conversely, let $X$ be Hausdorff then if $(x, y) \in \Delta^C$ then $x \neq y$ so by the Hausdorff property, $\exists U, V \in \topo$ s.t. $x \in U$ and $y \in V$ and $U \cap V = \emptyset$. If $(z, z) \in \Delta$ then $(z, z) \notin U \times V$ else $z \in U$ and $z \in V$. Therefore, $(x, y) \in U \times V \subset \Delta^C$. Therefore, $\Delta^C$ is open in the product topology which implies that $\Delta$ is closed. 

\section*{Problem 5.}

Let $f : X \rightarrow Y$ be continuous and $C \subset Y$ be closed and $D \subset X$ be dense. Let $f(D) \subset C$ then by continuity and Lemma \ref{contclosure}, $f(\overline{D}) \subset \overline{f(D)} \subset \overline{C}$. However, $D$ is dense so $\overline{D} = X$ and $C$ is closed so $\overline{C} = C$. Thus, $f(X) \subset C$.

\section*{Problem 6.} Let $f,g : X \rightarrow Y$ be continuous with $Y$ Hausdorff and let $D \subset X$ be dense. Also let $\forall z \in D : f(z) = g(z)$. Now suppose that $\exists x \in X : f(x) \neq g(x)$. Because $Y$ is Hausdorff, $\exists U, V \in \topo_Y$ s.t. $f(x) \in U$ and $g(y) \in V$ and $U \cap V = \emptyset$. Since $U, V$ are open and $f,g$ are continuous then $\invI{f}{U}$ and $\invI{g}{V}$ are open. Thus, $\invI{f}{U} \cap \invI{g}{V}$ is also open. However, $x \in \invI{f}{U}$ and $y \in \invI{g}{V}$ so $x \in \invI{f}{U} \cap \invI{g}{V}$. Thus, $\invI{f}{U} \cap \invI{g}{V} \neq \emptyset$. By Lemma \ref{denseintersect}, $\exists d \in D$ s.t. $d \in \invI{f}{U} \cap \invI{g}{V}$ but $f(d) = g(d)$ because $d \in D$. However, $d \in \invI{f}{U}$ and $d \in \invI{g}{V}$ so $f(d) \in U$ and $g(d) \in V$ so $f(d) = g(d) \in U \cap V$ which is a contradiction because $U \cap V = \emptyset$. Thus, $\forall x \in X : f(x) = g(x)$ so $f = g$. \\ \\
An alternative solution is given by considering the map $F : X \rightarrow Y \times Y$ given by \[F(x) = (f(x), g(x))\] This function is continuous by a previous homework problem because $f$ and $g$ are continuous. Now, $\forall x \in D : f(x) = g(x)$ so $F(D) \subset \Delta$ but $\Delta$ is closed in $Y \times Y$ because $Y$ is Hausdorff and $D \subset X$ is dense so by the previous problem, $F(X) \subset \Delta$. Therefore, $\forall x \in X : (f(x), g(x)) \in \Delta$ which gives $f(x) = g(x)$ for every $x \in X$.

\section*{Problem 7.}
\begin{enumerate}
\item Suppose that $A$ contains no limit points of itself. Take any $x \in A$ then $x \notin \overline{A \sm \{x \}}$ so $\exists U \in \topo$ s.t. $x \in U$ and $U \cap (A \setminus \{x \}) = \emptyset$. However, $x \in A$ and $x \in U$ so $x \in U \cap A$. Thus, $U \cap A = \{x \}$. But $U$ is open in $X$ so $U \cap A$ is open in $A$. Thus, every $\{x \}$ is open in $A$. For any $S \subset X$, $S = \bigcup\limits_{x \in S} \{x\}$ is open because each $\{x\}$ is open so every set is open in $A$. Conversely, if the subset topology on $A$ in $X$ is discrete then for any $x \in A$ there  must exist $U \in \topo$ s.t. $U \cap A = \{x\}$ because $\{x\}$ is open in $A$. Thus, $U \cap (A \setminus \{x \}) = \emptyset$ so $x$ is not a limit point of $A$ so $A$ contains no limit points. 
    
\item Take $S = \left\{ \frac{1}{n} \mid n \in \Zplus \right \}$ then for any $\delta > 0$ we have that $\exists n \in \Zplus$ s.t. $0 < \frac{1}{n} < \delta$ so $\frac{1}{n} \in \ball{\delta}{0}$ so $0$ is a limit point of $S$. However, for any $\frac{1}{n} \in S$ take $\delta = \frac{1}{n(n+1)}$ and $U = \ball{\delta}{\frac{1}{n}}$. Then $\frac{1}{k} - \frac{1}{n} = \frac{n - k}{nk} \ge \frac{1}{n(n+1)}$ so $U \cap S = \{\frac{1}{n}\}$ thus $S$ is discrete. 
\end{enumerate}

\section*{Lemmas}

\begin{lemma} \label{closedprods}
If $A \subset X$ and $B \subset Y$ are closed in $X$ and $Y$ respectivly, then $A \times B$ is closed in the product topology on $X \times Y$.
\end{lemma}
\begin{proof}
Let $A = X \sm C$ with $C \in \topo_X$ and $B = Y \sm D$ with $D \in \topo_Y$ then \[A \times B = (X \sm C) \times (Y \sm D) = (X \times Y) \sm ((C \times Y) \cup (X \times D))\] but $C \times Y$ and $X \times D$ are open in the product so $(C \times Y) \cup (X \times D)$ is also open and thus $A \times B$ is closed. 
\end{proof}

\begin{lemma} \label{contclosure}
Let $f : X \rightarrow Y$ be continuous and $A \subset X$ then $f(\bar{A}) \subset \overline{f(A)}$.
\end{lemma}
\begin{proof}
Let $y \in f(\bar{A})$ then $y = f(x)$ and $x \in \bar{A}$ thus for any open $U \subset X$, if $x \in U$ then $U \cap A \neq \emptyset$. Take a open $V \subset Y$ and $y \in V$ so $x \in \invI{f}{V}$. But $f$ is continous so $\invI{f}{V}$ is open and $x \in \invI{f}{V}$ so $\exists z \in \invI{f}{V} \cap A$ then $f(z) \in V$ and $z \in A$ thus $f(z) \in f(A)$. Thus, $f(z) \in V \cap f(A)$ so $V \cap f(A) \neq \emptyset$ thus $x \in \overline{f(A)}$.  
\end{proof}

\begin{lemma} \label{denseintersect}
Let $(X, \topo)$ be a topological space and $D \subset X$ be dense then $\forall U \in \topo \sm \{ \emptyset\} : \exists d \in U \cap D$.
\end{lemma}
\begin{proof}
If $U$ is a nonempty open set then $\exists x \in U$. $D$ is dense so $x \in \overline{D}$ thus because $x \in U$ and $U$ is open then we have $\implies U \cap D \neq \emptyset$. Thus, $\exists d \in U \cap D$.    
\end{proof}

\end{document}
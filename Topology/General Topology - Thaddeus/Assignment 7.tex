\documentclass[12pt]{extarticle}
\usepackage{import}
\import{./}{Includes}


\begin{document}
\atitle{7}
 
\section*{Problem 1.}
Let $X$ and $Y$ be topological spaces with $Y$ compact and let $\pi : X \times Y \to X$ be given by $\pi : (x, y) \mapsto x$. Let $C \subset X \times Y$ be closed and, assuming that $\pi(C)$ is not closed, consider $z \in \overline{\pi(C)}\sm \pi(C)$. Now, the preimage satisfies, 
\[\invI{\pi}{\{z\}} \subset (X \times Y) \sm C\]
because if $(x, y) \in \invI{\pi}{\{z\}}$ then $x = z$ but $\pi(x, y) \notin \pi(C)$ so $(x, y) \notin C$. For any $y \in Y$ the point $(z, y) \in \invI{\pi}{\{z\}}$ because $\pi(z, y) = z$. However, $(X \times Y) \sm C$ is open so for each $y \in Y$ there exists open sets $U_y \subset X$ and $V_y \subset Y$ such that, \[(z, y) \subset U_y \times V_y \subset (X \times Y) \sm C\]
Thus, for any $y \in Y$, $y \in V_y$ so $\mathcal{U} = \{V_y \mid y \in Y\}$ is an open cover of $Y$. By compactness, there exists a finite subcover, $\mathcal{U}_S$, indexed by a finite set $S \subset Y$. That is, \[ Y = \bigcup_{y \in S} V_y\]
Now, for each $y \in Y$, we have $z \in U_y$ and therefore, 
\[z \in \bigcap\limits_{y \in S} U_y = A\]     
which is open because $S$ is finite and each $U_y$ is open. However, $z \in \overline{\pi(C)}$ and $z \in A$ which is open in $X$ so $A \cap \pi(C) \neq \empty$. Therefore, $\exists t \in A \cap \pi(C)$ so for each $y \in S$ we have $t \in U_y$ and there exists some $y_t \in Y$ such that $(t, y_t) \in C$. However, $Y$ is covered by $\mathcal{U}_S$ so for some $y \in S$ we have $y_t \subset V_{y_S}$ but $t \in U_{y}$ so $(t, y_t) \in U_{y} \times V_{y}$. However, $(t, y_t) \in C$ which contradicts the fact that $U_y$ and $V_y$ were choosen such that $U_y \times V_y \subset (X \times Y) \sm C$. Thus, $ \overline{\pi(C)}\sm \pi(C)$ is empty but $\pi(C) \subset \overline{\pi(C)}$ so $\overline{\pi(C)} = \pi(C)$ and therefore, $\pi(C)$ is closed.   

\section*{Problem 2.}

Let $X$ be a $T_1$ space. Suppose that $X$ is countably compact. Because every infinite set contains a countable subset (assuming the axiom of countable choice), it suffices to prove that every infinite countable set has a limit point in $X$. Let $\Omega \subset X$ be a countable set with an enumeration given by $x_n \in \Omega$ for $n \in \N$. Suppose that $\Omega$ has no limit points in $X$ then, each $x_n \notin \overline{\Omega \sm \{x_n\}}$ so there exists an open $U_n \subset X$ with $x_n \in U_n$ and $U_n \cap (\Omega \sm \{x_n\}) = \empty$ so $U_n \cap \Omega = \{x_n\}$. Futhermore, because $\Omega$ has no limit points, $\Omega$ is closed so $X \sm \Omega$ is open. Thus, $V_n = U_n \cup (X \sm \Omega)$ is open and $X \sm \Omega \subset V_n$ and $\{x_n\} \in V_n$ however, $\Omega \cap V_n = \Omega \cap U_n = \{x_n \}$ so $V_n = (X \sm \Omega) \cup \{x_n \}$. Therefore, $\{V_n \mid n \in \N \}$ is an open cover of $X$ because,
\[ \bigcup_{n \in \N} V_n = \bigcup_{n \in \N} (X \sm \Omega) \cup \{x_n \} = (X \sm \Omega) \cup  \bigcup_{n \in \N} \{x_n \} = (X \sm \Omega) \cup \Omega = X\]

so by countable compactness, there exits a finite subcover indexed by a finite set $S \subset \N$. However,
\[\Omega \cap \bigcup_{n \in S} V_n = \bigcup_{n \in S} \Omega \cap V_n \subset \bigcup_{n \in S} \{x_n\}\] 
but $\bigcup\limits_{n \in S} V_n  = X$ and $\Omega \subset X$ so $\Omega \subset \{x_n \mid n \in S\}$. However, $S$ is finite and therefore, $\Omega$ is finite. Thus, if $\Omega$ is infinite then it must have a limit point. \\\\
Conversely, let $X$ be limit point compact. Suppose that $\{U_n \mid n \in \N \}$ is a countable open cover of $X$ with no finite subcover. Then, define $x_n \in X \sm (U_1 \cup \dots \cup U_{n})$ which exists because if $X \sm (U_1 \cup \dots \cup U_n) = \empty$ then $U_1 \cup \dots \cup U_n$ is a finite subcover. Let $A = \{x_n \mid n \in \N\}$. Because $\{U_n\}$ is a cover, for any $x \in X$ there exists some $N$ s.t. $x \in U_{N}$ and then for $i \ge N$ we have $x_i \notin U_{N}$ because $x_i \notin U_1 \cup \dots \cup U_N \cup \dots \cup U_i \supset  U_N$. Therefore, $A \cap U_{N} \subset \{x_n \mid n < N\}$ so $A \cap U_{N}$ is finite and thus, $C = A \cap  U_{N} \sm \{x\}$ is also finite. However, $X$ is $T_1$ so for any $y \in X$, the set $\{y\}$ is closed and thus, by finite unions, $C$ is closed. Therefore, $V =  U_{N} \cap (X \sm C)$ is open in $X$ but $x \notin C$ and $x \in U_{N}$ so $x \in V$. $(A \sm \{x\}) \cap V = (A \sm \{x\}) \cap U_{N} \cap (X \sm C) = (A \sm \{x\} \cap U_N) \sm C = \empty$. But $x \in V$ so $x$ is not a limit point of $A$ and thus $A$ is an infinite set with no limit points in $X$ contradicting limit point compactness. Therefore, we cannot have any countable cover without a finite subcover i.e. $X$ is countably compact.     

\section*{Problem 3.}
For nonempty $A, B \subset X$ define $d(A, B) = \inf\{d(x, y) \mid x \in A \text{ and } y \in B \}$ and $d(x, A) = d(\{x \}, A)$.  
\begin{enumerate}
\item $d(x, A) = 0$ iff $\forall \delta > 0 : \exists y_\delta \in A : d(x, y_\delta) < \delta$ so take any open $U \subset X$ with $x \in U$ then $\exists \delta > 0 : x \in \ball{\delta}{x} \subset U$ so $y_\delta \in \ball{\delta}{x} \subset U$ so $U \cap A \neq \empty$. Thus, $x \in \bar{A}$. Conversely, if $x \in \bar{A}$ then $x \in \ball{\delta}{x}$ is open so $\ball{\delta}{x} \cap A \neq \empty$ so $\exists y_\delta \in A$ with $d(x, y_\delta) < \delta$ so $d(x, A) = 0$.  

\item Let $A$ be compact then since $X$ is a metric space it is Hausdorff so $A$ is closed. Take $\delta_n = d(x, A) + 1/n$ and $A_n = C_{\delta_n} (x) \cap A$ where $C_\delta(x) = \{y \in X \mid d(x, y) \le \delta\}$. By Lemma \ref{closedball} and the intersection of closed sets, $A_n$ is closed and $A_n \subset A$ so $A_n$ is compact. Also, $\delta_n > \delta_{n+1}$ so $C_{\delta_n}(x) \supset C_{\delta_{n+1}}(x)$ and thus $A_{n} \supset A_{n+1}$. Furthermore, by approximation property, for any $n \in \N$ there exists $y \in A$ s.t. $d(x, y) < d(x, A) + 1/n$ so $y \in C_{\delta_n}(x) \cap A = A_n$. Thus, the sequence is nonempty. Since $X$ is Hausdorff, the intersection of these compact nonempty nested sets is nonempty. Take \[a \in \bigcap_{n \in \N} A_n\] 
Thus, for every $n$, we have $a \in A_n$ so $a \in C_{\delta_n}(x)$ thus $d(x, a) < d(x, A) + 1/n$ and $a \in A$ so $d(x, A) \le d(x, a)$. If $d(x, A) < d(x, a)$ then we can choose $n > 1/(d(x, a) - d(x, A))$ so $d(x, a) > d(x, A) + 1/n$ contradicting $a \in A_n$. Therefore, $d(x, a) = d(x, A)$ with $a \in A$.    

\item Define $\ball{\delta}{A} = \{ x \in X \mid d(x, A) < \delta\}$. Let $x \in \ball{\delta}{A}$ then $d(x, A) < \delta$ so $\epsilon = \delta - d(x, A) > 0$. Thus, by the approximtion property, there exist $a \in A$ such that $d(x, a) < d(x, A) + \epsilon = \delta$ so $x \in \ball{\delta}{a} \subset \bigcup\limits_{a \in A} \ball{\delta}{a}$. Thus, $\ball{\delta}{A} \subset \bigcup\limits_{a \in A} \ball{\delta}{a}$. \\\\
Conversely, if $x \in \bigcup\limits_{a \in A} \ball{\delta}{a}$ then for some $a \in A$ we have $x \in \ball{\delta}{a}$ so $d(x, a) < \delta$ but $d(x, A)$ is the infimum of all such numbers so $d(x, A) \le d(x, a) < \delta$ so $x \in \ball{\delta}{A}$. Therefore, $\ball{\delta}{A} = \bigcup\limits_{a \in A} \ball{\delta}{a}$.     

\item Let $A$ be compact and $U \subset X$ be open such that $A \subset U$. Because $U$ is open, for each $a \in A$, $\exists \delta_a > 0 : \ball{\delta_a}{a} \subset U$. Now, $\forall a \in A : a \in \ball{\frac{1}{2}\delta_a}{a}$ so $\{\ball{\frac{1}{2}\delta_a}{a} \mid a \in A\}$ is a open cover of $A$. By compactness, there exists a finite subcover indexed by $S \subset A$. Then, $\delta = \frac{1}{2} \min\limits_{a \in S} \delta_a$ exists and is positive. Let $x \in \ball{\delta}{A} = \bigcup\limits_{a \in A} \ball{\delta}{a}$ then, there exists $a_0 \in A$ such that $x \in \ball{\delta}{a_0}$ but $a_0 \in A$ so there must exist $a' \in S$ so that $a_0 \in \ball{\frac{1}{2}\delta_{a'}}{a'}$. Thus, 
\[d(x, a') < d(x, a_0) + d(a_0, a') < \delta + \frac{1}{2} \delta_{a'} \le \delta_{a'}\] because $a' \in S$ so $\delta \le \frac{1}{2} \delta_{a'}$. Therefore, $x \in \ball{\delta_{a'}}{a'} \subset U$ by the definition of $\delta_{a'}$ so $\ball{\delta}{A} \subset U$.       

\item Take $\Zplus \subset \R$ and $U = \bigcup\limits_{n \in \Zplus} \ball{\frac{1}{n}}{n}$. $\Zplus$ is closed in $\R$ because $\R / \Zplus = (-\infty, 1) \cup \bigcup\limits_{n \in \Zplus} (n, n + 1)$ which is a union of open sets and thus open. Also, each $\forall n \in \Zplus : n \in \ball{\frac{1}{n}}{n}$ thus $\Zplus \subset U$. However, suppose that $\ball{\delta}{\Zplus} = \bigcup\limits_{n \in \Zplus} \ball{\delta}{n} \subset U$ then choose $n > \frac{1}{\delta}$. Now, 
\[\ball{\delta}{\Zplus} \cap \ball{\delta}{n} \subset U \cap \ball{\delta}{n}\]
but, $\ball{\delta}{\Zplus} \cap \ball{\delta}{n} = \ball{\delta}{n}$ because $\ball{\delta}{n} \subset \ball{\delta}{\Zplus} = \bigcup\limits_{a \in \Zplus} \ball{\delta}{n}$ and $U \cap \ball{\delta}{n} = \ball{\frac{1}{n}}{n}$ because $\frac{1}{n} < \delta$. Therefore, $\ball{\delta}{n} \subset \ball{\frac{1}{n}}{n}$ which contradicts $\frac{1}{n} < \delta$. Thus, for every $\delta > 0$, $\ball{\delta}{\Zplus} \not\subset U$.     

\end{enumerate}

\section*{Problem 4.}

Let $f : X \to X$ be an isometry of a compact metric space $X$. We know that $f$ is injective and continuous. 

\begin{enumerate}
\item
Suppose that there exists $a \in X$ such that $a \notin f(X)$. Then because $X$ is compact and $f$ is continuous, $f(X)$ is compact. However, $X$ is a metric space so it is Hausdorff. The set $\{a\}$ is compact because it is finite. Since $X$ is Hausdorff, there exists open sets seperating $\{a\}$ and $f(X)$. Specifically, $\exists U, V \in \topo_X$ s.t. $\{a \} \subset U$, $f(X) \subset V$ and $U \cap V = \empty$. Since $a \in U$ is open, $\exists \epsilon > 0 : a \in \ball{\epsilon}{x} \subset U$. Thus, $\ball{\epsilon}{x} \cap V = \empty$ so $\ball{\epsilon}{x} \cap f(X) = \empty$. Because $d(f(x), f(y)) = d(x, y)$, by induction, $d(f^n(x), f^n(y)) = d(x, y)$. Now, consider $x_0 = a$ and $x_{n+1} = f(x_n)$. By induction, $x_n =  f^{n}(a)$. For natural numbers $n > m$ consider, 
\[d(x_n, x_m) = d(f^{n}(a), f^{m}(a)) = d(f^{n - m}(a), a) \ge \epsilon\] 
The last inequalty holds because $n - m \ge 0$ so we have $f^{n-m}(a) \in f(X)$ but $\ball{\epsilon}{a} \subset X \sm f(X)$ so $f^{n-m}(a)) \notin \ball{\epsilon}{a}$. Thus, no subsequence of $\{x_n\}$ can have a limit in $X$ because any ball of radius $\epsilon/2$ can contain at most one $x_i$. This contradicts the sequential compactness of $X$ which follows from the fact that $X$ is a compact metric space. Thus, $f(X) = X$.   

\item We have that $f : X \to X$ is a continuous bijection but $X$ is compact and $X$ is Hausdorff because it is a metric space. Thus, $f$ is a homeomorphism. 
\end{enumerate}   

\section*{Problem 5.}
Let $X$ be a compact metric space and let $f : X \to X$ be a contraction i.e. for $c \in [0, 1)$, \[\forall x, y \in X : d(f(x), f(y)) \le c \cdot d(x, y)\]

\begin{enumerate}
\item Let $U \subset X$ be open. Consider $x \in \invI{f}{U}$ and, equivalently, $f(x) \in U$. Because $U$ is open, $\exists \delta > 0 : f(x) \in \ball{\delta}{f(x)} \subset U$. Suppose that $y \in \ball{\delta}{x}$ then $d(x, y) < \delta$ so, 
\[d(f(x), f(y)) \le c \cdot d(x,y) < c \delta < \delta\] 
therefore $f(y) \in \ball{\delta}{f(x)} \subset U$. Thus, $y \in \invI{f}{U}$. Therefore, $x \in \ball{\delta}{x} \subset \invI{f}{U}$ so $\invI{f}{U}$ is open. Thus, $f$ is continuous. 

\item Since $f$ is continuous and $X$ is compact, $f(X) \subset X$ is compact. Now, suppose that $f^n(X) \subset X$ is compact, then $f(f^{n}(X)) = f^{n+1}(X)$ is compact by continuity. Thus, by induction, $f^{n}(X)$ is compact for all $n \in \N$. Consider,
\[C = \bigcap_{n \in \N} f^{n}(X)\]  
Now if $x \in f^{n+1}(X)$ then $x \in f^{n}(f(X))$ but $y = f(X) \in X$ so $x = f^{n}(y)$ thus $x \in f^{n}(X)$. Therefore, $f^{n+1}(X) \subset f^{n}(X)$. Futhermore, assuming $X$ is nonempty each $f^{n}(X)$ is nonempty. Thus, this sequence of nested nonempty compact sets in a metric space which is thus Hausdorff has a nonempty intersection. Thus, $C \neq \empty$. Take $x \in C$ thus for every $n \in \N$, $x \in f^{n}(X)$ so there is a sequence $y_n$ s.t. $x = f^{n}(y_n)$. \\\\
I claim that for any $a,b \in X$ we have $d(f^{n}(a), f^{n}(b)) \le c^n \cdot d(a, b)$. We proceed by induction: for $n = 1$ this is the definition of a contration. Suppose it holds for $n$ then, \[d(f^{n+1}(a), f^{n+1}(b)) = d(f(f^{n}(a)), f(f^{n}(b))) \le c \cdot d(f^{n}(a), f^{n}(b)) \le c^{n+1} \cdot d(a, b)\]
so the claim holds by induction. 
Now, consider, \[d(x, f(x)) = d(f^{n}(y_n), f^{n}(f(y_n))) \le c^n \cdot d(y_n, f(y_n))\]   
However, $X$ is a compact metric space so by Lemma \ref{compactbounded} it is bounded. Thus, there exists some $B \in \Rplus$ s.t. $\forall a, b \in X : d(a, b) < B$. Therefore, 
\[d(x, f(x)) < c^n \cdot B\]
for every $n \in \N$. However, $c < 1$ so if $d(x , f(x)) > 0$ then there exists some $n \in \N$ s.t. $c^n \cdot B < d(x, f(x))$ which contradicts the above formula. Thus, $d(x, f(x)) = 0$ so $f(x) = x$. This point is unique because if both $x$ and $y$ are fixed by $f$ i.e. $f(x) = x$ and $f(y) = y$, we would have $d(f(x), f(y)) = d(x, y)$ but $d(f(x), f(y)) \le c d(x,y)$ so $d(x, y) \le c \dot d(x,y)$ thus either $1 \le c$ or $d(x, y) = 0$. Since we know $c < 1$ we must have $d(x, y) = 0$ and thus $x = y$. Therefore, $x$ is the unique point such that $f(x) = x$.  
\end{enumerate} 

\section*{Problem 6.}
Take $X = \Rplus$ with the subspace topology in $\R$ with the standard topology. Define the harmonic series \[x_n = \sum_{k = 1}^n \frac{1}{k}\]
with $x_0 = 0$ and let $V_n = (x_{n}, x_{n+2})$ and $U_n = \bigcup\limits_{i = 0}^n V_n$ with $U_0 = (0, x_2)$. Suppose that $U_n = (0, x_{n+2})$ then $U_{n+1} = (0, x_{n+2}) \cup (x_{n+1}, x_{n+3}) = (0, x_{n+3})$ because $x_n$ is an increasing sequence. Thus, by induction, $U_{n} = (0, x_{n+2})$. Since the harmonic series diverges to infinity, for any $r \in \Rplus$ there exists $n \in \N$ s.t. $r < x_n < x_{n+2}$. Therefore, $r \in U_n$ so $r \in V_k$ for some $k \le n$. Therefore, $\{V_n \mid n \in \N \}$ is a open cover of $\Rplus$. However, suppose there existed a Lebesgue number $\delta$. Then take $n + 1 > \frac{1}{\delta}$ and, by the definition of a Lebesgue number, we must have $\ball{\delta}{x_{n + 1}} \subset V_{n}$ because $x_{n + 1} \in (x_{n}, x_{n+2})$ and no other $U_k$. However, $|x_{n+1} - x_{n}| = \frac{1}{n+1} < \delta$ so $x_{n} \in \ball{\delta}{x_{n + 1}} \subset V_n$ but $x_{n} \notin (x_{n}, x_{n+2}) = V_n$ which is a contradiction. Thus, there cannot exist a Lebesgue number for this cover.   
\section*{Lemmas}

\begin{lemma} \label{closedball}
In a metric space $X$ the set $C_\delta(x) = \{y \in X \mid d(x, y) \le \delta\}$ is closed.
\end{lemma}

\begin{proof}
Take $U = X \sm C_\delta(x)$ then $y \in U$ iff $d(x, y) > \delta$. For any $y \in U$ we have $d(x, y) > \delta$ so take $\epsilon = d(x, y) - \delta$ and then for any $z \in \ball{\epsilon}{y}$ we have $d(z, y) < \epsilon$ but $d(x, y) < d(x, z) + d(z, y) < d(x, z) + \epsilon$ so $d(x, z) > d(x, y) - \epsilon = \delta$. Thus, $z \notin C_\delta(x)$ so $z \in U$. Therefore, $y \in \ball{\epsilon}{y} \subset U$ so $U$ is open and thus $C_\delta(x)$ is closed. 
\end{proof}

\begin{lemma} \label{compactbounded}
Let $X$ be a compact metric space then $\exists B \in \Rplus : \forall x, y \in X : d(x, y) < B$. 
\end{lemma}
\begin{proof}
If $X = \empty$ we are done. Else, take $x_0 \in X$, consider the open cover $\{\ball{\delta}{x_0} \mid \delta \in \Rplus\}$ by compactness, there is a finite subcover indexed by a finite set $S \subset \Rplus$ s.t. \[\bigcup_{\delta \in S} \ball{\delta}{x_0} = X\]
But $S$ is finite so $\Delta = \max\limits_{\delta \in S} {\delta}$ exists. Then, 
\[x \in X \implies \exists \delta \in S : x \in \ball{\delta}{x_0} \implies d(x, x_0) < \delta \le \Delta\]  
So define $B = 2 \Delta$ then for any $x, y \in X$ we have $d(x, y) < d(x, x_0) + d(x_0, y) < \Delta + \Delta  = B$. 
\end{proof}

\end{document}

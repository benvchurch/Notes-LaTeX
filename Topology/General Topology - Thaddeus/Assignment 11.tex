\documentclass[12pt]{extarticle}
\usepackage{import}
\import{./}{Includes}

\begin{document}
\atitle{11}
 
\section*{Problem 1.}
\begin{enumerate}
\item Let $X = \R^2 \sm \{0\}$ and define $f : X \to S^1$ by $f(x,y) = \left(\frac{x}{\sqrt{x^2 + y^2}}, \frac{y}{\sqrt{x^2 + y^2}} \right)$ which is well defined because $\frac{x^2}{x^2 + y^2} + \frac{y^2}{x^2 + y^2} = 1$. First, because the function is resricted to $(x, y) \neq 0$ so $\sqrt{x^2 + y^2} \neq 0$ and thus $f$ is continuous by analysis. Futhermore, if $(x,y) \in S^1$ then $x^2 + y^2 = 1$ so $f(x,y) = (x,y)$ so $f \circ i_{S^1} = \id_{S^1}$. So $f$ is a retract. \bigskip \\
Furthmore, let $H : X \times I \to X$ by $H(x, y, t) = \left(tx + (1 - t)\frac{x}{\sqrt{x^2 + y^2}}, ty + (1 - t)\frac{y}{\sqrt{x^2 + y^2}} \right)$. Again because $(x, y) \neq 0$ this function is continuous by analysis. Now, 
\[H(x,y,0) = \left(\frac{x}{\sqrt{x^2 + y^2}}, \frac{y}{\sqrt{x^2 + y^2}} \right) = i_{S^1} \circ f(x, y)\]
Furthermore, $H(x,y,1) = (x,y) = \id_X$. Also, if $(x, y) \in S^1$ then $x^2 + y^2 = 1$ so $H(x,y,t) = (tx + (1 - t) x, ty + (1 - t)y) = (x, y)$. Thus, $f$ is a deformation retract. 
\item If $f : X \to A$ is a retract then for any $x_0 \in A$ the map $f_{*} : \pi_1(X, x_0) \to \pi_1(A, x_0)$ is a surjection. However, $\pi_1(S^1, (1, 0)) \cong \Z$ and $\pi_1(\R^2, (1,0)) \cong \{e\}$ because $\R^2$ is convex. Therefore, there does not exist a surjection from $\pi_1(S^1, (1, 0))$ to $\pi_1(\R^2, (1,0))$ because the former is larger then the latter. Therefore, $S^1$ is not a retract of $\R^2$. 
\end{enumerate}

\section*{Problem 2.}
\begin{enumerate}
\item Let $f : X \to A$ be a deformation retract and $a \in A$. Then, there exists a homotopy from $\id_X$ to $i \circ f$ i.e. $H : X \times I \to X$ such that $H(x, 0) = x$ and $H(x, 1) = i \circ f(x) $ and if $x \in A$ then $H(x, t) = x$. Therefore, for $x \in A$, $f(x) = H(x, 1) = x$ so $f$ is a retraction and thus the induced homomorphism $f_{*} : \pi_1(X, a) \to \pi_1(A, a)$ is a surjection. It suffices to show that $f_{*}$ is also an injection. Let $\gamma : I \to X$ be a loop at $a$. Then consider the map $G : I \times I \to X$ given by $G(x, t) = H(\gamma(x), t)$. Now, $G = H \circ (\gamma \times \id_I)$ which is a composition of continuous maps and therefore continuous.  

\begin{center}
\begin{tikzcd}[column sep=large]
I \times I \arrow[r, "\gamma \times \id_I"] \arrow[rr, "G", bend right = 30, dashed] & X \times I \arrow[r, "H"] & X 
\end{tikzcd}
\end{center}

However, $G(x,0) = H(\gamma(x), 0) = \gamma(x)$ and $G(x, 1) = i \circ f \circ \gamma(x)$ and $G(0, t) = H(a, t) = a$ and $G(1, t) = H(a, t) = a$ because $a \in A$. Thus, $G$ is a path-homotopy between $\gamma$ and $i \circ f \circ \gamma$. Suppose that $f_{*}([\gamma_1]) = f_{*}([\gamma_2])$ then $[f \circ \gamma_1] = [f \circ \gamma_2]$ so $f \circ \gamma_1 \sim f \circ \gamma_2$. Therefore, $i \circ f \circ \gamma_1 \sim i \circ g \circ \gamma_2$ because $i : A \to X$ is continuous. However, $\gamma_1 \sim i \circ f \circ \gamma_1$ and similarly $\gamma_2 \sim i \circ f \circ \gamma_2$ so by transitivity, $\gamma_1 \sim \gamma_2$ so $[\gamma_1] = [\gamma_2]$. Therefore, $f_{*}$ is an injection.   



\item Let $T = S^1 \times S^1$ i.e. the torus embedded in $\R^4$ and $x_0 = (1, 0) \in S^1$. Consider the projection $\pi_1 : S^1 \times S^1 \to S^1$ which is continuous by the definition of the product topology. Let $s : S^1 \to S^1 \times \{x_0\}$ be the map $s : x \mapsto (x, x_0)$. I claim that the map $f = s \circ \pi_1 : X \to S^1 \times \{x_0\} = A$ is a retraction. This is because if $p \in A$ then $p = (x, x_0)$ so $\pi_1(p) = x$ so $s(\pi_1(p)) = (x, x_0)$. Therefore, $f \circ i_A = \id_A$. So $A$ is a retract of $T$. However, 
\[\pi_1(T, x_0 \times x_0) = \pi_1(S^1 \times S^1, x_0 \times x_0) \cong \pi_1(S^1, x_0) \times \pi_1(S^1, x_0) \cong \Z \times \Z\]
Similarly, \[\pi_1(S^1 \times \{x_0 \}, x_0 \times x_0) \cong \pi_1(S^1, x_0) \times \pi_1(\{x_0\}, x_0) \cong \Z \times \{e\} \cong \Z\] 
By the folowing problem, $\Z \times \Z \not \cong \Z$ so $\pi_1(T, x_0 \times x_0) \not \cong \pi_1(S^1 \times \{x_0\}, x_0 \times x_0)$. Therefore, $S^1 \times \{x_0\}$ cannot be a deformation retract of $T$ because otherwise the fundamental groups would be isomorphic.     
\end{enumerate}    

\section*{Problem 3.}

\begin{enumerate}
\item Suppose that $\varphi : \Z \to \Z \times \Z$ is a homomorphism. Then, take $\phi(1) = (a, b) \in \Z \times \Z$. Because $\varphi$ is a homomorphism, $\varphi(n) = (an, bn)$. However, if $(1, 0) \in \Im{\varphi}$ then $b = 0$ since $n \neq 0$ if $an = 1$. Similarly, if $(0, 1) \in \Im{\varphi}$ then $a = 0$ which contradicts the claim that $an = 1$. Thus one of $(1,0)$ or $(0,1)$ is not in the image of $\varphi$ so the map cannot be surjective. 

\item $\pi_1(S^3, x_0) \cong \{e\}$ and $\pi_1(S^2 \times S^1, x_0' \times y_0) \cong \pi_1(S^2, x_0') \times \pi_1(S^1, y_0) \cong \{e \} \times \Z \cong \Z$ and, \[\pi_1(S^1 \times S^1 \times S^1, y_0 \times y_0 \times y_0) \cong \pi_1(S^1, y_0) \times \pi_1(S^1, y_0) \times \pi_1(S^1, y_0) \cong \Z \times \Z \times \Z \]
No two of these groups are isomorphic. The trivial group has one element which cannot be put into bijection with either infinite group. Futhermore, if there existed an isomorphism between $\Z$ and $\Z^3$ then by composing this map with the projection down to $\Z^2$ we would obtain a surjective homomorphism from $\Z$ to $\Z^2$ which we proved was impossible above. Thus, no two of these spaces are isomorphic.  
\end{enumerate}

\section*{Problem 4.}

In an exactly analogous fashion to question 1 (a), $S^{n - 1}$ is a deformation retract of $\R^n \sm \{0\}$. Therefore, $\pi(\R^n \sm \{0\}, x_0) \cong \pi_1(S^{n - 1}, x_0)$ so for $n > 2$ the fundamental group of $\R^n \sm \{0 \}$ is trivial because $S^{k}$ is simply connected for $k > 1$. However, $\pi_1(\R^2 \sm \{0 \}, x_0) \cong \pi_1(S^1, x_0) \cong \Z$. Therefore, $\R^2 \sm \{0 \}$ is not homeomorphic to $\R^n \sm \{0 \}$ for $n > 2$. However, if $\R^2 \cong \R^n$ then the subspace $\R^2 \sm \{0\}$ is homeomorphic to $\R^n \sm \{x\}$ which is homeomorphic to $\R^n \sm \{0\}$ by shifting. However the fomer is not simply connected and the latter is which is a contradiction. 

\section*{Problem 5.}

Let $X = Y = S^1$ and take the universal cover $\tilde{X} = \R$ with the standard covering map $g : \tilde{X} \to Y$ given by $g(r) = e^{2 \pi i r}$. Also, take $p : Y \to X$ be the covering map given by $p(z) = z^n$. Now, the continuous map $p$ induces an injective homomorphism $p_* : \pi_1(Y) \to \pi_1(X)$. Now, $\pi_1(Y) \cong \Z$ so the entire homomorphism is determined by the image of the generator. The path $\gamma : I \to Y$ given by $\gamma(t) = e^{2 \pi i t}$ generates the entire group because the path $\tilde{\gamma} : I \to \R$ given by $\tilde{\gamma}(t) = t$ is a lift of $\gamma$ to the universal cover since $g \circ \tilde{\gamma}(t) = e^{2 \pi i t} = \gamma(t)$. Therefore, $\gamma$ corresponds to the deck transformation taking $0$ to $1$ which generates the group of integer shifts. Thus, $\gamma$ generates $\pi_1(Y)$. Futhermore, $p_*([\gamma]) = [p \circ \gamma]$ where $p \circ \gamma(t) = (e^{2 \pi i t})^n = e^{2 \pi n i t} = \gamma^n(t)$ since $\gamma^n$ corresponds to a shift by $n$ in the group of deck transformations of $\tilde{X}$ over $X$. Therefore, the generator of $\pi_1(Y)$ is mapped to the $n^{\mathrm{th}}$ power of the generator of $\pi_1(Y)$ and thus, $p_*(\pi_1(Y)) = (\pi_1(X))^n \cong n \Z$. However, $n \Z \triangleleft \Z$ so $p_*(\pi_1(Y)) \triangleleft \pi_1(X)$ and by a theorem from class, this implies that $p : Y \to X$ is a Galois cover. Furthermore, when $p : Y \to X$ is a Galois cover, the group of deck transformations is given by, \[D_{Y \to X} \cong \pi_1(X) /p_*(\pi_1(Y)) \cong \Z/n\Z\]

\section*{Problem 6.}

Take the map $p : Y \to X$ which is a $3$-fold cover. Let $f \in D_{Y \to X}$ be a deck transformation. Consider the point $-1 \in S^1 \subset X$ whose preimage is $\pi^{-1}(-1) = \{r_1, r_2, r_3\}$ where $r_1 = (-1, a)$ and $r_2 = (i, b)$ and $r_3 = (-i, b)$. Let $\gamma : I \to S$ be the loop at $-1$ given by $\gamma(t) = - e^{2 \pi i t}$ which goes arround the left circle $S^1$ once counterclockwise. This path lifts uniquely at $r_1$ to the path $\tilde{\gamma}_1(t) = (-e^{2 \pi i t}, a)$. In particular, the lifted path is a loop. However, the lift of $\gamma$ at $r_2$ and $r_3$ are paths $\tilde{\gamma}_2(t) = (ie^{\pi i t}, a)$ and $\tilde{\gamma}_3(t) = (-ie^{\pi i t}, a)$   respectivly which are not loops but instead go arround half circles in $Y$. Suppose that $f(r_1) \neq r_1$ then, since $p \circ f = p$, we have $f(r_1) \in p^{-1}(x_0)$ so WLOG assume that $f(r_1) = r_2$. Then, $f \circ \tilde{\gamma}_1$ is a path in $Y$ such that $f \circ \tilde{\gamma}(0) = f(r_1) = r_2$ and $p \circ (f \circ \tilde{\gamma}_1) = (p \circ f) \circ \tilde{\gamma}_1 = p \circ \tilde{\gamma}_1 = \gamma$. Thus, $f \circ \tilde{\gamma}_1 = \tilde{\gamma}_2$, the unique lift of $\gamma$ at $r_2$. However, $\tilde{\gamma}_1(0) = \tilde{\gamma}_1(1)$ and therefore $f \circ \tilde{\gamma}_1(0) = f \circ \tilde{\gamma}_1(1)$ so $\tilde{\gamma}_2(0) = \tilde{\gamma}_2(1)$ which contradicts the fact that the unique lift at $r_2$ is not a closed loop. The same argument applies if $f(r_1) = r_3$ were assumed. Therefore, $f(r_1) = r_1$ so by Lemma \ref{freeaction}, $f = \id$ because the deck transformations act freely. Therefore, $D_{Y \to X} \cong \{\id\}$, the group of deck transformations is trivial. 
              
\section*{Lemmas}

Note: I assume that all spaces are path-connected

\begin{lemma} \label{freeaction}
Let $p : Y \to X$ be a covering map, if $f \in D_{Y \to X}$ is such that $f(y_0) = y_0$ for some $y_0 \in p^{-1}(x_0)$ then $f = \id$. Equivalently, the action of $D_{Y \to X}$ on a fiber $p^{-1}(x_0)$ is free. 
\end{lemma}

\begin{proof}
Suppose that $f(y_0) = y_0$ and take any $y \in Y$. Take a path $\gamma$ from $y_0$ to $y$ and consider its image under $p$. The path $p \circ \gamma$ takes $x_0$ to $p(y)$. By the path lifting lemma, there exists a unique lift of $p \circ \gamma$ at $y_0$. However, $\gamma(0) = y_0$ so $\gamma$ is clearly the unique lift. However, $f \circ \gamma(0) = f(y_0) = y_0$ and $p \circ (f \circ \gamma) = (p \circ f) \circ \gamma = p \circ \gamma$ because $f$ is a deck transformation. Therefore, $f \circ \gamma = \gamma$ by uniqueness. In particular, $f \circ \gamma(1) = f(y) = \gamma(1) = y$. Therefore, $f = \id$.    
\end{proof}



\end{document}

\documentclass[11pt]{amsart}
\usepackage{import}
\import{"../Algebraic Geometry/"}{AlgGeoCommands}

\newcommand{\Loc}[1]{\mathfrak{Loc}\left( #1 \right)}
\newcommand{\AbGrp}{\mathbf{AbGrp}}

\newtheorem*{defnn}{Definition}
\newtheorem*{conj}{Conjecture}

\usepackage{hyperref}

\usepackage{fancyhdr}

\pagestyle{fancy}
\fancyhead[LH]{\textbf{Benjamin V. Church}}
\fancyhead[RH]{\textbf{Statement of Purpose: Bonn Department of Mathematics}}
\setlength{\headheight}{15pt}
\setlength{\headsep}{0.2in}

\begin{document}

Mathematical research is what I want to spend my life doing. Although my specific interests have encompassed a wide array of topics in the sciences including mathematical modeling, physics, chemistry, and pure mathematics, I knew from a young age that I wanted to be a researcher and an academic. In high school, a summer course in number theory opened my eyes to the intricate, subtle, and surprising patterns seemingly the most familiar and mundane objects: the integers. As my courses abstracted and generalized, I began to glimpse the startling depth and overwhelming beauty that springs forth from even basic objects and their interelations. My college career solidified this amorphous pull into a strengthened and focused passon. Having completed a broad selection of upper-level mathematics coursework and dipped my toes into the world mathematical research, I am more excited than ever to pursue these mysteries in a doctoral program at Bonn where I hope to collaborate with innovative researchers and build my career as an academic.
\par
My primary research interests in algebraic geometry, arithmetic geometry, and number theory have been predominantly shaped by undergraduate research and independent study directly with professors. Working closely with professors through independent study was an invaluable opportunity to explore interesting topics, gain vital insight into the field, and prepare for graduate-level work. In spring 2018, I studied elliptic curves with Professor David Hansen initially from the perspective of number theory as I was concurrently taking graduate-level coursework in class field theory. This independent study was my first exposure to algebraic geometry and I became so enamored with the subject that I pivoted the course towards algebraic curves. Since then, algebraic geometry continues to by my primary interest. That said, I am fascinated also by the interplay between number theory, complex geometry of elliptic curves, and geometry in positive characteristic. To deepen my understanding of these interelations, I learned about modular forms and Galois representations through an independent study with Professor Chao Li covering Diamond and Shurman's book. 
\par
Although I have worked on a variety of research projects reflecting my broad interests in the sciences, especially in interdiciplinary work, mathematical research experience has been the most influential on my academic interests. I participated in the 2018 Columbia math REU studying the zeta functions of diagonal weighted-projective surfaces over finite fields using computational methods. We aimed to generalize Shioda’s classification of supersingular Fermat varieties to weighted-projective diagonal hypersurfaces. However, the naive generalization given by applying Shioda’s classification to the minimal covering Fermat surface provided a sufficient but not necessary condition for supersingularity. We approached the problem from an arithmetic perspective by using a result of Weil \cite{weil_counting} to compute the zeta functions of these diagonal hypersurfaces in terms of Gaussian sums. We then applied Stickelberger's theorem to determine the factorization of ideals generated by Gaussian sums and thus determine the roots and poles of the zeta function corresponding to eigenvalues of the Frobenius action on the variety’s l-adic cohomology reducing the problem to a numerical condition on the exponents and characteristic. Using a computer search, I was able to identify patterns in certain new examples of supersingular surfaces. From this observation, I proved the existence of an infinite family of supersingular weighted-projective surfaces such that the minimal Fermat surface parametrizing them fails to be supersingular. We then identified other infinite families with this same property. This project solidified my love of algebraic geometry, especially geometry in positive characteristic and its relations to arithmetic, and introduced me to the Weil conjectures. Learning about the Weil conjectures inspired me to study scheme theory and \etale cohomology, devoting myself to EGA, Hartshorne exercises, and Milne's \etale cohomology in order to understand the proofs by Grothendieck and Deligne. The introduction of \etale cohomology theory to explain, geometrically, these arithmetic phenomena about rational functions and counting solutions to polynomials over finite fields remains my absolute favorite piece of math.
\par
The summer of 2019, I had the wonderful opportunity to study toric geometry and inequalities in convex geometry in Paris through a joint program between Columbia and Paris Diderot University. Under the direction of Prof Huayi Chen, my group studied the relationship between intersection pairings of big nef divisors and inequalities in convex geometry and the relationships between these inequalities and constructions on toric varieties. Associated to such divisors are compact convex sets called Okunkov bodies whose volume reflects the intersection pairing and asymptotic number of sections. Variants of the Brunn-Minkowski, Alexandrov-Finchel, and isoperemetric inequalities applied to these Okunkov bodies can be strengthened by introducing probabilistic techniques \cite{probabiliste}. Specifically, an important term arising in these inequalities is the correlation between convex bodies which has a form similar to a Kantorovich optimal transport problem. We established an upper bound on the correlation between special cases of convex pairs using the Brenier map and Knothe map whose Jacobians and thus the Radon-Nikodym derivative of the induced measure can be bounded purely in terms of the volumes of the convex bodies.  
\par
My senior year, I focused primarily on independent study and my undergraduate thesis work. I worked through Deligne's proof that Hodge cycles on abelian varieties are absolutely Hodge in my independent study with Professor Michael Harris. The methods used in Delinge's proof brought together nearly all my mathematical studies from algebraic geometry, number theory and Galois theory, representation theory, and algebraic topology, linking together disparate pieces to form an incredibly beautiful, powerful, and surprising argument. 
Comparison theorems between cohomology theories played a major role which, coupled with Prof. Johan de Jong's seminar on Weil cohomology theories and algebraic de Rham cohomology I was concurrently attending, inspired me to read further about this cohomology zoo. In doing so, I read about Hodge theory to relate algebraic and smooth de Rham cohomology theories and Grothendieck's ``Tohoku'' paper to understand the cohomological methods like hypercohomology and universal $\delta$-functors employed in Milne's text. I also read Milne's notes on motives to understand the motivic perspective on absolute Hodge cycles presented in Milne's treatment. An important ingredient of Deligne's proof involved constructing a family of abelian varieties over a moduli space such that a distinguished fiber is of CM-type. This moduli space turns out to be a Shimura variety. The following semester, I continued independent study with Prof. Harris on Shimura varieties which solidifed my interest in arithmetic geometry. 

\par
The Paris Diderot program gave me a solid background in toric geometry which prepared me for my thesis topic. Under Prof Johan de Jong, I studied the problem of embedding smooth curves in toric surfaces. Using a result of Harris and Mumford \cite{harris1982kodaira}, I showed that very general curves cannot be embedded in any toric surface and I gave examples showing obstructions to these embeddings being Cartier. This project was inspired by a paper of Dokchitser \cite{models_of_curves} which provides an algorithm to construct, for a given curve, the minimal regular normal crossings model defined over a DVR. Dokchitser's method requires embedding the curve in a toric surface and constructs the model through gluing semi-toric surfaces associated to subdivisions of the Newton polygon. I constructed a degeneration of a genus 5 curve with a nontrivial Galois action on the components of its special fiber and showed that such a regular normal crossings model cannot result from Dokchitser's method. For my thesis, I was awarded the John Dash van Buren Jr. Prize in Mathematics. This work was an invaluable learning experience about how research is conducted in mathematics: what techniques to try, how to manage frustration, and how problems evolve. It clinched my decision to pursue graduate studies in mathematics. This problem was extremely rewarding and, largely due to Prof de Jong’s excellent mentorship and our working relationship, I am fully convinced that mathematical research is what I want to spend my life doing.  
\par
I recognize that research is not the only duty of a graduate student at [SCHOOL]. Teaching is also I major part of who I am and what I want to spend my life doing. I have been passionate about teaching since high school when I organized a student-run student-taught seminar propoting interest in pure mathematics. I presented elementary topics which would not appear in the standard high school curriculum which I thought would spark their interests. Examples include the Euler line of a triangle, the correspondence between Mersenne primes and perfect numbers, and Liouville numbers which give constructible examples of transcendental numbers. My proudest moment of high school was when another student told me afterwards: "I didn't know math could be so ... fun". Since then, teaching has remained a important part of my life. I have volunteered to teach over 20 splash classes at Columbia and MIT mostly on topics in physics and mathematics and I taught a six week course on elliptic curves for high school students through HSSP. Last year, in collaboration with the Columbia Association for Women in Mathematics, I helped create introductory talks, materials, and help sessions aimed at supporting freshman who were new to proof-based college level mathematics courses. I produced and edited a large portion of the materials which led new students through proof methods by example and I led guided help sessions and lectures to asist new students in putting these methods into practice. I was specifically asked by Prof. Brian Cole to serve as a teaching assistent for his accelerated physics course which I did for two years during which I taught weekly recitations on his recommendation\footnote{this is quite unusual for undergraduate TAs at Columbia, usually recitations are taught by graduate students}. Teaching these recitations was one the highlights of my college career as I got the pleaure of leading the next cohort of eager students through what had fascinated me about the subject and seeing that same fascination mirrored in them.   



My interests have coalesced around algebraic geometry, specifically birational geometry in positive characteristic. Currently, I am particularly interested in the Shioda conjecture and supersingular algebraic surfaces. Immediately following the completion of my senior thesis, I began working on a related problem (outlined in my research statement) under the direction of Prof de Jong. I intend to pursue this question further during my graduate studies as well as incorporating and exploring the deep interplay between algebraic geometry and branching out into related areas in number theory and complex geometry. I also hope to attend a graduate program with a strong mathematical physics group because I continue to be extremely interested in the ways algebraic geometry finds applications in modern physics. I hope that in my graduate studies, I will be able to unify these two interests towards solving research problems on the boundary between mathematics and physics. My ultimate goal is to become a professor. This goal goes beyond having a research position; I see teaching as a vital aspect of my career both in order to repay the excellent instruction I was given and because I believe teaching others to be an invaluable part of my own intellectual development.

The arithmetic algebraic geometry group at Bonn is world-class and extremely innovative. I am particularly inspired by Prof. Scholze’s introduction of perfectoid spaces, prismatic cohomology, and proofs of foundational results on the $B_{\mathrm{dR}}^+$-affine Grassmannian which are some of the most significant contributions to modern arithmetic geometry. Additionally, the partnership between the University of Bonn and the Max Planck Institute for Mathematics at Bonn, which specializes in topics relating to my interests such as arithmetic geometry and moduli problems, would provide me even more avenues to explore my academic interests and situate myself in the field. In particular, Prof. Gerd Faltings’ proof of the Mordell conjecture and more recent work on p-adic period domains are massive contributions which interest me significantly. The work of Prof. David Hansen, who advised my independent study of elliptic curves at Columbia and inspired my interest in algebraic geometry, on completed cohomology of Shimura varieties and his proof of Artin-Grothendieck vanishing in the rigid setting interests me significantly. My passion for mathematics has grown into dedication which motivated me to pursue intense coursework, research projects, and tenacious self-study. This dedication has prepared me with the technical background and perseverance necessary to thrive in Bonn’s rigorous PhD program. I strongly believe that Bonn an ideal place for my intellectual development and mathematical studies and I sincerely hope to be given the opportunity to learn from and contribute to this vibrant community.  

Thank you for your consideration. 


\bibliographystyle{unsrt}
\bibliography{bibliography}


\end{document}
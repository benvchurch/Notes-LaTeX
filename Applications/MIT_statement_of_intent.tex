\documentclass[10pt]{amsart}
\usepackage{import}
\import{"../Algebraic Geometry/"}{AlgGeoCommands}

\newcommand{\Loc}[1]{\mathfrak{Loc}\left( #1 \right)}
\newcommand{\AbGrp}{\mathbf{AbGrp}}

\newtheorem*{defnn}{Definition}
\newtheorem*{conj}{Conjecture}

\usepackage{hyperref}
\newcommand{\M}{\mathcal{M}}

\usepackage{fancyhdr}

\pagestyle{fancy}
\fancyhead[LH]{\textbf{Benjamin Church}}
\fancyhead[RH]{Research Plan}
\setlength{\headheight}{15pt}
\setlength{\headsep}{0.2in}

\begin{document}

Mathematical research is what I want to spend the rest of my live doing. The limited research experience ... 

\par

I recognize that research is not the only duty of a graduate student. Teaching is also I major part of who I am and what I want to spend my life doing. I have been passionate about teaching since high school when I oganized a student-run student-taught seminar propoting interest in pure mathematics. I presented elementary topics which would not appear in the standard high school curriculum which I thought would spark their interests. Examples include the Euler line of a triangle, the correspondence between Mersenne primes and perfect numbers, and Liouville numbers which give constructible examples of transcendental numbers. My proudest moment of high school was when another student told me afterwards: "I didn't know math could be so ... fun". Since then, teaching has remained a important part of my life. I have volunteered to teach over 20 (CHECK THIS NUMBER) splash classes at Columbia and MIT mostly on topics in physics and mathematics and I taught a six week course on elliptic curves for high school students through HSSP. Last year, in collaboration with the Columbia Association for Women in Mathematics, I helped create introductory talks, materials, and help sessions aimed at supporting freshman who were new to proof-based college level mathematics courses. I produced and edited a large portion of the materials which led new students through proof methods by example and I led guided help sessions and lectures to asist new students in putting these methods into practice. I was specifically asked by Prof. Brian Cole to serve as a teaching assistent for his accelerated physics course which I did for two years during which I taught weekly recitations on his recommendation\footnote{this is quite unusual for undergraduate TAs at Columbia, usually recitations are taught by graduate students}. Teaching these recitations was one the highlights of my college career as I got the pleaure of leading the next cohort of eager students through what had fascinated me about the subject and seeing that same fascination mirrored in them.   

\end{document}
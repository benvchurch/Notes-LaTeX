\documentclass[11pt]{article}
\usepackage{import}
\import{"../Algebraic Geometry/"}{AlgGeoCommands}

\newcommand{\Loc}[1]{\mathfrak{Loc}\left( #1 \right)}
\newcommand{\AbGrp}{\mathbf{AbGrp}}

\newtheorem*{defnn}{Definition}
\newtheorem*{conj}{Conjecture}

\usepackage{hyperref}
\usepackage{fancyhdr}

\geometry{margin=1in}

\pagestyle{fancy}
\fancyhead[LH]{\fontsize{8}{12} \textbf{Benjamin V. Church -- ID: \# 57998482}}
\fancyhead[RH]{\fontsize{8}{12} \textbf{Personal Statement: Rackham Graduate School}}
\setlength{\headheight}{15pt}
\setlength{\headsep}{0.2in}

\usepackage[backend=bibtex, citestyle=apa, style=phys]{biblatex}
\addbibresource{bibliography.bib}

\begin{document}
I aspire to earn a doctorate and pursue a career in academia because I am passionate about both research and teaching. My entire life, I have had the good fortune to be taught and guided by people who encouraged my curiosity and supported my studies, turning my interests into a dedication to research. In doing so, these people illustrated to me the importance of committed and passionate teachers at every level of education. Knowing the debt I owe to my mentors and the certain advantages I have had in access to these people and institutions, I have been involved in promoting equitable higher education through student groups, a commitment I intend to continue through graduate school. Additionally, given the profound impact that inspirational teachers have had on my own life and educational pursuits, it is my goal to someday be that sort of teacher for students of my own. 
\par
The internet, especially MIT OpenCourseWare, was an extremely influential resource in my adolescence. Early in high school, I devoured Walter Lewin’s lectures on physics and Gilbert Strang's linear algebra. Soon, the world became my laboratory; I began carrying around a polarizer film in my pocket wherever I went in order to investigate atmospheric optical phenomena. By this point, I was convinced I would become a physicist. Encouraged by a high school science teacher, I signed up to take the US Physics Olympiad qualifying exam. Although I qualified for the final round, my school had no logistics in place to proctor the exam so instead I convinced my English teacher to stay late after school as a proctor. I won a gold medal but narrowly missed the cutoff for the US team. Although I also participated in math competitions, the problems did not capture my imagination the way physics did. That changed when I enrolled in a summer number theory course. I was awestruck; the subtle and surprising proofs immediately captivated me, offering the same thrill of discovery that had drawn me to physics. Moreover, these beautiful and intricate patterns were drawn out of the most familiar and mundane objects: the integers. Just as with physics, I became enthralled by uncovering the wonders that lay hiding in plain sight.
\par
When I entered college, I dove head-first into both mathematics and physics coursework as well as a variety of research projects. It was again a course in number theory, this time taught by Prof.\ Michael Harris, which drew me deeper into pure mathematics. Bit by bit, I began to glimpse the startling depth and overwhelming beauty that springs forth from even basic mathematical objects and their interrelations, solidifying my curiosity into a focused passion. After Prof.\ Harris’s course, I began to seriously consider a career in mathematics and therefore I decided to apply to mathematics REUs. These research opportunities, as well as my undergraduate thesis work with Prof.\ Aise Johan de Jong, showed me that an academic career in mathematics is truly what I want to spend my life doing.
\end{document}
\documentclass[12pt]{article}
\usepackage{hyperref}
\hypersetup{
    colorlinks=true,
    linkcolor=blue,
    filecolor=magenta,      
    urlcolor=blue,
}

\usepackage{import}
\import{"../Algebraic Geometry/"}{AlgGeoCommands}

\newcommand{\Loc}[1]{\mathfrak{Loc}\left( #1 \right)}
\newcommand{\AbGrp}{\mathbf{AbGrp}}

\renewcommand{\tr}{\operatorname{tr}}

\newcommand{\LL}{\mathbb{L}}
\newcommand{\ob}{\mathrm{ob}}
\newcommand{\cM}{\mathcal{M}}
\newcommand{\cT}{\mathcal{T}}
\newcommand{\vir}{\mathrm{vir}}
\newcommand{\cN}{\mathcal{N}}

\begin{document}


\section{Foliation-obstruction setup}

\newcommand{\Ram}{\mathrm{Ram}}

Let $X$ be a smooth proper scheme over a noetherian scheme $S$. Let $s : \Spec{k} \embed S$ be a closed point and $C / k$ a smooth proper curve. Consider a separable map $f : C \to X$ (eg an immersion). Consider $P := \P(\Omega_{X/S}^1)$ then by base change $\P(\Omega_{X_s//k}) = P \times_S k$ to which by separability there is a lift $f' : C \to \P(\Omega_{X_s/k})$.

\begin{proof}
Let $L := \struct{P_s}(1)$ be the tautological bundle on $P$, and $\chi_C = 2 - 2 g(C)$ the geometric Euler characteristic of the curve. Then,
\[ L \cdot_{f'} C = - \chi_C - \Ram_f \le - \chi_C \]
\end{proof}

\begin{proof}
This is immediate by definition: the map $\pi^* \Omega_X \to L$ pulls back to $f^* \Omega_X \onto \L \subset \Omega_C$ hence 
\[ L \cdot_{f'} C = \deg{\L} = \deg{\Omega_C} - \Ram_f \] 
\end{proof}

Now suppose that $f : C \to X$ is invariant by a foliation $\F$ by curves. Outside of the singular locus $Z$ the foliation defines a section of $\P(\Omega_{X/S}) \to X$ whose closure over $Z$ is a blowup  $\pi : \wt{X} \to X$ with center supported in $Z$. Therefore, there is an exceptional divisor $E$ and $L|_{\wt{X}} = \pi^* K_{\F}(-E)$ and provided that $f$ doesn't factor through $Z$, the intersection $E \cdot_{f'} C$ can be identified with the Segre class $s_Z(f)$ so that,
\[ K_{\F} \cdot_{f} C = - \chi_C - \Ram_f + s_Z(f) \]

The main result we need is in controlling this Segre class $s_Z(f)$.


\section{Bigness of $K_\F$}

What is the condition on $\omega$ for it to define a foliation with $K_\F$ big?

Consider, $Y \to V(\omega) \subset \P(\Omega_X)$ the desingularization. Denote the maps $p : Y \to \P(\Omega_X)$ and $f : Y \to X$. Then we have,
\begin{center}
\begin{tikzcd}
f^* \Omega_X \arrow[d] \arrow[r] & \Omega_Y \arrow[d]
\\
L \arrow[r] & \pushout Q
\end{tikzcd}
\end{center}
Let $L = p^* \struct{X}(1)$.
Since $f^* \Omega_X \to L$ is surjective, it is immediate that $Q = \coker{(\ker{(f^* \Omega_X \to L)} \to \Omega_Y)}$.
Then we can take the torsion-free quotient of $Q$ which is rank $1$ and hence is $\I_Z \cdot \M$ for some $Z \subset Y$ closed subscheme of codimenson $2$. 

\begin{lemma}
Let $X$ be a smooth variety and $\F$ a torsion-free coherent sheaf of rank $1$. Then there is a unique decomposition $\F \cong \I_Z \cdot \L$ where $\L$ is a line bundle and $\I_Z$ is the ideal sheaf of a codimension $2$ subscheme.
\end{lemma}

\begin{proof}
Since $\F$ is torsion-free the map $\F \to \F^{\vee \vee}$ is injective. Furthermore, $\F^{\vee \vee}$ is a line bundle since it is reflexive of rank $1$ on a smooth variety. Let $\L = \F^{\vee \vee}$ and consider $\F \ot \L^{-1} \embed \struct{X}$. It suffices to show that this map is an isomorphism in codimension $1$. Indeed, if $\dim{\stalk{X}{x}} \le 1$ then $\stalk{X}{x}$ is a DVR and $\F_x$ is torsion-free and hence free therefore $\F$ is a vector bundle over an open $U$ of complementary codimension at least $2$ and thus $\F \to \F^{\vee \vee}$ is an isomorphism over $U$. 
\end{proof}


Now the claim is, if $L$ is big then $\M$ is big. Indeed, there is a map $L \to \M$ and hence $H^0(Y, L^{\ot m}) \embed H^0(Y, \M^{\ot m})$, 


WEIRD from McQuillan's paper: if $\Omega_X$ is big then it seems that the foliation is always big. WHAT ABOUT THE BIDISK FOLIATIONS?

PROBLEM: no reason for $L$ to be BIG, IS EXACTLY,
\[ 0 \to \Sym{k-m}{\Omega_X} \xrightarrow{\omega} \Sym{k}{\Omega_X} \to f_* L^{\ot m} \to 0 \]
for $m > k$ and taking cohomology we have to control a connecting map. Therefore, its only clearly big if some asymtotic cohomology statement holds which is possibly stronger than the asymtotic cohomology statement we need for the jet obstruction. 
\bigskip\\
In fact, if $L^{\ot m}$ has a nonconstant section at all then the base locus does not dominate $X$ so we immediately win by the resultant method. Thus we expect no sections. 


It looks like if $\F$ is $p$-closed then we ask are there finitely many or infinitely many $C$ such that $K_{\F} \cdot C < 0$. In the former case we win by McQuillan's theorem. In the latter case, we loose by Miyaoka's theorem since then there are infinitely many rational curves of bounded degree by Bend-and-Break hence uniruled. 

\section{Foliation Basics}

\begin{prop}
Let $f : X \to Y$ be a morphism of $S$-schemes and a morphism $\F \to \T_{Y/S}$ a subsheaf. Then there exists, a morphism $f^+ \F \to \T_{X/S}$ and the linear Lie bracket maps are compatible,
\begin{center}
\begin{tikzcd}
\wedge^2 f^+ \F \arrow[d] \arrow[r, "\ell"] & \T_{X/S} / f^+ \F \arrow[d]
\\
f^* \wedge^2 \F \arrow[r, "f^* \ell"] & f^* \T_{Y/S} / \F  
\end{tikzcd}
\end{center}
and if $S$ has characteristc $p$ then the $p$-curvature maps are also compatible,
\begin{center}
\begin{tikzcd}
\Frob^*_X f^+ \F \arrow[d] \arrow[r, "\psi_p"] & \T_{X/S} / f^+ \F \arrow[d]
\\
f^* \Frob^*_X\F \arrow[r, "f^* \psi_p"] & f^* \T_{X/S} / \F 
\end{tikzcd}
\end{center}
\end{prop}

\begin{proof}
We define $f^+ \F$ via the diagram,
\begin{center}
\begin{tikzcd}
f^+ \F \arrow[d] \pullback \arrow[r] & f^* \F \arrow[d]
\\
\T_{X/S} \arrow[r] & f^* \T_{Y/S}
\end{tikzcd}
\end{center}
Now the commutativity is obvious from the fact that $\T_{X/S} \to \pi^* \T_{Y/S}$ is compatible with bracket and $p$-power operations.
\end{proof}

Therefore, as long as $X$ does not map into the singular locus of the foliation $\F$, then we can define a foliation as the saturated image of $f^+ \F$ on $Y$ and then it will also be involutive and $p$-closed if $\F$ is. {\color{red} PROVE THIS!!}


\begin{cor}
If $f : C \to X$ is an entire curve invariant under a foliation $\F$ then 
\[ \im{f} \subset \Sing{\F} \cup (V^1(\psi_p) \cap V^1(\ell)) \]
where $V^1$ is the locus where a map of vector bundles is not full rank. For example, if $\F$ is a rank $1$ foliation then,
\[ \im{f} \subset \Sing{\F} \cup V(\psi_p) \]
If $X$ is a surface and $\F$ is not $p$-closed then $\Sing{\F}$ is a collection of points so any invariant curve is contained in $\Delta_p := V(\psi_p)$.
\end{cor}

\begin{theorem}[Miyaoka]
Let $L$ be a nef $\RR$-divisor on $X$. Let $f : C \to X$ be a nonconstant morphism from a smooth projective curve $C$ such that $X$ is a smooth along $f(C)$. Let $\F \subset \T_X$ be a ($p$-closed) $1$-foliation, smooth along $f(C)$. Assume that,
\[ c_1(\F) \cdot C > \frac{K_X \cdot C}{p-1} \]
then for every $x \in f(C)$ there is a rational curve $B_x \subset X$ passing through $x$ such that,
\[ L \cdot B_x \le 2n \frac{p L \cdot C}{(p-1) c_1(\F) \cdot C - K_X \cdot C} \]
\end{theorem}

\begin{proof}
Let $\rho : X \to Y = X / \F$ be the quotient. Then,
\[ K_X = (p-1) c_1(\F) + \rho^* K_Y \] 
and therefore,
\[ K_Y \cdot C = K_X \cdot C - (p-1) c_1(\F) \cdot C < 0 \]
Since the foliation is smooth along $f(C)$, the variety $Y$ is also smoth along the image of $C$. Therefore, by Bend-and-Break [Kollar rational curves, Theorem 5.8] we know that there exists such a family of rational curves $B_x'$ throught each point of the image of $C$ such  that for any nef $\RR$-divisor $M$ of $Y$ we have
\[ M \cdot B_x' \le 2n \frac{M \cdot C}{-K_Y \cdot C} \]
Let $L'$ be the pullback of $L^{(1)}$ via the canonical map $Y \to X^{(1)}$ then $\rho^* L' = (F_X)^* L = p L$. Since $\rho$ is purely inseparable of degree $p^{\rank{\F}}$ the reduced pre-image $B_x$ of $B_x'$ is also a rational curve. Note that,
\[ p (L \cdot B_x) = \rho^* L' \cdot B_x \le p L' \cdot B_x' \]
If the induced morphism $B_x \to B_x'$ is inseparable then we have equality. Otherwise, $\rho^* L' \cdot B_x = L' \cdot B_x'$ and the inequality is obvious. Therefore,
\[ L \cdot B_x \le L' \cdot B_x' \le 2n \frac{L' \cdot C}{-K_Y \cdot C} \le 2n \frac{p L \cdot C}{(p-1) c_1(\F) - K_X \cdot C} \]
proving the claim.
\end{proof}

\begin{rmk}
$c_1(\F) = - K_{\F}$ and therefore the theorem applies to curves $C$ with $K_{\F} \cdot C < 0$.
\end{rmk}


\section{Surfaces over $\ZZ$}

\begin{example}
Consider the deformation $X \to \A^1$ of a smooth quartic surface in $\P^3$ to the singular Kummer quadric. Then $\pi_1(X_0^{\text{sm}}) \neq 1$ even though $\pi_1(X_1 \sm S) = 1$ for any finite set of points $S$. This shows we cannot use deformation theory and Grothendieck's existence theorem to conlude that the fundamental group is zero. However, the resolution of singularities of $X_0$ is indeed simply-connected. This shows we need to carefully analyze the singularities of the special fiber in order to conclude.
\end{example}

\begin{prop}
Let $X / K$ be a smooth proper suface over a number field. Let $\p \subset K$ be a prime such that $N(\p) = q = p^n$ and $K / \Q$ is unramfied over $p$. Let $\ell \neq p$ be another prime. Suppose,
\begin{enumerate}
\item $\pi_1^{\et}(X_{\ol{K}}) = 0$
\item $\Frob_{\q}^n \acts H^2_{\et}(X_{\ol{K}}, \Q_{\ell})$ by $q^n \cdot \id$ for some $n > 0$ 
\item there is a flat proper model $\X \to \Spec{\stalk{K}{\p}}$ of $X$ such that $\X_\p$ is integral with isolated (GOOD DO THIS!!!) singularities.
\end{enumerate}
Then let $Y$ be a smooth proper surface over $\FF_q$ obtained as a resolution of singularities of $\X_{\p}$. Then,
\begin{enumerate}
\item $\pi_1^{\et}(Y_{\ol{\FF}_q}) = 0$
\item $Y$ is supersingular meaning $\Frob_{\q} \acts H^2(Y_{\ol{\FF}_q}, \Q_\ell)$ by $q^n \cdot \id$ for some $n > 0$. 
\end{enumerate}
\end{prop}

\begin{proof}
DO THIS PROOF!!!
\end{proof}







\section{Surfaces to Try}

\subsection{Horikawa surfaces}

Properties:
\begin{enumerate}
\item always $\pi_1 = 0$ (REF?)
\item usually general type 
\item can be orbifold hyperbolic
\end{enumerate}

Cons:

\begin{enumerate}
\item supersingularity seems hard
\item often are actually unirational
\end{enumerate}

\begin{enumerate}
\item 
\end{enumerate}

References:

\begin{enumerate}
\item \chref{https://arxiv.org/pdf/1201.5822.pdf}{How is the orbifold jets relevant here?}
\item \chref{https://arxiv.org/pdf/0805.3986.pdf}{there are lots of unirational horikawa surfaces}
\item 
\end{enumerate}

\subsection{Hilbert Modular Surfaces}


\subsection{Quotient of Products of Curves}

\subsection{Line Arrangement Varities}

\subsection{Explicit Complete Intersections}


\subsection{REU Examples}


\begin{theorem}
Let $p, q, w$ be primes such that $p, q, w \equiv 1 \mod{s}$ for some $s$ and let $X$ be the variety defined by,
\[ x_0^p + x_1^{ps} + x_2^{q} + x_3^{qs} = 0 \]
over $\FF_{w}$. If $w$ is a primitive root modulo $p$ and $q$ then $X$ is supersingular.  
\end{theorem}

\begin{example}
This works for,
\begin{enumerate}
\item $p = 7$
\item $q = 13$
\item $s = 3$
\item $w = 19$
\end{enumerate}
\end{example}

This example is supersingular for infinitely many explicit primes. Also it should be simply connected {\color{red} WHY?}
\bigskip\\
What the singularities? Note this is clearly quasi-smooth. Therefore, the singularities are only at the singularities of $P = \P(qs, ps, q, p)$. This is well-formed. 

\section{Foliations Methodology}

\subsection{What people say about McQuillan's paper}

\subsubsection{AN EXPLICIT BOUND FOR THE LOG-CANONICAL DEGREE OF CURVES ON
OPEN SURFACES}

Bogomolov proved the well known result according to which irreducible curves of fixed
geometric genus on X form a bounded family. Since Bogomolov’s argument depended on the
analysis of curves contained in a certain closed set (see [Des79] for an exposition), his remark-
able result was not effective. Indeed, Bogomolov was able to prove that curves in this closed set
form a bounded family by considerations involving algebraic foliations but without providing an
explicit bound on their degree. Because of this, in a deformation of the surface X , the number of
either rational or elliptic curves might in principle tend to infinity. This situation can be ruled
out providing an upper bound on the canonical degree of irreducible curves on X by a function
of the invariants of X and the geometric genus of the curve. The existence of such a function
and its form was then conjectured in various places and in slightly different contexts, see for
instance [Tia96, §9], with the function depending only on K2
X , c2(X ) and the geometric genus of
the curve. The conjecture was proved with some restrictive hypothesis on the singularities of the
curve involved by Langer in [Lan03] and finally in its full generality by Miyaoka in [Miy08]. It
is interesting to note that part of Miyaoka’s result can be recovered by methods closer in spirit to
the original argument of Bogomolov, see McQuillan [McQ17, Corollary 1.3], though one is able to
prove the existence of the afore mentioned function no explicit form can be established.


\subsubsection{Effective algebraic integration in bounded genus, Most interesting part of McQuillan's classification of foliations}

\newcommand{\HH}{\mathbb{H}}

\begin{theorem}
Let $\F$ be a relatively minimal foliation on a smooth projective surface $X$. If the numerical dimension of $\F$ does not coincide with the Kodaira dimension of $\F$ then,
\begin{enumerate}
\item $\nu(\F) = 1$
\item $\kappa(\F) = -\infty$
\item $X$ is the minimal desingularization of the Baily-Borel compactification of an irreducible quotient of $\HH \times \HH$ 
\item $\F$ is induced by one of the two natural fibrations on $\HH \times \HH$
\end{enumerate}
\end{theorem}

\subsubsection{Definitions}

(WHAT IS ALMOST ETALE??)

When the foliation is p-closed it follows from [2, Th´eor`eme 1] that there are infinitely
many algebraic solutions. On the other hand, if the foliation F is not p-closed then there
is a divisor F , the p-divisor, which is defined as the degeneracy locus of the p-curvature
morphism F . An interesting property of the p-divisor is that every irreducible algebraic
solution of the foliations is contained in the support of the p-divisor.


QUESTION: DO THE LEAVES OF SUCH A FOLIATION ALWAYS HAVE TO BE RATIONALLY CONNECTED? LOOK AT THE PAPER PROVING THIS USING MORI THEORY?

\section{Place to Collect Foliation Results}

\subsubsection{Paper of Langer}

Proves Popa-Schnell for surfaces in all characteristics

\chref{https://www.mimuw.edu.pl/~alan/papers/foliations.pdf}{Theorem }

Look at Theorem 5.1

Also gives a nice foliation producing rational curves result: Theorem 2.1

\subsection{Foliaitons To Read}

\begin{enumerate}
\item Myoka
\item Bogomolov-McQuillen
\end{enumerate}

\section{Product-Quotient Surfaces}

Let $G \acts C_1$ and $G \acts C_2$ be two faithful actions on smooth projective curves. Then we consider $X = (C_1 \times C_2) / G$ with the diagonal action and $\pi : S \to X$ the minimal desingualarizaton. Here we write down some results.

\begin{prop}
If $C$ is any curve and $H$ is a finite group acting faithfully on $C$ and $p \in C$ then the fixed point $H_p$ is cyclic. 
\end{prop}

\begin{proof}
 Hershel M. Farkas , Irwin Kra III.7.7 It follows from the next result.
\end{proof}

\begin{prop}
Let $h_1, \dots, h_n$ be holomorphic functions in the neighborhood of $0 \in \CC$ with $h_j(0) = 0$ and suppose these form a group $H$ under composition. Then $H$ is a rotation group, meaning there is an open disc around $0$ on which all $h_i$ are defined and a biolomorphism to $\Delta$ such that the $h_j$ are identidied with rotations. 
\end{prop}

\begin{cor}
The singularities of $X$ are cyclic quotient singulairites. If $\ol{(x,y)} \in X$ then analytically locally this point is $\CC^2 / C_n$ where $n = \# G_{(x,y)}$ defined by $\xi \cdot (z_1, z_2) = (\xi z_1, \xi^a z_2)$ for some $a$ coprime to $n$ and $\xi$ a primitive $n^{\text{th}}$-root of unity. This is a \textit{singularity of type} $\frac{1}{n}(1,a)$.
\end{cor}

\begin{rmk}
Note that $(n,a) = 1$ because otherwise some element of $G$ would act trivially on the second factor in an analytic neighborhood but by rigidity this means it acts trivially on one curve and hence the action is not faithful. 
\end{rmk}

\subsection{Singularity Types}

Note that $\frac{1}{n}(1,a) = \frac{1}{n}(1,a')$ where $a, a'$ are inverses in $(\Z / n \Z)^\times$. The exceptional fiber of the minimal resolution of a cyclic quotient singularity of type $\frac{1}{n}(1,a)$ are well-known and correspond to Hizerbruch-Jung strings:
\[ L = \sum_{i = 0}^\ell Z_i \]
a connected union of smmoth rational curves $Z_1, \dots, Z_\ell$ with self-intersection numbers at most $-2$ and ordered linearly so that $Z_i \cdot Z_{i+1} = 1$ and all other intersections are zero. Then the exceptional divisor $E$ on $S$ is the disjoint union of these HJ strings. 
\bigskip\\
The self-intersection numbers $Z_i^2 = -b_i$ are determined by the formula,
\[ \frac{n}{a} = b_i - \frac{1}{b_2 - \frac{1}{\cdots}} \]
We denote this fraction by the notation $[b_1, \dots, b_{\ell}]$ so we write,
\[ \frac{n}{a} = [b_1, \dots, b_\ell] \]
Moreover,
\[ \frac{n}{a} = [b_1, \dots, b_\ell] \iff \frac{n}{a'} = [b_\ell, \dots, b_1] \]
Cyclic quotient singularities of type $\frac{1}{n}(1, n-1)$ are particular cases of \textit{rational double points} or $A_n$ singularity: all the curves $Z_i$ have self-intersection $-2$. Singularities of type $\frac{1}{2}(1,1)$ are called \textit{ordinary double points} or $A_1$ singularities. However, type $\frac{1}{n}(1,a)$ for $a \neq n-1$ do not need to be canonical singularities.  


\begin{prop}[Serrano, Prop. 2.2]
$q(S) : = h^1(S, \struct{S}) = g(C_1/G) + g(C_2/G)$.
\end{prop}

Therefore, for our purposes we will always take $C_1 / G \cong C_2 / G \cong \P^1$. 
\bigskip\\
Furthermore, let $p_g = h^0(S, \omega_S) = h^2(S, \struct{S})$. Thus if $q(S) = 0$ we see that,
\[ \chi(\struct{S}) = 1 + p_g \]
and hence by Noether's formula,
\[ c_1^2 + c_2 = 12 (1 + p_g) \]

\begin{theorem}[Serrano, Theorem 4.1]
Let $S$ be as above and $\sigma_1 : S \to C_1 / G$ and $\sigma_2 : S \to C_2 / G$ the associated fibrations. Let $\{ n_i N_i \}_{i \in I}$ and $\{ m_j M_j \}_{j \in J}$ denote the components of all singular fibers of $\sigma_1$ and $\sigma_2$ respectively with the multiplicites attached. Finally, let $\{ Z_t \}_{t \in T}$ be the set of curves contracted to points by $\sigma_1 \times \sigma_2$ (i.e. the exceptional locus of $\pi : S \to X$) then,
\[ K_S = \sigma_1^* (K_{C_1/G}) + \sigma_2^*(K_{C_2 / G}) + \sum_{i \in I} (n_i - 1)N_i + \sum_{j \in J} (m_i - 1) M_i + \sum_{t \in T} Z_t \]
The fibrations induce foliations $\F_1, \F_2$ on $S$ such that Serrano's formula can be written as,
\[ K_S = \cN_{\F_1}^\vee \ot \cN_{\F_2}^\vee \ot \struct{S}(E) \]
\end{theorem}


\begin{center}
\begin{tikzcd}
& S  \arrow[d, "\pi"] \arrow[ldd, "\sigma_1"'] \arrow[rdd, "\sigma_2"]
\\
& (C_1 \times C_2) / G \arrow[ld, "p_1"] \arrow[rd, "p_2"'] 
\\
C_1 / G & & C_2 / G
\end{tikzcd}
\end{center}
We know that $X$ has only cyclic quotient singularities. 

\subsection{Surfaces with $p_g = 0$ and $c_1^2 = c_2$}

The conditions imply that $c_1^2 = c_2 = 6$ and $p_g = q = 0$. These have been classified and all have exactly two $A_1$ singularities.
\bigskip\\
Around one of the two singular points, $X$ is analytically isomorphic to the quotient $\CC^2 / C_2$ with action $(z_1, z_2) \mapsto (-z_1, -z_2)$. This is isomorphic to an affine subvariety of $\CC^3$ with coordinates,
\[ u = z_1^2 \quad v = z_1 z_2 \quad w = z_2^2 \]
defined by the equation $uw  =  v^2$. Moreover, if $\mu_1, \mu_2$ are the local coordinates on $S$, the resolution morphism is locally given by,
\[ \pi(\mu_1, \mu_2) = (\mu_1, \mu_1 \mu_2, \mu_1 \mu_2^2) \]
given by blowing up $(u,v,w)$. Therefore, we have the following relation between the local coordinates,
\[ z_1 = \mu_1^{1/2} \quad z_2 = \mu_1^{1/2} \mu_2 \]
The exceptional fiber is a single $(-2)$-curve. Using the local coordinates $\mu_1, \mu_2$ on $S$ we see that it is given by the set of point $\{ \mu_1 = 0 \}$. 

\subsection{Bigness of the Cotangent Bundle}

\renewcommand{\sm}{\mathrm{sm}}

Let $\Lambda$ be the set of points of $C_1 \times C_2$ with nontrivial stabilizer. We consider a procedure to produce sections of $S^{2m} \Omega_X$ from sections of $K_S^{\ot m}$. 
\bigskip\\
Let $\omega \in H^0(S, K_S^{\ot m})$. The pushforward $\pi_*$ gives a section of $K_X^{\ot m}$ away from the singularities which can be pulled back to a section of $(K_{C_1 \times C_2}^{\ot m})^G$ defined outside $\Lambda$. Since $\Lambda$ is a set of points it has codimension $\ge 2$ so this extends to a section over $C_1 \times C_2$. Moreover, we can identify sections of $(K_{C_1 \times C_2})^G$ with $(S^{2m} \Omega_{C_1 \times C_2})^G$ which descend to a section of $S^{2m} \Omega_X$ over the smooth locus which pulls back to a section $\Gamma(\omega)$ of $S^{2m} \Omega_S$ defined outside of $E$. 
\begin{center}
\begin{tikzcd}
H^0(S, K_S^{\ot m}) \arrow[ddd, "\Gamma"] \arrow[rddd, "\Theta"] \arrow[r, "\varphi_*"] & H^0(X^{\sm}, K_{X}^{\ot m}) \arrow[r, "p^*"] & H^0(C_1 \times C_2 \sm \Lambda, K_{C_1 \times C_2}^{\ot m})^G \arrow[d]
\\
& & H^0(C_1 \times C_2, K_{C_1 \times C_2}^{\ot m})^G \arrow[d]
\\
& & H^0(C_1 \times C_2, \Omega_{C_1}^{\ot m} \ot \Omega_{C_2}^{\ot m})^G \arrow[d]
\\
H^0(S \sm E, S^{2m} \Omega_S) & H^0(X^{\sm}, S^{2m} \Omega_X) \arrow[l, "\varphi^*"] & H^0(C_1 \times C_2, S^{2m} \Omega_{C_1 \times C_2})^G \arrow[l, "p_*"] 
\end{tikzcd}
\end{center} 

\begin{prop}
If $\omega \in H^0(S, \struct{S}(m(K_S - E))$ then $\Gamma(\omega)$ naturally extends to a global section $\Gamma(\omega) \in H^0(S, S^{2m} \Omega_S)$.
\end{prop}

\begin{proof}
We get $\Theta(\omega) \in H^0(X^{\sm}, S^{2m} \Omega_X)$ which, by definition, can be written locally on a punctured disk around an $A_1$ singularity,
\[ a(z_1, z_2) \d{z_1}^m \d{z_2}^m \]
Using the change of coordinates $z_1 = \mu_1^{1/2}$ and $z_2 = \mu_1^{1/2} \mu_2$ given by $\pi$ we get the pullback by $\pi^*$ of $\Theta(\omega)$ which is $\Gamma(\omega)$ can be written locally as,
\[ \sum_{j = 0}^m {m \choose j} \frac{\mu_2^{m-j} (a \circ \pi)(\mu_1, \mu_2)}{2^{2m - j} \mu_1^{m-j}} \d{\mu_1}^{2m-j} \d{\mu_2}^j \]
but $a \circ \pi$ canishes along $E$ at multiplicity at least $m$ so this is well-defined on $S$. 
\end{proof}

\begin{rmk}
Notice that all of these forms are of type $\d{z_1}^m \d{z_2}^m$ and therefore the resultant of any two is zero. Therefore, we need to use methods beyond the resultant. 
\end{rmk}


\begin{prop}
The line bundle $\struct{S}(K_S - E)$ is big.
\end{prop}

\begin{proof}
Since $S$ is a minimal surface of general type, the canonical divisor $K_S$ is nef. Therefore, by the asymtotic Riemann-Roch theorem,
\[ h^0(S, K_S^{\ot m}) = \tfrac{1}{2} m^2 c_1(S)^2 + O(m) \]
but $c_1(S)^2 = 6$ so we have,
\[ h^0(S, K_S^{\ot m}) = 3 m^2 + O(m) \]
Thus, there exists a positive real number $M$ such that,
\[ 3 m^2 - M m \le h^0(S, K_S^{\ot m}) \]
for $m$ large enough.
\bigskip\\
On the other hand, let $\omega$ be a section of $K_S^{\ot m}$. The corresponding section on $C_1 \times C_2$ can be written locally, around a fixed point, as
\[ a(z_1, z_2) (\d{z_1} \wedge \d{z_2})^m \]
where $a$ is a holomorphic function,
\[ a(z_1, z_2) = \sum_{i,j} a_{ij} z_1^i z_2^j \]
Using change of coordinates $z_1 = \mu_1^{1/2}$ and $z_2 = \mu^{1/2} \mu_2$ given by $\pi$ at an $A_1$ singularity. Then $\omega$ vanishes along $E$, at least with multiplicity $m$ if $a_{i,j} = 0$ for all $i,j$ such that $i + j < 2m$. This gives $1 + 2 + \cdots + 2m$ sufficent conditions. However, the section is invariant by the $G$-action, then $a_{i,j} = 0$ for all $i + j$ odd since any function defined on the quotient must be invariant under the stabilizer $C_2$ acting via $(z_1, z_2) \mapsto (-z_1, -z_2)$. Therefore, 
\begin{align*}
h^0(S, \struct{S}(m(K_S - E))) & \ge h^0(S, K_S^{\ot m}) - \# \{ A_1 \text{ singularities} \} \frac{1 + 2 + \cdots + 2 m}{2} 
\\ 
& \ge (3 m ^2 - Mm) - (2m^2 + m) = m^2 - (M+1)m 
\end{align*}
\end{proof}

It looks like the innequality we need is $c_1^2 > 2 a_1$ where $a_1$ is the number of $a_1$ singularities and we only have $a_1$ singularities. {\color{red} COMPARE TO ROUSEAUX}


\section{Explicit Formulas for Bigness}

\subsection{Orbifold}

Let $\X \to (X, \Delta)$ be a 2-dimensional orbifold for which $\Delta = 0$ and the singularities are ADE. Let $Y \to X$ be the minimal desingularization.

\begin{prop}
The Chern numbers are $c_1^2(\X) = c_1^2(X) = c_1^2(Y)$ and,
\[ c_2(\X) = c_2(Y) - \sum (n+1) (a_n + d_n + e_n) + \sum \left( \frac{a_n}{n+1} + \frac{d_n}{4(n-2)} + \frac{e_6}{24} + \frac{e_7}{48} + \frac{e_8}{180} \right) \]
Where $a_n$ is the number of $A_n$ singularities etc.
\end{prop}

\begin{rmk}
Note these numbers are the orders of the group which acts to form the standard group quotient representation of these singularities.
\end{rmk}

\begin{theorem}[\chref{https://arxiv.org/pdf/1303.3377.pdf}{RR13}]
Suppose that $s_2(Y) + s_2(\X) > 0$ then $\Omega_Y$ is big.
\end{theorem}

\begin{cor}
If there are only $a_n$ singularities this number is,
\[ 2 c_1^2(Y) - 2 c_2(Y) + \sum \frac{n^2 + 2n}{n + 1} a_n \]
\end{cor}

\subsection{Local Euler Characteristic}

Again let $\pi : Y \to X$ be the resolution of a surface. Then we have,
\[ \chi(Y, S^m \Omega^1_Y) = \frac{1}{12} \left( 2 (K^2 - \chi) m^3 - 6 \chi m^2 - (K^2 + 3 \chi) m + K^2 + \chi \right) \]
where $K^2 = c_1(Y)$ and $\chi = c_2(Y)$. 

\begin{theorem}[\chref{https://arxiv.org/pdf/1912.08908.pdf}{Bruin1}]
Let $X$ be an irreducible complex projective surface whose singular locus $S$ is a finite set of isolated du Val singularities. Let $\tau : Y \to X$ be a minimal resolution. Then for $m \ge 3$ we have,
\[ h^0(Y, S^m \Omega_Y^1) \ge \chi(Y, S^m \Omega^1_Y) + \sum_{s \in S} \chi^1(s, S^m \Omega^1_Y) \]
and for all $m \ge 1$
\[ h^0(Y, S^m \Omega^1_Y) \ge h^0(X, \hat{S}^m \Omega^1_X) - \sum_{s \in S} \chi^0(s, S^m \Omega_Y^1) \]
\end{theorem}

\begin{theorem}
If $s \in S$ is an $A_1$ singularity then $\chi^0 \sim \frac{11}{108} m^3$ and $\chi^1 \sim \frac{4}{27} m^3$. See the paper for exact formulas. 
\end{theorem}


\subsection{Complete Intersections of Quadrics}

Let $X \subset \P^n$ be the complete intersection of $n - 2$ with $\ell$ isolated $A_1$ singularities. Let $Y \to X$ be the minimal resolution. In this case we have,
\[ K^2 = c_1(Y)^2 = (n - 5)^2 2^{n-2} \quad \chi = c_2(Y) = (n^2 - 7n + 16)n^{n-3} \]
which is general type for $n \ge 6$.  Thene we have,
\[ \chi(Y, S^m \Omega_Y^1) = \tfrac{1}{3} 2^{n-5} (2 (n^2 - 13 n + 34) m^3 - 6(n^2 - 7n + 16) m^2 - (5 n^2 - 41 n + 98) m + 3 n^2 - 27 + 66) \]


\section{Surfaces with Many Nodes}

\begin{enumerate}
\item Barth's sextic is NOT known to have big $\Omega$ I think NOT know to be quasi-hyperbolic {\color{red} I THINK BOGOMOLOV'S PAPER HAS THE MISTAKE THAT MAKES PEOPLE THINK IT HAS BIG OMEGA}

\item Barth decic and Sarti's surface both are known to have big $\Omega$.
\end{enumerate}


{\color{red} magic squares surface looks to be not supersingular in most characteristics via quick computer search}



\subsection{Surface of degre 5 with four $A_9$ singularities}


\section{Questions}

Does McQuillan's Theorem actually imply that if $\Omega$ is big then it cannot be uniruled in any characteristic of good reduction? In particular in finitely many characteristics? 


\section{Ideas}

Maybe the following works: for $H^1(\X, S^n \Omega_{\X})$ I can get the usual Euler characteristic estimate for the stack but also use the formulas for $H^0(X, \hat{S}^n \Omega_X)$ which should be the same as the differentials on the stack (right?) and therefore the difference gives me a bound on the $H^1$ term. 

\end{document}
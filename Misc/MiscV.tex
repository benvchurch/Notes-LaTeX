\documentclass[12pt]{article}
\usepackage{hyperref}
\hypersetup{
    colorlinks=true,
    linkcolor=blue,
    filecolor=magenta,      
    urlcolor=blue,
}

\usepackage{import}
\import{"../Algebraic Geometry/"}{AlgGeoCommands}

\newcommand{\Loc}[1]{\mathfrak{Loc}\left( #1 \right)}
\newcommand{\AbGrp}{\mathbf{AbGrp}}

\begin{document}

\section{TODO!!}

\begin{enumerate}
\item Finish symplectic geometry course
\begin{enumerate}
\item figure out if symplectic toric is the same as projective toric variety (projectivity needed to come from a polytope and also to be Kahler)
\item review coisotropic reduced and write some notes
\item hyperkahler reduction examples 
\item are there examples of noncompact hyperkahlers?
\item work out the kinks in notes on hamiltonian actions
\end{enumerate}
\item review killing homotopy groups columbia lectures and write some notes
\item figure out those damn jet bundles and connections on principal bundles
\begin{enumerate}
\item RMK: $\pi^* E$ is NOT trivial for a vector bundle let alone a fiber bundle. it does get equiped with a canonical section but for a vector bundle this is just the trivial section, only for a principal bundle does giving a section trivialize it.
\item role of atiyah sequence vs jet bundle sequence
\item 
\end{enumerate}
\item spectral sequences for tor and ext in derived category 
\begin{enumerate}
\item application to universal coefficient theorem
\item Kunneth spectral sequence
\item Kunneth formula for smash product?
\item why are derived functors triangulated
\item derived functors in terms of Kan extensions (NOTES)
\end{enumerate}
\item write notes on universal morphisms
\item $G$-action of $X/Y$ induces map Descent data $X/Y$ to $G$-equivariant sheaves
\begin{enumerate}
\item isomorphism when $X/Y$ is a $G$-cover i.e. $X \to Y$ is a $G$-torsor
\item write down explicit $G$-equivariant structure on $\Omega_X$
\item Galois descent derive explicit form
\end{enumerate}
\item Weil restriction
\begin{enumerate}
\item write down trivialization after going back up
\item Galois descent in explicit form
\end{enumerate}
\item notes on Galois actions on schemes
\item notes on Frobenii
\item notes on universal constructions in math with examples
\end{enumerate}

\section{Some Connection Musings}

\begin{defn}
Let $f : E \to X$ be a smooth surjection (in the smooth category, what should it be in the algebraic category?) then an \textit{Erhesmann connection} is a  splitting of the sequence of vector bundles,
\begin{center}
\begin{tikzcd}
0 \arrow[r] & \ker{\d{f}} \arrow[r] & TE \arrow[r] & \pi^* TX \arrow[r] & 0
\end{tikzcd}
\end{center}
where we usually call $V = \ker{d{f}}$ the vertical bundle. In algebraic language, $V$ is the dual of the relative differentials so the connection corresponds to a splitting of,
\begin{center}
\begin{tikzcd}
0 \arrow[r] & f^* \Omega_X \arrow[r] & \Omega_E \arrow[r] & \Omega_{E/X} \arrow[r] & 0
\end{tikzcd}
\end{center}
\end{defn}

\begin{rmk}
Such splittings are supposed to correspond to smooth sections of the map $J^1(E) \to E$. We now explain how this works. Unfortunately, I don't know a good unified language to describe the jet bundles so I will give the algebraic and smooth definitions.
\end{rmk}

\begin{defn}
Given a smooth surjection $f : E \to X$, consider the $n$-th thickened diagonal, $X \to X_n \to X \times_S X$. Then we consider the functor sending $T \to S$ to pairs of maps $T \to X$ and $T \times_X X_n \to E$ such that the diagram,
\begin{center}
\begin{tikzcd}
E \arrow[r, "f"] & X
\\
T \times_X X_n \arrow[u] \arrow[dr, phantom, "\usebox\pullback" , very near start, color=black] \arrow[d] \arrow[r] & X_n \arrow[d, "\pi_2"] \arrow[u, "\pi_1"']
\\
T \arrow[r] & X
\end{tikzcd}
\end{center}
commutes. Then the jet scheme $J_n(E/X)$ with maps $J_n(E/X) \to X$ and $J^n(E/X) \times_X X_n \to E$ represents this functor.
\end{defn}

\section{Counterexamples In Geometry}

\begin{example}
The Hopf surface is the compact complex surface $H = \C^2 \setminus \{ 0 \} / \Z$ where $\Z \acts \C^2$ via $(z_1, z_2) \mapsto  (\lambda z_1, \lambda z_2)$ for $0 < \lambda < 1$. This surface has $h^{1,0} = 1$ but $h^{0,1} = 0$. Furthermore, $H$ is diffeomorphic to $S^3 \times S^1$. This provides:
\begin{enumerate}
\item a compact complex manifold that is not K\"{a}hler
\item a compact complex manifold without Hodge symmetry 
\item a compact complex manifold that is not symplectic ($H^2(H, \Z) = 0$)
\end{enumerate}
\end{example}

\begin{rmk}
From the exponential exact sequences,
\begin{center}
\begin{tikzcd}
0 \arrow[r] & \Z \arrow[r] & \struct{X} \arrow[r, "\exp"] & \struct{X}^\times \arrow[r] & 0
\end{tikzcd}
\end{center}
we have that,
\begin{center}
\begin{tikzcd}

\end{tikzcd}
\end{center}
\end{rmk}

\end{document}

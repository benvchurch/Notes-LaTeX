\documentclass[12pt]{article}
\usepackage{hyperref}
\hypersetup{
    colorlinks=true,
    linkcolor=blue,
    filecolor=magenta,      
    urlcolor=blue,
}

\usepackage{import}
\import{"../Algebraic Geometry/"}{AlgGeoCommands}


\begin{document}


\section{Two Capacitors}

Suppose we hook up two capacitors with capacitances $C_1, C_2$ one charged with charge $Q$ the other uncharged. The total internal resistance is $R$.
\bigskip\\
First, we can actually calculate the equilibrium is when the charge is distributed to equalize the voltage, i.e. $q_1 - q_2 = Q$ because 
Let $q_1(0)  = Q$ and $q_2(0) = 0$ and $I = \dot{q}_1 = \dot{q_2}$ so $q = q_1 - q_2$ is constant so $q_1 = Q + q_2$. Now,
\[ R \dot{q_1} + \frac{q_1}{C_1} + \frac{q_2}{C_2} = 0 \]
Therefore,
\[ \dot{q_1} = \left( \frac{Q}{R C_2} - q_1 \left( \frac{1}{R C_1} + \frac{1}{R C_2} \right) \right) = \frac{1}{R C_S} \left( Q \frac{C_S}{C_2} - q_1 \right) = \frac{1}{R C_S} \left( \frac{C_1}{C_1 + C_2} Q - q_1 \right) \]
Therefore,
\[ \dot{q_1} = \frac{1}{R C_2} \left(Q  - q_1 \frac{C_2}{C_S} \right) \]
where,
\[ \frac{1}{C_S} = \frac{1}{C_1} + \frac{1}{C_2} \]
viewing the capacitors in series. In particular, we have equilibrium when, 
\[ q_1 = Q \frac{C_S}{C_2} = \frac{C_1}{C_1 + C_2} Q \]
and thus at equilibrium,
\[ q_1 = \frac{C_1}{C_1 + C_2} Q \quad \text{and} \quad q_2 = - \frac{C_2}{C_1 + C_2} Q \]
At the begining there is energy,
\[ E_0 = \frac{Q^2}{2 C_1} \]
consider the energy afterwards,
\[ E_1 = \frac{q_1^2}{2 C_1} + \frac{q_2^2}{2 C_2} = \frac{Q^2}{2} \left( \frac{C_1}{(C_1 + C_2)^2} + \frac{C_2}{(C_1 + C_2)^2} \right) = \frac{Q^2}{2 (C_1 + C_2)} = \frac{Q^2}{2 C_P} \]
where $C_P = C_1 + C_2$ viewing the capacitors in parallel. Therefore the energy loss is,
\[ \Delta E = E_0 - E_1 = \frac{Q^2}{2 C_1} \left( 1 - \frac{C_1}{C_1 + C_2} \right) = \frac{Q^2}{2C_1} \cdot \frac{C_2}{C_1 + C_2} = E_0 \left( \frac{C_2}{C_1 + C_2} \right) \] 
Now likewise, we can solve the differential equation,
\[ q_1(t) = Q \left( \frac{C_1}{C_1 + C_2} +  \frac{C_2}{C_1 + C_2} e^{- \frac{t}{RC_S}} \right) \]
Furthermore, we can consider,
\begin{align*}
\Delta E & = \int_0^{\infty} P \, \d{t} = \int_0^{\infty} I^2 R \, \d{t} = \int_0^{\infty} \dot{q}_1^2 R \, \d{t} = \frac{1}{R C_S^2} \int_0^{\infty} \left( \frac{C_1}{C_1 + C_2} Q - q_1 \right)^2 \, \d{t}
\\
& = \frac{Q^2}{R C_S^2} \left( \frac{C_2}{C_1 + C_2} \right)^2 \int_0^\infty e^{-\frac{2 t}{R C_S}} \, \d{t} = \frac{Q^2}{R C_S^2} \left( \frac{C_2}{C_1 + C_2} \right)^2 \frac{R C_S}{2} = \frac{Q^2}{2 C_S} \left( \frac{C_2}{C_1 + C_2} \right)^2
\\
& = \frac{Q^2}{2 C_1} \frac{C_2}{C_1 + C_2} 
\end{align*}
agreeing with our previous result where I used,
\[ \frac{1}{C_S} \left( \frac{C_2}{C_1 + C_2} \right)^2 = \frac{C_1 + C_2}{C_1 C_2} \cdot \frac{C_2^2}{(C_1 + C_2)^2} = \frac{1}{C_1} \cdot \frac{C_2}{C_1 + C_2} \]

\section{Moving a Chain}

I claim that a chain (or string) in some smooth curve through space (without gravity) will follow its curve in the absense of gravity. 
\bigskip\\
First we illustrate this with a chain in a circular arc. Suppose it has tension $T$. Then on a chain element, $\d{F} = T \d{\theta}$ inwards. Furthermore, the acceleration needed to maintain circular motion is,
\[ a = \frac{v^2}{R} \]
where $R$ is the radius of curvature. Finally, the chain element has mass $\d{m} = \lambda R \d{\theta}$ where $\lambda$ is the linear mass density and thus,
\[ \d{F} = T \d{\theta} = a \, \d{m} = \mu v^2 \d{\theta} \iff T = \lambda v^2 \]
independent of the radius of curvature. Since infinitessimally every curve is a circular arc with some radius of curvature (that is we can fit it to second order and thus the second derivatives agree) we see that a constant tension gives exactly the required forces to mantain the chain's shape. 
\bigskip\\
What happens if we add gravity? 
\bigskip\\
What happens if we add constant friction? Let $f_s$ be the linear friction force density opposing the tension. The tension must increase throughout the length as,
\[ \d{T} = f_s \d{\ell} \]
which means we cannot have the constant tension necessary for maintaining curves. Thus,
\[ \d{F_{\perp}^{\text{eff}}} = a \, \d{m} - T \d{\theta} =  (\lambda v^2 - T) \d{\theta} \]
Therefore,
\[ a_\perp^{\text{eff}} = \frac{\lambda v^2 - T}{\lambda R} \]

\section{Schwarzian Derivative}

Let $f : \Omega \to \C$ be a holomorphic function and $z \in \Omega$ then consider the limit,
\[ \lim_{w \to z} \left[ \frac{f'(z) f'(w)}{(f(z) - f(w))^2} - \frac{1}{(z - w)^2} \right] \]
For sufficiently small $|z - w|$ we can write,
\[ f(w) = \sum_{n = 0}^\infty \frac{f^{(n)}(z)}{n!} (w - z)^n \]
and then,
\[ \frac{1}{(f(z) - f(w))^2} = \frac{1}{f'(z)^2 (z - w)^2} \cdot \left( \frac{1}{1 + \sum\limits_{n = 2}^\infty \frac{f^{(n)}(z)}{f'(z) n!} (w - z)^{n-1}} \right)^2 \]
Furthermore,
\[ f'(w) = \sum_{n = 0}^\infty \frac{f^{(n+1)}(z)}{n!} (w - z)^{n} \]
First we conmute the second-order leading terms of,
\begin{align*}
\left( \frac{1}{1 + \sum\limits_{n = 2}^\infty \frac{f^{(n)}(z)}{f'(z) n!} (w - z)^{n-1}} \right)^2 & = \left[ 1 - \left(\frac{1}{2} \frac{f''(z)}{f'(z)} (w - z) + \frac{1}{6} \frac{f'''(z)}{f'(z)} (w - z)^2 \right) + \left(\frac{1}{2} \frac{f''(z)}{f'(z)} (w - z) \right)^2 \right]^2
\\
& = \left[ 1 - \frac{1}{2} \frac{f''(z)}{f'(z)} (w - z) + \left( \frac{1}{2} \frac{f''(z)^2}{f'(z)^2} -  \frac{1}{6} \frac{f'''(z)}{f'(z)} \right) (w - z)^2  \right]^2
\\
& = 1 - \frac{f''(z)}{f'(z)} (w - z) + \left( \frac{5}{4} \frac{f''(z)^2}{f'(z)^2} - \frac{1}{3} \frac{f'''(z)}{f'(z)} \right) (w - z)^2  
\end{align*}
and therefore the leading terms are
\begin{align*}
\frac{f'(z) f'(w)}{(f(z) - f(w))^2} & = \frac{f'(z)  + f''(z) (w - z) + \tfrac{1}{2} f'''(z) (w - z)^2 }{f'(z) (z - w)^2} 
\\
& \cdot \left[ 1 - \frac{f''(z)}{f'(z)} (w - z) + \left( \frac{5}{4} \frac{f''(z)^2}{f'(z)^2} - \frac{1}{3} \frac{f'''(z)}{f'(z)} \right) (w - z)^2  \right]
\\
& = \frac{1}{(z - w)^2} \left[ 1 + \left( \frac{1}{6} \frac{f'''(z)}{f'(z)} - \frac{1}{4} \left( \frac{f''(z)}{f'(z)} \right)^2 \right) (w - z)^2 \right] 
\\
& = \frac{1}{(z - w)^2} + \left[ \frac{1}{6} \frac{f'''(z)}{f'(z)} - \frac{1}{4} \left( \frac{f''(z)}{f'(z)} \right)^2 \right]
\end{align*}
Therefore,
\[ \lim_{w \to z} \left[ \frac{f'(z) f'(w)}{(f(z) - f(w))^2} - \frac{1}{(z - w)^2} \right] =  \frac{1}{6} \frac{f'''(z)}{f'(z)} - \frac{1}{4} \left( \frac{f''(z)}{f'(z)} \right)^2 \]
Loot at \chref{https://en.wikipedia.org/wiki/Schwarzian_derivative}{this} page.
\end{document}

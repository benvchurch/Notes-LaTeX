\documentclass[12pt]{article}
\usepackage{hyperref}
\hypersetup{
    colorlinks=true,
    linkcolor=blue,
    filecolor=magenta,      
    urlcolor=blue,
}

\usepackage{import}
\import{"../Algebraic Geometry/"}{AlgGeoCommands}

\newcommand{\Loc}[1]{\mathfrak{Loc}\left( #1 \right)}
\newcommand{\AbGrp}{\mathbf{AbGrp}}
\usepackage{bbm}
\usepackage{cancel}
\usetikzlibrary{decorations.pathreplacing,calligraphy}


\begin{document}

\sloppy

\section{Curve Theory}

\subsection{Geometric Irreducibility of Generic Fibers}

\begin{lemma}[\chref{https://stacks.math.columbia.edu/tag/0553}{Tag 0553}]
Let $f : X \to Y$ be a morphism of schemes. Assume,
\begin{enumerate}
\item $Y$ is irreducible with generic point $\eta$,
\item $X_\eta$ is geometrically irreducible
\item $f$ is of finite type
\end{enumerate}
then there exists a nonempty open subscheme $V \subset Y$ such that $X_V \to V$ has geometrically irreducible fibers.
\end{lemma}

\begin{lemma} \label{lemma:normal_geom_integral}
Let $f : X \to Y$ be a morphism of schemes. Suppose that,
\begin{enumerate}
\item $X$ and $Y$ are integral
\item $X$ is normal
\item the fibers of $f$ are geometrically connected (e.g. $f_* \struct{X} = \struct{Y}$)
\end{enumerate}
then the generic fiber $X_\eta \to \Spec{\kappa(\eta)}$ is geometrically irreducible.
\end{lemma}

\begin{proof}
$X_\eta / \kappa(\eta)$ is geometrically irreducible iff $\kappa(\eta)$ is separable closed in $\kappa(\xi)$. This follows from \chref{https://stacks.math.columbia.edu/tag/054Q}{Tag 054Q} and \chref{https://stacks.math.columbia.edu/tag/0G33}{Tag 0G33}. Let $\alpha \in \kappa(\xi)$ be separably algebraic over $\kappa(\eta)$ i.e. a root of a separable polynomial $p \in \kappa(\eta)[x]$. There is a coordinate ring $A$ of $Y$ where all the denominators of $p$ are invertible. We claim that $A[\alpha] \subset B$ where $B$ is any coordinate ring of $X$ containing $A$. Indeed, $\alpha$ is integral over $A$ and hence over $B$ so by normality $\alpha \in B$ so we get morphisms,
\[ X_A \to \Spec{A[\alpha]} \to \Spec{A} \]
but the fibers of $X_A \to \Spec{A}$ are geometrically connected so we must have $\alpha \in A$ since otherwise the fibers of $\Spec{A[\alpha]} \to \Spec{A}$ and hence $X_A \to \Spec{A}$ are not geometrically irreducible.
\end{proof}

\begin{example}
We cannot ensure geometric reducedness of the fiber via Stein factorization however. Indeed, consider,
\[ X = \Proj{\FF_p[s,t][X,Y,Z]/(X^p + s Y^p + t Z^p)} \to \Spec{\FF_p[s,t]}= Y \]
satisfies $f_* \struct{X} = \struct{Y}$ and $X$ is normal but the generic fiber,
\[ X = \Proj{\FF_p(s,t)[X,Y,Z]/(X^p + s Y^p + t Z^p)} \to \Spec{\FF_p(s,t)} \]
is not geometrically reduced. Indeed, alhough $\FF_p(s,t)$ is algebraically closed in,
\[ \Frac{\FF_p(s,t)[x,y]/(x^p + s y^p + t)} \]
it is not separable since separability implies reducedness fo the base change by the field extension $\FF_p(s^{\frac{1}{p}}, t^{\frac{1}{p}})$.
\end{example}


\subsection{Genera of Curves}

\begin{defn}
A \textit{curve} $C$ over $k$ is a separated finite type scheme over $k$ of pure dimension $1$.
\end{defn}

\begin{defn}
Let $X$ be a proper curve over $k$. The \textit{arithmetic genus} of $X$ is,
\[ p_a(X/k) := \dim_k H^1(X, \struct{X}) \]
If $H^0(X, \struct{X}) = K$ is a field then we write,
\[ p_a(X) := \dim_K H^1(X, \struct{X}) \]
\end{defn}

\begin{rmk}
The arithmetic genus is stable under field extension by flat base change. However, if $X$ admits $X \to \Spec{k'} \to \Spec{k}$ then the arithmetic genus of $X$ viewed over $k$ is $[k' : k]$ times the arithmetic genus of $X$ viewed over $k'$. The point of the second definition is that when it it applies the base field is unambigious.
\end{rmk}

\begin{defn}
Let $X$ be a curve which is a disjoint union of finitely many smooth proper curves over an algebraically closed field $k$. Then the \textit{geometric genus} (or just \textit{genus}) of $X$ is,
\[ g(X) := p_a(X/k) = \sum_{i = 1}^n p_a(C_i / k) \]
\end{defn}


\begin{defn}
Let $X$ be a proper curve over a field $k$. Consider $\wt{X}$ which is the normalization of $(X_{\bar{k}})_{\red}$. This is a disjoint union of finitely many smooth proepr curves $C_i$ over $\bar{k}$. Thus we can define,
\[ g(X/k) := g(\wt{X}) \]
If $H^0(X, \struct{X}) = K$ is a field then we set,
\[ g(X) := g(X/k) \]
\end{defn}


\begin{rmk}
The geometric genus is stable under field extension by definition. However, notice that $g(X/k)$ does depend on the base field. If $X$ admits $X \to \Spec{k'} \to \Spec{k}$ then the geometric genus of $X$ viewed over $k$ is $[k' : k]$ times the geometric genus of $X$ viewed over $k'$. The point of the second definition is that when it it applies the base field is unambigious.
\end{rmk}

\begin{prop}
The geometric genus is a birational invarian of proper curves over $k$.
\end{prop}

\begin{proof}
This is almost by definition. Let $f : X \rat Y$ be a birational map of curves meaning there is a dense open on which it becomes an isomorphism. Then by functoriality this gives a birational map $f : \wt{X} \rat \wt{Y}$ which is an isomorphism since both sides are collections of regular curves over $\bar{k}$. Hence $g(X) = g(Y)$.
\end{proof}

\begin{lemma}
Let $f : X \to Y$ be a nonconstant map of proper regular curves over an algebrcially closed field $k$. Then $g(X) \ge g(Y)$.
\end{lemma}

\begin{proof}
Riemann-Hurwitz and Frobenius tricks [Hartshorne, Chapter IV]
\end{proof}

\begin{prop}
Let $f : X \to Y$ be a dominant map of proper curves over a field $k$. Then $g(X/k) \ge g(Y/k)$.
\end{prop}

\begin{proof}
By definition, we set $\wt{X}$ to be the normalization of $(X_{\bar{k}})_{\red}$ and then $g(X/k) = g(\wt{X})$. Then the induced map $f : \wt{X} \to \wt{Y}$ is also surjective since it is dominant (because this is preserved by base change and reduction and normalization) and proper. Therefore, each component of $\wt{Y}$ is hit by some component of $\wt{X}$ so we reduce to the previous lemma and conclude,
\[ g(X/k) \ge g(Y/k) \]
\end{proof}

\begin{example}
Say $E = \Proj{\RR[X,Y,Z]/(Y^2 Z - X^3 - x Z^2)}$ is an elliptic curve over $\RR$. It is important that we consider the genus of $E_{\CC}$ \textit{as a curve over $\RR$} as $2$ and not $1$ because,
\[ X = \Proj{\RR[X,Y,Z]/((Y^2 Z - X^3)^2 + (XZ^2)^2)} \]
has normalization $E_{\CC}$. However, $X$ has genus $2$ since $H^0(X, \struct{X}) = \RR$ so we must view it over $\RR$ and to compute its genus we base change to $X_{\CC}$ then our definition will give genus $2$. If we want the map $E_{\CC} \to X$ to satisfy the above lemma we must have $g(E_{\CC} / \RR) = 2$. 
\end{example}

\begin{cor}
Let $f : X \to Y$ be a dominant map of proper curves over $k$ with,
\[ k \to H^0(Y, \struct{Y}) \to H(X, \struct{X}) \] 
isomorphisms. Then $g(X) \ge g(Y)$.
\end{cor}

\begin{rmk}
The above example shows that the assumption on the fields is necessary. 
\end{rmk}

\subsection{Birational Maps of Curves and the Relationship Between Genera}

\begin{lemma} \label{genus_formulas}
Let $C, S$ be proper integral curves over $k$ which are birational over $k$. Suppose that $S$ is regular. Let $k_C = H^0(C, \struct{C})$ and $k_S = H^0(S, \struct{S})$. Then the genera satisfy,
\begin{enumerate}
\item $g(C) = g(S)$
\item $p_a(C) \ge p_a(S)$
\end{enumerate}
and if one of the following holds,
\begin{enumerate}
\item $p_a(C) = p_a(S)$ with $p_a(S) > 0$
\item $p_a(C) = p_a(S) = 0$ and $k_C = k_{S}$
\end{enumerate}
then $C \cong S$ so $C$ is regular.
\end{lemma}

\begin{proof}
We have already seen that $g$ is a birational invariant for all curves. Now focus on $p_a$. Given a birational map $S \birat C$ we can extend it to a birational morphism $S \to C$ since $S$ is regular. The morphism $f : S \to C$ is automatically finite since it is a non-constant map of proper curves. In particular, $f$ is affine so for each $y \in C$ we may choose an affine open $y \in V \subset C$ whose preimage $U = f^{-1}(V)$ is also affine. On sheaves, this gives a map of domains $\struct{C}(V) \to \struct{S}(U)$ which localizes to an isomorphism on the fraction fields. However, the localization map of a domain is injective so $\struct{C}(V) \embed \struct{S}(U)$ is an injection. This shows that $\struct{C} \to f_* \struct{S}$ is an injection of sheaves which is generically an isomorphism. Extending to an exact sequence,
\begin{center}
\begin{tikzcd}
0 \arrow[r] & \struct{C} \arrow[r] & f_* \struct{S} \arrow[r] & \Csh \arrow[r] & 0
\end{tikzcd}
\end{center} 
where $\dim{\Supp{}{\Csh}} = 0$ and hence $H^1(C, \Csh) = 0$. Then the long exact sequence of cohomology gives,
\begin{center}
\begin{tikzcd}[column sep = small]
0 \arrow[r] & H^0(C, \struct{C}) \arrow[r] & H^0(S, \struct{S}) \arrow[r] & H^0(X, \Csh) \arrow[r] & H^1(C, \struct{C}) \arrow[r] & H^1(S, \struct{S}) \arrow[r] & 0
\end{tikzcd}
\end{center}
Therefore, 
\[ p_a(C/k) \ge p_a(S/k) \]
and the fields $k_C = H^0(C, \struct{C})$ and $k_{S} = H^0(S, \struct{S})$ satisfy $k_C \embed k_{S}$. Therefore, dividing by the respective degrees of the extensions gives,
\[ p_a(C/k_C) \ge p_a(S/k_S) \]
since $[k_C : k] \le [k_S : k]$. Now if $p_a(C) = p_a(S)$ and are nonzero then the two constituent inequalities are equalities meaning $p_a(C/k) = p_a(S/k)$ and $k_C = k_S$. Indeed whenever this holds, the extact sequence shows that $H^0(C, \Csh) = 0$ so $\Csh = 0$ since it is supported on an affine scheme. Therefore $f : S \to C$ is an isomorphism since it is affine and $\struct{C} \iso f_* \struct{S}$ is an isomorphism. 
\end{proof}

\begin{example}
The normalization map,
\[ \P^1_{\CC} \to \Proj{\RR[X,Y,Z]/(X^2 + Y^2)} \]
gives an example where $p_a(C) = p_a(S) = 0$ but the map is not an isomorphism since $k_C \embed k_{S}$ is not an isomorphism. 
\end{example}

\begin{prop}
Let $C$ be a proper integral curve over $k$. Then $g(C) \le p_a(C)$. If $C$ is smooth, this is an equality. If equality holds and $C$ is geometrically reduced then $C$ is smooth.
\end{prop}

\begin{proof}
Change the field such that $k = H^0(C, \struct{C})$. Then $p_a(C) = p_a(C_{\bar{k}})$ by flat base change. Then $\wt{C} \to (C_{\bar{k}})_{\red}$ satisfies the above hypotheses so, 
\[ g(C) = p_a(\wt{C}/\bar{k}) \le p_a((C_{\bar{k}})_{\red} / \bar{k}) \le p_a(C_{\bar{k}}/\bar{k}) = p_a(C) \]
Now if $C$ is smooth then so is $C_{\bar{k}}$ so the above are equalities. Now if $g(C) = p_a(C)$ the above are equalities. This implies that $\wt{C} \iso (C_{\bar{k}})_{\red}$ is an  isomorphism so if $C$ is geometrically reduced then $C_{\bar{k}} \cong \wt{C}$ and hence $C$ is smooth. 
\end{proof}

\subsection{Degenerations of Curves}

Notation: let $(R, \m, \kappa)$ be a DVR with fraction field $K = \Frac{R}$. Let $S = \Spec{R}$. For $X \to S$ let $X_\eta = X_K$ be the generic fiber and let $X_s = X_\kappa$ the special fiber.

\begin{defn}
A \textit{degeneration of curves} is a proper flat family $X \to S = \Spec{R}$ over a DVR $R$ where $X_\eta$ is an integral normal projective curve over $K = \Frac{R}$. If $X$ is normal we say that $X$ is a \textit{model} of $X_\eta$ over $R$.
\end{defn}

\begin{lemma}
The total space $X$ of a degeneration of curves is integral.
\end{lemma}

\begin{proof}
We need to show that every affine open $\Spec{A} = U \subset X$ has $A$ a domain. Indeed, $R \to A$ is flat so $A \embed A_K$ is injective but $A_K$ is an affine open of $X_K$ which in integral so $A_K$ and hence $A$ is a domain.
\end{proof}

\begin{lemma} \label{lemma:normal_o_conn}
Let $f : X \to Y$ be a proper flat map of integral schemes with $Y$ normal. Then the following are equivalent,
\begin{enumerate}
\item $f_* \struct{Y} = \struct{Y}$
\item $H^0(X_\eta, \struct{X_\eta}) = \kappa(\eta)$
\end{enumerate}
\end{lemma}

\begin{proof}
Indeed, $f_* \struct{X}$ is a finite $\struct{Y}$-algebra and since $X$ is integral it is a sheaf of domains. We need to show that $\struct{Y} \to f_* \struct{X}$ is an isomorphism which is a local question so we reduce to $\Spec{A} \subset Y$ and $\Spec{B} \subset X$ such that $A \to B$. Then we have maps $A \to (f_* \struct{X})(A) \to B$ and $A \to B$ is flat hence injective since they are domains. Hence $\struct{Y} \to f_* \struct{X}$ is injective. Furthermore, by flat base change,
\[ H^0(X_\eta, \struct{X_\eta}) = (f_* \struct{X})_{\eta} \]
so if (b) holds then $(f_* \struct{X})_{\eta} = \kappa(\eta)$. Since $\struct{Y}$ is normal and $f_* \struct{X}$ is integral over $\struct{Y}$ we see that $\struct{Y} \to f_* \struct{X}$ is an isomorphism since it is contained in the fraction field.
\end{proof}

\begin{prop}
Let $X \to S$ be a degeneration of curves. Consider the following properties,
\begin{enumerate}
\item $X_\eta \to \Spec{\kappa(\eta)}$ is geometrically integral

\item $X_\eta \to \Spec{\kappa(\eta)}$ is geometrically irreducible

\item $X_\eta \to \Spec{\kappa(\eta)}$ is geometrically connected

\item $H^0(X_\eta, \struct{X_\eta}) = \kappa(\eta)$

\item $f_* \struct{X} = \struct{S}$
\end{enumerate}
then the following implications hold,
\begin{center}
\begin{tikzcd}
(a) \arrow[d] \arrow[r] & (d) \arrow[d]
\\
(b) \arrow[u, bend left, "X_\eta \text{ geom. \kern-0.5em red.}"] \arrow[d] & (e) \arrow[ld] \arrow[u]
\\
(c) \arrow[u, bend left, "X \text{ normal}"]
\end{tikzcd}
\end{center}
In particular, if $X$ is normal and $X_\eta$ is geometrically reduced all the properties are equivalent.
\end{prop}

\begin{proof}
The only nontrivial implications are:
\begin{itemize}
\item $(a) \implies (d)$ is \chref{https://stacks.math.columbia.edu/tag/0BUG}{Tag 0BUG} (8)
\item $(d) \implies (e)$ is exactly Lemma~\ref{lemma:normal_o_conn}
\item $(c) \implies (b)$ is Lemma~\ref{lemma:normal_geom_integral} and the fact that geometric connectedness of fibers can be checked generically in universally open (e.g. flat finitely presented) families [EGA IV, Cor. 15.5.4].
\end{itemize} 
\end{proof}

\begin{rmk}
Even if $f_* \struct{X} = \struct{S}$ we don't necessarily have that $X_\eta$ is geometrically reduced e.g. Example~\ref{example:geometrically_nonreduced}.
\end{rmk}


\subsection{Controlling the Arithmetic Genus in Families}

\subsubsection{Setup}

Let $X \to S$ be a normal degeneration of curves. Then consider the following data. Let $\Gamma_i \subset X_s$ be the (reduced) irreducible componetns of the special fiber and the following $\kappa$-algebras,
\begin{enumerate}
\item $A = H^0(X_s, \struct{X_s})$

\item $\kappa' = H^0((X_s)_\red, \struct{(X_s)_\red})$

\item $\kappa_i = H^0(\Gamma_i, \struct{\Gamma_i})$
\end{enumerate}
where $A$ is an Artin local $\kappa$-algebra and $\kappa'$ and $\kappa_i$ are finite field extensions of $\kappa$ by \chref{https://stacks.math.columbia.edu/tag/0BUG}{Tag 0BUG} (1) since these schemes are connected and the second two are reduced.

\subsubsection{Results in Any Characteristic}

For simplicitly, we should either restrict in this section to the case where the curve $C = X_\eta$ over $K$ is smooth and $H^0(C, \struct{C}) = K$ or the base DVR $R$ is excellent. I think the results should be true without this assumption but I want to quote the stacks project which makes this assumption without caution and use strong desingularization which either relies on smoothness of $C$ or excellence in an important way (see [Liu, Cor. 8.3.51] and [Liu, 8.3.44]).

\begin{prop} \label{prop:regular_inequality}
Let $X \to S$ be a regular degeneration of curves. For any irreducible component, 
\[ p_a(\Gamma_i / \kappa_i) \le p_a(X_K/K) \]
\end{prop}

\begin{proof}
By \chref{https://stacks.math.columbia.edu/tag/0C68}{Tag 0C68} and \chref{https://stacks.math.columbia.edu/tag/0C69}{Tag 0C69} the effective Cartier divisor,
\[ C = \sum_{i = 1}^r (m_i / d) C_i \]
where $m_i$ is the multiplicity of $C_i$ and $d = \gcd(m_i)$ satisfies,
\begin{enumerate}
\item $H^0(D, \struct{D}) = \kappa_D$ is a field and
\item $\chi(X_s, \struct{X_s}) = d \chi(D, \struct{D})$ 
\end{enumerate}
and hence,
\[ g - 1 = d [\kappa_D : \kappa] (g_D - 1) \]
where $g = p_a(X_K/K)$ and $g_D = p_a(D/\kappa_D)$. Therefore,
\[  g_D = \frac{g-1}{d [\kappa_D : \kappa]} + 1 \le g \]
since either $g = 0$ in which case $g_D = 0$ and $d = [\kappa_D : \kappa] = 1$ or $g > 0$ in which case we see that $g_D \le g$. Furthermore, since $(X_s)_{\red} = D_{\red}$ we see that $(X_s)_{\red}$ is a $\kappa_D$-scheme and,
\[ H^1(X_s, \struct{D}) \onto H^1(X_s, \struct{(X_s)_{\red}}) \]
so we conclude that,
\[ p_a((X_s)_{\red}/\kappa') \le p_a((X_s)_{\red}/\kappa_D) \le p_a(D/\kappa_D) = g_D \le g = p_a(X_K/K) \]
Let $Y = (X_s)_{\red}$ and consider the finite map $\pi : \bigsqcup_i \Gamma_i \to Y$ splitting the irreducible components. This gives a sequence of sheaves,
\begin{center}
\begin{tikzcd}
0 \arrow[r] & \struct{Y} \arrow[r] & \prod \struct{\Gamma_i} \arrow[r] & \Csh \arrow[r] & 0
\end{tikzcd}
\end{center}
and therefore since $\dim{\Supp{}{\Csh}} = 0$ we have,
\[ H^1(Y, \struct{Y}) \onto \bigoplus_{i} H^1(\Gamma_i, \struct{\Gamma_i}) \]
and hence,
\[ p_a(\Gamma_i/\kappa_i) \le p_a(\Gamma_i / \kappa') \le p_a(Y/\kappa') \le p_a(X_K/K) \]
\end{proof}

\begin{prop} \label{prop:normal_inequality}
Let $X \to S$ be a normal degeneration of curves and $\Gamma_i \subset X_s$ a (reduced) irreducible component which is \textit{normal} then,
\[ p_a(\Gamma_i / \kappa_i) \le p_a(X_K/K) \]
\end{prop}

\begin{proof}
Let $X$ be a normal degeneration of curves. Then there exists a strong desingularization [Liu, Cor. 8.3.51] $\pi : \wt{X} \to X$ meaning it is an isomorphism on the regular locus of $X$. Then $\wt{X} \to S$ is a regular model of $\wt{X}_\eta = X_\eta$ and hence verifies the inequality for \textit{every} irreducible component via Proposition~\ref{prop:regular_inequality}. However, since $\wt{X} \to X$ is an isomorphism away from a finite set of points, for each $\Gamma_i \subset X_s$ there is an irreducible component $\wt{\Gamma}_i \subset \wt{X}_s$ mapping birationally onto $\Gamma_i$. However, $\Gamma_i$ is a normal curve so the birational map $\wt{\Gamma}_i \to \Gamma_i$ is an isomorphism and hence we conclude.
\end{proof}


\section{Application to Specializing Genus of Fibrations}

As before let $(R, \m, \kappa)$ be a DVR with fraction field $K = \Frac{R}$ and spectrum $S = \Spec{R}$.

\begin{prop}
Let $f : X \to Y$ be a morphism of flat proper $S$-schemes. Suppose that,
\begin{enumerate}
\item $X$ and $Y$ are integral and normal
\item $X_s$ decomposes into (reduced) irreducible components $X_1, \dots, X_r$ with multiplicites $m_i$
\item $Y_s$ is integral 
\item the map $f_s : X_s \to Y_s$ is dominant
\item $f$ has relative dimension $1$ 
\end{enumerate}
Let $\xi \in Y$ and $\eta \in Y_s$ be the generic points. Then, the fiber $X_\eta$ of $f$ decomposes into (reduced) irreducble components $\Gamma_1, \dots, \Gamma_r$ with multiplicities $m_i$ such that $\Gamma_i$ is the generic fiber of $X_i \to Y_s$. Thus, for each normal $X_j$ we have,
\[ p_a(\Gamma_j / \kappa_j) \le p_a(X_\xi / \kappa(\xi)) \]
where $\kappa_j = H^0(\Gamma_j, \struct{\Gamma_j})$. and thus the Stein factorization,
\[ X_j \to B \to Y_s \]
has $X_j \to B$ a fibration by curves with generic fiber of genus $\le p_a(X_\xi/\kappa(\xi))$.
\end{prop}

\begin{proof}
Since $X$ is integral, the fiber dimension can only jump on a codim $\ge 2$ locus. In fact, the following lemma gives a strengthening of generic flatness. Now $(X_s)_{\eta} = X_{\eta}$ and the generic points of $\Gamma_i$ and of $X_i$ are the same in $X$ and hence they come with the same multiplicity. Indeed, this is a purely local situation. Consider the diagram of ring maps,
\begin{center}
\begin{tikzcd}
A \arrow[from=rd] \arrow[from=rr] & & B \arrow[from=ld]
\\
& R
\end{tikzcd}
\end{center}
with $\m \subset R$ the maximal ideal. Then $\m B$ is prime by assumption. Let $S = B \sm (\m B)$. Then,
\[ (R / \m) \ot_R A \ot_B (S^{-1} B) = [(B/\m) \sm \{ 0 \}]^{-1} (A / \m A) \]
has as its local rings those points of the special fiber of $X$ mapping to $\eta$. By flatness (see below) each generic point of $X_s$ maps to $\eta$ so we get the assumed decomposition. Now consider $X_D \to \Spec{D}$ where $D = \stalk{Y}{\eta}$ is a DVR since $Y$ is regular in codimension $1$. Since $X$ is irreducible and normal and $X_D \to X$ is a localization then $X_D$ is also irreducible and normal. Finally, $X_R \to \Spec{D}$ is proper by basechange and flat because it is dominant by assumption. Therefore, we conclude by an application of Propositione~\ref{prop:normal_inequality}.
\end{proof}


\end{document}
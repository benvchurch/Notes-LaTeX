\documentclass{article}
\usepackage[utf8]{inputenc}

\usepackage{import}
\import{"../Algebraic Geometry/"}{AlgGeoCommands}


\newcommand{\cQ}{\mathcal{Q}}
\newcommand{\Gr}{\mathrm{Gr}}
\newcommand{\MGL}{\mathrm{MGL}}
\newcommand{\BGL}{\mathrm{BGL}}


\newcommand{\SmP}{\mathrm{SmP}}
\newcommand{\Mot}{\mathrm{Mot}}
\newcommand{\ess}{\mathrm{ess}}
\newcommand{\Vect}{\mathrm{Vect}}
\newcommand{\Qbar}{\ol{\Q}}
\newcommand{\cV}{\mathcal{V}}
\newcommand{\LL}{\mathbf{L}}
\newcommand{\cO}{\mathcal{O}}
\newcommand{\Sm}{\mathrm{Sm}}
\newcommand{\Nis}{\mathrm{Nis}}
\newcommand{\Set}{\mathrm{Set}}
\newcommand{\Th}{\mathrm{Th}}
\newcommand{\cN}{\mathcal{N}}
\newcommand{\Spc}{\mathrm{Spc}}
\newcommand{\irr}{\mathrm{irr}}

\newcommand{\Def}{\mathrm{Def}}
\newcommand{\Spf}[1]{\mathrm{Spf}\left( #1 \right)}

\renewcommand{\top}{\mathrm{top}}
\newcommand{\pr}{\mathrm{pr}}

\DeclareMathOperator{\holim}{\mathrm{holim}}

\begin{document}


\section{PCMI Notes}

Define the moduli spaces.

\begin{prop}
$\M(X, r, \L, T_i) \to M_B(X, r, \L, T_i)$ is a $\Gm$-gerbe. 
\end{prop}

{\color{red} TODO}

\begin{theorem}
If there is one irreducible topological rank $r$ complex local system $\LL_{\CC}$ with determiniant $\L$ and monodromies in $T_i$ at infinity, then there is a non-empty open subscheme $S^\circ \subset S$ such that for any two closed points $s,s' \in |S|$ of residual characteristic $p \neq p'$ the following hold:
\begin{enumerate}
\item for any prime $\ell \neq p$ there is one arithmetic local system $\LL_{\ell, \bar{s}}$ on $X_{\bar{s}}$
\item which has determinant $\L$, with quasi-unipotent monodromies $T_{i,\ell, \bar{s}}$ at ininifty such that $T_i^{ss} = T_{i,\ell, \bar{s}}^{ss}$
\item which is irreducible over $\ol{\Q}_\ell$
\item for $\ell = p$ there is one arithmetic local system $\LL_{p,\bar{s}'}$ on $X_{\bar{s}'}$  with (s), (3) where $\ell$ is replaced by $p$
\item for any prime $\ell$, the topological pullback $(\mathrm{sp}_{\CC,\bar{s}}^{\top})^* \LL_{\ell, \bar{s}}$ have properties (2) and (3) as topological local systems. 
\end{enumerate}
\end{theorem}

\begin{proof}

\end{proof}

\renewcommand{\sp}{\mathrm{sp}}

\subsection{Grothendieck Specialization for Fundamental Groups}

Let $X_S \to S$ be a smooth morphism, where $S$ is any scheme. Consider two field valued points $\eta : \Spec{F} \to S$ and $s : \Spec{k} \to S$ with the property that $\im{s}$ lies in the Zariski closure of $\im{\eta}$. Therefore, there is an irreducible subscheme $Z \subset S$ such that $s \in Z$ is a point and $\cO(Z) \to F$ is injective. Let $\wh{Z}$ be the completion along $s$ and $F \embed \wh{F}$ a field extension such that $\wh{F}$ contains $\cO(\wh{Z})$.
\bigskip\\
If $X$ is not proper, we assume there exists a relative compactification: $X_S \embed \ol{X}_S$ such that,
\begin{enumerate}
\item $\ol{X}_S \to S$ is smooth proper,
\item $\wh{X}_X \sm X_S \to S$ is a relative normal crossings divisor with smooth components. 
\end{enumerate}
We call this a \textit{good} compactification. 
\bigskip\\
Therefore, we have a diagram
\[ \Spec{\wh{F}} \to \wh{Z} \leftarrow s \]
together with the scheme over it
\begin{center}
\begin{tikzcd}
X_{\wh{Z}} \arrow[d] \arrow[r] & X_{\wh{Z}} \arrow[from=r] & X_s \arrow[d]
\\
\Spec{\wh{F}} \arrow[r] & \wh{Z} \arrow[from=r] & s
\end{tikzcd}
\end{center}
We denote by $\ol{\wh{F}} \supset \wh{F}$ and $\ol{k} \supset k$ algebraic closures, the latter defining a morphism $\bar{s} \to s$. Then, upon choosing an $S$-point $x_S : S \to X_S$, one defines a specialization homomorphism
\[ \sp_{\wt{F}, s} : \pi^t_1(X_{\wh{F}}, x_{\wh{F}}) \to \pi^t_1(X_s, x_s) \]
which is the composite of the functoriality homomorphism
\[ \pi_1^t(X_{\wh{F}}, x_{\wh{F}}) \to \pi^t_1(X_{\wh{Z}}, x_s) \]
and the inverse of the base change isomorphism
\[ \pi_1^t(X_s, x_s) \iso \pi_1^t(X_{\wh{Z}}, x_s) \]
Finally, one has the functoriality homomorphism
\[ \pi_1^t(X_{\wh{F}}, x_{\wh{F}}) \to \pi_1^t(X_F, x_F) \]
which is an isomorphism in restriction to the geometric fundamental groups
\[ \pi^t_1(X_{\ol{\wh{F}}}, x_{\ol{\wh{F}}}) \iso \pi_1^t(X_{\ol{F}}, x_{\ol{F}}) \]
Taken together, this defines the specialization homomorphism
\[ \sp_{F,s} : \pi_1^t(X_F, x_F) \to \pi^t_1(X_s, x_s) \]
which, when retricted to the geometric fundamental groups, defines the specialization homomorphism
\[ \sp_{\ol{F}, \ol{s}} : \pi^t_1(X_{\ol{F}}, x_{\ol{F}}) \to \pi_1^t(X_{\ol{s}}, x_{\ol{s}}) \]

\begin{theorem}
$\sp_{F,s}$ and $\sp_{\ol{F}, \ol{s}}$ are surjective, and $\sp_{\ol{F}, \ol{s}}$ induces an isomorphism on the pro-p'-completion. 
\end{theorem}


\subsection{Open Questions}



\section{Lecture 1 Fabien Morel}

Brauer Degree: a map $f : S^n \to S^n$ has a degree $\deg{f} \in \Z$. In fact, 
\[ [S^n, S^n] \cong \Z \]
If $f$ is differentiable then for a regular value $y \in S^n$ we have
\[ \deg{f} = \sum_{x \in f^{-1}(y)} \epsilon(x) \]
where $\epsilon(x)$ is the sign given by the orientation on the tangent space at that point. 
\bigskip\\
Let $X$ be a CW complex. Then there is a classifying space $B \GL_n(\RR)$ such that it is a delooping $\Omega B \GL_n(\RR) \cong \GL_n(\RR)$ and 
\begin{theorem}
For any $X$ there is a natural isomorphism
\[ [X, B \GL_n(\RR)] \cong \Vect_{\RR}(X) \]
\end{theorem}

It follows from the theory of the Euler class that the adding a trivial bundle map gives a fibration,
\[ \GL_n(\RR) / \GL_{n-1}(\RR) \to B \GL_{n-1}(\RR) \to B \GL_n(\RR) \]
(this is just the shift of the tautological fibration $\GL_{n-1}(\RR) \to \GL_n(\RR) \to \GL_n(\RR) / \GL_{n-1}(\RR)$. Note that $\GL_n(\RR) / \GL_{n-1}(\RR) \cong S^{n-1}$.

\begin{theorem}
Let $X$ be a smooth manifold of dimension $n$. If $\xi$ is an oriented rank $r$ vector bundle there is an Euler class
\[ e(\xi) \in H^n(X, \Z) \]
such that $e(\xi) = 0$ iff $\xi$ splits off a trivial line bundle as a summand. 
\end{theorem}

If one considers $\Z \times B \GL_{\infty}(\RR)$ (the colimit of the inclusions)
and also the complex version $\Z \times B \GL_{\infty}(\CC)$ these classify $K$-theory
\[ [X, \Z \times B \GL_{\infty}(\RR)] = K_0^{\RR}(X) \]
and likewise
\[ [X, \Z \times B \GL_{\infty}(\CC)] = K_0^{\CC}(X) \] 

\begin{theorem}[Bott]
$\Omega^8 (\Z \times B \GL_{\infty}(\RR)) \cong \Z \times B \GL_{\infty}(\RR)$ and likewise $\Omega^2 (\Z \times B \GL_{\infty}(\CC)) \cong \Z \times B \GL_{\infty}(\CC)$. 
\end{theorem}

Spectra: 
\begin{enumerate}
\item $H \Z$ represents singular cohomology
\item homology 
\[ \H \xrightarrow{C_\bullet} D(Ab) \]
is taking the singular chains in the derived category of abelian groups where $\H$ is the homotopy category of CW-complexes. 
\item if $i : M \embed N$ is an embedding of a compact manifold, let $T \subset N$ be a tubular neighborhood of $M$ then $N \sm T$ is closed. If we colapse $N \sm T$ then $N / (N \sm T)$ is the Thom space of $N_{M|N}$. This is the Thom-Pontryagin construction. 
\end{enumerate}

\subsection{$\A^1$-algebraic topology over a field $k$}

Let $k$ be a field. $\Sm_k$ the category of smooth finite type separated $k$-schemes. First we define the notion of ''naive'' $\A^1$-homotopy. 

\begin{defn}
Two maps $f, g : X \to Y$ are \textit{naive} $\A^1$-homotopic if there exists $h : \A^1 \times X \to Y$ such that $h_0 = f$ and $h_1 = g$. 
\end{defn}

This gives a reasonable relation but it is not enough to be useful. It can also be badly behaved: it is not even an equivalence relation in general!

\begin{example}
$X \to \P^n_k$ is given by the data of $\struct{X}^{n+1} \onto \L$ on $X$. Then we consider,
\[ \Hom{k}{X}{\P^\infty_k} = \colim_{n \to \infty} \Hom{k}{X}{\P^n_k} \] 
If we mod out by naive homotopy we get exactly the monoid of line bundles generated by their global sections. This is not what we want, we want to get $\Pic{X}$. 
\end{example}

\begin{example}
The fibration $\Gm \to \A^{n+1} \sm \{ 0 \} \to \P^n$ give a fibration
\[ \Gm \to \A^{\infty} \sm \{ 0 \} \to \P^\infty \]
and the second term should be contractible so we see that we want $\P^{\infty} = B \Gm$ but the above calculation shows we need more that naive homotopy to realize this dream. 
\end{example}

We need to extend $\Sm_k \embed \Sp_k$ to spaces over $k$. There are many ways to do this. We will work with sheaves of sets in the Nisnevich topology. 

\begin{defn}
A sheaf of sets $\F$ on $(\Sm_k)_{\Nis}$ is a functor
\[ \F : (\Sm_k)^{\op} \to \Set \]
such that for all squares induced by \etale $f : Y \to X$ and open $U \embed X$,
\begin{center}
\begin{tikzcd}
V \arrow[r, hook, "\text{open}"] \arrow[d] \pullback & Y \arrow[d, "\et"]
\\
U \arrow[r, hook, "\text{open}"] & X
\end{tikzcd}
\end{center}
such that $Z := X \sm U$ with the reduced structure so that $f^{-1}(Z) := Z \times_X Y \to Z$ is an isomorphism then
\[ \F(X) \to \F(U) \times_{\F(V)} \F(Y) \]
is a bijection. 
\end{defn}

Then we define $\Sp_k := \Delta^{\op} \Sh((\Sm_k)_{\Nis})$  is the category of simplicial Nisnevich sheaves. 
\begin{enumerate}
\item if $G$ is a sheaf of groups then $B \G$ is given by the simplicial object
\[ \cdots \to G^2 \rightrightarrows G \to * \]
\item if $U \embed X$ then we can define the quotient sheaf $X / U$. 
\end{enumerate}

We define the notion of weak homotopy equivalence as generated by two types
\begin{enumerate}
\item given a Nisnevich hypercovering $U_\bullet \to X$ it is an equivalence
\item $X \times \A^1 \to X$ is an equivalence 
\end{enumerate}
This gives $\H_S(k)$ the simplicial homotopy category $\Sp_k [W_S^{-1}]$ where $W_S$ are the simplicial hypercovers and $\H_{\A^1}(k) := \H_S(k) [W_{\A^1}^{-1}]$ where $W_{\A^1}$ are the projections from $\A^1$.

\begin{example}
$\P^1_k = \A^1_k \cup \A^1_k$ and $\A^1_k \cap \A^1_k = \Gm$. Therefore,
\[ \P^1_k \cong_{W_{\A^1}} \Sigma \Gm \]
is the suspension of $\Gm$. 
\end{example}

\begin{theorem}[Purity]
If $i : Z \embed X$ is a closed subscheme in $\Sm_k$. Then there exists a canonical $\A^1$-weak equivalence
\[ X / (X \sm Z) \cong \Th(\cN_{Z|X}) = \cN_{Z|X} / (\cN_{Z|X} \sm \text{zero}) \]
\end{theorem}

You should think about the left as $X$ modulo the complement of a tubular neighborhood of $Z$ since this is homotopy equivalent to $X \sm Z$ and that one gives a nicer topological space if we took realization over $\CC$.
\bigskip\\
The same way, a smooth $k$-scheme $X$ ``looks locally'' at $x \in X$ like affine space
\[ X / (X \sm \{ x \}) \cong \A^n_k \sm (\A^n_k \sm \{ x \}) \]
Then we will be able to define $B \GL_n$ algebraically. 

\begin{rmk}
Why is it called ``purity''? There is a cofiber sequence
\[ X \sm Z \to X \to \Th(\cN_{Z|X}) \]
Then if we apply cohomology to this we get a long exact sequence this gives a purity satement if we understand $\Th(\cN_{Z|X})$ which by the Thom theorem gives shifted cohomology of $Z$. 
\end{rmk}

\section{Aravind Asok (USC)}

Let $M$ be a nice space (manifold, CW complex) of dimension $d$. Let $V_r(M)$ be the set of isomorphism classes of real rank $r$ vector bundles on $M$. 

Homotopy invariance:
\[ V_r(M) \to V_r(M \times I) \]
is a bijection. Consider $Gr_r^{\text{top}}$ be the Grassmannian of $r$-dimensional subspaces of an infinite dimensional real vector space which has a rank $r$ ``tautological'' vector bundle. 

\begin{theorem}
Pulling back the tautological bundle gives an isomorphism
\[ [M, Gr_r^{\text{top}}] \cong V_r(M) \]
There is a map $s_r : Gr_{r-1}^{\text{top}} \to Gr_r^{\text{top}}$ classifying the sum of the tautological bundle of rank $r-1$ and a trivial rank $1$ bundle.
\end{theorem}

A rank $r$ bundle has a nowhere vanishing section iff it has a trivial line bundle summand.


Take a closed manifold $M$ of dimension $d$ and fix $\xi : M \to Gr_r^{\text{top}}$. 
\begin{enumerate}
\item by the corank of $\xi$ we will mean $d-r$
\item if $r > d$ then a generic section of $\xi$ is nowhere vanishing
\end{enumerate} 

Take $r = d$. There is an obstruction to existence of a nowhere vanishing section: the cohomology class Poincare dual to the vanishing locus of a generic section the twisted (b/c we don't put an orientation on it) Euler class. Since $\pi_1(Gr_r^{\text{top}}) = \Z / 2$ with the determinant, $\xi$ yields an orientation character
\[ \omega_\xi : \pi_1(M) \to \Z / 2 \]
and hence an orientation local system on $M$.

\begin{theorem}
Given a topological space having the homotopy type of a CW complex of dim $d$ and $\xi$ as above then the vanishing of the twisted Euler class
\[ e(\xi) \in H^n(M, \Z/2[\xi]) \]
is the only obstruction. 
\end{theorem}

The failure of $s_r$ to be a weak homotopy equivalence is given by the fiber sequence
\[ S^{r-1} \to Gr_{r-1}^{\text{top}} \to Gr_{r}^{\text{top}} \] 
Use obstruction theory. The Euler class arises from:
\[ \deg : \pi_{r-1}(S^{r-1}) \iso \Z \]
for $r \ge 2$, dimension hypothesis implies this is the only obstruction. This group is stable: the group is independent of $r$. 

\begin{theorem}
Let $M$ be a manifold of dimension $d+1$, for $d \ge 4$. If $\xi$ is an oriendted rank $d$ vector bundle on $X$, then $\xi$ splits off a trivial rank $1$ summand iff 
\begin{enumerate}
\item $e(\xi) \in H^d(M, \Z)$ and
\item $o_2(\xi) \in H^{d+1}(M, \Z/2) / (\mathrm{Sq}^2 + w_2(\xi) \smile) H^{d-1}(M, \Z / 2)$. 
\end{enumerate}
both vanish.
\end{theorem}
Indeed, if the primary obstruction vanishes, there is a well-defined secondard obstruction arising from:
\[ \pi_d(S^{d-1}) = 
\begin{cases}
\Z & d = 3
\\
\Z / 2 & d \ge 4
\end{cases} \]
If $d \ge 4$ this is stable and unstable for $d = 3$. 

\subsection{Splitting algebraic vector bundles I}

Let $R$ be a commutative unital ring. Recall the following are equivalent:
\begin{enumerate}
\item $P$ is projective
\item $P$ is a summand of a free module
\item $P$ satisfies lifing over surjections
\item if $P$ is also finitely generated, then there exists $n$ and $\epsilon \in \End[R]{R^{\oplus n}}$ such that $P = \epsilon R^{\oplus n}$.
\end{enumerate}

Projective modules are vector bundles. 

\begin{defn}
Suppose $R$ is a Noetherian ring of Krull dim $d$ and $P$ is a projective $R$-module of rank $r$
\begin{enumerate}
\item $V_r(\Spec{R})$ is isomorphism classes of rank $r$ projective $R$-modules 
\item corank: $d - r$.
\end{enumerate}
Consider the stabilization map
\[ s_r : V_{r-1}(\Spec{R}) \to V_r(\Spec{R}) \]
\end{defn}

\begin{theorem}
The function $s_r$ is surjective for $r > d$ (i.e. negative corank). 
\end{theorem}

\begin{rmk}
Analogue of general position argument, a generic section of a topological vector bundle of of rank $r > d$ is nowhere vanishing. It is very important here we are working with affine schemes. It is also important such that ``existence of nowhwere vanishing section'' and ``splits off a trivial line bundle'' are equivalent (exact sequences of projectives split). 
\end{rmk}

What in general can we say about $s_r?$. First we consider the corank zero case.

Assume $k$ is a field, and $R$ is a smooth $k$-algebra of dim $d$ and $P$ a projective $R$-module f rank $d$. Obstruction: ``vanishing locus of generic section'' i.e. $c_d(P) \in \CH^d(\Spec{R})$. This obstruction is insufficient in general (unlike in topology): the tangent bundle to the real $2$-sphere ($X = V(x^2 + y^2 + z^2 = 1)$) has $c_2(T_X) = 0$ but does not split. 

\begin{theorem}
If $k$ is algebraically closed, and $X$ is a smooth affine $d$-fold over $k$, then the image of $s_{d-1, X}$ consists of those $\E$ of rank $d$ such that $c_d(\E) \in \CH^d(X)$ vanishes.
\end{theorem}

\subsection{Vector Bundles and Motivic Homotopy Theory}

$Gr_r$ is an infinite-dimensional limit of algebraic varieties. A map $\Spec{R} \to Gr_r$ corresponds to a vector bundle on $R$ and a collection of generating sections. The ``naive'' homotopies between $f,g : \Spec{R} \to Gr_r$ are those arising from $H : \Spec{R[t]} \to Gr_r$ with $H_0 = f$ and $H_1 = g$.

\begin{theorem}
If $k$ is a field, $R$ is a regular $k$-algebra then
\[ [\Spec{R}, Gr_r]_{\text{naive}} \iso V_r(\Spec{R}) \] 
\end{theorem}

This is much harder than the topological analog. For $R = k[t_1, \dots, t_n]$ it implies Quillen-Suslin theorem that all vector bundles on affine space are trivial.

The restriction to regular rings is necessary in order to have an analog of homotopy invariance. {\color{red} WHY}

Let $\Sm_k$ be the category of smooth $k$-varieties. Let $P(\Sm_k)$ be the space-valued presheaves. Then we invert equivalences:
\begin{enumerate}
\item Nisnevich local equivalences
\item $\A^1$-equivalences: $X \times \A^1 \to \A^1$.
\end{enumerate}

$\Spc_k$ is what we get by inverting in the $\infty$-category sense. 

If $k$ is regular, one can show $[X, \P^\infty]_{\A^1} = \Pic{X}$. Unfortunately, this fails for $Gr_r$ for $r \ge 2$ (because $V_r$ is not homotopy invariant).

\begin{theorem}
If $k$ is a field or $\Z$, then for any smooth affine $k$-scheme $X = \Spec{R}$,
\[ [\Spec{R}, Gr_r]_{\text{naive}} = [\Spec{R}, Gr_r]_{\A^1} \iso V_r(\Spec{R}) \]
\end{theorem}

\begin{theorem}
If $k$ is a field or a Dedekind ring with perfect residue fields, then there is an $\A^1$-fiber sequence of the form
\[ \A^r \sm 0 \to Gr_{r-1} \to Gr_r \]
\end{theorem}

There are two circles in $\A^1$-homotopy theory: $S^1, \Gm$. There are equivalences in $\Spc_k$ of the form $\P^1 \sim S^1 \wedge \Gm$ and $\A^n \sm 0 \sim S^{n-1} \wedge \Gm^{\wedge n}$.  Note that $\A^{n+1} \sm 0 \sim \P^1 \wedge \A^n \sm 0$.
We set:
\[ S^{p,q} := S^{p-q} \wedge \Gm^{\wedge q} \]

We must consider homotopy  sheaves not homotopy groups (gives correct form of Whiteheads' theorem, etc) but those associated with $S^i$ are most important.

\begin{theorem}[Morel]
For any $n \ge 2$, the space $\A^n \sm 0$ is $(n-2)$-$\A^1$-connected. 
\end{theorem}

\begin{theorem}[Morel]
For any $n \ge 2$, there is an isomorphism
\[ \pi_{n-1}^{\A^1}(\A^n \sm 0) \cong K_n^{MW} \]
Milnor-Witt K-theory. 
\end{theorem}

\begin{theorem}
Suppose $k$ is a field, $X$ is a smooth affine $k$-variety of dimension $d$, and $\xi : E \to X$ is a rank $d$ algebraic vector bundle on $X$. There exists a canonically defined ``Euler class'
\[ e(\xi) \in H^d(X, K_n^{MW}(\det{\xi})) = \wt{\CH}^d(X, \det{\xi}) \]
whose vanishing is sufficient to gauarantee $E$ splits off a free rank $1$ summand. There is a map $\wt{\CH}^d(X, \det{\xi}) \to \CH^d(X, \det{\xi})$ sending $e(\xi) \mapsto c_d(\xi)$. 
\end{theorem}

\begin{rmk}
When $k$ is algebraically closed, then $\wt{\CH}^d \to \CH^d$ is an isomorphism so we reduce to $c_d$ being the only obstruction.
\end{rmk}

\begin{conj}
Suppose $k$ is an algebraically closed field and $d \ge 1$ is an integer. If $X$ is a smooth affine $k$-variety of dimension $d + 1$, and $E$ is a rank $d$ vector bundle on $X$, then $E \cong E' \oplus \struct{X}$ if and only if $c_d(E) = 0$ in $\CH^d(X)$. 
\end{conj}

So we are saying that the secondary obstructions vanish when $k$ is algebraically closed. 

\begin{theorem}
The conjecture holds if $k$ has characteristic zero. 
\end{theorem}

\begin{theorem}
For any integer $n \ge 0$, the unit map
\[ S^n \to \Omega \Sigma S^n \]
is $2n - 1$-connected. 
\end{theorem}

Morel established an exact analog of this theorem for $S^1$-suspension. Since $\A^{n+1} \sm 0 \sim \P^1 \wedge \A^n \sm 0$, we need to understand $\P^1$-suspension. Note: Freudenthal fails for $\P^1$-suspension:
\[ \pi_n^{\A^1}(S^n) = \Z \quad \text{ but } \quad \pi_n^{\A^1}(\Omega^{2,1} S^{n+2,1}) = K_0^{MW} \]


\section{Notes}

\subsection{Irreducible Representations}

$M_B := M_B(\pi, r)$ the Betti moduli space. Then there are two notions of irreducible:
\[ M_B^{\irr} = M_B \sm Z \quad \text{ where } Z := \ol{(M_B)_{\CC} \sm (M_B)_{\CC}^{\irr}} \]
where we consider the irreducible complex representations. Moreover, we can consider the functor,
\[ A \mapsto \{ \rho : \pi \to \GL_r(A) \mid \text{ for all } \Spec{K} \to \Spec{A} : \rho : \pi \to \GL_r(K) \text{ is irreducible} \} \]
this implies absolutely irreducible at each place. This is representable by $M_B^{\text{ab.irr}}$ and we see
\[ M_B^{\text{ab.irr}} \subset M_B^{\irr} \subset M_B \]

\subsection{Helene and de Jong show}

At certain nice points, the Mazur deformation space of the residual representation agrees with the completion of $M_B$. We don't know if this holds for Chenevier deformation space. 

\section{Alexander Merkurjev lecture 1}

\begin{theorem}[Voevodsky-Rest] If $\zeta_p \in F$ then $H^\bullet(F, \Z / p \Z)$ is generated by $H^1(F, \Z / p \Z)$ with relations in degree $2$. 
\end{theorem}

Massey (1958). Let $\Gamma$ be a profinite group, $p$ a prime. Dwger (1975) gives a definition of the Massey product in this setting. Consider the subgroup of strictly upper triangular matrices ($1$ on diagonal),
\[ U_{n+1} \subset \GL_{n+1}(\Z / p \Z) \]
is the Sylow $p$-subgroup. The center $Z(U_{n+1})$ is given by matrices with one nonzero entry at the top right and 1 on the diagonal. Then there is a map $U_{n+1} \to (\Z / p \Z)^n$ given by the first off diagonal this factors through $\ol{U}_{n+1} = U_{n+1} / Z(U_{n+1})$. Consider $\chi_1, \dots, \chi_n \in H^1(\Gamma, \Z / p \Z) = \Hom{}{\Gamma}{\Z / p \Z}$. Thus we get a diagram,
\begin{center}
\begin{tikzcd}
& & \Gamma \arrow[d, "\chi"]  \arrow[ld, dashed, "\bar{\rho}"] \arrow[lld, dashed, "\rho"]
\\
U_{n+1} \arrow[r] & \ol{U}_{n+1} \arrow[r] & (\Z / p \Z)^n
\end{tikzcd}
\end{center} 
we want to understand when such liftings exist. Such a lift is given by a matrix of functions $\bar{\rho}_{ij} : \Gamma \to \Z / p \Z$ lift means that $\bar{\rho}_{i,i+1} = \chi_i$. To be a lifting we need that $\bar{\rho}$ is a homomorphism. Then $\rho$ extends $\bar{\rho}$ if $\rho_{ij} = \bar{\rho}_{ij}$ for all $(i,j) \neq (1, n+1)$. We need to check when $\rho$ is a group homomorphism. If we let $\eta = \bar{\eta}_{1,n+1}$ then we would require
\[ \eta(xy) = \eta(y) + \sum_{i = 2}^n \bar{\rho}_{1,i}(x) \bar{\rho}_{i, n+1}(y) + \eta(x) \]
This is exactly asking that $\Delta \in H^2(\Gamma, \Z / p \Z)$ is zero because we can rewrite the above in the form
\[ \Delta = - \partial \eta \]
Let the function in the middle be $\Delta(\bar{\rho})(x,y)$ is a $2$-cocycle. Consider the diagram
\begin{center}
\begin{tikzcd}
1 \arrow[r] & Z_{n+1} \arrow[r] \arrow[d,equals] & E \arrow[r] \arrow[d] & \Gamma \arrow[d, "\chi"] \arrow[r] & 1
\\
1 \arrow[r] & Z_{n+1} \arrow[r] & U_{n+1} \arrow[r] & \ol{U}_{n+1} \arrow[r] & 1
\end{tikzcd}
\end{center} 

\begin{defn}
If $\Gamma$ is a profinite group, $p$ is profinte, and $n \ge 2$. Given $\chi_1, \dots, \chi_n \in H^1(\Gamma, \Z / p \Z)$ the Massey product
\[ \left< \chi_1, \dots, \chi_n \right> = \{ \Delta(\bar{\rho}) \mid \bar{\rho} : \Gamma \to \ol{U}_{n+1} \text{ lifts } \chi \} \subset H^2(\Gamma, \Z / p \Z) \]
We say the Massey product is \textit{defined} if the set is nonempty i.e. there exists a lift to $\ol{U}_{n+1}$. We say the MAsset product \textit{vanishes} if the set contains zero i.e. $\chi$ admits a lift to $U_{n+1}$. 
\end{defn}

\begin{example}
Let $n = 2$ then the Massey product is always defined since $\ol{U}_{3} \iso (\Z / p \Z)^2$ is an isomorphism. Furthermore, $\Delta(\bar{\rho}) = \chi_1 \smile \chi_2$ is the cup prodcut so the Massey product vanishes iff $\chi_1 \smile \chi_2 = 0$. 
\end{example}

For $n \ge 3$ consider the following statements,
\begin{enumerate}
\item[$A_n$] $\left< \chi_1, \dots, \chi_n \right>$ vanishes
\item[$B_n$] $\left< \chi_1, \dots, \chi_n \right>$ is defined
\item[$C_n$] the cup products $\chi_i \smile \chi_{i+1} = 0$ for all $i$
\end{enumerate}
Then $A_n \implies B_n$ and $B_n \implies C_n$. The second statment holds by looking at $2 \times 2$ blocks looking like
\[ \begin{pmatrix}
\rho_{i,i+1} & \rho_{i+1,i+1}
\\
0 & \rho_{i+1,i+2}
\end{pmatrix} \]
then since this is a group homomorphism we see that $\Delta$ of this block is zero which is the cup product. 
\bigskip\\
In fact $B_3 = C_3$ since we only need to consider the $2 \times 2$ blocks to consider if the map is a homomorphism. 

For $\Gamma = \Gamma_F$ and $\zeta_p \in F$ then consider
\[ 1 \to \mu_p \to F_{\sep}^\times \to F_{\sep}^\times \to 1 \]
And $H^1(F, F^\times) = 1$ and $H^1(F, \mu_p) = H^1(F, \Z / p \Z)$ by the assumption $\zeta_p \in F$. Therefore,
\[ H^1(F, \Z / p \Z) = F^\times / (F^\times)^p \]
Furthermore,
\[ H^2(F, \Z / p \Z) = H^2(F, \mu_p) = H^2(F, F_{\sep}^\times)[p] = \Br(F)[p] \]

\begin{example}
For $a,b \in F^\times$ let $(a,b)$ be the $F$-algebra generated by $i,j$ with $i^2 = a, j^2 = b$ and $ij =  \zeta_p ji$, this is a quaternion algebra for $p = 2$. Then in the Brauer group $(a,b) = \chi_a \cup \chi_b$.
\end{example}

Results:
\begin{enumerate}
\item $A_3 = B_3 = C_3$ when $p = 2$ and $F$ a number field
\item $A_3 = B_3$ for $p = 2$ and any field
\item $A_3 = B_3$ for all $p$ for all $F$
\end{enumerate}

This motivates the conjecture:
\begin{conj}
For any $p$ and any field $F$, if $\left< \chi_1, \dots, \chi_n \right>$ is defined then it vanishes (i.e. $A_n \implies B_n$).
\end{conj}

\begin{theorem}[Harpas-Wittenberg, 2023]
If $F$ is a number field, then the conjecture is true. 
\end{theorem}

\begin{theorem}[M-Scava, 2024]
If $n = 4$ and $p = 2$ and any $F$ then the conjecture is true. 
\end{theorem}

\section{Burt Totaro I}

We will study the Chow groups: $\CH^\bullet(BG)$ of an algebraic group $G$. 

\subsection{Classifying Space}

Let $G$ be an affine group scheme finite type over $k$ meaning a closed subgroup $G \subset \GL_n$. 



What do we do to make $B G$ in motivic homotopy theory. Look at representations $V$ of $G$ over $k$ there is a closed subset $S \subset V$ such that $G \acts (V \sm S)$ freely. By making $V$ large we can make the codimension of $S$ go to infinity. 

\begin{defn}
$BG = \colim_V (V - S)/G$
\end{defn}

For $k = \CC$ we see that $(BG)^{\top} = B G(\CC)$. For example,
\[ H^\bullet(BG, \Z) = H^\bullet(B G(\CC), \Z) \]
so we don't get anything new from cohomology. However, we do also have Chow rings which we can't see in topology. There is a map
\[ \CH^\bullet(BG_{\CC}) \to H^\bullet(BG, \Z) \]
but this is not an isomorphism. 

\subsection{Chow Groups}

Let $X$ be a separated scheme of finite type over $k$ then

\begin{defn}
$\CH_i(X) = Z_i(X) / \text{rat.}$.
\end{defn}

If $X$ is smooth over $k$ of pure dimension $n$ we write $\CH^i(X) := \CH_{n-i}(X)$.
\bigskip\\
For $k = \CC$, there is a cycle map 
\[ \CH_i(X) \to H_{2i}^{BM}(X, \Z) \]
If $X / \CC$ is proper then Borel-More homology equals ordinary homology. 
\bigskip\\
For $X/\CC$ smooth, Poincare duality gives
\[ H_{i}^{BM}(X, \Z) = H^{2n-i}(X, \Z) \]
and therefore there is a cycle class map
\[ \CH^i(X) \to H^{2i}(X, \Z) \]

Formal properties: if $f : X \to Y$ is a \textit{proper} morphism there is a pushforward $f_* : \CH_i(X) \to \CH_i(Y)$. 

\begin{theorem}
For a scheme $X / k$ and a closed subscheme $Z \subset X$ there is an exact sequence,
\[ \CH_i(Z) \to \CH_i(X) \to \CH_i(X \sm Z) \to 0 \]
\end{theorem}

\begin{rmk}
The zero at the end is very different from how things behave in topology. For $k = \CC$, we have a similar exact sequence,
\[ H^{BM}_j(Z) \to H^{BM}_j(X) \to H^{BM}_j(X \sm Z) \to H^{BM}_{j-1}(Z) \to \cdots \]
\end{rmk}

\subsection{Motivic Cohomology}

Let $X / k$ be finite type. Then there is a bigraded cohomology theory
\[ H^i(X, \Z(j)) \]
called motivic cohomology. If $X / k$ is smooth then,
\[ \CH^i(X) = H^{2i}(X, \Z(i)) \]
and in general
\[ \CH_i(X) = H^{BM}_{2i}(X, \Z(i)) \]
is a motivic Borel-More cohomology. Then we use the exact sequences for motivic cohomology. In the smooth case, we get
\[ H^{2i - 1}(X \sm Z, \Z(i)) \to \CH^{i-r}(Z) \to \CH^i(X) \to \CH^i(X \sm Z) \to 0 \]
and we call
\[ H^{2i-1}(-,\Z(i)) = \CH^i(-,1) \]
the first higher Chow group so that it acts like a $\delta$-functor. 

\subsection{Chern classes}

An algebraic vector bundle $E$ on a scheme $X$ has Chern classes $c_i(E) \in \CH^i(X)$ satisfying for
\[ 0 \to A \to B \to C \to 0 \]
implies
\[ c(B) = c(A) \cdot c(C) \]
If $X$ is smooth over $k$ then $\CH^\bullet(X)$ is a commutative graded ring. Over $\CC$ the map
\[ \CH^\bullet(X) \to H^{2\bullet}(X(\CC), \Z) \]
is a ring map.

In the nonsmooth case, we don't have intersection product on $\CH$ or on $H^{BM}$. 

\subsection{Cobordism}

For $k = \CC$ there is a map from Levine-Morel cobordism to complex cobordism,
\[ \Omega^\bullet(X) \to MU^\bullet(X) \]
Note that
\[ \Omega^\bullet(X) \ot_{\Omega^\bullet(*)} \Z \cong \CH^\bullet(X) \]

\section{Burt Lecture 2}

For $i \ge 0$ define
\[ \CH^i(BG) = \CH^i((W \sm S)/G) \]
for any rep $W$ of $G$ and any $G$-invariant closed $S \subset W$ such that $G \acts (W \sm S)$ freely and $\codim{S, W} > i$.

\begin{lemma}
This is well-defined.
\end{lemma}

\begin{proof}
Let $(W_1, S_1)$ and $(W_2, S_2)$ two pairs as above. Then we can form the representation $W_1 \times W_2$ and $G$ acts freely outside of $S_1 \times S_2$. Moreover, there is a $G$-equivariant map on which $G$ acts freely
\[ (W_1 \sm S_1) \times W_2 \to (W_1 \sm S_1) \]
and so when we quotient we get a vector bundle over $W_1 \sm S_1$ (here we use that \etale vector bundles are Zariski vector bundles). Therefore,
\[ \CH^i((W_1 \sm S_1)/G) = \CH^i([(W_1 \sm S_1) \times W_2] / G) \]
and by the same argument
\[ \CH^i((W_2 \sm S_2)/G) = \CH^i([(W_2 \sm S_2) \times W_1] / G) \]
Therefore we just need to compare $(W_1 \sm S_1) \times W_2$ and $W_1 \times (W_2 \sm S_2)$ we compare both to $(W_1 \sm S_1) \times (W_2 \sm S_2)$. Indeed, 
\[ (W_1 \sm S_1) \times (W_2 \sm S_2) \embed (W_1 \sm S_1) \times W_2 \]
is an open immersion with closed complement $Z = (W_1 \sm S_1) \times S_2$ which has codimension $\codim{S_2, W_2} > i$ and therefore by excision the Chow groups for degree $i$ are the same so we win. 
\end{proof}


\section{De Jong's Conjecture}

Let $X$ be a smooth projective curve over a finite field $\FF_q$, and let 
\[ \rho : \pi_1(X) \to \GL_n(\FF_{\ell^n}((t))) \]
be a continuous representation. Assume that $\rho|_{\pi_1(\ol{X})}$ is absolutely irreducible then $\rho(\pi_1(\ol{X}))$ is finite. 

\begin{rmk}
How do you reduce to $X$ is a curve? De Jong does this.
\end{rmk}

\begin{theorem}
Let $X$ be a normal geometrically irreductible variety over $\FF_q$. Let
\[ \rho : \pi_1(X) \to \GL_n(\FF_{\ell^n}((t))) \]
be a continuous representation. Then $\rho|_{\pi_1(\ol{X})}$ has finite image.
\end{theorem}

Let $k_v$ be a finite field of characteristic $\ell$. Let $\bar{\rho} : \pi^t_1(X_{\ol{\FF}_p}) \to \GL_r(k_v)$ be absolutely irreducible, $\Def(\bar{\rho}) \to \Spf{\struct{v}}$ Mazur's deformation space which is a formal scheme formally of finite type. 

\begin{theorem}
If $\Def(\bar{\rho})_{\red}$ is formally smooth, then $\Def(\bar{\rho})_{\red}^{\Phi} \to \Spf{\struct{v}}$ is finite flat. Moreover, 
\[ \bigcup_{n \ge 1} \Def(\bar{\rho})_{\red}^{\Phi^n} \subset \Def(\bar{\rho}) \]
is Zariski dense. 
\end{theorem}

\section{Understanding Esnault-de Jong}

\subsection{Rumely's Local-Global Theorem}

\begin{theorem}
Let $X$ be an affine scheme finite type over a ring of integers $\struct{K}$. If $X_{\ol{K}}$ is irreducible then the following are equivalent,
\begin{enumerate}
\item $X$ has a $\ol{\Z}$-point (i.e. a $\struct{L}$-point for some finite extension $L/K$)
\item $X$ has a $\ol{\Z}_\ell$ point for each $\ell$ (i.e. for each place $\p \subset \struct{K}$ there is a $\p$-adic field $L / K_{\p}$ and $X$ has a $\struct{L}$-point).
\end{enumerate}
\end{theorem}


\section{Arithmetic Local Systems}

Let $X/k$ be a geometrically integral variety over a field $k$. Then there is a sequence,
\begin{center}
\begin{tikzcd}
1 \arrow[r] & \pi_1(X_{\bar{k}}) \arrow[r] & \pi_1(X) \arrow[r] & G_k \arrow[r] & 1
\end{tikzcd}
\end{center}
where $G_k$ is the absolute Galois group of $k$. This means there is a well-defined outer action
\[ G_k \to \mathrm{Out}(\pi_1(X_{\bar{k}})) \]
Hence given a local system $\LL_{\rho}$, meaning a representation up to conjugacy
\[ \rho : \pi_1(X_{\bar{k}}) \to \GL_r(A) \]
there is an action of $\sigma \in G_k$ to give $\LL_{\rho}^\sigma$ via 
\[ \rho^{\tilde{\sigma}}(g) = \rho(\tilde{\sigma} g \tilde{\sigma}^{-1}) \]
for any lift $\tilde{\sigma}$ of $\sigma$ which is well-defined up to conjugacy or equivalently we act on $\rho$ via the outer action which is well-defined at the level of $\LL_{\rho}$. 

\begin{defn}
$\LL_{\rho}$ is \textit{arithmetic} if its isomorphism class is fixed under an open subgroup of $G_k$.
\end{defn}

\begin{example}
If $X$ is an elliptic curve over $k = \FF_p$. Then $\pi_1(\ol{X}) \onto \ZZ_\ell$ for any $\ell \neq p$. Hence there exists a representation,
\[ \bar{\rho} : \pi_1(\ol{X}) \to \GL_1(\FF_{\ell}((t))) \]
with infinite image given by $\Z_\ell \to \FF_{\ell}((t))^\times$ mapping $1 \mapsto 1 + t$. This representation does not lift to $\pi_1(X)$ else it would contradict de Jong's conjecture. Indeed, 
\[ \Hom{}{\pi_1(X_{\bar{k}})}{\Z_\ell} = H^1(X_{\bar{\ell}}, \Z_\ell) = T_{\ell} E(-1) \]
is pure of weight $1$ by Weil II. Hence, there are no Frobenius fixed points.  
\end{example}


Suppose $\LL_{\rho}$ is arithmetic. Then we may pass to $k'/k$ such that $\LL_{\rho}$ is fixed under $G_{k'}$. Replace $k$ by $k'$ and base change $X$ to $k'$. Consider,
\begin{center}
\begin{tikzcd}
1 \arrow[r] & \pi_1(\ol{X}) \arrow[d] \arrow[r] & \pi_1(X) \arrow[ld, dashed, "\tilde{\rho}"] \arrow[r] & G_k \arrow[r] & 1
\\
& \GL_r(A)
\end{tikzcd}
\end{center}
By assumption, for each $\sigma \in G_k$ and lift $\tilde{\sigma} \in \pi_1(X)$ there is $\varphi_{\tilde{\sigma}} \in \GL_r(A)$ such that 
\[ \rho^{\tilde{\sigma}} = {}^{\varphi_{\tilde{\sigma}}} \rho \]
meaning
\[ \rho(\tilde{\sigma} g \tilde{\sigma}^{-1}) = \varphi_{\tilde{\sigma}} \circ \rho(g) \circ \varphi_{\tilde{\sigma}}^{-1} \]
for all $g \in \pi_1(\ol{X})$. In order to define $\tilde{\rho}$ we need to assign $\tilde{\rho}(\tilde{\sigma}) \in \GL_r(A)$ so it is a group homomorphism and
\[ \tilde{\rho}(\tilde{\sigma}) \rho(g) \rho(\tilde{\sigma})^{-1} = \rho(\tilde{\sigma} g \tilde{\sigma}^{-1}) = \varphi_{\tilde{\sigma}} \circ \rho(g) \circ \varphi_{\tilde{\sigma}}^{-1} \]
for all $g$. Therefore, since it commutes with $\rho(g)$ for all $g$,
\[ \tilde{\rho}(\tilde{\sigma}) \varphi_{\tilde{\sigma}}^{-1} \in \Aut{\rho} \]
Write $\tilde{\rho}(\tilde{\sigma}) = h(\tilde{\sigma}) \varphi_{\tilde{\sigma}}$ for $h(\tilde{\sigma}) \in \Aut{\rho}$. Compare,
\[ \tilde{\rho}(\tilde{\sigma} \tilde{\tau}) = h(\tilde{\sigma} \tilde{\tau}) \varphi_{\tilde{\sigma} \tilde{\tau}} \]
with
\[ \tilde{\rho}(\tilde{\sigma}) \tilde{\rho}(\tilde{\tau}) = h(\tilde{\sigma}) \varphi_{\tilde{\sigma}} h(\tilde{\tau}) \varphi_{\tilde{\tau}} \]
Therefore, we require,
\[ [\varphi_{\tilde{\sigma}} h(\tilde{\tau}) \varphi_{\tilde{\sigma}}^{-1}]^{-1} h(\tilde{\sigma})^{-1} h(\tilde{\sigma} \tilde{\tau}) = \varphi_{\tilde{\sigma}} \varphi_{\tilde{\tau}} \varphi_{\tilde{\sigma} \tilde{\tau}}^{-1} \]
which is saying that $\psi(\tilde{\sigma}, \tilde{\tau}) = \varphi_{\tilde{\sigma}} \varphi_{\tau} \varphi_{\tilde{\sigma} \tilde{\tau}}$ is a $2$-boundary. We should check it is a $2$-cycle. Note that 
\[ h \mapsto \varphi_{\tilde{\sigma}} \]

\subsection{Theorems on Arithmetic Local Systems}

\begin{theorem}[Lafforgue]
Let $\rho : \pi_1(X_{\FF_{q'}}, \bar{x}) \to \GL_r(\Qbar_\ell)$ be an arithmetic representation. Then the characteristic polynomial of $\rho(F_x)$ is valued in a \textit{fixed} number field $K$ as $x \in X$ varies over all closed points. 
\end{theorem}

\begin{theorem}[Grothendieck]
Arithmetic tame $\ell$-adic local systems on $X_{\ol{\FF}_p}$ have quasi-unipotent monodomries at infinity.
\end{theorem}

\section{Companions}

We first consider the case of pure motives over a field $K$. Let $\SmP_K$ be the category of smooth projective varities over $K$. Grothendieck produced a ``linearization'' or ``universal Weil cohomology theory''
\[ h : \SmP_K \to \Mot_K \]
to the category of pure motives. This is a map of $\otimes$-categories. The standard conjectures imply that
\begin{enumerate}
\item $\Mot_K$ is semisimple
\item $\Mot_K$ is Tannakian
\item for any Weil cohomology theory $H : \SmP_K \to \Vect_Q$ with $Q$ a field of characteristic $0$, there is a $Q$-linear realization fiber functor
\[ \ol{H} : \Mot_K \ot_{\Q} Q \to \Vect_Q \]
such that $H = \ol{H} \circ h$. 
\end{enumerate}

An important feature of Weil cohomology theiries is that they usually factor through a natural ``enriched'' Tannakian subcategory $\T_H \embed \Vect_Q$ for example
\begin{enumerate}
\item for $K = \CC$ and $H = H_{\text{sing}}$ then $\T_{H_{\text{sing}}}$ is he category of $\Q$-PHS polarizable hodge structures
\item for $\ell$-adic cohomology $H = H_{\ell} : \SmP_K \to \Vect_{\Q_\ell}$ then $\T_{H_{\ell}} = \Rep_{\Q_{\ell}}(\pi_1(K))$ is the category of finite-dimensional continuous $\pi_1(K)$-representations. 
\end{enumerate}

We expect $\ol{H} : \Mot_K \ot_{\Q} Q \to \Vect_{Q}$ induces an equivalence onto its essential image $\T_H^{\ess}$, understanding $\Mot_K$ essentially amounts to understanding the essential image. Indeed, the claim of full faithfulness of $\ol{H} : \Mot_K \ot_{\Q} Q \to \T_H$ is
\begin{enumerate}
\item for $K = \CC$ and $H = H_{\text{sing}}$ the Hodge conjecture
\item for $H = H_{\ell}$ the Tate conjecture for $\ell$.
\end{enumerate}

\begin{rmk}
$\Mot_K$ should be semisimple so we actually expect $\T_{H_\ell}^{\ess} = \Rep_{\Q_\ell}(\pi_1(K))^{ss}$ but it is still open if the cohomology of $X \in \SmP_K$ is semisimple. 
\end{rmk}

Grothendieck: the objects of $\T_H^{\ess}$ are those cut out by algebraic correspondences on the $Q$-vector spaces $H(X)$ for $X \in \SmP_K$ and the morphisms induced by algebraic correspondences. However, this is not useful since we don't understand correspondences.
\bigskip\\
The expected miracle is that $\T_H^{\ess} \subset \T_H$ will be an easy to understand subcategory. 

\subsection{Describing the Simple Objects}

Assuming these conjectures, we want to describe the simple objects of $\T_H^{\ess}$. It is easier to first ask about $\T_H^{\ess} \ot_Q \ol{Q}$. The companion conjecture concerns the case that $K = k(\eta)$ is the function field of a smooth geometrically irreducible variety $S$ over a finite field $k$ of characteristic $p > 0$ with generic point $\eta$.
\bigskip\\
Fix $\ell \neq p$. By construction, the simple objects in $\T_{H_\ell}^{\ess}$ are $\pi_1(K)$-subquotietns $V_\ell$ of $H^i(X_{\ol{K}}, \Qbar_\ell(j))$ for $X \in \SmP_K$. Since $X \to \Spec{K}$ extends to a smooth projective morphisms $f : \X \to U \subset S$ for a dense open then
\[ V_\ell = (\cV_\ell)_{\eta} \]
for a subquotient $\cV_\ell \embed R^i f_* \Qbar_\ell(j)$. In particular, for each closed point $s \in S$ with residue field $\kappa(s)$, 
\[ \det{(1 - T F_s | \cV_{\ell, \bar{s}})} \divides P_{i,j,s}(T) := \det{(1 - T F_s | H^i(X_{\bar{s}}, \Qbar_\ell(j)))} \]
which, according to the Weil conjectures, has coefficients in $\Z$ and is independent of $\ell$, and its roots are $\kappa(s)$-Weil numbers of weight $i - 2j$. The field of definition $Q_s$ denerated by the roots of $P_{i,j,s}$ is degree $\le \deg(P_{i,j,s})!$ over $\Q$, unramified outside $p$ (because the roots are $\kappa(s)$-Weil numbers) and since $\deg(P_{i,j,s}) = \dim H^i(X_{\bar{s}}, \Qbar_\ell(j)))$ is independent of $s$, by Hermite-Minkowski, there are only finitely many possibilities for $Q_s$ as $s \in S$ ranges over all closed points. This implies that $Q_{\cV_\ell}$ generated by the coefficients of $\det{(1 - T F_s | \cV_{\ell, \bar{s}})}$ is a number field. Morally, $Q_{\cV_\ell}$ is the field of definition of the motive $V \in \Mot_k \ot_{\Q} \ol{\Q}$ from which $V_\ell$ arises. 
\bigskip\\
If we fix a different prime $\ell' \neq p$, we can also consider the $\ell'$-adic realization $V_{\ell'} = H_{\ell'}(V)$. Again, the construction of $\Mot_K$ shows that (up to replacing $S$ by a dense open which only depends on $V$ and not on $\ell$) $V_{\ell'}$ also arises as the stack of a $\Qbar_{\ell'}$-local system $\cV_{\ell'}$ on $S$ which is compatible with (or a companion / camarade) of $\cV_\ell$ in the sense that
\[ \det{(1 - T F_s | \cV_{\ell, \bar{s}})} = \det{(1 - T F_s \mid \cV_{\ell', \bar{s}})} \quad s \in S \text{ closed } \]

We have explained that a simple object $V_\ell \in \T_{H_{\ell}}^{\ess} \ot \Qbar_\ell$ with finite determinant (to get rid of tate twists) should arise (after shrinking $S$) as the stalk of a simple $\Qbar_\ell$-local system $\cV_\ell$ with finite determinant on $S$ satisfying
\begin{enumerate}
\item Purity: the roots of $\det{(1 - T F_s \mid \cV_{\ell, \bar{s}})}$ are $\kappa(s)$-Weil numbers of weight $0$
\item Finiteness: $Q_{\cV_{\ell}}$ is a number field
\item Companions: for each $\ell' \neq p$ there is a $\Qbar_{\ell'}$-local system $\cV_{\ell'}$ on $S$ which is cimpatible with $\cV_\ell$. 
\end{enumerate}

The companion conjecture (now a Theorem of Deligne, Drinfeld, and Laforgue) predicts that these properties are true for all simple representations with finite determinant

\begin{thm}[Companions]
Let $\cV_\ell$ be a simple $\Qbar_{\ell}$-local system with finite determinant on $S$. Then $\cV_{\ell}$ satisfies the above properties (Purity), (Finiteness), and (Companions). 
\end{thm}

Thus, if $\Rep_{\Qbar_\ell}(\pi_1(K))^{ss, ur} \subset \Rep_{\Qbar_\ell}(\pi_1(K))^{ss}$ denotes the full subcategory of $\pi_1(K)$-represetnations which are unramified over a dense open subscheme of $S$, the companion conjecture + the Tate conjecture + standard conjectures predicts the $\ell$-adic realization functor
\[ \ol{H}_\ell : \Mot_K \ot_{\Q} \Qbar_\ell \to \Rep_{\Qbar_{\ell}}(\pi_1(K))^{ss, ur} \]
is an equivalence of categories. 


\begin{rmk}
What is an example of a representation that is not unramified over any open set? 
\end{rmk}



\subsection{Compaions Take II: Kedlaya}

The previous discussion did not emphasize enough that companions are expected to exist for \textit{arithmetic} representations which are much rarer than geometric representations (for which maybe we can easily see companions don't exist?)

\begin{conj}
For $\rho : \pi_1(X) \to \GL_r(K)$ an arithmetic representation irreducible with finite order determinant. Then,
\begin{enumerate}
\item $\det{(1 - \rho(F_x) T)} \in \Qbar[T]$ for all closed points $x \in X$
\item there are companions meaning for any $K, K'$ (say $\ol{\Q}_\ell$ and $\ol{\Q}_{\ell'}$) and a choice of (not continuous) isomorphism
\begin{center}
\begin{tikzcd}
\ol{\Q} \arrow[d, hook] \arrow[r, "\sigma"] & \ol{\Q} \arrow[d, hook]
\\
K \arrow[r, "\sigma"] & K'
\end{tikzcd}
\end{center}
then there is a $\rho^\sigma : \pi_1(X) \to \GL_r(K')$ such that
\[ \det{(1 - \rho^\sigma(F_x) T)} = \sigma(\det{(1 - \rho(F_x) T)} \]
for all closed points $x \in X$.
\end{enumerate}
\end{conj}

Note the second has content even for $K = K'$, it says we may form Galois conjugates of representations.

\subsection{Companions Take III: Esnault}

Given a field automorphism $\tau : \C \to \C$ we can postcompose the underlying monodromy representation of a local system $\LL_{\CC}$ to get,
\[ \rho^\tau : \pi_1(X(\CC)) \xrightarrow{\rho} \GL_r(\CC) \xrightarrow{\tau} \GL_r(\CC) \]
defining a new local system $\LL_{\CC}^\tau$. This works because there is no topology we need to respect. Because of the topologies we cannot play the same game with an isomorphism (or arbitrary automorphism) $\sigma : \Qbar_\ell \iso \Qbar_{\ell'}$.
\bigskip\\
Since $\pi_1(X(\CC))$ is finitely generated, taking the characteristic polynomials of the generators gives a map
\[ M_B(X, r)_{\CC}^\square \to N_{\C} := \prod_{i = 1}^s (\A^{r-1} \times \Gm)_{\CC} \]
via
\[ \rho \mapsto (\det{(T - \rho(\gamma_1))}, \dots, \det{(T - \rho(\gamma_s))}) \]
where $\gamma_1, \dots, \gamma_s \in \pi_1(X(\CC))$ is a large enough generating set (depening on $r$ and $\pi$). Since the characteristic polynomial is invariant under conjugation, this decends to a map
\[ \psi : M_B(X,r)_{\CC} \to N_{\CC} \]
which is a closed embedding since semisimple representations are determined by their characteristic polynomials and by a finite-generation argument we only need finitely many $\gamma_1, \dots, \gamma_s$. 
\bigskip\\
Let $\det{(T - \rho(\gamma))}^\tau := \tau(\det{(T - \rho(\gamma))}$. Then we get a diagram,
\begin{center}
\begin{tikzcd}
M^{\irr}_B(X,r)(\CC) \arrow[d, "\LL_{\CC} \mapsto \LL_{\CC}^{\tau}"] \arrow[r,hook] & M_B(X, r)(\CC) \arrow[r, "\psi"] \arrow[d, "\LL_{\CC} \mapsto \LL_{\CC}^{\tau}", hook] & N(\CC) \arrow[d,"(-)^{\tau}"]
\\
M_B^{\irr}(X,r)(\CC) \arrow[r, hook] & M_B(X,r)(\CC) \arrow[r, "\psi", hook] & N(\CC)
\end{tikzcd}
\end{center}
so we can recognize the operation $\LL_{\CC} \mapsto \LL_{\CC}^\tau$ by the more explicit operator $(-)^{\tau}$ on $N(\CC)$ via $\psi$. 
\bigskip\\
Now we want to do the same for $X$ smooth quasi-projective over a finite field $\FF_q$ of characteristic $p$. Let $\ell \neq p$ be a prime. Let $M^{\irr}_\ell(X_{\ol{\FF}_p}, r)$ be the set of $\ell$-adic Weil local systems $\LL_\ell$ of rank $r$ on $X_{\ol{\FF}_p}$ which are
\begin{enumerate}
\item arithmetic: menaing they descend to some $X_{\FF_{q'}}$ 
\item are irreducible over $\ol{\Q}_{\ell}$ as $\pi_1(X_{\FF_{q'}})$-reps.
\end{enumerate}
Then given a field isomorphism $\sigma : \Qbar_\ell \to \Qbar_{\ell'}$ the only ``continuous'' information it contains is that it induces an isomorphism $\sigma : \Qbar \iso \Qbar$ which may be nontrivial, i.e. it takes a number field $K \subset \Qbar_\ell$ to $K^\sigma \subset \Qbar_{\ell'}$. Therefore, we want to define $(-)^{\sigma}$ as in the above diagram but we can only do this is a continuous way if $\rho(\gamma)$ has its characteristic polynomial valued in a number field. 
\bigskip\\
To this aim, we desire a set of loops $\gamma \in \pi_1(X_{\FF_{q'}}, \bar{x})$ such that
\begin{enumerate}
\item $\det{(T - \rho(\gamma))} \in K[T]$ for some number field $K$
\item the $\gamma$ generate $\gamma \in \pi_1(X_{\FF_{q'}}, \bar{x})$ in the strong sense that the characteristic polynomials determine $\rho$ up to conjugation. 
\end{enumerate}

The set of frobenii $\{ F_x \mid x \in X \text{ closed } \}$ satisfy these two properties.

\begin{theorem}[Lafforgue]
Let $\rho : \pi_1(X_{\FF_{q'}}, \bar{x}) \to \GL_r(\Qbar_\ell)$ be an arithmetic representation. Then the characteristic polynomial of $\rho(F_x)$ is valued in a \textit{fixed} number field $K$ as $x \in X$ varies over all closed points. 
\end{theorem}

\begin{rmk}
If $\rho$ arises from geometry, this follows from the Weil conjectures and Hermite-Minkowski as above. 
\end{rmk}

Therefore, taking
\[ N^\infty = \prod_{x \in |X|} (\A^{r-1} \times \Gm) \]
and $\psi^\infty$ the characteristic polynomial map on those Frobenii at all closed points. Then Deligne's companion conjecture is the existence of a dahsed arrow making the diagram commute,
\begin{center}
\begin{tikzcd}
M^{\irr}_\ell(X, r) \arrow[d, dashed, "\LL_{\ell} \mapsto \LL_{\ell}^{\sigma}"] \arrow[r, "\psi^\infty"] & N^\infty(\Qbar) \arrow[d, "(-)^{\sigma}"] \arrow[r,hook] & N^\infty(\Qbar_\ell) \arrow[d]
\\
M^{\irr}_{\ell'}(X, r) \arrow[r, "\psi^\infty"] & N^\infty(\Qbar) \arrow[r, hook] & N^{\infty}(\Qbar_{\ell'}) 
\end{tikzcd}
\end{center}
then $\LL_\ell^{\sigma}$ is the \textit{companion} of $\LL_{\ell}$ for $\sigma$.


\subsection{Use of Mazur's Deformation Space}

Let $(S, X_S \embed \ol{X}_S, x_S)$ be a good relative compactification. We assume that
\[ \epsilon : M_B(X, r, \L, T_i) \to \Spec{\struct{K}} \]
is dominant. By generic smoothness, there is a nonempty open
\[ M^\circ \subset M_B(X, r, \L, T_i)_{\red} \]
such that $\epsilon|_{M^\circ}$ is smooth and dominant and has values in
\[ \Spec{\struct{K}}^\circ \subset \Spec{\struct{K}} \]
which is an open chosen to be \etale over $\Spec{\Z}$. The reason we make this further assumption is so that we can conclude that Mazur's deformation space is smooth over $W(\FF_{\ell^m})$ not just the completion of $\struct{K}$. Let $z \in |M^\circ|$ be a closed points, of residue field $\FF_{\ell^m}$ for some $\ell \ge 3$ and $m \in \N_{>0}$. Since the stack is a $\Gm$-gerbe over $M_B$ and the Brauer group of $\FF_{\ell^m}$ is trivial, the point lifts uniquely (up to isomorphism) to a point of the stack so there is a aboslutely irreducible,
\[ \rho_x : \pi \to \GL_r(\FF_{\ell^m}) \]
corresponding to a local system $\LL_z$ over $X$. 
\par 
Define $S^\circ \subset S$ to be the non-empty open which is the complement of the closed points of residual characteristic diving the order of $\GL_r(\FF_{\ell^m})$. We will show $S^\circ$ satisfies the main Theorem. Choose $s \in |S^\circ|$ of characteristic $p$ and consider the diagram
\[ \sp^{\top}_{\CC,\bar{s}} : \pi_1(X(\CC), x(\CC)) \to \pi_1(X_{\CC}, x_{\CC}) \to \pi_1^t(X_{\bar{s}}, x_{\bar{s}}) \]
since the last map is an isomorphism on prime to $p$ quotients, we see that $\LL_{x}$ gives rise to an absoltely irreducible $\FF_{\ell^m}$-system $\LL_{x,\bar{s}}$ over $X_{\bar{s}}$ with the correct determinant and behavior at infinity.
\par 
Let $D_{x,\bar{s}} = \Spf{R_{z, \bar{s}}}$ be Mazur's deformation space of rank $r$ representations of $\pi_1^r(X_{\bar{s}}, x_{\bar{s}})$ with residual representation $\LL_{z, \bar{s}}$. Inside there is a formal closed subscheme
\[ D_{z, \bar{s}}(r, \L, T_i) \subset D_{z, \bar{s}} \]
corresponding to deformations preserving the determinant and monodromies at infinity. Thus we get a morphism
\[ D_{z, \bar{s}}(r, \L, T_i) \to \M(X, r, \L, T_i) \to M_B(X, r, \L, T_i) \]
via composing with $\sp_{\CC, \bar{s}}^{\top}$.

\begin{prop}
The morphism of formal schemes,
\[ \iota : D_{z, \bar{s}}(r, \L, T_i) \to M_B(X, r, \L, T_i)_z^{\wedge} \]
is an isomorphism.
\end{prop}

\begin{proof}
Let $R$ be an artin local ring equipped with a map $R \to \FF_{\ell^m}$ inducing an isomorphism on the residue field. To construct the inverse to $\iota$ we will show that morphisms
\[ m : \Spec{R} \to M_B(X, r, \L, T_i) \]
deforming the point $z$ are in one-to-one correspondence with deformations of $\rho_{z,\bar{s}}$ in $D_{z, \bar{s}}(r, \L, T_i)$. 
\par 
First, the Brauer group of $R$ is trivial (the category of \etale $R$-algebras is just the category of \etale $\FF_{\ell^m}$-algebras) so any $m$ lifts (nonuniquely up to a choice of $R^\times$) to
\[ m : \Spec{R} \to \M(X, r, \L, T_i) \]
and the isomorphism class of the lift is well-defined. This gives an $R$-local system $\LL_m$ on $X$ such that $\LL_{R} \ot_R \FF_{\ell^m} = \LL_z$ canonically. 
\par 
Second, by exactly the same argument as above, this descends to a local $R$-system $\LL_{m, \bar{s}}$ on $X_{\bar{s}}$ (using that $R$ is finite so it factors through the profinite completion) unique up to isomorphism with determinant $\L$ and monofromies in $T_i$ at infinity. The corresponding continuous representation $\rho_{R, \bar{s}} : \pi_1^t(X_{\bar{s}}, x_{\bar{s}}) \to \GL_r(R)$ is the desired deformation. 
\end{proof}

\begin{cor}
$D_{x, \bar{s}}(r, \L, T_i)_{\red} \to \Spf{W(\FF_{\ell^m})}$ is formally smooth.
\end{cor}

\begin{proof}
Indeed, we choose
\[ M^\circ \to \Spec{\struct{K}}^\circ \to \Spec{\ZZ} \]
so that both maps are smooth. 
\end{proof}

\subsection{Using Compaions}

Recall that $D_{x, \bar{s}}(r, \L, T_i)$ is a deformation space for residual representations of $\pi_1^t(X_{\bar{s}}, x_{\bar{s}})$. There is an exact sequence
\[ 1 \to \pi_1^t(X_{\bar{s}}, x_{\bar{s}}) \to \pi_1^t(X_s, x_s) \to G_{\kappa(s)} \to 1 \]
of profinite groups. Denote by $\Phi \in G_{\kappa(s)}$ the Frobenius and $\Phi$ acts on the isomorphism classes of $\pi_1^t(X_{\bar{s}}, x_{\bar{s}})$-representations. Since $\rho_{z, \bar{s}}$ has finite image, a power $\Phi^n$ for some $n \ge 1$ stabilizes $\rho_{z, \bar{s}}$ (because $\pi_1^t(X_{\bar{s}})$ is topologically finitely generated and the target is finite so the set of all such representations is finite) hence $\Phi^n \acts D_{z,\bar{s}}(r)$. Furthermore, we can increase the power such that $\Phi^n$ fixes the data $\L, T_i$ and thus acts on $D_{z, \bar{s}}(r, \L, T_i) \embed D_{z,\bar{s}}(r)$. 
\par 
De Jong's conjecture (Theorem of Gaitsgory) implies that there is a $\ol{\Z}_\ell$-point of $D_{z,\bar{s}}(r,\L, T_i)_\red$ invariant under $\Phi^n$. This corresponds to an irreducible $\ell$-adic local system $\LL_{z, \bar{s}, \ell}$ on $X_{\bar{s}}$ which descends to a Weil sheaf hence to an arithmetic \etale local system with proper determinant and monodrmy. This gives the result for $\ell$. 
\par 
To get the representations for $\ell' \neq \ell$ we use companions $\LL_{z, \bar{s}, \ell}^{\sigma}$ which is the restriction to $X_{\bar{s}}$ of the companion of the arithmetic descent (we need arithmetic for the existence of companions) of $\LL_{z, \bar{s}, \ell}$. It is an $\ell'$-adic local system on $X_{\bar{s}}$ irreducible over $\Qbar_{\ell'}$ with determinant $\L$ and monodromies at infinity $T_{z, \bar{s}, \ell',i}$ having the same semisimplification as $T_i$. We set $\LL_{\ell, \bar{s}} = \LL_{z,\bar{s}, \ell}$ and $\LL_{\ell', \bar{s}} = \LL_{z,\bar{s}, \ell}^\sigma$ ofr all $\ell' \neq p,\ell$. For $\ell = p$ we just choose another closed point $s' \in S^0$ with residue characteristic $p' \neq p$ and redo the same construction and set $\LL_{p, \bar{s'}} = \LL_{z, p, \bar{s}'}$. 

\begin{theorem}
Assume $\epsilon : M_B(X,r,\L, T_i) \to \Spec{\struct{K}}$ is dominant. Then ther any prime $\ell$, there is an $\ell$-adic local system $\LL_{\ell}$ such that
\begin{enumerate}
\item has rank $r$ and is irreducible over $\Qbar_\ell$
\item has determinant $\L$
\item has monodromies $T_{i,\ell}$ at infinity such that $T_i^{ss} = T_{i,\ell}^{ss}$.
\end{enumerate}
\end{theorem}  


\begin{proof}
Just take $\LL_{\ell} := (\sp_{\CC, \bar{s}}^{\top})^* \LL_{\ell, \bar{s}}$ for $\ell \neq p$ and for $\ell = p$ take $\LL_p := (\sp^{\top}_{\CC, \bar{s}'})^* \LL_{p, \bar{s}'}$. 
\end{proof}

Note that the above local systems have more structure than just being topological $\ell$-adic local systems. Indeed, since they factor through specialization and the $\LL_{\ell, \bar{s}}$ are continuous representations the arise as,
\[ \pi_1(X(\CC) \to \pi_1(X_{\CC}) \to \GL_r(\Qbar_\ell) \]
where the second map is continuous and hence by compactness this is integral menaing conjugate to
\[ \pi_1(X(\CC)) \to \GL_r(\ol{\Z}_\ell) \]
giving integrality. 
\par 
Therefore, to prove the main group-theoretic theorem:

\begin{proof}
If $\pi$ is a geometric finitely presented group then $\pi$ is weakly integral.
\end{proof}

We just need to show that if 
\[ M_B(X, r, \L) \to \Spec{\Z} \]
is dominant then for some choice of monodrommies $T_i$ we have
\[ M_B(X, r, \L, T_i) \to \Spec{\struct{K}} \]
is dominant. Indeed, this will also follow from de Jong's conjecture since weakly arithmetric local systems are dense and the weakly arithmetic local systems all have quasi-unipotent monodromy at infinity. 

\section{Totaro III}

\begin{defn}
Let $X$ be a smooth $k$-scheme with an action of an affine group scheme $G/k$ define
\[ \CH^i_G(X) := \CH^i([X \times (V \sm S)]/G) \]
where $V$ is a rep of $G$ and $S \subset V$ is a closed $G$-invariant subscheme such that $\codim{S, V} > i$.
\end{defn}

If $G$ acts freely on $X$, then $\CH^i_G(X) = \CH^i(X/G)$. We always have $\CH^i_G(X)$ is a module over $\CH^i_G(*) = \CH^i(BG)$. 

\section{Helene's Lecture III} 

Sketch of the proof of the geometric theorem.
\par 
Let $X$ be a smooth connected quasi-projective variety over $\CC$. Fix a good compactification $X \embed \ol{X}$ meaning the boundary are normal crossing varities.

\begin{theorem}
Fix $\delta$ the order of the determinant and roots of unity at infinity $\lambda_{ij}$. Consider
\[ \epsilon : M \subset M_B(X, r, \delta, \lambda_{ij}) \to \Spec{\struct{K}} \]
where $M$ is the complement of the closure of the reducible complex local systems. If $\epsilon$ is dominant then it is surjective. 
\end{theorem}

\begin{rmk}
$\epsilon$ is dominant exactly if there exists an irreducible $\CC$-local system on $X$ with prescribed determinantal order $\delta$ and monodromy eigenvalues $\lambda_{ij}$ at $\infty$.
\end{rmk}

\begin{rmk}
In fact, we prove more. For any $\ol{\Z}_\ell$-point
\[ \struct{K} \embed \ol{\Z}_\ell \]
there is a compatible $\ol{\Z}_{\ell}$-point of $M$.
\end{rmk}

\begin{proof}
$\epsilon : M_\red \to \Spec{\struct{K}}$ is generically smooth on $M$. So there exsits a closed point $z \in M$ such that $\epsilon_{\red}$ is smooth at $z$ which has characteristic $\ell$. Thus
\[ M_{z,\red}^{\wedge} \to \Spf{\struct{K,v}} \]
where $v$ is a place of characteristic $\ell$. Then $z$ corresponds (using triviality of the Brauer group) to an absolutely irreducible representation
\[ \rho : \pi \to \GL_r(\FF_{\ell^m}) \]
If we choose $p$ large enough then it cannot divide the order of $\GL_r(\FF_{\ell^m})$. Now we spread out $X_{\CC}$ to $X_S \to S$ smooth with $S$ finite type over $\ZZ$. Shrinking $S$ we get a relative good compactification $X_S \embed \ol{X}_S$. Then by specialization, there is a factorization,
\begin{center}
\begin{tikzcd}
\pi_1 \arrow[r] \arrow[d] & \pi_1^t(X_{\CC}) \arrow[dl, dashed] \arrow[r] & \pi_1^t(X_{\bar{s}}) \arrow[dll, dashed, "\bar{\rho}_{\ol{\FF}_p}"]
\\
\GL_r(\FF_{\ell^m}) 
\end{tikzcd}
\end{center} 
as long as $\bar{s}$ has characteristic $p$ large enough. This is because the specialization map is an isomorphism on prime to $p$ part. This is the hidden $p$-direction of $M$. Since the downward left rep is absolutely irreducible so is $\bar{\rho}_{\ol{\FF}_p}$. 
\par 
Now we get an isomorphism
\[ M_{z}^{\wedge} \cong \Def(\bar{\rho}_{\ol{\FF}_p}, \delta, \lambda_{ij}) \]
to Mazur's deformation space. Therefore, Mazur's deformation space has smooth reduction over $\Spf{W(\FF_{\ell^m})}$. 
\bigskip\\
Now we consider
\begin{center}
\begin{tikzcd}
1 \arrow[r] & \pi_1^t(X_{\ol{\FF}_p}) \arrow[d, "\bar{\rho}_{\ol{\FF}_p}"] \arrow[r] & \pi_1^t(X_{\FF_q}) \arrow[r] & G_{\FF_q} \arrow[r] & 1
\\
& \GL_r(\FF_{\ell^m})
\end{tikzcd}
\end{center}
and since $\pi_1^t(X_{\ol{\FF}_p})$ is topologically finitely generated, there are only finitely many such $\bar{\rho}$. Thus the action of Frobenius to some power must stabilize $\bar{\rho}$. Therefore, after some extension $\FF_{q'}$ we get a representation
\[ \bar{\rho}_{\FF_{q'}} : \pi_1^t(X_{\FF_{q'}}) \to \GL_r(\FF_{\ell^n}) \]
Now we use the smoothness of the reduction of $\Def$ meaning that de Jong applies so that the Frob-invariant points are nonempty. Therefore, there exists $\LL_\ell$ an $\ell$-adic local system on $X_{\FF_{q'}}$ which is geometrically absolutely irreducible and has the correct data $\delta, \lambda_{ij}$. 
\par 
Now we will apply companions to $\LL_{\ell}$. For $\ell' \neq \ell,p$ and any $\sigma : \Qbar_{\ell} \iso \Qbar_{\ell'}$ there is $\LL_{\ell'} := \LL^{\sigma}_{\ell}$ an $\ell'$-adic local system and these $\LL^{\sigma}_{\ell}$ are geometrically absolutely irreducible and have the correct data $\delta$ and $\lambda_{ij}$.
\par 
We started with the topological fundamental group $\pi = \pi_1(X(\CC))$ then we specialize,
\begin{center}
\begin{tikzcd}
\pi_1(X(\CC)) \arrow[rrrd] \arrow[r] & \pi_1(X_{\CC}) \arrow[r,"\sp"] & \pi_1^t(X_{\ol{\FF}_q}) \arrow[r, hook] & \pi_1^t(X_{\FF_{q'}}) \arrow[d, "\LL_{\ell'}"]
\\
& & & \GL_{r}(\Qbar_{\ell'})
\end{tikzcd}
\end{center}
the diagonal map gives the local system we want. Thus we get all $\ell' \neq p$. Okay, just do it again for a bigger $p' \neq p$. Then we win.
\par 
To review: if $\epsilon$ is dominant then we get for any $\sigma : \CC \iso \Qbar_{\ell}$ we get an $\ell$-adic local system $\LL_{\ell}$. Remember $\ell$-adic here means it factors through a continuous representation on $\hat{\pi}$ and thus is integral (has values in $\ol{\Z}_\ell$).  
\end{proof}

\section{Burt IV}

\begin{theorem}[T, 2019]
For an affine group scheme $G$ over a field $k$ with a faithful representation $V$ of dim $n$ then,
\[ \CH^\bullet(BG) / (c_1(V), \dots, c_n(V)) \cong H^\bullet(\GL_n / G) \]
and hence is concentrated in finitely many degrees. 
\end{theorem}

This follows from the fibration sequence
\[ \GL_n / G \to B G \to B \GL_n \]
and noting that $\GL_n$-bundles are Zariski locally trivial and hence give nice relations between Chow groups. 

\begin{rmk}
Therefore, $\CH^\bullet(BG)$ is a finitely generated $\Z$-algebra iff $\CH^\bullet(\GL_n / G)$ is. Furthermore, we see $\CH^\bullet(BG)$ is a finitely generated $\Z$-algebra iff $\CH^i(BG)$ is a finitely generated $\Z$-module for all $i$.
\end{rmk}

\begin{example}
Let $k$ be a field of characteristic not $2$ then,
\[ \CH^\bullet(BSO(2n)_k) \cong \Z[c_2, c_3, \dots, c_{2n}, y_m]/(2 c_{\text{odd}}, y_m c_{\text{odd}}, y_m - (-v^m 2^{2m-2} c_{2m})) \]
\end{example}

Question: for $k = \bar{k}$ is $\CH^\bullet(BG)$ always finitely generated? 

\newcommand{\ur}{\mathrm{ur}}

\begin{theorem}[T, 2016]
Let $X$ be a smooth proper variety over $k$ and $R$ a nonzero commutative ring. The following are equivalent,
\begin{enumerate}
\item for any field $F / k$ the homomorphism 
\[ \CH_0(X) \to \CH_0(X_F) \]
is surjective
\item for any field $F / k$ the degree map
\[ \deg : \CH_0(X_F) \to \Z \]
is an isomorphism
\item For every $R$-linear rust cycle module $M$ over $k$ the homomorphism $M(k) \to M(k(X))_{\ur}$ is an isomorphism i.e. $X$ has trivial unramified cohomology in the most general sense
\item there is a nonempty Zariski open $U \subset X$ such that $\CH_i(U_F)_R = 0$ for all $F/k$ and all $i < \dim{X}$
\item there is a nonempty Zariski open $U \subset X$ such that $\CH_i(U)_R \to \CH_i(U_F)_R$ is surjective for all $F/k$ and $i < \dim{X}$. 
\end{enumerate}
\end{theorem}

\begin{example}
$X = \P^1 \times E$ for $E$ an elliptic curve. Then $\CH_0(X) \to \CH_0(E)$ is an isomorphism. However, $U = \A^1 \times E$ has $\CH_0(U_F) = 0$ for all $F / k$ but $\CH_1(U_F) = \CH_0(E_F)$ since it is an affine bundle over $E_F$ so we see the equivalence of (a) and (d) in the above theorem in this case.
\end{example}

\begin{rmk}
These equivalent conditions are what we call being ``universally $\CH_0$-trivial''. 
\end{rmk}

\begin{proof}
Proof that $(a) \implies (b)$. We assume that $\CH_0(X) \to \CH_0(X_F)$ is an isomorphism for all $F / k$. We actually just need this for $F = k(X)$. Then we know that the class of the diagonal $\delta \in CH_0(X_{k(X)})$ arises from $z \in \CH_0(X)$. Therefore, applying $\delta - z$ which acts by zero on $\CH_0(X_F)$ we get any $\gamma \in \CH_0(X_F)$ is $\gamma \sim z \deg{\gamma}$ so we win. 
\bigskip\\
Alternative,
\[ \CH_0(X_{k(X)}) = \dlim_{U \subset X} \CH_n(X \times U) \]
so we can write
\[ [\Delta] = z \times X + B \]
where $B$ is supported on some $X \times S$ for proper closed $S \subset X$. Intersection with $[\Delta_X]$ induces the identity on $\CH_i(X_F)$ for all $F / k$. However, for $\gamma \in \CH_0(X_F)$ we have $(z \times X)_* \gamma = z \cdot \deg{\gamma}$ and $B_* \gamma = 0$ because we can move $\gamma$ outside of $S$.  
\end{proof}

For $p$-groups of order $\le p^4$ and $2$-groups of order $\le 2^5$ then 
\[ \CH^i(BG_k) \]
are independent of $k$ for all fields containing all $\# G$-roots of unity. 

\begin{theorem}[T]
There are $p$-groups of order $p^5$ or $2$-groups of order $2^6$ such that for some $i > 0$
\[ \CH^i(BG_k) \]
and some field $k$ of characteristic zero there is an extension $F / k$ where the chow group gets bigger. In fact, $\CH^i(BG_F)$ can be arbitrarily large cardinality (for huge weird fields) so can be not finitely generated. 
\end{theorem}

\begin{proof}
Use a theorem of Bogomolov et al that there are groups $G$ as above such that for any faithful rep $V$ of $G$ over $\CC$ then a smooth compactification of $U/G$ has nontrivial Brauer group (hence $U/G$ is not rational). This is an unramified cohomology group so the previous result tells us that for all opens (in particular $U/G$) there is some $i$ such that
\[ \CH^i(U/G) \to \CH^i((U/G)_F) \]
is not surjective for some extension $F / k$. 
\par 
Note, if $i$ we small compared to the codimension of the nonfree locus $S \subset U$ then this would be the same as $\CH^i(BG)$. However, consider the localization sequence of equivariant Chow groups for $S \subset U$,
\[ \CH_G^i(V) \onto \CH_G^i(V \sm S) \]
But $V/G$ is a vector bundle over $BG$ so we see that $\CH_G^i(V) = \CH^i(BG)$. Therefore, we get a diagram,
\begin{center}
\begin{tikzcd}
\CH^i_G(V) \arrow[r, two heads] \arrow[d, equals] & \CH^i_G(V \sm S) \arrow[d, equals]
\\
\CH^i(BG) \arrow[r, two heads] \arrow[d, "\alpha"] & \CH^i(U/G) \arrow[d, "\beta"]
\\
\CH^i(BG_F) \arrow[r, two heads] & \CH^i((U/G)_F) 
\end{tikzcd}
\end{center}
if we assume that $\alpha$ is surjective this would show $\beta$ is surjective which we know is not the case. The bottom map is surjective by the same argument as the middle horizontal map.
\end{proof}

\section{Wickelgren}

Proof of rationality in the Weil conjectures. 
\bigskip\\
Consider a symmetric monoidal category $(\C, \ot, 1, \tau)$ a dual of $X \in \C$ is an object $\D X$ with maps
\[ \epsilon : X \ot \D X \to 1 \quad \quad \eta : 1 \to \D X \ot X \]
such that
\[ X \xrightarrow{\id \ot \eta} X \ot \D X \ot X \xrightarrow{\epsilon \ot \id} X \]
and 
\[ \D X \xrightarrow{\eta \to \id} \D X \ot X \ot \D X \xrightarrow{\id \ot \epsilon} \D X \]
are the identity. 

\begin{example}
$\Hom{\C}{X \ot Y}{Z} = \Hom{\C}{Y}{\D X \ot Z}$
\end{example}

For $X \in \C$ dualizable with dual $\D X$ then any $F : X \to X$ has a trace $\Tr{f} \in \End{1}$ via
\[ 1 \xrightarrow{\eta} \D X \ot X \xrightarrow{\id \ot f} \D X \ot X \xrightarrow{\tau} X \ot \D X \xrightarrow{\epsilon} 1 \]

\begin{example}
For $X$ a finite-dimensional vector space over $k$ a field then
\[ k \to \D X \ot X \to \D X \ot X \to X \ot \D X \to k \]
is given by
\[ 1 \mapsto \sum_i e_i^* \ot e_i \mapsto \sum_i e_i^* f(e_i) \mapsto \sum_i e_i^*(f(e_i)) = \sum_i f_{ii} = \tr{f} \]
where $f_{ij} = e_j^*(f(e_i))$. 
\end{example}

\begin{example}
Let $R$ be a commutative ring and $\C = D^{\text{perf}}(X)$ is the bounded derived category of perfect complexes. Then given $F : C_\bullet \to C_\bullet$ we get that $\D C_\bullet = \underline{\mathrm{Hom}}(C_\bullet, R)$. Then,
\[ \tr{F} = \sum_i (-1)^i \tr{F|_{H_i(C_\bullet)}} \] 
\end{example}


\begin{example}
Let $\mathrm{Spaces}_\bullet$ be the homotopy theory of pointed spaces and $X \wedge Y$ is the tensor product. Let $V \to X$ be a $\RR$-vector bundle. Then the Thom space is $\Th_X(V) = \P(V \oplus \struct{X}) / \P(V)$ is the same as putting a metric and taking the disk bundle modulo the sphere bundle or taking $V / (V \sm \text{zero})$. Then
\[ \Th_X(V \oplus \struct{X}) \cong S^1 \wedge \Th_X(V) \]
and therefore in the category of spectra
\[ \mathrm{Sp} = \mathrm{Spaces}_\bullet[(\wedge S^1)^{-1}] \]
is a twisted monodal category where $1 = \S$ is the sphere spectrum. Then we can define,
\[ \Th_X(-V) := \Omega^n \Th_X(W) \]
where $W \oplus V \cong \struct{X}^n$ because $\struct{X}^n - V = W$ in the grothendieck group so by the above formula we should have
\[ \Sigma^n \Th_X(-V) = \Th_X(W) \]
This gives a map $\Th : K_0(X) \to \Sp$. Then manifolds in $\Sp$ are invertible. We say that Purity is the statement: if $Z \embed X$ is a closed embedding of closed manifolds then,
\[ X / (X \sm Z) \cong \Th_Z(N_Z X) \]
using tubular neighborhoods. Write $\Th_X(V) = X^V$. 
\end{example}

\begin{theorem}[Atyiah Duality]
For $X$ a smooth compact manifold $\D X_{+} = X^{-TX}$ in $\Sp$. More generally $\D X^V = X^{-V - TX}$.
\end{theorem}

\begin{proof}
construct
\[ \epsilon : X^V \ot X^{-V - TX} = (X \times X)^{\pi_1^* V + \pi_2^* (-V - T X)} \to (X \times X)^{W} / (X \times X - \Delta_X)^W \cong X^{T_X + W |_{\Delta}} \cong X^{0} = X_+ \to S_0 \to \S \]
where $W = \pi_1^* V \pi_2^* (-V - T X)$ where the map $X_{+} \to S_0$ sends all of $X$ to the non-base point of $S_0$. The coevaluation (for $V = 0$) is the Thom collaps map followed by the Thom diagonal
\[ \S \to X^{-T X} \to X^{-T X} \ot X_{+} \]
where the Thom diagonal is
\[ X^V \xrightarrow{(\id, \pr)} X^V \wedge X_+ \]
For the Thom collapse map, embed $X \embed S^N$ into a large sphere then there is a map
\[ S^N \to S^N / (S^N \sm X) \cong S^N \wedge X^{-TX} \]
using purity and the fact that $N_{X} S^N \ot TX \cong T S^N$ which is trivial on $X$. Then we need to show compatibility. 
\end{proof}

Then for $f : X \to X$ the trace is ``given by the Lefschetz trace formula'' $\tr{f} \in \End{\S} = \Z$ in the sense that if $f$ has regular fixed points
\[ \tr{f} = \sum_{x \in X^f} \mathrm{ind}_x(f) \]

\begin{proof}
$\tr{f}$ is the composition
\[ S^N \to S^N / (S^N \sm X) \to S^N \times X/(S^N \sm X) \times X) \xrightarrow{\id \ot f}   S^N \times X/(S^N \sm X) \times X) \to S^N \times X / (S^N \times X \sm \Delta_X) \to S^N \wedge X_+ \to S^N \wedge S^0 = S^N \]
this sends any non-fixed point to the base point and this factors through $S^N / (S^N \sm X^F) = \bigvee_{X^F} S^N$ so we get a sum. 
\end{proof}

\begin{example}
$\A^1$-homotopy theory. Let $B$ be qcqs and consider the Nisnevich topology
\[ \Spc(B) \subset \mathrm{Fun}(\mathrm{Sm}_B^\op, \mathrm{Spaces}) \]
then we need to Nisnevich and $\A^1$-localize. Then we have a Thom space
\[ \Th_X(\Theta_X) \cong \frac{\A^1 \times X}{\Gm \times X} \cong \frac{\P^1 \times X}{\A^1 \times X} \cong \frac{\P^1 \wedge X}{\infty \wedge X} \cong \P^1 \wedge X_+ \]
Then there are $\P^1$-spectra
\[ SH(B) = \Spc(B)[(\wedge \P^1){-1}] \]
\end{example}

\begin{theorem}
If $k$ is a field and $X$ smooth proper over $k$ and $V \to X$ a vector bundle,
\[ D X^V = X^{-V - TX} \]
\end{theorem}

\begin{proof}
For $\epsilon$ the same construction works. For $\eta$, the Thom diagonal is the same. For the Thom collapse map: $X^{-TX}$ is contravariant in $X$ because for
\[ Z \embed X \]
we get
\[ X^{-TX} \to X^{-TX} / (X \sm Z)^{-TX} \cong Z^{TX - TZ - TX} = Z^{-TZ} \]
so it suffices to get the collapse map for $\P^n$ which we construct by hand. 
\end{proof}

\newcommand{\ind}{\mathrm{ind}}

\begin{theorem}[Hoyois]
Let $X$ be smooth proper over $B$ and $f : X \to X$ then
\[ \tr{f} = \sum_{x \in X^f} \ind_x f \]
which everything here properly defined. 
\end{theorem}


\section{Problems}

\subsection{1}

Let $M$ be a finitely presented $R$-module. Consider the functor:
\[ F : S \mapsto S \ot_R M \]
from $R$-algebras to groups. Then the claim is that if $F$ is representable by a group scheme $V \to X = \Spec{R}$ then $M$ is locally free and $V \to X = \Spec{R}$ is a vector bundle.
\bigskip\\
First, we show that $V \to \Spec{\Z}$ is smooth. Consider a square zero extension of Artin rings $A' \onto A$ and a diagram,
\begin{center}
\begin{tikzcd}
\Spec{A} \arrow[d, hook] \arrow[r] & V \arrow[d]
\\
\Spec{A'} \arrow[ru, dashed] \arrow[r] & X
\end{tikzcd}
\end{center}
this is the same as a map $\varphi : R \to A'$ and $m \in M \ot_R A$ we need to find $m' \in M \ot_R A'$ such that $m'|_A = m$. Indeed, this holds because $M \ot_R -$ preserves surjections (this is the step where affine is essential, in FGA eplained it is shown this is always representable for $X$ proper). Hence $V \to X$ is smooth. Now consider the tangent bundle $\Omega_{V/X}$ which must be locally free. Strictly, first we must show that $V \to X$ is locally finitely presented to do this we show $F$ satisfies
\[ F(\colim A_i) = \colim F(A_i) \]
but this is just that $M \ot_R -$ commutes with colimits since it is a left adjoint. Therefore $\Omega_{V/X}$ is a finite rank vector bundle on $X$. Furthermore, because $\pi : V \to X$ is a group scheme it must be the case that $\Omega_{V/X} = \pi^* e^* \Omega_{V/X}$ and $e^* \Omega_{V/X}$ is a vector bundle on $X$. We instead consider $\E = e^* (\Omega_{V/X})^\vee$ the tangent bundle on $X$. The claim is that $\E \cong \wt{M}$ which would prove that $M$ is finite locally free. Indeed, for any 

\subsection{Problem 2}

\subsection{Problem 3}

\subsubsection{(a)}

\subsubsection{(b)}

\renewcommand{\Mod}{\mathrm{Mod}}
\newcommand{\GW}{\mathrm{GW}}

Let $R$ be a DVR and $K = \Frac{R}$ and $\hat{R}$ its completion. First we claim that
\[ \GL_n(\hat{K}) = \GL_n(K) \cdot \GL_n(\hat{R}) \]
Since $\hat{K} = \hat{R}[\pi^{-1}]$ this is clear because we can just write $A \in \GL_n(\hat{K})$ as $\pi^{-r} \bar{A}$ for $\bar{A} \in \GL_n(\hat{R})$ so we win taking $\pi^{-1} I \in \GL_n(K)$. Consider the map
\[ \gamma : \Mod_n(R) \to \Vect_n(K) \times_{\Vect_n(\hat{K})} \Mod_n(\hat{R}) \]
Now $\gamma$ is clearly faithful since the projection to $\Mod_n(\hat{R})$ is faithful. To show fullness, consider $M, N$ free modules of rank $n$ and $\alpha : M_K \to N_K$ and $\beta : M_{\hat{R}} \to N_{\hat{R}}$ such that they agree on $M_{\hat{K}} \to N_{\hat{K}}$ we need to produce $(\alpha, \beta) : M \to N$. Indeed, chosing bases we see that $\alpha, \beta$ are represented by a matrix with coefficients in $K \cap \hat{R} \subset \hat{K} = R$ since it lies in $K$ so it suffices to show it has a positive power of $\pi$ for each entry. 


WAIT ITSNT IT THE INTERSECTION NOT THE PRODUCT THAT IS RELEVANT??


\section{Motivic Brauer Degree}

Let $X,Y$ be topological spaces then let $\pi(X,Y)$ be the homotopy classes of continuous map. If $\pi_\bullet(X, Y)$ is the homotopy classes of pointed maps $(X, x_0) \to (Y, y_0)$ then 
\[ \pi(X,Y) = \pi_\bullet(X,Y) / \pi_1(Y, y_0) \]
if $X,Y$ are connected. 

\begin{defn}
For $f : S^n \to S^n$ then we define using the canonical orientation:
\[ \sum_{x \in f^{-1}(y)} \epsilon(f, x) = \deg{(f,y)} \]
\end{defn}

\begin{theorem}
$\deg{(f,y)} \in \Z$ is independent of $y$ and
\[ \deg : \pi_n(S^n) \to \Z \]
is an isomorphism. 
\end{theorem}

\begin{example}
If $f = \frac{p}{q} \in \RR(x)$ with $p,q \in \RR[x]$ coprime then $f_{\RR} : S^1 \to S^1$ has a unique extension to $\mathrm{RP}^1$ and $f_{\C} : S^2 \to S^2$ has a unique extension to $\mathrm{CP}^1$. Observation: $\deg{f_{\CC}}$ and $\deg{f_{\RR}}$ need not be equal. However, because the non-real roots come in pairs we see that
\[ \deg{f_{\RR}} \equiv \deg{f_{\CC}} \mod 2 \]

\end{example}

\begin{defn}
Consider the fiber product:
\begin{center}
\begin{tikzcd}
\GW(\RR) \arrow[r] \arrow[d] & \Z \arrow[d]
\\
\Z \arrow[r] & \Z / 2 
\end{tikzcd}
\end{center}
The data of $\lambda \in \NN$ and $\epsilon \in \Z$ with $|\epsilon| \le \lambda$ and $\lambda \equiv \epsilon \mod 2$ is the data of a quadratic form $f$ over $\RR$ (signature and rank). 
\end{defn}

This is really where our degree is going to be valued. Let $k$ be a field (commutative, lol he said this) and $\P^1_k$ projective space of dimension $1$. Then $\P^1(\RR) = \RP^1$ and $\P^1(\CC) = \CP^1$ so for $k = \RR$ we got a degree for any map $f : \P^1_{\RR} \to \P^1_{\RR}$ valued in $\GW(\RR)$. 

\begin{example}
$\P^1 = \Sigma \Gm$ and $\A^n \sm \{ 0 \} \cong \Sigma^{n-1} (\Gm^{\wedge n})$.
\end{example}

Now given a map $f : \P^1_k \to \P^1_k$ it is given by $f = \frac{p}{q} \in k(x)$ for $p,q \in k[x]$ coprime. Here, the analog of Sard's theorem (which we could use to see that there is a regular value for $S^m \to S^n$ and hence for $m = n$ a point where there is an isomorphism on tangent spaces at each preimage so we can define the index for $m < n$ we see this means it is not surjective and hence nullhomotopic) is generic smoothness if $k$ has characteristic zero. 

\begin{defn}
$\deg{f, y} = \sum_{x \in f^{-1}(y)} \mathrm{Tr}^{k(x)}_k \left< \d{f}(x) \right> \in \GW(k)$ 
where $\d{f}(x) : T_x \P^1 \to T_y \P^1$ and $y \in \P^1_k(k)$ is a $k$-points.
\end{defn}

\begin{rmk}
For $k$ characteristic zero, it is infinite there is always a point $y$ such that the morphism is smooth over $y$. 
\end{rmk}

\begin{theorem}
$\deg{(f,x)}$ is independent of $x$ so we denote by $\deg{f} \in \GW(k)$ this element. 
\end{theorem}

\section{Kirsten's Problems}

\subsection{Problem 1}

Let $X$ be a dualizable object of a symmetric monoidal category $(\C, \ot, 1, \tau)$ with dual $\D X$. Consider the map
\[ \Hom{\C}{X \ot Y}{Z} \to \Hom{\C}{Y}{\D X \ot Z} \]
given by 
\[ Y \to \D X \ot X \ot Y \xrightarrow{\id \ot \epsilon} \to \D \ot Z \]
It should not be hard to show this is a bijection.

\subsubsection{Problem 2}


\subsubsection{Problem 3}

\[ \zeta_{\P^n} = \prod_{i = 0}^n (1 - q^i t)^{(-1)^i} \]

\subsubsection{Problem 4}

We want to show that $\Th_{\P^1}(\struct{}(2)) \cong \Th_{\P^1}(\struct{}(2)) \cong S^0 \vee \P^1$. 

\subsubsection{Problem 5}


\section{Torsors over affine curves II}

\subsection{4}

The important lemma is the following:

\begin{lemma}
Let $f : X \to Y$ be a finite locally free morphism. For any nonempty open $U \subset X$ there is a nonempty open $f^{-1}(V) \subset U$ for some $V \subset Y$.
\end{lemma}

\begin{proof}
Since $f$ is affine we may assume everything is affine to get $\Spec{R'} \to \Spec{R}$ with a finite locally free $\varphi : R \to R'$ ring map. Then we need to show that for all $f \in R'$ there is $g \in R$ such that $D(\varphi(g)) \subset D(f)$ which is true if we can show that $\varphi(g)$ is divisible by $f$. Indeed, we can let $g$ be the norm of $f$ defined as follows:
\begin{center}
\begin{tikzcd}
f_* \struct{X} \arrow[rd, "\mathrm{Nm}"'] \arrow[r] & \shEnd{\struct{Y}}{f_*\struct{X}} \arrow[d, "\det"]
\\
& \struct{Y}
\end{tikzcd}
\end{center} 
Notice that this says $g = \mathrm{Nm}(f)$ vanishes at those points where multiplication by $f$ is not invertible on the fiber. Indeed, $f$ is a $\kappa(y)$-linear endomorphism of $R' \ot_R \kappa(y)$ and $\mathrm{Nm}(f)$ is zero at $y$ iff this is invertible.
If $\varphi(g) \notin \p$ for a prime $\p \subset R'$ then multiplication by $f$ on $R' / \p$ is nontrivial hence $f \notin \p$. Thus $D(\varphi(g)) \subset D(f)$ so we win. 
\end{proof}


\begin{rmk}
However, these cannot cover $X$.
\end{rmk}


\begin{prop}
If $f : X \to Y$ is finite then any vector bundle $\E$ on $X$ is trivialized on some open $f^{-1}(V) \subset X$ for $V \subset Y$.
\end{prop}

\begin{proof}
Indeed, since $X_y$ is zero dimensional $\F|_{X_y}$ is trivial. Therefore, we can spread out so that $\F|_{f^{-1}(V)}$ is trivial for some open $y \in V \subset Y$. Indeed, precisely this is affine local since $f$ is affine so we have $A \to B$ a finite ring map and $M$ a finite locally-free $B$-module then $B \ot_A \kappa(y)$ is free and hence we can spread out the isomorphism:
\[ B^n \ot_A \kappa(y) \iso M \ot_A \kappa(y) \]
by lifting generators and using Nakayama's lemma to 
\[ B^n \ot_A A_f \iso M \ot_A A_f \]
for $f$ nonzero at $y$. Indeed, precisely we get a surjection
\[ B^n_f \onto M_f \]
after lifting a basis to give a surjection at the stalk (by Nakayama) and then this cokernel is killed by localization by some element $f \in A$ by finite generation. Then we have a map of rank $n$ vector bundles which is an isomorphism along $X_y$ but the locus on $X$ where it is an isomorphism is open so taking and open $V \subset Y$ whose preimage lands inside this locus we win. Alternatively we can argue using the inverse.  
\end{proof}

\section{Ayoub Problems}

\newcommand{\mot}{\mathrm{mot}}

\subsection{1}

To show $\Gm : X \mapsto \struct{X}(X)^\times$ is $\A^1$-invariant we just need to show it on smooth affine schemes so we need to show that,
\[ R^\times \to (R[t])^\times \]
is an isomorphism. Indeed, if $f g = 1$ then we must have $f(0) g(0) = 1$ so the constant terms are units. We can localize to assume that $R$ is a domain. Now, the top degree terms are zero divisors so we get $\deg{f} = 0$. 

\begin{rmk}
Consider $R = \Z / 4 \Z$ then $(1 + 2 t)^2 = 1$ in $R[t]$ so this fails for singular schemes. 
\end{rmk}

\subsection{2}

Let $K_{\Nis}(\Gm, 1) ;= L_{\Nis} K(\Gm, 1)$ where $K(\Gm, 1)$ is the Eilenberg Maclane space. We need to show this is $\A^1$-invariant. 

\subsection{3}

It is clear that $\Ga$ is not $\A^1$-invariant since $R \to R[t]$ is not an isomorphism. However, it is a Nisnevich sheaf.
\bigskip\\
Consider $L_{mot} \Ga$. Now, $\Ga = \Hom{}{-}{\A^1}$ and therefore,
\[ \L_{mot} \Ga = \Hom{}{-}{\L_{\mot} \A^1} = \Hom{}{-}{*} = * \]
because $\L_{\mot}$ of $\A^1 \to *$ is an equivalence. 

\subsection{4}

We say $F$ is birational if $F(X) = F(U)$ for any dense open immersion $U \embed X$ and $F(\sqcup_i X_i) = \prod_i F(X_i)$.
\bigskip\\
Let $F$ be birational. To show it is a Nisnevich sheaf we need to show for any square,
\begin{center}
\begin{tikzcd}
V' \arrow[r, hook] \arrow[d] \pullback & X' \arrow[d]
\\
V \arrow[r, hook] & X
\end{tikzcd}
\end{center}
where $X' \to X$ is an \etale map which is an isomorphism over $X \sm V$ and $V \embed X$ is an open immersion. Then we need that
\begin{center}
\begin{tikzcd}
F(X) \arrow[r] \arrow[d] & F(V) \arrow[d]
\\
F(X') \arrow[r] & F(V') 
\end{tikzcd}
\end{center}
is Cartesian. Since the schemes are smooth, using the disjoint union condition, we can assume $X$ is irreducible and $V \embed X$ is nonempty hence dense so $V' \embed X'$ is also dense. However, the horizontal maps are isomorphisms so the diagram is automatically Cartesian. 
\bigskip\\
Now suppose $F$ is a birational presheaf of abelian groups. We want to show that $F$ is strictly $\A^1$-invariant. This means for all $i$ and all smooth schemes $X$ the map,
\[ H^i(X, F) \to H^i(X \times \A^1, F) \]
is an isomorphims. Since $X \times \A^1 \to X$ has a section the maps
\[ H^i(X, F) \to H^i(X \times \A^1, F) \]
are injective so it suffices to prove surjectivity.


{\color{red} WHY EVEN FOR $i = 0$. }

\subsection{5}

Let $k$ be a field. A $k$-scheme $X$ is said to be $\A^1$-rigid if for any $Y \in \Sm_k$, the projection map $\A^1 \times Y \to Y$ induces a bijection $X(Y) \to X(Y \times \A^1)$ meaning $h^X$ is $\A^1$-invariant as a presheaf.

\subsubsection{(a)}

We already showed that $\Gm$ is rigid since $\Hom{}{Y}{\Gm} = \struct{Y}(Y)^\times$. Note that $\pi_0^{\A^1}(X) = \pi_0(L_{\A^1}(X))$ meaning the Nishnevich sheafification of the connected components of the $\A^1$-stabilization of $X$. But $L_{\A^1} \Gm = \Gm$ (I THINK!!) and therefore $\pi_0^{\A^1}(\Gm) = \Gm$.

\subsection{(b)}

Let $X = C$ be a curve of genus $1$. Then consider
\[ \Hom{}{Y}{C} \to \Hom{}{Y \times \A^1}{C} \]
Note that $\A^1 \to C$ is constant and hence the map $Y \times \A^1 \to C$ factors through the projection $Y \times \A^1 \to Y$ so the above is an isomorphism.


{\color{red} WHY WOULDNT $\pi_0^{\A^1}(C) = C$}

\section{Wickelgren II}

\begin{theorem}[Morel]
Let $k$ be a field. Then,
$\End{1_{SH(k)}} = \GW(k)$ is the group completion of the monoid of isomorphism classes of symmetric nondegenerate bilinear forms. 
\end{theorem}

\begin{defn}
For any commutative ring $A$ let $\GW(A)$ be the group completion of symmetric nondegenerate bilinear forms.
\end{defn}

\begin{example}
\[ \GW(\FF_q) \cong \frac{\ZZ[\left< u \right>]_{u \in \FF_q^\times \sm (\FF_q^\times)^2}}{(\left<u\right>^2 - 1, 2 (\left< u \right> - 1))} \cong \Z \times \FF_q^\times / (\FF_q^\times)^2 \cong \Z \times \Z / 2 \]
the first projection is the rank, the second is the discriminant of the quadratic form.
\end{example}

Note that if $F : \C_1 \to \C_2$ is a $\ot$-functor meaning a map of symmetric monoidcal categories and $X \in \C_1$ is a dualizable object then $F(X)$ is dualizable and moreover for any $f : X \to X$ we have, under the unique map $F(1_{\C_1}) \to 1_{\C_2}$ which is an isomorphism,
\begin{center}
\begin{tikzcd}
F(1_{\C_1}) \arrow[d] \arrow[r, "F(\tr{f})"] & F(1_{\C_1}) \arrow[d]
\\
1_{\C_2} \arrow[r, "\tr{F(f)}"] & 1_{\C_2} 
\end{tikzcd}
\end{center}
thus we say $F(\tr{f}) = \tr{F(f)}$. In particular, this shows that any (reduced\footnote{We have to take reduced cohomology so that $\P^1$ has cohomology only in one degree and smash product is taken to $\ot$ under reduced cohomology so it just multiplies by $H^2(\P^1)$ the tate motive which is invertible.}) cohomology theory,
\[ \RR \Gamma : SH(B) \to D^{\text{perf}}(\mathrm{Vect}_K) \]
meaning a $\ot$-functor then it automatically satisfies a Lefchetz trace formula. Indeed, 
\[ \End{1_{SH(B)}} \to \End{1_{D^{\text{perf}}(\mathrm{Vect}_K)}} = K \]
is just $\GW(\FF_q) \to \Z \to K$ and
\[ \tr{f} \mapsto \tr{f^*|_{\RR \Gamma}} = \sum_{i} (-1)^i \tr{f^*|_{H^i}} \]

\begin{defn}
Let $X$ be $SH(B)$ be dualizable equipped with a self-map $F : X \to X$ then we define the $\A^1$-logarithmic $\zeta$ function
\[ \d{\log{\zeta^{\A^1}_{X,F}}(t)} := \sum_{m \ge 1} \tr{F^m} t^{m-1} \in \End{1_{SH(B)}}[[t]] \]
\end{defn}

\begin{prop}
Let $X$ be smooth, projective over $\FF_q$ then
\[ \rank \d{\log}{\zeta^{\A^1}_X} = \d{\log}{\zeta_X} \in \Z[[t]] \]
\end{prop}

\renewcommand{\Tr}{\mathrm{Tr}}

Let $A \subset A'$ be a finite \etale extension then
\[ \Tr_{A'/A} : \GW(A') \to \GW(A) \]
sends $[V \times V \to A'] \mapsto [V \times V \to A' \xrightarrow{\Tr_{A'/A}} A]$.

\begin{example}
\[ \Tr_{\FF_{q^d} / \FF_q} \left< 1 \right> = 
\begin{cases}
d & d \text{ odd}
\\
d - 1 + \left< u \right> & d \text{ even} 
\end{cases} \]
where $u$ is some nonsquare. 
Indeed, the rank is $d$ and the discriminant is the discriminant of the extension $\FF_{q^d} / \FF_q$ which is exactly asking if the Vandermonde determinant is Frobenius fixed which is asking about the sign of a $d$-cycle.
\end{example}

\begin{theorem}[Hoyois]
Let $K$ be a field. Let $X$ be a smooth proper $K$-scheme. Let $F : X \to X$ be an endomorphism with \etale fixed points. Then,
\[ \tr{F} = \sum_{x \in X^F} \Tr_{k(x)/k} \left< \det{(I - \d{F}_x)} \right> \]
Note $X^F$ is etale if and only if $(I - \d{F}_x) : T_x X \to T_x X$ is invertible so the class of the determinant is well-defined.
\end{theorem}

\begin{example}
Then $X / \FF_q$ with the action of Frobenius, $\ind_x F = \Tr_{k(x) / k} \left< 1 \right>$ since $\d{F} = 0$. Therefore,
\[ \d{\log}{\zeta^{\A^1}_X}(t) = \sum_{m \ge 1} \sum_{d \divides m} \alpha_X(d) \Tr_{\FF_{q^d} / \FF_q} \left< 1 \right> t^{m-1} \]
where $\alpha_X(d)$ is the number of points of $X$ with $\kappa(x) \cong \FF_{q^d}$. There is a mobius inversion formula relating $\alpha_X(d)$ with the sequence $\# X(\FF_{q^d})$. 
\end{example}

\begin{theorem}[X, Hu]
$X$ smooth and proper over $\FF_q$ then $\d{\log}{\zeta^{\A^1}_X(t)}$ is rational. 
\end{theorem}

\begin{example}
For the complete hyperelliptic curve $y^2 = x^3 + x + 5$ over $\FF_{13}$ then
\[ \d{\log{\zeta_X^{\A^1}(t)}} = q + (170 + \left< u \right>) t + 2268 t^2 + (28898 + \left< u \right>) t^3 + \cdots \]
\end{example}

\begin{rmk}
$\d{\log}{\Tr^{\A^1}_F(\zeta_X^{\text{mot}})} \in \Z[[t]]$ so we don't get the same thing as what is defined above.
\end{rmk}

\begin{theorem}
Let $X$ be smooth projective ``cellular'' scheme over a perfect field $K$ and $F : X \to X$ then 
\[ \d{\log{\zeta_X^{\A^1}(t)}} = \sum_{r = 0}^n - \left< -1 \right>^r \d{\log{P_r(t)}} \]
where 
\[ P_r(t) = \det{(1 - t F|_{C_\bullet^{\text{cell}}(X)})} \]
is a square matrix of elements in $\GW(k)$ and ${C_\bullet^{\text{cell}}(X)}$ is the $\A^1$-cellular complex of Morel-Sawant. 
\end{theorem}

\begin{example}
$F \acts \P^n_{\FF_q}$ then
\[ P_r(t) = (1 - q_\epsilon^r t) \]
where $q_\epsilon := $ and therefore
\[ \d{\log{Z^{\A^1}_{\P^n}(t)}} = \sum_{r = 0}^d - \left< - 1 \right>^r (1 - q_\epsilon^r t) \]
\end{example}

Inside the category of Nisnevich sheaves of abelian groups there are the strictly $\A^1$-invariant ones. Consider
\[ H_n^{\A^1}(X) := \pi_n^{\A^1} \Z[X] \]
and likewise
\[ \wt{H}^{\A^1}_n(X) := \pi_n^{\A^1} \Z(X) \]
Then $X$ is \textit{cellular} if there are open $\Omega_i \subset X$ filtration such that $\Omega_i \sm \Omega_{i-1} = \sqcup \A^{d-i}$ then
\[ \Omega_i / \Omega_{i-1} = (\P^1)^{\wedge i} \wedge (\prod \A^{d-i})_+ \]
then there is a long exact sequence
\[ \cdots \to \wt{H}^{\A^1}_i(\Omega_i) \to \wt{H}^{\A^1}_i(\Omega_i / \Omega_{i-1}) \to \wt{H}^{\A^1}_{i-1}(\Omega_{i-1}) \to \cdots \]

\begin{defn}[Morel-Sawant]
$C^{\text{cell}}_{\bullet} = (\wt{H}^{\A^1}_i(\Omega_i / \Omega_{i-1}), \d)$. 
\end{defn}

\section{Ayoub Problems II}

\subsection{1}

Since $B \ZZ = K(\ZZ, 1) = S^1$ is a $K(\pi,1)$ is it automatic that it is $1$-truncated. Isn't this just the fact that $\pi_{>1}(S^1) = 0$ which is obvious from the fact that it has contractible uniersal cover.

\subsection{2}

The universal cover $\wt{X} \to X$ is a $\pi_1(X)$-torsor and hence we get a fibration
\[ \wt{X} \to X \to B \pi_1(X) \]
Furthermore, by definition, $\pi_1(\wt{X}) = 0$ and $\pi_0(\wt{X}) = *$ so $\wt{X}$ is $2$-connective (i.e. $1$-connected). 

\subsection{3}

In the motivic homotopy category

\begin{enumerate}
\item $\Sigma \Gm \cong \P^1$ because $\P^1$ is the pushout of $\A^1, \A^1$ along $\Gm$ and the  $\A^1$ is contractible
\item $\P^n / \P^{n-1} = \Th_{\P^{n-1}}(\struct{}(1))$ because this is the normal bundle and we use Purity. However, the total space of $\struct{}(1)$ over $\P^{n-1}$ is $\A^n$ so by definition
\[ \Th_{\P^{n-1}}(\struct{}(1)) = \A^n / (\A^n \sm \{ 0 \}) \]
\item 
\end{enumerate} 

\subsection{4}

We need to show that $\pi_0(L_{\mot} \SL_n) = *$. 

\section{Gilles III}

\begin{theorem}
Let $k$ be an algebraically closed field and $C/k$ an affine algebraic curve. Let $G/k$ be a \textit{semisimple} algebraic group over $k$ then
\[ H^1_{\fppf}(C, G) = * \]
\end{theorem}

The proof is by strong approximation.

\begin{rmk}
Recally that $G$ is semisimple if every smooth connected solvable normal subgroup of $G$ is trivial. For example, $\SL_n$ is semisimple but $\GL_n$ is not because it has a nontrivial center. Indeed, there are many affine curves with nontrivial line bundles. However, the above theorem implies that no nontrivial vector bundle has trivial determinant. Indeed, on an affine curve any vector bundle is of the form,
\[ \E \cong \struct{}^{r - 1} \oplus \L \]
so $\det{\E} = \L$ determines if $\E$ is trivial. This follows from the same argument that $K_0(C) = \Z \times \Pic{C}$ for any curve plus the fact that sequences split (or using Schanual's lemma).
\end{rmk}

\begin{example}
This implies that the Brauer group is trivial. Indeed, $\PGL_n$ is semisimple. We also know this because $k(C)$ is a $C^1$ field. 
\end{example}

\section{Gilles IV}

\begin{theorem}
Let $G/k$ be a reductive group. Then
\[ \ker{(H^1(\A^1_k, G) \to H^1(\A_{k_s}^1, G))} = 1 \]
If $k$ is perfect then $H^1(\A^1_k, G) = 1$.
\end{theorem}

There are examples in ``small characteristic'' for $k$ imperfect eg $G = \PGL_p = \GL_p / \Gm$ and $H^1(k[t], G)$ is Azumaya $k[t]$-algebras of degree $p$. For example,
\[ X^p = x + t \quad y^p = a \quad y x y^{-1} = x + 1 \]
Then the class of this algebra is nontrivial in $H^1(k[t], \PGL_p)$ and nontrivial in $H^1(k_s((1/t)), \PGL_p)$. 


\section{Kirsten Problems III}



\section{Ayoub Problems III}

\section{1}

\newcommand{\Map}{\mathrm{Map}}

Let $M$ be a sheaf of abelian groups. We want to show that $M$ is $n$-strongly $\A^1$-invariant iff $K(M, n)$ is motivic. Now the definition of $n$-strongly $\A^1$-invariant is that the map
\[ H^i(X, M) \to H^i(X \times \A^1, M) \]
is an isomorphism for $i \le n$ and all smooth $X$. 
\par 
We know that,
\[ H^i(X, M) \cong [X, K(M,i)] = \pi_0 K(M,i)(X)\]

\begin{lemma}
Let $X, Y$ be motivic spaces. Then the natural map
\[ \Omega \Map(X, Y) \to \Map(X, \Omega Y) \]
is an equivalence. 
\end{lemma}

\begin{proof}
Indeed, $\Map(X, -)$ preserves all homotopy limits. Therefore, it takes the homotopy pullback diagram,
\begin{center}
\begin{tikzcd}
* \arrow[r] & Y 
\\
\Omega Y \arrow[u] \arrow[r] & * \arrow[u]
\end{tikzcd}
\end{center}
to the homotopy pullback diagram
\begin{center}
\begin{tikzcd}
* \arrow[r] & \Map(X, Y)
\\
\Map(X, \Omega Y) \arrow[u] \arrow[r] \arrow[r] & * \arrow[u]
\end{tikzcd}
\end{center}
\end{proof}

Assume that $K(M,n)$ is motivic meaning that the map
\[ \Map(X, K(M,n)) \to \Map(X \times \A^1, K(M,n)) \]
is an equivalence. Therefore the maps
\begin{center}
\begin{tikzcd}
\Omega^{n-i} \Map(X, K(M,n)) \arrow[r] \arrow[d] & \Omega^{n-i} \Map(X \times \A^1, K(M,n)) \arrow[d]
\\
\Map(X, \Omega^{n-i} K(M,n)) \arrow[d] \arrow[r] & \Map(X, \Omega^{n-i} K(M,n)) \arrow[d]
\\
\Map(X, K(M,i)) \arrow[r] & \Map(X \times \A^1, K(M,i))
\end{tikzcd}
\end{center} 
are equivalences for $0 \le i \le n$. In particular, the top map is an isomorphism in the diagram,
\begin{center}
\begin{tikzcd}
\pi_0 \Map(X, K(M,i)) \arrow[r] \arrow[d] & \pi_0 \Map(X \times \A^1, K(M, i)) \arrow[d]
\\
H^i(X, M) \arrow[r] & H^i(X \times \A^1, M)
\end{tikzcd}
\end{center}
but we already saw the downward maps are isomorphims so we get that the bottom map is an isomorphism proving that $M$ is $n$-strongly $\A^1$-invariant.
\bigskip\\
Conversely, suppose that $M$ is $n$-strongly $\A^1$-invariant. We need to show that
\[ \Map(X, K(M,n)) \to \Map(X \times \A^1, K(M,n)) \]
is an equivalence. To do this, we will show that all $\pi_i$ of this are isomorphisms. Consider,
\begin{center}
\begin{tikzcd}
\pi_i  \Map(X, K(M,n)) \arrow[r] \arrow[d, equals] & \pi_i \Map(X \times \A^1, K(M,n)) \arrow[d, equals]
\\
\pi_0 \Omega^i  \Map(X, K(M,n)) \arrow[r] \arrow[d, equals] & \pi_0 \Omega^i \Map(X \times \A^1, K(M,n)) \arrow[d, equals]
\\
\pi_0 \Map(X, \Omega^i K(M,n)) \arrow[r] \arrow[d, equals] & \pi_0 \Map(X \times \A^1, \Omega^i K(M,n)) \arrow[d, equals]
\\
\pi_0 \Map(X, K(M,n-i)) \arrow[d,equals] \arrow[r] & \pi_0 \Map(X \times \A^1, K(M, n-i)) \arrow[d,equals]
\\
H^{n-i}(X, M) \arrow[r] & H^{n-i}(X \times \A^1, M)
\end{tikzcd}
\end{center}
and for $i > n$ we have $\Omega^i K(M,n) \cong *$ so the mapping space is trivial. Therefore, since the bottom map is an isomorphism by assumption, we see that so is the top map for all $i$ so we win.

\section{2}

Let $S = \Spec{R}$ of an essential smooth dvr with closed point $x$. We want to compute the cohmology with supports $H^i_x(S, \Gm)$ in the Nisnevich topology. There is an exact sequence,
\[ \cdots \to H^i_x(S, \Gm) \to H^i(S, \Gm) \to H^i(U, \Gm) \to \cdots \]
Since $U = \Spec{K}$ there is no Nisnevich cohomology for $i > 0$. Therefore,
\[ 0 \to H^0_x(S, \Gm) \to R^\times \to K^\times \to H^1_x(S, \Gm) \to H^1(S, \Gm) \to 0 \]
and for $i \ge 2$ we have,
\[ H^i_x(S, \Gm) \cong H^i(S, \Gm) \]
Therefore,
\[ H^i_x(S, \Gm) = 
\begin{cases}
0 & i = 0
\\
K^\times / R^\times \cong \Z \pi & i = 1
\\
0 & i > 1
\end{cases} \]
because $\dim{S} = 1$ so the cohomology vanishes in the Nisnevich topology for larger $S$. {\color{red} CHECK}

\section{3}

Consider the action $\SL_n \acts (\A^n \sm \{ 0 \})$. This is transitive. The stabilizer of $(1,0,\dots, 0)$ is a matrix with this as the first column. The claim is this kernel is $\A^1$-equivalent to $\SL_{n-1}$. This is clear because the matrix is a block
\[ A = \begin{pmatrix}
1 & \vec{v}
\\
\vec{0} & B
\end{pmatrix} \]
where $B \in \SL_{n-1}$ and $\vec{v}$ is arbitrary since the determinant of this matrix is $\det{B}$. Sending $\vec{v} \to \vec{0}$ gives the $\A^1$-deformation retract.  

\section{Gilles Problesm IV}

\subsection{1}

Let $d \ge 1$ be an integer and let $R'$ be a $G = \Z/d\Z$-Galois extension. 

\subsubsection{1}

The map $N(y) = \prod_{\sigma \in G} \sigma(y)$ gives a map $R_{R'/R}(\Gm) \to \Gm$. Indeed, this is functorial and a group homomorphism. 

\subsubsection{2}

To show that
\[ 1 \to \ker{N} \to R_{R'/R}(\Gm) \to \Gm \to 1 \]
is exact it suffices to show the second map is surjective. It is surjective in the fppf topology almost by definition since it amounts to solving a separable polynomial.

\subsubsection{3}

\subsubsection{4}

There is a natural map $\Gm \to R_{R'/R}(\Gm)$ given by the unit of the adjunction
\[ \Hom{}{X_{R'}}{Y} = \Hom{}{X}{R_{R'/R} Y} \]
which is the map
\[ \id \in \Hom{}{X_{R'}}{X_{R'}} = \Hom{}{X}{R_{R'/R}(X_{R'})} \]
This almost gives a section in the sense that $N \circ \iota : z \mapsto z^d$. However, we can get a map $\sigma - 1 : R_{R'/R}(\Gm) \to \ker{N}$ via $x \mapsto \sigma(x) x^{-1}$. The kernel of this is exactly those $x$ fixed by the Galois group which is $\Gm \embed R_{R'/R}(\Gm)$ so we get the exact sequence,
\[ 1 \to \Gm \to R_{R'/R}(\Gm) \to \ker{N} \to 1 \]
To see surjectivity, we can evaluate on $R'$ where we have the map
\[ (R' \times \cdots \times R')^\times \to \ker{N_{R'}} \]
Indeed, the map 
\[ (x_1, \dots, x_n) \mapsto (x_1 x_2^{-1}, \dots, x_d x_1^{-1}) \]
for $x_i \in (R')^\times$ surjects onto $(y_1, \dots, y_d)$ with $y_1 \cdots y_d = 1$ via
\[ x_{i+1} = x_i y_i^{-1} \]
and $x_1$ arbitrary so the kernel is exactly those of the form $(\lambda, \dots, \lambda)$. Indeed, we need to check only the last relation
\[ y_d = x_d x_1^{-1} = y_{d-1}^{-1} \cdots y_1^{-1} x_1 x_1^{-1} \]
which holds since
\[ y_1 \cdots y_d = 1 \]

\subsubsection{5}

The exact sequence
\[ 1 \to \Gm \to R_{R'/R}(\Gm) \to \ker{N} \to 1 \]
gives
\[ R^\times \to (R')^\times \xrightarrow{\sigma - 1} \ker{N}(R) \to \ker{(\Pic{R} \to \Pic{R'})} \]

\subsubsection{6}

What does this mean?

\subsubsection{7}

For $R' = \CC$ and $R = \RR$ we have $\ker{N} = \Spec{\RR[x,y]/(x^2 + y^2 - 1)}$ the circle with multiplication ``as complex numbers''. We want to know if 
\[ R_{R'/R} \Gm \to \ker{N} \]
is a trivial or nontrivial $\Gm$-bundle. Note that $\ker{N}$ as a scheme is $\P^1_{\RR}$ minus a $\CC$-point so this $\Pic{\ker{N}} = \Z / 2 \Z$ generated by the ``Mobius bundle''. In this case, there is a trick consider the $\RR$-points with the Euclidean topology, 
\[ \CC^\times \to U(1) \]
given by $z \mapsto \bar{z} z^{-1}$. This does not even have a topological section because the mao $\pi_1(\CC^\times) \to \pi_1(U(1))$ is $\Z \to \Z$ via $1 \mapsto -2$ since $e^{2 \pi i t} \mapsto e^{-4 \pi i t}$. This is not surjective so it cannot have a section.
\bigskip\\
More generally, if $K'/K$ is a Galois field extension of degree $2$ then $R_{K'/K} \Gm$ corresponds to the Galois module $\Z^2$ with action $(x,y) \mapsto (y,x)$. Then the sequece
\[ 1 \to \Gm \to R_{R'/R}(\Gm) \to \ker{N} \to 1 \]
is given by the sequence
\[ 0 \to \Z \to \Z^2 \to \Z \to 0 \]
where the first map is $x \mapsto (x,x)$ and the second is $(x,y) \mapsto x - y$. The Galois action on the first $\Z$ is trivial and on the second is $1 \mapsto -1$. This sequence does not have a section. Indeed, to be Galois invariant we must have $(x,y) \mapsto x + y$ for a left section but this takes $x \mapsto (x,x) \mapsto 2 x$. Likewise to be Galois invariant a right section must send $x \mapsto (x,-x)$ so $x \mapsto -x$ is compatible with $(x,y) \mapsto (y,x)$. Then $x \mapsto (x,-x) \mapsto 2 x$. Therefore, we only have sections after a $2$-isogeny this corresponds to the fact that,
base changing along $2 : \ker{N} \to \ker{N}$ gives a section since the composition 
\[ \ker{N} \to R_{K'/K} \Gm \xrightarrow{\sigma - 1} \ker{N} \] 
is multiplication by $2$ and likewise so is
\[ \Gm \to R_{K'/K} \Gm \xrightarrow{N} \Gm \]

\subsection{2}

Let $k$ be a field of characteristic $p$. Let $F$ be the Frobenius and $G$ a smooth \textit{connected} algebraic group. Then $G \acts G$ via 
\[g \cdot t =  g t F(g)^{-1} \]
For $t \in G(\bar{k})$ consider the orbit map
\[ \phi_t : G \to G \quad g \mapsto g t F(g)^{-1} \]
We can compute that $\phi_t$ is \etale and hence the orbits are open but $G$ is irreducible so the obits must intersect but orbits are distinct hence there is a unique orbit meaning $\phi_t$ is surjective for all $t$ including $t = 1$ which proves that $g \mapsto g F(g)^{-1}$ is surjective. 
\par 
To prove $\phi_t$ is \etale consider the derivative. The map is 
\[ G \xrightarrow{(\id, t, F(\iota))} G^3 \xrightarrow{m} G \]
But the derivatives of the second two are zero so we have $\d{m} \circ \d{(\id, t, F(\iota))} = \d{m} \circ (\id, 0, 0)$ is an isomorphism because $(\d{\phi_t})_{g_0}$ is equal to the derivative of $g \mapsto g t F(g_0)^{-1}$ at $g_0$ and this is clearly an isomorphism since this map is an isomorphism. Therefore, $\phi_t$ is \etale.  
\par 
In particular $\pi_1(\SL_{n, \bar{\FF}_p}) \neq 0$ since it has a connected \etale cover.


\subsection{3}

This was on the last pset. 

\subsection{4}

\newcommand{\HH}{\mathbb{H}}

Let $\HH$ be the quaternions. Let $A = \HH[x,y]$ and we want to construct an invertible right $A$-module which is not free. Consider
\[ 0 \to P \to A^2 \xrightarrow{f} A \to 0 \]
where
\[ f : (\gamma, \mu) \mapsto (x + i) \gamma - (y + j) \mu \]
This is an invertible right module (since we multiply on the left in $f$ it is a right module map) we need to show $P$ is not free. We will find two elements of $P$ which are not multiples. Consider
\[ (\gamma_1, \mu_1) = ((x-i)(y + j),x^2 + 1) \quad \quad (y^2 + 1, (y - j)(x+i)) \]
We need to show that these are not multiples of eachother on the right by $A$. Suppose there is $\lambda \in A$ such that
\[ \gamma_1 \lambda = \gamma_2 \quad \mu_1 \lambda = \mu_2 \]
If we set $x = i$ then $\gamma_1 = 0$ but $\gamma_2 \neq 0$ giving a contradiction. Likewise, if 
\[ \gamma_1 = \lambda \gamma_2 \quad \mu_1 = \lambda \mu_2 \]
then setting $y = j$ we get $\mu_2 = 0$ but $\mu_1 \neq 0$ giving a contradiction.

\section{Morel}

\begin{theorem}
For any $n \ge 2$ 
\[ \pi_{n-1}^{\A^1}(\A^{n} \sm \{ 0 \}) = K^{MW}_n \]
where $K^{MW}_\bullet$ is the graded sheaf whose stalks on a field $k$ are given by,
\[ K^{MW}_\bullet = \text{Tensor}(k^\times, \eta) / I \]
the tensor algebra on the \textit{set} $k^\times$ and an extra element $\eta$ where $I$ contains the following relations
\begin{enumerate}
\item for $u \neq 0,1$ we have $[u] [u-1] = 0$
\item $[ab] = [a] + [b] + [a][b] \eta$
\item $\eta (1 + (1 + \eta [-1])) = 0$
\item $\eta [a] = [a] \eta$
\end{enumerate} 
we give $k^\times$ degree $1$ and $\eta$ degree $-1$. The element $1 + (1 + \eta [a])$ is in degree zero. We write
\[ \left< a \right> = (1 + \eta [a]) \]
and we get that $K_0^{MW}(k) = GW(k)$ where these symbols are interpreted as the usual generators of $\GW(k)$. Then the third relation is
\[ \eta \cdot H = \eta \left( \left<1 \right> +  \left< - 1 \right> \right) = 0 \]
\end{theorem}

\begin{rmk}
We have
\begin{enumerate}
\item $(k^\times)^{\wedge n} \to K_n^{MW}(k)$ via $(\mu_1, \dots, \mu_n) \mapsto [\mu_1] \cdots [\mu_n]$
\item $K_\bullet^{MW}(k) / \eta = K_\bullet^M(k)$ is Milnor K-theory
\item $K_\bullet^{MW}(k) / H = K_\bullet^W(k)$ is Witt K-theory
\item for $n < 0$ we have $K_n^{MW}(k) = W(k)$
\item for all $n \in \Z$ there is a cartesian square
\begin{center}
\begin{tikzcd}
K_n^{MW}(k) \arrow[r] \arrow[d] & K^M_\bullet(k) \arrow[d]
\\
I^n(k) \arrow[r] & I^n(k) / I^{n+1}(k)
\end{tikzcd}
\end{center}
\item there is a diagram
\begin{center}
\begin{tikzcd}
K_0^{MW}(k) \arrow[d] \arrow[r, two heads] & \Z \arrow[d]
\\
W(k) \arrow[r] & \Z / 2
\end{tikzcd}
\end{center}
\end{enumerate}
\end{rmk}

\newcommand{\B}{\mathrm{B}}
\newcommand{\Spaces}{\mathrm{Spaces}}
\newcommand{\Cov}{\mathrm{Cov}}


Consider the fibration
\[ \GL_n / \GL_{n-1} \to B \GL_{n-1} \to B \GL_n \]
the first is equivalent to $\A^n \sm \{ 0 \}$ so for any $\xi : X \to \B \GL_n$ we get an obstruction class
\[ e(\xi) \in H^n(X, \pi_{n-1}^{\A^1}(\A^n \sm \{0\})(\det \xi)) = H^n(X, K^{MW}_n(\det)) \]

\subsection{On $\pi_1^{\A^1}$}

If $X \in \Spaces(k)$ then $\pi_0^{\A^1}(X) = *$. Consider $\Cov(X)$ to be maps $Y \to X$ satisfying the homotopy lifting property 



There is a universal cover $\wt{X} \to X$ with an action of $\pi_1^{\A^1}$ is universal

\begin{example}
For $n \ge 2$ we have a fibration
\[ \Gm \to \A^{n+1} \sm \{ 0 \} \to \P^n \]
which gives the universal cover of $\P^n$ with $\Gm$ is the fundamental group.
\bigskip\\
For $n = 1$ we still have
\[ \Gm \to \A^{2} \sm \{ 0 \} \to \P^1 \]
but now $\A^2 \sm \{ 0 \}$ is not $\A^1$-1-connected. Considering the map $\P^1 \to B \Gm$ we get 
\[ 0 \to \pi_1^{\A^1}(\A^2 \sm \{ 0 \}) \to \pi_1^{\A^1}(\P^1) \to \Gm \to 1 \]
there is a map $\sigma : \Gm \to \pi_1^{\A^1}(\P^1)$ (not group theoretic) because $\P^1 = \sigma \Gm$ from the suspension. It turns out that $\pi_1^{\A^1}(\P^1)$ is not abelian.
\end{example}

\begin{theorem}
If $X$ is an $\A^1$-connected smooth projective surface over $k = \bar{k}$ then
\[ X \sm \{ x \} \cong \bigvee_{i = 1}^r \P^1 \]
where $r = \rank{\NS(X)}$.
\end{theorem}

\begin{rmk}
This is like in topology if you take a closed surface and remove one point you get a wedge of spheres. 
\end{rmk}


\section{Deglise I}

Note that inverting $- \ot \P^1$ also inverts $-\ot S^1$ and $- \ot \Gm$ since $\P^1 = S^1 \wedge \Gm$. 
\bigskip\\
Let $\1(1)  = \Sigma^\infty \P^1 \wedge S^{-2}$ and $\![1] = \Sigma^\infty S^1$.

\begin{defn}
$E \in SH_0(S)$ is a ring spectrum if it is a commutative monoid in the homotopy category of $SH_0(S)$. Then we define
\[ E^{n,i}(X) = [\Sigma^\infty X, E(i)[n]]^*_S \]
where
\[ E(i)[m] = E \ot \1_S(1)^{\ot i} \ot \1_S[1]^{\ot m} = E \wedge S^{n-i} \wedge \Gm^{\wedge i} \]
\end{defn}

There are maps
\[ \Sigma^\infty X_+ \xrightarrow{\delta} \Sigma^\infty (X \times X)_+ \cong \Sigma^\infty X_+ \ot \Sigma^\infty X_+ \]
therefore for two classes
\[ a \in E^{n,i}(X) \quad b \in E^{m,j}(X) \]
we get a composition
\[  \Sigma^\infty X_+ \xrightarrow{\delta} \Sigma^\infty (X \times X)_+ \cong \Sigma^\infty X_+ \ot \Sigma^\infty X_+ \xrightarrow{a \ot b} E(i)[n] \ot E(j)[m] \xrightarrow{\mu} E(i+j)[n+m]
\]
where $\mu$ is the multiplication. Therefore we get a cup product. Note, for any homotopy type $X$ we can evaluate $E$ on it by the same formula, $X$ does not need to be a smooth scheme.

\begin{example}
\begin{enumerate}
\item Any Weil cohomology theory gives an $E$ (except apparently Crystaline cohomology is not $\A^1$-invariant...) we call the associated spectrum $H_{\E}$. 
\item For $\sigma " k \embed \RR$ consider $H^\bullet_{\text{sing}}(X(\RR), \Z)$ is also representable by a ring spectrum $H_{\sigma}$ note $H_{\sigma}^{n,i}(X) = H^{n-i}(X(\RR), \Z)$ since wedge with $\Gm(\RR)$ does nothing. 
\item ``Absolute''' cohomologies:
\begin{enumerate}
\item motivic cohomology over $S$ represented by a ring spectrum $H \Z_S$ such that
\[ H^{2n, n}(X, \Z) \cong \CH^n(X) \]
when $S$ is a field and 
\[ H^{n,n}(k, \Z) = K^M_n(k) \] 
\end{enumerate}
\end{enumerate}
\end{example}

\section{Characteristic Classes I}

Let $(\E, c)$ be an oriented cohomology theory (or represented by an oriented $\E_\infty$-motivic ring spectrum) this means we have an element
\[ c \in \wt{\E}^{2,1}(\P^\infty_S) \]
such that under the canonical $\iota : \P^1_S \to \P^\infty_S$ based at $\infty$ we have
\[ \iota^* c \in \wt{\E}^{2,1}(\P^1) = \E^{0,0}(S) \]
is identified with $1_{\E}$. 
\bigskip\\
Let $V \to S$ he a vector bundle of rank $r$. Let $\struct{V}(1)$ be the canonical line bundle on $\P(V)$. Show that the map
\[ \bigoplus_{i = 0}^{r - 1} \E(S) \to \E(\P(V)) \]
given by
\[ (\lambda_i)_i \mapsto \sum_{i = 0}^{r-1} p^*(\lambda_i) c_1(\struct{V}(1))^{i} \] 
is an isomorphism. 

\section{Field Arithmetic and Galois Cohomology}

Let $K$ be a nonreal field (cannot be ordered or something) then $u(K)$ is the maximal dimension of a non-isotropic quadratic form. For example,
\begin{enumerate}
\item $u(\FF_q) = 2$ 
\item $u(\CC) = 1$
\item $u(\Q_p) = 4$
\item $u(\Q_p(t)) = 8$
\item for $F$ a number field $u(F) = 4$
\item $u(F(t))$ is unknown but expected to be $8$
\end{enumerate}

Question: what is the relationship between quadratic forms and Brauer-Severi varities. For rank $3$ these correspond to conics so the smooth quadratic forms correspond to Brauer-Severi curves and it is isotropic iff the Brauer-Severi is trivial. 
\bigskip\\
However, quadric equations in $\P^3$ correspond to forms of $\P^1 \times \P^1$ not of $\P^2$. 

\begin{rmk}
When I write $O_n$ I mean for $\floor{n/2}$ copies of $H$ and one copy of $\left< 1 \right>$. 
\end{rmk}

\newcommand{\PGO}{\mathrm{PGO}}

This is explained by the following exceptional isomorphisms,
\[ \PGO_2 \iso \Gm \rtimes (\Z / 2 \Z) \]
so we see that rank $2$-quadratic forms are identified with $(\Z / 2 \Z)$-torsors over fields. Also
\[ \PGO_3 \iso \PGL_2 \]
this is basically the isomorphism
\[ \SL_2 / \mu_2 \iso \SO_3 \]
gives that conics are $\P^1$-forms. Also
\[ \PGO_4 \iso (\PGL_2 \times \PGL_2)/(\Z / 2 \Z) \]
means that quadrics in $\P^3$ are forms of $\P^1 \times \P^1$ with the structure of a basis of lines up to swap. 

\[ \PGO_5 \iso \mathrm{PGSp}_2 \]
and
\[ \PGO_6 \iso \PGL_4 \]
The inverse map
sends an automorphism $\varphi \in \End{k^4}$ to $\varphi \wedge \varphi \in \End{\wedge^2 k^4}$ which preserves the symmetric bilinear form
\[ B : (\wedge^2 k^4) \times (\wedge^2 k^4) \to \wedge^4 k^4 \]
therefore any $\varphi \mapsto \varphi \wedge \varphi$ sends this bilinear form to $\det{\varphi}$ times it. The kernel is $\lambda I$ but these are the same in $\PGL_4$. This
says that quadrics in $\P^5$ correspond to forms of $\P^4$. 

\subsection{Milnor Conjecture}

We have $I(F) \subset W(F)$ corresponding to even dimensional forms. Let $I^n(F) = I(F)^n$. Then let $K^M_\bullet(F)$ is generated by $F^\times$ with relation $[a][1-a] = 0$. Then
\[ K^M_n(F) \to I^n(F) / I^{n+1}(F) \]
via
\[ [a_1] \cdots [a_n] \mapsto \left<1,-a_1\right> \ot \cdots \ot \left<1, -a_n \right> \]
gives a surjective map whose kernel is multiples of $2$. Therefore,
\[ K^M_\bullet(F) / 2 \iso \mathrm{gr}_I W(F) \]
Furthermore, both are isomophic to $H^\bullet(F, \mu_2)$. 

\section{Remark on $L_{\A^1}$}

Usually $L_{\A^1}$ and $L_{\text{mot}}$ mean the same thing. They definitely do when Fabian Morel writes
\[ \pi_i^{\A^1}(X) := \pi_i(L_{\A^1} X)_{\text{Nis}} \]
However, for some reason, sometimes people write
\[ L_{\A^1}(X) := (U \mapsto X(U \times \Delta) \]
Then you would define $L_{\text{mot}}$ as a colimit.  

\section{Deglise II}

\renewcommand{\th}{\mathrm{th}}
\newcommand{\EE}{\mathbb{E}}

Given an embedding $\sigma : k \embed \RR$ there is a functor
\[ \sigma^*: SH(k) \to \mathrm{Spectra} \]
generated by $\Sigma^\infty X_{+} \mapsto \Sigma^\infty X(\RR)_+$. This has an adjoint $\sigma_*$ the ``constant spectrum''. Then $H_\sigma(\ZZ) = \sigma_* (H \ZZ)$. 

\begin{rmk}
Show that $H_\sigma \Z$ is \textit{not} orientable. Indeed,
\[ H_\sigma^{2,1}(\P^r) = H^{2-1}(\P^r(\RR), \ZZ) = H^{1}(\RP^r, \ZZ) = 
\begin{cases}
\Z & n = 1
\\
0 & n > 1
\end{cases} \]
so we cannot orient past $n = 1$. 
\end{rmk}

\begin{defn}
Thom class of a vector bundle $V \to S$. We let
\[ \th(V) = \sum_{i = 0}^m c_i(V) \cdot (-c_1(\struct{}(-1))^{i-1} \]
it is naturally defiend as a class $\ol{\th}(V) \in \EE^{2m,m}(\Th(V))$ but by Thom isomorphism defines, if $(\EE, c)$ is oriented
\[ 0 \to \EE(\Th(V)) \to \EE(\P(V \oplus \struct{S})) \to \EE(\P(V)) \to 0 \]
\end{defn}

Thom isomorphism
\[ \EE(X) \to \EE(\Th(V)) \]
given by
\[ \lambda \mapsto \lambda \ol{\th}(V) \]



\begin{rmk}
Given the exact sequence
\[ 0 \to \struct{\P(V \oplus \struct{})}(-1) \to \pi^* (V \oplus \struct{}) \to \cQ \to 0 \]
then $\th(V) = c_m(\cQ)$. 
\end{rmk}

For any map $p : X \to S$
Then let $\E_X = p^* \E$ and there is an isomorphism,
\[ \ol{\th}(V) : \E_X \ot \Th(V) \to \E_X(m)[2m] \]

\subsection{Cobordism}

Consider
\[ B\GL_n \cong \colim_{n \to \infty} \Gr_k(\A^N) \]
where the map is stabilization $\gamma_N \oplus \A^r \to \gamma_{N+r}$. Therefore,
\[ (\P^1)^{\wedge r} \wedge \Th(\gamma_m) \iso \Th(\gamma_m \oplus \A^r) \to \Th(\gamma_{m+r}) \]
Therefore we get a map
\[ \Sigma^\infty \Th(\gamma_r) \to \Sigma^\infty \Th(\gamma_{m+r})(r)[-2n] \]
and therefore we can form a colimit
\[ \MGL_S = \colim_m \Sigma^\infty \Th(\gamma_m)(-m)[-2m] \]
under $\gamma_{m} \times \gamma_{r} \to \gamma_{m + r}$. We could also consider the map
\[ J : K_0(S) \to \Pic{SH(S)} \]
given by $[V] \mapsto \Th(V)$. This is the Voevodsky algebraic cobordism spectrum. 

\begin{lemma}
There exists a canonical $\A^1$-equivalence $\Th(\gamma_1) \cong B \Gm \cong \P^\infty_S$. 
\end{lemma}


Then there is a map
\[ \Th(\gamma_1)(-1)[-2] \to \MGL_S \]
which is the same as a map
\[ c : \P^\infty_S \to \MGL_S(1)[2] \]
hence giving a canonical orientation. 

\begin{theorem}
Let $\EE$ be a motivic ring specturm, there is a bijection between
\[ \{ \text{ring maps } \varphi : \MGL_S \to \EE \} \to \{ \text{orientations of } \EE \}  \]
given by $\varphi(c_{\MGL_S}) \in \wt{\EE}^{2,1}(\P^\infty_S)$. 
\end{theorem}

Therefore, $\MGL$ is the universal oriented ring spectrum just like $\mathrm{MU}$ in topology. 

\begin{proof}
Idea: let $(\EE, c)$ be oriented then,
\begin{enumerate}
\item compute 
\[ \EE(\BGL) = \lim_n \EE(\BGL_m) \cong \lim_m \EE(S) [[c_1, c_2, \cdots, c_m]] \]
using the projective bundle formula where the $c_i$ is the chern class of the universal bundle. There is a subtitly here
\[ 0 \to \lim^1_m \EE(\BGL_m) \to \EE(\BGL) \to \lim_m \EE(\BGL_m) \to 0 \]
from the Milnor sequence
and we need to use Mitag-Leffler to show that $\lim^1 = 0$. 
\item likewise we compute 
\[ \EE(\MGL_S)  \to \lim_n \EE(\Th(\gamma_n)) \xrightarrow{\th} \lim_n \EE(\BGL_n) \th(\gamma_n) \cong \EE(\BGL) \]
\end{enumerate}
\end{proof}

\subsection{Milnor sequence}

Consider a tower of fibrations
\[ \cdots X_3 \to X_2 \to X_1 \to X_0 \]
for example a tower of similicial sets with each map a Kan fibration and $X_0$ a Kan complex. Then there is a short exact sequence
\[ 0 \to \lim^1_n \pi_{q+1}(X_i) \to \pi_q(\lim_n X_i) \to \lim_i \pi_q(X_i) \to 0 \]

\begin{proof}
Since $\lim_n$ in this case is a homotopy limit since the maps are fibrant therefore we have a pullback diagram (say in the infinity category)
\begin{center}
\begin{tikzcd}
\holim X_\bullet \arrow[r] \arrow[d] & \prod_i \mathrm{Path}(X_i) \arrow[d]
\\
\prod_i X_i \arrow[r, "(\id \, p_i)"] & \prod_i X_i \times X_i 
\end{tikzcd}
\end{center}
since the downward arrow on the right is a fibration so the left is also a fibration giving a fiber sequence
\[ \prod_i \Omega X_i \to \holim X_\bullet \to \prod_i X_i \]
since the fiber of the right over the base point $* \times *$ is exactly the product over the loop spaces. Therefore, there is a long exact sequence
\[ \cdots \to \prod_i \pi_{q+1}(X_i) \to \prod_i \pi_{q+1}(X_i) \to \pi_q(\lim X_\bullet) \to \prod_i \pi_q(X_i) \to \prod_i \pi_q(X_i) \to \cdots \]
and therefore taking the kernel and cokernel we conclude using the following sequence
\[ 0 \to \lim_i A_i \to \prod_i A_i \to \prod_i A_i \to \lim_i^1 A_i \to 0 \]
where the middle map is the product over the $\prod_i A_i \to A_j$ given by projecting to the coordinates $A_{j+1} \oplus A_j$ and considering $f_{j} - \id$ where $f_j : A_{j+1} \to A_j$ is the projection.
\end{proof}

\newcommand{\KGL}{\mathrm{KGL}}

\begin{example}
$\beta \in \KGL^{2,1}(S)$ defines the map
\[ \KGL \to \KGL(1)[2] \]
Bott periodicity. Since this is a $\KGL$-module map, this is defined by where $1$ maps to which is $\beta$. Then the formal group law of $\KGL_S$ is 
\[ F(x,y) = x + y - \beta^{-1} xy \]
\end{example}

\section{Deglise II Problems}

\subsection{The Lazard Ring}

\newcommand{\FGL}{\mathrm{FGL}}
Consider the functor
\[ R \mapsto \FGL_2(R) \]
sending a ring to formal group laws in two variables over $R$. We want to show this is representable. 
Consider
\[ \LL = \Z[c_{i,j}]_{i,j \ge 0} / I \]
where $I$ is some yet to be determined ideal. Then we think of a map
\[ \varphi : \LL \to R \]
as a formal power series
\[ F(x,y) = \sum_{i,j \ge 0} \varphi(c_{i,j}) x^i y^j \]
We need to satisfy
\begin{enumerate}
\item $F(x,y) = x + y + O(x^2, xy, y^2)$ so we must have $c_{0,0} = 0$ and $c_{1,0} = c_{0,1} = 1$ 
\item $F(x,y) = F(y,x)$ so $c_{i,j} = c_{j,i}$
\item $F(x, F(y,z)) = F(F(x,y), z)$ therefore we have the following universal relation
\[ \sum_{i,j} c_{i,j} x^i \left( \sum_{k,\ell} c_{k,\ell} y^k z^\ell \right)^j = \sum_{i,j} c_{i,j} \left( \sum_{k,\ell} c_{k,\ell} x^k y^{\ell} \right)^i z^j \]
which gives an infinite number of relations generating $I$.
\end{enumerate}

Note that $\LL$ is graded with $\deg{c_{i,j}} = i + j - 1$ this is consistent with $c_{1,0} = c_{0,1} = 1$ and the associativity relations. 

\subsubsection{2}

Suppose $R$ is a $\Q$-algebra and let $F \in R[[x,y]]$ be a formal group law. We need to find $f \in R[[t]]$ such that
\[ f(x + y) = F(f(x),f(y)) \]
Write
\[ f(t) = \sum a_i t^i \]
Work in the ring $R[\epsilon]/(\epsilon^n)$ then,
\[ f(t + \epsilon) = F(f(t), f(\epsilon)) \]
and therefore

\subsubsection{4}

Consider the additive formal group $x + y \in \ZZ[[x,y]]$. We want to find a map
\[ \varphi : \LL \to \ZZ[b_1, b_2, \dots] \]
this means a formal group law on $\ZZ[b_1, b_2, \cdots]$. There is a universal power series
\[ f \in \ZZ[b_1, b_2, \dots][[t]] \]
given by
\[ f = t(1 + \sum b_i t^i) \]
Consider
\[ F(x,y) = f^{-1}(f(x) + f(y)) \]
where $f^{-1}(t)$ is the inverse for composition i.e. $f^{-1}(f(t)) = t$. This is a formal group law. We claim this is graded, indeed we need to show that the term $f^{-1}(f(x) + f(y))$ of $x^i y^j$ has degree $i + j - 1$.

\subsubsection{5}

Let $I \subset \LL$ be the positive degree elements and $J \subset \Z[b_1, b_2, \dots]$ be the maximal ideal $J = (b_1, b_2, \dots)$ also the ideal generated by positive degree elements. Then Lurie proves
\[ \varphi : (I/I^2)_n \embed (J/J^2)_n \cong \Z \]
such that the image is $p \Z$ if $n + 1 = p^m$ for some $m$ and otherwise is an isomorphism. 
Then we construct
\[ \psi : \ZZ[t_1, \dots, t_n, \dots] \to \LL \]
by sending $t_1$ to a lift in $I_n$ of a generator of $(I/I^2)_n$ meaning to $p$ under the above injection. This $\psi$ is graded by definition.
\bigskip\\
First we need to show that $\psi$ is surjective. Indeed, it is clear that $\psi$ surjects onto $\LL / I^2$. The fact that it is graded means it factors through the map
\[ \ZZ[t_1, \dots, t_n, \dots] \to \bigoplus_i \nSym{i}{I/I^2} \to \LL \]


 Furthermore, the composition
\[ \varphi \circ \psi \]
is an injection since it sends $t_i \mapsto p_i b_i$ in $(J/J^2)_n$. Therefore, the map
\[  \ZZ[t_1, \dots, t_n, \dots] \to \LL \to \ZZ[b_1, b_2, \dots] \to \mathrm{gr}_I \ZZ[b_1, b_2, \dots]  \]
is injective because it is injective on tangent spaces and is a map of polynomial rings. 

\section{Some Notes}

\begin{enumerate}
\item Where does the cofiber sequence
\[ \A^2 \sm \{ 0 \} \to \P^1 \to \P^2 \]
come from? I don't know but I can write down a \textit{fiber} sequence
\[ \A^2 \sm \{ 0 \} \to \P^1 \to \P^\infty \]
which gives the same result. Indeed consider a fiber sequence
\[ X \to Y \to Z \]
meaning $X$ is the homotopy pullback of $* \leftarrow Y \to Z$ then we have a diagram,
\begin{center}
\begin{tikzcd}
X \arrow[r] \pullback & Y \arrow[r] & * \arrow[d] & 
\\
* \arrow[r] & Z \arrow[r] & B X
\end{tikzcd}
\end{center}
since the total square is a pullback we see that the right square is also a pullback. 
Applying this to the fibration sequence
\[ \Gm \to \A^2 \sm \{ 0 \} \to \P^1 \]
we get a fiber sequence
\[ \A^2 \sm \{ 0 \} \to \P^1 \to \P^\infty \]
We can then apply the exactness of $\EE$ and the orientation to show that $\EE(\P^\infty) \to \EE(\P^1)$ admits a section and thus from the long exact sequence $\eta \ot \EE = 0$.

\item I am not sure the above works because we usually work with cofiber sequences for cohomology theories since $\Sigma^\infty$ is a left adjoint and hence preserves colimits but not always limits. Furthermore, cohomology is about mapping out so (into the spectrum) so we want colimits which via hom are sent to limits.

\item how can we understand why
\[ \Pic{X} \iso [X, \P^\infty] \]
\end{enumerate}

\section{Algebraic $K$-Theory}

\newcommand{\perf}{\mathrm{perf}}

Let $X$ be a qcqs scheme. Let $D(X)$ be the derived category of quasi-coherent sheaves on $X$. If $X = \Spec{R}$ this is the derived category of $R$-modules. Now $D(X)$ is compactly generated, the subcategory of compact objects $D(X)^\omega = D_{\text{perf}}(X)$ generates $D(X)$ under colimits.  This gives a compactly generated stable $\infty$-category. 
\bigskip\\
For any compactly generated stable $\infty$-category there is a $K$-theory spectrum. 
\bigskip\\
Let $X = \Spec{R}$ then consider finite projective modules with isomorphisms 
\[ \left( \text{Proj}_R^{\simeq} \right)^{\infty\text{-group}} = K(R) \]
is the $K$-theory spectrum. But the issue is that $\infty$-group completion is complicated but the $K$-theory of a stable $\infty$-category will be more explicit. 
\bigskip\\
Let $V$ be a finite projective $R$-module then $-[V] \in K_0(R)$ is not representable by an $R$-module. However, in $D_{\perf}(R)$ this is true. If $E$ is a perfect complex, it has a class $[E] \in K_0(R)$ and $[E[1]] + [E] = 0$. Therefore, $D_{\perf}(R)$ is closer to being a group. 


\subsection{Notes}

Recall that $\Delta^n$ the simplex is NOT a Kan complex. In fact, rarely are ``finite'' simplicial sets Kan complexes. Recall that associated to a model structure to get the homotopy category we have to take maps between fibrant cofibrant replacements. Indeed if we compute maps $\Delta^1 / \partial \Delta^1 \to \Delta^1 / \partial \Delta^1$ they are all trivial. We need to take the Kan replacement frst. 


\section{Gille}

\subsection{Part 1}

Let $R$ be dedekind and $R_0 = R_f$ then $\Spec{R} \sm \Spec{R_0}$ is a finite set $\Sigma = \{ p_1, \dots, p_r \}$. Let $K = \Frac{R}$ then each $p_i$ defines a discrete valuation
\[ v_i : K^\times \to \Z \]
Let $G / R$ be a afine flat group scheme. We let $H = H_{\fppf}$ for notational convenience. Then we can define
\[ c : \ker{ \left( H^1(R, G) \to H^1(R_0, G) \times \prod_i H^1(R_{v_i}, G) \right) } \to c_\Sigma(R, G) = G(R_0) \backslash \prod_i [G(K_{v_i}) / G(R_{v_i})] \]
as follows. Take a trivialization $\sigma : T_{R_0} \iso G_{R_0}$ over $R_0$ and trivializations $\sigma_i : T_{R_{v_i}} \iso R_{v_i}$ over the points of $\Sigma$. Then the difference makes sense over the point $\Spec{K_{v_i}} \to \Spec{R}$ so we can compose these to get elements $\sigma_i \circ \sigma^{-1} \in G(K_{v_i})$ and changing each $\sigma_i$ multiplies by an element of $G(R_{v_i})$ while changing $\sigma$ multiplies all of them by an element of $G(R_0)$ so we see the class in $c_{\Sigma}(R, G)$ is well-defined.

\begin{lemma}
The map $c$ is injective.
\end{lemma}

\begin{proof}
If $c(T) = c(T')$ in $c_{\Sigma}(R, G)$ then this means we can choose trivializations $\sigma : T_{R_0} \to G_{R_0}$ and $\sigma_i : T_{R_{v_i}} \to G_{R_{v_i}}$ and likewse for $T'$ such that $\sigma' \circ \sigma^{-1}_i = \sigma' \circ \sigma_i'^{-1}$ agree in $G(K_{v_i})$ for each $i$. This gives us maps $T \to T'$ over $R_0$ and $\Sigma$ that ``agree on the overlaps''. It turns out that the cover 
\[ \Spec{R_0} \sqcup \Spec{R_{v_i}} \to \Spec{R} \]
is an fpqc cover, Indeed, any affine $U \subset \Spec{R}$ either does not meet $\Sigma$ in which case it is covered by an affine of $\Spec{R_0}$ or it does meet $\Sigma$ in which case its image is a union of the affine $U \cap \Spec{R_0}$ and the finite number of $\Spec{R_{v_i}}$ that it hits. This shows that $\Sigma$ being finite is very important otherwise the cover is not fpqc. Since we have fpqc descent we see that these glue to an isomorphism $T \to T'$.
\end{proof}

\begin{theorem}
$c$ is surjective: patching. 
\end{theorem}

If $G(R_0)$ is dense in $\prod_i G(K_{v_i}) / G(R_{v_i})$ (using the topology induced from $G \embed \GL_n$ and the nonarchemedian topology on $K_{v_i}$ inherited to $\GL_n(K_{v_i})$) then because $G(R_{v_i}) \subset G(K_{v_i})$ is open the quotient is discrete so any dense set is everything. Hence we get

\begin{cor}
If $G(R_0)$ is dense in  $\prod_i G(K_{v_i}) / G(R_{v_i})$ (e.g. if it is dense in $\prod_i G(K_{v_i})$) then 
\[  H^1(R, G) \embed H^1(R_0, G) \times \prod_i H^1(R_{v_i}, G) \]
is injective.
\end{cor}

\subsection{Strong Approximation}


\begin{prop}
Let $G$ be a split chevallay group scheme meaning it has a pair $(T, B)$ of a Borel and a split torus $T \subset B$ e.g. $\GL_n, \PGL_n, \Sp_{2n}, \SL_n$. Then $B = U \rtimes T$ with $U$ the unipotent radical.
\begin{enumerate}
\item there is a surjective map
\[ H^1(R, T) \onto \ker{ \left( H^1(R,G) \to H^1(K, G) \right) } \]

\item if $G$ is semisimple and simply connected then the target of the above map is trivial meaning
\[  H^1(R,G) \embed H^1(K, G) \]
is injective.
\end{enumerate}
\end{prop}

\begin{rmk}
Rreductive over a field means affine smooth connected with trivial geometric unipotent radical. Over a scheme, reductive means smooth affine group that is reductive on fibers.
\bigskip\\
Over a field, semisimple means reductive and perfect meaning $D G = G$ is its own derived subgroup. Over a scheme we mean reductive and semisimple on fibers.
\bigskip\\
Simply connected over $\CC$ means its analytification is simply connected. In general we need to look at the root lattice of the fibers and say the root lattice equals the root lattice does this mean no nontrivial isogenies to $G$?
\end{rmk}

\begin{proof}
For the first, let $E$ be a $G$-torsor such that $E(K) \neq \empty$ so trivialized at the generic point. Let $G/B$ be the flag variety which is a projective $R$-scheme. Then we can twist to get $E/B = E \times_G (G/B)$ which is a projective form of $(G/B)$ over $R$. By the valuative criterion of properness, $E/B$ has an $R$-point iff it has a $K$-point. However, it has a $K$-point because $E_K \cong G_K$ so $(E/B)_K \cong (G/B)_K$ which has a rational point (namely the image of the identity). Therefore, $E$ arises as the base change of a $B$-torsor. Now $U$ admits a $T$-equivariant filtration by copies of $\Ga^{r_i}$ therefore
\[ H^1(R, T) \iso H^1(R, B) \]
because $H^1(R, \Ga) = H^1(R, \struct{R}) = 0$ since $R$ is affine. Therefore we see that $E$ arises from a $T$-torsor proving the claim. 
\bigskip\\
For the second, we use strong approximation. Let $B^{-} \supset T$ be the opposite Borel. Then $B ^{-} = U^{-} \rtimes T$. Note that $G(K_{v_i})$ is generated by $U(K_{v_i})$ and $U^{-}(K_{v_i})$. Therefore, we reduce the density of $G(R_f)$ in $\prod_i G(K_{v_i})$ to that of $U$ but this reduces to $\Ga$ and hence the well-known fact that $R_f \subset \prod_i G(K_{v_i})$ is dense. 
\bigskip\\
Now we prove the second. Suppose we have $E$ such that $E_K$ is trivial. From (a) we know $E$ arises from a $T \cong \Gm^r$-torsor and hence is Zariski locally trivial hence there is some $R_0 = R_f$ on which it is trivial and let $\Sigma$ be the complement. Since a DVR has trivial Picard group we see that $E$ is trivial over $R_0$ and each $R_{v_i}$ so it suffices to show that the set $c_{\Sigma}(R, G)$ is trivial since by our previous results this shows $E$ is trivial. Now strong approximation shows that $R_f \subset \prod_i G(K_{v_i})$ is dense and hence by a previous remark this means $c_{\Sigma}(R, G)$ is trivial. 
\end{proof}


\begin{prop}
Let $k$ be algebraically closed and $C/k$ a smooth affine curve and $G/k$ a \textit{semisimple split} algebraic group over $k$. Then
\[ H^1_{\fppf}(C, G) = * \]
\end{prop}

Our previous results show we can understand torsors that are trivial over $K$ very well. Therefore we want to show that $G_K$-torsors are all trivial. We can do this by Tens's theorem that says $K = k(C)$ is a $C_1$-field. Therefore, it has cohomological dimension $1$. We need a bit more than just vanishing of Galois cohomology however. We use the following theorem of Steinberg:

\begin{theorem}[Steinberg]
Let $F$ a field and $H / F$ quasi-split semisimple group scheme. Then there is a surjeciton
\[ \sqcup_{T \subset H} H^1(F, T) \onto H^1(F, H) \]
for all maximal tori $T \subset H$. 
\end{theorem}

\begin{rmk}
We need \textit{all} maximal tori in the above not just the split tori otherwise the left hand side would be automatically zero. The point however, is that  the cohomology of a nonsplit torus over a field just reduces to Galois cohomology.
\end{rmk}

This finishes the claim because if $F$ has cohomological dimension $\le 1$ then the cohomology of any torus is zero. Indeed, if $T$ is split by a field of degree $n$ then pull-push shows that $H^1(F, T)$ is $n$-torsion since it is zero once the torus becomes trivial. Then we use the sequence
\[ 1 \to T[n] \to T \xrightarrow{n} T \to 1 \]
and since the multiplication map is zero on $H^1(F, T)$ we get an injection,
\[ H^1(F, T) \embed H^2(F, T[n]) \] 
and $H^2(F, T[n]) = 0$ because of the assumption on cohomological dimension. 

\begin{proof}
If $G$ is semisimple, part (2) of the previous theorem showed that it suffices to show that every torsor is trivial over $K$. We showed why $G_K$ torsors are trivial in the above discussion. Let's first reduce to this case.
Let $f : G^{sc} \to G$ be the simply connected cover of $G$ and $\mu = \ker{f}$. Let $T^{sc}$ be a maximal torus of $G^{sc}$ and $T = T^{sc} / \mu$ is then a maximal torus of $G$. Consider the diagram
\begin{center}
\begin{tikzcd}
H^1(C, T^{sc}) \arrow[d] \arrow[r] & H^1(C, T) \arrow[d, two heads]
\\
H^1(C, G^{sc}) \arrow[r] & H^1(C, G)
\end{tikzcd}
\end{center}
but we just showed that $H^1(C, G^{sc})$ is trivial and part (1) of the previous theorem said that $H^1(C, T) \onto H^1(C, G)$. Therefore, it suffices to show that the top map is surjecitve. Since $G$ was split, we can choose $T$ to be a split torus then $f$ corresponds to a map on cocharacters $\Z^r \to \Z^r$ with determinant nonzero (i.e. in $\Q^\times$ but not in $\Z^\times$ unless $\mu = 1$) and therefore the top map is isomorphic to
\[ \Pic{C}^r \to \Pic{C}^r \]
for some matrix of maps with nonzero integer determinant. Since $\Pic{C}$ is divisible for an affine curve over $k = \bar{k}$ we see that the map is surjective so we win. 
\end{proof}

\subsection{Part 2}

From last time we can prove.

\begin{theorem}
Let $G / k$ be split reductive over $k = \bar{k}$. Then,
$H^1(\A^1_{k}, G) = *$. 
\end{theorem} 

\begin{proof}
Let $T \subset G$ be a maximal torus. Then we showed last time
\[ H^1(\A^1_k, T) \onto H^1(\A^1_k, G) \]
is surjective. Since $k$ is algebraically closed $T$ is split and $\Pic{\A^1} = 0$ so we conclude that $H^1(\A^1_{\bar{k}}, G) = *$.
\end{proof}


\begin{proof}

\end{proof}



\begin{theorem}
Let $G / k$ be a reductive group. Then
\[ \ker{\left( H^1(\A^1_k, G) \to H^1(\A^1_{k^{\sep}}, G) \right) } \]
is trivial.
\end{theorem}

\begin{cor}
If $k$ is perfect and $G / k$ is reducitive then $H^1(\A^1_k, G) = 0$. 
\end{cor}

\begin{rmk}
There are examples for $k$ imperfect and $a \in k \sm k^p$ then for $G = \PGL_p$ there is an azumaya algebra on $\A^1_k$ which is nontrivial it is given by
\[ x^p = x + t \quad y^p = a \quad y x y^{-1} = x + 1 \]
and the Galois action is
\[ x \mapsto x + 1 \]
This is nontrivial in $H^1(\A^1_k, \PGL_p)$ and even in $H^1(\A^1_{k^{\sep}} , \PGL_p)$. 
\end{rmk}

\subsection{Todo}

\begin{enumerate}
\item 
Ask these questions to Markus Spitzweck. 

\item What was he saying about stacks with charts that are Nisnevich locally split. 

\item Ask for a reference for where the Bott map is written down.

\item Ask about removing the $\Sigma$ in the argument we sketched and show Marc this Hilb paper where they don't seem to be able to get rid of it.
\end{enumerate}

Consider on $\P^1 \times \P^1$ the resolution of the diagonal
\[ 0 \to \struct{}(-1,-1) \to \struct{} \to \struct{\Delta} \to 0 \]
Then for any vector bundle $\E$ on $\P^1_S$ over any base scheme $S$ we can consider
\[ \RR \pi_{1*} (\struct{\Delta} \ot^{\LL} \LL \pi_2^* \E) = \E \]
since it is the pullback to $\Delta$ composed with the isomorphism $\pi_1 : \Delta \to \P^1$. Therefore, 
\[ \E \cong \RR \pi_{1*} [\struct{}(-1,-1) \ot \pi_2^* \E \to \pi_2^* \E] \] 
Now let's assume that $\E$ and $\E(-1)$ have no higher cohomology. Then the spectral sequence,
\[ E^{p,q}_1 = \RR \pi_{1*}^q \K^p \implies \RR \pi_{1*}^{p+q} \K^\bullet \]
degenerates at $E_2$ and shows that there is an exact sequence,
\[ 0 \to \pi_{1*} (\struct{}(-1,-1) \ot \pi_2^* \E) \to \pi_{1*} \pi_{2}^* \E \to \E \to 0 \]
However, since $s : \P^1_S \to S$ is flat we have, by flat base change
\[ \RR^p \pi_{1*} \pi_{2}^* \E = s^* \RR^p s_* \E \]
and therefore by the projection formula there is an exact sequence,
\[ 0 \to s^* s_*(\E(-1))(-1) \to s^* s_* \E \to \E \to 0 \]

\section{Real Pointed Motivic Eilenberg-MacLane spaces}

There is a real realization:
\begin{center}
\begin{tikzcd}
\Sm_{\RR} \arrow[r] \arrow[rd, "X \mapsto X(\RR)"] & \Spc_{\RR} \arrow[d, dahsed, "r_{\RR}"]
\\
& \Spc
\end{tikzcd}
\end{center}

There are motivic Eilenberg-MacLane spaces $H \ZZ \in \SH(k)$ and we let
\[ K(\Z(i), j) = \Omega^\infty \Sigma^{j,i} H \Z \]

Main question what is
\[ r_{\RR}(K(\Z(i), j)) \]

\begin{rmk}
$r_{C_2} K(\Z(i), j) \cong K(\Z, S^{j,i})$ where there are $i$ many sign reps and $j-i$ trivial reps. Then we need
\[ r_{\RR}(-) = r_{C_2}(-)^{C_2} \]
for homotopy fixed points. 
\end{rmk}

\subsection{Orientation}

\subsubsection{Realizations} $r$ preserves finite products and colimits and therefore it preserves commutative monoids. Therefore we get
\[ \mathrm{CMon}(\Spc_{\RR}) \to \mathrm{CMon}(\Spc) \]
Therefore,
\[ r_{\RR}(K(\Z(i), j+1)) = r_{\RR} BK(\Z(i), j) = B r_{\RR}(K(\Z(i), j)) \]
so we can do this for $j = 1$ and just need to consider $i$. 

\begin{example}
We know $K(\Z(0), 0) = \ul{\Z}$ so we get
\[ r_{\RR}(K(\Z(0), 0)) = K(\Z, 0) \]
Consider
\[ K(\Z(1), 1) = \Gm \]
therefore 
\[ r_{\RR}(K(\Z(1), 1)) = r_{\RR}(\Gm) = \RR^\times \cong S^0 \cong \Z / 2 \Z \in \mathrm{CMon} \]
And thus,
\[ r_{\RR}(K(1), 2)) = B \Z / 2 = \RP^\infty \]
This is because $K(\Z(1), 2) = \P^\infty$ so it makes sense. 
\end{example}

\subsubsection{$\rho$-realization}

$r_{\RR} : \SH(\RR) \to \Sp$ lands in spectra. We can sometimes understand what this real realization of spectra does. 

\begin{theorem}[B]
Let $E \in \SH(\RR)$ be a motivic spectrum. Then
\[ \pi_* r_{\RR}(E) = (\pi_{*,*} E(\RR) [\rho^{-1}])_{*,0} \]
where $\rho : S^0 \to \Gm$ given by $\{ \pm 1 \} \subset \Gm$. 
\end{theorem}
Unfortunately we dont know $\pi_{*,*} E(\RR)$ for $H \ZZ$ this is related to the algebraic $K$-theory of $\RR$. 

\begin{theorem}[Voevodsky]
$\pi_{*,*} (H \Z / 2)(\RR) = \Z/2 [\rho, \tau]$
\end{theorem}

Therefore
\[ \pi_* r_{\RR} H \Z / 2 = (\Z / 2[\rho, \rho^{-1}, \tau])_{*,0} = \Z / 2 [t] \]
where $t = \rho^{-1} \tau$ has degree $1$. 
\bigskip\\
Some magic, $h \in \pi_{0,0} \S$ with the property that $H \Z / h = H \Z/2$ and $\rho^2 h = 0$. Consider the cofiber sequence
\[ H \Z \xrightarrow{h} H \Z \to H \Z / 2 \]
and now we invert $\rho$ to get
\[ H \Z[\rho^{-1}] \xrightarrow{0} H \Z[\rho^{-1}] \to H \Z / 2 [\rho^{-1} \]
and therefore
\[ H \Z / 2[\rho^{-1}] = H \Z [\rho^{-1} \oplus \Sigma H \Z[\rhi^{-1} \]
thus we get
\[ \pi_{*} r_{\RR} H \Z = \Z / 2 [t^2] \]

\subsubsection{Rational Part}

Recall there is a motivic algebraic $K$-theory spectrum $\KGL \in \SH(k)$ with 
\[ \Omega^\infty \KGL = \Z \times \BGL \]
and when you rationalize you get
\[ \KGL_{\Q} \cong \bigoplus_{n \in \Z} \Sigma^{2nmn} H \Q \]
by the Chern character. Therefore (since $\Omega^\infty$ is a right adjoint it preserves infinite products)
\[ \Q \times \BGL_{\Q} \cong \Omega^\infty \KGL_{\Q} \cong \prod_{n \ge 0} K(\Q(n), 2n) \] 
and 
\[ r_{\RR}(\BGL) = \B r_{\RR}(\GL) = \B \GL(\RR) \cong BO \]
therefore
\[ \prod_{n \ge 0} r_{\RR} K(\Q(n), 2n) = r_{\RR}(\BGL_{\Q} \times \Q)  = BO_{\QQ} \times \Q \cong \prod_{n \ge 0} K(\Q, 4n) \]
so maybe what happens is 
\[ r_{\RR} K(Q(n), 2n) = 
\begin{center}
0 & n \text{ odd}
\\
K(\Q, 2n) & n \text{ even}
\end{center} \] 

\subsection{Main result}

We need to use the ring spectrum structure 
\[ \Sigma^{i,i} H \Z \ot \Sigma^{j,j} H \Z \to \Sigma^{i+j,i+j} H \Z \]
We let $K_n := K(\Z(n),n)$ out of this ring structure we get 
\[ K_i \wedge K_j \to K_{i+j} \]
Then $\pi_* K_\bullet$ is a bigraded ring where addition respects $*$ and multiplication respects $\bullet$. There is a canonical element $\rho \in \pi_0 r_{\RR}(K_1)$ we need a map
\[ S^0 \to K_1 \]
given by the map
\[ \rho : \SS \to \Sigma^{1,1} H \Z \]

\begin{theorem}
there is an element $t^2 \in \pi_2 r_{\RR}(K_2)$ such that
\[ \pi_* r_{\RR} K_\bullet \cong \Z[\rho, t^2] / (2 \rho) \]
\end{theorem}

\subsection{Notes}

Basically the reason that $r_{\CC} K(\Z(i), j) = K(Z, j)$ topologically is that $r_{\CC} (\nSym{n}{X}) = \nSym{n}{X(\CC)}$ for any smooth variety this is just because quotients of schemes on algebraically closed field points are idenified with the quotient of sets of points. However, this fails over $\RR$. 
\bigskip\\
Why are the EM spectra related to symmetric powers? It arises from Voevodsky's construction of $K(A, n)$ via Dold-Kan as $\Gamma(A[n])$ where $A[n]$ is the chain complex with $A$ in degree $n$.

\subsection{Note}

\[ \pi_i^{\A^1} K(M,n) = \begin{cases}
M & i = n
\\
0 & i \neq n
\end{cases} \]

\end{document}


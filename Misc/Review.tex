\documentclass[12pt]{article}
\usepackage{import}
\import{"../Algebraic Geometry/"}{AlgGeoCommands}

\newcommand{\Loc}[1]{\mathfrak{Loc}\left( #1 \right)}
\newcommand{\AbGrp}{\mathbf{AbGrp}}

\renewcommand{\K}{\mathbb{K}}

\newcommand{\inner}[2]{\left< #1, #2 \right>}

\newcommand{\B}{\mathcal{B}}
\newcommand{\R}{\mathbb{R}}

\newcommand\eqae{\mathrel{\stackrel{\makebox[0pt]{\mbox{\normalfont\tiny a.e.}}}{=}}}
\renewcommand{\F}{\mathcal{F}}
\renewcommand{\K}{\mathcal{K}}

\begin{document}

\tableofcontents

\section{Groups of Lie Type}

\section{Galois Groups of Cubics}

\section{Products of Ideals}

\begin{lemma}
Let $I, J \subset R$ be ideals. Then,
\[ V(IJ) = V(I \cap J) = V(I) \cup V(J) \]
\end{lemma}

\begin{proof}
If $I \subset \p$ then $\p \supset I \cap J \subset IJ$ so it is clear that,
\[ V(I) \cup V(J) \subset V(I \cap J) \subset V(IJ) \]
Thus suppose that $\p \supset IJ$ but $\p \notin V(I) \cup V(J)$. Then there is $x \in I$ and $y \in J$ such that $x, y \notin \p$ so that $\p \not\supset I$ and $\p \not \supset J$. Then $x y \in IJ \subset \p$ so $x y \in \p$ contradicting the primality of $\p$ and proving the claim.
\end{proof}

\begin{prop}
Let $R$ be a comutative ring and $I, J \subset R$ are ideals.
If any of the following are true,
\begin{enumerate}
\item $I + J = R$
\item $\nilrad{R / IJ} = (0)$
\end{enumerate}
then $I \cap J = IJ$.
\end{prop}

\begin{proof}
If $I + J = R$ then for any $r \in I \cap J$ consider $1 = x + y$ with $x \in I$ and $y \in J$ and $r = r x + ry \in IJ$ so $I \cap J \subset IJ \subset I \cap J$ proving equality. 
\bigskip\\
Now suppose that $\nilrad{R / IJ} = (0)$. Consider the ideal $(I \cap J)/IJ \subset R / IJ$. I claim that it is contained in the nilradical. Indeed, for any prime $\p$ of $R / IJ$, that is a prime of $R$ above $IJ$ because $V(IJ) = V(I \cap J)$ and thus $(I \cap J)/IJ \subset \nilrad{R / IJ}$ so $I \cap J = IJ$.
\end{proof}

\end{document}
\documentclass[12pt]{article}
\usepackage{hyperref}
\hypersetup{
    colorlinks=true,
    linkcolor=blue,
    filecolor=magenta,      
    urlcolor=blue,
}

\usepackage{import}
\import{../}{AlgGeoCommands}

\begin{document}

\section{Rigidifying the space of stable maps}

quasi-stable curves: 

\begin{defn}
A \textit{stable} map $(\mu : C \to X, p_1, \dots, p_n)$ with $p_i \in C$ is a map from a quasi-stable curve such that
\begin{enumerate}
\item if $E \subset C$ is a genus zero component and $\mu(E)$ is a point then $E$ contains at least three special (marked or singular) points
\item if $E \subset C$ is a genus one component and $\mu(E)$ is a point then $E$ contains at least one special point. 
\end{enumerate}
\end{defn}

Coarse moduli of stable maps should be a quotient of a locus in $\Hilb(\P(W) \times \P^r)$.

Want a canonical way to rigidify the problem (meaning removing the automorphisms of stable maps). 

Idea: consider $C \to \P^r$ and pullback hyperplanes and these impose additional marked points. 
\bigskip\\
Let $\P^r = \P(V)$ then $V^* = H^0(\P^r, \struct{}(1))$ and let $\bar{t} = (t_0, \dots, t_r)$ be \textit{any} basis for $V^*$. 

\begin{defn}
A $\bar{t}$-stable rigid family of degree $d$ maps is a $(\pi : \C \to S, \{ \sigma_i \}_{0 \le i \le n}, \{ \tau_{i,j} \}_{\substack{0 \le i \le r \\ 0 \le j \le d}}, \mu: \C \to \P^r )$ such that
\begin{enumerate}
\item $\pi : \C \to S$ is a quasi-stable curve
\item $(\pi, \{ \sigma_i \}, \mu)$ is a stable family
\item $(\pi, \{ \sigma_i \}, \{ \tau_{i,j} \})$ is a family of pointed stable curves (meaning the $\tau_{i,j}$ are distinct points missing the already special points of $(\pi, \{ \sigma_i \})$ 
\item for any $0 \le i \le r$ we have $\div(\mu^* (t_i)) = [q_{i,1}] + \cdots + [q_{i, d}]$ as divisors 
\end{enumerate}
\end{defn}

\begin{rmk}
Given a stable family and a choice of $\bar{t}$ which is ``good'' meaning that the intersection of the $V(t_i)$ with $\C_t$ are all sets of $d$ distinct points, we don't quite get a $\bar{t}$-rigid stable family. Indeed, we need to choose an ordering of the $q_{i,d}$ which may require taking an $S_d$-cover of $S$. 
\end{rmk}

\subsection{Special Cases}

\newcommand{\barM}{\ol{\mathcal{M}}}

\begin{enumerate}
\item 
For $r = 0$ we get stable curves with no map
\item $d = 0$ the coarse space if $\barM_{g,n} \times \P^r$ 
\item $r = d = 1$ and $g = n = 0$ then there aren't enough points to make it stable, we will ignore this case. 
\end{enumerate}

\begin{defn}
Consider the functor:
$\barM_{g,n}(\P^r, d, \bar{t})(S)$ is the set of isomorphism classes of stable families over $S$. 
\end{defn}

\begin{theorem}
Let $d > 0$ and $r > 0$.
There exists a quasi-projective scheme $\ol{M}_{g,n}(\P^r, d, \bar{t})$ which is a coarse space for $\barM_{g,n}(\P^r, d, \bar{t})$. If $g = 0$ then additionally this coarse space represents the functor.  
\end{theorem}

\begin{proof}
Build the scheme explicitly such that $(\C, \sigma_i, \tau_{i,}, \mu)$ exists universally. Combining the marked points yields $S \to \ol{M}_{0,m}$ where $m = n + d(r+1)$. This map is not enough however. Suppose $S = \Spec{k}$ and $k = \bar{k}$. The map $C \to \P^r$ is not quite determined by the cartier divisors $\div(\mu^*(t_i))$, it is only determined up to a torus action. 
\end{proof}

\begin{defn}
Let $\H_1 = \struct{\U_{0,m}}(q_{i,1} + \cdots + q_{i,d})$ for the points on $\barM_{0,m}$ ordered as the first $n$ are called $p_1, \dots, p_n$ and the rest are called $q_{i,j}$. 
Suppose the diagram
\begin{center}
\begin{tikzcd}
\C \arrow[r, "\bar{\gamma}"] \arrow[d] \pullback & \U_{0,n} \arrow[d, "\pi"]
\\
X \arrow[r, "\gamma"] & \ol{M}_{0,m}
\end{tikzcd}
\end{center}
is Cartesian. We say $\gamma$ is $\H$-\textit{balanced} if for all $i$ we have,
\begin{enumerate}
\item $(\pi_X)_* \bar{\gamma}^* (\H_i \ot \H_0^{-1})$ is locally free
\item $(\pi_X)^* (\pi_X)_*  \bar{\gamma}^* (\H_i \ot \H_0^{-1}) \iso  \bar{\gamma}^* (\H_i \ot \H_0^{-1})$ is an isomorphism. 
\end{enumerate}
\end{defn}

\begin{rmk}
This is the right condition because in our case $\H_i$ will just be $\struct{}(1)$ because all Cartier divisors $q_{i,1} + \cdots + q_{i,d}$ are hyperplane sections. Therefore, any two $\bar{\gamma}^* \H_i$ are actually isomorphic in our case of interest. 
\end{rmk}

Fact: there exists a locally closed subscheme $\iota: B \embed \ol{M}_{0,m}$ which is universal for $\H$-balanced maps. Let $\G_i = (\pi_B)_* \bar{\iota} (\H_i \ot \H_0^{-1})$ a line bundle let $\tau_i : Y_i \to B$ be the $\Gm$-bundle corresponding to $\G_i$ and $Y_i$ represents trivializations of $\G_i$. 
\bigskip\\
Let $Y = Y_1 \times_B \cdots \times_B Y_r$ and let $\U$ be the pullback of $\U_{0,n}$ along $Y \to \ol{M}_{0,m}$. We will show that $\pi_Y$ is a universal $\bar{t}$-rigid stable family. 

\begin{proof}
We construct a map $\mu : \U \to \P^r$ given by $\mu^* \struct{}(1) = \H_0$ and the sections corresponding to the Cartier divisors $\{ q_{i,1} + \cdots + q_{i,d} \}_{i}$ which are sections of $\H_i \iso \H_0$ the canonical isomorphisms existing over $Y$. 

Since $(\U \to Y, \{ p_i \}, \{ q_i \})$ is a stable curve then $(\U \to Y, \{ p_i \}, \{ q_i \}, \U \to \P^r)$ is a stable map. We need to show the map is stable dropping the $\{ q_i \}$. Indeed, the number of $q_i$ on each component is its degree so the contracted components have no $q_i$ proving the claim. 
\bigskip\\
Universality: fix $(\pi : \C \to S, \{ \sigma_i \}, \{ \tau_{i,j} \}, \mu : \C \to \P^r)$ this gives a map $\lambda : S \to \ol{M}_{0,m}$ representing the data $(\mu : \C \to S, \{ \sigma_i \}, \{ \tau_{i,j} \})$. Then $\bar{\lambda}^* \H_i \cong \mu^* \struct{}(1)$ canonically and the divisors $\div \mu^* (t_i) = [q_{i,1}] + \cdots [q_{i,d}]$ are equal meaning there is an isomorphism of bundles with sections. These isomorphism between the $\bar{\lambda}^* \H_i$ give a lift $\tilde{\lambda} : S \to Y$ where the map $\mu$ is the pullback of the universal $\U \to \P^r$. 
\end{proof}

\section{Constructing $\Mbar_{g,n}(X, \beta)$}

We focus on the case $g = 0$. Last time: for any basis $\bar{t}$ of $V^* = H^0(\P^r, \struct{}(1))$ we constructed a nonsingular quasi-projective variety $\ol{M}_{g,n}(\P^r, \beta, \bar{t})$. Today we
\begin{enumerate}
\item construct $\ol{M}_{g,h}(\P^r, \beta)$ via patching the rigid guys
\item construct $\ol{M}_[g,n}(X, \beta)$ as a closed subscheme. 
\end{enumerate}

Step 1, by permuting the $q$ we get a quotient $\ol{M}(\bar{t})/G$ where $G = (S_d)^{r+1}$ permuting the points $\{ q_{i,j} \}$ given by the intersection with the $t_j$.  Then $\ol{M}(\bar{t})/G$ is also quasi-projective. Note that this action is \textit{not} free. Indeed, consider a map $\P^1 \to \P^1$ of degree $2$ and the $\bar{t}$ is given by a point on $\P^1$ that is not a branch point. Then $G$ swaps the two points in the fiber but these are isomorphic as stable curves with marked points. 
\bigskip\\
For two bases $\bar{t}, \bar{t}'$ we define an open $\ol{M}(\bar{t}, \bar{t}') \subset \ol{M}(\bar{t})$ where the basis $\bar{t}'$ is also transverse to the curve (but we are not fixing the order of the point on the intersection with $\bar{t}'$ so $\ol{M}(\bar{t}, \bar{t}')$ is not symmetric in $\bar{t}$ and $\bar{t}'$). We can show however that
\[ \ol{M}(\bar{t}, \bar{t}')/G \cong \ol{M}(\bar{t'}, \bar{t})/G \cong \mathcal{E}(\bar{t}, \bar{t}')/(G \times G) \]
where the last space is the one were we label the divisors for both $\bar{t}$ and $\bar{t}'$. 
\bigskip\\
These satisfy the cocycle condition for these isomorphisms because of the existence and symmetry of the spaces $\mathcal{E}(\bar{t}, \bar{t}')/(G \times G)$. Therefore, we can glue to get 
\[ \ol{M}_{g,n}(\P^r, \beta) \]

\subsection{Sepratedness}

We will use the following fact. If $X$ has a cover by opens 

\subsection{Properness}

Let $X$ be a test curve (i.e. a dvr) with special closed point $x \in X$. We need to show we can fill in the diagram,
\begin{center}
\begin{tikzcd}
\C \arrow[d] \arrow[r] & \P^r
\\
X \st \{ x \} 
\end{tikzcd}
\end{center}

\section{SAGS}

Recall: $X$ is convex if for all $f : \P^1 \to X$ then $H^1(\P^1, f^* \T_X) = 0$. 


Goal: if $X$ is a smooth projective convex variety then $\ol{\M}_{0,n}(X, \beta)$ is a smooth DM stack with boundary $\partial \ol{\M}_{0,n}(X, \beta) = \ol{\M}_{0,n}(X, \beta) \sm M_{0,n}(X, \beta)$ a divisor with normal crossings.

\begin{theorem}
Let $C$ be a proper algebraic curve without embedded point and $f : C \to M$ be a map to a smooth $M$ of pure dimension $n$. Then,
\[ \dim_{[f]} \Hom{}{C}{M} \ge h^0(C, f^* \T_M) - h^1(C, f^* \T_M) = \chi(C, f^* \T_M) =  \deg{f^* \T_M) + n \chi(\struct{C}) = - K_M \cdot [C] + n \chi(\struct{C}) \]
Furthermore, if we have equality then $\Hom{}{C}{M}$ is lci cut out by $h^1$ equations in its tangent space of dimension $h^0$.
\end{theorem}

Consider the sequence,
\begin{center}
\begin{tikzcd}
0 \arrow[r] & \bigoplus_{i = 1}^n T_{p_i}(\P^1) \arrow[r] & \Def_{\text{fix}}(f) \to H^0(\P^1, f^* \T_X) \arrow[r] & 0
\end{tikzcd}
\end{center}
Therefore, 
\[ \Def(f) = \dim H^0(\P^1, f^* \T_X) + n - 3 = \dim(X) + c_1(\T_X) \cdot \beta + (N-3) \]


Recall $\partial \ol{\M}_{0,n}(*, 0) = \partial \ol{\M}_{0,n}$ is stratified by paritions of $n$ which is the locus of curves where the number of marked points on each component corresponds to the partition.

\begin{defn}
Let $A_1 \sqcup \cdots \sqcup A_n = [n]$ be a partition and $\beta_1, \dots, \beta_k \in A_1(X)$ effective curve classes with $\beta_1 + \cdots + \beta_n = \beta$. Then we define,
\[ D(A_1, \dots, A_k ; \beta_1, \dots, \beta_k) \]
to be the locus of $\mu : C \to X$ such that there exists a decomposition
\[ C = C_1 \cup \cdots \cup C_n \]
such that marks from $A_i$ lie on $C_i$ and $\mu_*([C_i]) = \beta_i$. 
\end{defn}

These are closed subsets. They are not irreducible because we do not fix the dual graph. 

Plan:

\begin{enumerate}
\item if $k = 2$ then $D(A_1, A_2 ; \beta_1, \beta_2)$ are (Weil) divisors
\item a $D(A_1, \dots, A_{k+1} ; \beta_1, \dots, \beta_{k+1})$ is an intersection of $k$ divisors of type considered above
\item tangent space to $D(A_1, \dots, A_k ; \beta_1, \dots, \beta_k)$ has the expected dimension at each generic point
\end{enumerate}


\subsection{Part (a)}

Let $\ol{\M}_{A} := \M_{0, A \cup \{ * \}}(X, \beta_1)$ where we are chosing markings from a set $A \subset [n]$. And let $\ol{\M}_{B} = \ol{\M}_{0, B \cup \{ * \}}(X, \beta_2)$. There are evaluation maps $e_A : \ol{\M}_{A} \to X$ and $e_B : \ol{\M}_B \to X$ which evaluate at $*$. We want to consider gluing two curves at a marked point which is still stable. To do this we need the marked points on each piece to go to the same point. Thus we define,
\[ \wt{\D}(A, B, \beta_1, \beta-2) = \ol{\M}_A \times_X \ol{\M}_B \]
over the maps $e_A, e_B$. These are pairs
\[ \{ \mu_A : C_A \to X, \mu_B : C_B \to X, (\mu_A)_* [C_A] = \beta_1, (\mu_B)_* [C_B] = \beta_2, \mu_A(*) = \mu_B(*) \} \]
There is a canonical map $\wt{\D}(A, B, \beta_1, \beta_2) \to D(A, B ; \beta_1, \beta_2)$. Since $X$ is convex $\mu^* \T_X$ is generated by global sections this means that,
\[ \d{e_A} : H^0(C_A, \mu^* \T_X) \to T_{\mu(*)} X \]
is surjective (at least over the interior points). Therefore, the fiber product has the correct dimension:
\begin{align*}
\dim{\wt{\D}} & = \dim{\ol{\M}_A} + \dim{\ol{\M}_B} - \dim(X) = \dim(X) + c_1(\T_X) \cdot \beta_1 + \# A - 2 + \dim(X) + c_1(\T_X) \cdot \beta_2 + \# B - 2 - \dim(X)
\\
& = \dim(X) + c_1(\T_X) \cdot \beta + n - 4 
\end{align*}
which means it could be a divisor. Notice that the map $\wt{\D} \to D$ has self intersections. Indeed, we could glue $\P^1 \cup \P^1$ and $\P^1$ to get the same thing as $\P^1$ glued together with $\P^1 \cup \P^1$. 

\subsection{Part (b)}

Let $\D(A_1, \dots, A_k ; \beta_1, \dots, \beta_k)$ be a stratum. Let $A_i^* = [n] \sm A_i$ and $\beta* = \beta - \beta_i$ then 
\[ \D(A_1, \dots, A_k ; \beta_1, \dots, \beta_k) = \bigcap_{i = 1}^n \D(A_i, A_i^* ; \beta_i, \beta_i^*) \]
If the same term appears twice in the intersection it means we are looking for the locus where a divisor self-intersects exactly because of this trouble we described earlier. 

\begin{lemma}
Let $X$ be a nonsingular projective convex variety. Let $\mu : C \to X$ be a map from a projective, connected, reduced nondal curve of arithmic genus zero. Then $H^1(C, \mu^* \T_X) = 0$ and $\mu^* \T_X$ is generated by global sections.
\end{lemma}

\begin{proof}
Let $E \subset C$ be an irreducible component. Let $\mu^* \T_X |_E = \bigoplus \struct{}(a_i)$. By convexity we must have all $a_i \ge 0$ otherwise we could compose with a squaring map and violate convexity. We are going to prove by induction that,
\[ H^1(C, \mu^* \T_X \ot \struct{C}(-p)) = 0 \]
for any smooth point $p \in C$ which means that $\mu^* \T_X$ is generated by global sections (breaking a node into two parts and applying the proof to each part gives the global generation at nodes). We checked this if $C$ is irreducible by convexity. Let $C = C' \cup E$ and $C' \cap E = \{ p_1, \dots, p_r \}$ and $p \in E$. Then consider
\begin{center}
\begin{tikzcd}
0 \arrow[r] & \mu^* \T_X |_{C'} (- \sum p_i) \arrow[r] & \mu^* \T_X(-p) \arrow[r] & \mu^* \T_X |_E (-p) \arrow[r] & 0
\end{tikzcd}
\end{center}
the first term is a sum over irreducible componets twisted down by one point so we can apply the induction hypothesis. The last term has vanishing $H^1$ by the base case so we win by the exact sequence. 
\end{proof}

Deformations of nodal curves: inside $\Def(\mu)$ for each node, there is a hyperplane of deformations that do not smooth a given node. Hence, the space $\Def_G(\mu)$ of deformations which preserve all nodes satisfies $\dim{\Def_G(\mu)) \ge \dim{\Def(\mu)} - k$. There is a sequence,
\begin{center}
\begin{tikzcd}
0 \arrow[r] & \Def_G(\mu : C \to X) \arrow[r] & \Def_G(\mu) \arrow[r] & \Def_G(C) \arrow[r] & 0
\end{tikzcd}
\end{center}
Using the automorphism we get,
\begin{center}
\begin{tikzcd}
0 \arrow[r] & H^0(C, \T_C(-\text{nodes})) \arrow[r] & \Def_{C \text{fixed}}(\mu) \arrow[r] & \Def_G(\mu : C \to X) \arrow[r] & 0
\end{tikzcd}
\end{center}
First $\dim{\Def_G(C)}$. If a component has at least 4 special points then there are moduli given by cross ratios:
\[ \dim{\Def_G(C)} = \sum_{C_i} \max \{ ( \text{valence}_{C_i} - 3 ), 0 \} \]
Infinitesimal automorphisms exist only for components of valence $< 3$. Therefore,
\[ \dim H^0(C, \T_C(-\text{nodes})) = \sum_{C_i} \max \{ (3 - \text{valence}_{C_i}), 0 \} \]
Next,
\begin{align*}
\dim \Def_G(\mu) = \dim \Def_G(C) + \dim \Def(\mu : C \to X) & = \dim \Def_G(C) - \dim H^0(C, \T_C(-\text{nodes})) + \dim H^0(C, \mu^* \T_C) 
\\
& = \sum_{C_i} (\text{valence}_{C_i} - 3) + \chi(C, \mu^* \T_X) = 2 k - 3 (k + 1) + \dim(X) + c_1(\T_X) \cdot \beta 
\\
& = (\dim(X) - 3 + c_1(\T_X) \cdot \beta) - k
\end{align*}

\section{Matt}

Throughout, let $X$ be smooth, projective, convex variety.

\begin{example}
\begin{enumerate}
\item $X = G / P$ where $G$ is a reductive group and $P$ is a parabolic
\item $X = A$ is an abelian variety
\end{enumerate}
\end{example}

Consider for each $1 \le i \le n$ the evaluation map $\ev_i : \ol{M}_{0,n}(X, \beta) \to X$ at the $i^{\text{th}}$-marked point. Let $\gamma_1, \dots, \gamma_n \subset X$ be cycles. We define,
\[ I_\beta(\gamma_1 \cdots \gamma_n) = \int_{[ \ol{M}_{0,n}(X, \beta) ]} \ev_i \gamma_1 \smile \cdots \smile_n \gamma_n \]
Now we consider
\[ \dim{\ol{M}_{0,n}(X, \beta)} = \dim{X} + (n-3) + c_1(\T_X) \cdot \beta \]
These numbers satisfy the following properties:
\begin{enumerate}
\item $I_\beta$ is multilinear in the $\gamma_1, \dots, \gamma_n$.
\item if $\sum \codim{\gamma_i} \neq \dim(X) + (n-3) + c_1(\T_X) \cdot \beta$ then $I_\beta(\gamma_1 \dots \gamma_n) = 0$
\item $I_\beta(\gamma_1 \dots \gamma_n)$ is symmetric up to signs from the cup product
\item if $\beta = 0$ then $\ol{M}_{0,n}(X, \beta) = \ol{M}_{0,n} \times X$ and thus ($n \ge 3$)
\[ I_\beta(\gamma_1 \dots \gamma_n) = 
\begin{cases}
\int \gamma_1 \smile  \gamma_2 \smile \gamma_3 & n = 3
\\
0 & \text{else}
\end{cases} \] 
\item For $\gamma_1 = 1$ we get
\[ I_\beta(1 \cdot \gamma_2 \cdots \gamma_n) = 
\begin{cases}
\int \gamma_2 \smile \gamma_3 & n = 3, \beta = 0
\\
0 & \text{else}
\end{cases} \]
Using that the intersection is pulled back along $\ol{M}_{0, n}(X, \beta) \to \ol{M}_{0,n-1}(X, \beta)$ but the class on the base has the wrong dimension. 
\item if $\beta$ is not effective then $I_\beta(\gamma_1 \cdots \gamma_r) = 0$
\item Divisor axiom: if $\gamma_1 \in H^2(X)$ then
\[ I_\beta(\gamma_1 \cdots \gamma_n) = \left( \int_\beta \gamma_1 \right) I_\beta(\gamma_2 \cdots \gamma_n) \] 
\end{enumerate}

\begin{example}
Suppose $G \acts X$ transitively and $\gamma_i = [\Gamma_i]$ are classes of subvarities then for general $g \in G$
\[ I_\beta(\gamma_1, \dots, \gamma_n) = \# \{ \mu : \P^1 \to X, (0,1, \infty, p_4, \dots, p_n) \mid \forall i : \ev_i(p_i) = g \Gamma_i \text{ and } \mu_* [\P^1] = \beta \} \]
Why: the locus of automorphism free curves with smooth source $M_{0,n}(X, \beta)^\circ \subset \ol{M}_{0,n}(X, \beta)$ is a dense open. Then we use Kleiman-Bertini theorem which says that all the intersections of $\ev_i^{-1}(\Gamma_i)$ can be made transverse and lie inside $M_{0,n}(X, \beta)^\circ$ for a general choice of $g \in G$. 
\end{example}

\begin{example}
Consider the diagram,
\begin{center}
\begin{tikzcd}
\ol{M}_{0,n}(X, \beta) \arrow[r, "\ev"] \arrow[d, "f"] & X^n
\\
\ol{M}_{0,n}
\end{tikzcd}
\end{center}
Then we consider $f_* \ev^* : H^\bullet(X)^{\ot n} \to H^\bullet(\ol{M}_{0,n})$ which generalizes the Gromov-Witten invariants which appear as the top degree part of the image of this map. 
\end{example}

\section{Ben}

Let's review the Gromov-Witten theory of $X = \P^2$. Here $X$ is a homogenous space for $G = \PGL_3$. Using the notation of last time $T_0, T_1, T_2$ is a basis for the cohomology of $X$ where
\begin{enumerate}
\item $T_0 = 1$
\item $T_1 = [\text{line}]$
\item $T_2 = [\text{point}]$
\end{enumerate}
Furthermore, $\beta \in H^2(X, \Z)$ must be of the form $\beta = d T_1$ and the effective ones have $d \ge 0$ where $d$ is the degree of the curve. Then we write $\ol{M}_{0,n}(\P^2, d)$ for the moduli space of genus $0$ stable maps of degree $d$ with $n$ marked points to $\P^2$.
\bigskip\\
Recall the Gromov-Witten invariants,
\[ I_\beta(\gamma_1 \cdots \gamma_n) = \int_{[\ol{M}_{0,n}(X, \beta)]} \ev_1^{*}(\gamma_1) \smile \cdots \smile \ev_n^*(\gamma_n) \]
satisfy the following properties:
\begin{enumerate}
\item $I_\beta$ is multilinear in the $\gamma_1, \dots, \gamma_n$.
\item if $\sum \codim{\gamma_i} \neq \dim(X) + (n-3) + c_1(\T_X) \cdot \beta$ then $I_\beta(\gamma_1 \dots \gamma_n) = 0$
\item $I_\beta(\gamma_1 \dots \gamma_n)$ is symmetric up to signs from the cup product
\item if $\beta = 0$ then $\ol{M}_{0,n}(X, \beta) = \ol{M}_{0,n} \times X$ and thus ($n \ge 3$)
\[ I_\beta(\gamma_1 \dots \gamma_n) = 
\begin{cases}
\int \gamma_1 \smile  \gamma_2 \smile \gamma_3 & n = 3
\\
0 & \text{else}
\end{cases} \] 
\item For $\gamma_1 = 1$ we get
\[ I_\beta(1 \cdot \gamma_2 \cdots \gamma_n) = 
\begin{cases}
\int \gamma_2 \smile \gamma_3 & n = 3, \beta = 0
\\
0 & \text{else}
\end{cases} \]
Using that the intersection is pulled back along $\ol{M}_{0, n}(X, \beta) \to \ol{M}_{0,n-1}(X, \beta)$ but the class on the base has the wrong dimension. 
\item if $\beta$ is not effective then $I_\beta(\gamma_1 \cdots \gamma_r) = 0$
\item Divisor axiom: if $\gamma_1 \in H^2(X)$ then
\[ I_\beta(\gamma_1 \cdots \gamma_n) = \left( \int_\beta \gamma_1 \right) I_\beta(\gamma_2 \cdots \gamma_n) \] 
\end{enumerate}

\end{document}


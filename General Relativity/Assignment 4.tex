\documentclass[12pt]{article}
\usepackage[english]{babel}
\usepackage[utf8]{inputenc}
\usepackage[english]{babel}
\usepackage[a4paper, total={7.25in, 9.5in}]{geometry}
\usepackage{tikz-feynman}
\tikzfeynmanset{compat=1.0.0} 
\usepackage{subcaption}
\usepackage{float}
\floatplacement{figure}{H}
\usepackage{simpler-wick}
\usepackage{mathrsfs}  
\usepackage{dsfont}
\usepackage{relsize}
\DeclareMathAlphabet{\mathdutchcal}{U}{dutchcal}{m}{n}


\newcommand{\field}{\hat{\Phi}}
\newcommand{\dfield}{\hat{\Phi}^\dagger}
 
\usepackage{amsthm, amssymb, amsmath, centernot}
\usepackage{slashed}
\newcommand{\notimplies}{%
  \mathrel{{\ooalign{\hidewidth$\not\phantom{=}$\hidewidth\cr$\implies$}}}}
 
\renewcommand\qedsymbol{$\square$}
\newcommand{\cont}{$\boxtimes$}
\newcommand{\divides}{\mid}
\newcommand{\ndivides}{\centernot \mid}

\newcommand{\Integers}{\mathbb{Z}}
\newcommand{\Natural}{\mathbb{N}}
\newcommand{\Complex}{\mathbb{C}}
\newcommand{\Zplus}{\mathbb{Z}^{+}}
\newcommand{\Primes}{\mathbb{P}}
\newcommand{\Q}{\mathbb{Q}}
\newcommand{\R}{\mathbb{R}}
\newcommand{\ball}[2]{B_{#1} \! \left(#2 \right)}
\newcommand{\Rplus}{\mathbb{R}^+}
\renewcommand{\Re}[1]{\mathrm{Re}\left[ #1 \right]}
\renewcommand{\Im}[1]{\mathrm{Im}\left[ #1 \right]}
\newcommand{\Op}{\mathcal{O}}

\newcommand{\invI}[2]{#1^{-1} \left( #2 \right)}
\newcommand{\End}[1]{\text{End}\left( A \right)}
\newcommand{\legsym}[2]{\left(\frac{#1}{#2} \right)}
\renewcommand{\mod}[3]{\: #1 \equiv #2 \: \mathrm{mod} \: #3 \:}
\newcommand{\nmod}[3]{\: #1 \centernot \equiv #2 \: mod \: #3 \:}
\newcommand{\ndiv}{\hspace{-4pt}\not \divides \hspace{2pt}}
\newcommand{\finfield}[1]{\mathbb{F}_{#1}}
\newcommand{\finunits}[1]{\mathbb{F}_{#1}^{\times}}
\newcommand{\ord}[1]{\mathrm{ord}\! \left(#1 \right)}
\newcommand{\quadfield}[1]{\Q \small(\sqrt{#1} \small)}
\newcommand{\vspan}[1]{\mathrm{span}\! \left\{#1 \right\}}
\newcommand{\galgroup}[1]{Gal \small(#1 \small)}
\newcommand{\bra}[1]{\left| #1 \right>}
\newcommand{\Oa}{O_\alpha}
\newcommand{\Od}{O_\alpha^{\dagger}}
\newcommand{\Oap}{O_{\alpha '}}
\newcommand{\Odp}{O_{\alpha '}^{\dagger}}
\newcommand{\im}[1]{\mathrm{im} \: #1}
\renewcommand{\ker}[1]{\mathrm{ker} \: #1}
\newcommand{\ket}[1]{\left| #1 \right>}
\renewcommand{\bra}[1]{\left< #1 \right|}
\newcommand{\inner}[2]{\left< #1 | #2 \right>}
\newcommand{\expect}[2]{\left< #1 \right| #2 \left| #1 \right>}
\renewcommand{\d}[1]{ \mathrm{d}#1 \:}
\newcommand{\dn}[2]{ \mathrm{d}^{#1} #2 \:}
\newcommand{\deriv}[2]{\frac{\d{#1}}{\d{#2}}}
\newcommand{\nderiv}[3]{\frac{\dn{#1}{#2}}{\d{#3^{#1}}}}
\newcommand{\pderiv}[2]{\frac{\partial{#1}}{\partial{#2}}}
\newcommand{\fderiv}[2]{\frac{\delta #1}{\delta #2}}
\newcommand{\parsq}[2]{\frac{\partial^2{#1}}{\partial{#2}^2}}
\newcommand{\topo}{\mathcal{T}}
\newcommand{\base}{\mathcal{B}}
\renewcommand{\bf}[1]{\mathbf{#1}}
\renewcommand{\a}{\hat{a}}
\newcommand{\adag}{\hat{a}^\dagger}
\renewcommand{\b}{\hat{b}}
\newcommand{\bdag}{\hat{b}^\dagger}
\renewcommand{\c}{\hat{c}}
\newcommand{\cdag}{\hat{c}^\dagger}
\newcommand{\hamilt}{\hat{H}}
\renewcommand{\L}{\hat{L}}
\newcommand{\Lz}{\hat{L}_z}
\newcommand{\Lsquared}{\hat{L}^2}
\renewcommand{\S}{\hat{S}}
\renewcommand{\empty}{\varnothing}
\newcommand{\J}{\hat{J}}
\newcommand{\lagrange}{\mathcal{L}}
\newcommand{\dfourx}{\mathrm{d}^4x}
\newcommand{\meson}{\phi}
\newcommand{\dpsi}{\psi^\dagger}
\newcommand{\ipic}{\mathrm{int}}
\newcommand{\tr}[1]{\mathrm{tr} \left( #1 \right)}
\newcommand{\C}{\mathbb{C}}
\newcommand{\CP}[1]{\mathbb{CP}^{#1}}
\newcommand{\Vol}[1]{\mathrm{Vol}\left(#1\right)}

\newcommand{\Tr}[1]{\mathrm{Tr}\left( #1 \right)}
\newcommand{\Charge}{\hat{\mathbf{C}}}
\newcommand{\Parity}{\hat{\mathbf{P}}}
\newcommand{\Time}{\hat{\mathbf{T}}}
\newcommand{\Torder}[1]{\mathbf{T}\left[ #1 \right]}
\newcommand{\Norder}[1]{\mathbf{N}\left[ #1 \right]}
\newcommand{\Znorm}{\mathcal{Z}}
\newcommand{\EV}[1]{\left< #1 \right>}
\newcommand{\interact}{\mathrm{int}}
\newcommand{\covD}{\mathcal{D}}
\newcommand{\conj}[1]{\overline{#1}}

\newcommand{\SO}[2]{\mathrm{SO}(#1, #2)}
\newcommand{\SU}[2]{\mathrm{SU}(#1, #2)}

\newcommand{\anticom}[2]{\left\{ #1 , #2 \right\}}


\newcommand{\pathd}[1]{\! \mathdutchcal{D} #1 \:}

\renewcommand{\theenumi}{(\alph{enumi})}


\renewcommand{\theenumi}{(\alph{enumi})}

\newcommand{\atitle}[1]{\title{% 
	\large \textbf{Physics GR8040 General Relativity
	\\ Assignment \# #1} \vspace{-2ex}}
\author{Benjamin Church }
\maketitle}

\theoremstyle{definition}
\newtheorem{theorem}{Theorem}[section]
\newtheorem{definition}{definition}[section]
\newtheorem{lemma}[theorem]{Lemma}
\newtheorem{proposition}[theorem]{Proposition}
\newtheorem{corollary}[theorem]{Corollary}
\newtheorem{example}[theorem]{Example}
\newtheorem{remark}[theorem]{Remark}
\begin{document}

\atitle{4}

\section*{1. }

Consider a homogenesous magnetic field directed along the $y$-axis. In $(t,x,y,z)$ coordinates, the Electromagnetic tensor takes the form,
\[ 
F^{\alpha \mu} = \begin{pmatrix}
0 & 0 & 0 & 0
\\
0 & 0 & 0 & -B
\\
0 & 0 & 0 & 0
\\
0 & B & 0 & 0
\end{pmatrix} 
\]
and then using,
\[ T^{\mu \nu} = \frac{1}{4 \pi} F^{\mu \alpha} F^{\nu \beta} \eta_{\alpha \beta} - \frac{1}{16 \pi} F_{\mu \nu} F^{\mu \nu} \eta^{\alpha \beta} \] 
the stress-energy tensor is,
\[ T^{\mu \nu} = \frac{B^2}{8 \pi}
\begin{pmatrix}
1 & 0 & 0 & 0
\\
0 & 1 & 0 & 0
\\
0 & 0 & -1 & 0
\\
0 & 0 & 0 & 1
\end{pmatrix} \]
Now appling the boost matrix by $\beta = \tanh{\eta}$ along the $x$-direction,
\[ \Lambda^{\mu'}_{\nu} = 
\begin{pmatrix}
\cosh{\eta} & - \sinh{\eta} & 0 & 0 
\\
- \sinh{\eta} & \cosh{\eta} & 0 & 0
\\
0 & 0 & 1 & 0 
\\
0 & 0 & 0 & 1
\end{pmatrix} \]
Then,
\[ T^{\mu' \nu'} = T^{\mu \nu} \Lambda^{\mu'}_{\mu} \Lambda^{\nu'}_{\nu} \]
which gives,
\[ T^{\mu' \nu'} = \frac{B^2}{8 \pi}
\begin{pmatrix}
\cosh{2 \eta} & - \sinh{2 \eta} & 0 & 0
\\
- \sinh{2 \eta} & \cosh{2 \eta} & 0 & 0
\\
0 & 0 & -1 & 0
\\
0 & 0 & 0 & 1
\end{pmatrix} \]

\section*{2. }

Consider the Lagrangain for a scalar field, 
\[ \lagrange = - \tfrac{1}{2} \left( \partial_\mu \phi \: \partial^\mu \phi + m^2 \phi \right) \]
From the Euler-Lagrange equations,
\[ \pderiv{\lagrange}{\phi} = \partial_\mu \pderiv{\lagrange}{(\partial_\mu \phi)} \]
we find the equations of motion,
\[ - m^2 = \partial_\mu \left( - \partial^\mu \phi \right) \implies (\partial_\mu \partial^\mu - m^2) \phi = 0 \]
Furthermore, if we upgrade this Lagrangian to be valid in arbitrary coordinates on a (possibly) curved manifold,
\[ \lagrange = - \tfrac{1}{2} \left( g^{\mu \nu} \partial_\mu \phi \: \partial_\nu \phi + m^2 \phi^2 \right) \]
Then we may use the formula,
\begin{align*}
T_{\alpha \beta} & = - \frac{2}{\sqrt{g}} \frac{\delta \lagrange \sqrt{-g}}{\delta g^{\alpha \beta}} = - 2 \frac{\delta \lagrange}{\delta g^{\alpha \beta}} + g^{\alpha \beta} \lagrange 
\\
& = \partial_\alpha \phi \: \partial_\beta \phi -  \tfrac{1}{2} g_{\alpha \beta} \left( g^{\mu \nu} \partial_\mu \phi \: \partial_\nu \phi + m^2 \phi \right)
\end{align*}

\section*{3. }

In flat space, the Maxwell equation,
\[ \partial_\nu F^{\mu \nu} = \frac{4 \pi}{c} j^\mu \]
explicitly gives,
\[ \partial_\nu (\partial^\mu A^\nu - \partial^\nu A^\mu) = \partial_\nu \partial^\mu A^\nu - \partial_\nu \partial^\nu A^\mu = \frac{4 \pi}{c} j^\mu \]
If we choose the Lorentz gauge $\partial_\nu A^\nu = 0$ then because in flat space derivatives commute we find,
\[ \partial_\nu \partial^\nu A^\mu = - \frac{4 \pi}{c} j^\mu \]
One possible generalization of this equation to curved spacetime is,
\[ \nabla_\nu \nabla^\nu A^\mu = - \frac{4 \pi}{c} j^\mu \]
with the generalized gauge, $\nabla_\nu A^\nu = 0$. However, in curved spacetime,
\[ F^{\mu \nu} \neq \partial^\mu A^\nu - \partial^\nu A^\mu \]
so the above derivation does not hold and the difference includes derivatives of Christoffel symbols so it will not even hold in locally interal coordinates. However, what is true is that,
\[ F_{\mu \nu} = \nabla_\mu A_\nu - \nabla_\nu A_\nu = \partial_\mu A_\nu - \partial_\nu A_\mu \]
but index rasing breaks the nice cancelations. Regardless, applying the Maxwell equation generalized to curved spacetime,
\[ \nabla_\nu F^{\mu \nu} = \frac{4 \pi}{c} j^\mu \]
we find,
\[ \nabla_\nu (\nabla^\mu A^\nu - \nabla^\nu A^\mu) = \nabla_\nu \nabla^\mu A^\nu - \nabla_\nu \nabla^\nu A^\mu = \frac{4 \pi}{c} j^\mu \]
However, imposing the Gauge condition $\nabla_\nu A^\nu = 0$ will not reduce this equation to the above form. Rather we find,
\begin{align*}
\nabla_\nu \nabla^\nu A^\mu & = - \frac{4 \pi}{c} j^\mu + \nabla_\nu \nabla^\mu A^\nu 
\\
& = - \frac{4 \pi}{c} j^\mu + [\nabla^\nu, \nabla^\mu] A_\nu + \nabla^\mu \nabla^\nu A_\nu 
\\
& = - \frac{4 \pi}{c} j^\mu + R^{\quad \nu \mu}_{\nu \alpha} A^\alpha
\\
& = - \frac{4 \pi}{c} j^\mu + R^{\mu \alpha} A_\alpha
\end{align*}
Therefore,
\[ \nabla_\nu \nabla^\nu A^\mu = - \frac{4 \pi}{c} j^\mu + R^{\mu \alpha} A_\alpha \]
so this wave equation is corrected by the curvature which is the missing term constructed from derivatives of the Christoffel symbols aluded to above which cannot be detected by flat space computations. 

\section*{4. }

Consider a perfect fluid with stress-energy tensor,
\[ T^{\alpha \beta} = (w + p) u^\alpha u^\beta c^{-2} + g^{\alpha \beta} p \]
and baryon current $j^\alpha = n u^\alpha$. The fluid satisfies the conservation of energy,
\[ \nabla_\beta T^{\alpha \beta} = 0 \]
and conservation of baryons,
\[ \nabla_\alpha j^\alpha = 0 \]
For any extensive quantity $q$ denote its corresponding intensive ratio by $\tilde{q} = q/n$.

\begin{enumerate}
\item First consider,
\begin{align*}
u_\alpha \nabla_\beta T^{\alpha \beta} & = u_\alpha \nabla_\beta [ (w + p) u^\alpha u^\beta c^{-2} + g^{\alpha \beta} p ] 
\\
& = u_\alpha \nabla_\beta [ (\tilde{w} + \tilde{p}) u^\alpha (nu^\beta) c^{-2} + g^{\alpha \beta} p ] 
\\
& = u_\alpha (n u^\beta) \nabla_\beta [(\tilde{w} + \tilde{p}) u^\alpha] c^{-2} + u^\alpha \nabla_\alpha p 
\end{align*}
Furthermore,
\[ u_\alpha \nabla_\beta (q u^\alpha) = \nabla_\beta (q u_\alpha u^\alpha)- q u_\alpha \nabla_\beta u^\alpha = -c^2 \nabla_\beta q \]
Because $u_\alpha u^\alpha = -c^2$ and thus,
\[ u_{\alpha} \nabla_\beta u^\alpha = \tfrac{1}{2} \nabla_{\beta} u_\alpha u^\alpha = 0 \]
Plugging in, 
\[ u_\alpha \nabla_\beta T^{\alpha \beta} = - n u^\beta \nabla_\beta (\tilde{w} + \tilde{p}) + u^\beta \nabla_\beta p \]
Therefore, using the conservation of evergy,
\[ u^\alpha \nabla_\alpha (\tilde{w} + p/n) = n^{-1}  u^\alpha \nabla_\alpha p \]
which implies,
\[ u^\alpha \nabla_\alpha \tilde{w} + p u^\alpha \nabla_\alpha (1/n) = 0 \]
which implies that along streamlines,
\[ \d{\tilde{w}} = - p \d{(1/n)} \]
From the first law of theormodynamics,
\[ \d{\tilde{w}} = - p \d{(1/n)} + T \d{S} \]
and thus $\d{S} = 0$ along streamlines.

\item Likewise, consider,
\begin{align*}
(g_{\alpha \beta} + u_\alpha u_\beta / c^2) \nabla_\mu T^{\mu \beta} & = (g_{\alpha \beta} + u_\alpha u_\beta / c^2) \nabla_\mu [ (w + p) u^\mu u^\beta c^{-2} + g^{\mu \beta} p ] 
\\
& = \nabla_\mu [(w + p) u^\mu u_\alpha] c^{-2} + \nabla_\alpha p + u_\alpha u_\beta \nabla_\mu [(w + p)u^\mu u^\beta] c^{-4} + u_\alpha u^\mu \nabla_\mu p c^{-2}
\end{align*}
Now using the equality $u_\beta \nabla_\mu (q u^\beta) = -c^2 \nabla_\mu q$ we find,
\begin{align*}
u_\alpha u_\beta \nabla_\mu [(w + p)u^\mu u^\beta] c^{-4} = - u_\alpha \nabla_\mu [(w + p) u^\mu ] c^{-2} 
\end{align*}
Combining this with the first term we find,
\[ u^\mu (w + p) \nabla_\mu u_\alpha c^{-2} + \nabla_\alpha p + u_\alpha u^\mu \nabla_\mu p c^{-2} = 0 \]
Therefore, setting this expression equal to zero via energy conservation gives the relativistic Euler equation,
\[ (w + p) u^\mu \nabla_\mu u_\alpha + c^2 \nabla_\alpha p + u_\alpha u^\mu \nabla_\mu p = 0 \]
\end{enumerate}
Now in the Newtonian limit,
\[ g_{00} = - 1 - 2 \Phi c^{-2} \quad p \ll w \quad (w - n m c^2)/(nm c^2) \ll 1 \quad u^0 \sim 1  \quad u^i \sim v^i \ll c \]
and dropping small terms we find,
\begin{align*}
w u^\mu \nabla_\mu u_i + c^2 \partial_i p = 0 
\end{align*}
We can drop the last term because for spatial components $u_i u^\mu \nabla_\mu p$ this is at best first-order in $c$ and $p$ while $c^2 \nabla_i p$ is second-order in $c$ and first-order in $p$ and thus dominates. Furthermore, 
\begin{align*}
u^\mu \nabla_\mu u_i = u^\mu \partial_\mu u^i + \Gamma^i_{\mu \nu} u^{\mu} u^{\nu} 
\end{align*}
Since $u^i \sim v^i \ll c \sim u^0$ the $\Gamma^i_{00}$ term dominates. Furthermore, we have previously calculated that for a stationary spacetime,
\[ \Gamma^i_{00} = \partial_i \Phi c^{-2} \]
Therefore,
\[ u^\mu \nabla_\mu u_i = \partial_0 v^i + v^j \partial_j v^i + \partial_i \Phi \]
Putting everything together,
\[ \partial_0 v^i + v^j \partial_j v^i + \partial_i \Phi + \frac{c^2}{w} \partial_i p = 0 \]
Noting that $w c^{-2} = n m$ to zeroth order, we may write this equation in vector form,
\[ \pderiv{\bf{v}}{t} + (\bf{v} \cdot \nabla) \bf{v}  = - \nabla \Phi - \frac{1}{nm} \nabla p \]
which is exactly the nonrelativistic Euler equation. 

\section*{5. }

Let $M$ be the space-time manifold. Choose a space-like hyperplane $H \subset M$ and a time-like (nonzero) section $s \in (T H)^\perp$. Then for each $q \in H$ we may take $\bf{v}_q = s(q)/|s(q)|$ (i.e. such that $\bf{v}_q$ is normalized such that $|\bf{v}_q|^2 = -1$) which is a normalized time-like vector. Now we parametrize $M$ with coordinates $(t, q)$ via $(t, q) \mapsto \exp_q(t \bf{v}_q)$ i.e. take a geodesic through $q$ whose tangent vector at $H$ is normal to the tangent space at $q$. First, note that the tangent of the coordinate geodesic curves $\exp_q(t \bf{v}_q)$ has constant norm since they are parametrized by arclength. Thus, $g_{00} = -1$ everywhere since it holds on $H$. Furthermore, the curves $t \mapsto (t, q)$ are geodesics. Therefore,
\[ u^\alpha \nabla_\alpha u^\mu = \partial_t u^\mu + \Gamma^\mu_{\alpha \beta} u^{\alpha \beta} = 0 \]
However, in these coordinates these geodesics are $t$ coordinate curves so $u^\alpha = (1, \vec{0})$. Therefore, $\partial_t u^\mu = 0$ and thus $\Gamma^\mu_{\alpha \beta} u^{\alpha \beta} = \Gamma^\mu_{00} = 0$. Finally,
\[ \partial_0 g_{0i} = \Gamma^\mu_{00} g_{\mu i} + \Gamma^\mu_{0i} g_{\mu 0} = \Gamma^\mu_{0i} g_{0 \mu} = \Gamma^j_{0i} g_{0j} + \Gamma^0_{0i} g_{00} \]
Furthermore,
\[ \Gamma^0_{0i} = \tfrac{1}{2} g^{0 \alpha} \left( \partial_0 g_{i \alpha} + \partial_i g_{0 \alpha} - \partial_\alpha g_{0i} \right) \]
I claim that $g_{0i} = 0$ solution to this ODE everywhere. To see this, note that if $g_{0i} = 0$ then $g^{0i} = 0$ so,
\[ \Gamma^0_{0i} = \tfrac{1}{2} g^{00} \left( \partial_0 g_{i 0} + \partial_i g_{00} - \partial_0 g_{0i} \right) = 0 \]
vanishes because we have shown that $g_{00}$ is constant and we assume that $\partial_0 g_{0 i} = 0$. Thus, the ODE reduces to,
\[ \partial_0 g_{0i} = \Gamma^j_{0i} g_{0j} \]
which is solved by $g_{0i} = 0$ everywhere. 
Furthermore, since $\Gamma^{\mu}_{0i}$ is first-order in $g_{0i}$, this is a first-order ODE and thus the solution $g_{0i} = 0$ is unique given the boundary condition $g_{0i} = 0$ at $t = 0$ i.e. everywhere on $H$. Since these coordinates satisfy the boundary condition on the hypersurface $H$ that the time geodesics are perpendicular to $H$ i.e. $g_{0i} = 0$ on $H$, we must have $g_{0i} = 0$ everywhere in these coordinates since it is the unique solution satisfying those boundary conditions. 


\section*{6. }

Consider dust with a stress energy tensor,
\[ T^{\mu \nu} = \rho u^\mu u^\nu \]
Firstly, in the case that, $G_{\alpha \beta} = \kappa T_{\alpha \beta}$ then we know,
\[ \nabla_\mu G^{\mu \nu} = 0 \]
which implies that $\nabla_\mu T^{\mu \nu} = 0$. Expanding this in the Newtonian limit,
\begin{align*}
\nabla_\mu T^{\mu \nu} = \partial_\mu T^{\mu \nu} + \Gamma^\mu_{\mu \alpha} T^{\alpha \nu} + \Gamma^\nu_{\mu \alpha} T^{\mu \alpha} = 0
\end{align*}
However, the Christoffel symbols are proportional to derivatives of the metric and thus are at order $c^{-2}$. Thus, in the Newtonian limit, restricting to the case $\nu = 0$ we find,
\begin{align*}
\nabla_\mu T^{\mu 0} = \partial_\mu T^{\mu 0} = \partial_0 (\rho u^0 u^0) + \partial_i (\rho u^i u^0) = 0
\end{align*}
However, to first order, $u^0 = c$ and $u^i = v^i$ and thus,
\[ \pderiv{\rho}{t} + \nabla \cdot (\rho \bf{v}) = 0 \]
reproducing Newtonian mass conservation. Thus we have shown that $G_{\alpha \beta} = \kappa T_{\alpha \beta}$ is consistent with mass conservation in the Newtonian limit. In fact, this analysis shows that $\nabla_\mu T^{\mu \nu}$ must be of order $c^{0}$ to be consistent with mass conservation in the Newtonian limit. 
\bigskip\\
Alternatively, to show that the equation $R_{\alpha \beta} = \kappa' T_{\alpha \beta}$ is inconsistent we need only to show that under this assumption,
\[ \kappa'^{-1} \nabla_\mu (R^{\mu \nu} - G^{\mu \nu}) = \kappa'^{-1} \tfrac{1}{2} \nabla_\mu g^{\mu \nu} R = \kappa'^{-1} \tfrac{1}{2} \nabla^\nu R  \]
is order $c^1$ or greater in the Newtonian limit since in that case,
\[ \nabla_\mu R^{\mu \nu} = \nabla_\mu (R^{\mu \nu} - G^{\mu \nu})  + \nabla_\mu G^{\mu \nu} = \nabla_\mu (R^{\mu \nu} - G^{\mu \nu}) = \kappa' \nabla_\mu T^{\mu \nu} \]
which would imply that $\nabla_\mu T^{\mu \nu}$ does not vanish at order $c^1$. 
However, taking the trace of ,
\[ R_{\alpha \beta} = \kappa' T_{\alpha \beta} \] 
we find,
\[ \kappa'^{-1} R = T \]
and thus we must have, in the Newtonian limit,
\[  \nabla_\nu T = 0 \]
However, for dust,
\[ T = \rho u_\alpha u^\alpha = -c^2 \rho \]
and thus,
\[ - \nabla_\nu T = c^2 \nabla_\nu \rho = c^2 \partial_\nu \rho = 0 \]
yet $\partial_\nu \rho \neq 0$ for a general dust configuration, in particular, $\partial_\nu \rho \sim c^{-1} \rho$. Therefore $\kappa'^{-1} \nabla_\nu R$ cannot vanish at order $c^1$ implying that $\nabla_{\mu} T^{\mu \nu}$ cannot vanish at order $c^1$ which is inconsistent with mass conservation. 

\section*{7. Carroll 4.6 }

Let $K$ be a Killing vector field and set the vector potential $A_\mu = K_\mu$ generating the Electromagnetic tensor,
\[ F_{\alpha \beta} = \nabla_\alpha K_\beta - \nabla_\beta K_\alpha \] Now consider the Maxwell equation,
\begin{align*}
g^{\nu \alpha} \nabla_\nu F_{\alpha \beta} = g^{\nu \alpha} [\nabla_\nu \nabla_\alpha K_\beta - \nabla_\nu \nabla_\beta K_\alpha] 
\end{align*}
We have previously proven that if $K$ is a Killing field then,
\[ \nabla_\nu \nabla_\sigma K^\nu = R^\nu_{\: \sigma \nu \rho} K^\rho \]
Therefore,
\begin{align*}
g^{\nu \alpha} \nabla_\nu F_{\alpha \beta} = g^{\nu \alpha} [ R_{\beta \alpha \nu \rho}  - R_{\alpha \beta \nu \rho} ] K^\rho = -2 g^{\nu \alpha} R_{\alpha \beta \nu \rho} K^\rho = - 2 R_{\beta \rho} K^\rho
\end{align*}
Thus,
\[ \nabla_\nu F^{\mu \nu} = - 2 R^{\mu \rho} K_\rho \]
Thus if the metric of this spacetime satisfies the Einstein vacuum equation: $R_{\mu \nu} = 0$ then,
\[ \nabla_\nu F^{\mu \nu} = 0 \]
and thus $A_\mu = K_\mu$ satisfies the Maxwell equations. 

\section*{8. }

Consider the Riemann tensor,
\[ R^\rho_{\: \sigma \mu \nu} = \partial_\mu \Gamma^{\rho}_{\nu \sigma} - \partial_\nu \Gamma^\rho_{\mu \sigma} + \Gamma^\rho_{\mu \lambda} \Gamma^\lambda_{\nu \sigma} - \Gamma^\rho_{\nu \lambda} \Gamma^\lambda_{\mu \sigma} \]
Now consider the contraction,
\[ R = g^{\sigma \nu} \partial_\mu \Gamma^\mu_{\nu \sigma} - g^{\sigma \nu} \partial_\nu \Gamma^\mu_{\mu \sigma} + \Gamma^\mu_{\mu \lambda} \Gamma^{\lambda}_{\nu \sigma} g^{\sigma \nu} - \Gamma^\mu_{\sigma \lambda} \Gamma^\lambda_{\mu \nu} g^{\sigma \nu} \] 
Now define,
\[ A^\alpha = g^{\sigma \nu} \Gamma^\alpha_{\nu \sigma} - g^{\sigma \alpha} \Gamma^\lambda_{\lambda \sigma} \]
Recall the identity,
\[ \frac{1}{\sqrt{-g}} \pderiv{\sqrt{-g}}{x^\alpha}  = \Gamma^\beta_{\beta \alpha}  \]
Then we have,
\begin{align*}
\frac{1}{\sqrt{-g}} \pderiv{(\sqrt{-g} A^\alpha)}{x^\alpha} & = \partial_\alpha A^\alpha + \Gamma^\beta_{\beta \alpha} A^\alpha 
\\
& = \partial_\alpha (g^{\sigma \nu} \Gamma^\alpha_{\nu \sigma} - g^{\sigma \alpha} \Gamma^\lambda_{\lambda \sigma}) + \Gamma^{\beta}_{\beta \alpha} (g^{\sigma \nu} \Gamma^\alpha_{\nu \sigma} - g^{\sigma \alpha} \Gamma^\lambda_{\lambda \sigma}) 
\end{align*}
Thus,
\begin{align*}
\frac{1}{\sqrt{-g}} \pderiv{(\sqrt{-g} A^\alpha)}{x^\alpha} - R  & = \partial_\alpha (g^{\sigma \nu} \Gamma^\alpha_{\nu \sigma} - g^{\sigma \alpha} \Gamma^\lambda_{\lambda \sigma}) + \Gamma^{\beta}_{\beta \alpha} (g^{\sigma \nu} \Gamma^\alpha_{\nu \sigma} - g^{\sigma \alpha} \Gamma^\lambda_{\lambda \sigma})
\\
& - (g^{\sigma \nu} \partial_\mu \Gamma^\mu_{\nu \sigma} - g^{\sigma \nu} \partial_\nu \Gamma^\mu_{\mu \sigma}) - (\Gamma^\mu_{\mu \lambda} \Gamma^{\lambda}_{\nu \sigma} g^{\sigma \nu} - \Gamma^\mu_{\sigma \lambda} \Gamma^\lambda_{\mu \nu} g^{\sigma \nu})
\\
& = \Gamma^{\alpha}_{\nu \sigma} \partial_\alpha g^{\sigma \nu} - \Gamma^{\lambda}_{\lambda \sigma} \partial_\alpha g^{\sigma \alpha} + \Gamma^\mu_{\sigma \lambda} \Gamma^\lambda_{\mu \nu} g^{\sigma \nu} - \Gamma^{\beta}_{\beta \alpha}  \Gamma^\lambda_{\lambda \sigma} g^{\sigma \alpha}
\end{align*}
Therefore, let,
\[ H = \Gamma^{\lambda}_{\lambda \sigma} \partial_\alpha g^{\sigma \alpha} - \Gamma^{\alpha}_{\nu \sigma} \partial_\alpha g^{\sigma \nu}   +  \Gamma^{\beta}_{\beta \alpha} \Gamma^\lambda_{\lambda \sigma} g^{\sigma \alpha} - \Gamma^\mu_{\sigma \lambda} \Gamma^\lambda_{\mu \nu} g^{\sigma \nu}\]
which only depends on first derivatives of the metric not on second derivatives. Thus we may write,
\[ R = H +  \frac{1}{\sqrt{-g}} \pderiv{(\sqrt{-g} A^\alpha)}{x^\alpha} \]
Furthermore, using the fact that,

\section*{9. Carroll 4.1}

Consider the Lagrangian,
\[ \lagrange = \sqrt{-g} \left( - \tfrac{1}{4} F_{\mu \nu} F^{\mu \nu} + A_\mu J^\mu \right) \]
We choose to hold fixed $A_\mu$ and $\sqrt{-g} J^\mu$ (the latter such that currents in regions of spacetime are preserved) in varying the Lagrangian. Writing the Lagrangian explicitly in terms of the metric we find,
\[ \lagrange = \sqrt{-g} \left( - \tfrac{1}{4} g^{\alpha \gamma} g^{\beta \delta} F_{\alpha \beta} F_{\gamma \delta} + A_\alpha J^\alpha \right) \]
where,
\[ F_{\mu \nu} = \nabla_\mu A_\nu - \nabla_\nu A_\mu = \partial_\mu A_\nu - \partial_\nu A_\mu \]
is independent of the metric. Therefore,
\begin{align*}
T_{\mu \nu} & = - \frac{2}{\sqrt{-g}} \frac{\delta \lagrange}{\delta g^{\mu \nu}}
\\
& = \tfrac{1}{2} (\delta_{\mu}^{\alpha} \delta_{\nu}^{\gamma} g^{\beta \delta} + g^{\alpha \gamma} \delta_{\mu}^{\beta} \delta_{\nu}^{\delta}) F_{\alpha \beta} F_{\gamma \delta} + \frac{2}{\sqrt{-g}} \frac{\partial \sqrt{-g}}{\partial g^{\mu \nu}} \left( \tfrac{1}{4} F_{\alpha \beta} F^{\alpha \beta} \right) 
\\
& = \tfrac{1}{2} g^{\beta \delta} F_{\mu \beta} F_{\nu \delta} + \tfrac{1}{2} g^{\alpha \gamma} F_{\alpha \mu} F_{\gamma \nu} - \tfrac{1}{4} g_{\mu \nu} \: F_{\alpha \beta} F^{\alpha \beta}   
\\
& = F_{\mu \alpha} F_{\nu \beta} \: g^{\alpha \beta} - \tfrac{1}{4} g_{\mu \nu} \: F_{\alpha \beta} F^{\alpha \beta}
\end{align*}
Thus we have found,
\[ T_{\mu \nu} =  F_{\mu \alpha} F_{\nu \beta} \: g^{\alpha \beta} - \tfrac{1}{4} g_{\mu \nu} \: F_{\alpha \beta} F^{\alpha \beta} \]




\end{document}
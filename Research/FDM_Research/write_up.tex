\documentclass{article}
\usepackage[utf8]{inputenc}
\usepackage[margin=1in]{geometry}
\usepackage{amsmath,amsthm,amssymb}
\usepackage{tikz-cd}
\usepackage{comment}
\usepackage{fancyhdr}
\usepackage{hyperref}
\hypersetup{
    colorlinks,
    citecolor=black,
    filecolor=black,
    linkcolor=black,
    urlcolor=black
}

\newcommand{\bb}[1]{\mathbb{#1}}
\newcommand{\R}{\mathbb{R}}
\renewcommand{\P}{\mathbb{P}}
\newcommand{\Z}{\mathbb{Z}}
\newcommand{\Q}{\mathbb{Q}}
\newcommand{\C}{\mathbb{C}}
\newcommand{\A}{\mathbb{A}}
\newcommand{\F}{\mathbb{F}}
\newcommand{\I}{\mathbb{I}}
\newcommand{\Pro}{\mathbb{P}}
\newcommand{\pf}{\mathfrak{p}}
\newcommand{\Pf}{\mathfrak{P}}
\newcommand{\multzp}[1]{\left(\bb{Z}/#1\bb{Z}\right)^{\times}}
\newcommand{\iO}{\mathcal{O}}
\newcommand{\ia}{\mathfrak{a}}
\newcommand{\bphi}{\bar{\varphi}}

\newcommand{\into}{\hookrightarrow}
\newcommand{\Gal}[0]{\mathrm{Gal}}
\newcommand{\Hom}[0]{\mathrm{Hom}}
\newcommand{\Aut}[0]{\mathrm{Aut}}
\newcommand{\Tr}[0]{\mathrm{Tr}}
\newcommand{\Nm}[0]{\mathrm{Nm}}
\newcommand{\Ker}[0]{\mathrm{Ker}}
\newcommand{\Frac}[0]{\mathrm{Frac}}
\newcommand{\Stab}[0]{\mathrm{Stab}}
\newcommand{\Spec}[0]{\mathrm{Spec}}
\newcommand{\coker}[0]{\mathrm{coker}}
\newcommand{\im}[0]{\mathrm{im}}
\newcommand{\Res}[0]{\mathrm{Res}}
\newcommand{\Ind}[0]{\mathrm{Ind}}
\newcommand{\Frob}[0]{\mathrm{Frob}}
\newcommand{\finfield}[1]{\mathbb{F}_{#1}}
\newcommand{\finunits}[1]{\mathbb{F}_{#1}^\times}
\newcommand{\divides}{\mid}
\newcommand{\ndivides}{\centernot \mid}
\newcommand{\s}[1]{s\left( #1 \right)}

\newcommand{\lcm}[0]{\mathrm{lcm} \,}
\newcommand{\cis}[0]{\mathrm{cis}}
\newcommand{\ord}[0]{\mathrm{ord}}
\newcommand{\Sing}[0]{\mathrm{Sing }}
\newcommand{\mathfrac}[0]{\mathfrak}

\newcommand{\frp}[2]{\left\{\frac{#1}{#2}\right\}}
\newcommand{\pdiv}[2]{\frac{\partial #1}{\partial #2}}
\newcommand{\leg}[2]{\left(\frac{ #1}{ #2}\right)}

%%%%Theorem + Equation Styles%%%%%%%
\newtheorem{theorem}{Theorem}[section]
\newtheorem{corollary}{Corollary}[theorem]
\newtheorem{lemma}[theorem]{Lemma}
\newtheorem{proposition}[theorem]{Proposition}
\newtheorem{conjecture}[theorem]{Conjecture}

\theoremstyle{definition}
\newtheorem{problem}[theorem]{Problem}
\theoremstyle{definition}
\newtheorem{definition}[theorem]{Definition}
\newtheorem{example}[theorem]{Example}
\newtheorem{fact}[theorem]{Fact}

\theoremstyle{remark}
\newtheorem*{remark}{Remark}


\begin{document}

\title{\Huge \textbf{Research Notes}}
\author{Matthew Lerner-Brecher, Benjamin Church, Chunying Huangdai, Ming Jing, Navtej Singh}
\maketitle
\tableofcontents
\newpage

\section{On Affine Varieties}

\begin{theorem}
\label{thm:affine_reduct}
Suppose $X$ is the affine variety over $F_{q}$ defined by the zero set of:
\[a_0x_0^{n_0} + a_1x_1^{n_1} + \cdots + a_r x_r^{n_r}\]
For each $0 \le i \le r$, let $L_i = \lcm(\{n_j\}|_{j\neq i})$ and let $n_i' = \gcd(n_i, L_i)$. Then the affine variety $X'$ over $\F_q$ defined by the zero set of:
\[a_0x_0^{n_0'} + a_1x_1^{n_1'} + \cdots + a_r x_r^{n_r'}\]
has $|X'| = |X|$.
\end{theorem}
\begin{proof}
Let $d_i = \gcd(n_i, q - 1)$ and let $d_i' = \gcd(n_i', q - 1)$. By equation $(3)$ from Weil's paper we have:
\[|X| = q^r + (q-1)\sum_{\alpha \in S}\chi_{\alpha_0}(a_0^{-1})\cdots\chi_{\alpha_r}(a_r^{-1}) j(\alpha)\]
where $S = \{\alpha = (\alpha_0, \ldots, \alpha_r) : d_i\alpha_i \in \Z; \sum \alpha_i \in \Z; 0 < \alpha_i < 1 \}$. Similarly, we get:
\[|X'| = q^r + (q-1)\sum_{\alpha \in S'} \chi_{\alpha_0}(a_0^{-1})\cdots\chi_{\alpha_r}(a_r^{-1})j(\alpha)\]
where  $S' = \{\alpha = (\alpha_0, \ldots, \alpha_r) : d_i'\alpha_i \in \Z; \sum \alpha_i \in \Z; 0 < \alpha_i < 1 \}$. We will show that $S = S'$ and hence the two expressions must be equal. Note that as $n_i' | n_i$, $d_i' | d_i$. Thus $d_i'\alpha \in \Z$ implies $d_i\alpha \in \Z$. As such, $S' \subset S$. Now suppose $\alpha \in S$. If $d_i = d_i'$ for all $i$, the two sets are equal and we're done. As such assume $j$ is such that $d_j' \neq d_j$. As gcd is commutative, $d_j' = \gcd(d_j, L_j)$. Then we can write, $d_j = d_j'm$. Now for each $i$, as $d_i\alpha_i \in \Z$ and $0 < \alpha_i < 1$, there exists $a_i$ such that $\alpha_i = \frac{b_i}{d_i}$. Now, as $\alpha \in S$,
\[\frac{b_j}{d_j'm} + \sum_{i \neq j}\frac{b_i}{d_i} \in \Z\]
Let $\frac{B}{D} = \sum_{i \neq j}\frac{b_i}{d_i} \in \Z $ be a fraction in simplest form. Thus we have
\[\frac{b_j}{d_j'm} + \sum_{i \neq j}\frac{b_i}{d_i} = \frac{b_j}{d_j'm} + \frac{B}{D} = \frac{b_jD + d_j'mA}{d_j'mD} \in \Z\]
As $d_i | n_i | L_j$ for all $i \neq j$, we have $D | L_j$. For the above expression to be an integer we must have $d_j'm | b_jD$. As $d_j' = \gcd(d_j'm, D)$, this implies $m | b_j$. However, this means $d_j'\alpha_j = \frac{b_j}{m} \in \Z$. By our reasoning, this holds for all $j$. Thus $S' \subset S$.
\par
As explained before, this implies $S = S'$ and thus $|X| = |X'|$.
\end{proof}
\begin{theorem} Let $X$ be the affine variety over $\F_q$ defined by the zero set of:
\[a_0x_0^{n_0} + \cdots + a_rx_r^{n_r}\]
where the $a_i$ are nonzero and the $n_i$ are positive integers. If for all $1 \le i \le r$ we have $\gcd(n_0, n_i) = 1$, then $X$ is supersingular.
\end{theorem}
\begin{proof}
By theorem $\ref{thm:affine_reduct}$, $X$ has the same number of solutions as the variety $X'$ defined by the zero set of 
\[a_0x_0^{n_0'} + \cdots + a_rx_r^{n_r'}\]
As $n_0$ is relatively prime to the other $n_i$, $n_0' = 1$. However, then $a_0x_0$ achieves every element of $\F_q$ exactly once. Hence, regardless of the choice of $x_1, \ldots, x_r$ there is precisely one value of $x_0$ for which the defining equation of $X'$ is 0. Thus $|X| = q^r$. By the same reasoning if we define $N_k$ to be the number of points of $X$ defined over $\F_{q^k}$, we have
\[N_k = (q^k)^r = q^{rk}\]
As such the zeta function $\zeta_X$ is:
\begin{align*}
\zeta_X(T) &= \exp\left(\sum_{m \ge 1} \frac{q^{rm}}{m}T^m\right) \\
&= \exp\left(-\log(1 - q^rT)\right)\\
&= \frac{1}{1-q^rT}
\end{align*}
which implies that $X$ is supersingular, as desired.
\end{proof}

\section{On Projective Varieties}
\subsection{Conversion to Weighted Projective Space}
Note on notation. From now on, unless otherwise specified, let $X$ be an affine variety over $\F_q$ defined to be the zero set of
\[a_0x_0^{n_0} + \cdots + a_rx_r^{n_r}\]
such that the $a_i$ are nonzero. Let $L = \lcm(n_i)$ and $N_i = L/n_i$. For a given point $P = (P_0, \ldots, P_r)$ let 
\[S_P = \{ N_i \ : \ P_i \neq 0\}\]
Let $d_P = \gcd(S_P)$. We also define $V$ to be the image of $X$ in weighted projective space.
\begin{theorem} Suppose $\lambda$ acts on $X$ as follows: For any point $(x_0, \ldots, x_r)$ we have
\[\lambda \cdot (x_0, \ldots, x_r) = (\lambda^{N_0}x_0, \ldots, \lambda^{N_r}x_r)\]
Then for all $P = (P_0,\ldots, P_r) \in X$, 
\[|\Stab(P)| = \gcd(S_P)\]
In particular, $P_i \neq 0$ for all $i$, $|\Stab(P)| = 1$.
\end{theorem}
\begin{proof}
Suppose $\lambda\cdot P = P$. Then we have:
\[((\lambda^{N_0}-1)P_0, \ldots, (\lambda^{N_r}-1)P_r) = (0, \ldots, 0)\]
This holds if and only if $\lambda^{N_i} = 1$ for all $P_i \neq 0$. This is equivalent to $\lambda^{\gcd(d_P, q - 1)} = 1$, which has exactly $\gcd(d_P, q - 1)$ solutions.
\end{proof}
\begin{corollary}
\label{cor:orbit_count_V}
\[|V| = \sum_{P \in X/\{0\}} \frac{\gcd(d_P, q - 1)}{q - 1}\]
\end{corollary}
\begin{proof}
By the orbit-stabilizer theorem, under the scaling action of weighted projective space, $orb(P) = \frac{q-1}{\gcd(d_P, q-1)}$. This then follows from the fact that:
\[|V| = \sum_{P \in X/\{0\}} \frac{1}{orb(P)}\]
\end{proof}
We'll now introduce one more piece of notation. Suppose $t = (t_0, \ldots, t_r) \in \{0,1\}^{r+1}$. Say
\[C_t := \{P \in X \ : \ P_i = 0 \iff t_i = 0\}\]
and
\[S_t := \{N_i \ : \ t_i = 1\}\]
and as before $d_t = \gcd(S_t)$. Note that the $C_t$s form a partition of $X$. We also define an ordering on $\{0,1\}^{r+1}$. Suppose $u = (u_0,\ldots, u_r),t = (t_0, \ldots, t_r) \in \{0,1\}^{r+1}$. We say that $t \prec u$ if for all $i$, $u_i = 0 \implies t_i = 0$. Let
\[X_u = \bigcup_{t \prec u} C_t\]
(Note that there is a bijection between $X_u$ and the zero set of the equation: $\sum_j a_{i_j}x^{n_{i_j}}$ where $i_j$ ranges only over the values of $i$ such that $u_i = 1$. We make this note because using Weil's paper we can count $X_u$ more directly than $C_u$). Lastly, for convenience, let $T = \{0,1\}^{r+1}/\{(0,0,\ldots, 0)\}$
\begin{theorem}
\[|C_u| = \sum_{t \prec u} (-1)^{sum(u) - sum(t)} |X_u|\]
\end{theorem}

\begin{proof}
As the $C_t$ are disjoin we have:
\[|X_u| = \sum_{t \prec u} |C_t|\]
Let $p_0, p_1, \ldots, p_r$ be distinct primes and for $t \in \{0,1\}^{r+1}$ let:
\[P(t) = \prod_{i = 0}^r p_i^{t_i}\]
Let $Q$ be the inverse of $P$. Note then that $P(t) | P(u)$ if and only if $t \prec u$. Thus our above equation becomes:
\[|X_u| = \sum_{d | P(u)} |C_{Q(d)}|\]
By the Mobius Inversion formula:
\[|C_u| = \sum_{d | P(u)} |X_{Q(u)}| \mu\left(\frac{P(u)}{d}\right)\]
Let $t = Q(u)$. As $P(u), d$ are squarefree, $\mu\left(\frac{P(u)}{d}\right) = \mu(P(u))/\mu(d)$. Note that $\mu(P(u)) = (-1)^{sum(u)}$. Thus, by the equivalence between $P(t) | P(u)$ and $t \prec u$, this summation is equivalent to 
\[|C_u| = \sum_{t \prec u} (-1)^{sum(u) - sum(t)} |X_u|\]
as desired.
\end{proof}
\begin{theorem}
\label{V_formula}
\[|V| = \sum_{t \in T} |C_t| \frac{\gcd(d_t, q - 1)}{q - 1}\]
\end{theorem}
\begin{proof}
Note that for all $P \in C_t$, $d_P = d_t$. As the $C_t$ form a partition of $X$, this formula is just a restatement of Corollary $\ref{cor:orbit_count_V}$
\end{proof}






\subsection{Supersingular Projective Varieties}
\begin{lemma} 
\label{lem:gcd_sum}
For a given prime power $q$ and integer $N$. Suppose $N'$ is the largest divisor of $N$ relatively prime to $q$. Define:
\[g(k) = \gcd(N,q^k-1)\]
Furthermore define
\[f_r(k) = \begin{cases} 1 & r|k \\
0 & \text{else}
\end{cases}\]
Then 
\[g(k) = \sum_{i=1}^M a_if_i(k)\]
where $M = \ord_{N'}(q)$ and 
\[a_i = \sum_{d|i} g(d)\mu(i/d)\]
for $i | M$ and $a_i = 0$ otherwise with $\mu$ the moebius function.
\begin{proof}
Set $a_i$ to be as claimed in the lemma statement. Note that 
\[g(k) = \gcd(N,q^k-1) = \gcd(N',q^k-1)\]
By the Moebius inversion formula for $k | M$ we have:
\[g(k) = \sum_{i|k} a_i\]
As $f_i(k) = 1$ if $i|k$ and 0 otherwise this is equivalent to:
\[g(k) = \sum_{i=1}^M a_if_i(k)\]
We now claim $g(k) = g(\gcd(k,M))$. Clearly if $A | q^{\gcd(k,M)} - 1$, then $A | q^k - 1$. Thus $g(\gcd(k,M)) | g(k)$. Now suppose $A | q^k - 1$ for $A | N'$. As $A | N'$, $ A | q^M - 1$. Thus for all $x,y$ $A |q^{kx + My} - 1$. By Bezout's identity, $A | q^{\gcd(k,M)} - 1$. Thus $g(k) | g(\gcd(k,M))$ and so $g(k) = g(\gcd(k,M))$. Now let $k$ be any integer. Note that $a_i$ and $f_i(k)$ are both nonzero only if $i$ divides $M$ and $k$ and hence $\gcd(i,k)$. Thus we have:
\[\sum_{i=1}^M a_if_i(k) = \sum_{i | \gcd(k,M)} a_i\]
However, as $\gcd(k,M)$ divides $M$ we have already shown the latter expression to be $g(\gcd(k,M))$. As this equals $g(k)$, we have for all $k$: 
\[g(k) = \sum_{i=1}^M a_if_i(k)\]
as desired
\end{proof}
\end{lemma}
\begin{lemma}
\label{lem:a_w_div}
For a given prime power $q$ and integer $N$, define $g(k)$ and $a_i$ and $M$ as in the preceding lemma. Then for all $w$, we have $w | a_w$. 
\end{lemma}
\begin{proof}
If $w$ is not a divisor of $M$ then $a_w = 0$ and so the statement follows immediately. As such, from now on we will assume $w$ is a divisor of $M$ so that we may use the inversion formula for $a_w$.
\\
We'll begin by showing this is true for all $N,q$ in the case where $w = p^i$ for some prime $p$. We have:
\[a_w = \sum_{d|w} g(d)\mu(w/d) = g(p^i) - g(p^{i-1})\]
If $g(p^i) = g(p^{i-1})$ then we have $a_w = 0$ and so $w | a_w$. Suppose $g(p^i) \neq g(p^{i-1})$. As $q^{p^{i-1}} -1 | q^{p^i} - 1$, we have $g(p^{i-1}) | g(p^i)$. Now let $B$ be such that $g(p^i) = Bg(p^{i-1})$. Note that 
\[\gcd\left(\frac{q^{p^i} - 1}{q^{p^{i-1}} - 1},q^{p^{i-1}} - 1\right)\]
can only be a power of $p$. If $p|B$, then $p|q^{p^i} - 1$ which occurs if and only if $p | q - 1$. If $p | q - 1$, then by lifting the exponent lemma $p^i | q^{p^{i - 1}} - 1$. So either $p^i$ divides both  $g(p^{i-1})$ and $g(p^{i})$, in which case we're done or $p\nmid B$. As $p \nmid B$ and 
\[\gcd\left(\frac{q^{p^i} - 1}{q^{p^{i-1}} - 1},q^{p^{i-1}} - 1\right)\]
can only be a power of $p$, all prime factors of $B$ cannot be factors of $q^{p^{i-1}} - 1$. Thus for all primes $t|B$ we have $q^{p^{i-1}} \not\equiv 1 \pmod t$ but $q^{p^i} \equiv 1 \pmod t$ which implies $p^i | \ord_t(q) | t - 1$. As for all primes $t | B$ we have $t \equiv 1 \pmod p^i$, we have $B \equiv 1 \pmod p^i$. Now 
\[g(p^i) - g(p^{i-1}) = (B-1)g(p^{i-1})\]
and thus $p^i | g(p^i) - g(p^{i-1})$ as desired.
\\
\\
We'll now show that if $m,n$ are relatively prime positive integers such that regardless of the choice of $N,q$ we have $n|a_n$ and $m|a_m$, then $mn|a_{mn}$. For notational purposes let $g_{N,q}(k)$ be $g(k)$ for given $N,q$. We have
\begin{align*}
a_{mn} &= \sum_{d | mn} g(d)\mu(mn/d) \\
&= \sum_{x|m}\mu(m/x)\sum_{y|n} g(xy)\mu(n/y) \\
&= \sum_{x|m}\mu(m/x)\sum_{y|n} \gcd(N, (q^{x})^y - 1)\mu(n/y) \\
&= \sum_{x|m}\mu(m/x)\sum_{y|n} g_{N,q^x}(y)\mu(n/y) \\
\end{align*}
By our assumption that regardless of the choice of $N,q$ we have $n|a_n$ and $m|a_m$ we have $n | \sum_{y|n} g_{N,q^x}(y)\mu(n/y)$ (as the latter is the formula for $a_n$ for $N$, $q^x$ given). Thus $n$ divides the total expression and hence $a_{mn}$. By symmetry, $m | a_{mn}$.
\\
\\
Now suppose $w = \prod_{i} p_i^{e_i}$. By the first part of our proof $p_i^{e_i} | a_{p_i^{e_i}}$. By the second part of our proof all of these divisibility statements together imply 
\[ w = \prod_{i} p_i^{e_i} | a_{\prod_{i} p_i^{e_i}} = a_w \]
as desired.
\end{proof}

\begin{definition}
Let $\frac{p(T)}{s(T)}$ be a rational function. Define $\frac{p(T)}{s(T)}$ to be $\textit{supersingular}$ if every root of both $p,s$ is of the form $r\alpha$ where $r \in \R_{\ge 0}$ and $\alpha$ is a root of unity. 
\end{definition}
\begin{theorem}
\label{gcd_power}
For given $N, q$ let $g(k) = \gcd(N, q^k - 1)$. Suppose 
\[\exp\left(\sum_{k \ge 1} h(k)\frac{T^k}{k}\right)\]
defines a rational function $\frac{p(T)}{s(T)}$. Then, 
\[B(T) := \exp\left(\sum_{k \ge 1} h(k)g(k)\frac{T^k}{k}\right)\]
also defines a rational function equal to
\[\prod_{i = 1}^M \left( \frac{p_i(T^i)}{s_i(T^i)}\right)^{b_i}\]
for some integers $b_i, M$ and with $p_k(T) = \prod_{j = 1}^k p(Te^{\frac{2\pi ij}{k}})$ and $s_k$ defined similarly. Furthermore, if $\frac{p(T)}{s(T)}$ is supersingular, then so is $B(T)$.
\end{theorem}

\begin{proof}
By Lemmas $\ref{lem:gcd_sum}$, for some $M$, we can write
\[g(k) = \sum_{i = 1}^M a_if_i(k)\]
Plugging this into our formula for $B(T)$ gives:
\begin{align*}
B(T) &= \exp\left(\sum_{k \ge 1} h(k)\sum_{i = 1}^M a_if_i(k)\frac{T^k}{k}\right) \\
&= \exp\left(\sum_{i = 1}^M a_i \sum_{k \ge 1} h(k)f_i(k)\frac{T^k}{k}\right) \\
&= \exp\left(\sum_{i = 1}^M a_i \sum_{k \ge 1} h(ik)\frac{T^{ik}}{ik}\right) \\
&= \prod_{i = 1}^M\exp\left(\sum_{k \ge 1} h(ik)\frac{T^{ik}}{k}\right)^{\frac{a_i}{i}}
\end{align*}
Let 
\[A(T) =\sum_{k \ge 1} h(k)\frac{T^k}{k}\]
so that $\frac{p(T)}{s(T)} = \log(A(T))$. Note note that if $\zeta_i$ is an $i$-th root of unity:
\begin{align*}
\sum_{k \ge 1} h(ik)\frac{T^{ik}}{ik} &= \frac{\sum_{j = 1}^i A(T\zeta_i^j)}{i} \\
\exp\left(\sum_{k \ge 1} h(ik)\frac{T^{ik}}{k}\right) &= \prod_{j = 1}^i \exp(A(T\zeta_i^j)) \\
&= \frac{p_i(T)}{s_i(T)}
\end{align*}
so our above expression becomes:
\[B(T) = \prod_{i = 1}^M \left( \frac{p_i(T)}{s_i(T)}\right)^{b_i}\]
with $b_i = \frac{a_i}{i} \in \Z$ by Lemma $\ref{lem:a_w_div}$. Now note that if $p, s$ are supersingular, so are $p_i(T)$ and $s_i(T)$ and thus $B(T)$. 
\end{proof}


\begin{corollary}
Let $V$ be the weighted projective space over $\F_q$ defined to be the zero set of 
\[x^{r_1} + x^{r_2} = 0\]
Then $V$ is supersingular over $\F_{q^i}$ for some $i$.
\end{corollary}
\begin{proof}
Let $X$ be the same curve just over affine space instead of projective space. Using our notation from before, note that $|C_{[0,1]}|=|C_{[1,0]}| = 0$ and $|C_{[0,0]}| = 1$ and thus $|C_{[1,1]}|  = |X| - 1$. By our definitions $d_{[1,1]} = 1$. Thus:
\[|V| = \frac{|X| - 1}{q - 1}\]
Let $R = \gcd(r_1, r_2)$. By Lemma $\ref{thm:affine_reduct}$, $|X| = |X'|$ where $X'$ is the set of solutions to
\[x_1^{R} + x_2^{R} = 0\]
over $\F_q$. There is one solution where one of the components is 0. If $x_1, x_2 \neq 0$, this equation is equivalent to:
\[(x_1x_2^{-1})^R = -1\]
If $y^R = -1$ has no solutions in $\F_q$, the number of solutions is 0. If it does have a solution, then it has precisely $\gcd(R, q - 1)$ solutions. In which case there are $(q-1)\gcd(R, q - 1)$ solutions as there are $R$ choices for which root $x_1x_2^{-1}$, $q - 1$ choices for $x_1$ and then 1 choice for $x_2$. In net, $|V| = \gcd(R, q - 1)$ if $y^R = -1$ has a solution as 0 otherwise. $y^R = - 1$ will have a solution if and only if $2\gcd(R, q - 1) | q - 1$. 
\par
Now consider when $y^R = - 1$ has a solution over various $\F_{q^k}$. As this will depend on what the highest power of 2 divising $q^k - 1$ is (we need $v_2(q^k - 1) \ge v_2(R) + 1$), there will exist an $i$ such that $y^R = -1$ has a solution if and only if $i | k$. Thus, over $\F_{q^i}$,
\[\zeta_V = \sum_{k \ge 1} \gcd(R, q^{ik} - 1)\frac{T^k}{k}\]
which is supersingular by theorem $\ref{gcd_power}$.
\end{proof}


\section{Conjectures and Other Theorems}
\begin{theorem} \label{thm:base_field_shifting}
Let $X$ be a variety. If $X$ is supersingular over $\F_q$ then it is supersingular over $\F_{q^k}$. Furthermore, if $X$ is nonsingular (weighted) projective and defined by the reduction modulo $p$ of a nonsingular variety over a number field, then if it is supersingular over $\F_{q^k}$ it is also supersingular over $\F_q$.
\end{theorem} 
\begin{proof}
Let $\zeta_X$ be the zeta function of $X$ over $\F_q$:
\[\zeta_X = \exp\left(\sum_{i\ge 0} a_i \frac{T^i}{i}\right)\]
Then the zeta function $\zeta_{X_k}$ for $X$ over $\F_{q^k}$ is:
\[\zeta_{X_k} = \exp\left(\sum_{i\ge 0}^{\infty} a_{ik} \frac{T^i}{i}\right)\]
Let
\[A(T) = \sum_{i\ge 0} a_i \frac{T^i}{i}\]
Let $\zeta$ be a $k$-th root of unity. Then 
\begin{align*} \frac{\sum_{j = 1}^k A(T\zeta^j)}{k} &= \sum_{i \ge 0} a_{ik} \frac{T^{ik}}{ik} \\
\sum_{j = 1}^k A({T^{1/k}}\zeta^j) &= \sum_{i \ge 0} a_{ik} \frac{T^{i}}{i} \end{align*}
And thus:
\[\zeta_{X_k} = \prod_{j = 1}^k \zeta_X(T^{1/k}\zeta^j)\]
Now suppose 
\[\zeta_X = \frac{P(T)}{S(T)} = \frac{\prod_{i = 1}^m (T - r_i)}{\prod_{i = 1}^m (T - s_i)}\]
Then 
\[\zeta_{X_k} = \pm\frac{\prod_{i = 1}^m (T - r_i^k)}{\prod_{i = 1}^m (T - s_i^k)}\]
which implies that $\zeta_{X_k}$ is supersingular if $\zeta_X$ is.
\par
We'll now do the second part. WLOG assume $\frac{P}{S}$ is in simplest form. Note that the only way $\zeta_{X_k}$ is supersingular but $\zeta_X$ is not is if the roots that do not have complex unit part a root of unity cancel in $\zeta_{X_k}$. However, by the fourth part of the weil conjectures, the numerator and denominator of the rational functions of $\zeta_X$ and $\zeta_{X^k}$ have the same degree. Thus there is no cancellation, and so $\zeta_X$ is supersingular.
\end{proof}



\begin{comment}
\begin{theorem}
Given \[x_0^{n_0} + \dots + x_3^{n_3} = 0\] over field $F_p$, there exists d such that the variety is unirational if $q \equiv -1 \mod d$, where $d = \text{gcd}(n_0, \dots, n_3)$. 
\end{theorem}
\begin{proof}
Suppose that $n_i = d k_i$, and $i = 0,1,2,3$ and let $w_i = x_i^{k_i}$.
\end{proof}
\end{comment}

\begin{theorem}
Given \[x_0^{n_0} + \dots + x_3^{n_3} = 0\] over field $F_p$, there exists d such that the variety is unirational if $q \equiv -1 \mod d$, where $d = \text{lcm}(n_0, \dots, n_3)$.
\end{theorem}
\begin{proof}
Given \[x_0^{n_0} + \dots + x_3^{n_3} = 0,\] let $l = \text{lcm}(n_0, n_1,n_2,n_3)$
Let $x_i' = x_i^{l/n_i}$. Then we get a homogeneous equation of degree $l$, which is unirational over $\F_p$ if there exists a $v$ such that $p^v \equiv -1 \mod l$ by Shioda's paper.
\end{proof}

\begin{comment}
\begin{example}
The automorphism of the weighted projective space $WP_k^n$. Suppose that the weight of this space is $w_0, \dots, w_n$. Then if $\{a_{ij}\}$ is an invertible $(n+1) \times (n+1)$ matrix of elements of a field $k$, then $x_i' = \lambda^{w_i}\sum a_{ij}x_j$ determines an automorphism of the weighted projective space $WP_k^n$, and this automorphism can be written in matrix form as 
\[\begin{bmatrix}
    a_{00} & \dots  & a_{0n} \\
    \vdots & \vdots & \vdots \\
    a_{n0} & \dots  & a_{nn}
\end{bmatrix} \cdot 
\begin{bmatrix}
\lambda^{w_0} & \dots & \lambda^{w_n}
\end{bmatrix}
\]
So we can consider the group $\text{WPGL}(n,k) = \text{GL}(n+1,k)/\begin{bmatrix}
\lambda^{w_0} & \dots & \lambda^{w_n}
\end{bmatrix}k^\ast$.

Consider the points $(1,0,\dots,0), (0,1,\dots,0),\dots, (0,0,\dots, 1),(1,1,\dots, 1)$, we find that if $g \in \text{WPGL}(n,k)$ induces the trivial automorphism, then \[1 = \lambda^{w_i}a_{ii}; \\ \sum_{i\ne j} a_{ij} = 0 \text{ for each fixed } j = 0, \dots, n.\] This means that $g$ is of the form \[\begin{bmatrix}
    1/\lambda^{w_0} & 0 &\dots  & 0 \\
    0 & 1/\lambda^{w_1} & \dots & 0\\
    \vdots & \vdots & \ddots & \vdots\\
    0 & \dots  & \dots & 1/\lambda^{w_n}
\end{bmatrix}
\]
And we know how to count how many element of the group has trivial action. Namely, this number is identical to the number of elements of the form $\begin{bmatrix}
\lambda^{w_0} & \dots & \lambda^{w_n}
\end{bmatrix}$. 
But the problem that the orbit could have different sizes may still remain.

\end{example}
\end{comment}

\begin{theorem}
Let $X$ be the variety defined by \[a_0 x_0^{n_0} + \dots + a_r x_r^{n_r}.\] If all the exponents are coprime, then $X$ is isomorphic to the hyperplane $H_{r-1}$ in $\Pro^r$, where $r$ is the dimension of image of Veronese embedding.
\end{theorem}

\begin{proof}
Notice that $X$ is in the weighted projective space $\Pro(w_0, \dots, w_r)$. If $d = \lcm(n_0, \dots, n_r)$, then $w_i = d/n_i$, and we see that our equation has weighted homogeneous degree $d$. 
Then the image of our variety by Vernose embedding will be in $\Pro^R$, and the coordinate ring of the image is generated by $y_i = x_i^{n_i}$, and these elements only. \\
The reason is that a monomial $\prod x_i^{a_i}$ has weighted degree $d$ is and only if $\sum a_iw_i = d$, which is equivalent to \[\sum \frac{a_i}{n_i} = 1\] because we know $w_i = d/n_i$. 
And again, we can write this sum as \[\frac{a_0}{n_0} + \frac{A}{N} = \frac{a_0N + An_0}{n_0N} = 1, a_i \in \Z^+.\]
Since $n_0$ divides $a_0N + An_0$, we will have $n_0|a_0N$. But we assume that all the exponents are coprime, so $\gcd(n_0, N) = 1$, and $n_0|a_0$, so either $a_0 = 1$ or $a_0 = n_0$. We know that $a_0$ cannot be any larger because $\sum \frac{a_i}{n_i} = 1$. Therefore, we know that the only monomial that will appear in the image of Vernose embedding are of the form $y_i = x_i^{n_i}$, and there will be no other cross terms. Then we also know that the only relation that these new coordinate satisfies is the diagonal equation that we have, i. e., $y_0 + \dots + y_r = 0$. Since a variety is isomorphic to the image of the Vernose embedding, and the image of the Vernose embedding give us a hyperplane in $\Pro^r$, we know that $X$ is isomorphic to a hyperplane in $\Pro^r$. 
%\color{red}{With appropriate change of coordinate can we show that a hyperplane in $\Pro^r$ is isomorphic to \Pro^{r-1}?}
\end{proof}

\begin{theorem}
A variety $X$ defined by \[a_0 x_0^{n_0} + \dots + a_r x_r^{n_r}.\] in weighted projective space is singular in $\F_q$ if and only if (i) $q|n_i$ for some $i$, or (ii) in weighted projective space $\Pro(w_0, \dots, w_r)$, there exists a prime number $p$ such that set $x_j = 0$ when $p$ does not divide $n_j$, we get a new equation that has solution over $\F_q$.
\end{theorem}

\begin{proof}
First, if $q|n_i$ for some $i$, then the Jacobian ring for $X$ will be \[(n_0x_0^{n_0-1}, \dots, 0, \dots, n_r x_r^{n_r-1}).\] And we see that this ideal can be zero for some nonzero point. Thus $(i)$ is true.\\

Second, we claim that the only singular points of the weighted projective space $\Pro(w_0, \dots, w_r)$ are of the form \[\Sing_p \Pro(w_0, \dots, w_r) = \{x \in \Pro(w_0, \dots, w_r) : x_i \ne 0 \text{ only if } p | w_i\}\] for some prime p.

We contend that \[\Sing \Pro(w_0, \dots, w_r)  = \bigcup \Sing_p \Pro(w_0, \dots, w_r).\]



\begin{comment}
The reason is that when we try to find the local affine coordinate ring of $X$, we first do the change of variable $$x_i^{(0)} = \frac{x_i}{x_0^{w_i/w_0}},$$ 
Then \[\F_p(x_0^{(0)}, x_1^{(0)}, \dots, x_r^{(0)})\] is a homogeneous ring of degree 0. The problem we introduce is that by adjoining $x_0^{1/w_0}$ we get an algebraic extension of $\F_p$, and the coordinate ring is will defined up to the action of $\zeta$, where $\zeta$ is the $a_0$-th root of unity. 
\end{comment}

\end{proof}
\begin{corollary}
If X is singular over $\F_q$, then it is singular over $\F_q^k$.
\end{corollary}


\begin{theorem} \label{thm:resolution_of_sigularities}
Let $X$ be a variety defined by,
\[ a_0 x^{n_0} + \cdots + a_r x^{n_r} = 0 \]
over $\F_q$ where $q = p^f$ and let $\tilde{n}_i = \frac{n_i}{p^{v_p(n_i)}}$ i.e. $n_i$ with all powers of $p$ removed. Define the "base" variety $\bar{X}$ by the equation,
\[ a_0 x^{\tilde{n}_0} + \cdots + a_r x^{\tilde{n}_r} = 0 \]
over $\F_q$. Then $\bar{X}$ is smooth as an affine variety away from zero. Furthermore, There exits a bijective morphism $X \to \bar{X}$ so $\#(X) = \#(\bar{X})$ over each $\F_q$ and thus $\zeta_X = \zeta_{\bar{X}}$. 
\end{theorem}

\begin{proof}
Let $t_i = v_p(n_i)$. Let $\Frob_p : \finfield{q} \to \finfield{q}$ denote the Frobenius automorphism $x \mapsto x^p$. Now we define the Frobenius morphism $X \to \bar{X}$ via $(x_0, \cdots, x_r) \mapsto (\Frob_p^{t_0} (x_0), \cdots, \Frob_p^{t_r} (x_r)) = (x_0^{p^{t_0}}, \cdots, x_r^{p^{t_r}})$. This map is well defined because if,
\[ a_0 x_0^{n_0} + \cdots + a_r x_r^{n_r} = 0 \]
then we have,
\[  a_0 (x_0^{p^{t_0}})^{\tilde{n}_0} + \cdots + a_r (x_r^{p^{t_r}})^{\tilde{n}_r} = 0  \]
Clearly this map is a morphism and it is bijective because I can exhibit an inverse map, $(x_0, \cdots, x_r) \mapsto (\Frob_p^{-t_0} (x_0), \cdots, \Frob_p^{-t_r} (x_r))$. Therefore, $\#(X) = \#(\bar{X})$ over any $\F_q$. This implies that $\zeta_X = \zeta_{\bar{X}}$.
Furthermore, as an affine variety, $\bar{X}$ has Jacobian,
\[ (a_0 \tilde{n}_0 x_0^{\tilde{n}_0 - 1}, \cdots, a_r \tilde{n}_r x_r^{\tilde{n}_r - 1})  \]
Since $p \nmid \tilde{n}_i$ for the Jacobian to have rank zero we must have $a_i \tilde{n}_i x_i^{\tilde{n}_i - 1} = 0 \implies x_i = 0$ for each $i$. Therefore, $\bar{X}$ is smooth away from zero. 
\end{proof}


\section{Facts from Daniel Litt and Alex Perry?}

\begin{fact}
A variety is rational over affine space if and only if it is rational over weighted projective space.
\end{fact}

\begin{fact}
$\Pro(w,x,y,z) \cong \Pro(w,xd,yd,zd)$
\end{fact}
\begin{corollary}
The two varieties described in Theorem $\ref{thm:affine_reduct}
$ are isomorphic over weighted projective space
\end{corollary}
\begin{fact}
Let $X$ be the variety defined by the curve:
\[a_0x_0^{n_0} + \cdots + a_r x_r^{n_r} = 0\]
Let $L = \lcm(n_0,\ldots, n_r)$ and let $w_i = L/n_i$. If
\[\sum_i w_i - L > 0\]
then $X$ is rational.
\end{fact}


\section{On Zeta functions} 


\begin{definition}
For a $r$-tuple of exponents $n$,
\[ A_{n, q} = \left\{ (\alpha_0, \dots, \alpha_r) : 0 < \alpha_i < 1 \text{ and } d_i \alpha_i \in \Z \text{ and } \sum \alpha_i \in \Z \text{ where } d_i = \gcd{(n_i, q-1)} \right\} \]
\end{definition}

\begin{theorem} \label{thm:change_coeffs}
The variety $X$ defined by,
\[ x_0^{n_0} + \cdots + x_r^{n_r} = 0 \]
and the variety $X_a$ defined by,
\[ a_0 x_0^{n_0} + \cdots + a_r x_r^{n_r} = 0 \]
have equal zeta functions up to multiplication of the roots by $z^{\mathrm{th}}$-roots of unity where \[ z = [E : \F_q] \]
and $E$ is the splitting field of the polynomial,
\[ \prod_{i = 0}^r (x^n_i - a_i) \]
over $\F_q$.
\end{theorem}

\begin{proof}
Consider the variety $X_a$ defined over $E$. Each $a_i$ has all $n_i^{\mathrm{th}}$ roots so we can write $a_i = b_i^{n_i}$ for each $i$. Therefore, $X_a$ is defined by the polynomial equation over $E$,
\[ b_0^{n_0} x^n_0 + \cdots + b_r^{n_r} x_r^{n_r} = (b_0 x_0)^{n_0} + \cdots + (b_r x_r)^{n_r} = 0 \]
Therefore, over $E$ the varieties $X_a$ and $X$ are isomorphic via the linear $E$-map $(x_0, \cdots, x_r) \mapsto (b_0 x_0, \cdots, b_r x_r)$ so $\zeta_{X_E} = \zeta_{X_{a,E}}$. However, the zeta function over $E$ and over $\F_q$ are equal up to replacing each root and pole of $\zeta$ by a $z^{\mathrm{th}}$ root. Thus $\zeta_{X}$ and $\zeta_{X_a}$ are equal up to choices of $z^{\mathrm{th}}$ root and thus up to multiplications by $z^{\mathrm{th}}$ roots of unity. 
\end{proof}

\begin{theorem} \label{thm:decomp_zeta_function}
For the weighted projective variety (with points counted via the stack quotient) defined by 
\[ a_0 x_0^{n_0} + \dots + a_r x_r^{n_r} = 0 \] 
over $\F_q$ such that $q \equiv 1 \mod (\lcm(n_i))$, the zeta function of $X$ equals, 
\[
\zeta_X(t) = \prod_{i = 0}^{r-1} \frac{1}{1 - q^i t} \cdot 
\left[ \prod_{\alpha}  \bigg( 1 + (-1)^r B(\alpha)j_q(\alpha)t \bigg) \right]^{(-1)^{r}},
\] 
where $B(\alpha) = \chi_{\alpha_0}(a_0^{-1}) \dots \chi_{\alpha_r}(a_r^{-1})$ is a root of unity determined by $\alpha$ and the coefficients.
\end{theorem} 

\begin{proof}
Notice that $A_{n,\alpha}$, the set of all possible $(\alpha_i)$, is the same for $\F_{q^k}$ for any positive integer $k$. The reason is that \[q \equiv 1 \mod (\lcm(n_i)) \iff q \equiv 1 \mod n_i.\] Then $d_i = \gcd(n_i, q-1) = n_i$, and we know $d_i \leq n_i$, so $d_i$ will not increase as the size of field increase. Thus the set $A_{n,p}$ is completely determined by the situation in $\F_q$. And we shall determine $A_{n,p}$ explicitly later.
By Weil's paper, the formula for the number of solution over $F_{q}$ is 
\[N_1 = q^{r} + (q - 1) \sum_{\alpha \in A_{n,p}} B(\alpha) j_q(\alpha),\]
where, 
\[B(\alpha) = \chi_{\alpha_0}(a_0^{-1}) \dots \chi_{\alpha_r}(a_r^{-1}) \quad \text{and} \quad j_q(\alpha) = \frac{1}{q} g(\chi_{\alpha_0}) \dots g(\chi_{\alpha_r})\] 
are algebraic numbers depends on $r$-tuple $\alpha$.
Because the set of $\alpha$ for each extension of $\F_{q}$ are defined over $\F_{q}$ we can use the reduction formula,
\[g'(\chi_\alpha') = -[-g(\chi_\alpha)]^k\] where $g'$ is the gaussian sum in the extension $\F_{q^k}$. 
Furthermore, for $x \in \F_q$,
\[ \chi'_{\alpha}(x) = \chi_{\alpha}(x)^k\] Therefore, the number of solution in $\F_{q^k}$ is,
\[N_k = q^{rk} + (q^k - 1) \sum_{\alpha \in A_{n,p}} (-1)^{(r+1)(k+1)}B(\alpha)^k j(\alpha)^k.\]
Using the stack quotient, we get the formula for the number of solution in weighted projective space: 
\[ N_k' = \frac{N_k - 1}{q^k - 1} = \sum_{i = 0}^{r-1}(q^{ik}) + \sum_{\alpha \in A_{n,p}} (-1)^{(r+1)(k+1)}B(\alpha)^k j(\alpha)^k.\]
Thus, the zeta function becomes,
\begin{align*}
\zeta_X(t) &= \exp \left( \sum_{i = 0}^{r-1} \sum_{k = 1}^{\infty} \frac{q^{ik}}{k}t^k + \sum_{\alpha \in A_{n,p}} (-1)^{r+1} \sum_{k = 1}^{\infty} (-1)^{k(r + 1)} \frac{B(\alpha)^k j(\alpha)^k}{k} t^k \right)
\\
&= \exp \left( - \sum_{i = 0}^{r-1} \log\left[ 1 - q^i t \right] - (-1)^{r+1} \sum_{\alpha \in A_{n,p}} \log \left[ 1 - (-1)^{(r + 1)} B(\alpha) j(\alpha) t \right] \right)
\\
&= \prod_{i = 0}^{r-1} \frac{1}{1 - q^i t}\cdot \left[ \prod_{\alpha} \bigg(1 + (-1)^r B(\alpha)j(\alpha)t \bigg) \right]^{(-1)^{r}}
\end{align*}
\end{proof}

\begin{proposition}
Up to multiplying the roots by roots of unity, the zeta function of the weighted projective variety (with points counted via the stack quotient) defined by 
\[ a_0 x_0^{n_0} + \dots + a_r x_r^{n_r} = 0 \] 
over any $\F_q$ is equal to,
\[
\zeta_X(t) = \prod_{i = 0}^{r-1} \frac{1}{1 - q^i t} \cdot 
\left[ \prod_{\alpha}  \bigg( 1 + (-1)^r B(\alpha)j_q(\alpha)t \bigg) \right]^{(-1)^{r}},
\] 
where $B(\alpha) = \chi_{\alpha_0}(a_0^{-1}) \dots \chi_{\alpha_r}(a_r^{-1})$ is a root of unity determined by $\alpha$ and the coefficients.
\end{proposition}

\begin{proof}
By Theorem \ref{thm:base_field_shifting} we can reduce the zeta function for $X$ over $\F_q$ to zeta function for $X$ over $\F_{q^v}$, where $v = \ord_n(q)$ and $n = \lcm(n_i)$ such that $q^v \equiv 1 \mod (\lcm(n_i))$. We know that $\zeta_{X_q}$ is equal to $\zeta_{X_{q^v}}$ with each root $\beta$ replaced by $\beta^{1/v}$. Therefore, $\zeta_{X_q}$ is determined up to roots of unity by Theorem \ref{thm:decomp_zeta_function}.
\end{proof}

\begin{corollary} \label{cor:supersingular_iff_gauss_sum_products_roots_of_unity}
The variety $X$ is supersingular if and only if $j_q(\alpha) = \omega q^{\frac{r-1}{2}}$ where $\omega$ is a root of unity for each $\alpha \in A_{n,q^v}$. 
\end{corollary}

\begin{proof}
By Theorem \ref{thm:decomp_zeta_function} the roots and poles of the zeta function have the form $(-1)^r B(\alpha) j_q(\alpha)$ or $q^i$. Since $B(\alpha)$ is a product of characters it is always a root of unity. Therefore, each root of $\zeta_X$ has argument a root of unity if and only if $j_q(\alpha)$ does for each $\alpha$. 
\end{proof}

\begin{corollary}
Note that $|g(\chi_\alpha)| = q$ and thus,
\[ | j_q(\alpha) | = \frac{1}{q} | g(\chi_{\alpha_0})| \cdots | g(\chi_{\alpha_r})| = \frac{1}{q} q^{\frac{r+1}{2}} = q^{\frac{r-1}{2}} \]
Since the characters are roots of unity,
\[ \left| (-1)^{(r + 1)} B(\alpha) j(\alpha) \right| = q^{\frac{r-1}{2}} \]
By the Riemann hypothesis, each of the $\alpha$-derived roots are roots of $P_{r-1}$ in Weil's factorization of the zeta function. If $r-1$ is even then a factor of $(1 - q^{\frac{r-1}{2}} t)$ from the zeta function of $\P^r$ will also appear in $P_{r-1}$. Therefore, we can write,
\[ \zeta_X = \zeta_{\P^r} \cdot \tilde{P}_{r-1}^{(-1)^r} \]
where $\zeta_{\P^r}$ is the zeta function of projective $r$-space and,
\[ \tilde{P}_{r-1}(t) = \prod_{\alpha} \bigg(1 + (-1)^r B(\alpha)j(\alpha)t \bigg) \]
Therefore, we can write the Weil factorization of $\zeta_X$ as,
\[ P_i(t) = 
\begin{cases}
1 - q^{\frac{i}{2}} t  & 0 \le i \le 2(r-1) \text{ is even and } i \neq r - 1 \\
(1 - q^{\frac{r-1}{2}} t) \cdot \tilde{P}_{r-1}(t) & i = r-1 \text{ is even} \\
 \tilde{P}_{r-1}(t) & i = r - 1 \text{ is odd}
\end{cases}
\]
\end{corollary}

\begin{remark}
The only interesting cohomology group is $H^{r-1}$ which shows up in the dimension of the surface.
\end{remark}

\begin{theorem}
Let $X$ be the weighted projective variety (with points counted via the stack quotient) defined by 
\[ a_0 x_0^{n_0} + \dots + a_r x_r^{n_r} = 0 \] 
over any $\F_q$. Then the Betti numbers are determined,
\[ 
\dim H^i(X) = 
\begin{cases}
1  & 0 \le i \le 2(r-1) \text{ is even and } i \neq r - 1 \\
|A_{n,q}| + 1 & i = r-1 \text{ is even} \\
|A_{n,q}| & i = r - 1 \text{ is odd}
\end{cases}
\]
\end{theorem}

\begin{proof}
By Theorem \ref{thm:base_field_shifting}, changing the base field only changes the zeta function by multiplying its roots by roots of unity. In particular, the magnitudes of the degrees of each $P_i$ and thus the Betti numbers are not changed. Therefore, given $X$ defined over $\F_q$ take $v = \ord_n(q)$ and $n = \lcm(n_i)$ such that $q^v \equiv 1 (\mod n)$. Then we know that $\zeta_{X_{p^v}}$ factors with,  
\[ P_i(t) = 
\begin{cases}
1 - q^{\frac{i}{2}} t  & 0 \le i \le 2(r-1) \text{ is even and } i \neq r - 1 \\
(1 - q^{\frac{r-1}{2}} t) \cdot \tilde{P}_{r-1}(t) & i = r-1 \text{ is even} \\
 \tilde{P}_{r-1}(t) & i = r - 1 \text{ is odd}
\end{cases}
\]
Therefore, the Betti numbers of $X$ which are equal to the Betti numbers of $X_{p^v}$ are equal to the degrees of these polynomials. 
\end{proof}

\begin{remark}
Notice that whether a variety is supersingular or not is now determined explicitly by one computation of Gaussian sum.
\end{remark}
 
\begin{proposition}
If $\alpha_1 + \alpha_2 = 1$, then $g(\chi_{\alpha_1})g(\chi_{\alpha_2}) = \chi_{\alpha_1}(-1) p$.
\end{proposition}
\begin{proof}
Notice that if $\alpha_1 + \alpha_2 = 1$, then $\chi_{\alpha_1} = \overline{\chi_{\alpha_2}}$. We know that 
\begin{align*}
g(\chi)g(\overline \chi) &= \sum_{x \ne 0}\sum_{y \ne 0} \chi(xy^{-1}) \psi(x+y)\\
&= \sum_{x \ne 0} \chi(x)\sum_{y \ne 0} \psi[(x+1)y]
\end{align*}
The second sum has the value $p-1$ for $x = -1$, and $-1$ when $x \ne 0$. As sum over all $x \in k^*$ is $0$, we get $g(\chi_{\alpha_1})g(\chi_{\alpha_2}) = \chi_{\alpha_1}(-1) p$.
\end{proof}
In our example when $n = 4$ and $\alpha_1 = 1/4$, $\chi_{1/4}(-1) = 1$ if $p \equiv 1 \mod 8$, and $\chi_{1/4}(-1) = -1$ otherwise.


\begin{fact}
Let $K = \Q(\zeta_n)$ be a cyclotomic field. Then $\iO_{K}$ is a PID if and only if $n = m$ or, when $m$ is odd, $n = 2m$ where $m$ is one of the following,
\[ 1, 3, 4, 5, 7, 8, 9, 11, 12, 13, 15, 16, 17, 19, 20, 21, 24, 25, 27, 28, 32, 33, 35, 36, 40, 44, 45, 48, 60, 84 \]
\end{fact}

\begin{lemma}[Coyne]
Let $d = \lcm{(n_i)}$ and $w_i = d/n_i$ then,
\[ \# \left\{ (x_0, \cdots, x_r) : \sum_{i = 0}^{r} w_i x_i \equiv 0 \mod{(d)} \text{ and } 0 \le x_i < n_i  \right\} = \frac{1}{\lcm{(n_i)}} \prod\limits_{i = 0}^r n_i  \] 
\end{lemma}

\begin{proof}
Consider the homomorphism,
\[ \Phi : \prod_{i = 0}^r (\Z / n_i \Z) \to \Z / d \Z \]
via $(x_0, \cdots, x_r) \mapsto w_0 x_0 + \cdots + w_r x_r$. Thus,
\[ \ker{\Phi} = \left\{ (x_0, \cdots, x_i) : \sum_{i = 0}^{r} w_i x_i \equiv 0 \mod{(d)} \text{ and } 0 \le x_i < n_i  \right\} \]
Suppose that $p^r \divides \divides d$ then we know that $p^r \divides \divides n_i$ for some $n_i$. Thus, $p \nmid w_i$ so each prime dividing $d$ cannot divide all $w_i$. However, $w_i \divides d$ so the list $w_0, \cdots, w_r$ cannot share any common factors. Thus, the ideal $(w_0, \cdots, w_r) = \Z$ so the map $\Phi$ is surjective. Therefore, by the first isomorphism theorem,
\[ \#(\ker{\Phi}) = \#\left( \prod_{i = 0}^r \Z / n_i \Z \right) \bigg/ \#(\Z / d \Z) = \frac{1}{d} \prod_{i = 0}^r n_i \]
\end{proof}

\begin{lemma}
The number of alphas $A_{n,q}$ is given by the formula,
\[ \#(A_{n,q}) = \sum_{t \in T} \frac{(-1)^{r + 1 - sum(t)}}{\lcm{(d_i \mid t_i = 1)}} \prod_{i \in \{ i :  t_i = 1 \} } d_i \] 
where $d_i = \gcd{(n_i, q - 1)}$. 
\end{lemma}

\begin{proof}
For each $t \in T$, define the number,
\[ C_t = \# \left\{ (x_0, \cdots, x_r) : \sum_{i = 0}^{r} w_i x_i \equiv 0 \mod{\lcm(d_i)} \text{ and } 0 \le x_i < d_i \text{ and } x_i = 0 \text{ if } t_i = 0 \right\} \]
By inclusion-exclusion,
\[ \#(A_{n,q}) = \# \left\{ (x_0, \cdots, x_r) : \sum_{i = 0}^{r} w_i x_i \equiv 0 \mod{\lcm{(d_i)}} \text{ and } 0 < x_i < d_i \right\} = \sum_{t \in T} (-1)^{r+1 - sum(t)} C_t \]
However, letting,
\[ g = \frac{\lcm{(d_i)}}{\lcm{(d_i \mid t_i = 1)}} \]
then we know that $g \divides w_i$ for $t_i = 1$ since $w_i = \lcm{(d_i)} / d_i$ and thus,
\[ \tilde{w}^t_i = \frac{w_i}{g} = \frac{\lcm{(d_i \mid t_i = 1)}}{d_i} \in \Z \]
since $d_i$ is such that $t_i = 1$. Therefore, the conditions,
\[ \sum_{i = 0}^r w_i x_i \equiv 0 \mod \lcm{(d_i)} \iff \sum_{i = 0}^r \tilde{w}^t_i x_i \equiv 0 \mod \lcm{(d_i \mid t_i = 1)} \] 
are equivalent when $x_i = 0$ for $t_i = 0$. 
By Coyne's Lemma,
\[ C_t = \frac{1}{\lcm{(d_i \mid t_i = 1 )}} \prod_{i \in \{ i :  t_i = 1 \} } d_i \] 
and thus the lemma follows. 
\end{proof}


\section{On Gaussian Sums}

\subsection{Previously Known Facts and Some Lemmas}
\begin{theorem}
\label{thm:chowla_root_unity}
$g(\chi_{\alpha}) = \omega q^{\frac{1}{2}}$ where $\omega$ is a root of unity if and only if $\alpha = 1, \tfrac{1}{2}$. 
\end{theorem}

\begin{proof}
See Chowla.
\end{proof}

\begin{lemma} 
\label{lem:g_power_m} Let $\chi$ be a character on $\F_q$ of order $m$. Then $g(\chi)^m \in \Q(\zeta_m)$.
\end{lemma}

\begin{proof} 
Well-known fact. See Evans' generalization of Chowla's paper.
\end{proof}

\begin{lemma} \label{lem:galois_gauss_sum} Let $\chi$ be a character of order $m$ on $\F_q$ for $q = p^r$. Let $K = \Q(\zeta_{pr})$ with $m | r$ and $a$ an integer 1 $\pmod m$ with $(a,2p(q-1)) = 1$. Let $\sigma\in \Gal(K/\Q)$ be the element such that
\[\sigma(\zeta_{2p(q-1)}) = \zeta_{2p(q-1)}^a\]
Then $\sigma(g(\chi)) = \bar{\chi}(a)g(\chi)$.
\end{lemma}

\begin{proof}
Let $\psi$ be the nontrivial additive character such that:
\[g(\chi) = \sum_{a \in \F_q} \chi(a)\psi(a)\]
Note that $\psi(x)^p = \psi(px) = \psi(0) = 1$. Thus $\psi(x) = \zeta_p^{t(x)}$ for $t: \F_q \to \Z$. We can select $\zeta_p$ to be the $p$-th root of unity so that $t(1) = 1$. Note that as $\psi(x + y) = \psi(x)\psi(y)$, $t(x + y) = t(x) + t(y)$. Thus as $a$ is an integer $t(a) = a$ and $t(ax) = at(x)$.
\[\sigma(\psi(x)) = \sigma(\zeta_p)^{t(x)} = \zeta_p^{at(x)} = \zeta_p^{t(ax)} = \psi(ax)\]
If $w$ is a generator of $\F_q^{\times}$, as $a \equiv 1 \pmod m$ and $\chi$ has order m, we have $\sigma(\chi(w)) = \chi(w)^a = \chi(w)$. Thus as $\chi$ is nontrivial,

\begin{align*}
\sigma(g(\chi)) &= \sum_{x \in \F_q^{\times}} \sigma(\chi(x))\sigma(\psi(x)) \\
&= \sum_{x \in \F_q^{\times}} \chi(x)\psi(ax) \\
\end{align*}
Making the substitution $ax \mapsto x$ gives,
\begin{align*}
\sigma(g(\chi)) &= \sum_{x \in \F_q^{\times}} \chi(a^{-1}x)\psi(x) \\
&= \bar{\chi}(a)\sum_{x \in \F_q^{\times}} \chi(x)\psi(x) \\
&= \bar{\chi}(a)g(\chi)
\end{align*}
\end{proof}


\begin{theorem} \label{thm:gauss_fact_lang}[See Lang's Algebraic Number Theory]
Let $\mathfrak{p}$ be a prime lying over $p$ in $\Q(\zeta_{m})$ and let $\mathfrak{P}$ be a prime lying over $\mathfrak{p}$ in $\mathbb{Q}(\zeta_{m}, \zeta_p)$. Let $f$ be the order of $p$ modulo $m$ and $q = p^f$. Let $\chi$ be a character of $\F=\F_q$ such that
\[\chi(a) \equiv a^{-(q-1)/m} \pmod{\mathfrak{p}}\]
Then for any integer $r \ge 1$ we have:
\[\tau\left(\chi^r\right) \sim \mathfrak{P}^{\alpha(r)}\]
where 
\[\alpha(r) = \frac{1}{f}\sum_{\mu} s\left(\frac{(q-1)\mu r}{m}\right)\sigma_{\mu}^{-1}\]
where the summation runs over all $0 < \mu < p-1$ relatively prime to $p-1$ and where $s(v)$ is the sum of the digits of the $p$-adic expansion of $v$ modulo $q - 1$. Furthermore, if $\mu, \mu'$ are such that $\sigma_{\mu}^{-1}\mathfrak{P} = \sigma_{\mu'}^{-1}\mathfrak{P}$ then
\[s\left(\frac{(q-1)\mu r}{m}\right) = s\left(\frac{(q-1)\mu' r}{m}\right)\]
\end{theorem}


\begin{remark} If $f = 1$, then $\sigma_{\mu}^{-1} \mathfrak{P}$ is distinct for all $\mu\in(\Z/m\Z)^{\times}$. In general, by cyclotomic reciprocity, there are $\frac{\phi(m)}{f}$ distinct values of $\sigma_{\mu}^{-1} \mathfrak{P}$ as $\mu$ ranges over all the elements of $(\Z/m\Z)^{\times}$
\end{remark}

\begin{lemma}
\[ s\left( v \right) = (p-1) \sum_{i = 0}^{f-1} \left\{ \frac{p^i v}{q - 1} \right\} \]
\end{lemma}

\begin{theorem} \label{thm:evans_product_root}(From Evans' Chowla Generalization) Let $\chi, \psi$ be two multiplicative characters modulo $p$ of order $> 2$. Then $g(\chi)^jg(\psi)^k$ has argument a root of unity if and only if $j = k$ and $\chi = \bar{\psi}$ or $j = 2k$, $\chi = \bar{\psi}^2$ and $\psi$ has order 6.
\end{theorem}

\subsection{Jacobi Sums}

\begin{proposition}
Let $J(\chi_1, \chi_2) = \sum_x \chi_1(x) \chi_2(1-x)$, where $\chi$ is a character of $\F_q$. If $\chi_1\chi_2 \ne 1$, then 
\[J(\chi_1, \chi_2) = \frac{g(\chi_1)g(\chi_2)}{g(\chi_1\chi_2)}\].
\end{proposition}
\begin{proof} 
\begin{align*}
g(\chi_1)g(\chi_2) &= \sum_x\sum_y \chi_1(x)\chi_2(y) \psi(x+y) \\
&= \sum_x \sum_y \chi_1(x)\chi_2(y-x)\psi(y) \\ 
&= \sum_x \sum_{a \ne 0} \chi_1(x)\chi_2(a-x)\psi(a) + \sum_x \chi_1(x)\chi_2(-x) \\ &= (\sum_a \chi_1\chi_2(a) \psi(a)) \cdot (\sum_x \chi_1(x)\chi_2(1-x)) 
\end{align*}
\end{proof}

\begin{proposition}
If $\chi_1 \dots \chi_4 |_{\finunits{q}} = \chi_0$ where $\chi_0$ is the trivial character then, \[g(\chi_1) \dots g(\chi_4) = J(\chi_1, \chi_2)J(\chi_3,\chi_1\chi_2) \chi_4(-1)q\]
\end{proposition}

\subsection{Products of Gauss Sums}
\begin{theorem}
Let $\chi_1, \ldots, \chi_n$ be nontrivial characters on $\F_q$ for $q = p^r$ with $p$ an odd prime. If $n$ is even and $\chi_1\cdots \chi_n|_{\F_p^{\times}}$ is not the trivial character or $n$ is odd and $\chi_1\cdots \chi_n|_{\F_p^{\times}}$ is not -1 or 1 everywhere, then 
\[\prod_{i = 1}^n g(\chi_i)\]
does not have argument equal to a root of unity.
\end{theorem}
\begin{proof} (adapted from theorem 1 in Evans' Generalizations of Chowla paper)
\\
Let $L$ be the lcm of the orders of the $\chi_i$. Let 
\[G = \prod_{i = 1}^n g(\chi_i)\]
By Lemma \ref{lem:g_power_m},  $g(\chi_i)^L \in \Q(\zeta_L)$. Thus $G^L \in \Q(\zeta_L)$. Let $\epsilon$ be the number of order 1 such that $G = q^{n/2}\epsilon$. Now suppose $G$ does have argument equal to a root of unity. As $G^L \in \Q(\zeta_L)$, $G^L$ must be a $2L$-th root of unity. Thus $\epsilon = \zeta_{2L^2}^v$ for some integer $v$. 
\par
Now let $a$ be an integer such that $a \equiv 1 \pmod 2L^2$ and $a \equiv g^{-1} \pmod p$ where $g$ is a generator modulo $p$. Note that such an $a$ exists as $L | q-1$ and hence must be relatively prime to $p$. Now consider the Galois group $Gal(\Q(\zeta_{2pL^2})/\Q(\zeta_{2L^2}))$ and the element $\sigma$ contained in it such that:
\[\sigma(\zeta_{2pL^2}) = \zeta_{2pL^2}^a\]
This is a well-defined element as $(a,2pL^2) = 1$ $a \equiv 1 \pmod 2L^2$ so it fixes $\Q(\zeta_{2L^2})$. Note that as $\epsilon$ is a $2L^2$-th root of unity $\sigma(\epsilon) = \epsilon$. Furthermore, $\sigma(\sqrt(q)) = \pm \sqrt{q}$. As
\[\sigma(G) = \sigma(q^{n/2})\sigma(\epsilon)\]
So $\sigma(G) = G$ if $n$ is even and $\sigma(G) = \pm G$ if $n$ is odd. However, we also have by lemma \ref{lem:galois_gauss_sum},
\[\sigma(G) = \prod_{i = 1}^n \sigma(g(\chi_i)) = \prod_{i = 1}^n \chi_i(a^{-1})g(\chi_i) = G\prod_{i = 1}^n \chi_i(a^{-1}) G\prod_{i = 1}^n \chi_i|_{\F_p}(g)\]
Hence if $n$ is even,
\[\prod_{i = 1}^n \chi_i|_{\F_p}(g) = 1\]
and if $n$ is odd,
\[\prod_{i = 1}^n \chi_i|_{\F_p}(g) = \pm 1\]
Thus, as $g$ is a generator, $\prod_{i = 1}^n \chi_i|_{\F_p}$ must be the trivial character if $n$ is even and take value $\pm 1$ everywhere if $n$ is odd.
\end{proof}



\begin{proposition}
If $\chi_1, \chi_2$ are two different nontrivial character on $\F_q$ of same order, and \[\mu = g^j(\chi_1)g^k(\chi_2)q^{(j+k)/2} \in U,\] where $q = p^r$, and $j \ne k$, $g(\chi)$ is gauss sum on $\F_q$, $U$ denote the group of all root of unity, then in $\Q(\zeta_{p(q-1)})$, we have $(q^{1/2})$ divides $(g(\chi_i))$, i.e., 
\[\iO g(\chi_1) = \iO(q^{1/2}) \ia.\]
\end{proposition}
\begin{proof}
Notice that \[\mu = \frac{g^j(\chi_1)  \chi_2^k(-1)}{q^{(j-k)/2}g^k(\overline{\chi_2})}.\] And \[V(g(\chi_1)) = V(g(\chi_2)) = \text{min}_{(a,q-1) = 1} \s{\frac{a(q-1)}{m}}\] But we also have $V(g^j(\chi_1)) = V(q^{(j-k)/2}g^k(\overline{\chi_2}))$, while $V(q^{1/2}) = (p-1)r/2$. This give us the result.
\end{proof}
\begin{remark}
When is $e_i = (p-1)r/2$ for each $i$? Let us just act by Galois group again. 
\end{remark}
\begin{remark}
When is the conjugate of a gauss sum a gauss sum?
Why is the equation \[\sigma_a(G_r(\chi)) = \overline{\chi}(a) G_r(\chi) ?\]
\end{remark}

\begin{lemma}
If $K/\Q$ is abelian then $|\sigma(z)|^2 = \sigma(|z|^2)$ for all $\sigma \in \Gal(K/\Q)$. In particular, if $|z|^2 \in \Q$ then $\sigma(|z|^2) = |z^2|$ and thus $|\sigma(z)| = |z|$.
\end{lemma}

\begin{proof}
Since $K/\Q$ is Galois complex conjugation $\tau : K \to K$ is an automorphism fixing $\Q$ so $\tau \in \Gal(K/\Q)$. Furthermore, $|\sigma(z)|^2 = \sigma(z) \tau(\sigma(z)) = \sigma(z) \sigma(\tau(z)) = \sigma(z \tau(z)) = \sigma(|z|^2)$ since $\Gal(K/\Q)$ is abelian. 
\end{proof}

\begin{lemma} \label{lem:all_conjugates_one_then_rootofunity}
Let $K$ be a number field and $z \in \iO_K$ such that $|\sigma(z)| = 1$ for all $\sigma \in \Gal(K/\Q)$ then $z$ is a root of unity.
\end{lemma}

\begin{proposition} \label{prop:gauss_sum_prod_alg_int}
The element $q^{-(r+1)/2} g(\chi_0) \dots g(\chi_r)$ is an algebraic integer if and only if it is a root of unity.
\end{proposition}
\begin{proof}
We know that $|q^{-(r+1)/2} g(\chi_0) \dots g(\chi_r)| = 1$ and since $\sigma$ takes $g(\chi)$ to another Gaussian sum which must also have magnitude $q^{\frac{1}{2}}$ we know that,
\[ |\sigma(q^{-(r+1)/2} g(\chi_0) \dots g(\chi_r))| = |\sigma(q^{-(r+1)/2})| |\sigma(g(\chi_0))| \cdots |\sigma(g(\chi_r))| = | \pm q^{-(r+1)/2} | q^{(r+1)/2} = 1 \]
Thus, if $q^{-(r+1)/2} g(\chi_0) \dots g(\chi_r)$ is an algebraic integer then by Lemma \ref{lem:all_conjugates_one_then_rootofunity} we know that $q^{-(r+1)/2} g(\chi_0) \dots g(\chi_r)$ is a root of unity. Conversely, if $q^{-(r+1)/2} g(\chi_0) \dots g(\chi_r)$ is a root of unity then clearly it is an algebraic integer. 
\end{proof}

\begin{corollary}
\label{cor:gauss_sum_prod_principal_ideal}
The element $q^{-(r+1)/2} g(\chi_0) \dots g(\chi_r)$ is a root of unity if and only if the principal fractional ideal generated by it in $K = \Q(\zeta_{m}, \zeta_p)$ is $\iO_K$ if and only if it is an algebraic integer.
\end{corollary}

\begin{proof} If it is a root of unity, then the ideal generated will be $\iO_K$. If it is not a root of unity, by the Proposition \ref{prop:gauss_sum_prod_alg_int} it is not an algebraic integer. Thus the ideal cannot be $\iO_K$.
\end{proof}

\begin{remark}
By Stickelberger's theorem, we can determine exactly when $q^{-(r+1)/2}g(\chi_0) \dots g(\chi_r)$ is a unit. 
\end{remark}

\begin{theorem} \label{thm:gauss_sum_is_root_of_unity_ideal_factorization_counting_condition}
Let $p$ be an odd prime (or $r + 1$ is even) and $q = p^f$. The normalized product $\omega = q^{-\frac{r+1}{2}} g(\chi^{e_0}) \cdots g(\chi^{e_r})$ is a root of unity if and only if,
\[ \sum_{i = 0}^r s\left(\frac{(q-1) \mu e_i}{m}\right) = \frac{r+1}{2} (p - 1) f \]
for each $\mu \in (\Z / m \Z)^\times$.
\end{theorem}

\begin{proof}
Consider the ideals generated by $g(\chi^{e_0}) \cdots g(\chi^{e_r})$ and by $q^{\frac{r+1}{2}}$ respectivly. By Lang's formula, we know the Gaussian sum factors into prime ideals as,
\[ (g(\chi^{e_0}) \cdots g(\chi^{e_r})) = \mathfrak{P}_1^{D_1} \cdots \mathfrak{P}_{w}^{D_{w}} \]
where,
\[ D_j = \sum_{i = 0}^r s\left(\frac{(q-1) \mu e_i}{m}\right) \]
Lang's formula contains a factor of $f^{-1}$. However, $\sigma_\mu^{-1} \mathfrak{P}$ ranges over each prime above $p$ a total of $f$ times because the decomposition group has order $f$. The sets of $\sigma_\mu$ mapping to a fixed prime are exactly the cosets of the decomposition groups of which there are $w = \phi(m)/f$. In the field $K = \Q(\zeta_m, \zeta_p)$ the ideal $(p)$ factors as,
\[ (p) = \mathfrak{P}_1^{(p-1)} \cdots \mathfrak{P}_w^{(p-1)} \] Therefore, since $\Q(\sqrt{p}) \subset \Q(\zeta_p)$ for $p$ an odd prime, the ideal $(q^{\frac{r+1}{2}}) = (p^{\frac{r+1}{2} f})$ fractors into primes as,
\[ (q^\frac{r+1}{2}) = (p^\frac{r+1}{2})^f = \mathfrak{P}_1^{\frac{r+1}{2} (p-1) f} \cdots \mathfrak{P}_w^{\frac{r+1}{2} (p-1) f} \]
Therefore, the principal fractional ideal genreated by $\omega$ factors as,
\[ (\omega) = (q^{\frac{r+1}{2}})^{-1} (g(\chi^{e_0}) \cdots g(\chi^{e_r})) = \mathfrak{P}_1^{D_1 - \frac{r+1}{2} (p-1) f} \cdots \mathfrak{P}_{w}^{D_{w} - \frac{r+1}{2} (p-1) f } \] 
Which implies that $\omega \in \iO_K$ if and only if,
\[ D_w =\sum_{i = 0}^r s\left(\frac{(q-1) \mu e_i}{m}\right) \ge \frac{r+1}{2} (p - 1) f \]
such that the fractional ideal it generates is an actual ideal of $\iO_{K}$. However, by Proposition \ref{prop:gauss_sum_prod_alg_int}, $\omega \in \iO_K$ if and only if $\omega$ is a root of unity. In particular, if $\omega \in \iO_K$ then $\omega$ is a unit. Therefore, $\omega$ is a root of unity if and only if,
\[ \sum_{i = 0}^r s\left(\frac{(q-1) \mu e_i}{m}\right) \ge \frac{r+1}{2} (p - 1) f \]
for each $\mu \in (\Z / m \Z)^\times$
if and only if
\[ \sum_{i = 0}^r s\left(\frac{(q-1) \mu e_i}{m}\right) = \frac{r+1}{2} (p - 1) f \]
for each $\mu \in (\Z / m \Z)^\times$. 
\end{proof}


\begin{theorem}\label{thm:gauss_factor_SS}
Let $X$ defined by,
\[ a_0 x_0^{n_0} + \dots + a_r x_r^{n_r} = 0 \]
be a variety over $\F_{p^t}$. Let $n = \lcm(n_i)$. And consider it's zeta function over $\F_q$, where $q = p^f$ such that $f = \text{ord}_n(p)$. This means that $q \equiv 1 \mod n$. Then $X$ is supersingular over $\F_q$ if and only if 
\[ \sum_{i = 0}^{r} \s{ \frac{(q-1)\mu \ell_i}{n} } = \frac{r+1}{2}(p - 1)f,\]
for each,
\[ \ell \in \left\{(\ell_0, \dots, \ell_r) : \ell_i \in \Z \text{ and } n \divides \sum_{i = 0}^{r} \ell_r \text{ and } 0 < \ell_i < n \text{ and } n \divides \ell_i n_i \right\} \]
and each $\mu \in (\Z/n\Z)^\times$. Notice in Lang (p97) that if $\sigma_\mu(\Pf_j) = \Pf_j$, then $s\left(\frac{(q-1)\mu r_i}{n}\right) = s\left(\frac{(q-1) r_i}{n}\right)$.
\end{theorem}

\begin{proof}
When $q = p^f$, then $X$ is supersingular over $\F_p$ if and only if $X$ is supersingular over $\F_q$ if and only if $X$ is supersingular over $\F_{p^t}$. Thus, we need only consider the supersingularity of $X$ over $\F_q$. However, by Lang, the above condition gives that the product of each tuple of Gaussian sums generates the same ideal as $q^{\frac{r+1}{2}}$ and thus their ratio is a unit. By Proposition \ref{prop:gauss_sum_prod_alg_int}, this implies that each product has argument root of unity. Therefore, by Corollary \ref{cor:supersingular_iff_gauss_sum_products_roots_of_unity}, we know that $X$ is supersingular over $\F_q$.  
\end{proof}


\begin{theorem}
Let $\chi$ be a multiplicative character of order $p-1$ modulo $p$. Let $\chi^a, \chi^b, \chi^c$ be three multiplicative distinct characters modulo $p$ of order $> 2$. Then $g(\chi^a)g(\chi^b)g(\chi^c)^2$ does not have argument a root of unity.
\end{theorem}
\begin{proof} Assume $g(\chi^a)g(\chi^b)g(\chi^c)^2$ is a root of unity. To begin note that the unit part of $g(\chi^a)g(\chi^b)g(\chi^c)^2$ is:
\[p^{-2}g(\chi^a)g(\chi^b)g(\chi^c)^2 = \frac{g(\chi^a)g(\chi^b)\chi^{c}(-1)}{g(\chi^{-c})^2 } \]
Thus the above must be a root of unity. Now consider the principal ideal generated by it in $\Q(\zeta_{p-1}, \zeta_p)$. By Theorem $\ref{thm:gauss_fact_lang}$, for each $\mu$ relatively prime to $p-1$, the prime ideal $\sigma_{\mu}^{-1}\mathfrak{P}$ has index:
\[s(\mu a) + s(\mu b) - 2s(-\mu c) = 0\]
WLOG assume $0 < a,b < p-1$ and let $0 < d < p-1$ be such that $d \equiv -c \pmod p - 1$. As $s(\mu a) = (p-1)\left\{\frac{\mu c}{p - 1}\right\}$, the above is equivalent to:
\[\frp{\mu a}{p-1} + \frp{\mu b}{p-1} = 2\frp{\mu c}{p-1}\]
for all $\mu$ relatively prime to $p - 1$. Taking $\mu = 1$ gives $2d = a + b$. Now let $c', t$ be such that $t = \gcd(d, p-1)$ and $d = c't$. As $\chi^c$ has order $> 2$ we must have $t < \frac{p-1}{2}$. Now there exists $\nu < \frac{p-1}{t}$ such that $\nu d \equiv t \pmod {p-1}$ and $\nu$ is relatively prime to $\frac{p-1}{t}$. Furthermore, for each $k$ we will have $\left(\nu + \frac{p-1}{t}k\right) d \equiv \pmod{p-1}$. Taking $\mu = \nu + \frac{p-1}{t}k$ for some $k$ gives:
\[\frp{\left( \nu + \frac{p-1}{t}k\right) a}{p-1} + \frp{\left( \nu + \frac{p-1}{t}k\right)  b}{p-1} = \frac{2t}{p-1} < 1\]
This implies that for all $k$:
\[\frp{\nu a + \frac{p-1}{t}ka}{p-1} \le \frac{2t}{p-1}\]
and similarly for $b$. Now let $s = \gcd(a,t)$ and take $a = a's$. Then this becomes:
\[\frp{\nu a + \frac{(p-1)}{t/s}ka'}{p-1} \le \frac{2t}{p-1}\]
Note that $k, a'$ are both relatively prime to $t/s$. Thus $\nu a + \frac{(p-1)}{t/s}ka' \pmod{p-1}$ ranges over all residues $x \equiv \nu a \pmod {\frac{p-1}{t/s}}$. Pick the $k$ that gives the largest $x = \nu a + \frac{(p-1)}{t/s}ka' \pmod{p-1} $ with $0 <  x < p-1$. We know $x \ge p - 1 - \frac{(p-1)}{t/s}$ (with equality if and only if $\frac{(p-1)}{t/s}$ divides $a$ and hence $\frac{(p-1)}{t}$ divides $a'$). 
\par
However, as $x \le 2t$ by the above, this implies:
\[2t + \frac{(p-1)}{t/s} \ge p-1\]
where equality can only occur if $\frac{(p-1)}{t}$ divides $a'$. If $s = t$ this follows immediately. Otherwise, note that $t$ is at most $\frac{p-1}{3}$ and $\frac{(p-1)}{t/s}$ is at most $\frac{p-1}{2}$. Thus we have the following possibilities:
\begin{enumerate}
\item $s = t$
\item $t  = 2s$, $t = \frac{p-1}{3}$
\item $t  = 2s$, $t = \frac{p-1}{4}$, and $\frac{(p-1)}{t} = 4$ divides $a'$
\item $t = 3s$, $t = \frac{p-1}{3}$, and $\frac{(p-1)}{t} = 3$ divides $a'$
\end{enumerate}
Note that possibilities $3$ and $4$ can't actually happen as the fact that $4 | a'$ contradicts $t = 2s$ and $3|a'$ contradicts $t = 3s$. This same reasoning can be applied to $b$. Now suppose $t < \frac{p-1}{3}$. Then for both $a,b$ we must have case $1$. Thus $t | a$ and $t | b$. Let $d = c't, a = a't, b = b't$. Note that the minimum value of $\frp{\mu a}{p-1}$ is $\frac{\gcd(a,p-1)}{p-1}$ and similarly the minimum of $\frp{\mu b}{p-1}$ is $\frac{\gcd(b,p-1)}{p-1}$. As $\gcd(a, p -1), \gcd(b, p -1) \ge t$ and taking $\mu = \nu$ gives us:
\[\frp{\nu a}{p-1} + \frp{\nu b}{p-1} = \frac{2t}{p-1}\]
We must have:
\[\frp{\nu a}{p-1} = \frp{\nu b}{p-1} = \frac{t}{p-1}\]
and thus $\gcd(a, p -1)=\gcd(b, p -1)=t$. Now note that $\nu$ satisfies: $\nu d \equiv t \pmod{p-1}$ and $\nu a \equiv t \pmod{p-1}$. This implies:
\[\nu(a - d) \equiv 0 \pmod{p-1}\]
which further gives:
\[\nu(a' - c') \equiv 0 \pmod{\frac{p-1}{t}}\]
But as $\nu$ is relatively prime to $\frac{p-1}{t}$ this implies $a' \equiv c' \pmod{\frac{p-1}{t}}$, which implies $a = d$. By the same reasoning $b = d$, which is a contradiction.
\par
Thus we have shown that $\chi^c$ must have order $3$. Let $s_1 = \gcd(t,a)$ and $s_2 = \gcd(t,b)$. As $s_1, s_2$ are either $t$ or $\frac{t}{2}$, $a$ and $b$ must both be multiples of $\frac{p-1}{6}$. However, as $c = \frac{p-1}{3}$ or $\frac{2(p-1)}{3}$ the only way that we can have $a+b=2c$ is if $a$ or $b$ is $\frac{p-1}{2}$, which is a contradiction on $\chi^a, \chi^b$ having order $> 2$.
\par
As we have exhausted all possibilities, 
\[g(\chi^a)g(\chi^b)g(\chi^c)^2\]
does not have argument a root of unity.
\end{proof}

\section{On Fermat Surfaces}

\begin{definition}
Let $F^n_r$ denote the projective variety of dimension $r-1$ in $\P^r$ defined by the polynomial,
\[ x_0^n + \cdots + x_r^n = 0 \] 
We call this the Fermat $n,r$ hypersurface.
\end{definition}

\begin{conjecture} Let $p$ be an odd prime. Let $\zeta_{X_p}$ be the zeta function of the Fermat-4,3 hypersurface over $\F_p$. Then
\[\zeta_{X_p} = \begin{cases} \frac{-1}{(T-1)(p^2T - 1)(pT + 1)^{10}(pT - 1)^{12}} & p \equiv 3 \pmod 4 \\ & \\
\frac{-1}{(T-1)(p^2T - 1)(pT - 1)^8g_p(T)h_p(T)} & p \equiv 1 \pmod 4\end{cases}\]
where
\[g_p(T) = \begin{cases} (pT + 1)^{12} & p \equiv 5 \pmod 8 \\ & \\ (pT - 1)^{12} & p \equiv 1 \pmod 8\end{cases}\]
and 
\[h_p(T) = \left (pT - \frac{s^2}{p}\right)\left(pT - \frac{\bar{s}^2}{p}\right)\]
where $s = a + bi$ is the unique complex number with $a$ an odd positive integer, $b$ an even positive integer, and $|s| = p$.
\end{conjecture}

\begin{proposition}
For Fermat variety $F^n_r$ defined over $\F_q$, the number of possible $\alpha$ is determined by the formula, \[\# A_{n,q} = \sum_{i = 1}^r (-1)^i(d-1)^i,\] where $d = \gcd(n, q-1)$.
\end{proposition}
\begin{proof}
Recall that $ A_{n, p} = \{ (\alpha_0, \dots, \alpha_r): 0 < \alpha_i < 1, \sum d\alpha_i \in \Z, i = 0, \dots, r\}$ in this case. Since $\alpha_i$ have the same denominator, we consider only the numerator of $\alpha_i$, and our problem become counting $x_i$ such that \[x_0 + x_1 + \dots + x_r \in d\Z.\]
Suppose we let $x_1, \dots, x_r$ take arbitrary value in $\{1, \dots, d-1\}$, then the value of $x_0$ is uniquely determined. This gives us $(d-1)^r$ possibilities. But we may be over counting. So apply the inclusion-exclusion formula.  
\end{proof}

\begin{corollary}
The Betti numbers of the Fermat n,r hypersurface are,
\[ 
\dim H^i(F^n_r) = 
\begin{cases}
1  & 0 \le i \le 2(r-1) \text{ is even and } i \neq r - 1 \\
\sum\limits_{j=0}^{r-1} (-1)^j (n-1)^j + 1 & i = r-1 \text{ is even} \\
\sum\limits_{j=0}^{r-1} (-1)^j (n-1)^j & i = r - 1 \text{ is odd}
\end{cases}
\]
\end{corollary}

\begin{corollary}
The Euler Characteristic of the Fermat n,r hypersurface is,
\[ \chi(F^n_r) = r + (-1)^{r-1} \sum\limits_{j=0}^{r-1} (-1)^j (n-1)^j \] 
\end{corollary}

\begin{theorem}
The Fermat hypersurface $F^n_{n-1}$ is never supersingular over $\F_p$ when $p \equiv 1 \mod n$ and $n > 2$.
\end{theorem}

\begin{proof}
The Gaussian sum $g(\chi_\alpha)$ over $\F_p$ is never a root of unity when normalized to the unit circle unless $\alpha = 1, 1/2$ (Chowla). Therefore, consider $\alpha = (1/n, \cdots, 1/n)$ which satisfied the conditions to be in $A_{n,p}$ since $r + 1 = n$. Therefore, 
\[ (-1)^r B(\alpha) j(\alpha) = (-1)^r B(\alpha) g(\chi_{1/n})^{n} \]
which is a root of $\zeta_X$ 
cannot be a root of unity when normalized to the unit circle because $(-1)^r B(\alpha)$ is a root of unity but $g(\chi_{1/n})^{n}$ is not since $g(\chi_{1/n})$ is not either by Chowla because $n > 2$. Therefore, $\zeta_X$ contains a root which is not of the form $\omega q^{\frac{i}{2}}$ where $\omega$ is a root of unity so $X$ is not supersingular. 
\end{proof}


\begin{theorem}
Let $n \ge 4$ be an integer and let $p \equiv 1 \pmod{n}$ be a prime number. Then the zeta function for the Fermat curve (with points counted via the "stack quotient") given by the zero set of:
\[w^n + x^n + y^n + z^n = 0\]
is not supersingular
\end{theorem}
\begin{proof}
By Theorem \ref{thm:decomp_zeta_function}, we just need to show that 
\[\prod_{i=0}^3 g(\chi_{\alpha_i})\]
has argument not equal to a root of unity. For $n = 4$ we take $\alpha_i = \frac{1}{4}$ for all $i$. By Theorem \ref{thm:chowla_root_unity} this is does not have argument equal to a root of unity. For $n = 6$ we take $\alpha_0 = \frac{1}{2}$ and $\alpha_i = \frac{1}{6}$ for $i \neq 0$. Again, by Theorem \ref{thm:chowla_root_unity} this is does not have argument equal to a root of unity. For all other $n \ge 4$ we take $\alpha_0 = \frac{n - 3}{n}$ and $\alpha_i = \frac{1}{n}$ for $i \neq 0$. By Theorem \ref{thm:evans_product_root} this does not have argument equal to a root of unity. 
\end{proof}

\section{On Non-Supersingularity using Factorization of Gauss sum}
In this section, let $X$ be a variety defined by,
 \[ a_0 x_0^{n_0} + \dots + a_r x_r^{n_r} = 0 \]
  over $\F_p$, where $p$ is a prime not dividing $m =  \lcm(n_0, \dots, n_r)$. Furthermore, let $f = \text{ord}_m(p)$.
\begin{proposition}
If $p \equiv 1 \mod m$ for $m \ge 4$ and $r \ge 3$ then $F^m_r$ is not supersingular.
\end{proposition}
\begin{proof}
Notice that in this case $f = 1$, and $q = p$.  If $F^m_r$ were supersingular then,  by Theorem \ref{thm:gauss_factor_SS}, for each choice of $\mu \in (\Z / m \Z)^\times$ and character powers $e_0, \cdots e_r$ that,
\[\sum_{i = 0}^{r} s \left(\frac{(q-1)\mu r_i}{m} \right) = \frac{r+1}{2} (p-1) f \] 
Consider the case $\mu = 1$ and choose a set of characters such that \[e_0 + \cdots + e_r = m \left \lfloor  \frac{r}{2} \right \rfloor\]
This is always possible with $0 < e_i < m$ since $r+1 \le m \left \lfloor  \frac{r}{2} \right \rfloor < mr$. In this case, since $f = 1$ and $\mu = 1$,
\[ \sum_{i = 0}^{r} s \left(\frac{(q-1)\mu r_i}{m} \right) = (p-1)\sum_{i = 0}^{r} \left\{ \frac{e_i}{m} \right\} = (p - 1) \sum_{i = 0}^r \frac{e_i}{m} = (p - 1) \left \lfloor  \frac{r}{2} \right \rfloor < (p-1) \frac{r+1}{2} \]
Therefore, by Theorem \ref{thm:gauss_sum_is_root_of_unity_ideal_factorization_counting_condition}, $F^m_r$ cannot be supersingular. 
\end{proof}

\begin{proposition}
Let $p$ be a prime, and $f > 2$, let $n = \frac{p^f - 1}{p-1}$. Then $F^n_3$ is not supersingular over $\F_p$.
\end{proposition}
\begin{proof}
Let $\mu = 1$, and $\overline r = (1,1,1,m-3)$. We know that $s(\frac{(q-1) \mu r}{m}) = p - 1$ when $r = 1$ using the fraction part formula for $s$ because all the terms are less than $1$.

Now consider \[s\left(\frac{(m-3)(q-1)}{m} \right) = (p-1) \sum_{i = 1}^{f-1} \left\{\frac{(m-3)p^i}{m}\right\}\]
If $i < f - 1$, then $3p^i < m$, so 
$$\left\{\frac{(m-3)p^i}{m}\right\} = 1 - \frac{3p^i}{m}$$. If $i = f - 1$, then use the relation 
\[p^{f - 1} = m - (1 + p + \dots + p^{f - 2}),\] 
so \[ \left\{\frac{(m-3)(m - (1 + p + \dots + p^{f - 2}))}{m} \right\} = \frac{3(1 + p + \dots + p^{f - 2})}{m}\].
As a result, 
$s \left(\frac{(q-1)(m-3)}{m}\right) = (p-1)(f-1)$. 
And \[\sum_{i = 0}^{r} s \left(\frac{(q-1) r_i}{n} \right) = (f+2)(p-1) < 2f(p-1)\] if $f > 2$. 
Therefore, $F^n_3$ cannot be supersingular if $f > 2$. 
\end{proof}
\begin{proposition}
When $f$ is even, and $n = \frac{p^f - 1}{p^2 - 1}$, then $F^n_3$ is not supersingular.
\end{proposition}
\begin{proof}
Let $\mu = 1$, $\overline{r} = (1,1,1, n-3)$, and write $m = 1 + p^2 + p^4 + \dots + p^{f - 2}$. Notice that $p^{f-1} = pm - (p + p^3 + \dots + p^{f-3})$. When $r = 1$, 
\begin{align*}
s(\frac{(q-1)}{m}) &= (p-1) \sum_{i = 1}^{f-1} \{\frac{p^i}{m}\} \\ &= (p-1)(\sum_{i = 0}^{f-2}(\frac{p^i}{m}) + \{\frac{pm - (p + p^3 + \dots + p^{f-3})}{m}\}) \\ &= (p-1)(1 + \frac{1 + p^2 + \dots + p^{f-2}}{m}) \\ &= 2(p-1).
\end{align*}

When $r = m - 3$, we have 
\begin{align*}
s(\frac{(q-1)(m-3)}{m}) &= (p-1) \sum_{i = 1}^{f-1} \{\frac{p^i(m-3)}{m}\} \\ &= (p-1)(\sum_{i = 0}^{f-2}(1 - \frac{3p^i}{m}) + \{\frac{(m-3)(pm - (p + p^3 + \dots + p^{f-3}))}{m}\}) \\ &= (p-1)(f-1 + \sum_{i = 0}^{f-2}(- \frac{3p^i}{m}) + \frac{3(p + p^3 + \dots + p^{f-3})}{m}) \\ &= (p-1)(f - 1 - \frac{3m}{m}) \\ &= (p-1)(f-4).
\end{align*}

In total we still have \[\sum_{i = 0}^{r} s(\frac{(q-1) r_i}{n}) = (f+2)(p-1) < 2f(p-1).\]
\end{proof}
\begin{proposition}
When $n = p+a$ for $1 < a < p$, and $\ord_n(p) = 2$, the Fermat variety $X_n$ is not supersingular.
\end{proposition}
\begin{proof}
Still consider $\mu = 1$, $\overline{r} = (1,1,1,n-3)$. We have $\{1/n\} + \{p/n\} = (1+p)/n < 1$ for $r = 1$.
And since $\ord_n(p) = 2$, $n$ does not divides $p-1$ but $n$ divides $p^2 - 1$, so $n|(p+1)$. Then $\{(n-3)/n\} + \{(n-3)p/n\}$ is an integer. Thus it has to be $1$. This tell us that the sum of the $s$ functions is less than $4(p-1)$. Therefore, $X_n$ is not supersingular.
\end{proof}
\begin{conjecture}
For $p$ a prime, and $f > 2$, let $n = \Phi_f(p) = \frac{p^f - 1}{k(p)}$, then $\ord_n(p) = f$, and the Fermat surface $F^n_3$ is not supersingular.
\end{conjecture}

\begin{lemma} \label{variety_supersingular_E_constant}
Let $X$ be a variety defined by the zero set of the equation:
\[a_0 x_0^{n_0} + a_1 x_1^{n_1} + a_2 x_2^{n_2} + a_3 x_3^{n_3} = 0\]
over $\finfield{p^k}$ with $a_i \in \Z, n_i \in \Z_{\ge 1}$. Let $m = \lcm(n_0, n_1, n_2, n_3)$ and let $w_i = \frac{m}{n_i}$ for $i = 0,1,2,3$. Then $X$ is supersingular if and only if for all $\mu \in (\Z / m \Z)^\times$ and $e_0, e_1, e_2, e_3 \in \Z$ with $m | e_0 + e_1 + e_2 + e_3$, $w_i | e_i$, $0 < e_i < m$ we have:
\begin{equation*}
\sum_{i = 0}^{f-1} \left( \left\{\frac{\mu e_0p^i}{m}\right\} + \left\{\frac{\mu e_1p^i}{m}\right\} \right) 
= \sum_{i = 0}^{f-1} \left( \left\{\frac{-\mu e_2p^i}{m}\right\} + \left\{\frac{-\mu e_3p^i}{m}\right\} \right)
\end{equation*}
\end{lemma}

\begin{proof}
By Theorem \ref{thm:base_field_shifting}, we only need to prove that it is supersingular over $\F_q$ for some power $q = p^f$. Suppose $r$ is the smallest positive integer such that $p^r \equiv -1 \pmod{m}$. We'll take $f = 2r$, so that $f$ is the minimal integer for which $m | p^f - 1$. \par
Let $\chi$ be a character of order $m$. Now, by Corollary \ref{cor:supersingular_iff_gauss_sum_products_roots_of_unity}, $X$ is supersingular if the product of Gaussian sums for each $\alpha$ has argument root of unity. That is,
\[ \prod_{i = 0}^{3}g(\chi^{e_i}) \]
must always have argument a root of unity where $m | e_0 + e_1 + e_2 + e_3$, $0 < e_i < m$, and $w_i | e_i$ for each $i$.
\par
Consider the ideal generated by,
\[q^{-2}\prod_{i = 0}^{3}g(\chi^{e_i}) = \frac{g(\chi^{e_0})g(\chi^{e_1})\chi^{e_2 + e_3}(-1)}{g(\chi^{-e_2})g(\chi^{-e_3})}\]
By Corollary \ref{cor:gauss_sum_prod_principal_ideal}, this is a root of unity if and only if the ideal generated by it is $\iO$, which will occur if and only if the valuation of each prime ideal in $\Q(\zeta_m, \zeta_p)$ is 0. By Theorem \ref{thm:gauss_fact_lang}, this will occur if and only if:
\[s\left(\frac{(q-1)\mu e_0}{m}\right) + s\left(\frac{(q-1)\mu e_1}{m}\right) = s\left(\frac{-(q-1)\mu e_2}{m}\right) + s\left(\frac{-(q-1)\mu e_3}{m}\right)\] 
for all $\mu$ relatively prime to $m$ where $s(n)$ is the sum of the digits of $n \pmod{q-1}$ in base $p$. Even Further, by [Lang's Algebraic Number Theory Page 96], this is equivalent to:
\begin{equation*}
\sum_{i = 0}^{f-1} \left( \left\{\frac{\mu e_0p^i}{m}\right\} + \left\{\frac{\mu e_1p^i}{m}\right\} \right) 
= \sum_{i = 0}^{f-1} \left( \left\{\frac{-\mu e_2p^i}{m}\right\} + \left\{\frac{-\mu e_3p^i}{m}\right\} \right)
\end{equation*}
as desired.
\end{proof}

\begin{definition}
Define the sum,
\[ S_\mu(e_0, \dots, e_t) = s\left( \frac{(q-1) \mu e_0}{m} \right) + \cdots + s\left( \frac{(q-1) \mu e_t}{m} \right)  = \sum_{i = 0}^{f-1} \left( \left\{\frac{\mu e_0p^i}{m}\right\} + \cdots + \left\{\frac{\mu e_t p^i}{m}\right\} \right)  \]
\end{definition}

\begin{corollary} \label{cor:supersingular_iff_constant_sum}
$X$ is supersingular if and only if the value of the sum,
\[ S_\mu(e_0, e_1) = \sum_{i = 0}^{f-1} \left( \left\{\frac{\mu e_0p^i}{m}\right\} + \left\{\frac{\mu e_1p^i}{m}\right\} \right)  \]
for each fixed value of $\mu \in (\Z / m \Z)^\times$ depends only on $E \equiv e_0 + e_1 \mod{m}$. 
\end{corollary}

\begin{proof}
We know that $X$ is supersingular if and only if,
\begin{equation*}
\sum_{i = 0}^{f-1} \left( \left\{\frac{\mu e_0p^i}{m}\right\} + \left\{\frac{\mu e_1p^i}{m}\right\} \right) 
= \sum_{i = 0}^{f-1} \left( \left\{\frac{-\mu e_2p^i}{m}\right\} + \left\{\frac{-\mu e_3p^i}{m}\right\} \right)
\end{equation*}
for each $\mu \in (Z / m \Z)^\times$ and $e_0, e_1, e_2, e_3$ such that $m \divides e_0 + e_1 + e_2 + e_3$ and $w_i \divides e_i$. Therefore, whenever,
\[ E \equiv e_0 + e_1 \equiv -e_2 -e_3 \mod{m} \]
we must have that $S_\mu(e_0, e_1) = S_\mu(-e_2, -e_3)$. This is equivalent to $S_\mu$ depending on $E$ alone. 
\end{proof}


\begin{lemma}
\label{lem:sum_frac_part_N(e_0,e_1)}
Let $p$ be a prime number, $f$ be a positive integer, $m$ be an integer not divisible by $p$, and $\mu \in (\Z / m \Z)^\times$. For integers $m \nmid e_0, e_1$ define:
\[N_\mu(e_0, e_1) = \#\left\{i \ : \ \left\{\frac{\mu (e_0 + e_1)p^i}{m}\right\} < \left\{\frac{\mu e_0 p^i}{m}\right\} \right\},\] 
where $i = 0, \dots, f-1$, then
\[ S_\mu(e_0, e_1) = \sum_{i = 0}^{f-1} \left( \left\{\frac{\mu e_0p^i}{m}\right\} + \left\{\frac{\mu e_1p^i}{m}\right\} \right) = N_\mu(e_0, e_1) + \sum_{i = 0}^{f-1}  \left\{\frac{\mu (e_0 + e_1)p^i}{m}\right\} = N_\mu(e_0, e_1) + S_\mu(e_0 + e_1).
\]
\end{lemma}
\begin{proof}
Note that 
\[\left\{\frac{\mu e_0p^i}{m}\right\} + \left\{\frac{\mu e_1p^i}{m}\right\}\]
is either equal to $\left\{\frac{\mu (e_0 + e_1)p^i}{m}\right\}$ or $\left\{\frac{\mu (e_0 + e_1)p^i}{m}\right\} + 1$. If it is equal to the former, then
\[\left\{\frac{\mu (e_0 + e_1)p^i}{m}\right\} \ge \left\{\frac{\mu e_0p^i}{m}\right\} \]
If it is equal to the latter, then
\[\left\{\frac{\mu e_0p^i}{m}\right\} = \left\{\frac{\mu (e_0 + e_1)p^i}{m}\right\} - \left\{\frac{\mu e_1p^i}{m}\right\} + 1 > \left\{\frac{\mu (e_0 + e_1)p^i}{m}\right\}\]
Thus we have:
\[\left\{\frac{\mu e_0p^i}{m}\right\} + \left\{\frac{\mu e_1p^i}{m}\right\} = \begin{cases} \left\{\frac{\mu (e_0 + e_1)p^i}{m}\right\} & \left\{\frac{\mu (e_0 + e_1)p^i}{m}\right\} \ge \left\{\frac{\mu e_0 p^i}{m}\right\} \\ 
& \\ 
\left\{\frac{\mu (e_0 + e_1)p^i}{m}\right\} + 1 & \left\{\frac{\mu (e_0 + e_1)p^i}{m}\right\} < \left\{\frac{\mu e_0 p^i}{m}\right\} \end{cases} \]
\end{proof}

\begin{corollary} \label{cor:if_divides_then_equal_f}
If $e_0 + e_1 \equiv 0 \mod{m}$ then $S_\mu(e_0, e_1) = N_\mu(e_0, e_1) = f$.
\end{corollary}

\begin{proof}
\[ S_\mu(e_0, e_1) = \sum_{i = 0}^{f-1} \left( \left\{\frac{\mu e_0p^i}{m}\right\} + \left\{\frac{\mu e_1p^i}{m}\right\} \right) = N_\mu(e_0, e_1) + \sum_{i = 0}^{f-1}  \left\{\frac{\mu (e_0 + e_1)p^i}{m}\right\} 
\]
However, $m \divides e_0 + e_1$ so the fractional part of all multiplies of their quotient is zero. Thus,
\[ \left\{\frac{\mu (e_0 + e_1)p^i}{m}\right\} = 0 \]
Therefore, the second sum is zero. Furthermore, since $m \nmid e_0$ and $(m, p) = (m, \mu) = 1$ we have that,
\[ 0 \le \left\{\frac{\mu e_0 p^i}{m}\right\} \]
for each $i$. Therefore, $N(e_0, e_1) = f$. 
\end{proof}

\begin{lemma}[Ming]\label{lem:Ming}
The product $q^{-2} g(\chi^{e_0}) g(\chi^{e_1}) g(\chi^{e_2}) g(\chi^{e_3})$ is a root of unity if and only if $N_\mu(e_0, e_1) + N_\mu(e_2, e_3) = f$ for each $\mu \in (\Z / m \Z)^\times$
\end{lemma}

\begin{proof}
By Theorem \ref{thm:gauss_sum_is_root_of_unity_ideal_factorization_counting_condition} we need only check if,
\[ \sum_{i = 0}^3 s\left(\frac{(q-1) \mu e_i}{m}\right) = 2 (p - 1) f \]
for each $\mu \in (\Z / m \Z)^\times$. However, because $m \divides e_0 + e_1 + e_3 + e_4$ by Corollary \ref{cor:if_divides_then_equal_f},
\[ S_\mu(e_0 + e_1) + S_\mu(e_2 + e_3) = S_\mu(e_0 + e_1, e_2 + e_3) = f \]
Furthermore, by Lemma, \ref{lem:sum_frac_part_N(e_0,e_1)},
\[ S_\mu(e_0, e_1) + S_\mu(e_2, e_3) = N_\mu(e_0, e_1) + N_\mu(e_2, e_3) + S_\mu(e_0 + e_1) + S_\mu(e_2 + e_3) = N_\mu(e_0, e_1) + N_\mu(e_2, e_3) + f \]
Thus,
\[ S_\mu(e_0, e_1) + S_\mu(e_2, e_3) = \frac{1}{p-1} \sum_{i = 0}^3 s\left(\frac{(q-1) \mu e_i}{m}\right) = 2 f \iff N_\mu(e_0, e_1) + N_\mu(e_2, e_3) = f \]
\end{proof}

\begin{theorem}\label{thm:shioda}
Let $X$ be a variety defined by the zero set of the equation:
\[ a_0 x_0^{n_0} + a_1 x_1^{n_1} + a_2 x_2^{n_2} + a_3 x_3^{n_3} = 0 \]
over $\finfield{p^k}$ with $a_i \in \Z, n_i \in \Z_{\ge 1}$. Let $m = \lcm(n_0, n_1, n_2, n_3)$. If $a_i \neq 0$ in $\F_p$ for all $i$ and there exists $r$ such that $p^r \equiv -1 \pmod{m}$, then $X$ is supersingular.
\end{theorem}

\begin{proof} 
By Corollary \ref{cor:supersingular_iff_constant_sum}, if we can show that for all $\mu \in (\Z / m \Z)^\times$ and $e_0, e_1$ with $0 < e_0, e_1 < m$ the sum $S_\mu(e_0, e_1)$
is only a function of $E = e_0 + e_1$, then $X$ is supersingular. Let $N(e_0, e_1)$ be as defined in lemma \ref{lem:sum_frac_part_N(e_0,e_1)}.
If $m | E$, then we will always have: 
\begin{equation*} \label{eq:e_0e_1_inequality} 
\left\{\frac{\mu (e_0 + e_1)p^i}{m}\right\} < \left\{\frac{\mu e_0 p^i}{m}\right\} 
\end{equation*}
and thus $N(e_0, e_1) = f$. If $m \nmid E$, then note that as $p^{r} \equiv -1 \pmod{m}$, we have:
\[\left\{\frac{\mu Ep^{i+r}}{m}\right\} = \left\{\frac{-\mu Ep^{i}}{m}\right\} = 1 - \left\{\frac{\mu Ep^{i}}{m}\right\}\]
Therefore, applying this procedure to the above inequality,
\[ \left\{\frac{\mu (e_0 + e_1)p^{i+r}}{m}\right\} < \left\{\frac{\mu e_0 p^{i+r}}{m}\right\}  \iff 1 - \left\{\frac{\mu (e_0 + e_1)p^i}{m}\right\} < 1 - \left\{\frac{\mu e_0 p^i}{m}\right\} \iff  \left\{\frac{\mu e_0 p^i}{m}\right\} < \left\{\frac{\mu (e_0 + e_1)p^i}{m}\right\}   \]
Furthermore, since $m \nmid e_0, e_1$ the inequality must always be strict. Since $f = 2r$, this symmetry implies that $N(e_0, e_1) = \frac{f}{2}$. Note that $N(e_0, e_1)$ is constant. Thus by Lemma \ref{lem:sum_frac_part_N(e_0,e_1)}, 
\[ S_\mu(e_0, e_1) = \sum_{i = 0}^{f-1} \left( \left\{\frac{\mu e_0p^i}{m}\right\} + \left\{\frac{\mu e_1p^i}{m}\right\} \right) \]
is a function of $E$ alone and thus $X$ is supersingular.
\end{proof}



\begin{theorem}
If there exists $v \in \Z$ such that $p^v \equiv -1 \mod{m}$ then $F^m_r$ is supersingular for any $r$.
\end{theorem}

\begin{proof}
Consider the sum,
\[ S_\mu(e_1, \dots, e_r) = \frac{1}{p-1} \sum_{i = 0}^r s \left( \frac{\mu (q-1) e_i }{m} \right) = \sum_{i = 0}^r \sum_{j = 0}^{f-1} \left\{ \frac{\mu e_i p^j}{m} \right\} \]
which we can rearrange as,
\[ S_\mu(e_1, \dots, e_r) = \sum_{i = 0}^r \left( \sum_{j = 0}^{\frac{f}{2}-1} \left\{ \frac{\mu e_i p^j}{m} \right\} + \sum_{j = 0}^{\frac{f}{2}-1} \left\{ \frac{\mu e_i p^{j + \frac{f}{2}}}{m} \right\} \right) \]
However, since $f = \ord_m{p}$ and the hypothesis, we know that $p^{\frac{f}{2}} \equiv -1 \mod{m}$. Thus,
\[ \left\{ \frac{\mu e_i p^{j + \frac{f}{2}}}{m} \right\} = \left\{ \frac{-\mu e_i p^{j}}{m} \right\}  = 1 - \left\{ \frac{\mu e_i p^{j}}{m} \right\} \]
Therefore, plugging in, 
\begin{align*} 
S_\mu(e_1, \dots, e_r) &= \sum_{i = 0}^r \left( \sum_{j = 0}^{\frac{f}{2}-1} \left\{ \frac{\mu e_i p^j}{m} \right\} + \sum_{j = 0}^{\frac{f}{2}-1} \left[ 1 - \left\{ \frac{\mu e_i p^{j}}{m} \right\} \right] \right) = \sum_{i = 0}^r \sum_{j = 0}^{\frac{f}{2} - 1} 1 = (r + 1) \frac{f}{2} 
\end{align*}
Thus, by Theorem \ref{thm:gauss_factor_SS}, $F^m_r$ is supersingular.
\end{proof}


\begin{lemma}
\label{lem:symmetric_group_order}
Let $\sigma \in S_n$ be a permutation and $C \in S_n$ be the standard $n$-cycle,
\[ C = (1 \: 2 \: 3 \: \cdots \: n) \]
Define the function,
\[ g(\sigma, k) = \# \{ i \in [n] \mid \sigma(i) < \sigma C^k (i) \} \]
Then $g(\sigma, k) + g(\sigma, n-k) = n$ for all $0 < k < n$.
\end{lemma}

\begin{proof}
Since $\sigma$ is a permutation, we can reindex the set in the definition of $g$ by $j = \sigma(i)$ such that,
\[ g(\sigma, k) = \# \{ j \in [n] \mid j < \sigma C^k \sigma^{-1}(j) \} \]
However, conjugation is an automorphism so,
\[ \sigma C^k \sigma^{-1} = (\sigma C \sigma^{-1})^k = C_{\sigma}^k \]
where $C_\sigma = \sigma C \sigma^{-1}$ is also an $n$ cycle (with order $n$) since conjugation preserves cycle type. Thus,
\[ g(\sigma, k) = \# \{ j \in [n] \mid j < C_\sigma^k (j) \} \]
However, if $j < C_\sigma^k (j)$ then define $\tilde{j} = C_\sigma^k (j)$ or equivalently $C_\sigma^{n - k}(\tilde{j}) = j$ such that,
\[ C_\sigma^{n - k}(\tilde{j}) < \tilde{j} \]
However, $n$ cycles act freely on $[n]$ so there are no fixed points of $C_\sigma^k$ for any $0 < k < n$. Thus, the set of $\tilde{j}$ such that $C_{\sigma}^{n-k}(\tilde{j}) < \tilde{j}$ is exactly the compliment of the set such that $\tilde{j} < C_{\sigma}^{n-k}(\tilde{j})$. Therefore, $j \in g(\sigma, k) \iff \tilde{j} \notin g(\sigma, n-k)$ so,
\[ g(\sigma, k) = \{ \tilde{j} \in [n] \mid C_\sigma^{n - k}(\tilde{j}) < \tilde{j} \} = n - g(\sigma, n - k) \]
\end{proof}

\begin{corollary}
If there exists $\sigma \in S_n$ such that $g(\sigma, k) = g(\sigma, n-k)$ then $g(\sigma, k) = \frac{n}{2}$. In particular, this is true if $g(\sigma, k)$ is constant for $0 < k < n$.
\end{corollary}

\begin{corollary}
If $n$ is odd then $g(\sigma, k) \neq g(\sigma, n - k)$ for all $0 < k < n$. In particular, this means that if $n$ is odd, then there cannot exits $\sigma \in S_n$ such that $g(\sigma, k)$ is constant for all $0 < k < n$. 
\end{corollary}


\begin{lemma} \label{lem:N(e_0,e_1)_constant} Let $m,p,e_0,e_1,f,N(e_0,e_1)$ be as in lemma \ref{lem:sum_frac_part_N(e_0,e_1)}. If $f > 1$, $m | p^{f} - 1$ and $E$ is such that $m \nmid E(p-1)$ and there exists a $K$ such that for all $e_1 + e_2 \equiv E \pmod{M}$ with $m \nmid e_1, e_2$, we have
\[N_\mu(e_0, e_1) = K\]
then $K = \frac{f}{2}$ where $\mu \in (\Z / m \Z)^\times$ is fixed.
\end{lemma}

\begin{proof}
Suppose that such an $E$ exists. Let
\[a_i = \left\{ \frac{\mu Ep^i}{m}\right\}\]
Note as $m | p^f - 1$, we have $a_{i + f} = a_i$. Suppose $a_i = a_j$ for some integers $i,j$. Then we have:
\[Ep^{i} \equiv Ep^{j} \pmod{m}\]
which is true if and only if
\[E(p^{i-j} - 1) \equiv 0 \pmod{m}\]
This we hold only when $i - j$ is multiple of some integer $t$. As a result $a_{i+t} = a_i$ but $a_0, a_1, \ldots, a_{t-1}$ are distinct. Furthermore, since $m \nmid E(p-1)$ we have $t > 1$. For notation purposes. We now let permutations $\pi \in S_{t}$ act on the sequence $a_i$. As $a_0, a_1, \ldots, a_{t-1}$ are distinct, there exists a permutation $\sigma \in S_t$ such that 
for $i = 0, \ldots, t - 1$. $a_\sigma(i) < a_\sigma(j)$ if and only if $i < j$ for $0 \le i,j \le t-1$. Since the condition $N_\mu(e_0, e_1) = K$ must hold for all $e_0 + e_1 \equiv E \mod{m}$ we may pick a particualr value of,
\[ e_0 |_j = E p^j \text{ and } e_1 |_j = E - e_0|_j \] 
for any $1 \le j \le t - 1$. In this case,
\[\left\{ \frac{\mu e_0|_j p^i}{m}\right\} = a_{i + j}\]
Thus if we let $C = (1 \; 2 \; \cdots \; t) \in S_t$, then this can be rewritten as:
\[\left\{ \frac{\mu e_0|_j p^i}{m}\right\} = a_{C^j(i)}\]
By definition,
\begin{equation*}
K = N_\mu( e_0 |_j ,  e_1 |_j ) = \# \{ 0 \le i < t \ : \ a_i < a_{i + j} \}
\end{equation*}
As $a_i$ is periodic, this is implies
\begin{align*}
K &= \frac{f}{t}\#\{i \ : \ a_i < a_{C^j(i)}\} \\
& = \frac{f}{t}\#\{ i \ : \ \sigma^{-1}(i) < \sigma^{-1}(C^j(i)) \} = \frac{f}{t} g(\sigma^{-1}, j)
\end{align*}
However, by lemma \ref{lem:symmetric_group_order},
\[ g(\sigma^{-1}, j) = g(k) = t - g(t - k) \]
As $t > 1$, taking $k = 1$ implies $g(\sigma^{-1}, k) = \frac{t}{2}$. Thus:
\[K = \left(\frac{f}{t}\right)\left(\frac{t}{2}\right) = \frac{f}{2}\]
\end{proof}

\begin{theorem} \label{thm:shioda_neg1}
If $f$ is odd and $f > 1$, then $F^m_3$ is not supersingular
\end{theorem}
\begin{proof}
By Corollary \ref{cor:supersingular_iff_constant_sum}, $F^m_3$ is supersingular only if for all $e_0, e_1$ with $0 < e_0, e_1 < m$ we have that 
\[ S_\mu(e_0, e_1) = \sum_{i = 0}^{f-1} \left( \left\{\frac{\mu e_0p^i}{m}\right\} + \left\{\frac{\mu e_1p^i}{m}\right\} \right) \]
is only a function of $E = e_0 + e_1$. Consider the case $E = 1$. Let $N(e_0, e_1)$ be defined as in lemma \ref{lem:sum_frac_part_N(e_0,e_1)}. By the same lemma, the above being a function of $E$ is equivalent to $N(e_0, e_1)$ being constant across $e_0 + e_1$. By lemma \ref{lem:N(e_0,e_1)_constant}, if it is constant for fixed $E$, then it must always be $\frac{f}{2}$. However, as $N(e_0, e_1)$ is integer-valued this is impossible. Thus we have a contradiction, so $F^m_3$ is not supersingular.
\end{proof}


\begin{theorem}
Let $f = \ord_n(p)$. If $f$ is odd and $f > 1$, then $F_2^n$ is not supersingular
\end{theorem}

\begin{proof}
By Theorem \ref{thm:base_field_shifting}, we only need to prove that it is supersingular over $\F_q$ for some power $q = p^f$.  Let $\chi$ be a character of order $n$. By Theorem \ref{thm:decomp_zeta_function}, we have that 
\[\zeta(T) = \frac{p(T)}{q(T)}\]
where $p(T) = -1$ and the roots of $q(T)$ are of the form:
\[\prod_{i = 0}^{2}\chi^{e_i}(a_i^{-1})\prod_{i = 0}^{2}g(\chi^{e_i})\]
where $m | e_0 + e_1 + e_2$ and $0 < e_i < n$, and $w_i | e_i$ for each $i$. The product $\prod_{i = 0}^{2}\chi^{e_i}(a_i^{-1})$ will always be a root of unity. Thus to show $\zeta(T)$ is supersingular, we just need to show that $\prod_{i = 0}^{2}g(\chi^{e_i})$ always has argument a root of unity. We will now do so.
\par
Consider the ideal generated by,
\[q^{-3/2}\prod_{i = 0}^{2}g(\chi^{e_i}) = \frac{g(\chi^{e_0})g(\chi^{e_1})\chi^{e_2}(-1)}{q^{-1/2}g(\chi^{-e_2})}\]
By Corollary \ref{cor:gauss_sum_prod_principal_ideal}, this is a root of unity if and only if the ideal generated by it is $R$, which will occur if and only if the valuation of each prime ideal in $\Q(\zeta_n, \zeta_p)$ is 0. By Theorem \ref{thm:gauss_fact_lang}, this will occur if and only if:
\[s\left(\frac{(q-1)\mu e_0}{n}\right) + s\left(\frac{(q-1)\mu e_1}{n}\right) = s\left(\frac{-(q-1)\mu e_2}{n}\right) + \frac{f}{2}\]
By [Lang Algebra Page 96] this is equal to,
\[\sum_{i = 0}^f\ \left(\frp{\mu e_0 p^i}{n} + \frp{\mu e_1 p^i}{n} - \frp{\mu -e_2 p^i}{n}\right) = \frac{f}{2}\]
However, as $e_0 + e_1 \equiv -e_2 \pmod{n}$, each term in the above summation must be either 1 or 0. Thus the left hand side is an integer. However, if $f$ is odd, the right hand side is not. Thus this equality cannot possibly happen.
\end{proof}





\begin{theorem}
Let $f$ be odd and $m$ be even, then the Fermat variety $F^m_3$ is not supersingular.
\end{theorem}
\begin{proof}
We know that $X$ is supersingular if and only if $q^{-2}\prod_{i = 0}^3 g(\chi^{e_i})$ is a root of unity, where $m| e_0 + e_1 + e_2 + e_3$ and $0 < e_i < m$ for each $i$.\\
Let $e_0 + e_1 = E_0$, and $e_2 + e_3 = E_2$. By lemma \ref{lem:Ming}, we know that $V_m$ is supersingular if and only if $N(e_0, e_1) + N(e_2, e_3) = f$.
Now let $E_0 + E_2 = 3m$, and $e_0 = e_2$, $e_1 = e_3$. Then $E_0 = 3/2 m$ is an integer because $m$ is even. But $N_0 \ne f/2$ because $N_0$ is an integer but $f$ is odd, so $f/2$ is not an integer. We also know that $N_0 = N_2$, since $e_0 = e_2$, $e_1 = e_3$. Thus it is impossible that $N_0 + N_2 = f$. Therefore, $F^m_3$ is not supersingular.
\end{proof}
\begin{theorem}\label{thm:converse_shioda_r_odd_f_odd}
Let $f$ be odd, the Fermat variety $F^m_{r}$ is not supersingular if $r$ is odd.
\end{theorem}
\begin{proof}
We prove this using Theorem \ref{thm:converse_shioda_r=3_f_odd} and Lemma \ref{lem:sum_frac_part_N(e_0,e_1)}. 

We know that $F^m_r$ is supersingular if and only if \[\sum_{i = 0}^r S_\mu (e_i) = (p-1)(r+1)f/2\]
for all $\mu \in (\Z/m\Z)^\times$, and $m| e_0 + e_1 + \dots + e_r$ and $0 < e_i < m$ for each $i$. 
Thus, we can choose $e_i$ for $i > 3$ such that $m| e_i + e_{i+1}$. Then for any given $\mu$, $S_\mu(e_i, e_{i+1}) = f$ by Lemma \ref{lem:sum_frac_part_N(e_0,e_1)}. 

On the other hand, choose $e_0, \dots, e_3$ as in Theorem  \ref{thm:converse_shioda_r=3_f_odd}, then $S_\mu (e_0, e_1, e_2, e_3) \ne 2f$. 

Therefore, we have \[\sum_{i = 0}^r S_\mu (e_i) \ne (p-1)(r+1)f/2\] for this chosen set of $e_i$, so $F^m_{r}$ is not supersingular.

\end{proof}

\begin{conjecture} Let $q = p^n$, $p$ a prime and $n \in \mathbb{Z}^+$, be the order of our finite field $\F_q$ and let $N_\mu$ be the number of solutions $(e_0, e_1, e_2, e_3)$ with $0 < e_i < q-1$ all distinct and $\mu \in \mathbb{Z}^+$ with $(\mu, q-1) = 1$ satisfying $s(\mu e_0) + s(\mu e_1) = s(\mu e_2) + s(\mu e_3)$. We conjecture that $N_1 = N_p$, and for $\mu_j, \mu_k > p$, $N_{\mu_{j}} = N_{\mu_{k}}$ if $\mu_j$ and $\mu_k$ share the same largest factor.
\end{conjecture}

\section{On Sum-Product Varieties}

\subsection{Introduction}
 
In this section we concern ourselves with the family of varieties,
\[ x_1 + \cdots + x_d = \lambda x_1 \cdots x_d \]
over the finite field $\finfield{q}$. In the process, we will study the $m$-values which are solutions to the set of simultaneous equations,
\[ x_1 + \cdots + x_d = z \quad \text{and} \quad x_1 \cdots x_d = y \]
over $\finfield{q}$. (Motivation?)

\begin{definition}
The integer, $m^{d,q}_{y,z}$ is the number of solutions to the set simultaneous of equations,
\begin{align*}
x_1 + & \cdots + x_d = z \\
x_1  & \cdots x_d = y
\end{align*}
over $\finfield{q}$. 
\end{definition}

\begin{definition}
The diagonal hyper-plane number is the number of solutions, 
\[H^d_z(S) = \# \left\{ x_1 + \cdots + x_d = z \mid x_i \in S \right\} \]
where $S \subset K$ and $z \in K$ for some field $K$.
\end{definition}

\begin{proposition}
For any $z \in \finfield{q}$ we have $H^d_z(\finfield{q}) = q^{d-1}$ and for $z \in \finfield{q}$ we have, 
\[ H^d_z(\finunits{q}) = \frac{1}{q} \left[ (q-1)^d + (q \delta_z - 1)(-1)^d \right] \]
\end{proposition}

\begin{proof}
For any choice of $x_1, \cdots, x_{d-1} \in \finfield{q}$ there is a unique $x_d \in \finfield{q}$ such that $x_1 + \cdots + x_d = z$. Thus, $H^d_z(\finfield{q}) = q^{d-1}$. We will no count how many solutions contain no zeros. By inclusion exclusion, 
\begin{align*}
H^d_z(\finunits{q}) & = H^{d}_z(\finfield{q}) - {d \choose 1} H^{d-1}_z(\finfield{q}) + {d \choose 2} H^{d-2}_z(\finfield{q}) + \cdots + {d \choose d} (-1)^d H^0_z(\finfield{d}) 
\\
& = \sum_{i = 0}^{d-1} {d \choose i} (-1)^i q^{d-1 - i} + (-1)^d \delta_z = \frac{1}{q} \left[ \sum_{i = 0}^{d-1} {d \choose i} (-1)^i q^{d - i} \right]  + (-1)^d \delta_z 
\\
& = \frac{1}{q} \left[ (q-1)^d - (-1)^d \right] + (-1)^d \delta_z
\end{align*}
where the factor of $\delta_z$ comes from the fact that for $z \neq 0$ the set $H^0_z(\finfield{q})$ is empty but for $z = 0$ has one element representing the all zero solution to the original problem. 
Therefore,
\[ H^d_z(\finunits{q}) = \frac{1}{q} \left[ (q-1)^d + (q \delta_z - 1)(-1)^d \right] \]
\end{proof}

\begin{proposition}
\[ m^{d,q}_{0,z} = q^{d-1} - \frac{1}{q} \left[ (q-1)^d + (q \delta_z - 1)(-1)^d \right] \]
\end{proposition}

\begin{proof}
Solutions to the set of simultaneous equations $x_1 + \cdots x_d = z$ and $x_1 \cdots x_d = 0$ are exactly those solutions to $x_1 + \cdots + x_d = z$ which are not all elements of $\finunits{q}$. Therefore,
\[ m^{d,q}_{0,z} = H^d_z(\finfield{q}) - H^d_z(\finunits{q}) = q^{d-1} - \frac{1}{q} \left[ (q-1)^d + (q \delta_z - 1) (-1)^d \right] \]
\end{proof}

\begin{corollary} \label{cor:zero_y_difference}
For $z \neq 0$ we have, $m^{d,q}_{0,z} - m^{d,q}_{0,0} = (-1)^d$
\end{corollary}

\begin{proposition}
\[ \sum_{y \in \finfield{q}} m^{d,q}_{y,z} = q^{d-1} \quad \text{and} \quad \sum_{z \in \finfield{q}} m^{d,q}_{y,z} = 
\begin{cases}
(q-1)^{d-1} & y \neq 0 
\\
q^d -(q-1)^d & y = 0
\end{cases} \]
\end{proposition}

\begin{proof}
\[ \sum_{y \in \finfield{q}} m^{d,q}_{y,z} = \# \left\{ x_1 + \cdots + x_d = z \mid x_i \in \finfield{q} \right\} = H^d_z(\finfield{q}) = q^{d-1} \]
Likewise,
\[ \sum_{z \in \finfield{q}} m^{d,q}_{y,z} = \# \left\{ x_1  \cdots  x_d = z \mid x_i \in \finfield{q} \right\} = \begin{cases}
(q-1)^{d-1} & y \neq 0 
\\
q^d -(q-1)^d & y = 0
\end{cases} \]
because if $y \neq 0$ then every solution to $x_1 \cdots x_d = y$ must have $x_i \neq 0$ for each $i$ and for any choice of $x_1, \cdots, x_{d-1} \in \finunits{q}$ there is a unique choice of $x_d$ such that $x_1 \cdots x_d = y$. Thus, in the case $y \neq 0$ there are exactly $(q-1)^{d-1}$ solutions. However, if $y = 0$ then the condition $x_1 \cdots x_d = 0$ is equivalent to not all $x_i$ being in $\finfield{q}$ and thus $\# (\finfield{q})^d - \# (\finunits{q})^d = q^d - (q-1)^d$. 
\end{proof}

\begin{proposition}
\[ \sum_{y \in \finunits{q}} m^{d,q}_{y,z} = \frac{1}{q}\left[ (q-1)^d + (q \delta_z - 1)(-1)^d \right] \]
\end{proposition}

\begin{proof}
Since having some product $y \neq 0$ is equivalent to all $x_i \neq 0$ we have,
\[ \sum_{y \in \finunits{q}} m^{d,q}_{y,z} = \#\left\{ x_1 + \cdots x_d = z \mid x_i \neq 0 \right\} = H^d_z(\finunits{q}) = \frac{1}{q}\left[ (q-1)^d + (q \delta_z - 1)(-1)^d \right] \]
\end{proof}

\subsection{Relationships Between $m$-values}

\begin{lemma}
\[ \# \left( \finunits{q} / (\finunits{q})^n \right) = \gcd(n, q-1) \] 
\end{lemma}

\begin{proof}
Let $w \in \finunits{q}$ be a generator. The group, $\finunits{q})^n$ is generated by $w^n$ which has order $\frac{q-1}{\gcd(n, q-1)}$. Therefore, $\#(\finunits{q})^n = \frac{q-1}{\gcd(n, q-1)}$ and thus,
\[ \# \left( \finunits{q} / (\finunits{q})^n \right) = \gcd(n, q-1) \]
\end{proof}


\begin{proposition}
Let $\pi : \finunits{q} \to \finunits{q} / (\finunits{q})^d$ be the projection map. If $\pi(y) = \pi(y')$ then $m^{d,q}_{y,0} = m^{d,q}_{y',0}$. 
\end{proposition}

\begin{proof}
Suppose that $\pi(y) = \pi(y')$. Then, $y' = y \lambda^d$. Suppose that $x_1 + \cdots + x_d = 0$ and $x_1 \cdots x_d = y$ is a solution for $m^{d,q}{y,0}$. Then, consider the point $\lambda x_1, \cdots, \lambda x_d$. We have, 
\[ \lambda x_1 + \cdots + \lambda x_d = \lambda (x_1 + \cdots + x_d) = 0 \]
and
\[ \lambda x_1 \cdots \lambda x_d = \lambda^d (x_1 \cdots x_d) = \lambda^d y = y' \]
Therefore, $\lambda x_1, \cdots, \lambda x_d$ is a solution for $m^{d,q}_{y', 0}$. Furhtermore, $\lambda \neq 0$ so multiplication by $\lambda$ is invertible. 
\end{proof}



\begin{corollary}
If $\gcd(d, q-1) = 1$ then $m^{d, q}_{y, 0} = m^{d, q}_{y', 0}$ for all $y, y' \in \finfield{q}$. 
\end{corollary}

\begin{proposition}
Let $\sigma$ be an automorphism of $\finfield{q}$ then $m^{d,q}_{y,z} = m^{d,q}_{\sigma(y), \sigma(z)}$.
\end{proposition}

\begin{proof}
Since $\sigma$ is an automorphism, it is an invertible map which preserves the structure of polynomial equations and therefore gives a bijection between $m^{d,q}_{y,z}$ and $m^{d,q}_{\sigma(y), \sigma(z)}$.
\end{proof}

\begin{proposition}
If $y,z \neq 0$ then for any $\lambda \in \finunits{q}$ we have $m^{d,q}_{y,z} = m^{d,q}_{\lambda^d y, \lambda z}$. 
\end{proposition}


\begin{proof}
Multiplication by $\lambda \in \finunits{q}$ is invertible and takes solutions for $m^{d,q}_{y,z}$ to solutions for $m^{d,q}_{\lambda^d y, \lambda z}$.
\end{proof}

\begin{corollary} \label{cor:equal_for_all_z}
If $q - 1 \divides d$ then for $y, z, z' \neq 0$ we have $m^{d,q}_{y,z} = m^{d,q}_{y,z'}$. 
\end{corollary}

\begin{proof}
We know that for any $\lambda \in \finunits{q}$ we have $m^{d,q}_{y,z} = m^{d,q}_{\lambda^d y, \lambda z}$. However, $q-1 \divides d$ so $d$ is an exponent of $\finunits{q}$ so $\lambda^{d} = 1$. 
\end{proof}

\begin{lemma}
Let $Z_y = \frac{1}{q-1} m^{d,q}_{y,0}$. If $q-1 \divides d$ then $Z_y$ is an integer.
\end{lemma}

\begin{proof}
Any solution $x_1 + \cdots + x_d = 0$ and $x_1 \cdots x_y = y$ can be taken to another distinct solution $\lambda x_1 + \cdots + \lambda x_d = \lambda (x_1 + \cdots + x_d) = 0$ and $\lambda x_1 \cdots \lambda x_d = \lambda^d (x_1 \cdots x_d) = \lambda^d y = y$ by multiplication by $\lambda$. Since $y \neq 0$ we have that $x_1, \cdots, x_d \in \finunits{q}$ for any such solution (since their product is nonzero) and thus multiplication by $\lambda \in \finunits{q}$ acts freely on the set of solutions. Thus, each orbit has size $\#(\finunits{q}) = q - 1$ but the orbits form a partition so $q - 1 \divides m^{d,q}_{y, 0}$. 
\end{proof}


\begin{lemma} \label{lem:determine_all_nonzero}
If for $y, z, z' \neq 0$ we have $m^{d,q}_{y,z} = m^{d,q}_{y,z'}$ then,
\[ m^{d,q}_{y, z} = (q-1)^{d-2} - Z_{y} \]
\end{lemma}

\begin{proof}
For $y,z \neq 0$ we have that,
\[  (q-1) m^{d,q}_{y,z} + m^{d,q}_{y,0}  = \sum_{z \in \finfield{q}} m^{d,q}_{y,z} = (q-1)^{d-1}  \] 
Thus,
\[ m^{d,q}_{y, z} = \frac{1}{q-1} \left[ (q-1)^{d-1} -  m^{d,q}_{y, 0} \right] \]
\end{proof}

\begin{lemma} \label{lem:equal_Z}
If $m^{d,q}_{y, 0} = m^{d,q}_{y', 0}$ for all $y, y' \in \finunits{q}$ then,
\[ m^{d,q}_{y, 0} = \frac{1}{q} \left[ (q-1)^{d-1} + (-1)^d \right] \]
for each $y \in \finunits{q}$. 
\end{lemma}

\begin{proof}
We have that,
\[ (q-1) m^{d,q}_{y, 0} = \sum_{y \in \finfield{q}} m^{d,q}_{y,0} = \frac{1}{q} \left[ (q-1)^d + (q-1)(-1)^d \right] \]
Therefore, 
\[ m^{d,q}_{y, 0} = \frac{1}{q} \left[ (q-1)^{d-1} + (-1)^d \right] \]
\end{proof}



\subsection{Powers of Gaussian Sums}


\begin{theorem} \label{thm:gauss_sums_via_m_vals}
Let $\chi : \finfield{q} \to \C^\times$ be a multiplicative character. If $q - 1 \divides d$ then,
\[ g(\chi)^d = q \sum_{y \in \finunits{q}} Z_y \chi(y) - \delta_{\chi} \cdot \left[ (q-1)^{d-1} + (-1)^d \right] \] 
\end{theorem}

\begin{proof}
Let $\phi : \finfield{q} \to \C^\times$ be a nontrivial additive character. Consider,
\begin{align*}
g(\chi)^d & = \left[ \sum_{x \in \finfield{q}} \chi(x) \psi(x) \right]^d = \sum_{x_1 \in \finfield{q}} \cdots \sum_{x_d \in \finfield{q}} \chi(x_1) \cdots \chi(x_d) \psi(x_1) \cdots \psi(x_d) 
\\
& = \sum_{x_1 \in \finfield{q}} \cdots \sum_{x_d \in \finfield{q}} \chi(x_1 \cdots x_d) \psi(x_1 + \cdots + x_d) = \sum_{y \in \finfield{q}} \sum_{z \in \finfield{q}} \sum_{\substack{x_1 + \cdots + x_d = z \\ x_1 \cdots x_d = y}} \chi(y) \psi(z) 
\\
& = \sum_{y \in \finfield{q}} \chi(y) \sum_{z \in \finfield{q}} m^{d,q}_{y, z} \psi(z) 
\end{align*}
However, since $q - 1 \divides d$, by Lemma \ref{cor:equal_for_all_z} we know that $m^{d,q}_{y,z} = m^{d,q}_{y,z'}$ if $y, z, z' \in \finunits{q}$. Therefore,
\begin{align*}
g(\chi)^d & = \sum_{y \in \finunits{q}} \chi(y) \sum_{z \in \finfield{q}} m^{d,q}_{y, z} \psi(z) + \chi(0) \sum_{z \in \finfield{q}} m^{d,q}_{0, z} \psi(z)
\\
& = \sum_{y \in \finunits{q}} \chi(y) \left[ m^{d,q}_{y,0} \psi(0) + m^{d,q}_{y,z} \sum_{z \in \finunits{q}} \psi(z) \right] + \chi(0) \left[ m^{d,q}_{0, 0} \psi(0) + m^{d,q}_{0,z} \sum_{z \in \finfield{q}} \psi(z) \right]
\end{align*}
Because $\psi$ is a nontrivial character,
\[ \sum_{z \in \finfield{q}} \psi(z) = 0 \implies \sum_{z \in \finunits{q}} \psi(z) = - 1 \]
since $\psi(0) = 1$. Therefore, 
\begin{align*}
g(\chi)^d & = \sum_{y \in \finunits{q}} \chi(y) \left[ m^{d,q}_{y,0} - m^{d,q}_{y,z} \right] + \chi(0) \left[ m^{d,q}_{0, 0} - m^{d,q}_{0,z} \right]
\end{align*}
where $z$ is an arbitrary nonzero element (since these numbers are independent of choice of $z \neq 0$). Furthermore, by Lemma \ref{lem:determine_all_nonzero} we know that,
\[ m^{d,q}_{y,0} - m^{d,q}_{y,z} = m^{d,q}_{y,0} + \frac{1}{q-1} m^{d,q}_{y,0} - (q-1)^{d-2} = q Z_y - (q-1)^{d-2} \]
Furthermore, by Lemma \ref{cor:zero_y_difference}, $m^{d,q}_{0,z} - m^{d,q}_{0,0} = (-1)^d$. Putting these facts together,
\begin{align*}
g(\chi)^d & = \sum_{y \in \finunits{q}} \chi(y) \left[ q Z_y - (q-1)^{d-2} \right] - \chi(0) (-1)^d
\end{align*}
Now we consider the case when $\chi$ is the trivial character $\chi_0$ and when $\chi \neq \chi_0$. When $\chi \neq \chi_0$ we know that $\chi(0) = 0$ and that,
\[ \sum_{y \in \finunits{q}} \chi(y) = 0 \]
Therefore we get,
\[ g(\chi)^d = q \sum_{y \in \finunits{q}} Z_y \chi(y) \]
When $\chi$ is the trivial character, $\chi(y) = 1$ for all $y \in \finfield{q}$. Therefore,
\begin{align*}
g(\chi)^d = q \sum_{y \in \finunits{q}} Z_y \chi(y) - \left[ (q-1)^{d-1} + (-1)^d \right]  
\end{align*}
\end{proof}

\begin{theorem} \label{thm:m_vals_via_gauss_sums}
Let $\widehat{\finfield{q}}$ be the character group of $\finfield{q}$ and $q - 1 \divides d$. Then,
\[ Z_y = \frac{1}{q(q-1)} \left( \sum_{\chi \in \widehat{\finfield{q}}} g(\chi)^d  \: \overline{\chi}(y) + \left[ (q-1)^{d-1} + (-1)^d \right] \right) \]
\end{theorem}

\begin{proof}
By Theorem \ref{thm:m_vals_via_gauss_sums}, we know that,
\[ q \sum_{y \in \finunits{q}} Z_y \chi(y) = g(\chi)^d + \delta_{\chi} \left[ (q - 1)^{d-1} + (-1)^d \right]  \]
We will make use the character orthogonality relation,
\[ \sum_{\chi \in \widehat{\finfield{q}}} \chi(x) \overline{\chi}(y) = 
\begin{cases}
(q - 1) & x = y
\\
0 &  x \neq y
\end{cases} \]
for $x,y \in \finunits{q}$.
Using this relation,
\begin{align*}
\sum_{\chi \in \widehat{\finfield{q}}} \left( g(\chi)^d + \delta_{\chi} \left[ (q - 1)^{d-1} + (-1)^d \right] \right) \overline{\chi}(y) = q \sum_{\chi \in \widehat{\finfield{q}}} \sum_{z \in \finunits{q}} Z_z \chi(z) \overline{\chi}(y) = q \sum_{z \in \finunits{q}} Z_z (q - 1) \delta_{y - z} = q (q - 1) Z_z
\end{align*}
Furthermore, for $\chi = \chi_0$ we have $\overline{\chi}(y) = 1$. Thus, 
\[ q (q - 1) Z_z = \sum_{\chi \in \widehat{\finfield{q}}} g(\chi)^d \: \overline{\chi}(y) + \left[ (q - 1)^{d-1} + (-1)^d \right] \]

\end{proof}



\subsection{Special Cases of Sum-Product Varieties}

\begin{definition}
The sum-product variety, $V_{\lambda}^{d,q}$ is defined by the equation $x_1 + \cdots + x_d = \lambda x_1 \cdots x_d$ over $\finfield{q}$. Clearly, the number of points on a sum-product variety is given by,
\[ \#(V^{d,q}_{\lambda}) = \sum_{y \in \finfield{q}} m^{d,q}_{y, \lambda y} \] 
\end{definition}

\begin{proposition}
Suppose that $m^{d,q}_{y, z} = m^{d,q}_{y, z'}$ for all $y,z,z' \in \finunits{q}$ then,
\[ \#(V^{d,q}_{\lambda}) = q^{d-1} - (-1)^d \]
\end{proposition}

\begin{proof}
We know that,
\begin{align*}
\#(V^{d,q}_{\lambda}) & = \sum_{y \in \finfield{q}} m^{d,q}_{y, \lambda y} = m^{d,q}_{0,0} + \sum_{y \in \finunits{q}} m^{d,q}_{y, \lambda y} = m^{d,q}_{0,0} + \sum_{y \in \finunits{q}} m^{d,q}_{y, 1} = \sum_{y \in \finfield{q}} m^{d,q}_{y, 1} + [m^{d,q}_{0,0} - m^{d,q}_{0,1}]
\\
& = q^{d-1} - (-1)^d
\end{align*}
\end{proof}

\begin{corollary}
If $q - 1 \divides d$ then,
\[ \#(V^{d,q}_{\lambda}) = q^{d-1} - (-1)^d \]
\end{corollary}

\begin{proposition}
The number of points on a sum-product variety is determined entirely by $m^{d,q}_{\lambda^{-1}, 0}$ via,
\[ \#(V^{d,q}_{\lambda}) = \#(V^{d,q}_{\lambda}) = q^{d-1} - (q-1)^{d-2} + q m^{d,q}_{\lambda^{-1}, 0} \]
\end{proposition}

\begin{proof} \label{prop:points_sum_product_V}
Choose any $x_1, \cdots, x_{d-1} \in \finfield{q}$. Denote $S = x_1 + \cdots + x_{d-1}$ and $P = x_1 \cdots x_{d-1}$. Then finding a point on the variety is equivalent to solving,
\[ S + x_d = \lambda P x_d \iff x_d = \frac{S}{\lambda P - 1} \]
when $P \neq \lambda^{-1}$. Therefore, for any choice of $x_1, \cdots, x_{d-1} \in \finfield{q}$ there is a unique point on the variety when $P \neq \lambda^{-1}$. When $P = \lambda^{-1}$ there are no solutions for $S \neq 0$ and any $x_d$ gives a point on the variety if $S = 0$. There are $q^{d-1} - (q-1)^{d-2}$ choices for $x_1, \cdots, x_{d-1} \in \finfield{q}$ which do not have $P = \lambda^{-1}$ since to get $P = \lambda^{-1}$ we can take the first $d-2$ to be arbitrary elements of $\finunits{q}$ and then there is a unique $x_{d-1} \in \finunits{q}$ such that $x_1 \cdots x_{d-1} = \lambda^{-1}$. Thus, the total number of solutions is,
\[ \#(V^{d,q}_{\lambda}) = q^{d-1} - (q-1)^{d-2} + q m^{d,q}_{\lambda^{-1}, 0} \]
\end{proof}


\begin{proposition}
If $m^{d,q}_{y, 0} = m^{d,q}_{y', 0}$ for all $y, y' \in \finunits{q}$ then,
\[ \#(V^{d,q}_{\lambda}) = q^{d-1} + (q-2) (q-1)^{d-2} + (-1)^d \]
for each $\lambda \in \finunits{q}$.
\end{proposition}

\begin{proof}
By Lemma \ref{lem:equal_Z} we know that,
\[ m^{d,q}_{\lambda^{-1}, 0} = \frac{1}{q} \left[ (q-1)^{d-1} + (-1)^d \right] \]
Therefore, by Proposition \ref{prop:points_sum_product_V},
\[ \#(V^{d,q}_{\lambda}) = q^{d-1} - (q-1)^{d-2} + (q-1)^{d-1} + (-1)^d  = q^{d-1} + (q-2) (q-1)^{d-2} + (-1)^d \]
\end{proof}


\begin{corollary} \label{cor:coprime_sum_product_points}
If $\gcd(d, q-1) = 1$ then for each $\lambda \in \finunits{q}$,
\[ \#(V^{d,q}_{\lambda}) = q^{d-1} + (q-2) (q-1)^{d-2} + (-1)^d \]
\end{corollary}

\begin{theorem}
Let $q = p^r$ and $d = p^s$ then, for each $\lambda \in \finunits{q}$, the zeta function of the variety, $V^{d, q}_{\lambda}$ equals,
\[ \zeta_{V^{d, q}_{\lambda}} = \frac{1}{1 - q^{d-1} t} \left[ \frac{1}{1 - t} \right] ^{(-1)^d}  \prod_{i = 0}^d \left[ \frac{(1 - q^i t)^2}{1 - q^{i+1} t} \right]^{{d \choose i} (-1)^{d-i}} \]
and therefore, $V^{d,q}_{\lambda}$ is supersingular.
\end{theorem}

\begin{proof}
\begin{align*}
\zeta_{V^{d, q}_{\lambda}} = \exp{\left( \sum_{k \ge 1} \frac{\#(V^{d,q^k}_{\lambda})}{k} t^k \right)}
\end{align*}
However, $(d, q^k - 1) = (p^s, p^{rk} - 1) = 1$ for all $k$. Therefore, by Corollary \ref{cor:coprime_sum_product_points}, 
\[ \#(V^{d,q^k}_{\lambda}) = q^{(d-1)k} + (q^k - 2) (q^k-1)^{d-2} + (-1)^d = q^{k(d-1)} + (-1)^d + (q^k - 2) \sum_{i = 0}^d {d \choose i} (-1)^{d-i} q^{ki} \] 
Thus,
\begin{align*}
\zeta_{V^{d, q}_{\lambda}} & = \exp{\left(  \sum_{k \ge 1} \frac{q^{k(d-1)}}{k} t^k  + \frac{(-1)^d}{k} t^k + (q^k - 2) \sum^d_{i = 0} \left[ {d \choose i} (-1)^{d-i} \sum_{k \ge 1} \frac{q^{ki}}{k} t^k \right] \right)} 
\\
& = \exp{\left(  \sum_{k \ge 1} \frac{q^{k(d-1)}}{k} t^k  + \frac{(-1)^d}{k} t^k + \sum^d_{i = 0} \left[ {d \choose i} (-1)^{d-i} \sum_{k \ge 1} \frac{q^{k(i+1)}}{k} t^k \right] - 2 \sum^d_{i = 0} \left[ {d \choose i} (-1)^{d-i} \sum_{k \ge 1} \frac{q^{ki}}{k} t^k \right] \right)} 
\\
& = \exp{\left( - \log{[1 - q^{d-1} t]}  - (-1)^d \log{[1 - t]} - \sum^d_{i = 0} \left[ {d \choose i} (-1)^{d-i} \log{[1 - q^{i+1}]} \right] + 2 \sum^d_{i = 0} \left[ {d \choose i} (-1)^{d-i} \log{[1 - q^{i}]} \right] \right)} 
\\
& = \frac{1}{1 - q^{d-1} t} \left[ \frac{1}{1 - t} \right] ^{(-1)^d}  \prod_{i = 0}^d \left[ \frac{(1 - q^i t)^2}{1 - q^{i+1} t} \right]^{{d \choose i} (-1)^{d-i}}
\end{align*}
\end{proof}

\begin{lemma}
Let $w \in \finunits{q}$ be a generator. Then, $a = w^r$ is a $n^{\mathrm{th}}$ power if and only if $\gcd(n q-1) \divides r$.  
\end{lemma}

\begin{proof}
Suppose that $a = b^n$ where $b = w^x$. Then, $w^r = w^{n x}$ which is equivalent to $nx \equiv r \mod (q-1)$. This equation has solutions if and only if $\gcd(n, q-1) \divides r$.  
\end{proof}

\section{On the Relationships Between Diagonal Varieties}

\begin{lemma}
Let $\varphi : X \to Y$ be a surjective morphism then the induced map on $\ell$-adic cohomology $\varphi^* : H^*(Y, \Q_{\ell}) \to H^*(X, \Q_{\ell})$ is injective. 
\end{lemma}

\begin{proof}
See Kleiman, Algebraic Cycles and the Weil Conjectures, Proposition 1.2.4. Further, use the fact that $\ell$-adic cohomology is a Weil cohomlogy theory. 
\end{proof}

\begin{proposition} \label{prop:surjective_induces_injection_cohomology}
We say a scheme $X$ over $\finfield{q}$ is supersingular if and only if the frobenius map $F_X : X \to X$ induces a map $F^*_X : H^i(X, \Q_{\ell}) \to H^i(X, \Q_{\ell})$ on $\ell$-adic cohomology with all eigenvalues of the form $\omega q^{\frac{i}{2}}$ where $\omega$ is a root of unity.  
\end{proposition}

\begin{theorem} \label{lem:map_gives_supersingular_implication}
Let $\varphi : X \to Y$ be a surjective morphism then $X$ being supersingular implies that $Y$ is supersingular. 
\end{theorem}

\begin{proof}
The induced map $\varphi^* : H^i(Y, \Q_{\ell}) \to H^i(X, \Q_{\ell})$ is injective by Proposition \ref{prop:surjective_induces_injection_cohomology} and commutes with the Frobenuius maps,
\begin{center}
\begin{tikzcd} 
H^i(Y, \Q_{\ell}) \arrow[r, hook, "\varphi^*"] \arrow[d, "F_Y^*"] & H^i(X, \Q_{\ell}) \arrow[d, "F_X^*"]
\\
H^i(Y, \Q_{\ell}) \arrow[r, hook, "\varphi^*"] & H^i(X, \Q_{\ell})
\end{tikzcd}
\end{center}
Suppose that $X$ is supersingular then every eigenvalue of $F*_X : H^i(X, \Q_{\ell}) \to H^i(X, \Q_{\ell})$ has the form $\lambda = \omega q^{\frac{i}{2}}$ where $\omega$ is a root of unity. Suppose that $v \neq 0$ is an eigenvector of $F^*_Y$ such that $F^*_Y = \lambda v$. By commutativity of the diagram,
\[ \varphi^* \circ F^*_Y(v) = F^*_X(\varphi^*(v)) \]
Furthermore, since $\varphi^*$ is a linear map, 
\[ \varphi^* \circ F^*_Y(v) = \varphi^*(\lambda v) = \lambda \varphi^*(v) \]
and therefore,
\[ F^*_X(\varphi^*(v)) = \lambda \varphi^*(v) \]
Since $\varphi^*$ is injective and $v \neq 0$ we know that $\varphi^*(v) \neq 0$ so $\varphi^*(v)$ is an eigenvector of $F^*_X$ with eigenvalue $\lambda$. Therefore, since $X$ is supersingular, $\lambda = \omega q^{\frac{i}{2}}$ with $\omega$ a root of unity. Since $\lambda$ is an abitrary eigenvalue of $F^*_Y$ we have that $Y$ is supersingular. 
\end{proof}

\begin{definition}
Let $X$ and $Y$ be diagonal varieties of dimension $r-1$ over the field $k$, defined respectively by the equations, 
\[ a_0 x_0^{n_0} + \cdots + a_r x_r^{n_r} = 0 \text{ and } b_0 x_0^{m_0} + \cdots + b_r x_r^{n_r} = 0 \]
Then we say that $X \divides Y$ iff $n_i \divides m_0$ for each $0 \le i \le r$. 
\end{definition}


\begin{lemma} \label{lem:if_divides_then_map}
If $X$ and $Y$ are diagonal varieties of dimension $r-1$ over an algebraically closed field $k$ and $X \divides Y$ then there exists a surjective morphism, $\varphi : Y \to X$.  
\end{lemma}

\begin{proof} 
  
Define the map $\varphi : Y \to X$ via,
\[ (x_0, \dots, x_r) \mapsto (x_0^{\frac{m_0}{n_0}}, \dots, x_r^{\frac{m_0}{n_0}}) \]
This map is well-defined because if the point $(x_0, \dots, x_r)$ satisfies,
\[ x_0^{m_0} + \cdots + x_r^{m_r} = 0 \]
Then the point $(y_0, \dots, y_r) = (x_0^{\frac{m_0}{n_0}}, \dots, x_r^{\frac{m_0}{n_0}})$ satisfies the equation,
\[ y_0^{n_0} + \cdots + y_r^{n_r} \]
Furthermore, $\varphi$ is surjective because $k$ is algebraically closed and thus each $y_i \in k$ is an $\left(\frac{m_i}{n_i}\right)^{\mathrm{th}}$ power. 
\end{proof}

\begin{remark}
Theorem \ref{thm:resolution_of_sigularities} is a special case of this result in which the map $\varphi$ has additional properties due to the characteristic of $k$.
\end{remark}

\begin{corollary} \label{cor:divides_then_supersingular_implication}
Suppose $X \divides Y$. If $Y$ is supersingular then $X$ is supersingular. 
\end{corollary}

\begin{proof}
This follows immediately from Lemma \ref{lem:map_gives_supersingular_implication} and Lemma \ref{lem:if_divides_then_map}. However, we also give an elementary proof. Take $q$ to be a power of $p$ such that $q \equiv 1$ modulo the LCM for $X$ and $Y$. Since $X \divides Y$ each $\alpha \in A_{X,q}$ for $X$ satisfies the correct divisibility relations for $Y$. Thus, $A_{X,q} \subset A_{Y,q}$. Therefore, if $Y$ is supersingular then each $\alpha \in A_{Y,q}$ gives a product of gauss sums which is a root of unity. Since $A_{X,q} \subset A_{Y,q}$ the same holds for $X$ so $X$ is supersingular.  
\end{proof}

\begin{corollary}
Let $X$ be a diagonal variety over an algebraically closed field $k$ defined by the equation,
\[ a_0 x_0^{n_0} + \cdots + a_r x_r^{n_r} = 0 \]
Define the LCM extension $X_{\ell}$ and GCD reduction $X_g$ of $X$ by,
\[ X_{\ell} = F^{\lcm{(n_i)}}_r \text{ and } X_g = F^{\gcd{(n_i)}}_r \]
respectively. Then there exist surjective maps,
\begin{center}
\begin{tikzcd}
X_{\ell} \arrow[r, "\varphi_{\ell}"] & X \arrow[r, "\varphi_g"] & X_g
\end{tikzcd}
\end{center}
\end{corollary}

\begin{corollary} \label{cor:important_reduction_maps}
If $X_{\ell}$ is supersingular then $X$ is supersingular. If $X_g$ is not supersingular then $X$ is not supersingular. 
\end{corollary}

\begin{theorem}
Let $X$ be a diagonal variety. Then $X$ is supersigular over $\finfield{p}$ if there exists $v \in \Z$ such that $p^v \equiv -1 \mod {\lcm(n_i)}$ and $X$ is not supersingular if for all $v \in \Z$ we have $p^v \not\equiv -1 \mod{\gcd{(n_i)}}$.  
\end{theorem}

\begin{proof}
This follows from Shioda's theorem via Corollary \ref{cor:important_reduction_maps}.
\end{proof}

\section{On Newton Polygon}
\begin{proposition}
The set of slopes that appear in the Newton polygon is determined by 
\[\frac{1}{(p-1) f}\sum_{i = 0}^3 s(\frac{(q-1) r_i}{m}) - 1,\]
where $\sum \frac{r_i}{m} \in \Z$, i. e., the set of $\frac{r_i}{m}$ is in the set of all possible $\alpha$.
\end{proposition}
\begin{proof}
See Koblitz's paper p-adic variation of the zeta function over the families of varieties defined over finite fields. 
\end{proof}

\begin{proposition}
When $f = 1$, the Newton Polygon of the Fermat variety $F_{p, r}^n$ is of the form \[(0, 0),\; (0, a), (b_2 - a, b_2 - 2a), (b_2, b_2),\] 
where $a = \binom{m-1}{3}$, and $b_2$ is the second betti number.
\end{proposition}
\begin{proof}
Since $f = 1$, we know that \[\sum_{i = 0}^3 s(\frac{(q-1) r_i}{m}) = \sum_{i = 0}^3 \left\{  \frac{r_i}{m}  \right\} \]
But $m | r_0 +  r_1 + r_2 + r_3$, so the only possible value for $\sum_{i = 0}^3 \left\{  \frac{r_i}{m}  \right\} $ is $1, 2, 3$, and these corresponds to slope $0, 1, 2$. 

To count the length of $x$-axis where the slope is $0$, we need to find the number of solution to the equation \[r_0 + r_1 + r_2 + r_3 = m,\] which is $\binom{m-1}{3}$. By duality of the cohomology, this length is equal to the length of the last segment, i. e., the segment with slope $2$.
\end{proof}

\section{On Surface of the form $x^p + y^q + z^{ps} + w^{qs}$}

\begin{theorem}
Let $p, q, w$ be primes such that $p, q, w \equiv 1 \mod{s}$ for some $s$ and let $X$ be the variety defined by,
\[ x_0^p + x_1^{ps} + x_2^{q} + x_3^{qs} = 0 \]
over $\finfield{w}$. If $w$ is a primitive root modulo $p$ and $q$ then $X$ is supersingular.  
\end{theorem}

\begin{proof}
By Theorem \ref{thm:gauss_sum_is_root_of_unity_ideal_factorization_counting_condition}, we need only check that for each $\alpha = (e_0 / m, \dots, e_3/m) \in A(X)$ that,
\[ S_\mu(e_0, e_1, e_2, e_3) = \sum_{i = 0}^3 \sum_{j = 0}^{f-1} \left\{ \frac{\mu e_i w^j}{m} \right\} = 2 f \]
where $m = pqs$ and $f = \ord_{pqs}(w)$. However, we also know that $\alpha$ can be written as a tuple, $(a_0, \dots, a_3)$ such that,
\[ \frac{a_0}{p} + \frac{a_1}{ps} + \frac{a_2}{q} + \frac{a_3}{qs} = \frac{sa_0 + a_1}{ps} + \frac{s a_2 + a_3}{qs} = \frac{q (sa_0 + a_1) + p(s a_2 + a_3)}{pqs} \in \Z \]
Since $p$ and $q$ are coprime, we must have,
\[ p \divides s a_0 + a_1 \quad \text{and} \quad q \divides s a_2 + a_3 \]
Thus, let, $s a_0 + a_1 = p n_p$ and $s a_2 + a_3 = q n_q$. This reduces the above condition to,
\[ \frac{n_p}{s} + \frac{n_q}{s} \in \Z  \iff n_p + n_q \equiv 0 \mod{s} \]
Now, using Lemma \ref{lem:sum_frac_part_N(e_0,e_1)},
\begin{align*}
S_\mu(e_0, e_1, e_2, e_3) & = S_\mu(e_0, e_1) + S_\mu(e_2, e_3)
\\
& = N_\mu(e_0, e_1) + N_\mu(e_2, e_3) + \sum_{j = 0}^{f-1} \left[ \left\{ \frac{\mu (e_0 + e_1) w^j}{m} \right\} +  \left\{ \frac{\mu (e_2 + e_3) w^j}{m} \right\} \right]
\end{align*}
However, $e_0 + e_1 = q(s a_0 + a_1) = pq n_p$ and $e_2 + e_3 = p (s a_2 + a_3) = pq n_q$ and thus,
\begin{align*}
\sum_{j = 0}^{f-1} \left[ \left\{ \frac{\mu (e_0 + e_1) w^j}{m} \right\} + \left\{ \frac{\mu (e_2 + e_3) w^j}{m} \right\} \right] & = \sum_{j = 0}^{f - 1} \left[ \left\{ \frac{\mu n_p w^j}{s} \right\} + \left\{ \frac{\mu n_q w^j}{s} \right\}  \right] = \sum_{j = 0}^{f - 1} 1 = f
\end{align*}
since $\mu w^j(n_p + n_q) \equiv 0 \mod s$. We need not worry about the case $n_p \equiv n_q \equiv 0 \mod s$ because in that case $m \divides e_0 + e_1$ and $m \divides e_2 + e_3$ so $S_\mu(e_0, e_1) = S_\mu(e_2, e_3) = f$ which is the condition we need. 
\bigskip\\
It remains to show that,
\[ N_\mu(e_0, e_1) + N_\mu(e_2, e_3) = f \implies S_\mu(e_0, e_1, e_2, e_3) = 2 f\]
Consider the number, $N_\mu(e_0, e_1)$ which counts all $0 \le j < f$ such that,
\[ \left\{ \frac{\mu n_p w^j}{s} \right\} < \left\{ \frac{\mu a_0 w^j}{p} \right\}  \]
However, $w \equiv 1 \mod s$ and thus,
\[ \left\{ \frac{\mu n_p w^j}{s} \right\} = \left\{ \frac{\mu n_p}{s} \right\} = \frac{[\mu n_p]_s}{s} \] 
Furthermore, $w$ is a primitve root modulo $p$ so the numbers $\mu a_0 w^j$ give a complete set of residues modulo $p$. Because $p - 1 = \ord_{p}(w) \divides \ord_{pqs}(w) = f$ we can write $f = u_p(p - 1)$ and similarly $f = u_q(q - 1)$. Therefore,
\[ N_\mu(e_0, e_1) = u_p \left[\#\left\{ 0 \le i < p - 1 \: \bigg| \: \frac{[\mu n_p]_s}{s} < \frac{i}{p} \right\}\right] = u_p \left( p - 1 - \left\lfloor \frac{p [\mu n_p]_s}{s} \right\rfloor \right) \]
However, $p \equiv 1 \mod s$ so $p = s k_p + 1$ and thus because $0 < [\mu n_p]_s < s$ we have,
\[ \left\lfloor k_p [\mu n_p]_s + \frac{[\mu n_p]_s}{s} \right\rfloor  = k_p [\mu n_p]_s \]  
Finally,
\[ N_\mu(e_0, e_1) = f - u_p k_p [\mu n_p]_s = f - u_p \frac{p - 1}{s} [\mu n_p]_s = f \left(1 - \frac{[\mu n_p]_s}{s} \right) \] 
and identical argument gives,
\[ N_\mu(e_2, e_3) = f \left(1 - \frac{[\mu n_q]_s}{s} \right) \] 
and thus,
\[ N_\mu(e_0, e_1) + N_\mu(e_2, e_3) = f \left(2 - \frac{[\mu n_p]_s + [\mu n_q]_s}{s} \right) = f \]
because $[\mu n_p]_s + [\mu n_q]_s = s$. 
\end{proof}

\begin{theorem}
Let $X$ be the variety defined by,
\[ a_0 x_0^{n_0} + \cdots + a_r x_r^{n_r} = 0 \]
and let $n = \lcm{n_i}$. Now define the polynomial,
\[ B_X(x) = \left[ \prod_{i = 0}^r \frac{x^{2n} - x^{2 w_i}}{x^{2w_i} - 1} - \prod_{i = 0}^r \frac{x^{n(r+1)} - x^{w_i(r+1)}}{x^{w_i(r+1)} - 1} \right] \]
Suppose that $p \equiv 1 \mod{n}$ then the total degree of $X$ minus the picard number of $X$ is given by,
\[ P^C(X) = \sum_{i = 1}^{n(r+1)} B_X(\zeta_{n(r+1)}^i)  \]
In particular, $X$ is supersingular iff $P^C(X) = 0$.
\end{theorem}

\begin{proof}
When $p \equiv 1 \mod{n}$ then $f = 1$ so we know that a given product of Gaussian sums applied for $\alpha \in A_{n,p}$ is a root of unity if and only if,
\[ \sum_{i = 0}^r \left\{ \frac{\mu e_0}{n} \right\} = \frac{r+1}{2} \]
for each $\mu \in (\Z / n \Z)^\times$. (WIP)
\end{proof}

\section{On Rationality}
\begin{theorem}
The variety $X$ defined by equation \[x^q + y^q + z^p + w^p = 0\] is rational when $\gcd(p, q) = 1$.
\end{theorem}
\begin{proof}
This variety is in the weighted projected space $\Pro(p,p,q,q)$. 
We want to define a map $f$ from $\Pro(p,p,q,q)$ to $\Pro \times \Pro$ by \[(x_0 : x_1 : x_2: x_3) \mapsto ((x_0:x_1), (x_2: x_3)),\] and we consider the locus $D_+(x_0x_2) \subset \Pro(p,p,q,q)$ and its image $D_+(x_0) \times D_+(x_2) \cong \A \times \A \subset \Pro \times \Pro$ under $f$.

We know that \[D_+(x_0x_2) = \Spec \,R \mathrm{, where} \; R = k\left[x_0, x_1, x_2, x_3\right]\left[\frac{1}{x_0x_2} \right]_0 \]

Define the change of variable \[x_{1,0} = \frac{x_1}{x_0}, \;\; x_{3,2} = \frac{x_3}{x_2}, \;\; x_{2,0} = \frac{x_2^p}{x_0^q},\]
we content that $D_+(x_0x_2) = \Spec(k[x_{1,0}, x_{3,2}, x_{2, 0}, x_{2, 0}^{-1}])$, as proved in lemma. \\

On the other hand, we can write $D_+(x_0) \times D_+(x_2) = \Spec(k[s] \otimes_k k[t]) = \A \times \A$ by let \[s = \frac{x_1}{x_0}, \;\; t = \frac{x_3}{x_2}.\]

Then we can define the ring map
\[f_*: k[s] \otimes_k k[t] \to R\] by \[s \mapsto x_{1,0}, \;\; t \mapsto x_{3,2}.\]

Now consider the variety $X = V(x_0^q + x_1^q  + x_2^p + x_3^p) = V(I)$ in the affine patch $D_+(x_0x_2)$. The defining equation of $X$ after change of variable can be written as
\[1 + x_{1,0}^q + x_{2,0} + x_{3,2}^p x_{2,0} = x_{2,0}(1 + x_{3,2}^p) + (1 + x_{1,0}^q)\]

Thus it is clear that \[k[x_{1,0}, x_{3,2}, x_{2, 0}, x_{2, 0}^{-1}]/(x_{2,0}(1 + x_{3,2}^p) + (1 + x_{1,0}^q)) \cong \Frac(R/I)\]

Notice that $\overline{f^*} : k[s] \otimes_k k[t] \to \Frac(R/I)$ is surjective because we can write $x_{2,0}$ and $x_{2,0}^{-1}$ as a rational function in term of $x_{1,0}$ and $x_{3,2}$. 
Furthermore, it is easy to see that $f^*$ is injective. Thus, $f^*$ is a bijective rational map. For the inverse map of $f^*$, we map 
\[x_{1,0} \mapsto s, \;\; x_{3,2} \mapsto t.\]
We thus show that $X$ is birationally equivalent to $\P \times \P$. 
\end{proof}

\begin{lemma}
Let $R = k\left[x_0, x_1, x_2, x_3\right]$ be a weighted ring with weight $(p,p,q,q)$ and $\gcd(p, q) = 1$. Then \[R_+ = k\left[x_0, x_1, x_2, x_3\right]\left[\frac{1}{x_0x_2} \right]_0 \cong k[x_{1,0}, x_{3,2}, x_{2, 0}, x_{2, 0}^{-1}],\] where \[x_{1,0} = \frac{x_1}{x_0}, x_{3,2} = \frac{x_3}{x_2}, x_{2,0} = \frac{x_2^p}{x_0^q}.\]
\end{lemma}

\begin{proof}
We proceed by showing that if \[m = \frac{x_0^{a_0}x_1^{a_1}x_2^{a_2}x_3^{a_3}}{x_0^{b_0}x_2^{b_2}}\] for $a_i, b_j > 0$ with $i = 0, 1,2,3$ and $j = 0, 1$, and $m$ has degree $0$, then $m$ can be written as a product  of $x_{1,0}, x_{3,2}, x_{2,0}, \text{ or } x_{0,2}$.

If $a_0 > b_0$ and $a_2 > b_2$, then it is impossible for $m$ to have degree $0$.

If $a_0 > b_0$ and $a_2 < b_2$, then let $b_2 - a_2 = c_2$ and $a_0 - b_0 = c_0$. For $m$ to have degree $0$, we need 
\[pc_0 + pa_1 + qa_3 = qc_2.\] 
Since $\gcd(p, q) = 1$, it must be the case that $q | (c_0 + a_1)$. Write $c_0 + a_1 = qk$ for some $k \in \Z$. Our equation now become 
\[pk + a_3 = c_2\] Thus we can write $m$ as 
\[m = \left( \frac{x_0^{a_1}x_0^{c_0} x_1^{a_1} x_3^{a_3}}{x_0^{a_1}x_2^{pk}}\right) \left( \frac{x_3}{x_2}\right)^{a_3}  = x_{1,0}^{a_1} x_{3,2}^{a_3} x_{0,2}^{k}\]

If $a_0 < b_0$ and $a_2 < b_2$, let $c_0 = b_0 - a_0$ and $c_2 = b_2 - a_2$. Then we have the equation 
\[pa_1 + qa_3 = pc_0 + qc_2\] with $a_1, a_3, c_0, c_2 > 0$.

Since $\gcd(p, q) = 1$, we can write $d_1 p + d_2 q = 1$, and $|d_1|  < q$ and $|d_2| < p$. Notice that  $d_1d_2 < 0$. 

Moreover, any other such equation can be written as $(d_1 + qr)p +  (d_2 - pr)q = 1$ for $r \in \Z$. Without loss of generality, let $d_1 > 0$ and $d_2 < 0$. Then 
\begin{align*}
(d_1 + qr)(d_2 -pr) &= d_1d_2 - prd_1 + r(1-d_1p) - pqr^2 \\
&= d_1d_2 + r - 2d_1pr - pqr^2
\end{align*}
If $r > 0$, the only positive term is $r$ thus we know $(d_1 + qr)(d_2 -pr) < 0$. 

If $r < 0$, we have $-2d_1 pr > 0$, but $2d_1p < pq|r|$ since $|d_1| < q$. Thus, it is impossible for both of the coefficient to be positive at the same time. However, $a_1, a_3, c_0, c_2 > 0$. Therefore, it is also impossible for $m$ in this case to have degree $0$.
\end{proof}



\section{Surfaces of the Form $x^{a}+y^{b}+z^{c}+w^{abc}$}

\begin{lemma}  \label{lem:prim_char_supersing_contradiction} Let $p$ be a prime, $n$ be an integer not divisible by $p$, and $f = \ord_n(p)$. Suppose that for all $\mu$ relatively prime to $n$:
\[\sum_{i=0}^{f-1} \frp{\mu p^i}{n} = \frac{f}{2}\]
Then there does not exist a primitive character $\chi$ modulo $n$ such that $\chi(-1) = -1$ and $\chi(p) = 1$.
\end{lemma}
\begin{proof} (Modified from a theorem in Shioda's On Fermat Varieties) Suppose there does exist such a character. As $\chi$ is primitive with $\chi(-1) = -1$,
\[0 \neq L(1, \chi) = \frac{i\pi g(\chi)}{n^2} \sum_{k = 1}^{n} \bar{\chi}(k)k\]
As $g(\chi)$ is non-zero we must have: 
\[\sum_{k = 1}^{n}\bar{\chi}(k)k \neq 0\]
Now let $G$ be $(\Z/abc\Z)^{\times}$ and let $H$ be the subgroup of $G$ generated by $p$. As $\chi$ is trivial on $H$:
\[\sum_{k = 1}^{n}\bar{\chi}(k)k = \sum_{\mu \in G/H} \chi(\mu) \sum_{k \in \mu H} k\]
Now we have that:
\[ \frac{f}{2} = \sum_{i=0}^{f-1} \frp{\mu p^i}{n} = \sum_{k \in \mu H} \frac{k}{n}\]
Thus 
\[\sum_{k = 1}^{n}\bar{\chi}(k)k = \frac{nf}{2} \sum_{\mu \in G/H} \chi(\mu)\]
Note that $\chi$ is a nontrivial character on $G/H$. Thus
\[\sum_{\mu \in G/H} \chi(\mu) = 0\]
and so we have a contradiction.
\end{proof}

\begin{lemma} \label{lem:char_prim_conditions} Let $p,a_1,a_2, \ldots,a_r$ be distinct primes. Suppose $f = \ord_{abc}(p)$ and $f_i = \ord_{a_i}(p)$. There exists a primitive character modulo $a_1a_2\cdots a_r$ such that $\chi(-1) = -1$ and $\chi(p) = 1$ if and only if there exist integers $0 < \alpha_i < a_i - 1$ for each $i$ such that
\[ \sum_{i=1}^{r} \frac{\alpha_r}{f_r}\in \Z\]
and $\alpha_1 + \alpha_2 +\cdots + \alpha_r$ is odd.
\end{lemma}
\begin{proof} Let $A = a_1a_2\cdots a_r$ and $\chi: (\Z/A\Z)^{\times} \to S^1$ be a character. As:
\[ (\Z/A\Z)^{\times}  =  \prod_{i=1}^r (\Z/a_i\Z)^{\times} \]
There exists characters $\chi_i: (\Z/a_i\Z)^{\times} \to S^1$ such that
\[\chi(j) = \chi_1(j)\chi_2(j)\cdots\chi_r(j)\]
As the $a_i$ are prime, there exists generators $g_i$ modulo $a_i$ for each $i$ such that:
\begin{equation*}
g_i^{\frac{a_i-1}{f_i}} \equiv p \pmod{a_i}
\end{equation*}
Now there exists $\alpha_i$ for each $i$ such that:
\begin{equation*}
\chi(g_i) = \exp\left(\frac{2\pi \alpha_i}{a_i-1}\right) 
\end{equation*}
Using these above definitions, the condition $\chi(p) = 1$ is equivalent to
\[ \sum_{i=1}^{r} \frac{\alpha_r}{f_r}\in \Z\]
and the condition $\chi(-1) = -1$ translates to $\alpha_1 + \alpha_2 +\cdots + \alpha_r$ is odd. Lastly, the condition that $\chi$ is primitive just implies that $\chi_1, \chi_2, \chi_3$ are not trivial. Thus we lastly need $\alpha_1 \neq a - 1, \alpha_2 \neq b - 1, \alpha_3 \neq c - 1$, as desired.
\end{proof}

\begin{lemma} \label{lem:char_prim_powers_of_two} Let $a,b,c,p$ be distinct primes. Suppose $f = \ord_{abc}(p), f_1 = \ord_a(p), f_2 = \ord_b(p),$ and $f_3 = \ord_c(p)$ and let $2^r, 2^s, 2^t$ be the highest power of 2 dividing $f_1, f_2, f_3$ respectively. Then there exists a character $\chi$ primitive modulo $abc$ such that $\chi(-1)=-1$ and $\chi(p)=1$ only if one of the following holds
\begin{itemize}
\item $p^{f/2} \equiv -1 \pmod{abc}$
\item $f_2 = b - 1, f_3 = c - 1, r > s, s = 1, t = 1$
\item $f_1 = a - 1, f_2 = b - 1, f_3 = c - 1, r > s, s = 2, t = 1$
\end{itemize}
\end{lemma}
\begin{proof} 
We will do this by casework, using the result of lemma \ref{lem:char_prim_conditions}. To make things easier for ourselves suppose $f_1', f_2', f_3'$ are the largest odd numbers dividing $f_1, f_2, f_3$ respectively. Let $\alpha_1, \alpha_2, \alpha_3$ be as in the statement of lemma \ref{lem:char_prim_conditions}:
\\
\par
\textit{Case ($r=s=t$): } This is simply equivalent to $w^{f/2} \equiv -1 \pmod{p}$.
\\
\par
\textit{Case ($r>s>t$): } If $t \neq 1$ taking $\alpha_1 = f_1'2^{r-s}, \alpha_2 = f_2'(2^{s-t}-1), \alpha_3 = f_3'2^{t-1}$ gives us a primitive character satisfying the desired conditions. If $t = 1$ and $s \neq 2$, taking $\alpha_1 = f_1'2^{r - t - 1}, \alpha_2 = f_2'2^{s-t-1}, \alpha_3 = f_3'(2^{t} - 1)$ gives us a primitive character satisfying the desired conditions. As there exists no such characters, these cases are impossible. Hence $r >s=2>t=1$. 
\par
Now suppose we have $r >s=2>t=1$. Consider the case $\alpha_1 = f_1'2^{r-s}, \alpha_2 = 3f_2', \alpha_3 = 2f_3'$. This implies that $f_3 = 2f_3' =c - 1$, as otherwise this gives a character and hence a contradiction. Similarly, consider the case $\alpha_1 = f_1'2^{r-s+1}, \alpha_2 = 4f_2', \alpha_3 = f_3'$. By the same reasoning, this implies that $f_2 = 4f_2' = qb- 1$. Lastly, consider the case $\alpha_1 = f_1'2^{r}, \alpha_2 = 2f_2', \alpha_3 = f_3'$. Again, this implies that $f_1 = 2^rf_2' = a - 1$. This completes our analysis of this case.
\\
\par
\textit{Case ($r=s>t$): } Taking $\alpha_1 = f_1', \alpha_2 = f_2'(2^{s-t}-1), \alpha_3 = f_3'(2^{t}-1)$ gives us a primitive character satisfying the desired conditions. Thus we get a contradiction, so this case is impossible.
\\
\par
\textit{Case ($r>s=t$): } If $t \neq 1$, taking $\alpha_1 =2^{r-s}f_1', \alpha_2 = f_2'(2^{s}-2), \alpha_3 = f_3'$ gives us a primitive character satisfying the desired conditions. Hence $t = 1$.
\par
Now suppose we have $r > s=t=1$. Consider the case $\alpha_1 = f_1'2^{r-1}, \alpha_2 = f_2', \alpha_3 = 2f_3'$. This implies that $f_3 = 2f_3' =c - 1$, as otherwise this gives a character and hence a contradiction. Similarly, consider the case $\alpha_1 = f_1'2^{r-1}, \alpha_2 = 2f_2', \alpha_3 = f_3'$. By the same reasoning, $f_2 = 2f_2' = b - 1$.
\\
\par
We have now exhausted all possible cases and have shown that the only possible choices are those in the theorem statement.
\end{proof}


\begin{lemma} (Coyne) \label{lem:bless_the_coyne} Let $R$ be a positive integer and let $a_1, a_2, \ldots, a_k$ be positive integers all dividing $R$. Then the number of solutions $(b_1, \ldots, b_k) \in \prod_{i=1}^k \Z/a_i\Z$ to
\[\sum_{i = 1}^k \frac{Rb_i}{a_i} \equiv 0 \pmod{R}\]
is equal to 
\[\displaystyle\frac{\gcd(a_1, a_2, \ldots, a_k)\prod_{i=1}^k a_i}{R}\]
\end{lemma}
\begin{proof} 
Consider the homomorphism:
\[\phi : \prod_{i=1}^k \Z/a_i\Z \to \Z/R\Z \]
given by 
\[\phi(b_1, \ldots, b_k) = \sum_{i = 1}^k \frac{Rb_i}{a_i} \pmod{R}\]
The size of the kernel of this map is precisely the quantity we are looking for. Now consider $\im \ \phi$. This will be the elements of $\Z/R\Z$ with nonzero image in $\Z/\gcd(a_1, a_2, \ldots, a_k)\Z$. Thus:
\[|\im \ \phi| = \frac{R}{\gcd(a_1, a_2, \ldots, a_k)}\]
Lastly, by the first isomorphism theorem,
\[|\ker\phi| = \frac{|\prod_{i=1}^k \Z/a_i\Z|}{|\im \phi|} = \frac{\gcd(a_1, a_2, \ldots, a_k)\prod_{i=1}^k a_i}{R}\]
\end{proof}

\begin{lemma} \label{lem:char_prim_classification}  Let $a,b,c,p$ be distinct primes. Suppose $f = \ord_{abc}(p), f_1 = \ord_a(p), f_2 = \ord_b(p),$ and $f_3 = \ord_c(p)$ and let $2^r, 2^s, 2^t$ be the highest power of 2 dividing $f_1, f_2, f_3$ respectively. Lastly, let $f_1', f_2', f_3'$ be the largest odd integers dividing $f_1, f_2, f_3$ respectively. If $r \ge s \ge t \ge 1$ and $p^{f/2} \not\equiv -1 \pmod{abc}$, there does not exist a character $\chi$ primitive modulo $a,b,c$ such that $\chi(-1) = -1$ and $\chi(p) = 1$ if and only if $f_1', f_2', f_3'$ are pairwise coprime and one the following two conditions holds:
\begin{enumerate}
\item $f_2 = b - 1, f_3 = c - 1, r > s, s = 1, t = 1$
\item $f_1 = a - 1, f_2 = b - 1, f_3 = c - 1, r > s, s = 2, t = 1$
\end{enumerate}
\end{lemma}
\begin{proof} By lemma \ref{lem:char_prim_powers_of_two}, all that is left to show is that if one of the two cases holds then $f_1', f_2', f_3'$ being pairwise coprime is a necessary and sufficient condition on the existence of a character. By lemma \ref{lem:char_prim_conditions}, such a character exists if and only if we can find $\alpha_1, \alpha_2, \alpha_3$ such that:
\[S := \frac{\alpha_1}{2^rf_1'} + \frac{\alpha_2}{2^sf_2'} + \frac{\alpha_3}{2^tf_3'} \in \Z\]
and $\alpha+\alpha_2+\alpha_3 \in \Z$. In the first of our two conditions, the only possible values of $\alpha_1, \alpha_2, \alpha_3$ modulo $2^r, 2^s, 2^t$ such that the sum of the $\alpha_i$ is odd and the denominator of $S$ is odd are $\alpha_1 \equiv 2^{r-1} \pmod{2^r}$ and exactly one of $\alpha_2, \alpha_3$ is odd. Thus, as the choice of $\alpha_1, \alpha_2, \alpha_3$ modulo $f_1', f_2', f_3'$ will determine if $S$ is an integer,  there does not exist such a primitive character if and only if the only choices of $\alpha_2, \alpha_3$ have $f_2' | \alpha_2$ and $f_3' | \alpha_3$.
\par
Similarly, in the second of our two conditions, the only possible values have one of $\alpha_1, \alpha_2, \alpha_3$ modulo $2^r, 2^s, 2^t$ that do  give rise to a character has one of the $\alpha$s 0 in the respective modulus. Furthermore, there exists at least one choice of modular remainders for which each of them is 0 and no others are. Thus there does not exist such a primitive character if and only the only choices of $\alpha_1, \alpha_2, \alpha_3$ are divisible by $f_1', f_2', f_3'$ respectively.
\par
In both cases, this comes down to determining whether there are solutions to:
\[T(\gamma_1, \gamma_2, \gamma_3) := \frac{\gamma_1}{f_1'} + \frac{\gamma_2}{f_2'} + \frac{\gamma_3}{f_3'} \in \Z\]
with $f_i \nmid \gamma_i$ as we can pick $\alpha_1, \alpha_2, \alpha_3$ modulo $f_1', f_2', f_3'$ respectively such that $\gamma_1 = 2^{i}\alpha_1, \gamma_2 = 2^j\alpha_2, \gamma_3 = 2^k\alpha_3$ for any $i,j,k$. 
\par
Let $R = \lcm(f_1'f_2'f_3')$ and $w_i$. Any choice of $\gamma_i$ with $T \in Z$ will have $f_2' | \alpha_2, f_3' | \alpha_3$ if and only if  $f_1' | \alpha_1$. Thus $T \in Z$ if and only if the number of solutions to:
\[\frac{R\gamma_1}{f_1'} + \frac{R\gamma_1}{f_1'} + \frac{R\gamma_1}{f_1'}  \equiv 0 \pmod{R}\]
is 1. By lemma \ref{lem:bless_the_coyne}, this occurs if and only if:
\[f_1f_2f_3\gcd(f_1,f_2,f_3) = \lcm(f_1,f_2,f_3)\]
Which occurs if and only if $f_1, f_2, f_3$ are pairwise coprime, as desired.
\end{proof}

\begin{theorem} \label{thm:sum_mu_abc_supersing_cond}
Let $a,b,c,p$ be distinct primes. Suppose that the order of $p$ modulo each of $a,b,c$ is even. Then
he projective variety $V$ defined by
\[w^{abc} + x^a + y^b + z^c =0\]
over $\F_p$ is supersingular if and only if for all $\mu$ relatively prime to $abc$,
\[\frp{\mu p^i}{abc} = \frac{f}{2}\]
\end{theorem}
\begin{proof}
By (Insert Citation), $V$ is supersingular if and only if for all $a\nmid \beta_1, b\nmid \beta_2, c\nmid \beta_3, abc \nmid \beta_4$ such that
\[\frac{\beta_1}{a} + \frac{\beta_2}{b} + \frac{\beta_3}{c} + \frac{\beta_4}{abc} \in Z\] 
we have:
\[\sum_{i = 0}^f \left[\frp{ \mu\beta_1 p^i}{a} + \frp{\mu \beta_2 p^i }{b} + \frp{\mu \beta_3 p^i}{c} + \frp{\mu \beta_4 p^i}{abc}\right] = 2f \]
As $p$ has even order modulo each of $a,b,c$ there exists a power of it which is -1 modulo each of $a,b,c$. As such we can pair up to get
\[\sum_{i = 0}^f \frp{ \mu\beta_1 p^i}{a} = \sum_{i = 0}^f \frp{ \mu\beta_2 p^i}{b} = \sum_{i = 0}^f \frp{ \mu\beta_3 p^i}{c} = \frac{f}{2}\]
Hence the above condition is equivalent to:
\[\frp{\mu \beta_4 p^i}{abc} = \frac{f}{2}\]
As $\mu\beta_4$ ranges over the same set as just $\mu$, this is equivalent to for all $\mu$ relatively prime to $abc$:
\[\frp{\mu p^i}{abc} = \frac{f}{2}\]
as desired
\end{proof}

\begin{theorem} \label{thm:a_b_c_supersingular_cond1} Let $a,b,c,p$ be distinct primes. Suppose $f = \ord_{abc}(p), f_1 = \ord_a(p), f_2 = \ord_b(p),$ and $f_3 = \ord_c(p)$ and let $2^r, 2^s, 2^t$ be the highest power of 2 dividing $f_1, f_2, f_3$ respectively. Lastly, let $f_1', f_2', f_3'$ be the largest odd integers dividing $f_1, f_2, f_3$ respectively. If $r \ge s \ge t \ge 1$ and the projective variety $V$ defined by
\[w^{abc} + x^a + y^b + z^c =0\]
over $\F_p$ is supersingular and $p^{f/2} \not\equiv -1 \pmod{abc}$ then $f_1', f_2', f_3'$ are pairwise coprime and one the following two holds:
\begin{itemize}
\item $f_2 = b - 1, f_3 = c - 1, r > s, s = 1, t = 1$
\item $f_1 = a - 1, f_2 = b - 1, f_3 = c - 1, r > s, s = 2, t = 1$
\end{itemize}
\end{theorem}

\begin{proof}
By theorem  \ref{thm:sum_mu_abc_supersing_cond}, we have for all $\mu$ relatively prime to $abc$:
\[\frp{\mu p^i}{abc} = \frac{f}{2}\]
The result of lemma $\ref{lem:prim_char_supersing_contradiction}$ then implies that there does not exist a character $\chi$ primitive modulo $abc$ such that $\chi(p) = 1, \chi(-1) = -1$. From this, lemma \ref{lem:char_prim_classification} gives us the desired result.
\end{proof}

\begin{corollary} \label{thm:a_b_c_supersingular_cond2}
Let $a,b,c,p$ be distinct primes. Suppose $f = \ord_{abc}(p), f_1 = \ord_a(p), f_2 = \ord_b(p),$ and $f_3 = \ord_c(p)$ and let $2^r, 2^s, 2^t$ be the highest power of 2 dividing $f_1, f_2, f_3$ respectively with $r \ge s \ge t$. If the projective variety $V$ defined by
\[w^{abc} + x^a + y^b + z^c =0\]
over $\F_p$ is supersingular and $p^{f/2} \not\equiv -1 \pmod{abc}$ then $p$ is a primitive root modulo $b$ and $c$, $f = \frac{\phi(abc)}{4}$ or $f = \frac{\phi(abc)}{8}$, and $r > s$:
\end{corollary}
\begin{proof}
This is implied by theorem \ref{thm:a_b_c_supersingular_cond1}
\end{proof}









\begin{lemma}
\label{thm:partial_sum_abc_to_bc}
Suppose $a,b,c,p$ are primes with $f = \ord_{abc}(p)$ and $f_1 = \ord_{bc}(a)$. Let $H$ be the subgroup of $(\Z/a\Z)^{\times}$ generated by $p^{f_1}$. Then for all $\mu$ not divisible by $a,b,c$ we have:
\[\sum_{h \in (\Z/a\Z)^{\times}/H} \sum_{i = 0}^{f-1} \frp{\mu hp^i}{abc} = \frac{f_1(a-1)}{2}\]
if and only if for all $\mu$ not divisible by $b,c$ we have:
\[\sum_{i = 0}^{f_1-1} \frp{\mu p^i}{bc} = \sum_{i = 0}^{f_1-1} \frp{\mu up^i}{bc}\]
where $u \equiv a^{-1} \pmod{bc}$.
\end{lemma}
\begin{proof}
Note that we have:
\begin{equation} \label{eq:sum_abc_to_bc} \sum_{h \in H} \sum_{i = 0}^{f-1} \frp{\mu hp^i}{abc} = \sum_{k \in (\Z/a\Z)^{\times}} \sum_{i = 0}^{f_1 - 1} \frp{\mu kp^i}{abc} =  \sum_{k \in (\Z/a\Z)} \sum_{i = 0}^{f_1 - 1} \frp{\mu kp^i}{abc} - \sum_{i = 0}^{f_1 - 1} \frp{\mu up^i}{bc} \end{equation}
where we view $k \in (\Z/a\Z)^{\times}$ as the element $x$ for which:
\begin{align*} x &\equiv k \pmod{a} \\
 x &\equiv 1 \pmod{b} \\
 x &\equiv 1 \pmod{c}
\end{align*}
Now as $f_1 = \ord_{p}(bc)$ for each pair of remainders $f \pmod{b}, g\pmod{c}$ there exists at most one remainder modulo $e \pmod{a}$ such that there exists an $i$ for which $p^i$ is equivalent to each of those in the respective modulus. As such we have:
\[\sum_{k \in (\Z/a\Z)^{\times}} \sum_{i = 0}^{f_1 - 1} \frp{\mu kp^i}{abc} = \sum_{j = 0}^{a-1} \sum_{i = 0}^{f_1 - 1} \frp{\mu p^i + jbc}{abc}\]
Now for each $i$ let $j_i$ be the $j$ for which
\[\frp{\mu p^i + jbc}{abc} < \frac{1}{a}\]
We then get:
\begin{align*}
\sum_{k \in (\Z/a\Z)^{\times}} \sum_{i = 0}^{f_1 - 1} \frp{\mu kp^i}{abc} &= \sum_{j = 0}^{a-1} \sum_{i = 0}^{f_1 - 1} \frp{\mu p^i + j_0bc + jbc}{abc} \\
&= \sum_{j = 0}^{a-1} \left[\sum_{i = 0}^{f_1 - 1} \frp{\mu p^i + j_0bc}{abc} + \frac{j}{a}\right] \\
&= \frac{(a-1)f_1}{2} +  \sum_{i = 0}^{f_1 - 1} a\frp{\mu p^i + j_0bc}{abc} 
\end{align*}
Now as $\frp{\mu p^i + j_0bc}{abc} < \frac{1}{a}$ we have
\[a\frp{\mu p^i + j_0bc}{abc}  = \frp{\mu a p^i +j_0abc}{abc} =  \frp{\mu p^i}{bc}\]
Thus we get:
\[\sum_{k \in (\Z/a\Z)^{\times}} \sum_{i = 0}^{f_1 - 1} \frp{\mu kp^i}{abc} = \frac{(a-1)f_1}{2} +  \sum_{i = 0}^{f_1 - 1}  \frp{\mu p^i}{bc} \]
Plugging this back into equation \label{eq:sum_abc_to_bc} gives:
\[\sum_{h \in H} \sum_{i = 0}^{f-1} \frp{\mu hp^i}{abc} =  \frac{(a-1)f_1}{2} +  \sum_{i = 0}^{f_1 - 1}  \frp{\mu p^i}{bc} - \sum_{i = 0}^{f_1 - 1} \frp{\mu u p^i}{bc}\]
Rearranging we get:
\[ \sum_{i = 0}^{f_1 - 1} \frp{\mu u p^i}{bc}=   \sum_{i = 0}^{f_1 - 1}  \frp{\mu p^i}{bc} +  \frac{(a-1)f_1}{2} - \sum_{h \in H} \sum_{i = 0}^{f-1} \frp{\mu hp^i}{abc}\]
which implies the desired result.
\end{proof}

\begin{corollary} Suppose $a,b,c,p$ are primes with the order of $p$ modulo each of a,b,c even. If the projective variety $V$ defined by
\[w^{abc} + x^a + y^b + z^c =0\]
over $\F_p$ is supersingular and $b\equiv c \equiv 3 \pmod{4}$ then there exists $i,j$ such that $p^i \equiv b \pmod{ac}$ and $p^i \equiv \pmod{ab}$
\end{corollary}
\begin{proof}
Define $f_1, f_2, f_3, f_1', f_2', f_3', r, s, t$ as in theorem \ref{thm:a_b_c_supersingular_cond1}. As$b\equiv c \equiv 3 \pmod{4}$, we must be in the case $s=t=1$. By the results of theorem \ref{thm:a_b_c_supersingular_cond1}, $p$ generates s subgroup of order $\frac{\phi(ac)}{2}$. Thus if there does not exist an $i$ for which $p^i \equiv b \pmod{ac}$, $b,p$ must generate $(\Z/ac\Z)^{\times}$. By theorem \ref{thm:sum_mu_abc_supersing_cond} and lemma \ref{thm:partial_sum_abc_to_bc}, we must have for each $\mu$ relatively prime to $ac$
\[\sum_{i = 0)}^{\frac{\phi(ac)}{2} - 1}\frp{\mu p^i}{ac} = \sum_{i = 0)}^{\frac{\phi(ac)}{2} - 1}\frp{\mu bp^i}{ac}\]
However, as $b,p$ generate $(\Z/ac\Z)^{\times}$, this implies for each $\mu$
\[\sum_{i = 0)}^{\frac{\phi(ac)}{2} - 1}\frp{\mu p^i}{ac}\]
is constant and thus equal to $\frac{\phi(ac)}{2}$ as summing the sums for $\mu = 1, \mu = -1$ gives $\phi(ac)$ by cancellation. However, by lemma \ref{lem:prim_char_supersing_contradiction}, this implies there cannot exist a character modulo $ac$ with $\chi(-1) = -1, \chi(p) = 1$. However, if we take $\alpha_1 = 2^{r-1}f_1', \alpha_3 = f_3'$ then:
\[\frac{\alpha_1}{f_1} + \frac{\alpha_3}{f_3} \in \Z\]
and $\alpha_1 + \alpha_3$ is odd. Thus by lemma \ref{lem:char_prim_conditions}, there should exist such a character satisfying those conditions, which gives us a contradiction. Thus $b$ is in the group generated by $p$ modulo $ac$. By the same reasoning, $c$ is in the group generated by $p$ modulo $ab$, as desired.
\end{proof}


\begin{theorem}  Suppose $a,b,c,p$ are primes with $f = \ord_{abc}(p)$. Let $f_1 = \ord_a(p), f_2 = \ord_b(p), f_3 = \ord_c(p)$. Let $2^r, 2^s, 2^t$ be the highest power of 2 dividing $f_1, f_2, f_3$ respectively. If $r > s \ge t \ge 1$, $f = \frac{\phi(abc)}{4}$, and there exists $i,j$ such that $p^i \equiv b \pmod{ac}$ and $p^i \equiv \pmod{ab}$ then the projective variety $V$ defined by
\[w^{abc} + x^a + y^b + z^c =0\]
over $\F_p$ is supersingular. 
\end{theorem}


\begin{proof}
Note that as $r > s \ge t \ge 1$ and $f = \frac{\phi(abc)}{4}$ we must have $s = t = 1$. Let $u$ be defined to be the integer satisfying the following equivalences:
\begin{align*}
u &\equiv 1 \pmod{a} \\
u &\equiv -1 \pmod{b} \\
u &\equiv 1 \pmod{c}
\end{align*}
Similarly let $v$ be an integer such that 
\begin{align*}
v &\equiv -1 \pmod{a} \\
v &\equiv 1 \pmod{b} \\
v &\equiv -1 \pmod{c}
\end{align*}
Let $H$ be the subgroup of $(\Z/abc\Z)^{\times}$ generated by $p$. We claim $H, -H, uH, vH$ are the distinct cosets of $H$. Note that as $r > s=t > 0$ $-1, u, v$ cannot be powers of $p$. Thus $uH, vH, -H$ are distinct from $H$. Now note that $uv = -1$ and $u^2 = v^2 = 1$. Thus $(uH)^2 = H, (vH)^2 = H, (uH)(vH) = -H$. Thus implies $H, -H, uH, vH$ are the distinct cosets of $H$ and $(\Z/abc\Z)^{\times}/H$ is the Klein-Four group. Now define:
\[g(\mu) := \sum_{i = 1}^f \frp{\mu p^i}{abc}\]
By theorem \ref{thm:sum_mu_abc_supersing_cond}, $V$ is supersingular if and only if:
\[g(1) = g(-1) = g(u) = g(v) = \frac{f}{2}\]
We will now show that those equivalences holds. Due to pairing up:
\[g(1) + g(-1) = f\]
and
\[g(v) + g(u) = f\]
Now as $b$ lies in the subgroup generated by $p$ modulo $ac$, we have for all $\mu$:
\[\sum_{i = 0}^{f_2 - 1} \frp{\mu p^i}{ac} = \sum_{i = 0}^{f_2 - 1} \frp{\mu bp^i}{ac}\]
Thus by lemma \ref{thm:partial_sum_abc_to_bc}, for all $\mu$ relatively prime to $abc$,
\[ \sum_{g \in (\Z/b\Z)^{\times}/G} \sum_{i=1}^{f-1} \frp{\mu g p^i}{abc}= \frac{f_2(b-1)}{2}\]
where $G$ is the subgroup of $(\Z/b\Z)^{\times}$ generated by $p^{f_2}$. Note that the conditions of the problem imply $f_1 = a - 1, f_2 = b - 1, f_3 = c - 1$ and the odd parts of $f_1, f_2, f_3$ are coprime. As $r > s=1$, $G$ will be the set of squares modulo $b$. As $s = 1$, $b \equiv 3 \pmod{4}$ and so $-1$ is not a square modulo $b$. As such, $1, u$ are the coset representatives of $(\Z/b\Z)^{\times}/G$. Taking $\mu = 1$ gives:
\[g(1) + g(u) = f\]
and taking $\mu = v$ gives:
\[g(-1) + g(v) = f\]
Applying the same reasoning to the subgroup generated by $p$ modulo $ab$:
\[g(1) + g(v) = f\]
and
\[g(-1) + g(u) = f\]
Combining all of our equations gives:
\[g(1) = g(-1) = g(u) = g(v) = \frac{f}{2}\]
which as stated before implies $V$ is supersingular.
\end{proof}

\begin{conjecture} Let $a,b,c,p$ be distinct primes. Let $f = \ord_{abc}(p), f_1 = \ord_{a}(p), f_2 = \ord_{b}(p), f_3 = \ord_{c}(p)$ and let $2^r, 2^s, 2^t$ be the largest powers of 2 dividing $f_1, f_2, f_3$ respectively. If $r \ge s \ge t$, the variety $V$ defined by the equation:
\[x^a + y^b + z^c + w^{abc}\]
is supersingular if and only if $p^{f/2} \equiv -1 \pmod{abc}$ or if conditions 1,2 hold and either of 3,4 hold:
\begin{enumerate}
\item $r > s$ and $\frac{f_1}{2^r}, \frac{f_2}{2^s}, \frac{f_3}{2^t}$ are pairwise coprime. 
\item $f_2 = b - 1, f_3 = c - 1$ and there exists an integer $j$ such that $p^j \equiv c \pmod{ab}$
\item $s=t=1$ and there exists an integer $i$ such that $p^i \equiv b \pmod{ac}$ 
\item $s=2, t= 1$, $f_1 = a - 1$, and there exists an integer $i$ such that $p^i \equiv a \pmod{bc}$ or there exists an integer $j$ such that $p^j \equiv b \pmod{ac}$ 
\end{enumerate}
\end{conjecture}

\section{Varieties of the Form $w^{a} + x^{a} + y^{ab} + z^{ab}$}

Let $X$ be the diagonal surface defined by $w^{a} + x^{a} + y^{ab} + z^{ab}$ over $\finfield{p}$.

\begin{lemma} \label{lem:surjective_quotient_maps}
Let $H_1, H_2 \triangleleft G$ be normal subgroups with quotient maps $\pi_i : G \to G / H_i$ and consider the maps,
\[ \varphi_{i,j} : H_i \hookrightarrow G \overset{\pi_j}{\twoheadrightarrow} G/H_j \]
Then $\varphi_{1,2}$ is surjective iff $\varphi_{2,1}$ is surjective. 
\end{lemma}

\begin{proof}
Consider the commutative diagram with exact rows and columns,
\begin{center}
\begin{tikzcd}[column sep = large, row sep = huge]
& 0 \arrow[d] & 0 \arrow[d] & 0 \arrow[d] 
\\
0 \arrow[r] & H_1 \cap H_2 \arrow[d, hook] \arrow[r, hook] & H_1  \arrow[d, hook] \arrow[r, two heads] \arrow[dr, "\varphi_{1,2}"] & K_1 \arrow[d, hook, "\bphi_{1,2}"] \arrow[r] & 0  
\\
0 \arrow[r] & H_2 \arrow[rd, "\varphi_{2,1}"] \arrow[r, hook] \arrow[d, two heads] & G \arrow[d, "\pi_1", two heads] \arrow[r, "\pi_2", two heads] & G / H_2 \arrow[d, two heads] \arrow[r] & 0  
\\
0 \arrow[r] & K_2 \arrow[d] \arrow[r, hook, "\bphi_{2,1}"] & G/H_1 \arrow[d] \arrow[r, two heads] & C \arrow[r] \arrow[d] & 0
\\
& 0 & 0 & 0 
\end{tikzcd}
\end{center}
where $K_i = H_i / (H_1 \cap H_2)$ and the maps $\bphi_{i,j} : K_i \to G/H_j$ are induced by the maps $\varphi_{i,j}$ and are injective by the first isomorphism theorem. Exactness and commutativity are obvious except at $C$ which I have yet to define! By commutativity and surjectivity, $\im{\bphi_{i,j}} = \pi_j(H) \triangleleft \im{\pi_j} = G/H_j$ so $\Im{\bphi_{i,j}}$ is a normal subgroup and thus $\coker{\bphi_{i,j}} = (G/H_j) / \im{\bphi_{i,j}} $ exists. Take $C = \coker{\bphi_{1,2}}$. Furthermore, the exactness of columns gives a surjective map $G/H_1 \to C$ which makes the bottom right square commute. By the nine lemma, the bottom row is exact proving that $C = \coker{\bphi_{2,1}}$. Finally, by exactness, 
\[ \bphi_{1,2} \text{ is an isomorphism } \iff C = 0 \iff \bphi_{2,1} \text{ is an isomorphism}  \]
But $\varphi_{i,j}$ is a surjection iff $\bphi_{i,j}$ is an isomorphism so $\varphi_{1,2}$ is surjective iff $\varphi_{2,1}$ is surjective. 
\end{proof}


\begin{lemma} \label{lem:surjective_map_coset_reps}
Let $p : G \to G'$ be surjective and $H \triangleleft G$ a normal subgroup. Then there exist coset representatives for $G/H$ with fixed image in $G'$ if and only if $p(H) = G'$. Furthermore, we if this holds, we may take the coset representatives to be trivial in $G'$. 
\end{lemma}

\begin{proof}
A set $S \subset G$ contains a full set of coset represenatives for $G/H$ if $\pi(S) = G/H$. Therefore, we require that $\pi(p^{-1}(x)) = G/H$ for some $x \in G'$. Since we must hit the identity, $H \cap p^{-1}(x) \neq \varnothing$ so there exits $h \in H$ such that $p(h) = x$. Thus, $p^{-1}(x) = h \ker{p}$ so $\pi(p^{-1}(h)) = \pi(h) \pi(\ker{p}) = \pi(\ker{p})$ so we may take $h = e$. The conclusion holds if and only if $\pi(\ker{p}) = G/H$. 
\bigskip\\    
Take $H_1 = H$ and $H_2 = \ker{p}$ in Lemma \ref{lem:surjective_quotient_maps} and thus, 
\[\im{\varphi_{2,1}} = \pi(\ker{p}) = G/H \iff \im{\varphi_{1,2}} = \pi_2(H) = G/\ker{p} \]
but the map $p$ naturally factors through $G / \ker{p}$ as,
\begin{center}
\begin{tikzcd}
H \arrow[r, hook] & G \arrow[rr, two heads, "p"] \arrow[rd, hook, "\pi_2"] & & G'
\\
& & G /\ker{p} \arrow[ru, "\sim"]
\end{tikzcd}
\end{center}
so $p(H) = G' \iff \pi_2(H) = G / \ker{p}$. 
\end{proof}

\begin{theorem}
Suppose there exists a subgroup $H \subset (\Z / ab \Z)^\times$ such that $p \in H$ and $-1 \notin H$
\[ H \hookrightarrow (\Z / ab \Z)^\times \to (\Z / a \Z)^\times\]
is surjective. Then $X$ is not supersingular. 
\end{theorem}

\begin{proof}
By Theorem \ref{thm:gauss_factor_SS},  if $X$ is supersingular then,
\[ \sum_{i = 0}^3 \sum_{j = 0}^{f-1} \left\{ \frac{\mu e_i p^j}{ab} \right\} = 2f \]
However, there is a projection map $X \to F^3_{a}$ so $F^3_a$ is supersingular and thus, by Shioda, $p^v \equiv -1 \mod a$. However, we know that,
\[ \frac{e_0'}{a} + \frac{e_1'}{a} + \frac{e_2'}{ab} + \frac{e_2'}{ab} = \frac{b(e_0' + e_1') + e_2' + e_3'}{ab} \in \Z\]
and thus $b \divides e_2' + e_3'$. Thus we have,
\[  \sum_{j = 0}^{f-1} \left\{ \frac{\mu e_0' p^j}{a} \right\} + \sum_{j = 0}^{f-1} \left\{ \frac{\mu e_1' p^j}{a} \right\} +  \sum_{j = 0}^{f-1} \left\{ \frac{\mu e_2' p^j}{ab} \right\} + \sum_{j = 0}^{f-1} \left\{ \frac{\mu e_3' p^j}{ab} \right\} = 2f 
\]
however because $p^v \equiv -1 \mod a$,
\[ \sum_{j = 0}^{f-1} \left\{ \frac{\mu e_0' p^j}{a} \right\} + \sum_{j = 0}^{f-1} \left\{ \frac{\mu e_1' p^j}{a} \right\} = f \]
so we know that,
\[ \sum_{j = 0}^{f-1} \left\{ \frac{\mu e_2' p^j}{ab} \right\} + \sum_{j = 0}^{f-1} \left\{ \frac{\mu e_3' p^j}{ab} \right\} = f \]
Define the sum,
\[ S(x) = \sum_{j = 0}^{f-1} \left\{ \frac{x p^j}{ab} \right\} \]
The above gives the functional equation,
\[ S(x) + S(y) = f \]
whenever $x + y \equiv 0 \mod{b}$. In particular, if $x \equiv y \mod b$ then $S(x) = S(y)$. \bigskip\\
Let $\chi : (\Z / ab \Z)^\times \to \C^\times$ be a Dirichlet character such that $\chi(H) = 1$ and $\chi(-1) = -1$. This is possible assuming that $-1 \notin H$. Let $m_0$ be the conductor of $\chi$ with a map $\varphi : (\Z / ab \Z)^\times \to (\Z /m_0 \Z)^\times$ and $H_0 = \varphi(H)$ and character $\chi_0 : (\Z / m_0\Z)^\times \to \C^\times$ inducing $\chi$. Now define the sum,
\[ S_0(x) = \sum_{t \in \varphi(\left<p\right>)} \left\{ \frac{xt}{m_0} \right\} = \frac{1}{| \left< p \right> \cap \ker{\varphi} |} \sum_{t \in \left< p \right>} \left\{ \frac{(ab/m_0) x t}{ab} \right\} = \frac{1}{|\left< p \right> \cap \ker{\varphi}|} S\left( \frac{ab}{m_0} x \right) \]  
Thus, $S_0(x) = S_0(y)$ whever $m_0 \divides a(x - y) \iff x \equiv y \mod \bar{m_0} = m_0/(m_0,a)$. Next, let $G = (\Z / m_0 \Z)^\times$ and $K = \varphi(\left< p \right>)$ and consider,
\begin{align*} 
\sum_{x \in G} \chi_0(x) \frac{x}{m_0} & = \sum_{gH_0 \in G/H_0} \sum_{h \in H_0 / K} \sum_{x \in hg K} \chi_0(x) \frac{x}{m_0} = \sum_{gH_0 \in G/H_0} \chi_0(g)  \sum_{h \in H_0 / K} \sum_{x \in gh K} \frac{x}{m_0} 
\\
& = \sum_{gH_0 \in G/H_0} \chi_0(g)  \sum_{h \in H_0 / K} S_0(gh)
\end{align*}
since $\chi_0$ is trivial on $H_0$ and thus descends to a nontrivial character on $G/H_0$. By Lemma \ref{lem:surjective_map_coset_reps}, the surjective map,
\[ H \hookrightarrow (\Z / ab \Z)^\times \to (\Z / a \Z)^\times \]
alows us to choose coset representatives of $G / H_0$ which are all trivial under the map $(\Z / m_0 \Z)^\times \to (\Z / \bar{m_0} \Z)^\times$. Therefore, $gh \equiv h \mod \bar{m_0}$ and thus,
\[ \sum_{x \in G} \chi_0(x) \frac{x}{m_0} = \sum_{gH_0 \in G/H_0} \chi_0(g)  \sum_{h \in H_0 / K} S_0(h) = \left( \sum_{h \in H_0 / K} S_0(h) \right) \cdot \left( \sum_{gH_0 \in G/H_0} \chi_0(g) \right) = 0 \]
since $\chi_0$ is a nontrivial character on $G / H_0$. This is a contradiction because,
\[ \sum_{gH_0 \in G/H_0} \chi_0(g) \sim L(1 ; \chi_0) \neq 0 \]
\end{proof}

\section{Varieties of the Form $w^{a} + x^{ar} + x^{br} + x^{ab}$}


\begin{thebibliography}{99}
\bibitem{Chowla}
S. Chowla, On Gaussian Sums, \textit{Proceedings of the National Academy of Sciences}, 48 (7), 1127-8, 1962.

\bibitem{Evans}
R. Evans, Generalizations of a Theorem of Chowla on Gaussian Sums, \textit{Houston Journal of Mathematics}, 3, 1977.

\bibitem{Koblitz}
N. Koblitz, p-adic variation of the zeta-function over families of varieties defined over finite fields, \textit{Compositio Mathematica}, 31, 119-218, 1975.

\bibitem{Lang}
S. Lang, \textit{Algebraic Number Theory}, Springer, 1994.

\bibitem{Shioda-Unirational}
T. Shioda, An example of Unirational Surfaces in Characteristic p, \textit{Mathematische Annalen}, 221, 233-236, 1974.

\bibitem{Shioda-Katsura}
T. Shioda, T. Katsura, On Fermat Varieties, \textit{Tohoku Math Journal}, 31, 97-115, 1979.

\bibitem{Weil}
A. Weil, Numbers of Solutions of Equations in Finite Fields, \textit{Bulletin of the American Mathematical Society}, 55 (5), 497-508, 1949



\end{thebibliography}

\end{document}

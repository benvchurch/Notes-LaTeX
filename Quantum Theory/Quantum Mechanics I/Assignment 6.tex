\documentclass[12pt]{extarticle}
\include{QuantumCommands}


\begin{document}
\atitle{6}
 
\section*{Problem 19.}
Let a particle of mass $m$ and energy $E$ move under the influence of a one dimensional potential with $V(0) = E$ so the problem transition to the classically forbidden region (region II) for $x > 0$. Consider, 
\[\psi_I(x) = \frac{C}{(E - V(x))^{1/4}} \cos{\left(\frac{1}{\hbar} \int_x^0 \sqrt{2m(E - V(x))} \: \d{x} + \frac{\pi}{4} \right)}\] 
In the region close to $0$, $V(x) \approx E + x V'(x)$ so
\[\psi_I(x) = \frac{C}{(-x V'(0))^{1/4}} \cos{\left(i \frac{\sqrt{2m V'(0)}}{\hbar} \frac{2}{3} x^{3/2} + \frac{\pi}{4} \right)}\] 
Let $z = \left(\frac{\sqrt{2m V'(0)}}{\hbar} \frac{2}{3} \right)^{2/3} \cdot x = r e^{i \phi}$,
\[\psi_I(z) = \frac{\tilde{C}}{(-z)^{1/4}} \frac{1}{2} \left[ e^{-z^{3/2} + i \frac{\pi}{4}} + e^{z^{3/2} - i \frac{\pi}{4}} \right]   \] 
for $\phi = \pi$ (consistently using positive phase) this formula gives: \[\frac{1}{(-r e^{i \pi})^{1/4}} \frac{1}{2} \left[ e^{-r^{3/2} e^{i 3\pi/2} + i \frac{\pi}{4}} + e^{r^{3/2} e^{i 3\pi/2} - i \frac{\pi}{4}} \right] = r^{-1/4} \frac{1}{2} \left[ e^{i r^{3/2} + i \frac{\pi}{4}} + e^{ - ir^{3/2} - i \frac{\pi}{4}} \right] = r^{-1/4} \cos{\left(r^{3/2} + \frac{\pi}{4} \right)}\]
Now, we analytically extend this solution into region II. However, the inital point $x < 0$ corresponds to $z$ having a phase of $\pi$ which is on the branch cut of $z^{1/2}$ (where $z = re^{i\phi} \mapsto z^{1/2} = r^{1/2} e^{i \phi/2}$ for $\phi \in (-\pi, \pi)$) and the two terms correspond to $f(z) = z^{-1/4} e^{z^{3/2}}$ whith the phase of $z$ constained to $\phi \in (-\pi, \pi)$ just above and just below this cut. Where just above the cut, \[f_{+}(z) = z^{-1/4} e^{z^{3/2}} = r^{-1/4} e^{-i \phi/4} \exp{\left(r^{3/2} e^{ 3/2 \: i \phi} \right)} = r^{-1/4} e^{-i \pi/4} \exp{\left( -i r^{3/2} \right)} \]
And just below, \[ f_{-}(z) = z^{-1/4} e^{z^{3/2}} = r^{-1/4} e^{-i (-\phi)/4} \exp{\left(r^{3/2} e^{ 3/2 \: i (-\phi)} \right)} = r^{-1/4} e^{i \pi/4} \exp{\left(i r^{3/2} \right)} \]
Thus, $\psi_I(z) = \frac{\tilde{C}}{2} (f_-(z) + f_+(z))$ where $f$ is holomorphic everywhere except the negative real axis. These terms agree for real positive $z$ where both equal $r^{-1/4} e^{r^{3/2}}$. We continue these terms to the positive real axis about mirror image contours so as to not cross the branch cut. From the WKB approximation, we know the form of the wavefunction in region II is,
\[\psi_{II}(z) = \frac{\tilde{C}_{+}}{z^{1/4}} e^{z^{3/2}} + \frac{\tilde{C}_{-}}{z^{1/4}} e^{- z^{3/2}}\]
on both contours, the $f_+$ and $f_{-}$ functions match the positive exponential which is exponentially growing in the region to the right of the $\pm 60^\circ$ anti-Stokes lines but we get no information about the exponentially supressed negative exponental because the WKB approximation is not accurate to subdominant terms. Thus, for positive real $z \rightarrow r$
\[\psi_{II}(r) \approx \frac{\tilde{C}_{+}}{r^{1/4}} e^{r^{3/2}} \approx \frac{\tilde{C}}{2} \left[f_{-}(z) + f_{+}(z) \right] \rightarrow \frac{\tilde{C}}{r^{1/4}} e^{r^{3/2}}\]
Therefore, $\tilde{C}_+ = \tilde{C}$ and $\tilde{C}_{-} \approx 0$ although we cannot make the second conclusion accurately. Now, we match this solution to the general WKB wavefunction in region II,
\[\psi_{II}(x) = \frac{C}{(V(x) - E)^{1/4}} \exp{\left(\frac{1}{\hbar} \int_0^{x} \sqrt{2m(V(x) - E)} \: \d{x} \right)}\] 
If we apply the same analysis to the region I wavefunction,
\[\psi_I(x) = \frac{C}{(E - V(x))^{1/4}} \sin{\left(\frac{1}{\hbar} \int_x^0 \sqrt{2m(E - V(x))} \: \d{x} + \frac{\pi}{4} \right)}\] 
we would get $\psi_I(z) = \frac{\tilde{C}}{2} (f_-(z) - f_+(z))$ and thus conclude that $\psi_{II}(z) \rightarrow \frac{\tilde{C}}{2} (f_-(r) - f_+(r)) = 0$ but then the undetermined falling exponential dominates so we have, in reality, not determined the solution in region II.  
\section{Problem 20.}
\subsection*{(a)}
A particle of mass $m$ and energy $E$ is incident from the left on a potential barrier $V(x)$. First, define three regions, $x < x_1$ is region I, $x_1 \le x \le x_2$ is region II, and $x > x_2$ is region III. Regions I and III are classically allowed where as region II is classically forbidden. Because we do not have a wave incident on the right, the wavefunction far in region III must have the form of a left-propagativg wave, $t \: e^{ikx + i\phi}$ where $k= \frac{\sqrt{2mE}}{\hbar}$. Choose the phase of the wave to match the WKB solution in region III and let $|p(x)| = |\sqrt{2m(E - V(x))}|$,
\begin{align*}
\psi_{III}(x) &= \frac{t}{\sqrt{|p(x)|}} \exp{\left(\frac{i}{\hbar} \int_{x_2}^x |p(x)| \d{x} + \frac{\pi}{4} \right)} \\ &= \frac{t}{\sqrt{|p(x)|}} \left[ \cos{\left(\frac{1}{\hbar} \int_{x_2}^x |p(x)| \d{x} + \frac{\pi}{4} \right)} + i \sin{\left(\frac{1}{\hbar} \int_{x_2}^x |p(x)| \d{x} + \frac{\pi}{4} \right)} \right]  
\end{align*}
In region I, we will in general have a incident wave of the form $e^{ikx}$ and a reflected wave of the form $e^{-ikx}$ so there is no restriction on the form of the WKB solution is this region.
\subsection*{(b)}
We extend the solution in region III into region II by using the above connection formula. The sin term gives us no information but the cos term connects to,
\[ \frac{t}{\sqrt{|p(x)|}} \cos{\left(\frac{1}{\hbar} \int_{x_2}^x |p(x)| \d{x} + \frac{\pi}{4} \right)} \implies \psi_{II}(x) = \frac{t}{\sqrt{|p(x)|}} \exp{\left(\frac{1}{\hbar} \int_{x}^{x_2} |p(x)| \d{x} \right)}\]
In order to use the other connection formula, rewrite the wavefuntion in region II as,
\[\psi_{II}(x) = \frac{t}{\sqrt{|p(x)|}} \exp{\left(\frac{1}{\hbar} \int_{x_1}^{x_2} |p(x)| \d{x} \right)} \exp{\left(-\frac{1}{\hbar} \int_{x_1}^{x} |p(x)| \d{x} \right)} \]
the forbidden to allowed connection formula gives,
\[\frac{C}{\sqrt{|p(x)|}} \exp{\left(-\frac{1}{\hbar} \int_{x_1}^{x} |p(x)| \d{x} \right)} \implies \psi_I(x) = \frac{2C}{\sqrt{|p(x)|}} \sin{\left( \frac{1}{\hbar} \int_{x}^{x_1} |p(x)| \d{x} + \frac{\pi}{4} \right)} \]
Therefore, 
\[\psi_I(x) = \frac{2t}{\sqrt{|p(x)|}} \exp{\left(\frac{1}{\hbar} \int_{x_1}^{x_2} |p(x)| \d{x} \right)} \sin{\left( \frac{1}{\hbar} \int_{x}^{x_1} |p(x)| \d{x} + \frac{\pi}{4} \right)} \]
This wave represents the sum of the incident and transmitted waves. For $x \ll x_1$,
\[\psi_I(x) = A e^{ikx + i\frac{\pi}{4}} + Be^{-ikx - i\frac{\pi}{4}} \]
where the extra phases are introduced to simplyify $A$ and $B$. Therefore, 
\[A = -B = \frac{t}{i} \exp{\left(\frac{1}{\hbar} \int_{x_1}^{x_2} |p(x)| \d{x} \right)} \] 
The probability current in the incident wave is proportional to $|A|^2$ and the probability current in the transmitted wave is proportional to $|t|^2$ so the probability that a given particle penetrates the barrier and propagates to large positive $x$ is given by,
\[T = \left|\frac{t}{A} \right|^2 = \exp{\left( - \frac{2}{\hbar} \int_{x_1}^{x_2} \sqrt{2m(V(x) - E)} \d{x} \right)} \]
This holds because a wavepacket formed from energy eigenstates in region I will connect to a the same combination of energy eigenstates in region III which are identical functions up to an overall phase shift and a reduction in probability by $T$. By linearily, the overall probaility of the wavepacket in region III is reduced by the same factor $T$ as long as the wavefuntion is highly localized in energy so that $T$ has negligible variation over the sum of eigenstates in the wavepacket. If the barrier is much higher than the energy in most of the forbidden region then,
\[T \approx \exp{\left( - \frac{2 \Delta x }{\hbar}\sqrt{2m(V_0 - E)} \right)} \]
with $V_0$ being the average hight of the barrier in region II and $\Delta x = x_2 - x_1$ the width of the forbidden region. 

\section{Problem 21.}
Consider the angular momentum operators acting on a $j = 1$ multiplet.
\[\bra{j = 1, m} \J_3 \ket{j = 1, m'} = \bra{j = 1, m} \hbar m' \ket{j = 1, m'} = \hbar m' \delta_{mm'}\] because these are eigenstates of a Hermitian operator with distinct eigenvalues. Therefore, expressed in the basis \[\ket{j = 1, m = 1}, \quad \ket{j = 1, m = 0}\quad, \ket{j = 1, m = -1}\]
we have,
\[(\J_3)_{mm'} = \hbar
\begin{pmatrix}
1 & 0 & 0 \\
0 & 0 & 0 \\
0 & 0 & -1
\end{pmatrix}
\]
Now, write $\J_1 = \frac{1}{2} (\J_{+} + \J_{-})$ so we can compute,
\begin{align*}
\bra{j, m} \J_1 \ket{j, m'} & = \frac{1}{2}\big(\bra{j, m} \J_{+} \ket{j, m'} + \bra{j, m} \J_{-} \ket{j, m'}\big) \\ & = \frac{\hbar}{2} \big( \sqrt{j(j+1) - m'(m'+1)} \inner{j, m }{j, m' + 1} + \sqrt{j(j+1) - m'(m'-1)} \inner{j, m }{j, m' - 1} \big) \\ & = \frac{\hbar}{2} \big( \sqrt{2 - m'(m'+1)} \: \delta_{m \: m'+1} + \sqrt{2 - m'(m'-1)} \: \delta_{m \: m'-1} \big)
\end{align*}
Therefore, using the above basis,
\[(\J_1)_{mm'} = \frac{\hbar}{2}
\begin{pmatrix}
0 & \sqrt{2} & 0 \\
\sqrt{2} & 0 & \sqrt{2} \\
0 & \sqrt{2} & 0
\end{pmatrix}
\]
Similarly, $\J_2 = \frac{1}{2i} (\J_{+} - \J_{-})$ so
\begin{align*}
\bra{j, m} \J_2 \ket{j, m'} & = \frac{1}{2i}\big(\bra{j, m} \J_{+} \ket{j, m'} - \bra{j, m} \J_{-} \ket{j, m'}\big) \\ & = \frac{\hbar}{2i} \big( \sqrt{j(j+1) - m'(m'+1)} \inner{j, m }{j, m' + 1} - \sqrt{j(j+1) - m'(m'-1)} \inner{j, m }{j, m' - 1} \big) \\ & = \frac{\hbar}{2i} \big( \sqrt{2 - m'(m'+1)} \: \delta_{m \: m'+1} - \sqrt{2 - m'(m'-1)} \: \delta_{m \: m'-1} \big)
\end{align*}
Therefore, using the above basis,
\[(\J_2)_{mm'} = \frac{\hbar}{2i}
\begin{pmatrix}
0 & \sqrt{2} & 0 \\
-\sqrt{2} & 0 & \sqrt{2} \\
0 & -\sqrt{2} & 0
\end{pmatrix}
\]
Now, we compute the matrix of the commutatior,
\begin{align*}
[\J_1, \J_2] & = \J_1 \J_2 - \J_2 \J_1 \\ = & \frac{\hbar^2}{4i} 
\left[
\begin{pmatrix}
0 & \sqrt{2} & 0 \\
\sqrt{2} & 0 & \sqrt{2} \\
0 & \sqrt{2} & 0
\end{pmatrix}
\begin{pmatrix}
0 & \sqrt{2} & 0 \\
-\sqrt{2} & 0 & \sqrt{2} \\
0 & -\sqrt{2} & 0
\end{pmatrix}
-
\begin{pmatrix}
0 & \sqrt{2} & 0 \\
-\sqrt{2} & 0 & \sqrt{2} \\
0 & -\sqrt{2} & 0
\end{pmatrix}
\begin{pmatrix}
0 & \sqrt{2} & 0 \\
\sqrt{2} & 0 & \sqrt{2} \\
0 & \sqrt{2} & 0
\end{pmatrix} 
\right]
\\ = & \frac{\hbar^2}{4i} 
\left[
\begin{pmatrix}
-2 & 0 & 2 \\
0 & 0 & 0 \\
-2 & 0 & 2
\end{pmatrix}
-
\begin{pmatrix}
2 & 0 & 2 \\
0 & 0 & 0 \\
-2 & 0 & -2
\end{pmatrix} 
\right]
= 
\frac{\hbar^2}{4i} 
\begin{pmatrix}
-4 & 0 & 0 \\
0 & 0 & 0 \\
0 & 0 & 4
\end{pmatrix}
=
i \hbar \left[ \hbar
\begin{pmatrix}
1 & 0 & 0 \\
0 & 0 & 0 \\
0 & 0 & -1
\end{pmatrix} \right]
= i \hbar \J_3
\end{align*}

\end{document}


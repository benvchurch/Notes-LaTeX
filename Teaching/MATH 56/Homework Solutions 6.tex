\documentclass[12pt]{article}
\usepackage[margin=2.5cm]{geometry}
\usepackage{amsmath,amsthm,amssymb,graphicx,float}
\usepackage{mdframed}
\usepackage{pgfplots}
\usepackage{comment}
\usepgfplotslibrary{fillbetween}
\pgfplotsset{compat=1.15}
\usepackage[labelsep=space]{caption}
\usepackage{float}
\usepackage{wrapfig}
\usepackage{tikz-cd}
\usepackage{tikz}
\usepackage{enumitem}
\setlist[enumerate]{leftmargin=*}
\theoremstyle{definition}
	\newmdtheoremenv{prob}{Problem}
\theoremstyle{definition}
	\newtheorem*{soln}{Solution}
\newcommand{\N}{\mathbb{N}}
\newcommand{\Z}{\mathbb{Z}}
\newcommand{\Q}{\mathbb{Q}}
\newcommand{\R}{\mathbb{R}}
\newcommand{\C}{\mathbb{C}}
\newcommand{\F}{\mathbb{F}}
\let\i\relax
\newcommand{\i}{\mathbf{i}}
\let\j\relax
\newcommand{\j}{\mathbf{j}}
\newcommand{\T}{\mathbf{T}}
\let\r\relax
\newcommand{\r}{\mathbf{r}}
\let\k\relax
\newcommand{\k}{\mathbf{k}}
\newcommand{\Ker}{\operatorname{Ker}}
\let\Im\relax
\newcommand{\Im}{\operatorname{Im}}
\newcommand{\Coker}{\operatorname{Coker}}
\newcommand{\Ext}{\operatorname{Ext}}
\newcommand{\Hom}{\operatorname{Hom}}
\newcommand{\Span}{\operatorname{span}}
\newcommand{\Null}{\operatorname{null}}
\newcommand{\range}{\operatorname{range}}

\begin{document}

\title{Math 56: Proofs and Modern Mathematics\\ Homework 5 Solutions}
\author{Naomi Kraushar}
\maketitle

\begin{prob}[Axler 3.D.7]
Suppose $V$ and $W$ are finite dimensional and $v\in V$. Let 
\[E=\{T\in\mathcal{L}(V,W):Tv=0\}\]
\begin{enumerate}[label=\arabic*.]
\item Show that $E$ is a subspace of $\mathcal{L}(V,W)$.

\item Suppose $v\neq0$. What is $\dim E$?
\end{enumerate}
\end{prob}

\begin{soln}
\begin{enumerate}[label=\arabic*.]
\item We need to show that $E$ contains $0$ and is closed under addition and scalar multiplication. The zero element in $\mathcal{L}(V,W)$ is the map that sends everything to $0$ in $W$, which clearly sends $v$ to $0$ and so is in $E$. If $T_1$ and $T_2$ are in $E$, so that $T_1v=T_2v=0$, then $(T_1+T_2)(v)=T_1v+T_2v=0+0=0$, so $T_1+T_2\in E$. Finally, if $T\in E$ and $\lambda$ is a scalar, then $(\lambda T)(v)=\lambda(Tv)=\lambda(0)=0$, so $\lambda T\in E$. Hence $E$ is a subspace of $\mathcal{L}(V,W)$.

\item There are a few different ways we can prove this. I'll give you a few that I thought of or saw in your solutions.
\begin{enumerate}[label=\textbf{Method \arabic*.}]
\item Define the following map:
\begin{align*}
F:\mathcal{L}(V,W) &\to W\\
F(T) &= Tv.
\end{align*}
This is a linear map, since $(T+T')(v)=Tv+T'v$, and $(\lambda T)(v)=\lambda (Tv)$. By definition, $\Null F=\{T\in\mathcal{L}(V,W):Tv=0\}=E$. Since $v\neq 0$, we can extend $v$ to a basis $v=v_1,v_2,\dots,v_n$ of $V$, where $n=\dim V$. This means that for any $w\in W$, we can find $T\in\mathcal{L}(V,W)$ such that $Tv=w$: just let $Tv=w$ and $Tv_i=0$ for $i\geq 2$, and extend linearly to a map on $V$. Hence $F$ is surjective, so $\dim \range F=\dim W$. By the rank-nullity theorem, we have
\begin{align*}
\dim \Null F+\dim\range F &= \dim\mathcal{L}(V,W)\\
\implies \dim E = \dim\Null F &= \dim \mathcal{L}(V,W)-\dim\range F\\
&=\dim\mathcal{L}(V,W)-\dim W\\
&= \dim V\dim W-\dim W\\
&=(\dim V-1)\dim W.
\end{align*}

\item Since $v\neq 0$, we can extend $v$ to a basis $v=v_1,v_2,\dots,v_n$ of $V$, where $n=\dim V$. Let $V'=\Span(v_2,\dots,v_n)$, so that $V=\Span(v)\oplus V'$. Define the following map.
\begin{align*}
F:E &\to \mathcal{L}(V',W)\\
F(T)(v') &= T(v'),
\end{align*}
where $v'\in V'\subset V$ is any element in $V'$. What $F$ does is take a map in $E$, which means it's a map $V\to W$, and restrict it to $V'\subset V$. Since $(T+T')v'=Tv'+T'v'$, and $(\lambda T)v'=\lambda (Tv')$, the map $F$ is linear. Since $V=\Span(v)\oplus V'$, we can write every element in $V$ uniquely as $av+v'$ for some scalar $a$ and some $v'\in V'$. Using this, we define an inverse for $F$ given by 
\begin{align*}
F^{-1}:\mathcal{L}(V',W) &\to E\\
F^{-1}(T)(av+v') &= T(v').
\end{align*}
Note that for any $T\in\mathcal{L}(V',W)$, we have $F^{-1}(T)(v)=F^{-1}(T)(v+0)=0$, so that this is indeed a map in $E$. In the same way as above, $F^{-1}$ is linear. For a map $T\in E$, and an element $av+v'\in V$, we have $FF^{-1}(T)(av+v')=Tv'=T(av+v')$, since $T$ is linear and $Tv=0$, and for a map $T\in \mathcal{L}(V,W)$ and an element $v'\in V'$ we have $F^{-1}F(T)v'=Tv'$. Hence $F^{-1}$ is indeed an inverse for $F$, so $F$ is an isomorphism, and therefore
\[\dim E=\dim\mathcal{L}(V',W)=\dim V\dim W=(\dim V-1)\dim W.\]

\item We can prove this explicitly by finding a basis for $E$. Since $v\neq 0$, we can extend $v$ to a basis $v=v_1,v_2,\dots,v_n$ of $V$, where $n=\dim V$. Let $w_1,\dots,w_m$ be a basis for $W$. Define the linear maps $T_{ij}$, for $i=2,\dots,n$ and $j=1,\dots,m$ by
\begin{align*}
T_{ij}(v_i) &= w_j \\
T_{kj}(v_k) &= 0 \tag{for $k\neq i$.}
\end{align*}
Note that all of these map $v_1=v$ to $0$, so they are all in $E$. We claim that the set of $T_{ij}$ is a basis for $E$. We need to prove that this set is linearly independent and spans $E$.

\underline{Linearly independent:} Suppose that we have a set of scalars $a_{ij}$, $i=2,\dots,n$, $j=1,\dots,m$ such that
\[a_{21}T_{21}+\dots+a_{2m}T_{2m}+\dots+a_{n1}T_{n1}+\dots+a_{nm}T_{nm}=0.\]
Consider the vector $v_2\in V$. Since $a_{21}T_{21}+\dots+a_{2m}T_{2m}+\dots+a_{n1}T_{n1}+\dots+a_{nm}T_{nm}=0$, we have
\[a_{21}T_{21}v_2+\dots+a_{2m}T_{2m}v_2+\dots+a_{n1}T_{n1}v_2+\dots+a_{nm}T_{nm}v_2=0.\]
Terms on the left-hand side of the form $a_{ij}T_{ij}v_2$ where $i\neq 2$ are $0$, and $T_{2j}v_2=w_j$, so this simplifies to
\[a_{21}w_1+\dots+a_{2m}v_m=0.\]
By linear independence of $w_1,\dots,w_m$, this means that $a_{21},\dots,a_{2m}=0$. Applying the same method for each $v_i$ gives $a_{ij}=0$ for $j=1,\dots,m$. Hence $a_{ij}=0$ for all $i=2,\dots,n$ and $j=1,\dots,m$, so the set of these $T_{ij}$ is linearly independent as required.

\underline{Spans $E$:} Let $T$ be an arbitrary map in $E$, so that $Tv=0$. For $i=2,\dots,n$, we have
\[Tv_i=a_{i1}w_j+\dots+a_{im}w_m\]
for some scalars $a_{i1},\dots,a_{im}$. Since $T$ is completely determined by the images of the basis elements $v=v_1,v_2,\dots,v_n$, we can conclude that
\[T=a_{21}T_{21}+\dots+a_{2m}T_{2m}+\dots+a_{n1}T_{n1}+\dots+a_{nm}T_{nm},\]
since applying the maps on both sides to a basis vector gives us the same output, namely $0$ for $v$, and $a_{i1}w_j+\dots+a_{im}w_m$ for $v_i$, where $i=2,\dots, n$. Hence the set of $T_{ij}$ spans $E$ as required.

We have a basis for $E$, namely, the set of $T_{ij}$, where $i=2,\dots,n$ and $j=1,\dots,m$. This gives $\dim E=(n-1)m=(\dim V-1)\dim W$.

\item Since $v\neq 0$, we can extend it to a basis $v=v_1,\dots,v_n$ of $V$. Let $w_1,\dots,w_n$ be a basis for $W$. For each linear map $T\in\mathcal{L}(V,W)$, we can define its matrix $M(T)$ with respect to these bases; this gives us an isomorphism from $\mathcal{L}(V,W)$ to $\F^{m,n}$, which has dimension $M$. By definition, $T$ is in $E$ if and only if the first column of $M(T)$ is $0$, so $E$ corresponds to the subspace of $\F^{m,n}$ where the first column is all $0$-s, hence $\dim E=m(n-1)=\dim W(\dim V-1)$.
\end{enumerate}
\end{enumerate}
\end{soln}

\begin{prob}[Axler 3.D.14]
Suppose $v_1,v_2,\dots,v_n$ is a basis of $V$. Prove that the map $T:V\to \F^{n,1}$ defined by $Tv=M(v)$ is an isomorphism of $V$ onto $\F^{n,1}$. Here $M(v)$ is the matrix of $v$ with respect to the basis $v_1,\dots,v_n$.
\end{prob}

\begin{soln}
Recall that for $v=a_1v_1+\dots+a_nv_n$, the matrix of $v$ with respect to the basis $v_1,\dots,v_n$ is given by
\[\mathcal{M}(v)=\begin{bmatrix} a_1 \\ \vdots \\ a_n \end{bmatrix}.\]
This defines a map $T:V\to\F^{n,1}$ given by $Tv=\mathcal{M}(v)$. We want to show that this map is an isomorphism, i.e., that it is linear and invertible.

\underline{Linear:}
Suppose that $v,v'$ are two elements in $V$. Using the given basis, we can write $v=a_1v_1+\dots+a_nv_n$ and $v'=b_1v_1+\dots+b_nv_n$. Then
\begin{align*}
T(v+v') &= T\left((a_1v_1+\dots+a_nv_n)+(b_1v_1+\dots+b_nv_n)\right) \tag{writing $v,v'$ in terms of the given basis}\\
&= T\left((a_1+b_1)v_1+\dots+(a_n+b_n)v_n\right) \tag{distributivity and commutative of addition in $V$}\\
&= \begin{bmatrix} a_1+b_1 \\ \vdots \\ a_n+b_n \end{bmatrix} \tag{definition of $T$}\\
&= \begin{bmatrix} a_1 \\ \vdots \\ a_n \end{bmatrix}+\begin{bmatrix} b_1 \\ \vdots \\ b_n \end{bmatrix} \tag{addition in $\F^{n,1}$}\\
&= Tv+Tv' \tag{definition of $T$.}
\end{align*}
Hence $T$ is additive.

Next, suppose that $v$ is an element of $V$, with $v=a_1v_1+\dots+a_nv_n$, and $\lambda$ is a scalar. Then
\begin{align*}
T(\lambda v) &= T\left(\lambda(a_1v_1+\dots+a_nv_n)\right)\\
&= T(\lambda a_1v_1+\dots+\lambda a_nv_n) \tag{distributivity in $V$}\\
&= \begin{bmatrix} \lambda a_1 \\ \vdots \\ \lambda a_n \end{bmatrix} \tag{definition of $T$}\\
&= \lambda \begin{bmatrix} a_1 \\ \vdots \\ a_n \end{bmatrix} \tag{scalar multiplication in $\F^{n,1}$}\\
&= \lambda Tv \tag{definition of $T$.}
\end{align*}
Hence $T$ is also homogeneous, and so is a linear map.

We now want to prove that $T$ is invertible. There are two similar ways we can approach this.
\begin{enumerate}[label=\textbf{Method \arabic*.}]
\item We can define an inverse for $T$. Define the map 
\begin{align*}
S:\F^{n,1} &\to V\\
S\left(\begin{bmatrix} a_1 \\ \vdots \\ a_n \end{bmatrix}\right) &= a_1v_1+\dots+a_nv_n
\end{align*}
By definition, we have $STv=v$ and $TS\left(\begin{bmatrix} a_1 \\ \vdots \\ a_n \end{bmatrix}\right)=\left(\begin{bmatrix} a_1 \\ \vdots \\ a_n \end{bmatrix}\right)$. Hence $S$ is an inverse for $T$, so $T$ is invertible and therefore an isomorphism.

\item We can show that $T$ is both injective and surjective.

\underline{Injective:} Suppose that $Tv=0$. We can write $v=a_1v_1+\dots+a_nv_n$, so that $Tv=\begin{bmatrix} a_1 \\ \vdots \\ a_n \end{bmatrix}$. By definition, if this is the zero matrix in $\F^{n,1}$, then we must have $a_1=\dots=a_n=0$, so that $v=0$. Hence $\null T$ contains only the zero vector, so $T$ is injective.

\underline{Surjective:} Consider an arbitrary matrix $M=\begin{bmatrix} a_1 \\ \vdots \\ a_n \end{bmatrix}$ in $\F^{n,1}$. Let $v=a_1v_1+\dots+a_nv_n$; by definition, this means that $Tv=M$. Hence $T$ is surjective.

\textbf{Note:} We can use the rank-nullity theorem, along with the fact that $\dim V=\dim\F^{n,1}=n$, to show that $T$ is invertible using only one of the above properties.
\end{enumerate}
\end{soln}

\break

\begin{prob}
Compute the matrix of the linear map $T:V\to W$ with respect to the bases $\mathcal{B}$ of $V$ and $\mathcal{C}$ of $W$ for the following choices: $V=\mathcal{P}_2(R)$, $W=\mathcal{P}+3(R)$, $\mathcal{B}=(1,x,x^2)$, $\mathcal{C}=(1,x,x^2,x^3)$,
\[(Tp)(x) =\int_0^x p(t)dt+p'(x).\]
\end{prob}

\begin{soln}
We compute the image of each basis element in $\mathcal{B}$ under $T$. For $p(x)=x^n$, we have
\[\int_0^xp(t)dt=\int_0^x t^ndt=\left[\frac{1}{n+1}t^{n+1}\right]_0^x=\frac{1}{n+1}x^{n+1},\]
and $p'(x)=nx^{n-1}$. This gives us
\begin{align*}
T1(x) &= x\\
Tx(x) &= \frac{1}{2}x^2+1\\
Tx^2(x) &= \frac{1}{3}x^3+2x.
\end{align*}
Hence the matrix of $T$ with respect to the bases $\mathcal{B}$ and $\mathcal{C}$ is
\[\begin{bmatrix} 0 & 1 & 0 \\ 1 & 0 & 2 \\ 0 & 1/2 & 0 \\ 0 & 0 & 1/3 \end{bmatrix}.\]
\end{soln}

\break

\begin{prob}
Let $v_1,v_2$ be  the  basis $\mathcal{B}$ of $V$ and  define a new  basis $\mathcal{C}$ of $V$ by $w_1=v_1$, $w_2=v_1+v_2$. Let $T\in\mathcal{L}(V,V)$ be defined by $Tw_1=2w_1$, $Tw_2=-3w_2$. What is $M(T)$ in the basis $\mathcal{C}$? What is $M(T)$ in the basis $\mathcal{B}$?
\end{prob}

\begin{soln}
By definition, the matrix for $T$ with respect to $\mathcal{C}$ is just
\[\begin{bmatrix} 2 & 0 \\ 0 & -3 \end{bmatrix}.\]
To compute $M(T)$ in the basis $\mathcal{B}$, we're going to need to compute $Tv_1$ and $Tv_2$ in terms of $v_1$ and $v_2$. To do this, we note that $v_1=w_1$ and $v_2=w_2-v_1=w_2-w_1$. So we have
\begin{gather*}
Tv_1 = Tw_1 = 2w_1 = 2v_1 \\
Tv_2 = T(w_2-w_1) = Tw_2-Tw1 = -3w_2-2w_1 = -3(v_1+v_2)-2v_1 = -5v_1-3v_2
\end{gather*}
Hence $M(T)$ with respect to $\mathcal{B}$ is 
\[\begin{bmatrix} 2 & -5 \\ 0 & -3 \end{bmatrix}.\]
\end{soln}

\end{document}
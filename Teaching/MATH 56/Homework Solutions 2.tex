\documentclass[12pt]{article}
\usepackage[margin=2.5cm]{geometry}
\usepackage{amsmath,amsthm,amssymb,graphicx,float}
\usepackage{mdframed}
\usepackage{pgfplots}
\usepackage{comment}
\usepgfplotslibrary{fillbetween}
\pgfplotsset{compat=1.15}
\usepackage[labelsep=space]{caption}
\usepackage{float}
\usepackage{wrapfig}
\usepackage{tikz-cd}
\usepackage{tikz}
\usepackage{enumitem}
\setlist[enumerate]{leftmargin=*}
\theoremstyle{definition}
	\newmdtheoremenv{prob}{Problem}
\theoremstyle{definition}
	\newtheorem*{soln}{Solution}
\newcommand{\N}{\mathbb{N}}
\newcommand{\Z}{\mathbb{Z}}
\newcommand{\Q}{\mathbb{Q}}
\newcommand{\R}{\mathbb{R}}
\newcommand{\C}{\mathbb{C}}
\newcommand{\F}{\mathbf{F}}
\let\i\relax
\newcommand{\i}{\mathbf{i}}
\let\j\relax
\newcommand{\j}{\mathbf{j}}
\newcommand{\T}{\mathbf{T}}
\let\r\relax
\newcommand{\r}{\mathbf{r}}
\let\k\relax
\newcommand{\k}{\mathbf{k}}
\newcommand{\Ker}{\operatorname{Ker}}
\let\Im\relax
\newcommand{\Im}{\operatorname{Im}}
\newcommand{\Coker}{\operatorname{Coker}}
\newcommand{\Ext}{\operatorname{Ext}}
\newcommand{\Hom}{\operatorname{Hom}}

\begin{document}

\title{Math 56: Proofs and Modern Mathematics\\ Homework 1 Solutions}
\author{Naomi Kraushar}
\maketitle

\begin{prob}
Suppose $X$ is a non-empty set, and let $S$ be the collection of maps $f:X\to X$. Show that $S$ is a monoid, with composition of maps as the operation: $\circ:S\times S\to S$.
\end{prob}

\begin{soln}
Since $X$ is nonempty, there exist maps from $X$ to itself, so $S$ is non-empty. We need to prove that composition is associative, and that there exists an identity element.
\underline{Associativity:} Let $f,g,h$ be maps from $X$ to itself; we want to show that $(f\circ g)\circ h = f\circ (g\circ h)$. Let $x$ be an arbitrary element in $X$. By definition, we have
\[((f\circ g)\circ h)(x)=(f\circ g)(h(x))=f(g(h(x))).\]
Similarly, we have
\[(f\circ(g\circ h))(x)=f((g\circ h)(x))=f(g(h(x))).\]
Hence $((f\circ g)\circ h)(x) = (f\circ (g\circ h))(x)$ for all $x\in X$, so $(f\circ g)\circ h = f\circ (g\circ h)$, as required.

\underline{Identity:} Define the identity function $e:X\to X$ by $e(x)=x$ for all $x\in X$. Let $f$ be an arbitrary function in $S$, and $x$ any element in $X$. We then have
\[(e\circ f)(x)=e(f(x))=f(x), \quad (f\circ e)(x)=f(e(x))=f(x).\]
Hence $(e\circ f)(x)=(f\circ e)(x)$ for all $f\in S$ and $x\in X$, so $e\circ f=f\circ e$ for all $f\in S$. Hence   and as required.

Having proven associativity of composition, and the existence of an identity element, we conclude that $S$ is a monoid.
\end{soln}

\begin{prob}
Suppose $(F,+,\cdot)$ is a field.  Show that $x,y\in F$ and $x\cdot y= 0$ imply that either $x=0$ or $y=0$. 
\end{prob}

\begin{soln}
First, we will need the fact that for any $a\in F$, we have $a\cdot 0=0$. You have seen this already, but I'll prove it again here to make sure: we have
\begin{align*}
a\cdot 0 &= a \cdot (0+0) \tag{since $0$ is the additive identity}\\
&= a\cdot 0 + a \cdot 0 \tag{by the distributive law}\\
\implies 0 &= a\cdot 0 \tag{adding the additive inverse of $a\cdot 0$ to both sides}
\end{align*}
so $a\cdot 0=0$ as required.

Now suppose that we have $x,y\in F$ with $x\cdot y=0$; we want to show that $x=0$ or $y=0$. Suppose therefore that $x\neq 0$; we now have to show that this forces $y=0$. Since $x\neq 0$, $x$ has a multiplicative inverse $x^{-1}$. Multiplying both sides of the equation $x\cdot y=0$ by $x^{-1}$ on the left, we have
\begin{align*}
x^{-1}\cdot (x\cdot y) &= x^{-1} \cdot 0 \\
\implies (x^{-1}\cdot x)\cdot y &= 0 \tag{associativity of multiplication, also $a\cdot 0=0$ for all $a\in F$}\\
\implies 1\cdot y &= 0 \tag{by definition of the multiplicative inverse $x^{-1}$}\\
\implies y &= 0 \tag{since $1$ is the multiplicative identity.}
\end{align*}
Hence $y=0$ as required.
\end{soln}

\begin{prob}
Let $F$ be the subset of $\R$ given by numbers of the form
\[\{a+b\sqrt{2}:a,b\in\Q\},\]
and define $+$ and $\cdot$ to be the usual operations inherited from $\R$.
\begin{enumerate}[label=(\alph*)]
\item Show that for $x,y\in F$, one has $x+y$, $x\dot y\in F$.

\item Show that $(F,+,\cdot)$ is a field.
\end{enumerate}
\end{prob}

\begin{soln}
\begin{enumerate}[label=(\alph*)]
\item Let $x,y$ be elements of $F$, so we have $x=a+b\sqrt{2}$, $y=c+d\sqrt{2}$ for some $a,b,c,d\in \Q$. We then compute
\begin{align*}
x+y &= (a+b\sqrt{2})+(c+d\sqrt{2}) \\
&= (a+c)+(b\sqrt{2}+d\sqrt{2}) \tag{using associativity and commutativity of addition in the field $\R$}\\
&= (a+c)+(b+d)\sqrt{2} \tag{using distribution in $\R$.}
\end{align*}
Since $\Q$ is a field, we have $a+c\in Q$ and $b+d\in Q$, so this is an element of $F$. Similarly for multiplication, we have
\begin{align*}
x\cdot y &= (a+b\sqrt{2})(c+d\sqrt{2}) \\
&= ac+ad\sqrt{2}+bc\sqrt{2}+bd\sqrt{2}\sqrt{2} \tag{using distribution in $\R$}\\
&= (ac+2bd)+(ad+bc)\sqrt{2} \tag{using $\sqrt{2}^2=2$, distribution in $\R$.}
\end{align*}
Again, since $\Q$ is a field, we have $ac+2bd\in\Q$ and $ad+bc\in Q$, so this is an element of $F$.

\item Part (a) shows us that $F$ is closed under addition and multiplication; in addition, $0$ and $1$ are elements of $F$, since we can take $a=0, b=0$ for the former and $a=1, b=0$ for the latter in the definition of $F$. Since $F$ is a subset of $\R$ with the same addition and multiplication, $F$ inherits the associativity, commutativity, and identity axioms for both addition and multiplication, as well as the distribution axioms. It remains to prove the inverse axioms in $F$.

Let $x=a+b\sqrt{2}$ be any element of $F$. Since $a,b\in\Q$ and $\Q$ is a field, we also have $-a,-b\in \Q$, so $y=-a-b\sqrt{2}\in F$. We also have
\[x+y=(a+b\sqrt{2})+(-a-b\sqrt{2})=(a-a)+(b-b)\sqrt{2}=0,\]
using axioms from $\R$. Hence every element $x\in F$ has an additive inverse.

Now suppose that $x\neq 0$, so $x=a+b\sqrt{2}$ where $a$ and $b$ are not both $0$. As many of you may have seen, for the inverse $x^{-1}=\frac{1}{a+b\sqrt{2}}$ that we want, we can rationalize the denominator to get the expression
\[\frac{a-b\sqrt{2}}{(a+b\sqrt{2})(a-b\sqrt{2})} = \frac{a-b\sqrt{2}}{a^2-2b^2}=\frac{a}{a^2-2b^2}+\frac{-b}{a^2-2b^2}\sqrt{2}.\]
We need to show that this is an element of $F$, i.e. that the expressions $\frac{a}{a^2-2b^2}$ and $\frac{-b}{a^2-2b^2}$ are rational. Both numerators and denominators are rational, so these are rational numbers so long as the denominator is nonzero, so we'll need to prove that $a^2-2b^2\neq 0$.

We can show that in two different ways. First method: if $x=a+b\sqrt{2}$ is nonzero, then $a,b$ are nonzero, so $a-b\sqrt{2}$ is also nonzero. We have two nonzero elements in the field $\R$, and we know from problem $2$ that the product of two nonzero elements in a field is nonzero, so $a^2-2b^2\neq 0$. Alternatively, suppose $a^2-2b^2=0$. If $b=0$, we then have $a^2=0$, so $a=0$ by Problem 2, but this gives $x=0$, which is false. If $b\neq 0$, we can divide by $b$ to get $2=a^2/b^2$, but $2$ is not the square of a rational number, by Problem 1, so this is also impossible. Hence $a^2-2b^2\neq 0$ for all $a,b\in\Q$ not both $0$. Either way, we find that $\frac{a}{a^2-2b^2}\in \Q$ and $\frac{-b}{a^2-2b^2}\in\Q$. Multiplying this by $x$ gives
\[(a+b\sqrt{2})\left(\frac{a}{a^2-2b^2}+\frac{-b}{a^2-2b^2}\sqrt{2}\right)=\frac{(a+b\sqrt{2})(a-b\sqrt{2})}{a^2-2b^2}=\frac{a^2-2b^2}{a^2-2b^2}=1,\]
as required. Hence if $x\neq 0$, it has a multiplicative inverse in $F$, and this completes the proof.
\end{enumerate}
\end{soln}

\begin{prob}
Show that if $n\ge 2$ is an integer then $\Z/n\Z$ is a commutative ring with a unit.  (You may use that $(\Z/n\Z,+)$ is a commutative group, as shown in class.)
\end{prob}

\begin{soln}
We already know that $(\Z/n\Z,+)$ is a commutative group, so it remains to show that $(\Z/n\Z,\cdot)$ is a commutative monoid, and that distributivity holds. First, we define multiplication (as you might expect) by $[a][b]=[ab]$; we need to show that this is well-defined, that it obeys associativity and commutativity, that there is an identity element, and that it is distributive.
\begin{itemize}
\item Well-defined: suppose we have integers $a$, $a'$, $b$, and $b'$ such that $[a]=[a']$ and $[b]=[b']$; we need to show that $[a][b]=[a'][b']$, so that it does not matter which integer we choose in a particular equivalence class. By definition, since $[a]=[a']$, we have $a-a'=pn$ for some integer $p$, and similarly $b-b'=qn$ for some integer $q$. Using the distribution law in $\Z$, we have
\[ab-a'b' = ab-ab'+ab'-a'b' = a(b-b')+(a-a')b' = aqn+pnb' = n(aq+pb'),\]
so that $[a][b]=[a'][b']$, by definition. Hence multiplication is well defined.

\item Associativity: let $[a],[b],[c]$ be elements of $\Z/n\Z$. We have
\begin{align*}
\left([a][b]\right) [c] &= [ab][c] \tag{by definition of multiplication in $\Z/n\Z$}\\
&= [(ab)c] \tag{definition of multiplication in $\Z/n\Z$}\\
&= [a(bc)] \tag{associativity of multiplication in $\Z$}\\
&= [a][bc] \tag{definition of multiplication in $\Z/n\Z$}\\
&= [a]\left([b][c]\right) \tag{definition of multiplication in $\Z/n\Z$.}
\end{align*}
So multiplication is associative.

\item Commutativity: let $[a], [b]$ be elements of $\Z/n\Z$. We have
\begin{align*}
[a][b] &= [ab] \tag{multiplication in $\Z/n\Z$}\\
&= [ba] \tag{commutative of multiplication in $\Z/n\Z$}\\
&= [b][a] \tag{multiplication in $\Z/n\Z$.}
\end{align*}
So multiplication is associative.

\item Identity: let $[a]$ be an element of $\Z/n\Z$. We have
\begin{align*}
[1][a] &= [1a] \tag{multiplication in $\Z/n\Z$}\\
&= [a] \tag{identity in $\Z$.}
\end{align*}
By commutativity, we also have $[a][1]=[a]$. Hence multiplication has an identity (or unit), $[1]$.

\item Distributivity: let $[a], [b], [c]$ be elements of $\Z/n\Z$. We have
\begin{align*}
[a]\left([b]+[c]\right) &= [a][b+c] \tag{addition in $\Z/n\Z$}\\
&= [a(b+c)] \tag{multiplication in $\Z/n\Z$}\\
&= [ab+ac] \tag{distribution in $\Z$}\\
&= [ab]+[ac] \tag{addition in $\Z/n\Z$}\\
&= [a][b]+[a][c] \tag{multiplication in $\Z/n\Z$.}
\end{align*}
By commutativity, we also have $([b]+[c])[a]=[b][a]+[c][a]$. Hence the distributive properties hold.
\end{itemize}
Hence $(\Z/n\Z,+,\cdot)$ is a commutative ring with unit.
\end{soln}

\begin{prob}
Assuming all properties but that non-zero elements have multiplicative inverses (i.e. assuming that $\Z/p\Z$ is a commutative ring with a unit), as you may by Problem 4, show that $\Z/p\Z$ is a field when $p$ is a prime.

{\small (Hint:Let $a\in\{1,\dots,p-1\}$.  Show that it suffices to find $b\in\Z$ such that $ab-1\in p\Z$.  On the other hand, to prove this, consider the $p-1$ integers, $1\cdot a,2\cdot a,\dots,(p-1)a$.  Note that none of these is a multiple of $p$ (since $p$ is a prime, and $1\le a\le p-1$), so none of these lies in $[0]$, the equivalence class of 0 modulo $p$ (a.k.a. none of them is a multiple of $p$).  Since there are exactly $p-1$ non-zero equivalence classes modulo $p$, there are two cases: either no two of these $p-1$ numbers lies in the same class (i.e. they all lie in different classes), or two lie in the same class, i.e. for  some $b, c\in\{1,\dots,p-1\}$, $b\neq c$, $ba-ca$ is a multiple of $p$. Show that the latter cannot happen.)}
\end{prob}

\begin{soln}
As noted in the question, problem 4 already tells us that $\Z/p\Z$ is a commutative ring with unit, so the only property of a field that remains to be proven is the existence of multiplicative inverses for all nonzero elements. Let $[a]$ be a nonzero element of $\Z/p\Z$; we can assume without loss of generality that $a\in\{1,\dots,p-1\}$ since every equivalence class has an integer between $0$ and $p-1$, and $[a]\neq [0]$. We want to find $[b]$ such that $[a][b]=1$; again we may assume that $b\in\{1,\dots,p-1\}$ since the inverse of $[a]$ cannot of $[0]$. By definition, $[a][b]=[1]$ if and only if $ab-1$ is divisible by $p$, so we have reduced the problem to finding $b\in\{1,\dots,p-1\}$ such that $ab-1\in p\Z$. If we consider all possible values of $ab$ for $b\in\{1,\dots,p-1\}$, we have the list of integers $a,2a,\ldots,(p-1)a$. Since each of these is the product of two integers less than $p$, and $p$ is prime, none of these are divisible by $p$, and so must be in one of the equivalence classes $[1],[2],\dots,[p-1]$. We have a list of $p-1$ integers that must all be in one of $p-1$ equivalence classes, so either each is in a different equivalence class, or two distinct integers in the list are in the same equivalence class. Suppose we have two elements in the list, $ab$ and $ac$, that are in the same equivalence class. By definition, this means that $ab-ac\in p\Z$, so that $a(b-c)$ is divisible by $p$. But $a,b,c\in\{1,\dots,p-1\}$, so $a(b-c)$ is not divisible by $p$ unless $b-c=0$. Hence $ab=ac$, which means that distinct integers in the list must be in different equivalence classes. Since there are $p-1$ integers and $p-1$ equivalence classes, there must be some $b\in\{1,\dots,p-1\}$ such that $ab$ is in $[1]$, i.e. $ab-1\in p\Z$, which is what we needed to prove. Hence $[a]$ has an inverse and $\Z/p\Z$ is a field, as required.
\end{soln}

\end{document}
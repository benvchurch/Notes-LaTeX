\documentclass[12pt]{article}
\usepackage{hyperref}
\hypersetup{
    colorlinks=true,
    linkcolor=blue,
    filecolor=magenta,      
    urlcolor=cyan,
}
 
\urlstyle{same}
\usepackage{import}
\import{./}{AlgGeoCommands}


\AtBeginDocument{\renewcommand{\d}{\mathrm{d}}}
\newcommand{\ch}[1]{\mathrm{char}\left( #1 \right)}
 

\begin{document}

\tableofcontents

\newpage

\section{I Varieties}

\section{II Schemes}

\subsection{1}

\subsubsection{1.8}

Given a continuous map $f : X \to Y$ the functor $f^{-1} : \Sh(Y) \to \Sh(X)$ is a left-adjoint to the functor $f_* : \Sh(X) \to \Sh(Y)$. Therefore $f^{-1}$ is cocontinuous and right-exact and $f_*$ is continuous and left-exact. In fact, $f^{-1}$ is exact. 

\begin{lemma}
The functor $f^{-1}$ preserves stalks.
\end{lemma}

\begin{proof}
Let $\F$ be a sheaf on $Y$ and $f : X \to Y$ a continuous map. Then $f^{-1} \F$ is the sheafification of the presheaf,
\[ U \mapsto \varinjlim_{V \supset f(U)} \F(V) \]
The stalks of this presheaf are,
\[ S_x = \varinjlim_{x \in U} \varinjlim_{V \supset f(U)} \F(V) = \varinjlim_{f(x) \in V} \F(V) = \F_x \]
Since sheafification preserves stalks we have shown that $(f^{-1} \F)_x = \F_x$.
\end{proof}

\begin{proposition}
The functor $f^{-1}$ is exact. 
\end{proposition}

\begin{proof}
The functor $f^{-1}$ comutes with talking stalks. Therefore, applying $f^{-1}$ to an exact sequence preserves exactness on the stalks and thus exactness of the sequence.
\end{proof}

Consider two special cases. First, consider the constant map $C : X \to *$ sending all of $X$ to a point. Then $C_* \F = \Gamma(X, \F)$ is the sheaf which sends the only nonempty open set of $*$ to $\F(C^{-1}(*)) = \F(X)$. Furthermore, any abelian group $A$ is a sheaf on $*$ so $C^*(A)$ is the sheafification of $U \mapsto A$ and thus the constant sheaf $\underline{A}$ on $X$. Thus $\Gamma(X, -)$ is left-exact and $A \mapsto \underline{A}$ is exact.
\bigskip\\
Second, consider the inclusion $\iota_x : * \to X$ sendng $*$ to $x \in X$. Then given a sheaf $\F$ on $X$ we have,
\[ \iota_x^{-1} \F = \varinjlim_{x \in U} \F(U) = \F_x \]
and for an abelian group $A$ (as a sheaf on $*$) we have $(\iota_x)_* A$ is the skysraper sheaf at $x$ with stalk $A$. Thus, taking skyscrapers is left-exact and takiking stalks is exact.
\bigskip\\
Finally, this is easily proven directly. Given an exact sequence of sheaves,
\begin{center}
\begin{tikzcd}
0 \arrow[r] & \F \arrow[r, "f"] & \G \arrow[r, "g"] & \H 
\end{tikzcd}
\end{center}
then we must show that the sequence,
\begin{center}
\begin{tikzcd}
0 \arrow[r] & \Gamma(U, \F) \arrow[r, "f"] & \Gamma(U, \G) \arrow[r, "g"] & \Gamma(U, \H)
\end{tikzcd}
\end{center}
for any open set $U \subset X$. (FINISH)

\subsubsection{1.14}

Let $\F$ be a sheaf on $X$ and $s \in \F(U)$ a section on some open set $U$. Then consider the set,
\[ \Supp{\F}{s} = \{ x \in U \mid s_x \neq 0 \} \]
Suppose $x \in U \setminus \Supp{\F}{s}$ then $s_x = 0$. Thus, there exists some open neighborhood $x \in V \subset U$ such that $s|_V = 0$. Then for each $y \in V$ we have $_y = (s|_V)_y = 0$ so $y \in U \setminus \Supp{\F}{s}$ and thus $V \subset U \setminus \Supp{\F}{s}$. Therefore, $U \setminus \Supp{\F}{s}$ is open so $\Supp{\F}{s}$ is closed.
\bigskip\\
We furthermore define $\Supp{}{\F} = \{ x \in X \mid \F_x \neq 0 \}$ which is not necessarily closed without further assumptions on $\F$. If $\F$ is a coherent $\struct{X}$-module then this holds because on affine opens $\Supp{}{\F} \cap U = \Supp{}{M} = V(\Ann{A}{M})$ which is closed in $U$ where $U = \Spec{A}$ and $\F |_U = \widetilde{M}$ a finitely-generated $A$-module. 

\subsubsection{1.15}

Let $\F$ and $\G$ be sheaves of abelian groups on $X$ (in fact, $\F$ need only be a presheaf). Consider the presheaf $\shHom{}{\F}{\G}$ given by sending $U \mapsto \Hom{}{\F|_U}{\G|_U}$. I claim that this presheaf is actually a sheaf. First, let $f : \F|_U \to \G|_U$ be a morphism of sheaves and $\{ V_i \}$ and open cover of $U$ such that $f|_{V_i} = 0$ on each $V_i$. Let $\tilde{U} \subset U$ be any open subset and consider $f_{\tilde{U}} : \F(\tilde{U}) \to \G(\tilde{U})$. There is an open cover $\tilde{V}_i = \tilde{U} \cap V_i$ of $\tilde{U}$ and since $\tilde{V}_i \subset V_i$ we have $f|_{\tilde{V}_i} = 0$. Then for $s \in \F(V)$ we have \[ \res_{\tilde{V}_i, \tilde{U}} \circ f_{\tilde{U}}(s) = f_{\tilde{V}_i} \circ \res_{\tilde{V}_i, \tilde{U}}(s) = 0 \]
Therefore, $f_{\tilde{U}}(s)$ restricted to the cover $\tilde{V}_i$ is zero so by the sheaf property of $\G$ we have $f_{\tilde{U}}(s) = 0$. Thus, $f = 0$ proving the locality property of $\shHom{}{\F}{\G}$. 
\bigskip\\
Now, suppose that $V_i$ is an open cover of the open subset $U \subset X$ as before and we have $f_i \in \shHom{}{\F}{\G}(V_i) = \Hom{}{\F|_{V_i}}{\G|_{V_i}}$ which agree on the overlaps. Take any open $\tilde{U} \subset U$ and cover it viw $\tilde{V}_i = \tilde{U} \cap V_i$. Now define a morphism $f : \F|_U \to \G|_U$ such that,
\begin{center}
\begin{tikzcd}[column sep = large, row sep = large]
\F(\tilde{U}) \arrow[r, "f_{\tilde{U}}"] \arrow[d, "\res_{\tilde{V}_i, \tilde{U}}"']  & \G(\tilde{U}) \arrow[d, "\res_{\tilde{V}_i, \tilde{U}}"] 
\\
\F(\tilde{V}_i) \arrow[r, "(f_i)_{\tilde{V}_i}"] & \G(\tilde{V}_i)
\end{tikzcd}
\end{center}
as follows. Given $s \in \F(\tilde{U})$ let $s_i = s |_{V_i}$. Then the sections $(f_i)_{\tilde{V}_i}(s_i)$ agree on overlaps because,
\[ \res_{\tilde{V}_i \cap \tilde{V}_j, \tilde{V}_i} \circ (f_i)_{\tilde{V}_i}(s_i) = (f_i)_{\tilde{V}_i \cap \tilde{V}_j} \circ \res_{\tilde{V}_i \cap \tilde{V}_j, \tilde{V}_i}(s_i) = (f_i)_{\tilde{V}_i \cap \tilde{V}_j}(s |_{\tilde{V}_i \cap \tilde{V}_j}) \]
However, by assumption, $(f_i)_{\tilde{V}_i \cap \tilde{V}_j} = (f_j)_{\tilde{V}_i \cap \tilde{V}_j}$ and thus,
\[ \res_{\tilde{V}_i \cap \tilde{V}_j, \tilde{V}_j} \circ (f_j)_{\tilde{V}_j}(s_j) = (f_j)_{\tilde{V}_i \cap \tilde{V}_j}(s |_{\tilde{V}_i \cap \tilde{V}_j}) = \res_{\tilde{V}_i \cap \tilde{V}_j, \tilde{V}_i} \circ (f_i)_{\tilde{V}_i}(s_i) \]
Therefore, by the sheaf property of $\G$ these sections glue to form a unique section $f_{\tilde{U}}(s) \in \G(\tilde{U})$. We must check that the constructed $f$ is a homomorphism and satisfies the naturality conditions. Take $s,t \in \tilde{U}$ then,
\[ (f_i)_{\tilde{V}_i}((s + t)|_{\tilde{V}_i}) = (f_i)_{\tilde{V}_i}(s_i + t_i) = (f_i)_{\tilde{V}_i}(s_i) + (f_i)_{\tilde{V}_i}(t_i) \]
We know that these sections lift to $f_{\tilde{U}}(s)$ and $f_{\tilde{U}}(s)$ respectively showing that the sum lifts to $f_{\tilde{U}}(s) + f_{\tilde{U}}(t)$ because restriction is linear. Therefore, be definition the lift of these sections gives,
\[ f_{\tilde{U}}(s + t) = f_{\tilde{U}}(s) + f_{\tilde{U}}(t) \]
so $f$ is a collection of homomorphisms. Furthermore, take any open $W \subset \tilde{U}$. Then, consider the diagram,
\begin{center}
\begin{tikzcd}[column sep = large, row sep = large]
\F(\tilde{U}) \arrow[d, "\res_{W, \tilde{U}}"'] \arrow[r, "f_{\tilde{U}}"] & \G(\tilde{U}) \arrow[d, "\res_{W, \tilde{U}}"] 
\\
\F(W) \arrow[r, "f_{W}"'] & \G(W)
\end{tikzcd}
\end{center}  
Given a cover $V_i$ of $U$ we get covers $\tilde{V}_i = \tilde{U} \cap V_i$ of $\tilde{U}$ and $W_i = W \cap V_i = W \cap \tilde{V}_i$ of $W$. For any section $s \in \F(\tilde{U})$ consider $f_{W}(s|_W)$ which is the lift of $(f_i)_{W_i}(\res_{W_i, W} (s|_W))$ to $\G(W)$. However,
\[ \res_{W_i, W}(s|_W) = \res_{W_i, W} \circ \res_{W, \tilde{U}}(s) = \res_{W_i, \tilde{U}}(s) = \res_{W_i, \tilde{V}_i} \circ \res_{\tilde{V}_i, \tilde{U}}(s) = \res_{W_i, \tilde{V}_i}(s_i) \]
Therefore, using the naturality of $f_i$ on subsets of $V_i$,
\[ (f_i)_{W_i}(\res_{W_i, W} (s|_W)) = (f_i)_{W_i}(\res_{W_i, \tilde{V}_i}(s_i)) = \res_{W_i, \tilde{V}_i} \circ (f_i)_{\tilde{V}_i}(s_i) \]
Furthermore, we know that the sections $(f_i)_{\tilde{V}_i}(s_i)$ lift to $f_{\tilde{U}}(s)$. Thus,
\[ (f_i)_{W_i}(\res_{W_i, W} (s|_W)) = \res_{W_i, \tilde{V}_i} \circ \res_{\tilde{V}_i, \tilde{U}} \circ f_{\tilde{U}}(s) = \res_{W_i, W} \circ (\res_{W, \tilde{U}} \circ f_{\tilde{U}}(s)) \]
Therefore, the sections which lift to $f_W(s|_W)$ (i.e. the restrictions of $f_W(s|_W)$ to $W_i$) are exactly the restrictions of $\res_{W, \tilde{U}} \circ f_{\tilde{U}}(s)$. By the sheaf property of $\G$, gluing is unique so $f_W(s|_W) = \res_{W, \tilde{U}} \circ f_{\tilde{U}}(s)$. Thus locality gives,
\[ f_W \circ \res_{W, \tilde{U}} = \res_{W, \tilde{U}} \circ f_{\tilde{U}} \]
Therefore the morphisms $f_i$ glue to a unique $f \in \shHom{}{\F}{\G}(U) = \Hom{}{\F|_U}{\G|_U}$ so $\shHom{}{\F}{\G}$ is a sheaf. 


\subsubsection{1.16}

\begin{enumerate}
\item[(a)] Let $X$ be an irreducible space and $\underline{A}$ a constant sheaf on $X$. Take any open sets $V \subset U \subset X$. By Lemmas \ref{open_of_irreducible} and \ref{irreducible_implies_connected} the sets $V$ and $U$ are connected. Therefore, any continuous map $f : V \to A$ (with $A$ given the discrete topology) is constant (since the only connected sets in the discrete topology are points) so $f : V \to A$ is the restriction of the corresponding constant map $\tilde{f} : U \to A$. Therefore, the restriction map $\res_{V,U} : \underline{A}(U) \to \underline{A}(V)$ is surjective. Thus, the constant sheaf $\underline{A}$ is flasque.  

\item[(b)] Consdier the exact sequence of sheaves over $X$,
\begin{center}
\begin{tikzcd}
0 \arrow[r] & \F \arrow[r, "f"] & \G \arrow[r, "g"] & \H \arrow[r] & 0
\end{tikzcd}
\end{center}
where $\F$ is flasque. For an open set $U \subset X$, applying the left-exact functor $\Gamma(U, -)$ we get an exact sequence,
\begin{center}
\begin{tikzcd}
0 \arrow[r] & \F(U) \arrow[r, "f_U"] & \G(U) \arrow[r, "g_U"] & \H(U)
\end{tikzcd}
\end{center}
It suffices to show that the map $\G(U) \to \H(U)$ is surjective. For each $x \in U$, consider the induced maps on stalks, 
\begin{center}
\begin{tikzcd}
0 \arrow[r] & \F(U) \arrow[d] \arrow[r, "f_U"] & \G(U) \arrow[r, "g_U"] \arrow[d] & \H(U) \arrow[d]
\\
0 \arrow[r] & \F_x \arrow[r, "f_x"] & \G_x \arrow[r, "g_x"] & \H_x \arrow[r] & 0 
\end{tikzcd}
\end{center}
For any section $s \in \H(U)$ its inclusion in the stalk $\H_x$ lifts to $t_x \in \G_x$. Therefore, there exists some open $W$ nbd. of $x$ such that $t_x \in \G(W)$ maps to $s|_{W} \in \H(W)$.  
\bigskip\\
Consder the poset $\mathcal{T}$ of pairs $(V, t)$ where $V \subset U$ is open, $t \in \G(V)$, and $g_V(t) = s|_V$. The ordering is $(V, t) \le (V', t')$ if and only if $V \subset V'$ and $t'|_V = t$. To apply Zorn's lemma, consider a totally ordered subset $(V_\alpha, t_\alpha) \subset \mathcal{T}$ with totally ordered index set $\alpha \in I$. Then take,
\[ V = \bigcup_{\alpha \in I} V_\alpha \]
and the unique $t$ which glues all $t_{\alpha}$ by the sheaf condition of $\G$. Such a gluing exists because for $\alpha < \alpha'$ we have $V_\alpha \subset V_{\alpha'}$ and $t_{\alpha'}|_{V_{\alpha}} = t$ where $V_{\alpha} \cap V_{\alpha'} = V_{\alpha}$ so these sections agree on the overlap. 
\bigskip\\
Now, by Zorn's lemma, there exists a maximal element $(V, t)$ in $\mathcal{T}$. It suffices to show that $V = U$ since then $g_U(t) = s$. For each $x \in U$ we have $(W, t_x) \in \mathcal{T}$ from before. Then,
\begin{align*}
g_{W \cap V}(t_x|_{W \cap V} - t|_{W \cap V}) & =  \res^\H_{W \cap V, W} \circ g_{W}(t_x) - \res^\H_{W \cap V, V} \circ g_V(t)
\\
& = \res^\H_{W \cap V, W}(s |_{W}) - \res^\H_{W \cap V, V}(s|_V) =  s|_{W\cap V} - s|_{W \cap V} = 0
\end{align*}
Therefore, the section $d = t_x|_{W \cap V} - t|_{W \cap V}$ lies in the image of $f_{W \cap V}$ and thus lifts to $q \in \F(W \cap V)$. 
\begin{center}
\begin{tikzcd}
0 \arrow[r] & \F(W) \arrow[d, two heads] \arrow[r, "f_W"] & \G(W) \arrow[r, "g_W"] \arrow[d] & \H(W) \arrow[d]
\\
0 \arrow[r] & \F(W \cap V) \arrow[r, "f_{W \cap V}"] & \G(W \cap V) \arrow[r, "g_{W \cap V}"] & \H(W \cap V)
\end{tikzcd}
\end{center}
Because $\F$ is flasque, the section $q$ lifts to $q' \in \F(W)$. Now, \[ \res^\F_{W \cap V, W} \circ f_W(q') = f_{W \cap V} \circ \res^\G_{W \cap V, W}(q') = f_{W \cap V}(q) = d \]
Therefore, 
\[ \res^\G_{W\cap V, W} (t_x - f_W(q')) = t_x |_{W \cap V} - d = t_{W \cap V} \]
Thus $t_x - f_W(q') \in \G(W)$ and $t \in \G(V)$ agree on the overlap and thus glue to a section $t' \in \G(W \cup V)$ by the sheaf property of $\G$. Furthermore, let $s' = g_{W \cup V}(t') \in \F(W \cup V)$. Then by exactness,
\[ s'|_W = \res^\H_{W, W \cup V} \circ g_{W \cup V}(t') = g_W(t'|_W) = g_W(t_x - f_W(q')) = g_W(t_x) = s|_W \]
and likewise,
\[ s'|_V = \res^\H_{V, W \cap V} \circ g_{W \cup V}(t') = g_V(t'|_V) = g_V(t) = s|_V \]
Then $g_{W \cup V}(t') = s' = s|_{W \cup V}$ since they restrict to the same sections on the open cover $W, V$ of $W \cup V$ so $(W \cup V, t') \in \mathcal{T}$. However, $W \cup V \supset W$ and, by construction, $t'|_V = t$  contradicting the maximality of $(V, t)$ unless $V = W \cup V$ i.e. $W \subset V$. Since $W$ was, by construction, a neighborhood of $x$, then for each $x \in U$ we have $x \in V \subset U$ so $V = U$ proving the claim.


\item[(c)] Suppose that,
\begin{center}
\begin{tikzcd}
0 \arrow[r] & \F \arrow[r, "f"] & \G \arrow[r, "g"] & \H \arrow[r] & 0
\end{tikzcd}
\end{center}
is an exact sequence of sheaves over $X$ with $\F$ and $\G$ flasque. Now for any open sets $V \subset U \subset X$, consider the commuative diagram,
\begin{center}
\begin{tikzcd}
0 \arrow[r] & \F(U) \arrow[d, two heads, "\res^\F_{V,U}"'] \arrow[r, "f_U"] & \G(U) \arrow[r, "g_U"] \arrow[d, two heads, "\res^\G_{V,U}"'] & \H(U) \arrow[d, "\res^\H_{V,U}"'] \arrow[r] & 0
\\
0 \arrow[r] & \F(V) \arrow[r, "f_{V}"] & \G(V) \arrow[r, "g_{V}"] & \H(V) \arrow[r] & 0
\end{tikzcd}
\end{center}
where the rows are exact by part (b) since $\F$ is flasque and the first two downward maps are surjective because $\F$ and $\G$ are flasque. Given a section $s \in \H(V)$ we can lift $s$ under $g_V$ (which is a surjection since $\F$ is flasque) and under $\res^\G_{V,W}$ (which is a surjection since $\G$ is flasque) to get a section $s' \in \G(U)$. By the commutativity of the diagram,
\[ \res^\H_{V, U} \circ g_U(s') = g_V \circ \res^\G_{V,U}(s') = s \]
Therefore the restriction map $\res^\H_{V, U} : \H(U) \to \H(V)$ is surjective so $\H$ is flasque. 

\item[(d)] Let $f : X \to Y$ be a continuous map and $\F$ a flasque sheaf on $X$. Then consider the sheaf $f_* \F$ on $Y$. For open sets $V \subset U \subset Y$, we have restriction maps,
\[ \res^{f_* \F}_{V,U} : f_* \F(U) \to f_* \F(V) \quad \text{given by} \quad \res^{\F}_{f^{-1}(V), f^{-1}(U)} : \F(f^{-1}(U)) \to \F(f^{-1}(V)) \]
which is surjective since $\F$ is flasque. Therefore, $f_* \F$ is flasque. 

\item[(e)] Let $\F$ be a sheaf on $X$. Consider the sheaf $\G$ constructed by sending open sets $U \subset X$ to the maps,
\[ s : U \to \coprod_{x \in U} \F_x \quad \text{such that} \quad \forall x \in U : s(x) \in \F_x \]
or equivalently,
\[ U \mapsto \prod_{x \in U} \F_x \]
This sheaf is globally,
\[ \G = \prod_{x \in X} (\iota_x)_*(\F_x) \]
where $\iota_x : \{x\} \to X$ is the inclusion of the point and $\F_x$ is viewed as a constant sheaf over $\{ x \}$. For open sets $V \subset U \subset X$, consider the restriction maps,
\[ \res^\G_{U, V} : \G(U) \to \G(V) \quad \text{given by} \quad \prod_{x \in V} \pi_x : \prod_{x \in U} \F_x \to \prod_{x \in V} \F_x \]
Clearly, this map is surjective so $\G$ is flasque. Furthermore, consider the canonical morphism $\F \to \G$ which is locally
\[ \F(U) \to \prod_{x \in U} \F_x \]
defined by mapping $s \in \F(U)$ to its image in the stalk at each $x \in U$. Suppose that $s \in \F(U)$ maps to zero under this canonical map i.e. that the image of $s$ in $\F_x$ is zero at each $x \in U$. Then there exists an open neighborhood of each $x \in U$ on which $s$ restricts to zero. Thus by locality of the sheaf $\F$ we have $s = 0$ since their restrictions are equal on an open cover of $U$. 
\end{enumerate}

\begin{proposition}
Flasque abelian sheaves on a space $X$ are $\Gamma(X, -)$-acyclic.
\end{proposition}

\begin{proof}
Let $\F$ be a flasque abelian sheaf on $X$. Since the category of abelian sheaves on $X$ has enough injectives we may form an exact sequence of sheaves on $X$,
\begin{center}
\begin{tikzcd}
0 \arrow[r] & \F \arrow[r] & \I \arrow[r] & \G \arrow[r] & 0
\end{tikzcd}
\end{center} 
where $\I$ is injective. Now both $\F$ and $\I$ are flasque so $\G$ is also flasque. Since $\F$ is flasque, applying the functor $\Gamma(X, -)$ we get an exact sequence,
\begin{center}
\begin{tikzcd}
0 \arrow[r] & \Gamma(X, \F) \arrow[r] & \Gamma(X, \I) \arrow[r] & \Gamma(X, \G) \arrow[r] & 0
\end{tikzcd}
\end{center} 
Furthermore, applying the long exact cohomology sequence we get,
\begin{center}
\begin{tikzcd}[column sep = small]
0 \arrow[r] & \Gamma(X, \F) \arrow[r] & \Gamma(X, \I) \arrow[r] & \Gamma(X, \G) \arrow[r] & H^1(X, \F) \arrow[r] & H^1(X, \I) \arrow[draw=none]{d}[name=Z, shape=coordinate]{} \arrow[r] & H^1(X, \G)
\arrow[dlllll,
rounded corners, crossing over,
to path={ -- ([xshift=2ex]\tikztostart.east)
|- (Z) [near end]\tikztonodes
-| ([xshift=-2ex]\tikztotarget.west)
-- (\tikztotarget)}]
\\ 
& H^2(X, \F) \arrow[r] & H^2(X, \I)  \arrow[r] & H^2(X, \G) \arrow[r] & H^3(X, \F) \arrow[r] & H^3(X, \I) \arrow[r] & H^3(X, \G) \arrow[r] & \cdots
\end{tikzcd}
\end{center}
Since $\I$ is an injective sheaf, $H^r(X, \I) = 0$ for $r > 0$ which gives an exact sequence,
\begin{center}
\begin{tikzcd}
0 \arrow[r] & \Gamma(X, \F) \arrow[r] & \Gamma(X, \I) \arrow[r] & \Gamma(X, \G) \arrow[r] & H^1(X, \F) \arrow[r] & 0 
\end{tikzcd}
\end{center}
and isomorphisms $H^r(X, \G) \cong H^{r+1}(X, \F)$ for $r > 0$. Combining this exact sequence with the earlier one derived from the flasqueness of $\F$ shows that the cokernel of $\Gamma(X, \I) \to \Gamma(X, \G)$ is zero and thus $H^1(X, \F) = 0$. Since $\G$ is also a flasque sheaf on $X$ we can use the isomorphisms $H^{r+1}(X, \F) \cong H^r(X, \G)$ for $r > 0$ to show that $H^r(X, \F) = 0$ for all $r > 0$ by induction.  
\end{proof}

\begin{proposition}
Let $(X, \struct{X})$ be a ringed space. The derived functors of $\Gamma(X, -)$ computed over the category $\Ab(X)$ of sheaves of abelian groups on $X$ and those computed over the category $\Mod{\struct{X}}$ of $\struct{X}$-modules agree. 
\end{proposition}

\begin{proof}
There are enough injectives in the category of $\struct{X}$-modules. Taking an injective resolution of $\struct{X}$-modules is a resolution  of flasque sheaves of abelian groups which we have shown computes the derived functors of $\Gamma(X, -)$ in the full category $\Ab(X)$ since flasque sheaves are acyclic. 
\end{proof}

\subsubsection{1.17}

Let $x \in X$ be some point and $\iota_x : \{ x \} \to X$ the inclusion. Then consider the sheaf $\iota_x(A) = (\iota_x)_*(\underline{A})$ where $\underline{A}$ is the constant sheaf on $\{ x \}$. Now for any open $U \subset X$, we have,
\[ \iota_x(A)(U) = \underline{A}(\iota_x^{-1}(U)) = \begin{cases}
A & x \in U
\\
0 & x \notin U
\end{cases} \] 
Now consider the stalks,
\[ \iota_x(A)_y = \lim_{y \in U} \iota_x(A)(Y) \]
If there exists some open $U$ containing $y$ but not $x$ then $\iota_x(A)_y = 0$. Otherwise, for any open with $y \in U$ then $x \in U$ so $\iota_x(A)(U) = A$ and thus $\iota_x(A)_y = A$. Furthermore, there exists such an open exactly when $y$ is not a limit point of $x$ and not equal to $x$ i.e. $y \notin \overline{\{ x \}}$. Therefore, 
\[ \iota_x(A)_y = \begin{cases}
A & y \in \overline{\{ x \}} 
\\
0 & y \notin \overline{\{ x \}}
\end{cases} \]

\subsubsection{1.18}

Let $f : X \to Y$ be a continuous map, $\F$ a sheaf on $X$ and $\G$ a sheaf on $Y$. The restriction maps define a map,
\[ \varinjlim_{V \supset f(U)} \F(f^{-1}(V)) \to \F(U) \]
since $f^{-1}(V) \supset U$ gives restriction maps $\F(f^{-1}(V)) \to \F(U)$ compatilbe with restricton. Sheafifying gives a natural map $f^{-1} f_* \F \to \F$. Furthermore, we can define a map $\F \to f_* f^{-1} \F$ as follows. Consider the sheafification map $(f^{-1} \F)^P \to f^{-1} \F$ giving $\F \to f_* (f^{-1} \F)^P \to f_* f^{-1} \F$. The first map is defined by,
\[ \F(U) \to \varinjlim_{V \supset f(f^{-1}(U))} \F(V) \]
given since $U \supset f(f^{-1}(U))$ and then take the inclusion map of the colimit. These maps are natural. We have produced two natural transformations, a unit $\eta : \id \to f_* f^{-1}$ and a counit $\epsilon : f^{-1} f_* \to \id$. Now we use the following proposition.

\begin{prop}
Let $F : \C \to \D$ and $G : \D \to \C$ be functors with unit and counit natural transformations $\eta : 1_{\C} \to G \circ F$ and $\epsilon : F \circ G \to \id_{\D}$ satisfying the coherence relations that,
\begin{center}
\begin{tikzcd}
F \arrow[r, "F \eta"] & F \circ G \circ F \arrow[r, "\epsilon F"] & F
\\
G \arrow[r, "\eta G"] & G \circ F \circ G \arrow[r, "G \epsilon"] & G
\end{tikzcd}
\end{center}
are the identity $\id_F$ and $\id_G$ respectively. 
Then $F$ is left-adjoint to $G$.
\end{prop}

\begin{proof}
We construct a natural transformation $\phi_{X,Y} : \Hom{\D}{F(X)}{Y} \to \Hom{\C}{X}{G(Y)}$ via sending,
\[ (f : F(X) \to Y) \mapsto (G(f) \circ \eta_X : X \to G\circ F(X) \to G(Y))\]
and a natural transformation $\psi_{X,Y} : \Hom{\C}{X}{G(Y)} \to \Hom{\D}{F(X)}{Y}$ given by sending,
\[ (g : X \to G(Y)) \mapsto (\epsilon_Y \circ F(g) : F(X) \to F \circ G(Y) \to Y) \]
I claim these are inverse to each other:
\[ f \mapsto G(f) \circ \eta_X \mapsto \epsilon_Y \circ FG(f) \circ F(\eta_X) \]  
However, $\epsilon : F \circ G \to \id_{\D}$ is a natural transformation so,
\begin{center}
\begin{tikzcd}
FG(F(X)) \arrow[d, "\epsilon_{F(X)}"] \arrow[r, "FG(f)"] & FG(Y) \arrow[d, "\epsilon_Y"]
\\
F(X) \arrow[r, "f"] & Y
\end{tikzcd}
\end{center}
commutes and therefore,
\[ \eta_Y \circ FG(f) \circ F(\eta_Y) = f \circ \eta_{F(X)} \circ F(\eta_X) = f \]
using the first coherence relation showing that $\psi _{X,Y} \circ \phi_{X,Y} = \id$. Furthermore, consider,
\[ g \mapsto \epsilon_Y \circ F(g) \mapsto G(\epsilon_Y) \circ GF(g) \circ \eta_X  \]
Hpwever, $\eta : \id_{\C} \to G \circ F$ is a natural transformation so,
\begin{center}
\begin{tikzcd}
X \arrow[r, "g"] \arrow[d, "\eta_X"] & G(Y) \arrow[d, "\eta_{G(Y)}"]
\\
GF(X) \arrow[r, "GF(f)"] & GF(G(Y))
\end{tikzcd}
\end{center}
commutes and therefore,
\[ G(\eta_Y) \circ GF(g) \circ \eta_X = G(\eta_Y) \circ \eta_{G(Y)} \circ g = g \]
by the second coherence relation showing that $\phi_{X,Y} \circ \psi_{X,Y} = \id$ and providing a natural isomorphism,
\begin{center}
\begin{tikzcd}
\Hom{\D}{F(X)}{Y} \arrow[r, "\phi_{X,Y}", shift left] & \Hom{\C}{X}{G(Y)} \arrow[l, "\psi_{X,Y}", shift left] 
\end{tikzcd}
\end{center}
\end{proof}
\noindent
Therefore it suffices to prove the coherence relations in our case. (DO THIS!)

\subsubsection{1.19}

Let $X$ be a topological space, $Z \subset X$ a closed subspace and $U = X \setminus Z$ open. Furthermore denote the inclusions $\iota : Z \to X$ and $j : U \to X$. 

\begin{enumerate}
\item Let $\F$ be a sheaf on $Z$. Then consider the sheaf $\iota_*(\F)$ on $X$. For $x \in Z$ we have, 
\[ (\iota_* \F)_x = \varinjlim_{x \in V} (\iota_* \F)(V) = \varinjlim_{x \in V} \F(V \cap Z) = \varinjlim_{x \in V \cap Z} \F(V \cap Z) = \F_x \]
where the equality holds because every open set of the subspace $Z$ is of the form $V \cap Z$  for some open $V \subset X$ and $x \in V \iff x \in V \cap Z$ since $x \in Z$.
For $x \notin Z$ then for any $x \in V \subset U$ we have $\iota_*(\F)(V) = \F(\varnothing) = 0$ so $(\iota_* \F)_x = 0$. 

\item Let $\F$ be a sheaf on $U$. Now consider the sheaf $j_! \F$ as the sheafification of the presheaf defined by,
\[ (j_! \F)^P(V) = \begin{cases}
\F(V) & V \subset U 
\\
0 & V \not\subset U
\end{cases} \]
The stalks of the sheaf $j_! \F$ are the same as those of the presheaf and thus may be computed as follows.
For $x \in U$ we have,
\[ (j_! \F)^P_x = \varinjlim_{x \in V} (j_! \F)^P(V) = \varinjlim_{x \in V \subset U} \F(V) = \F_x \]
because both direct limits satisfy the same universal properties. For $x \notin U$ then any open $V$ containing $x$ cannot be contained in $U$ so,
\[ (j_! \F)^P_x = \varinjlim_{x \in V} (j_! \F)^P(V) = 0 \]
Now suppose that $\G$ is some sheaf on $X$ such that $\G|_U = \F$ and for which $\G_x = \F_x$ for all $x \in U$ and $\G_x = 0$ otherwise. To prove that $\G = j_! \F$, it suffices to show that $\G = j_! (\G|_U)$ since $\G|_U = \F$ by assumption. Consider the inclusion map $(j_! \G|_U)^P \to \G$. Since $\G$ is a sheaf this inclusion factors uniquely through the sheafification as $(j_! \G |_U)^P \to j_! (\G|_U) \to \G$. By assumption, the inclusion $(j_! \G|_U)^P \to \G$ is an isomorphism on stalks since $\G_x = 0$ for $x \notin U$. Thus $j_! (\G|_U) \to \G$ is an isomorphism but $\G|_U = \F$ so we get an isomorphism $j_! \F \to \G$. 


\item Let $\F$ be a sheaf on $X$. By adjunction, there is a morphism $\F \to \iota_* \iota^* \F$.  By definition, $\iota^* \F = \F|_Z$ so we have a map $\F \to \iota_* (\F |_\Z)$ and the sheaf $\iota_* \iota^* \F$ has stalks,
\begin{align*}
(\iota_* \iota^* \F)_x & = 
\begin{cases}
(\iota^* \F)_x & x \in Z
\\
0 & x \notin Z
\end{cases}
\\
& = 
\begin{cases}
\F_x & x \in Z
\\
0 & x \notin Z
\end{cases}
\end{align*}
On stalks at $x \notin Z$ this gives $\F_x \to 0$ and on stalks at $x \in Z$ it gives the identity $\F_x \to \F_x$. Furthermore, we have shown there exists a map $j_! (\F |_U) \to \F$ above which is an isomorphism on stalks at $x \in U$ and is the map $0 \to \F_x$ on stalks at $x \notin U$. Thus consider the sequence,
\begin{center}
\begin{tikzcd}
0 \arrow[r] & j_! (\F |_U) \arrow[r] & \F \arrow[r] & \iota_* (\F|_Z) \arrow[r] & 0
\end{tikzcd}
\end{center}
On stalks at $x \in Z$ this sequence is,
\begin{center}
\begin{tikzcd}
0 \arrow[r] & 0 \arrow[r] & \F_x \arrow[r, "\id"] & \F_x \arrow[r] & 0
\end{tikzcd}
\end{center}
and on stalks at $x \notin Z$ i.e. $x \in U$ this sequence is,
\begin{center}
\begin{tikzcd}
0 \arrow[r] & \F_x \arrow[r, "\id"] & \F_x \arrow[r] & 0 \arrow[r] & 0
\end{tikzcd}
\end{center}
both of which are exact so the sequence of sheaves is exact.
\end{enumerate}

\subsubsection{1.20}

Let $Z \subset X$ be closed and $\F$ a sheaf on $X$. We say a section $s \in \F(X)$ has support in $Z$ if $\Supp{\F}{s} \subset Z$. In that case $s |_{X \setminus Z} = 0$ since for each $x \in X \setminus Z$ we have $x \notin \Supp{\F}{s}$ so $s_x = 0$ so $s |_{X \setminus Z} = 0$ by seperatedness. Conversely, if $s|_{X \setminus Z} = 0$ then for any $x \in X \setminus Z$ we have $s_x = 0$ so $x \notin \Supp{\F}{s}$ and thus $\Supp{\F}{s} \subset Z$. We denote the subgroup of $\Gamma(X, \F)$ of sections with support in $Z$ by $\Gamma_Z(X, \F)$.

\begin{enumerate}
\item Consider the presheaf $V \mapsto \Gamma_{Z \cap V}(V, \F|_V)$. Let $U \subset X$ be an open set and $\{ V_i \}$ be an open conver of $U$. Suppose that $s \in \Gamma_{Z \cap U}(U, \F|_U)$ is a section on $U$ with support in $Z \cap U$ such that $s|_{V_i} = 0$. Then since $\F$ is a sheaf $s = 0$. Furthermore, given sections $s_i \in \Gamma_{Z \cap V_i}(V_i, \F|_{V_i})$ with supports in $Z \cap V_i$ which agree on the overlaps, then since $\F$ is a sheaf, these sections glue to give $s \in \Gamma(X, \F)$. It suffices to prove that $s$ has support in $Z \cap U$. We know that $s |_{V_i} = s_i$ and thus for $x \in V_i$ we have $s_x = (s_i)_x$. Thus, 
\[ \Supp{\F}{s} = \bigcup_{i \in I} \Supp{\F}{s_i} \subset \bigcup_{i \in I} Z \cap V_i \subset Z \cap U \]
so this is a sheaf which we denote $\H^0_Z(\F)$. 
\item Let $U = X \setminus Z$ and $j : U \to X$ be the inclusion. Consider the map $\F \to j_* (\F |_U)$ given by adjunction. For a section $s \in \F(V)$ on some open set $V \subset X$ to be in the kernel we must have $s_x \mapsto 0$ at each stalk. Consider,
\begin{align*}
(j_* (\F |_U))_x = \varinjlim_{x \in V} \F|_U(U \cap V) = \varinjlim_{x \in V} \F(U \cap V) 
\end{align*}
Thus, if $x \in Z$ then $(j_* (\F |_U))_x = 0$. Otherwise, if $x \in U$, suppose that the map $\F_x \to (j_* (\F |_U))_x$ take $s_x \mapsto 0$. Then $s|_{U \cap V} = 0$ on some $V$ meaning that $s_x = 0$ since $x \in U \cap V$. Therefore, the map $\F_x \to (j_* (\F |_U))_x$ is injective for $x \in U$. Thus $s$ is in the kernel exactly when $s_x = 0$ for each $x \in U$ i.e. $\Supp{\F}{s} \subset Z$ so $\H^0_Z(\F)$ is the kernel of the map $\F \to j_* (\F|_U)$ making the following sequence exact,
\begin{center}
\begin{tikzcd}
0 \arrow[r] & \H^0_Z(\F) \arrow[r] & \F \arrow[r] & j_* (\F|_U) 
\end{tikzcd}
\end{center}
Furthermore, if $\F$ is flasque then the restriction map $\F(V) \to \F(U \cap V)$ is surjective meaning that the stalk maps $\F_x \to (j_* (\F |_U))_x$ are surjective which implies that the morphism of sheaves $\F \to j_* (\F |_U)$ is surjective.
\end{enumerate}

\subsubsection{1.21}

\subsubsection{1.22}

\subsection{2}

\subsubsection{2.2}

\begin{proposition}
Let $A$ be a ring. Then $A_\red = A / \nilrad{A}$ is reduced.
\end{proposition}

\begin{proof}
Take $f \in A$ then if $f^n \in \nilrad{A}$ then $f \in \sqrt{\nilrad{A}} = \nilrad{A}$ since $\nilrad{A} = \sqrt{(0)}$ is a radical idel. Thus, if $f^n = 0$ in $A_\red$ then $f = 0$ in $A_\red$. 
\end{proof}

\subsubsection{2.3}

\begin{enumerate}
\item 
Let $X$ be a reduced schemes. Take $x \in X$ and consider the stalk,
\[ \stalk{X}{x} = \varinjlim_{x \in U} \struct{X}(U) \]
Each $\struct{X}(U)$ is a reduced ring so if $f \in \stalk{X}{x}$ satisfies $f^n = 0$ then on some open neighborhood $x \in U$ we have $f^n = 0$ and thus $f = 0$ on $U$ which shows that $f = 0$ in $\stalk{X}{x}$. Conversely, if all stalks are reduced then for any open set $U \subset X$ conisder an element $f \in \struct{X}(U)$. If $f^n = 0$ then $f^n = 0$ in each stalk $\stalk{X}{x}$ at $x \in U$ which implies $f = 0$ since $\stalk{X}{x}$ is reduced. Thus $f = 0$ in $\struct{X}(U)$ so $X$ is reduced. Thus,
\[ X \text{ is reduced} \iff \forall x \in X : \stalk{X}{x} \text{ is reduced} \]

\item Let $(X, \struct{X})$ be a scheme. Let $(\struct{X})_\red$ be the sheaf associated to the presheaf $U \mapsto \struct{X}(U)_\red$. Consider the ringed space $X_\red = (X, (\struct{X})_\red)$ which is locally ringed because the stalks of $(\struct{X})_\red$ are $(\stalk{X}{x})_\red$ which are reduced rings. Furthermore, there is a morphism of locally ringed spaced $(\id_X, f^\#) : X_\red \to X$ where $f^\#$ is the natural morphism of sheaves induced by the sheafification of the morphism of presheaves $\struct{X}(U) \to \struct{X}(U)_\red$. This is indeed a morphism of locally ringed spaces because the induced map $\stalk{X}{x} \to (\stalk{X}{x})_\red$ is local. It suffices to show that $X_\red = (X, (\struct{X})_\red)$ is indeed a scheme. Let $U_i = \Spec{A_i}$ be an affine cover of $X$ then I claim that $\tilde{U}_i = \Spec{(A_i)_\red}$ is an affine cover of $X_{\red}$. Firstly, $A$ and $A_\red$ have the same prime ideals because all primes lie above $\nilrad{A}$ so $U_i = \tilde{U}_i$ as topological spaces. Furthermore, the structure sheaf $\struct{\Spec{(A_i)_\red}}$ has exactly the correct structure to be the unique sheaf $(\struct{\Spec{A_i}})_\red = (\struct{X})_\red|_{U_i}$. Therefore, this cover is affine. 
\bigskip\\
To be clever, define the sheaf of ideals $\mathcal{N}_X$ to be the kernel of the sheaf map $\struct{X} \to (\struct{X})_\red$ or alternatively the sheaf associated to the presheaf 
\[ \sN_X(U) = \nilrad{\struct{X}(U)} \] 
Then there is an exact sequence of sheaves,
\begin{center}
\begin{tikzcd}
0 \arrow[r] & \sN_X \arrow[r] & \struct{X} \arrow[r] & (\struct{X})_\red \arrow[r] & 0
\end{tikzcd}
\end{center}
Then $X_\red$ is a closed subscheme of $X$. 

\item Let $f : X \to Y$ be a morphism of schemes and assue that $X$ is reduced. Consider the cokernel diagram,
\begin{center}
\begin{tikzcd}[column sep = large, row sep = large]
\sN_Y \arrow[r] & \struct{Y} \arrow[d, "\iota^\#"'] \arrow[r, "f^\#"] & f_* \struct{X}
\\
& (\struct{Y})_\red \arrow[ru, dashed, "g^\#"']
\end{tikzcd}
\end{center}
The top row composes to zero because on the stalks $(\sN_Y)_y \to (f_* \struct{X})_y$ the ring $(\sN_Y)_y = \nilrad{\stalk{Y}{y}}$ which only contains nilpotent elements. Furthermore, $X$ is reduced so $(f_* \struct{X})_y$ is a limit of reduced rings and thus reduced. Thus the image of $f$ is nilpotent in $(f_* \struct{X})_y$ and therefore zero. Thus the map of sheaves $\sN_Y \to f_* \struct{X}$ is zero so it factors through the cokernel $(\struct{Y})_\red$ uniquely as $g^\# : (\struct{Y})_\red \to f_* \struct{X}$. Therefore, the morphism of schemes $f : X \to Y$ factors via a unique morphism $g : X \to Y_\red$ with $g = (f, g^\#)$ through the closed immersion $Y_\red \to Y$.
\end{enumerate}

\subsubsection{2.16}

Let $X$ be a locally ringed space and $f \in \Gamma(X, \struct{X})$. Define,
\[ X_f = \{ x \in X \mid f_x \notin \m_x \subset \stalk{X}{x} \} \]
\begin{enumerate}
\item For any $x \in X_f$ then $f_x \notin \m_x$ so $f_x \in \stalk{X}{x}^\times$ since the stalk is a local ring. Therefore, there exists some open $U  \subset X$ with $x \in U$ such that $f|_U$ is invertible $g \cdot f|_U = 1$. Under the restriction to $\stalk{X}{y}$ for any point $y \in U$ we have $g_y \cdot f_y = 1$ so $f_y \in \stalk{X}{y}$ is invertible and thus $f_y \notin \m_y$. Therefore, $x \in U \subset X_f$ so $X_f$ is open. Furthermore, since inverses are unique, the inverses of $f|_{U_x}$ for each $x \in X_f$ agree on overlaps and thus glue to an inverse of $f|_{X_f}$.
\bigskip\\
Furthermore, let $X$ be a scheme and $U = \Spec{B}$ be an affine open subscheme $U \subset X$ with $f|_U = \bar{f} \in \struct{X}(U) = B$. 
Consider,
\[ U \cap X_f = \{ \p \subset B \mid \bar{f} \notin \p B_\p \} \]
However, if $\bar{f} \in \p$ then $\bar{f} \in \p B_\p$ if $\bar{f} \notin \p$ then $\bar{f} \in B_\p^\times$ so $\bar{f} \notin \p B_\p$. Thus,
\[ U \cap X_f = \{ \p \subset B \mid \bar{f} \notin \p B_\p \} = \{ \p \in \Spec{B} \mid \bar{f} \notin \p \} = D(\bar{f}) \]
which is open in $U = \Spec{B}$. Thus we see again that $X_f$ is open.

\item Let $X$ be a quasi-compact scheme and $A = \Gamma(X, \struct{X})$. Take $a \in A$ such that $a|_{X_f} = 0$. Now take an affine open $U \subset X$ with $U = \Spec{B}$ and consider $a |_{U \cap X_f} = 0$ i.e. $\bar{a} |_{D(\bar{f})} = 0$. Therefore, $\bar{a} \in \struct{X}(U \cap X_f) = \struct{\Spec{B}}(D(\bar{f})) = B_{\bar{f}}$ is zero so $\bar{f}^n \bar{a} = 0$ for some $n$. Thus, on each affine open $U$ there is some $n$ such that $(f^n a)|_U = 0$. Now since $X$ is quasi-compact we may take a finite affine cover $\{ U_i \}$ of $X$ such that $(f^{n_i} a)|_{U_i} = 0$. Let $N = \max_i n_i$, which exists by the finiteness of the cover, such that $(f^N a)|_{U_i} = 0$ for each open $U_i$ implying that $f^N a = 0$.  

\item Suppose that $X$ has a finite affine open cover $\{ U_i \}$ with $U_i = \Spec{B_i}$ such that $U_i \cap U_j$ is quasi-compact. Let $b \in \Gamma(X_f, \struct{X_f})$. Now $b|_{U_i \cap X_f} \in (B_i)_{\bar{f}}$ and thus there exists $n_i$ such that $\bar{f}^{n_i}(b|_{U_i \cap X_f})$ is in the image of $B_i$. By finiteness of the cover, $n = \max_i n_i$ exists such that we may take $(f^{n} b)|_{U_i \cap X_f} = b_i |_{U_i \cap X_f}$ for some $b_i \in B_i$ i.e. some section $b_i \in \struct{X}(U_i)$. Now consider, $s_{ij} = (b_i - b_j)|_{U_i \cap U_j}$ which satisfies,
\[ s_{ij} |_{U_i \cap U_j \cap X_f} = b_i|_{U_i \cap U_j \cap X_f} - b_j|_{U_i \cap U_j \cap X_f} = (f^{n} b)|_{U_i \cap U_J \cap X_f} - (f^{n} b)|_{U_i \cap U_j \cap X_f} = 0 \]
By the quasi-compactness of $U_i \cap U_j$ we may apply the previous part to get some $n_{ij}$ such that $f^{n_{ij}} |_{U_i \cap U_j} s_{ij} = 0$. Using the finiteness of the cover again, we may take $m = \max_{ij} n_{ij}$ to find that $f^m |_{U_i \cap U_j} s_{ij} = 0$ and thus,
\[ (f^m b_i - f^m b_j) |_{U_i \cap U_j} = 0 \]
Therefore, the sections $f^m b_i \in B_i$ agree on overlaps and thus glue to a global section $a \in A = \Gamma(X, \struct{X})$. Furthermore, 
\[ \res_{U_i \cap X_f, X_f}(a|_{X_f}) = (f^m b_i)|_{U_i \cap X_f} = (f^{n+m} b)|_{U_i \cap X_f} \]
since $U_i \cap X_f$ is an open cover of $X_f$ we find that $a|_{X_f} = f^{n + m} b$ so $f^{n + m} b$ has a lift to a global section. 

\item With the above hypothesis, consider the restricton map 
\[ \res_{X_f, X} : \Gamma(X, \struct{X}) \to \Gamma(X_f, \struct{X_f}) \] 
under which $f$ is mapped to a unit. 
Therefore, this map factors uniquely through the localization, 
\[ r : A_f \to \Gamma(X_f, \struct{X_f}) \]
However, if $\res_{X_f,X}(a) = 0$ then $f^n a = 0$ for some $n$ i.e. $a = 0$ in $A_f$ so $\ker{r} = 0$. Furthermore, for any $b \in \Gamma(X_f, \struct{X_f})$ there is some $n$ such that $f^n b = \res_{X_f, X}(a)$ for $a \in A$. Thus, 
\[ r(a/f^n) = \res_{X_f, X}(a) / f^n = f^n b / f^n  = b \]
so $r$ is surjective making  $r$ an isomorphism giving $\Gamma(X_f, \struct{X_f}) \cong A_f$. Therefore,
\[ \Gamma(X_f, \struct{X_f}) \cong \Gamma(X, \struct{X})_f \]
\end{enumerate}

\subsubsection{2.17}

\begin{enumerate}
\item Suppose that $f : X \to Y$ is a morphism of schemes such that $Y$ can be covered by open subsets $U_i$ such that for each $i$, the induced map $f_i : f^{-1}(U_i) \to U_i$ is an isomorphism. Let $g_i : U_i \to f^{-1}(U_i)$ be its inverse. Note that on the overlaps $f_i$ and $f_j$ agree,
\[ f_i |_{f^{-1}(U_i \cap U_j)} = f_j |_{f^{-1}(U_i \cap U_j)} = f |_{f^{-1}(U_i \cap U_j)}  \]
therefore, by uniqueness inverses, we know that the maps $g_i$ also agree on overlaps,
\[ g_i |_{U_i \cap U_j} = g_j |_{U_i \cap U_j} \]
Therefore, these functons glue to give a map $g : Y \to X$ such that $g_i = g |_{U_i}$. Now consider,
\[ (g \circ f)|_{f^{-1}(U_i)} = g |_{u_i} \circ f |_{f^{-1}(U_i)} = g_i \circ f_i = \id_{f^{-1}(U_i)} \]
and likewise,
\[ (f \circ g)|_{U_i} = f|_{f^{-1}(U_i)} \circ g|_{U_i} = f_i \circ g_i = \id_{U_i} \]
Therefore $g \circ f = \id_X$ and $g \circ f = \id_Y$ since these functions are locally the identity.

\item Let $X$ be a scheme and $A = \Gamma(X, \struct{X})$. Suppose that $f_1, \dots, f_n \in A$ generate the unit ideal and further suppose that the open subsets $X_{f_i}$ are affine. First, the open sets $X_{f_i}$ cover $X$ since if $x \notin X_{f_i}$ then $f_i \in \m_x$ however $f_i$ generate the unit ideal so we cannot have $f_i \in \m_x$ for all $i$ so $x \in X_{f_i}$. There is a natural map $a : X \to \Spec{A}$ via adjunction of the identity on global sections. Consider the open cover $U_i = D(f_i)$ of $\Spec{A}$. For each open $U = U_i$ and $f = f_i$ consider the restriction of the map, $a : a^{-1}(U) \to U = D(f) = \Spec{A_{f}}$. Recall that $a(x) = \id^{-1} \circ \res^{-1}_x(\m_x) \in \Spec{A}$ so,
\[ f \in a(x) \iff f \in \res^{-1}_x(\m_x) \iff f_x \in \m_x \]
and therefore,
\[ x \in a^{-1}(U) \iff a(x) \in D(f) \iff f \notin a(x) \iff f_x \notin \m_x \iff x \in X_f \]
Thus, $a^{-1}(D(f)) = X_f$. However, by assumption, $X_{f}$ is an affine scheme so the map $a : X_{f} \to \Spec{A_{f}}$ is determined uniquely by the ring map on global sections $r : A_f \to \Gamma(X_f, \struct{X_f})$ which we have shown is an isomorphism. Thus $a : X_{f} \to \Spec{A_{f}}$ is an isomorphism of affine schemes for each $f$. Applying part (a) we find that $a : X \to \Spec{A}$ is an isomorphism so $X$ is affine.
\bigskip\\
Conversely, if $X$ is an affine scheme $X = \Spec{A}$ then take $f = 1 \in A$ which generates the unit ideal and $X_f = D(f) = \Spec{A}$ satisfying the criterion.
\end{enumerate}

\subsubsection{2.18}

\begin{enumerate}
\item Let $A$ be a ring and $X = \Spec{A}$ and $f \in A$. Then,
\[ f \in \nilrad{A} \iff \forall \p \in \Spec{A} : f \in \p \iff D(f) = 0 \]

\renewcommand{\P}{\mathfrak{P}}

\item Let $\varphi : A \to B$ be a homomorphism of rings, $X = \Spec{A}$ and $Y = \Spec{B}$, and let $f : Y \to X$ be the induced morphism of affine schemes. Suppose that $\varphi : A \to B$ is injective. Then the sheaf map $f^\# : \struct{X} \to f_* \struct{Y}$ on the standard open $D(g)$ is the map $A_g \to B_{f(g)}$ which is injective since $a / g^n \mapsto f(a) / f(g)^n$ is zero exactly when $f(g)^k f(a) = 0$ for some $k$ but $f(g)^k f(a) = f(g^k a) = 0$ thus $g^k a = 0$ by injectivity meaning that $a / g^n = 0$ in $A_g$. Therefore, the morphism of sheaves $f^\# : \struct{X} \to f_* \struct{Y}$ is injective. We may also check this on the stalks. For $\p \in \Spec{A}$ consider the stalk map $f^\#_\p : \stalk{X}{\p} \to (f_* \struct{Y})_\p$. Now,
\[ (f_* \struct{Y})_\p = \varinjlim_{g \notin \p} \struct{Y}(f^{-1}(D(g)) = \varinjlim_{g \notin \p} \struct{Y}(D(f(g)) = \varinjlim_{g \notin \p} B_{f(g)} = B_\p \]
where $B_\p = S_\p^{-1} B = B \otimes_A A_\p$. Since localization of $A$-modules is exact, the map $\varphi_\p : A_\p \to B_\p$ remains an injection. Thus the stalk maps are injections so $f^\# : \struct{X} \to f_* \struct{Y}$ is an injective morphism of sheaves.
\bigskip\\
Conversely, if $f^\# : \struct{X} \to f_* \struct{Y}$ is injective then it is injective on sections so in particular $f^\# : \struct{X}(X) \to \struct{Y}(Y)$ which is the map $\varphi : A \to B$ is injective.
\bigskip\\
\begin{lemma}
$\varphi^{-1}(\sqrt{I}) = \sqrt{\varphi^{-1}(I)}$
\end{lemma}
\begin{proof}
\begin{align*}
x \in \varphi^{-1}(\sqrt{I}) & \iff \varphi(x) \in \sqrt{I} \iff \varphi(x)^n = \varphi(x^n) \in I
\\
&  \iff x^n \in \varphi^{-1}(I) \iff x \in \sqrt{\varphi^{-1}(I)} 
\end{align*}
\end{proof}
\begin{lemma}
$\overline{f(V(I))} = V(\varphi^{-1}(I))$.
\end{lemma}
\begin{proof}
Consider $f(V(I)) \subset V(J)$ then $J \subset \varphi^{-1}(\P)$ for each prime ideal $\P \subset B$ above $I$ so $\varphi(J) \subset \sqrt{I}$. By the above lemma, $J \subset \varphi^{-1}(\sqrt{I}) = \sqrt{\varphi^{-1}(I)}$ and thus,
\[ V(J) \supset V(\sqrt{\varphi^{-1}(I)}) = V(\varphi^{-1}(I)) \] Furthermore, if $\p \supset f(V(I))$ then $\p = \varphi^{-1}(\P)$ with $\P \supset I$ so then $\p \supset \varphi^{-1}(I)$ and thus $\p \in V(\varphi^{-1}(I))$. Thus $f(V(I)) \subset V(\varphi^{-1}(I))$ which proves that $\overline{f(V(I))} = V(\varphi^{-1}(I))$.
\end{proof}

\begin{corollary}
$\overline{f(Y)} = V(\ker{\varphi})$ so $f$ is dominant iff $\ker{\varphi} \subset \nilrad{A}$.
\end{corollary} 
Therefore, in this case, $\ker{\varphi} = 0$ so $f$ is dominant. 

\item If $\varphi : A \to B$ is surjective then the stalk map $\varphi : A_{\varphi^{-1}(\P)} \to B_\P$ is clearly surjective because any $s'$ mapping to $s \in B \setminus \P$ lies in $\varphi^{-1}(B \setminus \P) = A \setminus \varphi^{-1}(\P)$. Thus, the sheaf map $f^\# : \struct{Y} \to f_* \struct{X}$ is surjective. Furthermore, let $I = \ker{\varphi}$ then $f : Y \to X$ is a homeomorphism of $Y$ to the closed subspace $V(I) \subset X$ by the lattice isomorphism theorem. 

\begin{proposition}
Let $\varphi : A \to B$ be a surjective map of rings with $I = \ker{\varphi}$. Then the induced map $f : \Spec{B} \to \Spec{A}$ is a homeomorphism onto its image, the closed subspace $V(I) \subset \Spec{A}$.
\end{proposition}

\begin{proof}
Define the map $g : V(I) \to \Spec{B}$ via $\p \mapsto \varphi(\p)$. We must show that this map is well-defined and continuous. However, first note that because $\varphi$ is surjective that 
$g \circ f(\P) = \varphi(\varphi^{-1}(\P)) = \P$
and $f \circ g(\p) = \varphi^{-1}(\varphi(\p))$ but,
\[ x \in \varphi^{-1}(\varphi(\p)) \iff \varphi(x) \in \varphi(\p) \iff \exists y \in \p : \varphi(x) = \varphi(y) \iff x \in \p + I \]
so if $\p \supset I$ then $f \circ g(\p) = \varphi^{-1}(\varphi(\p)) = \p$. Thus, these maps are inverses as maps of subsets.
\bigskip\\
Let $\p \supset I$ is prime, then $\varphi(\p)$ is an ideal because $\varphi$ is surjective. Furthermore, if $f(x) \cdot f(y) \in \varphi(\p)$ then $f(xy) \in \varphi(\p)$ so $xy \in \varphi^{-1}(\varphi(\p)) = \p$ implying that $x \in \p$ or $y \in \p$ and thus $f(x) \in \varphi(\p)$ or $f(y) \in \varphi(\p)$. Therefore, $\varphi(\p) \subset B$ is a prime ideal so $g$ is well-defined. 
\bigskip\\
Take an ideal $J \subset B$ corresponding to the closed subset $V(J) \subset \Spec{B}$. Consider,
\[ \p \in g^{-1}(V(J)) \iff \varphi(\p) \in V(J) \iff \varphi(\p) \supset J \iff \p \supset \varphi^{-1}(J) \iff \p \in V(\varphi^{-1}(J)) \]
where I have used the fact that $f$ and $g$ are inclusion preserving inverses and $\p \in V(I)$. Thus, $g^{-1}(V(J)) = V(\varphi^{-1}(J))$ which is closed in $V(I)$ because $\varphi^{-1}(J)$ is an ideal of $A$ containing $I$ so $V(I) \cap V(\varphi^{-1}(J)) = V(\varphi^{-1}(J))$. Therefore, $g : V(I) \to \Spec{B}$ is a continuous inverse of $f : \Spec{B} \to V(I)$. 
\end{proof}

\item Let $f : Y \to X$ be a morphism of schemes such that $f^\# : \struct{X} \to f_* \struct{Y}$ surjective. Consider the ring maps,
\begin{center}
\begin{tikzcd}
A \arrow[rr, "\varphi"] \arrow[rd, "\pi"'] & & B 
\\
& A / \ker{\varphi} \arrow[ru, "\tilde{\varphi}"']
\end{tikzcd}
\end{center}
Then consider the scheme $X' = \Spec{A / \ker{\varphi}}$ and the induced morphism of affine schemes,
\begin{center}
\begin{tikzcd}
X   & & Y \arrow[ll, "f"'] \arrow[dl, "\tilde{f}"] 
\\
& X' \arrow[ul, "p"]
\end{tikzcd}
\end{center}
These morphisms of schemes give a morphism of sheaves on $X$,
\begin{center}
\begin{tikzcd}
\struct{X} \arrow[rd, "p^\#"'] \arrow[rr, two heads, "f^\#"] & & f_* \struct{Y}
\\
& p_* \struct{X'} \arrow[ru, "p_* \tilde{f}^\#"', hook, two heads]
\end{tikzcd}
\end{center}
By assumption $f^\# : \struct{X} \to f_* \struct{Y}$ is surjective so $p_* \tilde{f}^\# : p_* \struct{X'} \to f_* \struct{Y}$ is surjective as well. Furthermore, the ring map $\tilde{\varphi} : A / \ker{\varphi} \to B$ is injective meaning that $\tilde{f}^\#$ is an injective morphism of sheaves and, since $p_*$ is a right-adjoint functor, $p_* \tilde{f}^\#$ is also injective. Therefore, $p_* \tilde{f}^\# : p_* \struct{X'} \to f_* \struct{Y}$ is a bijection of sheaves over $X$ and, in particular, surjective on sections i.e. in the sense of pre-sheaves. Furthermore, $\pi : A \to A / \ker{\varphi}$ is a surjection and thus $p^\#$ is surjective on global sections. Thus, the composition $f^\# = p_* \tilde{f}^\# \circ p^\#$ is surjective on global sections i.e. $f^\# : \struct{X}(X) \to (f_* \struct{Y})(X) = \struct{Y}(Y)$ which is the map $\varphi : A \to B$ is surjective. 
\end{enumerate}

\subsubsection{2.19}

Let $A$ be a ring. Suppose that $\Spec{A}$ is disconnected so there exist disjoint nonempty closed sets $V(I_1), V(I_2) \supset \Spec{A}$. Therefore, 
\[ V(I_1) \cap V(I_2) = V(I_1 + I_2) = \varnothing \implies I_1 + I_2 = A \]
and likewise,
\[ V(I_1) \cup V(I_2) = V(I_1 I_2) = \Spec{A} \implies I_1 I_2 \subset \nilrad{A} \]
Therefore, there must exist elements $e_1 \in I_1$ and $e_2 \in I_2$ such that $e_1 + e_2 = 1$ and furthermore $e_1 e_2 \in \nilrad{A}$. Note that,
\[ (e_1 + e_2)^n = e_1^2 + n e_1 e_2^{n-1} + \cdots + n e_1^{n-1} e_2 + e_n^n = 1 \]
Therefore, $1 - (e_1^n + e_2^n) \in \nilrad{A}$ so $e_1^n  + e_2^n \in A^\times$ and let $u \in A^\times$ be its inverse. Since $e_1 e_2$ is nilpotent there exists some $n \ge 0$ such that $(e_1 e_2)^n = 0$. Now set $\tilde{e}_1 = u e_1^n$ and $\tilde{e}_2 = u e_2^n$. Thus $\tilde{e}_1 + \tilde{e}_2 = u(e_1^n + e_2^n) = 1$ and $\tilde{e}_1 \tilde{e}_2 = u^2 e_1^n e_2^n = u^2 (e_1 e_2)^n = 0$. Finally, consider,
\begin{align*}
\tilde{e}_1 & = 1 \cdot \tilde{e}_1 = (\tilde{e}_1 + \tilde{e}_2) \tilde{e}_1 = \tilde{e}_1^2 + \tilde{e}_1 \tilde{e}_2 = \tilde{e}_1^2
\\
\tilde{e}_2 & = 1 \cdot \tilde{e}_2 = (\tilde{e}_1 + \tilde{e}_2) \tilde{e}_2 = \tilde{e}_1 \tilde{e}_2 + \tilde{e}_2^2 = \tilde{e}_2^2
\end{align*}
so $e_1$ and $e_2$ are perpendicular idempotent generators proving (i) $\implies$ (ii). 
\bigskip\\
First, note that because $e_i$ is idempotent the ideal $(e_i)$ is actually a ring with identity element $e_i$ since $e_i \cdot (a e_i) = a e_i^2 = a e_i$. Now, consider the ring map $\Phi : A \to (e_1) \times (e_2)$ via $a \mapsto (a e_1, a e_2)$ which indeed maps $1 \mapsto (e_1, e_2)$ the identity. Now suppose that $\Phi(a) = 0$ then $a e_1 = a e_2 = 0$ so $a = 1 \cdot a = (e_1 + e_2) \cdot a = 0$. Thus $\Phi$ is injective. Furthermore, for any $(a e_1, b e_2) \in (e_1) \times (e_2)$ consider the element $a e_1 + b e_2 \in A$. Then, 
\[ \Phi(a e_1 + b e_2) = (a e_1^2 + b e_2 e_1, a e_1 e_2 + b e_2^2) = (a e_1, b e_2) \]
so $\Phi$ is surjective. Thus $\Phi : A \xrightarrow{\sim} (e_1) \times (e_2)$ is an isomorphism.
\bigskip\\
Finally, suppose that $A = A_1 \times A_2$. Then $A_1, A_2 \subset A$ are ideals such that $A_1 A_2 = 0$ and $A_1 + A_2 = A$. Therefore in $\Spec{A}$ we have closed subsets $V(A_1)$ and $V(A_2)$ such that $V(A_1) \cup V(A_2) = V(A_1 A_2) = \Spec{A}$ and $V(A_1) \cap V(A_2) = V(A_1 + A_2) = V(A) = \varnothing$. Therefore, $\Spec{A}$ is disconnected. 

\renewcommand{\P}{\mathbb{P}}

\subsection{3}

\subsubsection{3.1}

Let $f : X \to Y$ be a morphism such that there exist affine open coverings $U_i = \Spec{A_i}$ and $V_i = \Spec{B_i}$ such that $f : U_i \to V_i$ makes $\varphi : B_i \to A_i$ finite type. Let $\Spec{B} = V \subset Y$ be any affine open and consider $V \cap V_i$ which is open and thus covered by affines $D(f_{ij}) \subset V_i$ for $f_{ij} \in B_i$. Then, inside the affine $U_i$ we have $U \cap U_i$ covered by $D(\varphi(f_{ij})) = f^{-1}(D(f_{ij}))$. Therefore, $U$ is covered by the affine opens $D(f_{ij}) = \Spec{(A_i)_{f_{ij}}}$. Furthermore, consider the maps $B \to (A_i)_{f_{ij}}$. Since $(B_i)_{f_{ij}} \to (A_i)_{f_{ij}}$ are finite type it suffices to show that $B \to (B_i)_{f_{ij}}$ via the inclusion of affine schemes $D(f_{ij}) \subset U$ is finite type. This is clear since restriction maps are generated by localization maps $B \to B_f$ which are finite type. Therefore, for any affine $V \subset Y$ then $f^{-1}(V)$ is covered by affine opens such that the ring map induced by $f$ is finite type. 

\subsubsection{3.2}

Suppose that $f : X \to Y$ is locally finit type according to Hartshorne i.e. there exists an open affine cover $V_i = \Spec{B_i}$ such that $f^{-1}(V_i)$ is quasi-compact. Let $V \subset Y$ be affine then $V \cap V_i$ is covered by affine opens $D(f_{ij}) \subset V \cap V_i$ and by quasi-compactness there are finitely many $D(f_{ij})$ covering $V$. For each $V_i$ by quasi-compactness we can write $f^{-1}(V_i)$ as a finite union of affine opens $U_{ik} = \Spec{A_{ik}}$. Now, $f^{-1}(D(f_{ij}) ) \cap U_{ik} \subset U_{ik}$ is the affine open $(f|_{U_{ik}})^{-1}(D(f_{ij})) = D(\varphi(f_{ij}))$ for the map $\varphi : B_i \to A_{ik}$. Therefore, $f^{-1}(V)$ is the union of the finitely many affine opens $f^{-1}(D(f_{ij})) \cap U_{ik}$. Thus, $f^{-1}(V)$ is quasi-compact as the finite union of quasi-compacts.

\begin{definition}
A continuous map $f : X \to Y$ is quasi-compact if for any quasi-compact open $K \subset Y$ we have $f^{-1}(K)$ is quasi-compact. A morphism of schemes $f : X \to Y$ is quasi-compact if the underlying map of topologcial spaces is quasi-compact
\end{definition}  

\begin{theorem}
A morphism $f : X \to Y$ is quasi-compact iff the equivalent conditions above.
\end{theorem}

\begin{proof}
If $f : X \to Y$ is quasi-compact then for any affine open $V \subset Y$ we have $V$ is quasi-compact (since any open cover can be refined to $D(f_i)$ and $\bigcup D(f_i) = D(\sum (f_i) ) = D(1)$ so $\sum f_i$ generates the unit ideal so there must be some finite sum $f_{i_1} + \cdots + f_{i_n} = 1$ so $D(f_{i_1}) \cup \cdots \cup D(f_{i_n})$ is a finite subcover). Thus $f^{-1}(V)$ is quasi-comact. Coversely, suppose that any affine open $V \subset Y$ satisfies $f^{-1}(V)$ is quasi-compact. Then take any quasi-compact open $K \subset Y$ and consider $f^{-1}(K)$. By quasi-compactness we can write $K$ as a finite union of affine opens and then each has quasi-compact preimage. Thus $f^{-1}(K)$ is a finite union of quasi-compacts and thus is quasi-compact.
\end{proof}


\subsubsection{3.3}

\begin{definition}
Let $f : X \to Y$ be a morphism of schemes.
\begin{enumerate}
\item $f$ is finite type at $x \in X$ if there exist affine opens $\Spec{A} = U \subset X$ and $\Spec{B} = V \subset Y$ with $f(U) \subset V$ and $x \in U$ such that $B \to A$ is finite type
\item $f$ is locally finite type if it is finite type at each $x \in X$
\item $f$ is finite type if it is locally finite type and quasi-compact.
\end{enumerate}
\end{definition}

\begin{lemma}
A morphism $f : X \to Y$ is quasi-compact iff there exists an affine open cover $V_i \subset Y$  such that $f^{-1}(V_i)$ is the finite union of affine opens. 
\end{lemma}

\begin{proof}
If $f : X \to Y$ is quasi-compact then $f^{-1}(V)$ is quasi-compact for any affine open $V \subset Y$. Since affine opens form a base of the topology on $X$ the open $f^{-1}(V)$ is a union of affine opens which can by made finite by quasi-compactness. 
\bigskip\\
Conversely, if each $f^{-1}(V_i)$ is a finite union of affine opens then $f^{-1}(V_i)$ is quasi-compact so $f : X \to Y$ is quasi-compact by the above problem.
\end{proof}

\begin{enumerate}
\item If $f : X \to Y$ is finite type (see above) then it is quasi-compact by definition and for some affine open cover $V_i \subset Y$ we know $f^{-1}(V_i)$ can be covered by affine open $U_{ij}$ such that $U_{ij} \to V_i$ is finite type on rings. By quasi-compactness we can take the covering $U_{ij}$ of $f^{-1}(V_i)$ to be finite. Conversely, suppose $f : X \to Y$ is finite type according to Hartshorne then it is trivially locally finite type and $f^{-1}(V_i)$ is covered by finitely many affine opens and thus, by the lemma, is quasi-compact. 

\item If $f : X \to Y$ is finite type according to Hartshorne then we know it is locally finite type and by problem 3.2 we know for any affine open $V \subset Y$ we have $f^{-1}(V)$ is covered by affine opens $U_{i}$ such that $U_{i} \to V$ is finite type on rings. Furthermore, we have shown that $f$ is quasi-compact so $f^{-1}(V)$ is quasi-compact so we may take a finite subcover $U_i$ so $f^{-1}(V)$ is a finite union of affine opens with the finite type property. 

\item Let $f : X \to Y$ be locally of finite type according to Hartshorne and let $\Spec{B} = V \subset Y$ and $\Spec{A} = U \subset f^{-1}(V)$ be affine opens. Then consider $\varphi : B \to A$. We know that $f^{-1}(V)$ has a cover of affine opens $U_i = \Spec{A_i}$ with $B \to A_i$ finite type. Then consider $U \cap U_i$ which is open in $U_i$ and thus covered by principal opens $D(f_{ij}) = \Spec{(A_i)_{f_{ij}}}$. Now, $B \to A_i \to (A_i)_{f_{ij}} = A_i[f_{ij}^{-1}]$ is finite type. Now the maps $A \to (A_i)_{f_{ij}}$ are the restriction maps from $D(f_{ij}) \subset U$ which are finite type since they are localizations. Then consider,
\begin{center}
\begin{tikzcd}
B \arrow[r, "\varphi"] \arrow[rd, "\varphi"] & A \arrow[d]
\\
& (A_i)_{f_{ij}}
\end{tikzcd}
\end{center} 
Since $D(f_{ij})$ cover $U$ I claim that $B \to A$ is finite type. (PROVE THIS CLAIM)
\end{enumerate}

\subsubsection{3.4}

Let $f : X \to Y$ be finite and $\Spec{V} = V \subset Y$ be an affine open. Take an affine open cover $\Spec{B_i} = V_i \subset Y$ such that $f^{-1}(V_i) = U_i = \Spec{A_i}$ is an affine open cover of $X$ and $B_i \to A_i$ is finite. Consider $f^{-1}(V) \cap U_i = f^{-1}(V \cap V_i)$. Since $V_i$ is affine open we have $V \cap V_i$ covered by prinipal opens $D(f_{ij})$ and then $f^{-1}(V) \cap U_i = D(\varphi(f_{ij}))$ inside the affine open $U_i$. Thus replacing $X$ by $f^{-1}(V)$ we reduce to the case $a : X \to \Spec{B}$ where $D(f_i) \subset \Spec{B}$ is an open affine cover (which we may take to be finite) and $a^{-1}(D(f_i)) = X_{a^\#(f_i)} = \Spec{A_i}$ is affine open with $B_{f_i} \to A_i$ is finite. Then $f_1, \dots, f_n$ generate the unit ideal of $B$ since they cover $\Spec{B}$. Thus, $a^\#(f_1), \dots, a^\#(f_n) \in \Gamma(X, \struct{X})$ generate the unit ideal. Therefore, by 2.17, $X = \Spec{A}$ is affine with $A_i = A_{a^\#(f_i)}$. Now, the map $a^\# : B \to A$ localizes to $B_{f_i} \to A_{a^\#(f_i)}$ which is finite. Then by lemma \ref{finiteness_local} $B \to A$ is finite. 

\begin{lemma} \label{finiteness_local}
Let $\varphi : A \to B$ be a ring map such that for $f_1, \dots, f_n$ generating the unit ideal of $A$ the localized maps $\varphi : A_{f_i} \to B_{\varphi(f_i)}$ are finite then $\varphi$ is finite.
\end{lemma}

\begin{proof}
Let $x_{i1}, \dots, x_{in} \in B_{\varphi(f_i)}$ generate $B_{\varphi(f_i)}$ as an $A_f$-module. Multiplying by a suitable power of $f_i$ we may assume these elements lift to $B$. I claim that $\{ x_{ij} \}$ generate $B$ as an $A$-module. For any $b \in B$ we know that $f_i^{N_i} (a_1 \cdot x_{i1} + \cdots + a_n \cdot x_{in} - b)$ in $B$ for some $n_i$.  Now, $f_1^N, \dots, f_n^N$ generate the unit ideal of $B$ where $N = \max_i N_i$ so for each $i$ we get,
\[ f_i^N b \in A x_{i1} + \cdots + A x_{in} \]
and thus,
\[ b \in  \sum A x_{ij} \]
since $f_1^N, \dots, f_n^N$ generate the unit ideal so $\varphi : A \to B$ is finite. 
\end{proof}

\subsubsection{3.5}

\begin{definition}
We say that $f : X \to Y$ is quasi-finite if for each $y \in Y$ the set $f^{-1}(y)$ is finite.
\end{definition}

\begin{enumerate}
\item Let $f : X \to Y$ be finite. For any $y \in Y$ there must exist affine open sets $\Spec{A} = U \subset X$ and $y \in \Spec{B} = V \subset Y$ such that $U = f^{-1}(V)$ and $B \to A$ is finite. Then $f^{-1}(y) \subset U$ so it suffices to show that the set of primes above $\p \in \Spec{B}$ is finite. The fibre is $X_y = \Spec{A \otimes_B \kappa(y)} \to \Spec{\kappa(y)}$ which is finite then we use the fact that a finite-dimensional $k$-algebra has finitely many prime ideals. 
\item Let $f : X \to Y$ be finite and $Z \subset X$ be closed. Because $f$ is finite we can find an affine open cover $V_i = \Spec{B_i}$ of $Y$ such that $U_i = f^{-1}(V)$ is affine, $U_i = \Spec{A_i}$ and $\varphi_i : B_i \to A_i$ is finite. Then $Z \cap U_i$ is closed in $\Spec{A_i}$ so there is an ideal $I_i \subset A_i$ such that $Z \cap U_i = V(I_i)$. Now, I claim that finite ring maps induce closed maps on spectra. 
\bigskip\\
Consider $V(I) \subset \Spec{A}$ and $\varphi : B \to A$. Then consider $\varphi^*(V(I))$
We can reduce to the case $I = 0$ since $B \to A \to A / I$ is finite and $\varphi^*(V(I))$ is the image of $\Spec{A / I} \to \Spec{B}$. We can also reduce to $B \to A$ injective since the image of $\Spec{A}$ is contained in $\Spec{B / \ker{(B \to A)}}$ which is closed in $\Spec{B}$. Thus, take $B \to A$ injective and consider $\Spec{A} \to \Spec{B}$. Since $B \to A$ is finite it is integral so the going up property holds. Thus it suffices to show that minimal primes of $B$ are in the image. If $\p \in \Spec{B}$ is minimal then $B_\p$ has a unique prime ideal then the localization $B_\p \to A_\p$ is injective so $\Spec{A_\p}$ is nonempty and hits the unique prime $\p \in \Spec{B_\p}$ so $\p$ is in the image of $\Spec{A} \to \Spec{B}$. Since $\Spec{A} \to \Spec{B}$ hits all minimal primes and has going up then it must be surjective. 
\bigskip\\
Therefore $f(Z \cap U_i)$ is closed in $V_i = \Spec{A_i}$. Now, $y \in f(Z) \cap V_i$ if $y \in V_i$ and $\exists x \in Z$ such that $f(x) = y$ so $x \in Z \cap f^{-1}(V_i) = Z \cap U_i$. Furthermore, $f(Z \cap U_i) \subset f(Z) \cap f(U_i) \subset f(Z) \cap V_i$ so $f(Z) \cap V_i = f(Z \cap U_i)$ so $f(Z) \cap V_i$ is closed. Then, I claim that $f(Z)$ is closed. 
\bigskip\\
If $x \in f(Z)^C$ then for some $V_i$ we have $x \in V_i \setminus f(Z)$ is open and $x \in V_i \setminus f(Z) \subset f(Z)^C$ so $f(Z)$ is closed. 

\item Consider the map $\Gm^k \coprod \A^1_k \to \A^1_k$ via $k[x] \to k[x,x^{-1}]$ and the identity. This is clearly surjective and finitely generated since on rings it is,
\[ k[x] \to k[x, x^{-1}] \times k[x] \]
Furthermore, this map is quasi-finite since the fibers have at most two points. To see this, consider, $y = (x - a) \in \Spec{k[x]}$ then $\kappa(y) = k[x]/(x - a)$ and the fibre is,
\begin{align*}
X_y & = \Spec{(k[x, x^{-1}] \times k[x]) \otimes_{k[x]} k[x]/(x  - a)} 
\\
& = \Spec{k[x, x^{-1}]/(x - a) \times k[x] / (x - a)} 
\\
& = \Spec{k[x, x^{-1}/(x - a)} \coprod \Spec{k[x]/(x - a)} 
\\
& = 
\begin{cases}
\Spec{k} & a = 0
\\
\Spec{k} \coprod \Spec{k} & a \neq 0
\end{cases}
\end{align*}
However, this map is not closed since $\Gm^k \subset \Gm^k \coprod \A^1_k$ is closed but its image is $\A^1_k \setminus \{ 0 \}$ which is not closed. Thus the map cannot be finite. In particular,
\[ k[x, x^{-1}] = \bigoplus_{n \ge 0} x^{-n} k[x] \]
so $k[x, x^{-1}]$ is not a finitely-generated $k[x]$-module.  
\end{enumerate}

\subsubsection{3.6}

Let $X$ be an integral scheme. Then $X$ is irreducible so it has a unique generic point $\xi \in X$. Since $\xi$ is generic, all points are its limit points i.e. it lies in every nonempty open $U \subset X$. In particular, if $U = \Spec{A}$ is an affine open then $\xi \in U$ corresponding to $\p_\xi \subset A$ such that $V(\p_\xi) = \Spec{A}$. Since $X$ is integral, $A$ is a domain then $\stalk{X}{\xi} = A_{\p_\xi} = \Frac{A}$ is a field.

\subsubsection{3.7}

\begin{definition}
A morphism $f : X \to Y$ with $Y$ irreduclbe is \textit{generically finite} if $X_\eta$ is finite at the generic point $\eta \in Y$. 
\end{definition}

\begin{definition}
A morphism $f : X \to Y$ is \textit{dominant} if $f(X) \subset Y$ is dense.
\end{definition}

\begin{lemma}
Let $f : X \to Y$ be a morphism of irreducible schemes and let $\eta_X \in X$ and $\eta_Y \in Y$ be their generic points. Then $f$ is dominant iff $f(\eta_X) = \eta_Y$.
\end{lemma}

\begin{proof}
If $f(\eta_X) = \eta_Y$ then $\overline{f(\eta_X)} = \overline{\eta_Y} = Y$. Conversely, if $\overline{f(X)} = Y$ then since $f$ is continuous $f(\overline{A}) \subset \overline{f(A)}$ for any set $A$. Thus,
\[ f(X) = f(\overline{\eta_X}) \subset \overline{f(\eta_X)} \]
Thus, $\overline{f(\eta_X)} = \overline{f(X)} = Y$ so $f(\eta_X) = \eta_Y$ since it is a point whose closure is $Y$.
\end{proof}

Let $f : X \to Y$ be a dominant, generically finite, finite-type morphism of integral schemes. Let $\eta_X \in X$ and $\eta_Y \in Y$ be their generic points. Then $f(\eta_X) = \eta_Y$ so we get a map $f^\# : \stalk{Y}{\eta_Y} \to \stalk{X}{\eta_X}$ whch is an extension of residue function, $K(Y) \embed K(X)$. 
\bigskip\\
First, take affine opens $\Spec{B} = U \subset X$ and $\Spec{A} = V \subset Y$ with $f : U \to V$ then $A$ and $B$ are domains and $\varphi : A \to B$ is finite type so there is a surjecton $A[x_1, \dots, x_n] \twoheadrightarrow B$. Since $f(\eta_X) = \eta_Y$ (because $f : U \to V$ is dominant), then $\ker{\varphi} = \varphi^{-1}(0) = (0)$ so $\varphi$ is injective so we get an extension of domains $A \subset B$. Furthermore, $K = \Frac{A} = K(Y)$ and $F = \Frac{B} = K(X)$. The morphism $f : U \to V$ must be generically finite which implies that the fibre,
\[ U_{\eta_Y} = \Spec{B \otimes_A K} = \Spec{B \otimes_A S^{-1}_A A} = \Spec{S_A^{-1} B} \]
is finite. However, $B_K = B \otimes_A K(Y)$ is a finitely generated $K$-algebra because the base change of the map $A[x_1, \dots, x_n] \twoheadrightarrow B$ gives $K[x_1, \dots, x_n] \twoheadrightarrow B_K$. Now we apply Noetherian normalization to the domain $B_K = S_A^{-1} B$ to get a finite (and hence integral) extension of domains $B_K \supset K[x_1, \dots, x_d]$ with $d = \dim{B_K}$. By Cohen, $\Spec{B_K} \to \Spec{K[x_1, \dots, x_d]}$ is surjective but for $d > 0$ the space $\Spec{K[x_1, \dots, x_d]}$ is infinite so $\dim{B_K} = 0$ and thus $B_K$ is a domain finite over $K$. Therefore, $B_K$ is a field but $B_K = S_A^{-1} B \subset F$ so $B_K = \Frac{B} = F$ meaining that $F / K$ is a finite extension of fields since $B_K / K$ is finite. 
\bigskip\\
Now, $F / K$ is generated by $x_1, \dots, x_n \in B$ all of which must satisfy monic $K$-equations since $F / K$ is finite. Let $g \in K$ be the product of the denominators of the coefficients then $A_g \subset B_g$ is a finite extension since $B_g$ is generated as an $A_g$-algebra by finitely many integral elements. Then $f : U_g \to V_g$ is finite with $U_g = \Spec{B_g}$ and $V_g = \Spec{A_g}$ and $V_g$ is dense since $Y$ is irreducible. Since $f^{-1}(V_g) \subset X$ is dense ($X$ is irreducible) replacing $X$ by $f^{-1}(V_g)$ and $Y$ by $V_g$ reduces to the case of $f : X \to \Spec{A}$ with an affine covering by $U_i = \Spec{B_i}$ such that $f : U_i \to \Spec{A}$ is finite and $A \embed B_i$ is a finite extension of domains. Then take,
\[ W = \bigcap U_i \]
which is nonempty since $X$ is irreducible. Now $U_i \setminus W$ is closed in $U_i$ so there is some ideal $\a_i \subset B_i$ strictly containing the nilradical (i.e. nonzero since these are domains) such that $U_i \setminus W = V(\a_i)$ (since $U_i \setminus W \subsetneq U_i$). If $\a_i \cap A = (0)$ then by Cohen $\a_i = (0)$ since there cannot be inclusions in the fibres of an integral extension. Therefore, $\a_i \cap A \supsetneq (0)$ so take some nonzero $f_i \in \a_i \cap A$ then in $U_i = \Spec{B_i}$ we have $D(f_i) \subset V(\a_i)^C = W$. Take $V = \Spec{A_{f_i}}$ which is open in $\Spec{A}$ and $f^{-1}(V) \cap U_i = D(f_i) \subset W \cap U_i$ meaning that $f^{-1}(V) \subset W$ since $U_i$ form a cover of $X$ so $f^{-1}(V) \subset U_i$ and so $f^{-1}(V) = \Spec{(B_i)_{f_i}}$ is affine. Finally, since $A \subset B_i$ is finite we know $A_{f_i} \to (B_i)_{f_i}$ is finite and thus $f : f^{-1}(V) \to V$ is finite and since $Y$ is irreducible and $V$ is nonempty open it is dense.

\subsubsection{3.8}

\subsubsection{3.9}

\begin{enumerate}
\item Let $\A^1_k = \Spec{k[x]}$ where $k$ is algebraically closed. The points of $\A^1_k$ are ideals $(x - \mu)$ for $\mu \in k$ and $(0)$. However, the points of $\A^2_k = \A^1_k \times_k \A^1_k = \Spec{k[x,y]}$ are ideals $(x - \mu, y - \lambda)$ with $\mu, \lambda \in k$ plus $(f(x,y))$ for any irreducible $f(x,y) \in k[x,y]$ plus $(0)$. Therefore, $\A^2_k$ has all points in the product plus a bunch of generic points of closed subschemes.

\item Consider $k(s)$ and $k(t)$ with two independent inteterminants. These are fields so $\Spec{k(t)}$ and $\Spec{k(t)}$ are point point spaces. However, consider,
\[ X = \Spec{k(s)} \times_{\Spec{k}} \Spec{k(t)} = \Spec{k(s) \otimes_k k(t)} \]
This has at least as many closed points as $k^\times$ because the map $k(s) \otimes_k k(t) \to k(x)$ sending $s \otimes 1 \mapsto x$ and $1 \otimes t \mapsto r x$ for $r \in k^\times$ is surjective making its kernel $(rs \otimes 1 - 1 \otimes t)$  a maximal ideal. 
\end{enumerate}

\subsubsection{3.10}

\begin{enumerate}
\item Let $f : X \to Y$ be a morphism and $y \in Y$. Then consider the fibre $X_y$ defined as the pushout,
\begin{center}
\begin{tikzcd}[row sep = large]
X_y \arrow[r] \arrow[d] & X \arrow[d, "f"] 
\\
\Spec{\kappa(y)} \arrow[r] & Y 
\end{tikzcd}
\end{center}
First, note that $X_y \to X$ as a map of topological spaces has image inside the fibre $f^{-1}(y)$ since the diagram commues and the image of $\Spec{\kappa(y)} \to Y$ is the single point $y \in Y$. Thus it suffices to show that $X_y \to f^{-1}(y)$ is a homeomorphism.
\bigskip\\
For any point $x \in f^{-1}(y)$ there is a morphism $\Spec{\kappa(x)} \to X$ and $f$ gives a map $\kappa(y) \to \kappa(x)$ and thus a morphism $\Spec{\kappa(x)} \to \Spec{\kappa(y)}$ such that the diagram commutes,
\begin{center}
\begin{tikzcd}[row sep = large]
\Spec{\kappa(x)} \arrow[rrd, bend left] \arrow[rd, dashed] \arrow[ddr, bend right]
\\
& X_y \arrow[r] \arrow[d] & X \arrow[d, "f"] 
\\
& \Spec{\kappa(y)} \arrow[r] & Y
\end{tikzcd}
\end{center} 
Thus we get a point $\Spec{\kappa(x)} \to X_y$. Therefore, the map $X_y \to f^{-1}(y)$ is bijective. Therefore, it suffices to prove that $\iota : X_y \to f^{-1}(y)$ is closed. (DO THIS)
\item Let $k$ be an algebraicall closed field of characteristic zero. Consider the scheme,
\[ X = \Spec{k[s,t]/(s - t^2)} \]
and $Y = \Spec{k[s]}$ and consider the morphism $f : X \to Y$ via $k[s] \to k[s,t]/(s - t^2)$. For the prime $y = (s - a) \in Y$ consider the residue field,
\[ \kappa(y) = k[s]_{(s - a)} / (s - a) = k[s]/(s - a) \]
then the fibre is,
\[ X_y = \Spec{k[s,t]/(s - t^2) \otimes_{k[s]} k[s]/(s - a)} \]
furthermore,
\[ k[s,t]/(s - t^2) \otimes_{k[s]} k[s]/(s - a) = k[t]/(a - t^2) \]
which implies that,
\[ X_y \cong V((a - t^2)) \subset \Spec{k[t]} \]
Thus, if $a \neq 0$ then $t^2 - a$ splits (since $k$ is algebraically closed and $\ch{k} = 0$) so $\Spec{k[t]/(a - t^2)}$ has two points and is reduced. For $a = 0$ we have $X_y = \Spec{k[t]/(t^2)}$ which is one point and not reduced since $\nilrad{k[t]/(t^2)} = (t)$.  
\bigskip\\
Finally, consider the fibre above the generic point $\eta = (0) \subset k[s]$ which has resuide field $\kappa(\eta) = k[s]_{(0)} = k(s)$. Therefore the fibre is,
\[ X_\eta = \Spec{k[s, t]/(s - t^2) \otimes_{k[s]} k(s)} = \Spec{k(s)[t]/(s - t^2)} \] 
The polynomial $t^2 - s \in k(s)[t]$ is irreducible then $(s - t^2)$ is maximal. Then $k(s)[t]/(s - t^2)$ is a field extension of $k(s)$ of degree 2 and thus it has one prime so $X_\eta$ is a one point space.
\end{enumerate}

\subsubsection{3.11}

\begin{enumerate}
\item Let $f : Z \to Y$ be a closed immersion and $X \to Y$ a morphism then consider $f' : Z \times_Y X \to X$. Being a closed immersion is a local property since surjectivity of sheaves is local on the source and target and being a homeomorphism onto a closed set is local since the image of closed sets being closed is local on the source and target. Thus it suffices to prove the case of affine schemes $X = \Spec{A}$ and $Y = \Spec{B}$ and $Z = \Spec{C}$. Then we get,
\begin{center}
\begin{tikzcd}
C \otimes_B A \arrow[from=r, two heads] & A
\\
C \arrow[u] & B \arrow[u] \arrow[l, two heads]
\end{tikzcd}
\end{center}
If $B \to C$ is surjective then by right-exactness $A \to C \otimes_B A$ is surjective. Furthermore, by surjectivity of $B \to C$ we get $C = B / I$ and thus this is the closed immersion 
$\Spec{B/I} \to \Spec{C}$. Then, $A \to (B / I) \otimes_B A = B / I B$ gives the closed immersion $\Spec{B / I B} \to \Spec{B}$.

\item Let $X = \Spec{A}$ be affine and $\iota : Y \embed X$ a closed subscheme. Let $\Spec{B_i} = U_i \subset Y$ be an affine open cover of $Y$ and consider $\iota|_{U_i} : U_i \to \iota(U_i)$. Since $\iota : Y \to \iota(Y)$ is a homeomorphism with $\iota(Y) \subset X$ closed so $\iota(U_i)$ is open in $\iota(Y)$. Therefore, we can cover $\iota(U_i)$ by finitely may $D(f_{ij})$ and write,
\[ \iota(U_i) = \bigcup_{j = 1}^{n_i} D(f_{ij}) \cap \iota(Y) \]
Now, $\iota|_{U_i}^{-1}(D(f_{ij})) = U_i \cap \iota^{-1}(D(f_{ij})) = D(\varphi_i(f_{ij}))$ for $\varphi : A \to B_i$ inside $U_i = \Spec{B_i}$. However, $D(f_{ij}) \cap \iota(Y) \subset \iota(U_i)$ and $\iota$ is injective so,
\[ \iota^{-1}(D(f_{ij})) \subset U_i \]
and thus,
\[ \iota^{-1}(D(f_{ij})) = U_i \cap \iota^{-1}(D(f_{ij})) = \iota^{-1}(D(f_{ij})) = D(\varphi_i(f_{ij})) \]
Therefore, $Y$ is covered by affine opens $D(\varphi_i(f_{ij}))$. Furthermore, $\Spec{A}$ is quasi-compact so $\iota(Y) \subset X$ closed is quasi-comact and since $\iota : Y \to \iota(Y)$ is a homeomorphism then $Y$ is quasi-compact so we may take $D(\varphi_i(f_{ij}))$ to be a finite affine cover of $Y$. Since $\iota(Y)$ is closed we can cover $X \setminus \iota(Y)$ by finitely many principal opens $D(f_k)$ to give a finite cover of $X$ by opens $D(f_i)$ such that $\iota^{-1}(D(f_i))$ is empty or affine. Since $D(f_i)$ cover $X$ we have $f_1, \dots, f_n \in A$ generate the unit ideal so $\iota^\#(f_1), \dots, \iota^\#(f_n) \in \Gamma(Y, \struct{Y})$ generate the unit ideal and $\iota^{-1}(D(f_i)) = Y_{\iota^\#(f_i)} = D(\varphi(f_i))$. Therefore, by criterion 2.17 we have that $Y = \Spec{B}$ is affine. Furthermore, by 2.18d, since $Y \to X$ is a closed immersion we have $A \to B$ is surjective so $B \cong A / \a$ and our closed subscheme is equivalent to $\Spec{A / \a} \to \Spec{A}$.  

\item Let $Y \subset X$ be a closed subset and give $Y$ the reduced induced subscheme structure. Let $Y' \embed X$ be any other closed subscheme of $X$ whose underlying space is $Y$. This question is local so it suffices to show the case that $X = \Spec{A}$ is affine and thus $Y = \Spec{A / I}$ where,
\[ I = \bigcap_{\p \in V(I)} \p \]
i.e. $I = \sqrt{I}$ is radical. Then $Y' = \Spec{A / J}$ for any ideal such that $V(J) = V(I)$ i.e. $\sqrt{J} = \sqrt{I} = I$. Therefore, $J \subset I$ so  the map $\Spec{A / I} \to \Spec{A}$ factors through $\Spec{A/J} \to \Spec{A}$ since the ring map $A \to A / I$ factors through $A \to A / J$ because $J \subset I$.  

\item Let $f : Z \to X$ be a morphism. Consider the scheme theoretic image $Y$ of $f$ which is a closed subscheme of $Y$ such that $f$ factors $f : Z \to Y \to X$ and if $Y'$ is annother closed subscheme of $X$ such that $f$ factors as $f : Z \to Y' \to X$ then $Y \to X$ factors through $Y' \to X$,
\begin{center}
\begin{tikzcd}
Z \arrow[rr, "f"] \arrow[rd, "\tilde{f}"] \arrow[rdd, "\tilde{f}", bend right] &  & X
\\
& Y \arrow[ru, "\iota"] \arrow[d, dashed]
\\
& Y' \arrow[ruu, "\iota'", bend right]
\end{tikzcd}
\end{center}
Uniqueness is clear since if $Y$ and $Y'$ both satisfied this condition then we have morphisms $Y \to Y'$ and $Y' \to Y$ which compose to give an automorphism of $Y \to X$ which must be the identity since $Y \to X$ is a closed immersion.
\bigskip\\
We need to show that such a scheme exists. (SHOW THIS)
\end{enumerate}

\begin{lemma}
Given a morphism of schemes $f : X \to \Spec{A}$ and $g \in A$ we have, 
\[ f^{-1}(D(g)) = X_{f^\#(g)} = \{ x \in X \mid (f^\#(g))_x \notin \m_x \} \]
\end{lemma}

\begin{proof}
Recall that $f(x) = \p$ iff $\p = (f^\#)^{-1} \circ \res_x^{-1}(\m_x)$ because for $f(x) = \p$ the sheaf diagram,
\begin{center}
\begin{tikzcd}
A \arrow[d] \arrow[r, "f^\#"] & \struct{X}(X) \arrow[d, "\res_x"]
\\
A_\p \arrow[r, "f^\#_x"] & \stalk{X}{x}
\end{tikzcd}
\end{center} 
And furthermore, $f^\#_x : A_\p \to \stalk{X}{x}$ is local so $(f^\#_x)^{-1}(\m_x) = \p A_\p$ and thus, by commutativity, $\p = (f^\#)^{-1} \circ \res_x^{-1}(\m_x)$. 
Thus,
\begin{align*}
x \in f^{-1}(D(g)) & \iff f(x) \in D(g) \iff (f^\#)^{-1} \circ \res_x^{-1} (\m_x) \in D(g) 
\\
& \iff g \notin (f^\#)^{-1} \circ \res_{x}^{-1} (\m_x) \iff (f^\#(g))_x \notin \m_x 
\\
& \iff x \in X_{f^\#(g)}
\end{align*}
\end{proof}

\subsubsection{3.12}

\begin{enumerate}
\item Let $\varphi : S \to B$ be a surjective graded ring map. (DO 2.14)
\end{enumerate}

\subsubsection{3.13}

\subsubsection{3.14}

Let $X$ be a scheme locally of finite type over a field $k$. A point $x \in X$ is closed iff $\kappa(x)$ is a finite extension of $k$. Take any nonempty open $U \subset X$ which must contain an affine open $\Spec{A}$. Since $X$ is of finite type over $k$ the ring $A$ is a finite $k$-algebra which we may write as $k[x_1, \dots, x_n]/I$. Take a maximal ideal $\m$ containing $I$ such that $\m \in \Spec{A}$ and,
\[ \kappa(\m) = A_{\m} / \m A_{\m} = (A \setminus \m)^{-1} (A / \m) = A / \m \]
since $A / \m$ is a field. Furthermore, $A / \m$ is a finitely-generated $k$-algebra so, by the nullstellensatz, $A / \m$ is a finite extension of $k$ so $\m \in \Spec{A} \subset U$ is a closed point of $X$ (not just of $\Spec{A}$ which is obvious). Therefore, closed points are dense. 
\bigskip\\
Conversely, take any local ring $R$ which is not a field. Then $R$ has a unique maximal ideal which is a unique closed point so the closure of the closed points is a single point. However $\Spec{R}$ has more than one point.  

\subsubsection{3.15}

Let $X$ be a scheme of finite type over $k$.

\begin{enumerate}
\item We say that $X$ is \textit{geomerically irreducible} if one of the three conditions hold,
\begin{enumerate}
\item[(i)] $X \times_k \Spec{\overline{k}}$ is irreducible
\item[(ii)] $X \times_k \Spec{k^{\text{sep}}}$ is irreducible
\item[(iii)] $X \times_k \Spec{K}$ is reduced for every extension $K / k$
\end{enumerate}
The map $k^{\text{sep}} \embed \overline{k}$ induces $X \times_k \Spec{\overline{k}} \to X \times_k \Spec{k^{\text{sep}}}$. Since the map $\Spec{k^{\text{sep}}} \to \Spec{\overline{k}}$ is surjective its base change to,
\[ X \times_k \Spec{\overline{k}} \to X \times_k \Spec{k^{\text{sep}}} \]
is surjective. Now, the image of irreducible sets is irreducible so (i) $\implies$ (ii). Furthermore, 
(ii) $\implies$ (iii) (DO THIS)
Finally, (iii) $\implies$ (i) is trivial. 

\item We say that $X$ is \textit{geomerically reduced} if one of the three conditions hold,
\begin{enumerate}
\item[(i)] $X \times_k \Spec{\overline{k}}$ is reduced
\item[(ii)] $X \times_k \Spec{k^{\text{perf}}}$ is reduced
\item[(iii)] $X \times_k \Spec{K}$ is reduced for every extension $K / k$
\end{enumerate}

The map $k^{\text{perf}} \embed \overline{k}$ induces $X \times_k \Spec{\overline{k}} \to X \times_k \Spec{k^{\text{perf}}}$. Since the map $\Spec{k^{\text{perf}}} \to \Spec{\overline{k}}$ is surjective its base change to,
\[ X \times_k \Spec{\overline{k}} \to X \times_k \Spec{k^{\text{perf}}} \]
is surjective. Now, the image of irreducible sets is irreducible so (i) $\implies$ (ii). Furthermore, 
(ii) $\implies$ (iii) (DO THIS)
Finally, (iii) $\implies$ (i) is trivial. 

\item We say that $X$ is \textit{geometrically integral} if $X \times_k \overline{k}$ is integral. 
\end{enumerate}

\subsubsection{3.16}

Let $P$ be a property of closed subsets of a noetherian topological space $X$. Suppose that each closed subset $Y \subset X$ has the property that if $P$ is true for each proper closed subset of $Y$ then $P$ holds for $P$. Furthermore, suppose that $P$ holds for $\varnothing \subset X$.
\bigskip\\
Let $\Sigma = \{ Y \subset X \mid Y \text{ is closed and property P fails} \}$ which is a poset under inclusion. Assume that $\Sigma$ is nonempty. By the Noetherian propery all chains have a least element and thus, by Zorn's Lemma, $\Sigma$ has a least element $Y \in \Sigma$. Thus, any proper closed subset of $Y$ cannot lie in $\Sigma$ and thus has property $P$. By the induction assumption $Y$ has property $P$ contradicting $Y \in \Sigma$ so the assumption that $\Sigma \neq \varnothing$ must be false. Therefore, $X \notin \Sigma$ so $X$ has property $P$.

\subsubsection{3.20}

\begin{lemma}
Let $Y \subset X$ be any subset then $\dim{Y} \le \dim{X}$. 
\end{lemma}

\begin{proof}
Let $Y = Z_0 \supsetneq Z_1 \supsetneq \cdots \supsetneq Z_d$ be a maximal chain of closed irreducible sets in $Y$. Then consider $W_i = \overline{Z_i}$. I claim that $W_i$ is irreducible and forms a proper chain. First, if $W_i \supset C_1 \cup C_2$ for closed $C_1, C_2$ then $Z_i \subset W_i \subset C_1 \cup C_2$ but $Z_i$ is irreducible so (WLOG) $Z_i \subset C_1$ and thus $W_i = \overline{Z_i} \subset C_1$ since $C_1$ is cloed. Thus $W_i$ is irreducible. 
\bigskip\\
Now suppose that $\overline{Z_i} = \overline{Z_{i+1}}$ then $Z_i \subset \overline{Z_{i+1}}$ which implies that $Z_i \subset \overline{Z_{i + 1}} \cap Y = Z_{i + 1}$ (since $Z_{i + 1}$ is closed in $Y$) which contradicts the fact that $Z_i \supsetneq Z_{i+1}$. Thus the chain is proper so $\dim{X} \ge d = \dim{Y}$.  
\end{proof}

\begin{definition}
If $Z \subset X$ is irreducible closed then $\codim{Z,X}$ is the length of the longest chain,
\[ Z = Z_0 \subsetneq Z_1 \subsetneq \cdots \subsetneq Z_n \]
Furthermore, if $Y \subset X$ is closed then we define,
\[ \codim{Y, X} = \inf_{Z \subset Y} \codim{Z, X} \]
\end{definition}

\begin{lemma}
Let $X$ be a topological space and $Z \subset X$ be an irreducible closed subspace. Let $U \subset X$ be open such that $U \cap Z$ is nonempty then $\codim{Z, X} = \codim{Z \cap U, U}$.
\end{lemma}

\begin{proof}
Let $Z = Z_0 \subsetneq Z_1 \subsetneq \cdots \subsetneq Z_d$ be a maximal chain of closed irreducible subsets in $X$ so $d = \dim{X}$. Consider,
\[ U \cap Z_0 \subset U \cap Z_1 \subset \cdots \subset U \cap Z_d \]
Then $U \cap Z_i$ is nonempty (contains $U \cap Z$) and thus is irreducible and furthermore is closed in $U$. Consider $(U^C \cap Z_i) \cup \overline{U \cap Z_i} \supset Z_i$ and both are closed so $\overline{U \cap Z_i} \supset Z_i$ since $U \cap Z_i$ is non empty which implies that $U^C \cap Z_i$ does not contain $Z_i$. Furthermore, $Z_i \supset U \cap Z_i$ and $Z_i$ is closed so $Z_i \supset \overline{U \cap Z_i}$. Therefore, $\overline{U \cap Z_i} = Z_i$. Thus we cannot have $U \cap Z_{i} = U \cap Z_{i+1}$ since their closures are $Z_i \subsetneq Z_{i+1}$ which are distinct. Thus, $\codim{U \cap Z, U} \ge d = \codim{Z, X}$ and by the argument in the previous lemma, $\codim{U \cap Z, U} \le \codim{Z, X}$.
\end{proof}

\begin{corollary}
If $U \subset X$ is open and contains a point $x \in X$ such that $\dim_x(X) = \dim{X}$ then $\dim{U} = \dim{X}$.
\end{corollary}

\begin{proof}
We have, \[ \dim_x(X) = \codim{x, X} = \codim{x, U} = \dim_x(U) \le \dim{U} \]
Thus, $\dim{X} \le \dim{U}$ but we have shown that $\dim{U} \le \dim{X}$.
\end{proof}

\begin{lemma}
Let $X$ be sober and $Z \subset X$ a closed irreducible subspace and $\xi \in Z$ its generic point. Then $\codim{Z, X} = \dim{\stalk{X}{\xi}}$.
\end{lemma}

\begin{proof}
Choose any affine open $U \subset X$ containing the generic point $\xi \in Z$. Then,
\[ \codim{Z, X} = \codim{Z \cap U, U} \]
However, $U = \Spec{A}$ and $Z \cap U = V(\p)$ since it is closed irreducible in $U$ with $\p = \xi \in U$ its generic point. Then chains of irreducible subsets about $V(\p)$ correspond to chains of prime ideals below $\p$ so,
\[ \codim{Z \cap U, U} = \codim{V(\p), \Spec{A}} = \height{(\p)} = \dim{A_\p} = \dim{\stalk{X}{\xi}} \]
\end{proof}

Let $X$ be an integral scheme of finite type over a field $k$.
\begin{enumerate}
\item Let $x \in X$ be a closed point and $U \subset X$ any affine open containing $x$. Then $U = \Spec{A}$ with $A$ a finitely-generated $k$-algebra domain with $x$ corresponding to $\m \subset A$. In this case, the height satisfies,
\[ \dim{\stalk{X}{x}} = \dim{A_\m} = \height{\m} = \dim{A} - \dim{A / \m} \]
However, $A / \m$ is a field so $\dim{A / \m} = 0$. Thus,
\[ \dim{\stalk{X}{x}} =  \dim{A} \]
Now, let $X = Z_0 \supsetneq Z_1 \supsetneq \cdots \supsetneq Z_d = \{ p \}$ (with $p \in X$ closed) be a maximal chain of closed irreducible subsets so $d = \dim{X}$. Take an affine open $V = \Spec{B}$ with $p \in V$ then by the same argument $\dim{B} = \dim{\stalk{X}{p}}$. Furthermore, since $X$ is irreducible $U \cap V$ is nonempty open. Furthermore, since $X$ is finite type over $k$ closed points are dense so we can choose a closed point $q \in U \cap V$ and then we have,
\[ \dim{\stalk{X}{x}} = \dim{A} = \dim{\stalk{X}{q}} = \dim{B} = \dim{\stalk{X}{p}} \]
By the lemma, $\dim{V} = \dim{X}$ and thus,
\[ \dim{X} = \dim{V} = \dim{B} = \dim{\stalk{X}{p}} =  \dim{\stalk{X}{x}} \]
\item For any finitely-generated $k$-algebra domain $A$ we have $\dim{A} = \trdeg{k}{\Frac{A}}$ and thus, for any affine open $U \subset X$ with $U = \Spec{A}$ then we have shown,
\[ \dim{X} = \dim{A} = \trdeg{k}{\Frac{A}} \]
However, $\Frac{A} = K(X)$ is the function field $\stalk{X}{\xi}$ so we have,
\[ \dim{X} = \trdeg{k}{K(X)} \]
\item 
Now, in the case of integral schemes of finite type over $k$, consider a closed subset $Y \subset X$ then,
\[ \codim{Y, X} = \inf_{Z \subset Y} \codim{Z, X} \]
for irreducible closed subsets $Z \subset Y$. First, for an irreducible closed subset $Z \subset X$ 

We know that $\codim{Z, X} = \dim{\stalk{X}{\xi}}$ where $\xi \in Z$ is the generic point. Therefore, 
\[ \codim{Y, X} = \inf \{ \dim{\stalk{X}{\xi}} \mid \xi \in Y \} \]
since schemes are sober so points of $Y$ correspond exactly to closed irreducibe subspaces via closure. 

\item 

Let $Y \subset X$ be a closed subset. First, suppose that $Y$ is irreducible then for any affine open $U \subset X$ intersecting $Y$ we have,
\[ \codim{Y, X} = \codim{U \cap Y, U} \]
Furthermore, $U = \Spec{A}$ and $Y \cap U$ is irreducible closed in $U$ so $Y \cap U = V(\p)$. Thus,
\[ \codim{U \cap Y, U} = \codim{V(I), \Spec{A}} =\height{\p} \]
However, since $A$ is a finitely generated $k$-algebra domain,
\[ \dim{A} = \height{\p} + \dim{A / \p} \]
Furthermore, since $U \subset X$ and  $U \cap Y \subset U$ are nonempty open subspaces, we have shown that, $\dim{X} = \dim{U} = \dim{A}$ and,
\[ \dim{Y} = \dim{U \cap Y} = \dim{V(I)} = \dim{A / I} \]
Therefore, we have,
\[ \dim{X} = \codim{Y, X} + \dim{Y} \]
Now suppose that $Y$ is not irreducible we then define,
\[ \codim{Y, X} = \inf_{Z \subset Y} \codim{Z, X} \]
for all irreducible closed subsets of $Y$ which are thus irreducible closed subsets of $X$. Thus,
\[ \codim{Y, X} = \inf_{Z \subset Y} \left[ \dim{X} -  \dim{Z} \right] = \dim{X} - \sup_{Z \subset Y} \dim{Z} = \dim{X} - \dim{Y} \]
since by definition,
\[ \dim{Y} = \sup_{Z \subset Y} \dim{Z} \]
is the maximal length of irreducible closed chains in $Y$. 

\item Let $U \subset X$ be a nonempty open. Since $X$ is finite type over $k$ the closed points of $X$ are dense so there is a closed point $x \in U$ and then we have shown that,
\[ \dim{X} = \dim{\stalk{X}{x}} = \dim{U} \]
since $U$ is a scheme satisfying the conditions. Furthermore, for any affine open $\Spec{A} \subset U$ then we know,
\[ \dim{A} \le \dim{U} \le \dim{X} \]
and $\dim{A} = \dim{X}$ so $\dim{U} = \dim{X}$.
  
\item Assume that $k$ is a perfect field (HOW TO DO IT WHEN NOT PERFECT). Consider an extension $k \embed k'$ and base change to $X \times_k \Spec{k'}$. First, if $A$ is a finitely generated $k$-algebra domain then base chaning to $A' = A \otimes_k k'$ is a finitely generated $k'$-algebra. Furthermore, since $k' / k$ is seperable then $A' = A \otimes_k k'$ is reduced. Therefore, the irreducible components of $X \times_k k'$ are integral scheme of finite type over $k'$. Then, as we have shown, for any affine open $U = \Spec{A'}$ in a irreducible component $\dim{X'} = \dim{U} = \dim{A'}$. Furthermore, $\dim{X} = \dim{A}$ where $\Spec{A}$ is an affine open of $X$. Thus, it suffices to show that $\dim{A} = \dim{A'}$. However, these are finitely-generated $k$-algebra domains so we know that,
\[ \trdeg{k}{\Frac{A}} = \trdeg{k'}{\Frac{A'}} \]
and thus $\dim{A} = \dim{A'}$. (FIX THIS PROOF).
\end{enumerate}

\subsubsection{3.21}

Let $R$ be a discrete valuation ring containing its residue field $k$ for example $R = k[[X]]$ and let $K = \Frac{R} = R_\varpi$. Let $X = \Spec{R[t]}$. We have $\dim{X} = 2$. 
\bigskip\\
However, consider the ideal $\m = (\varpi t - 1) \subset R[t]$ and the quotient $R[t]/(\varpi t - 1) \to K$ via $t \mapsto \varpi^{-1}$ is an isomorphism so $\m = (\varpi t - 1)$ is maximal. However, $\height{\m} = 1$ because it is principal (see Lemma \ref{principal_ideals_height_one}). Thus,
\[ \dim{R[t]_{\m}} = \height{\m} = 1 \]
but $\dim{X} = 2$ so property (a) does not hold.
\bigskip\\
Now consider the closed set $Y = V(\m)$ where $\m = (\varpi t - 1)$ then $\dim{Y} = \dim{R[t] / \m} = \dim{K} = 0$ and $\codim{Y, X} = \height{\m} = 1$ however $\dim{X} = 2$ so,
\[ \dim{X} \neq \codim{Y, X} + \dim{Y} \]
and thus property (d) does not hold. 
\bigskip\\
Consider the open $D(\varpi) \subset X$ then $D(\varpi) \cong \Spec{R[t]_\varpi} = \Spec{K[t]}$. Therefore,
\[ \dim{D(\varpi)} = \dim{K[t]} = 1 \]
but $\dim{X} = 2$ so property (e) does not hold.

\begin{lemma} \label{principal_ideals_height_one}
Let $A$ be Noetherian. Any principal prime ideal $\p = (a)$ has $\height{\p} \le 1$.
\end{lemma}

\begin{proof}
If $\p = (a)$ is prime then consider the local ring$A_{(a)}$ which is Noetherian and has unique maximal ideal $a A_{(a)}$. Thus $A_{(a)}$ is Noetherian and has every maximal ideal prinipal so $A_{(a)}$ is a PID and thus $\height{(a)} = \dim{A_{(a)}} \le 1$.  
\end{proof}

\subsection{4}

\begin{definition}
A morphism $f : X \to Y$ is \textit{proper} if it is separated, of finite type, and universally closed. 
\end{definition}

\begin{lemma}
Finite morphisms are preserved under base change. 
\end{lemma}

\begin{proof}
This is local so we only need to check this for affine schemes. Then it follows from the fact that finite ring maps are preserved under tensor product because surjections are preserved by tensor products.
\end{proof}

\subsubsection{4.1}

Let $f : X \to S$ be a finite morphism. Finite morphisms are affine and thus separated and clearly finite morphisms are of finite type. Furthermore, finite morphisms are closed and finite morphisms are preserved under base change so they are universally closed.  

\subsubsection{4.2 DO!!}

Let $X$ be a reduced scheme over $S$ and $Y$ be a seperated scheme over $S$. Consider two $S$-morphisms $f,g : X \to Y$ and a dense open set $U \subset X$ such that $f|_U = g|_U$. Now, consider the map $F : X \to Y \times_S Y$ defined by $f$ and $g$. Now consider the diagonal map $\Delta : Y \to Y \times_S Y$ which is a closed immersion because $Y$ is seperated. Since $f|_U = g|_U$ we may factor $F |_U : X \to Y \times_S Y$ as $F|_U = \Delta \circ f|_U$ and thus $F(U) \subset \Delta(Y) \subset Y \times_S Y$ is closed. Therefore, $F(X) = F(\overline{U}) \subset \overline{F(U)} \subset \Delta(Y)$ since it is closed. Therefore, topologically, $f(x) = g(x)$ for $x \in X$. Thus, it remains to prove that the sheaf maps agree. By hypothesis, the map on stalks, $f^\#_x : \stalk{Y}{f(x)} \to \stalk{X}{x}$ and $g^\#_x : \stalk{Y}{g(x)} \to \stalk{Y}{x}$ agree for each $x \in U$. Consider a section $s \in \struct{Y}(V)$ for some open $V \subset Y$. Then consider the section $s' = f^\#(s) - g^\#(s) \in \struct{X}(f^{-1}(V))$. We know that for each $x \in f^{-1}(V) \cap U$ that $s'_x = f_x(s) - g_x(s) = 0$. Now the vanishing $V(s') = \{ x \in f^{-1}(V) \mid s'_x \in \m_x \}$ of the section $s'$ is closed in $f^{-1}(V)$. However, $\forall x \in U \cap f^{-1}(V)$ we know that $s'_x = 0$ so $x \in V(s')$. Thus, $V(s')$ is a closed set containing a dense set and thus $V(s') = f^{-1}(V)$. Therefore, $s'_x \in \m_x$ for each $x \in f^{-1}(V)$. Thus, on each affine open the restriction of $s'$ lies in every prime ideal and thus in the nilradical. Thus $s' \in \nilrad{\struct{X}(f^{-1}(V)}$. Since $X$ is reduced, $\nilrad{\struct{X}(f^{-1}(V)} = 0$ so $s' = 0$. Thus $f^\# = g^\#$. 

(FINISH)

\subsubsection{4.3}

Let $X$ be a separated scheme over an affine scheme $S = \Spec{A}$ and $U, V \subset X$ be affine open subsets. The diagonal morphism $\Delta : X \to X \times_S X$ is a closed immersion which is affine. Then conider $U \times_S V \subset X \times_S X$ which is affine and we have $U \cap V = \Delta^{-1}(U \times_S V)$ is affine since $\Delta$ is affine.
\bigskip\\
However, take $\A^2_k$ with two origins. Then each copy of $\A^2_k$ is clearly affine but their intersection is $\A^2_k \setminus \{ 0 \}$ which is not affine. 

\begin{lemma}
Closed immersions are affine.
\end{lemma}


\subsubsection{4.4 DO!!}

(LOOK UP IN NOTES)

\subsubsection{4.5 DO!!}

Let $X$ be an integral scheme of finite type over a field $k$ with function field $K$. We say that a valuation of $K / k$ has \textit{center} $x$ on $X$ if the valuation ring $R \subset K$ dominates the local ring $\stalk{X}{x}$.

\begin{enumerate}
\item Let $X$ be separated and suppose that $R \subset K$ has centers $x,y \in X$. Then $R$ dominates $\stalk{X}{x}$ and $\stalk{Y}{y}$. However, by the valuative criterion of separatedness for the diagram,
\begin{center}
\begin{tikzcd}
\Spec{K} \arrow[d] \arrow[r] & X \arrow[d]
\\
\Spec{R} \arrow[ru, dashed] \arrow[r] & \Spec{k}
\end{tikzcd}
\end{center}
any lift $\Spec{R} \to X$ is unique. However, either local map $\stalk{X}{x} \embed R$ and $\stalk{X}{y} \embed R$ gives such a lift via $\Spec{R} \to \Spec{\stalk{X}{x}} \to X$ and $\Spec{R} \to \Spec{\stalk{Y}{y}} \to Y$ so $x = y$.

\item Let $X$ be proper over $k$ and let $R \subset K$ be a valuation ring containing $k$. Then by the valuative criterion of properness for the diagram,
\begin{center}
\begin{tikzcd}
\Spec{K} \arrow[d] \arrow[r] & X \arrow[d]
\\
\Spec{R} \arrow[ru, dashed] \arrow[r] & \Spec{k}
\end{tikzcd}
\end{center}
there exists a unique lift $\Spec{R} \to X$ giving a local map $\stalk{X}{x} \to R$ for $x \in X$ the image of the maximal ideal of $R$. Since $R \subset K$ this is an injective local map $\stalk{X}{x} \embed R$ so $R$ has a unique center at $x$.

\item (DO THIS!!)

\item Let $X$ be proper over $k$. Then $\Gamma(X, \struct{X})$ is a finite (finiteness of cohomology) $k$-algebra domain and thus a finite field extension of $k$ so if $k$ is algebraically closed $\Gamma(X, \struct{X}) = k$.
\bigskip\\
Alternatively let $k$ be algebraically closed and let $a \in \Gamma(X, \struct{X})$ and $a \notin k$. Then (SHOW) there exists a valuating ring $R \subset K$ over $k$ such that $a^{-1} \in \m_R$. However, $R$ has center at some $x \in X$ so $\m_R \cap \stalk{X}{x} = \m_x$ but (FINISH THIS!!)  
\end{enumerate}

\subsubsection{4.6}

Let $f : X \to Y$ be a proper morphism of affine varieties $X = \Spec{B}$ and $Y = \Spec{A}$ over $k$. We can reduce to the case that $f : X \to Y$ is dominant i.e. $A \to B$ is an extension of $k$-algebra domains (HOW). By the valuative criterion for properness for any valuation ring $R$ and diagram,
\begin{center}
\begin{tikzcd}
\Spec{K} \arrow[d] \arrow[r] & X \arrow[d, "f"]
\\
\Spec{R} \arrow[ru, dashed] \arrow[r] & Y
\end{tikzcd}
\end{center}
there is a unique map $\Spec{R} \to X$ making the diagram commute. Let $K = \Frac{B}$ and $R \subset K$ be any valuating ring containing $A$ so there is a map $\Spec{R} \to \Spec{A}$ giving a diagram,
\begin{center}
\begin{tikzcd}
\Spec{K} \arrow[d] \arrow[r] & \Spec{B} \arrow[d, "f"]
\\
\Spec{R} \arrow[ru, dashed] \arrow[r] & \Spec{A}
\end{tikzcd}
\end{center}
corresponding to inclusions of rings,
\begin{center}
\begin{tikzcd}
B \arrow[r, hook] & K
\\
A \arrow[r, hook] \arrow[u, hook] & R \arrow[u, hook] \arrow[lu, hook, dashed]
\end{tikzcd}
\end{center}
Because the map $R \to K$ is injective we see that the dashed arrow $R \to B$ must also be injective so every valuation ring in $K$ containing $A$ also contains $B$. Thus $B$ is contained in the integral closure of $A$ in $K$ so the extension $A \subset B$ is integral. Furthermore, $A \to B$ is finite type since both are finite type $k$-algebras. Thus, by the lemma, $B$ is a finite $A$-module and thus $f : X \to Y$ is finite. It appears we only need that $X$ and $Y$ are integral affine schemes (i.e. $A, B$ are domains) and $f : X \to Y$ is proper since properness implies finite type.

\begin{lemma}
If $A \to B$ is finite type and integral then it is finite.
\end{lemma}

\begin{proof}
Tag 02JJ or Lang's Algebra, Proposition VII.1.2.
\end{proof}

\subsubsection{4.7}

\newcommand{\R}{\mathbb{R}}
\renewcommand{\C}{\mathbb{C}}

For any scheme $X_0$ over $\R$ let $X = X_0 \times_\R \Spec{\C}$ and $\alpha : \C \to \C$ be complex conjucation. Then $\sigma : X \to X$ is the automorphism $\id \times \alpha$ so $\sigma$ is semi-linear i.e. making the diagram,
\begin{center}
\begin{tikzcd}
X \arrow[d] \arrow[r, "\sigma"] & X \arrow[d]
\\
\Spec{\C} \arrow[r, "\alpha"] & \Spec{\C}
\end{tikzcd}
\end{center}
commute. Furthermore, $\sigma^2 = \id$ so $\sigma$ is an involution.

\begin{enumerate}
\item Let $X$ be a separated scheme of finite type over $\C$ and assume that $X$ is equiped with a semi-linear involution $\sigma : X \to X$. Furthermore, assume that for each pair of points $x_1, x_2 \in X$ there exists an affine open $U \subset X$ containing $x_1$ and $x_2$. Let $X_0 = \coeq{(\sigma, \id)}$. We need to show that $X_0$ exists and is the unique $\R$-scheme such that $X = X_0 \times_\R \Spec{\C}$ compatibly with $\sigma$.
\bigskip\\
Let $G = \Z/2\Z$ act on $X$ via $\sigma$. If $X = \Spec{A}$ is affine then take $X_0 = \Spec{A^G}$ which satisfies the universal property since $A^G = \ker{(\sigma - \id)}$. Otherwise, for any point $x \in X$ choose an affine open $U \subset X$ containing $x, \sigma(x)$. Since $\sigma^2 = \id$ we see that $\sigma(x), x \in \sigma(U)$. Therefore, $x, \sigma(x) \in V = \sigma(U) \cap U$ which is affine because $X$ is separated. Furthermore, $\sigma(V) = U \cap \sigma(U) = V$. Thus we can cover $X$ by $\sigma$-stable affine opens. The affine opens $V_i = \Spec{A_i}$ give gluing data for $X$ which descends to gluing data $V_i^G = \Spec{A^G_i}$ since $(-)^G$ is functorial which define $X_0$.
\bigskip\\
Define a map $X \to X_0 \times_\R \Spec{\C}$ via the canonical $\R$-maps $X \to X_0$ and $X \to \Spec{\C}$. This map is affine given locally on $X_0 \times_\R \Spec{\C}$ via $\varphi : A^G \otimes_\R \C \to A$. We can write any element as $a \otimes 1 + b \otimes i$ for $a,b \in A^G$. Suppose $\varphi(a \otimes 1 + b \otimes i) = a + i b = 0$ then $i b \in A^G$ so $i b = \sigma(ib) = -i \sigma(b) = -i b$ and thus $b = 0$ so $\ker{\varphi} = 0$. Furthermore, for any $a \in A$ we have $a + \sigma(a) \in A^G$ and $i (\sigma(a) - a)  \in A^G$ therefore,
\[ \tfrac{1}{2} (a + \sigma(a)) \otimes 1 + \tfrac{i}{2} (\sigma(a) - a) \otimes i \mapsto \tfrac{1}{2} (a + \sigma(a)) + \tfrac{1}{2}(a - \sigma(a)) = a \]
so $\varphi$ is surjective. Thus $X \to X_0 \times_\R \Spec{\C}$ is an isomorphism. Furthermore, $\alpha$ acts on $A^G \otimes_\R \C$ via $a \otimes 1 + b \otimes i \mapsto a \otimes 1 - b \otimes i$ which corresponds to $a + ib \mapsto a - bi$ and $\sigma(a + ib) = a - ib$ so $\id \times \alpha$ corresponds to $\sigma$. We can see this categorically,
\begin{center}
\begin{tikzcd}[column sep = small]
X \arrow[rd, "\sigma"] \arrow[dd] \arrow[rr] & & X_0 \times_\R \Spec{\C} \arrow[dd] \arrow[rd, "\id \times \alpha"] \arrow[rr] & & X_0 \arrow[dd] \arrow[rd, "\id"]
\\
& X \arrow[rr, crossing over] & & X_0 \times_\R \Spec{\C} \arrow[rr, crossing over] & & X_0 \arrow[dd] 
\\
\Spec{\C} \arrow[rd, "\alpha"] \arrow[rr, "\id"] & & \Spec{\C} \arrow[rd, "\alpha"] \arrow[rr] & & \Spec{\R} \arrow[rd, "\id"]
\\
& \Spec{\C} \arrow[from=uu, crossing over] \arrow[rr, "\id"] & & \Spec{\C} \arrow[from=uu, crossing over] \arrow[rr] & & \Spec{\R}
\end{tikzcd}
\end{center}
commutes because the map $X \to X_0$ is the coequalizer of $\id : X \to X$ and $\sigma : X \to X$ and thus the diagram,
\begin{center}
\begin{tikzcd}[row sep = large, column sep = large]
X \arrow[r, "\sigma"] \arrow[d] & X \arrow[d] 
\\
X_0 \arrow[r, "\id"] & X_0
\end{tikzcd}
\end{center}
commutes. Finally suppose that $X_0$ is any separated scheme of finite type over $\R$ equiped with an $\R$-morphism $\varphi : X \to X_0$ such that $\varphi \times \iota : X \to X_0 \times_\R \Spec{\C}$ is an isomorphim and the previous diagrams commute. However, via the earlier construction  we know that,
\[ X_0 = \coeq{(\id \times \alpha, \id : X_0 \times_\R : \Spec{\C} \to X_0 \times_\R \Spec{\C})} \]
Since the previous diagram commutes we see that $X_0 = \coeq{(\sigma, \id : X \to X)}$. Since this is a colimit it is unique up to unique isomorphism so there exists a unique such $X_0$.

\item By our construction, if $X = \Spec{A}$ is affine then $X_0 = \Spec{A^G}$ is affine as well and $X_0$ is unique so any such $X_0$ must be affine.

\item Let $X_0, Y_0$ be two such schemes over $\R$. Given a morphism $f_0 : X_0 \to Y_0$ over $\R$ we get a morphism $f : X \to Y$ with $f = f_0 \times \id$ satisfying,
\[ f \circ \sigma_X = (f_0 \times \id) \circ (\id \times \alpha) = f_0 \times \alpha = (\id \times \alpha) \times (f_0 \times \id) = \sigma_Y \circ f \]
Conversely, suppose that $f : X \to Y$ is a morphism satisfying $f \circ \sigma_X = \sigma_Y \circ f$. Projecting, we get a map $f' : X \to Y \to Y_0$ such that $f' \circ \sigma_X = f'$ and thus a map $f_0 : X_0 \to Y_0$ factoring $f'$ through $X \to X_0$ and thus we get a commutative diagram,
\begin{center}
\begin{tikzcd}[row sep = large, column sep = large]
X \arrow[d] \arrow[r, "f"] & Y \arrow[d] 
\\
X_0 \arrow[r, "f_0"] & Y_0 
\end{tikzcd}
\end{center}
Furthermore, since $f : X \to Y$ is a $\C$-morphism it commutes with $\id : \Spec{\C} \to \Spec{\C}$ so $f = f_0 \times \id$ by the universal property of the product.

\item Suppose that $X \cong \A^1_\C$. We need to classify semilinear involutions. If $\sigma : X \to X$ is a semilinear involution then $\bar{\sigma} = (\id \times \alpha) \circ \sigma$ is a $\C$-linear involution so we need to study automorphism of order $2$. Automorphisms of $\A^1_\C$ are given by $z \mapsto az + b$ for $a, b \in \C$. We need $z \mapsto az + b \mapsto a^2 z + ab + b = z$. Thus $b(a + 1) = 0$ and $a^2 = 1$ so either $b = 0$ and $a = \pm 1$ or $a = -1$ and $b = 1$. Thus either $\bar{\sigma} = \id$ or $\bar{\sigma}(z) = -z$ or $\bar{\sigma}(z) = 1 - z$.
\bigskip\\
However, I claim that each of these involutions is equivalent to $\id \times \alpha$. Indeed,

\begin{center}
\begin{tikzcd}[row sep = large, column sep = large]
\A^1_\C \arrow[d, "z \mapsto i z"'] \arrow[r, "z \mapsto -\bar{z}"] & \A^1_\C \arrow[d, "z \mapsto i z"]
\\
\A^1_\C \arrow[r, "z \mapsto \bar{z}"] & \A^1_\C
\end{tikzcd}
\qquad
\begin{tikzcd}[row sep = large, column sep = large]
\A^1_\C \arrow[d, "z \mapsto i(z - \frac{1}{2})"'] \arrow[r, "z \mapsto 1-\bar{z}"] & \A^1_\C \arrow[d, "z \mapsto i(z - \frac{1}{2})"]
\\
\A^1_\C \arrow[r, "z \mapsto \bar{z}"] & \A^1_\C
\end{tikzcd}
\end{center}
Therefore, up to $\C$-isomorphism there is only one semi-linear involution of $X$, namely $\sigma = \id \times \alpha$. Thus $X_0 \cong \A^1_\R$.

\item Suppose that $X \cong \P^1_\C$. We need to classify semilinear involutions. If $\sigma : X \to X$ is a semilinear involution then $\bar{\sigma} = (\id \times \alpha) \circ \sigma$ is a $\C$-linear involution so we need to study automorphism of order $2$. Automorphisms of $\P^1_\C$ are given by $\PGL{2}{\C} = \SL{2}{\C}/\{ \pm 1 \}$ matrices,
\[ A = \begin{pmatrix}
a & b 
\\
c & d 
\end{pmatrix} \]
with $ad - bc = 1$. Such a matrix has order $2$ if $a^2 + bc = bc + d^2$ and $ab + bd = 0$ and $c a + cd = 0$. So $a = \pm d$. If $a = d$ then $2 ba = 0$ and $2 ca = 0$ so either $a = d = 0$ or $b = c = 0$ giving,
\[ 
A_1 = \begin{pmatrix}
1 & 0
\\
0 & 1 
\end{pmatrix}
\quad \quad 
A_2 = \begin{pmatrix}
0 & -b
\\
b^{-1} & 0 
\end{pmatrix} \]
If $a = -d$ then we get,
\[ A_3 = \frac{1}{bc + 1}
\begin{pmatrix}
1 & b
\\
c & -1 
\end{pmatrix}
 \]
with $bc \neq -1$. These give,
\[ \bar{\sigma}(z) = z \quad \quad \bar{\sigma}(z) = - \frac{b^2}{z} \quad \quad \bar{\sigma}(z) = \frac{z + b}{cz - 1} \]
However, sending $z \mapsto b z$ transforms the second into $\bar{\sigma}(z) = - z^{-1}$ and the thrid can be diagonalized via the change of basis matrix,
\[ C =
\begin{pmatrix}
1 - \sqrt{1 + bc} & c
\\
1 + \sqrt{1 + bc} & c
\end{pmatrix}
\quad \quad C^{-1} A_3 C = 
\begin{pmatrix}
-1 & 0
\\
0 & 1
\end{pmatrix} \]
and thus $A_3$ is equivalent to the involution $\bar{\sigma}(z) = -z$. Therefore, we have classified the three $\C$-linear involutions of $X = \P^1_\C$,
\[ \bar{\sigma}_1(z) = z \quad \quad \bar{\sigma}_2(z) = - z^{-1} \quad \quad \bar{\sigma}_3(z) = - z\]
\bigskip\\
However, I claim that $\sigma_3$ is equivalent to $\sigma_1 = \id \times \alpha$. Indeed,

\begin{center}
\begin{tikzcd}[row sep = large, column sep = large]
\P^1_\C \arrow[d, "z \mapsto i z"'] \arrow[r, "z \mapsto -\bar{z}"] & \P^1_\C \arrow[d, "z \mapsto i z"]
\\
\P^1_\C \arrow[r, "z \mapsto \bar{z}"] & \P^1_\C
\end{tikzcd}
\end{center}
However, $\sigma_2$ and $\sigma_1$ are innequivalent up to $\C$-isomorphism and therefore there are two nonisomorphic $X_0$ such that $X \cong X_0 \times_\R \Spec{\C}$. Corresponding to $\sigma_1 = \id \times \alpha$ is $X_0 = \P^1_\R$. Corresponding to $\sigma_2$ is $X_0$ which we now investigate.
\bigskip\\
Consider $A = \C[z, z^{-1}]$ and $G = \Z / 2\Z$ acts on $A$ via $\sigma : A \to A$ defined by $z \mapsto - z^{-1}$ and complex conjuation on the coefficients. Then,
\[ A^G = \ker{(\sigma - \id)} = \R[\tfrac{1}{2}(z - z^{-1}), \tfrac{i}{2}(z + z^{-1})] \subset \C[z, z^{-1}] \]
Let $x = \tfrac{1}{2}(z - z^{-1})$ and $y = \tfrac{i}{2}(z + z^{-1})$ but these are not indpendent. In fact, the satisfy a single relation,
\[ x^2 + y^2 + 1 = \tfrac{1}{4} (z^2 - 2 + z^{-2}) - \tfrac{1}{4} (z^2 + 2 + z^{-2}) + 1 = 0 \]
Thus, $A^G \cong \R[x,y]/(x^2 + y^2 + 1)$. So an affine open of $X_0$ is isomorphic to $\Spec{\R[x,y]/(x^2 + y^2 + 1)}$. Continuing this argument we can see that,
\[ X_0 \cong \Proj{\R[x_0, x_1, x_2]/(x_0^2 + x_1^2 + x_2^2)} \]
Notice this has no $\R$-points but it does have $\C$-points and $X_0(\C) \cong \P^1_\C$ as analytic spaces and $X_0 \times_\R \Spec{\C} \cong \P^1_\C$ as $\C$-schemes.
\end{enumerate}

\renewcommand{\C}{\mathcal{C}}

\subsubsection{4.8 DO!!}

(DONE, LINK TO NOTES)

\subsubsection{4.9 DO!!}

(DONE, LINK TO NOTES)

\subsubsection{4.10 FINISH!!}

Let $X \to S$ be proper and $S$ noetherian. We want to show there is a projective $S$-scheme $X'$ and a morphism $g : X' \to X$ such that there is a dense open $U \subset X$ so that $g : g^{-1}(U) \to U$ is an isomorphism.

\begin{enumerate}
\item (DO!!)

\item Consider the proper map $f : X \to S$. Locally, let $V = \Spec{A} \subset S$ be an affine scheme and (DO!!)

\item Let $U = \bigcap U_i$ and consider the map,
\[ U \to X \times_S P_1 \times_S \cdots \times_S P_n \]
given by $U \to X$ and $U \to P_i$. Let $X'$ be the scheme theoretic image and $g : X' \to X$ the projection onto the first factor and $h : X' \to P$ projection onto the second factor where $P = P_1 \times_S \cdots \times_S P_n$. Because $X \to S$ is proper, the base change $X \times_S P \to P$ is also proper and therefore $h : X' \to P$ is closed. Furthermore, $U \to X \times_S P \to P$ is an immersion because each $U \to P_i$ is an open immersion then applying Lemma \ref{product_of_immersion}.  

\item Furthermore, there is a diagram,
\begin{center}
\begin{tikzcd}
U \arrow[r] \arrow[rrd] & X' \arrow[r, hook] & X \times_S P \arrow[d]
\\
& & X
\end{tikzcd}
\end{center}
Where $U \to X$ is an open immersion. Thus, if we restrict to $U \subset X$ the composition becomes an isomorphism and since scheme theoretic immage commutes with flat base change we see that $g^{-1}(U)$ is the scheme theoretic immage of $U \to U \times_S P$ which is isomorphic to $U$. 
\end{enumerate}

\subsubsection{4.11 DO!!}

\begin{enumerate}
\item Let $(\struct{}, \m)$ be a noetherian local domain with quotient field $K$ and let $L / K$ be a finitely generated field extension. 

\item Let $f : X \to Y$ be a morphism of finite type of noetherian schemes. Clearly valuative criteria imply valuative criteria for discrete valuation rings. First consider the uniqueness case,
\begin{center}
\begin{tikzcd}
\Spec{K} \arrow[r] \arrow[d] & X \arrow[d, "f"]
\\
\Spec{R} \arrow[ru, dashed] \arrow[r] & Y
\end{tikzcd}
\end{center}
where $R$ is a valuation ring and $K = \Frac{R}$.
Then there exists a discrete valuation ring $R' \subset K$ dominating $R$ i.e. $R \subset R'$ is a local map. Then the diagram becomes,
\begin{center}
\begin{tikzcd}
\Spec{K} \arrow[rr] \arrow[d] & & X \arrow[d, "f"]
\\
\Spec{R'} \arrow[r] & \Spec{R} \arrow[ru, dashed] \arrow[r] & Y
\end{tikzcd}
\end{center}
therefore we get two map $\Spec{R'} \to X$ making the diagram commute so by the uniqueness discrete valuative criterion there is a unique map $\Spec{R'} \to X$ corresponding to a local map $\stalk{X}{x} \to R \to R'$ since $R \to R'$ is injective this implies uniqueness of the map local map $\stalk{X}{x} \to R$. (CHECK THIS MAKE IT BETTER!!)

 The map $\Spec{K} \to X$ selects a point $x \in X$ and gives a map $\Spec{K} \to \Spec{L}$ where $L = \kappa(x)$ thus $K / L$ is a field extension.
\end{enumerate}

\subsubsection{4.12 DO!!}

Let $k$ be algebraically closed.

\begin{enumerate}
\item Let $K$ be a function field of dimension $1$ over $k$. Then choose a transendental $t \in K$ then $K$ is finite over $F = k(t)$. Let $R \subset K$ be a valuation ring. Then $R \cap F \subset F$ is a valuation ring which we know has the form (WHAT FORM) (WHY IS EXTENSION IMPLY $R$ IS DISCRETE).

\item Let $K/k$ be a function field of dimension two. Suppose that $X$ is a complete nonsingular surface with function field $K$.
\begin{enumerate}
\item If $Y \subset X$ is an irreducible curve with generic point $x_1$ then $R = \stalk{X}{x_1} \subset K$ then clearly $R$ is 
\end{enumerate}
\end{enumerate}

\subsection{5}

\subsubsection{5.1}

Let $(X, \struct{X})$ be a ringed space and $\E$ a locally free $\struct{X}$-module of finite rank. Define $\E^\vee = \shHom{\struct{X}}{\E}{\struct{X}}$. 

\begin{enumerate}
\item There is a natural map $\E \to (\E^\vee)^\vee$ defined by sending a section $s$ to $\varphi_s : \E^\vee \to \struct{X}$ defined via $\varphi_s(\psi) = \psi(s)$. Locally,  $\E|_U \cong \struct{U}^{\oplus n}$ so we need to check that the above map is an isomorphism for $\E = \struct{X}^{\oplus n}$. Indeed, $\shHom{\struct{X}}{\shHom{\struct{X}}{\struct{X}^{\oplus n}}{\struct{X}}}{\struct{X}} = \struct{X}$ because a map $\varphi : \struct{X}^{\oplus n} \to \struct{X}$ is determined by a section of $\struct{X}^{\oplus n}$ (where each basis is sent) via $\varphi \mapsto (\varphi(e_i))$. Under this identification the map in question becomes $(s_i e_i) \mapsto (\psi \mapsto \psi(s_i e_i)) \mapsto ((t_i e_i) \mapsto s_i t_i) = (s_i e_i)$ which is the identity. 

\item Let $\F$ be any $\struct{X}$-module. Consuder the map $\E^\vee \otimes_{\struct{X}} \F \to \shHom{\struct{X}}{\E}{\F}$ defined (on the tensor presheaf then sheafified) via $\varphi \otimes f \mapsto (s \mapsto \varphi(s) \cdot f)$. Locally $\E|_U \cong \struct{U}^{\oplus n}$ so we need to check this map is an isomorphism when $\E = \struct{X}^{\oplus n}$. However, $\shHom{\struct{X}}{\struct{X}^{\oplus n}}{\F} = \F^{\oplus n}$ via evaluating on the basis and $\E^\vee \otimes_{\struct{X}} \F = \F^{\oplus n}$ by multiplication. Under these identification the map becomes $(f_i) \mapsto e^i \otimes f_i \mapsto ((s_i e_i) \mapsto s_i f_i) \mapsto (f_i)$ which is the identity. 

\item Let $\F$ and $\G$ be $\struct{X}$-modules. Consider the map,
\[ \Hom{\struct{X}}{\F \otimes_{\struct{X}} \E}{\G} \to \Hom{\struct{X}}{\F}{\shHom{\struct{X}}{\E}{\F}} \]
via $F \mapsto (f \mapsto (s \mapsto F(f \otimes s)))$. Locally $\E|_U \cong \struct{U}^{\oplus n}$ so we need to check this for $\E = \struct{X}^{\oplus n}$. However,
\[ \Hom{\struct{X}}{\F \otimes_{\struct{X}} \struct{X}^{\oplus n}}{\G} = \Hom{\struct{X}}{\F^{\oplus n}}{\G} = \bigoplus_{i = 1}^n \Hom{\struct{X}}{\F}{\G} \]
Likewise,
\[ \Hom{\struct{X}}{\F}{\shHom{\struct{X}}{\struct{X}^{\oplus n}}{\G}} = \Hom{\struct{X}}{\F}{\G^{\oplus n}} = \bigoplus_{i = 1}^n \Hom{\struct{X}}{\F}{\G} \]

\item Consider a morphism $f : (X, \struct{X}) \to (Y, \struct{Y})$ of ringed spaces. Let $\E$ be a finite locally free $\struct{Y}$-module and $\F$ a $\struct{X}$-module. Consider the natural map $f^* f_* \F \to \F$ then tensoring with $f^* \E$ gives a natural map $f^* f_* \F \otimes_{\struct{X}} f^* \E = f^*(f_* \F \otimes_{\struct{Y}} \E) \to \F \otimes_{\struct{X}} f^* \E$. By adjunction, this is equivalent to giving a natural map $f_* \F \otimes_{\struct{Y}} \E \to f_*(\F \otimes_{\struct{X}} f^* \E)$. To show this is an isomorphism it suffices to check locally on $Y$ where $\E|_U \cong \struct{U}^{\oplus n}$ so we can reduce to the case that $\E = \struct{Y}^{\oplus n}$. Then,
\[ f_* \F \otimes_{\struct{Y}} \E = (f_* \F)^{\oplus n} = f_* (\F \otimes_{\struct{X}} f^* \E) \]
because $f^* \struct{Y} = \struct{X}$.
\end{enumerate}

\subsubsection{5.2}

Let $R$ be a DVR with $K = \Frac{R}$ and $X = \Spec{R}$. A $\struct{X}$-module $\F$ on $X$ has two pieces of data, an $R$-module $M = \F(X)$ and a $K$-module $L = \F(D(\varpi))$ where $\varpi$ is a uniformizer because these are the only two open sets. Furthermore, there is a restriction map $\rho : M \to L$ compatible with $R \to K$ which is equivalent to giving a map $\rho : M \otimes_R K \to L$ of $K$-modules.
\bigskip\\
Now $\F$ is quasi-coherent iff $\F = \wt{M}$ iff $\rho : M \to L$ is localization at $(0)$ i.e. $\rho : M_{(0)} = M \otimes_R K \to L$ is an isomorphism. 

\subsubsection{5.3}

Let $X = \Spec{A}$ be an affine scheme. Let $M$ be an $A$-module and $\F$ be a sheaf of $\struct{X}$-modules. Then there is a map,
\[ \Phi : \Hom{\struct{X}}{\wt{M}}{\F} \to \Hom{A}{M}{\Gamma(X, \F)} \]
defined by sending a sheaf map $f : \wt{M} \to \F$ to its value on global sections $f_X : M \to \Gamma(X, \F)$. Given a map $\varphi : M \to \Gamma(X, \F)$ we define a $\struct{X}$-module map $f_\varphi : \wt{M} \to \F$ on $D(s)$ via $f_{\varphi}(m/s) = s^{-1} \cdot \res(\varphi(s))$ since $\struct{X}(D(s)) = A_s$ acts on $\F(D(f))$. This construction $\Psi$ is clearly right inverse to $\Phi$. Furthermore, given a morphism of $\struct{X}$-modules $f : \wt{M} \to \F$ over $D(s)$ we see that $f_{D(s)}(m/s) = s^{-1} f_{D(s)}(m/1) = \res(f_X(m))$ and thus $\Psi$ is left inverse to $\Phi$ so $\Phi$ is an isomorphism. Clearly, the isomorphisms,
\[ \Hom{\struct{X}}{\wt{M}}{\F} = \Hom{A}{M}{\Gamma(X, \F)} \]
are natural so $\wt{(-)} : \Mod{A} \to \QCoh{X}$ is left adjoint to $\Gamma$ (in fact it is adjoint on the entire category of $\struct{X}$-modules).

\subsubsection{5.4 DO!!}

\subsubsection{5.5}

\begin{enumerate}
\item Consider $f : \A^1_k \to \Spec{k}$ then $f_* \struct{\A^1_k} = k[x]$ as a $k$-module which is not finite and thus not coherent.
\item Let $\iota : Z \embed X$ be a closed immersion. Choose an affine open $U \subset X$ with $U = \Spec{A}$. Then $Z \cap U = V(I)$ where $\wt{I} = \ker{(\struct{X} \to \iota_* \struct{Z})}|_U$ which is a module because it is quasi-coherent and $U$ is affine. Thus $\iota$ is affine and locally given by $A \to A/I$ which is clearly finite so $\iota$ is finite.
\item Let $f : X \to Y$ be a finite morphism of noetherian schemes and $\F$ a coherent $\struct{X}$-module. Cover $Y$ be affines $V_i = \Spec{A_i}$ such that $f^{-1}(V_i) = U_i = \Spec{B_i}$. Then $\F|_{U_i} = \wt{M_i}$ and $(f_* \F)|_{V_i} = \wt{(M_i)_{A_i}}$ where $M_i$ is a finite $B_i$-module. However, the map $A_i \to B_i$ is finite so $(M_i)_{A_i}$ is a finite $A_i$-module and thus $f_* \F$ is coherent.
\end{enumerate}

\subsubsection{5.6}

\begin{enumerate}
\item Let $A$ be a ring and $M$ an $A$-module. Let $X = \Spec{A}$ and $\F = \wt{M}$. For any $m \in M = \Gamma(X, \F)$ consider,
\[ \Supp{}{m} = \{ \p \in X \mid \nexists s \in A \setminus \p : s m = 0 \} = \{ \p \in X \mid \p \supset \Ann{A}{m} \} = V(\Ann{A}{m}) \]
\item Suppose that $A$ is Noetherian and $M$ is finitely generated. Then,
\[ \Supp{\struct{X}}{\F} = \{ \p \in X \mid \F_\p \neq 0 \} = \Supp{A}{M} \]
If $s \in A \setminus \p \cap \Ann{A}{M}$ then $M_\p = 0$ since $s \cdot m = 0$ for all $m \in M$ so $\Supp{A}{\F} \subset V(\Ann{A}{M})$. Furthermore, if $M_\p = 0$ then for each generator $e_i$ there exits $s_i \in A \setminus \p$ such that $s_i \cdot e_i = 0$. Then $s = s_1 \cdots s_n \in \Ann{A}{M}$ so $\p \not\supset \Ann{A}{M}$ and thus,
\[ \Supp{A}{M} = V(\Ann{A}{M}) \]
\item Let $\F$ be a coherent sheaf on $X$ a Noetherian scheme. Then affine locally on $U = \Spec{A}$ we have $\F|_U = \wt{M}$ and $\Supp{\struct{X}}{\F} \cap U = V(\Ann{A}{M})$ is closed so $\Supp{\struct{X}}{\F}$ is closed.
\item Let $\a \subset A$ be an ideal and define the submodule of $M$,
\[ \Gamma_\a(M) = \{ m \in M \mid \a^n m = 0 \text{ for some } n > 0 \} \]
Assume that $A$ is noetherian and $M$ is $A$-module and let $Z = V(\a)$ and $\F = \wt{M}$. Since,
\begin{center}
\begin{tikzcd}
0 \arrow[r] & \H_Z^0(\F) \arrow[r] & \F \arrow[r] & j_* (\F|_U) 
\end{tikzcd}
\end{center}
since $\F$ is coherent we see that $\H^0_Z(\F)$ is quasi-coherent. Furthermore, 
\begin{align*}
\Gamma_Z(\F) & = \{ m \in M \mid m_\p \neq 0 \implies \p \in Z \} = \{ m \in M \mid \p \supset \Ann{A}{m} \implies \p \supset \a \} 
\\
& = \{ \m \in M \mid \sqrt{\Ann{A}{m}} \supset \a \} = \{ m \in M \mid \a^n m = 0 \text{ for some } n > 0 \} 
\\
& = \Gamma_\a(M)
\end{align*}
The last equality follows because $A$ is Noetherian: if $\sqrt{\Ann{A}{m}} \supset \a$ then for all $a \in \a$ we have $a^n \in \Ann{A}{m}$ for some $n > 0$ and thus $a^n \cdot m = 0$ meaning $\a^n \cdot m = 0$ because $\a$ is finitely generated. Likewise, if $\a^n \cdot m = 0$ for some $n > 0$ then $\a^n \subset \Ann{A}{m}$ so $\a \subset \sqrt{\Ann{A}{m}}$. Thus, since $\H^0_Z(\F)$ is quasi-coherent and has global sections $\Gamma_Z(\F)$ we see that $\H^0_Z(\F) = \wt{\Gamma_{\a}(M)}$.

\item Let $X$ be a noetherian scheme and $Z \subset X$ a closed subset. Let $\F$ be quasi-coherent (resp. coherent). From the exact sequence,
\begin{center}
\begin{tikzcd}
0 \arrow[r] & \H_Z^0(\F) \arrow[r] & \F \arrow[r] & j_* (\F|_U) 
\end{tikzcd}
\end{center}
it is immediate that $\H^0_Z(\F)$ is quasi-coherent as long as $j_* (\F|_U)$ is quasi-cohernet. However, since $X$ is Noetherian $j : U \to X$ is quasi-compact (i.e. $U$ is retrocompact) and separated so indeed $j_* (\F|_U)$ is quasi-coherent. If furthermore $\F$ is coherent then the quasi-coherent submodule $\H^0_Z(\F) \subset \F$ is coherent since $X$ is noetherian.

\end{enumerate}

\subsubsection{5.7}

Let $X$ be a Noetherian scheme and $\F$ a coherent sheaf on $X$.

\begin{enumerate}
\item Suppose that the stalk $\F_x$ is a free $\stalk{X}{x}$-module for some $x$. We may reduced to an affine open $X = \Spec{A}$ with $A$ noetherian (since $X$ is a Noetherian scheme) and $\F = \widetilde{M}$ for some finitely generated $A$-module $M$ since $\F$ is coherent. Suppose that $M_\p$ is a free $A_\p$-module for some prime $\p \subset A$. Let $e_1, \dots, e_r$ be an $A_\p$ basis of $M_\p$ which we may choose to be elements of $M$ since we may reintroduce denominators via multiplication by $A_\p$. Now consider the exact sequence,
\begin{center}
\begin{tikzcd}
0 \arrow[r] & \ker{E} \arrow[r] & A^{\oplus r} \arrow[r, "E"] & M \arrow[r] & \coker{E} \arrow[r] & 0 
\end{tikzcd}
\end{center}
where $E(a_1, \dots, a_r) = a_1 e_1 + \cdots + a_r e_r$. However, $\coker{E}$ is finitely genrerated because $M$ is and $\ker{E} \subset A^{\otimes r}$ is finitely generated because $A$ is Noetherian. Furthermore, we know that,
\begin{center}
\begin{tikzcd}
0 \arrow[r] & (\ker{E})_\p \arrow[r] & A_\p^{\oplus r} \arrow[r, "E"] & M_\p \arrow[r] & (\coker{E})_\p \arrow[r] & 0 
\end{tikzcd}
\end{center}
remains exact and $A_\p^{\oplus r} \xrightarrow{\sim} M_\p$ is an isomorphism so $(\ker{E})_\p = (\coker{E})_\p = 0$. Therefore, there exists some $f \notin \p$ such that $(\ker{E})_f = (\coker{E})_f = 0$ since they are finitely generated (take the products of elements in $A \setminus \p$ killing their generating sets). Now localizing the exact sequence, we get an exact sequence,
\begin{center}
\begin{tikzcd}
0 \arrow[r] & (\ker{E})_f \arrow[r] & A_f^{\oplus r} \arrow[r, "E"] & M_f \arrow[r] & (\coker{E})_f \arrow[r] & 0 
\end{tikzcd}
\end{center}
but $(\ker{E})_f = (\coker{E})_f = 0$ so $A_f^{\oplus r} \to M_f$ is an isomorphism. Therefore, 
\[ \F|_{D(f)} = \widetilde{M_f} = \widetilde{A_f^{\oplus n}} = \struct{X}|_{D(f)}^{\oplus n} \] 
is a free sheaf. 

\item Suppose that $\F_x$ is a free $\stalk{X}{x}$-module for each $x \in X$. Then, by above, there exsits an open cover of $X$ on which $\F$ is free so $\F$ is a locally-free sheaf. Conversely, if $\F$ is a locally-free sheaf. Then for each $x \in X$ there exists an open neighbrohood with $x \in U$ such that $\F|_U \cong \struct{X} |_U^{\oplus n}$. Then the induced map $\F_x \cong \stalk{X}{x}^{\oplus n}$ is an isomorphism so $\F_x$ is a free $\stalk{X}{x}$-module for each $x \in X$. 

\item A invertible sheaf is a locally free sheaf of rank $1$. First, suppose there exists a coherent sheaf $\G$ such that $\F \otimes_{\struct{X}} \G = \struct{X}$. Then for each $x \in X$ we have,
\[ \F_x \otimes_{\stalk{X}{x}} \G_x = \stalk{X}{x} \]
Since $\stalk{X}{x}$ is local, Lemma \ref{tensor_inverse} implies that $\F_x \cong \stalk{X}{x}$ and $\G_x \cong \stalk{X}{x}$ for each $x \in X$. Therefore, $\F$ is is locally free of rank $1$.
\bigskip\\
Conversely, suppose that $\F$ is an invertible $\struct{X}$-module. Consider the dual module,
\[ \F^* = \shHom{\struct{X}}{\F}{\struct{X}} \]
and then the evaluation map $\ev : \F \otimes_{\struct{X}} \F^* \to \struct{X}$ which is a morphism of $\struct{X}$-modules.
Consider the induced map on stalks $\F_x \otimes_{\stalk{X}{x}} \F^*_x \to \stalk{X}{x}$. Since $\F$ is invertible, $\F_x \cong \stalk{X}{x}$. By Hartshorne III 6.8 we know that,
\[ \F^*_x = \Homover{\stalk{X}{x}}{\F_x}{\stalk{X}{x}} \cong \Homover{\stalk{X}{x}}{\stalk{X}{x}}{\stalk{X}{x}} = \stalk{X}{x} \] and thus, $\ev_x(r \otimes (1 \mapsto r')) = rr'$ gives the natural map $\stalk{X}{x} \otimes_{\stalk{X}{x}} \stalk{X}{x} \to \stalk{X}{x}$ which is an isomorphism. Thus $\ev$ is an isomorphism since it is on the stalks. Therefore, 
\[ \F \otimes_{\struct{X}} \F^* = \struct{X} \]
\end{enumerate}


\subsubsection{5.8}

Let $X$ be a noetherian scheme and $\F$ a coherent sheaf. Recall that,
\[ \rank_{x}(\F) = \dim_{\kappa(x)} \F_x \otimes_{\stalk{X}{x}} \kappa(x) \]
where $\kappa(x) = \stalk{X}{x} / \m_x$. 

\begin{enumerate}
\item Choose an affine open $U = \Spec{A}$ such that $\F|_U = \wt{M}$ for some $A$-module $M$. For a point $x \in U$ suppose $\rank_{x}(\F) = n$ then via Nakayama we can lift a basis of $M_\p \otimes_{A_\p} A_\p / \p A_\p$ to a generating set of $M_\p$. Clearing denominators we get an exact sequence, 
\begin{center}
\begin{tikzcd}
0 \arrow[r] & \ker{f} \arrow[r] & A^n \arrow[r, "f"] & M \arrow[r] & \coker{f} \arrow[r] & 0
\end{tikzcd}
\end{center}
However, localizing at $\p$ we get a surjection $A_\p^n \onto M_\p$ and thus $(\coker{f})_\p = 0$. However, $\coker{f}$ is finitely generated so there exists some $g \in A$ such that $(\coker{f})_g = 0$ and thus we get an exact sequence,
\begin{center}
\begin{tikzcd}
0 \arrow[r] & (\ker{f})_g \arrow[r] & A_g^n \arrow[r] & M_g \arrow[r] & 0
\end{tikzcd}
\end{center}
Therefore $M_g$ is generated by $n$ elements so for each $\q \in D(f) \subset U$ there is a surjection $\kappa(\q)^n \onto M_\q \otimes_{A_\q} \kappa(\q)$ so $\rank_\q(\F) \le n$ on $D(f)$. Therefore, for each $x \in X$ there exists a neighborhood $x \in U \subset X$ such that $\forall y \in U : \rank_y(\F) \le \rank_x(\F)$ thus, in particular, $\{ x \in X \mid \rank_x(\F) < n \}$ is open for any $n \in \Z$ proving that rank is upper semi-continuous.

\item Suppose further that $\F$ is locally free. Then locally $\F|_U \cong \struct{U}^{\oplus n}$ showing that $\rank_x{\F} = n$ for all $x \in U$ i.e. $\rank{\F}$ is locally constant. Then for any $n \in \Z$ the set $S_n = \{ x \in X \mid \rank_x(\F) = n \}$ is clopen because if $x \in S_n$ then $x \in U \subset S_n$ and if $x \notin S_n$ then $x \in U \subset S_n^C$. Thus if $X$ is connected then $S_n = X$ (choose $x \in X$ and set $n = \rank_x(\F)$ so $x \in S_n$ is not empty) so $\rank{\F} = n$ is constant.

\item Now suppose that $X$ is reduced and $\rank(\F)$ is constant. Choose a point $x \in X$ such that $\rank_x(\F) = n$ and an affine open neighborhood $x \in U = \Spec{A}$. Then $\F|_U = \wt{M}$ for some finite $A$-module $M$. Let $\p \subset A$ correspond to $x \in U$. Proceeding as above we lift a basis of $M \otimes_A (A/\p)_\p$ to $A^n \onto M$ (localizing if necessary) to get an exact sequence,
\begin{center}
\begin{tikzcd}
0 \arrow[r] & \ker{f} \arrow[r] & A^n \arrow[r] & M \arrow[r] & 0
\end{tikzcd}
\end{center}
For any $\q \in U$, tensoring by $\kappa(\q)$ gives a surjection $\kappa(\q)^n \onto M \otimes_{A} \kappa(\q)$. However, by assumption, this vectorspace has dimension $n$ since $\F$ has constant rank so
\[ f \otimes \id : A^n \otimes_{A} \kappa(\q) \to M \otimes_{A} \kappa(\q) \]
is an isomorphims for each $\q \in U$. 
Now using the next following lemma, we conclude that $\ker{f} \subset \q \cdot A^n$ for each $\q \in \Spec{A}$ and thus $\ker{f} \subset \nilrad{A} \cdot A^n = 0$. Thus, $M = A^n$ so $\F$ is locally free.

\begin{lemma}
Let $f : M \to N$ be a map of $A$-modules with $N$ finite and $\p \subset A$ prime such that $f \otimes \id : M \otimes \kappa(\p) \to N \otimes \kappa(\p)$ is an isomorphism. Then $\ker{f} \subset \p \cdot M$ and $(\coker{f})_\p = 0$.
\end{lemma}

\begin{proof}
If $m \in \ker{f}$ then $\bar{m} \in M/\p M$ is sent to zero under $f \otimes \id$ and thus $\bar{m} = 0$ so $m \in \p M$. Furthermore, by left-exactness $(\coker{f})_\p \otimes_{A_\p} \kappa(\p) = 0$ but $A_\p$ is local and $\coker{f}$ is finitely generated so $(\coker{f})_\p = 0$. 
\end{proof}
\end{enumerate}

\subsubsection{5.9 DO!!} 

Let $S$ be a graded ring generated by $S_1$ as an $S_0$-algebra and let $M$ be a graded $S$-module and $X = \Proj{S}$.

\begin{enumerate}
\item It suffices to produce a map $M_0 \to \Gamma(\wt{M})$. However, $\wt{M}(D_{+}(f)) = M_{(f)}$ and there is a map $M_0 \to M_{(f)}$ compatible with further localization. Thus these local sections glue giving a map $M_0 \to \Gamma(\wt{M})$.
\item Assume that $S_0 = A$ is a finitely generated $k$-algebra and $S_1$ is a finite $A$-module. Suppose that $M$ is a finitely generated graded $M$-module. (FINISH!!)

\item (FINISH!!)
\end{enumerate}

\subsubsection{5.10}

Let $A$ be a ring and $S = A[x_0, \dots, x_r]$ and $X = \Proj{S}$. 

\begin{enumerate}
\item For any homogeneous ideal $I \subset S$, we define the \textit{saturation},
\[ \overline{I} = \{ s \in S \mid \forall i : \exists n : x^n_i s \in I \} \]
and say that $I$ is \textit{saturated} if $\overline{I} = I$. Suppose that $s \in \overline{I}$. Write $s = s_0 + \cdots + s_d$ as a sum of homogeneous elements. Then $x_i^n(s_0 + \cdots + s_d) \in I$ but each term is still homogeneous and $I$ is a homogeneous ideal so $x_i^n s_k \in I$ and thus $s_k \in \overline{I}$ for each $k$. Thus $\overline{I}$ is homogeneous.

\item Notice that $\overline{I}_{(x_i)} = I_{(x_i)}$ because for any element $s/x_i^m \in \overline{I}_{(x_i)}$ we have $s/x_i^m = (x^n_i s)/x_i^{n+m}$ but $x^n_i s \in I$. Therefore, $\wt{\overline{I}} = \wt{I}$ because the natural map $\wt{I} \to \wt{\overline{I}}$ is an isomorphism on the open cover $D_{+}(x_i)$. In particular, $I$ and $\overline{I}$ define the same closed subscheme since this subscheme is determined by $\wt{I}$ so clearly if $I_1$ and $I_2$ have the same saturation they define the same closed subscheme. Conversely, suppose that $I_1, I_2 \subset S$ define the same closed subscheme $Y$. Then we know that $\I_Y \cong \wt{I_1} \cong \wt{I_2}$. In particular, we must have $(I_1)_{(x_i)} = (I_2)_{(x_i)}$ which implies that the saturations of $I_1$ and $I_2$ are equal.

\item Let $Y \subset X$ be a closed subscheme and define $I = \Gamma_*(\I_Y)$. Take $s \in S$ and suppose that $x_i^n s \in I$ for each $i$. Splitting up $s$ we may assume it is homogeneous of degree $d$ i.e. $s \in S_d$. Then $x_i^n s$ is a global section of $\I_Y \otimes_{\struct{X}} \struct{X}(d+n)$ 
However, $x_i^n$ globally generate $\struct{X}(n)$ so the induced map,
\[ \struct{X} \to \bigoplus_{i = 0}^r \struct{X}(n) \]
is injective. Therefore there is a morphism of exact sequences,
\begin{center}
\begin{tikzcd}
0 \arrow[r] & \I_Y(d) \arrow[d, "x_i^n", hook] \arrow[r] & \struct{X}(d) \arrow[r] \arrow[d, "x_i^n"] & \struct{Y}(d) \arrow[d, "x_i^n", hook] \arrow[r] & 0
\\
0 \arrow[r] & \bigoplus\limits_{i = 0}^r \I_Y(d+n) \arrow[r] & \bigoplus\limits_{i = 0}^r \struct{X}(d+n) \arrow[r] & \bigoplus\limits_{i = 0}^r \struct{Y}(d+n) \arrow[r] & 0
\end{tikzcd}
\end{center}
where the rightmost downward arrow is surjective because $x_i^n$ globally generate $\struct{Y}(n)$ as a $\struct{Y}$-module.  Thus if $x^n_i s$ is in the kernel for each $i$ then $s \in \Gamma(X, \I_Y(d))$ so $s \in I$ and thus $I$ is saturated.

\item There is a correspondence,
\[ \{ \text{closed subschemes } Y \subset X \} \leftrightarrow \{ \text{saturated ideals } I \subset S \} \]
given by $Y \mapsto \Gamma_*(\I_Y)$ which we showed is saturated and $I \mapsto \Proj{S/I}$. Furthermore, if $I_1$ and $I_2$ define the same subscheme $Y$ then their saturations are equal. In particular if $I_1$ and $I_2$ are saturated then $I_1 = I_2$ so this mapping is injective. Furthermore, we know that every closed subscheme arises from some ideal and thus from its saturation so $I \mapsto \Proj{S/I}$ is a bijection. Furthermore, $Y \mapsto \Gamma_*(\I_Y)$ is its inverse since $\Gamma_*(\wt{I})$ is the saturation of $I$ and $\Gamma_*(\I_Y)$ defines $Y$ as proven before (CHECK THIS).
\end{enumerate}

\subsubsection{5.11 DO!!}

Let $S$ and $T$ be graded rings with $S_0 = T_0 = A$ and define $S \times_A T$ as the graded ring,
\[ (S \times_A T)_n = S_n \otimes_A T_n \]
Let $X = \Proj{S}$ and $Y = \Proj{T}$. (LOOK AT FUNCTORIALITY OF PROJ) 

\subsubsection{5.12}

\begin{enumerate}
\item Let $X$ be a scheme over $Y$ and let $\L$ and $\M$ be two very ample invertible sheaves on $X$. These determine immersions $\iota_1 : X \embed \P^n_Y$ and $\iota_2 : X \embed \P^m_Y$. Consider the diagram,
\begin{center}
\begin{tikzcd}
& \P^n_Y 
\\
X \arrow[r, dashed] \arrow[ru, hook] \arrow[rd, hook] & \P^n_Y \times_Y \P^m_Y \arrow[u, "\pi_1"] \arrow[d, "\pi_2"] \arrow[r, hook] & \P^{N}_Y
\\
& \P^m_Y
\end{tikzcd}
\end{center}
where $s : \P^n_Y \times_Y \P^m_Y \embed \P^{N}_Y$ is the Segre embedding which satisfies,
\[ s^* \struct{\P^{N}_Y}(1) = \pi_1^* \struct{\P^n_Y}(1) \otimes_{\struct{}} \pi_2^* \struct{\P^m_Y}(1) \]
Now, consider,
\begin{align*}
(\iota_1, \iota_2)^* s^* \struct{\P^N_Y}(1) & = (\iota_1, \iota_2)^* [\pi_1^* \struct{\P^n_A}(1) \otimes_{\struct{}} \pi_2^* \struct{\P^m_A}(1)] 
\\
& = [(\iota_1, \iota_2)^* p_1^* \struct{\P^n_A}(1)] \otimes_{\struct{}} [(\iota_1, \iota_2)^* p_2^* \struct{\P^m_A}(1)] 
\\
& = [\pi_1 \circ (\iota_1, \iota_2)]^* \struct{\P^n_A}(1) \otimes_{\struct{}} [\pi_2 \circ (\iota_1, \iota_2)]^* \struct{\P^m_A}(1)
\\
& = \iota_1^* \struct{\P^n_A}(1) \otimes_{\struct{X}} \iota_2^* \struct{\P^m_A}(1)
\\
& = \L \otimes_{\struct{}} \M 
\end{align*}
Now we need to show that $(\iota_1, \iota_2) \circ \Delta$ is an immersion which follows because we can factor it as $\Gamma_{\iota_2} : X \to X \times_Y \P^m_Y$ then $\iota_1 \times \id : X \times \P^m_Y = \P^n_Y \times_Y \P^n_Y$. However, $\iota_1 \times \id$ is an immersion by base change and $\Gamma_{\iota_2}$ is a closed embedding because $\P^m_k$ is separated.

\item Let $f : X \to Y$ and $g : Y \to Z$ be two morphisms of schemes. Let $\L$ be a very ample invertible sheaf on $X$ relative $Y$, and let $\M$ be a very ample invertible sheaf on $Y$ relative to $Z$. These define closed immersions $\iota_1 : X \embed \P^n_Y$ and $\iota_2 : Y \embed \P^m_Z$.
Consider the following diagram,
\begin{center}
\begin{tikzcd}
& \P^n_Y \arrow[r, "\id \times g"] & \P^n_Z
\\
X \arrow[ru, "\iota_1", hook] \arrow[rd, "f"'] \arrow[rr, dashed, "q"] & & \P^n_Z \times_Z \P^m_Z \arrow[u, "\pi_1"'] \arrow[d, "\pi_2"] \arrow[r, hook] & \P^N_Z
\\
& Y \arrow[r, "\iota_2", hook] & \P^m_Z 
\end{tikzcd}
\end{center}
where $s : \P^n_Z \times_Z \P^m_Z \embed \P^{N}_Z$ is the Segre embedding which satisfies,
\[ s^* \struct{\P^{N}_Z}(1) = \pi_1^* \struct{\P^n_Z}(1) \otimes_{\struct{}} \pi_2^* \struct{\P^m_Z}(1) \]
Therefore,
\begin{align*}
(s \circ q)^* \struct{\P^N_Z}(1) & = q^* s^* \struct{\P^N_Z}(1) = q^* [\pi_1^* \struct{\P^n_Z}(1) \otimes_{\struct{}} \pi_2^* \struct{\P^m_Z}(1)] 
\\
& = (\pi_1 \circ q)^* \struct{\P^n_Z}(1) \otimes_{\struct{}} (\pi_2 \circ q)^* \struct{\P^m_Z}(1)
\\
& = ((\id \times g) \circ \iota_1)^* \struct{\P^n_Z}(1) \otimes_{\struct{}} (\iota_2 \circ f)^* \struct{\P^m_Z}(1)
\\
& = \iota_1^* (\id \times g)^* \struct{\P^n_Z}(1) \otimes_{\struct{}} f^* \iota_2^* \struct{\P^m_Z}(1)
\\
& = \iota_1^* \struct{\P^n_Y}(1) \otimes_{\struct{}} f^* \M 
\\
& = \L \otimes_{\struct{}} f^* \M
\end{align*}
Thus it suffices to show that $q$ is an immersion. We can factor the morphism as,
\[ X \xrightarrow{\iota_1} \P^n_Y \xrightarrow{\iota_2'} \P_Z^n \times_Z \P^m_Z \] where the second map is the base change,
\begin{center}
\begin{tikzcd}
\P_Y^n \arrow[d] \arrow[r, "\iota_2'"] & \P_Z^n \times_Z \P^m_Z \arrow[d, "\pi_2"]
\\
Y \arrow[r, "\iota_2"] & \P_Z^m
\end{tikzcd}
\end{center}
then $\iota_2' = \id \times \iota_2$ and thus the diagram becomes,
\begin{center}
\begin{tikzcd}
& \P^n_Y \arrow[r, "g'"] \arrow[rd, "\iota_2'"', hook] & \P^n_Z
\\
X \arrow[ru, "\iota_1", hook] \arrow[rd, "f"'] \arrow[rr, dashed, "q"] & & \P^n_Z \times_Z \P^m_Z \arrow[u, "\pi_1"'] \arrow[d, "\pi_2"] \arrow[r, hook] & \P^N_Z
\\
& Y \arrow[r, "\iota_2", hook] & \P^m_Z 
\end{tikzcd}
\end{center}
where the top triangle is the base change via $\P^n_Z \to Z$ of
\begin{center}
\begin{tikzcd}
Y \arrow[rd, "g"'] \arrow[r, "\iota_2", hook] & \P^m_Z \arrow[d]
\\
& Z
\end{tikzcd}
\end{center}
Since $\iota_1$ is an immersion and $\iota_2'$ is the base change of a closed immersion and thus a closed immersion as well we see that $q = \iota_2' \circ \iota_1$ is an immersion.

\end{enumerate}

\subsubsection{5.13 DO!!}

\subsubsection{5.14 DO!!}

\subsubsection{5.15 DO!!}

\begin{enumerate}
\item Let $X = \Spec{A}$ be a Noetherian affine scheme. Suppose that $\F = \wt{M}$ is a quasi-coherent sheaf for $M$ some $A$-module. Now the coherent subsheaves correspond to finite $A$-submodules $N \subset M$. Clearly,
\[ M = \bigcup_{N \subset M \text{ finite}} N \]
and thus $\F$ is the union of its finite submodules since union commutes with localization.

\item Let $X = \Spec{A}$ be an affine noetherian scheme and $U \subset X$ an open subset and $\F$ a coherent sheaf on $U$. Let $\iota : U \to X$ be the inclusion which is quasi-compact since $X$ is noetherian (meaning $U$ is retrocompact) and separated so $\iota_* \F$ is quasi-coherent. By (a) the quasi-coherent sheaf $\iota_* \F$ is the union of its coherent subsheaves. For any coherent subsheaf $\F' \subset \iota_* \F$ we have $\F'|_U \subset \F$ is a coherent subsheaf of $\F$. However, since $U$ is noetherian and $\F$ is coherent its sheaves satisfy the ascending chain condition (Tag 01Y8) so the union must stabilize at a finite point giving a coherent sheaf $\F'$ on $X$ such that $\F'|_U = \F$.

\item Let $X = \Spec{A}$ be an affine noetherian scheme and $U \subset X$ an open subset. Let $\F$ be a coherent sheaf on $U$ and $\G$ a quasi-coherent sheaf on $X$ such that $\F \subset \G|_U$. Consider the natural map $\rho : \G \to \iota_* (\G|_U)$. Now $\iota_* \F \subset \iota_* (\G|_U)$ so consider the quasi-coherent submodule $\rho^{-1}(\iota_* \F) \subset \G$. However, on $U$ the map $\rho : \G \to \iota_* (\G|_U)$ is an isomorphism so $\rho^{-1}(\iota_* \F) |_U = \F$. Therefore, for any coherent subsheaf $\F' \subset \rho^{-1}(\iota_* \F)$ we have $\F'|_U \subset \F$ and thus the union of coherent subsheaves stabilizes giving $\F' \subset \rho^{-1}(\iota_* \F)  \subset \G$ such that $\F'|_U = \F$.

\item Now let $X$ be any noetherian scheme, $U \subset X$ an open subset. Then let $\F$ be a coherent sheaf on $U$ and $\G$ a quasi-coherent sheaf on $X$ such that $\F \subset \G|_U$. Since $X$ is Noetherian there is a finite affine cover $U_i = \Spec{A_i}$ with $A_i$ Noetherian.
First, since $\F|_{U_1 \cap U}$ is coherent we can extend it to a coherent sheaf $\F' \subset \G|_{U_1}$ on $U_1$ and then glue to a coherent sheaf $\F'_1 \subset \G|_{U \cup U_1}$ such that $\F'_1 |_U = \F$ since $\F'|_{U_1 \cap U} = \F|_{U_1 \cap U}$. Then we have a coherent sheaf $\F'_1 \subset \G|_{U_1 \cup U}$ on $U_1 \cup U$. Repeating this process (replace $U$ by $U_1 \cup U$) we can extend to a coherent sheaf $\F'$ on $X = (U_1 \cup U_2 \cup \cdots \cup U_n) \cup U$ such that $\F'|_U = \F$ and $\F \subset \G|_U$.

\item Let $X$ be a noetherian scheme and $\F$ a quasi-coherent sheaf. To show that $\F$ is the union of its coherent subsheaves, we need to show that for each section $s \in \F(U)$ there is a coherent subsheaf $\F' \subset \F$ such that $s \in \F'(U)$. Let $\F_s \subset \F|_U$ be the image sheaf of $\struct{U} \xrightarrow{s} \F_U$ which is coherent because it is a quotient of $\struct{U}$ and $X$ is noetherian. Then by (d) there is a coherent subsheaf $\F' \subset \F$ such that $\F'|_U = \F_s$ so $s \in \F'(U)$ proving that $\F$ is the union of its coherent subsheaves.
\end{enumerate}

\subsubsection{5.16 DO!!}

PLease no

\subsubsection{5.17 DO!}

\subsubsection{5.18 DO!}

\subsection{6}

\subsubsection{6.1}

Let $X$ be a Noetherian integral separated scheme regular in codimension one. Then $X \times \P^n$ is also Noetherian integral and separated and locally $X \times \P^n$ is isomorphic to $X \times \A^n$ which is regular in codimension one by Proposition 6.6. Fix a closed immersion $\P^{n-1} \embed \P^n$ (the line at infinity) and base change to $X$ giving $X \times \P^{n-1} \embed X \times \P^n$ and write $Z = X \times \P^{n-1}$ for this closed subscheme. Then there is an exact sequence,
\begin{center}
\begin{tikzcd}
\Z \arrow[r, "1 \mapsto Z"] & \Cl{X \times \P^n} \arrow[r] & \Cl{X \times \A^n} \arrow[r] & 0 
\end{tikzcd}
\end{center}
where $U = X \setminus Z = X \times \A^n$. First we need to show that $\Z \to \Cl{X \times \P^n}$ is injective i.e. that $n [Z]$ is never a principal divisor. Let $\xi : \Spec{K} \to X$ be the generic point of $X$ then consider the map $\P^n_K \to X \times \P^n$ pulling back gives a map $\Cl{X \times \P^n} \to \Cl{\P^n_K} = \Z$ giving a diagram,
\begin{center}
\begin{tikzcd}
\Z \arrow[r, "1 \mapsto Z"] & \Cl{X \times \P^n} \arrow[d] \arrow[r] &  \Cl{X \times \A^n} \arrow[r] & 0 
\\
& \Cl{\P^n_K} \arrow[ul, "\deg"]
\end{tikzcd}
\end{center}
I claim that $\Z \to \Cl{X \times \P^n} \to \Cl{\P^n_K} \xrightarrow{\deg} \Z$ is the identity. We send $1 \mapsto [Z] \mapsto [\xi^{-1}(Z)]$ where $[\xi^{-1}(Z)] = [\P^{n-1}_K \embed \P^n_K]$ is the divisor ``at infinity'' which generates $\Cl{\P^n_K} = \Z$ because $\P^{n-1}_K$ is a prime divisor of degree $1$. Therefore $\Z \to \Cl{X \times \P^n}$ has a section meaning that it is injective and the sequence,
\begin{center}
\begin{tikzcd}
0 \arrow[r] & \Z \arrow[r, "1 \mapsto Z"] & \Cl{X \times \P^n} \arrow[r] & \Cl{X \times \A^n} \arrow[r] & 0 
\end{tikzcd}
\end{center}
splits. Finally, $\Cl{X \times \A^n} \cong \Cl{X}$ and thus the split sequence provides an isomorphism, 
\[ \Cl{X \times \P^n} \cong \Cl{X} \times \Z \]

\subsubsection{6.2 DO!!}

Let $k$ be an algebraically closed field and $X \subset \P^n_k$ be a closed subvariety regular in codimension one.

\begin{enumerate}
\item Let $V \subset \P^n_k$ be an irreducible (reduced) hypersurface (i.e. a prime divisor of $\P^n_k$) not containing $X$. Now let $Y_i$ be the irreducible components of $V \cap X$. Let $U_i$ be an affine open of $\P^n_k$ intersecting $Y_i$ then $V \cap U_i = V(f_i)$ for some $f_i \in \struct{\P^n_k}(U_i)$ because $V$ is a hypersurface. Under $\struct{\P^n_k} \to \struct{X}$ this pulls back to $\bar{f}_i \in \struct{X}(U_i \cap X)$ and we define,
\[ V \cdot X = \sum \nu_{Y_i}(\bar{f}_i) [Y_i] \]
We need to show this is independent of choices. Firstly, $f_i$ is determined up to units and thus $\nu_{Y_i}(\bar{f}_i)$ are independent of the choice of local equation. If we choose a different open patch $U_i'$ the equations are equal up to units at the generic stalk of $Y_i$. Extending linearly corresponds to alowing irreducible $f_i$ because the sum of two divisors is cut out by the product of equations and whose valuations are again summed.

\item Let $D = \div{f}$ be a principal divisor on $\P^n_k$ such that each prime component does not contain $X$. We want to show that $D \cdot X$ is principal. Write $f = \frac{g_1^{r_1} \cdots g_k^{r_k}}{h_1^{s_1} \cdots h_r^{s_p}}$ where $g_i,h_i \in k[x_0, \dots, x_n]$ are irreducible homogeneous of degree $d_i, d_i'$ such that $r_1 d_1 + \cdots + r_k d_k = s_1 d_1' + \cdots + s_p d_p'$. Then we have,
\[ \div{(f)} = \sum_{i = 1}^k d_i \cdot [V(g_i)] - \sum_{i = 1}^p d_i' \cdot [V(h_i)] \]
where the $V(g_i)$ and $V(h_i)$ are prime divisors i.e. irreducible hypersurfaces corresponding to the homogeneous polynomials. Under the map $\struct{\P^n_k} \to \struct{X}$ locally over $D(g_i)$ the element $\frac{f_i}{g_j}$ is nonzero because $X$ is not contained in either $V(f_i)$ for $V(g_j)$ and therefore $f$ maps to an element $\bar{f} \in K(X)$ and we see that $D \cdot X = \div{(\bar{f})}$ because each $f_i$ and $g_i$ reduce to local equations for the principal divisors. Hence we get a map $\Cl{\P^n_k} \to \Cl{X}$ given by $D \mapsto D \cdot X$.

\item Cover $\P^n_k$ by affine opens $U_i = \Spec{A_i}$ then locally $X \cap U_i = \Spec{A_i / I_i}$ (DOOOO!!!!)

\item Let $D$ be a principal divisor on $X$ so $D = \div{(f)}$ for $k \in K(X)$. Since locally $X \cap U = \Spec{A/I}$ we have $f \in K(A/I)$ so we can write $f = \frac{\bar{a}}{\bar{b}}$ for $a, b \in A$ and $\bar{a}, \bar{b} \in A$ the images. Therefore, we can consider $f' = \frac{a}{b} \in K(\P^n_k)$. Furthermore, let $D' = \div{(f')}$ then the local equations $a$ and $b$ reduce to $\bar{a}$ and $\bar{b}$ and therefore,
\[ D' \cdot X = D \]
proving that,
\[ \deg{D} = \deg{D' \cdot X} = (\deg{D'}) \cdot (\deg{X}) = 0 \]
because $D'$ is a principal divisor on $\P^n_k$ and thus $\deg{D'} = 0$. Therefore, we get a homomorphism $\deg : \Cl{X} \to \Z$ making the diagram,
\begin{center}
\begin{tikzcd}[row sep = large, column sep = large]
\Cl{\P^n_k} \arrow[r, "D \mapsto D \cdot X"] \arrow[d, "\deg"] & \Cl{X} \arrow[d, "\deg"]
\\
\Z \arrow[r, "\cdot \deg{X}"] & \Z
\end{tikzcd}
\end{center}
commute since $\deg{D \cdot X} = (\deg{D}) \cdot (\deg{X})$.
\end{enumerate}

\subsubsection{6.3 DO!!}

Let $V \subset \P^n_k$ be a projective variety with $\dim{V} \ge 1$ which is nonsingular in codimension $1$. Let $X = C(V)$ be the affine cone over $V$ in $\A^{n+1}_k$ and $\bar{X}$ be its projective closure in $\P^{n+1}_k$. Let $P \in X$ be the vertex of $X \subset \bar{X}$. 

\begin{enumerate}
\item Let $\pi : \overline{X} \setminus \{ P \} \to V$ be the linear projection. Let $U_i = D_{+}(x_i)$ be standard affines on $\P^n$ and $V$ is cut out by a homogeneous ideal $I \subset k[x_0, \dots, x_n]$. Then,
\[ \tilde{U}_i = U_i \cap V = \Spec{k[\tfrac{x_0}{x_i}, \dots, \tfrac{x_n}{x_i}]/I} \]
Then $C(V) = \Proj{k[x_0, \dots, x_n, x_{n+1}]/\tilde{I})}$ where $\tilde{I}$ and thus,
\begin{align*}
\pi^{-1}(U_i) & = \Spec{k[\tfrac{x_0}{x_i}, \cdots, \tfrac{x_n}{x_i}, \tfrac{x_{n+1}}{x_i}]/I} = \Spec{k[\tfrac{x_0}{x_i}, \cdots, \tfrac{x_n}{x_i}]/I \otimes_k k[\tfrac{x_{n+1}}{x_i}]} 
\\
& = \Spec{k[\tfrac{x_0}{x_i}, \cdots, \tfrac{x_n}{x_i}]/I} \times_k \Spec{k[\tfrac{x_{n+1}}{x_i}]} = U_i \times_k \A^1_k
\end{align*}
Therefore, consider the map $\Cl{V} \to \Cl{\overline{X} \setminus \{ P \}}$ (DO THIS!!)

\item We have $V \subset \overline{X}$ cut out as $V = V(x_{n+1})$. Choose a hyperplane $H$ not containing $V$. Then $H = V(f)$ is cut out by a linear form $f$. Since $H$ does not contain $V$, then $f' \frac{f}{x_{n+1}} \in K(C(V))$ is a nonzero rational function. Then $[V] + \div{(f)} = \pi^* [V \cdot H]$ because the local equation for $V \cdot H$ is $f$ which pulls back locally to $f' x_{n+1}$ inside $K(X)$ and $V(x_{n+1}) = V \subset \overline{X}$. Therefore, taking the map $X \setminus \{ P \} \embed \bar{X} \setminus \{ P \} \xrightarrow{\pi} V$, we get a map $\Cl{V} \to \Cl{X}$ and a sequence,
\begin{center}
\begin{tikzcd}
0 \arrow[r] & \Z \arrow[r, "1 \mapsto V \cdot H"] & \Cl{V} \arrow[r] & \Cl{X} \arrow[r] & 0
\end{tikzcd}
\end{center}
where $\Cl{X \setminus \{ P \}} = \Cl{X}$ because $\dim{X} \ge 2$.
The map $\Z \to \Cl{V}$ is injective because $\deg{n \cdot [V \cdot H]} = n \cdot (\deg{V}) \cdot  (\deg{H}) = n \cdot \deg{V}$ which is injective since $\deg{V} \neq 0$ meaning $\Z \to \Cl{V} \to \Z$ is injective and thus $\Z \to \Cl{V}$ is injective. Next, $X \setminus \{ P \} \embed \overline{X} \setminus \{ P \}$ is an open immersion with compliment $V$ which is a divisor and therefore the map $\Cl{\overline{X} \setminus \{ P \}} \to \Cl{X \setminus \{ P \}}$ is surjective with kernel exactly the subgroup generated by $[V]$ i.e. the image of $\Z \to \Cl{V} \to \Cl{\overline{X} \setminus \{ P \}}$. Furthermore, $\Cl{V} \to \Cl{\overline{X} \setminus \{ P \}}$ is an isomorphism implying that $\Cl{V} \to \Cl{X \setminus \{ P \}}$ is surjective with kernel exactly the image of $\Z \to \Cl{V}$ meaning the seqeunce is exact.

\item Let $S(V)$ be the homogeneous coordinate ring of $V$ and thus $X = \Spec{S(V)}$. Suppose that $S(V)$ is a UFD. Since $S(V)$ is a Noetherian domain, by Prop. 6.2 we see that $\Cl{X} = 0$ and $S(V)$ is integrally closed. Therefore by the exact sequence $\Cl{V} = \Z$ generated by the class of $V \cdot H$ and $S(V)$ being an integrally closed domain means that $V \embed \P^n_k$ is a projectively normal embedding. Conversely, if we assume that $V \embed \P^n_k$ is projectively normal, i.e. $S(V)$ is an integrally closed domain, and $\Cl{V} = \Z$ generated by the class of $V \cdot H$. From the exact sequence, $\Cl{X} = 0$ and since $S(V)$ is an integrally closed domain we see that $S(V)$ is a UFD by Prop. 6.2.  

\item Let $\stalk{X}{P}$ be the local ring at the cone point $P$ on $X$. Consider the natrual map $\Cl{X} \to \Cl{\Spec{\stalk{X}{P}}}$.
Let $R = S(V)$ then $X = \Spec{R}$ and $\stalk{X}{P} = R_\m$ where $\m = (x_0, \dots, x_n)$. Prime divisors $D$ of $\Spec{\stalk{X}{P}}$ correspond to height one primes $\q \subset R$ contained in $\m$ and thus the closure $\overline{D}$ in $X$ corresponds to $V(\q) \subset \Spec{R}$. Therefore $\Cl{X} \to \Cl{\Spec{\stalk{X}{P}}}$ is surjective. To establish that this map is injective we need to show for any divisor,
\[ D = \sum_{\height{\q} = 1} n_\q \cdot \q \]
such that,
\[ D_P = \sum_{\q \subset \m} n_\q \cdot \q = \div{(f)} \]
then we have $D = \div{(f')}$ for some $f, f' \in \Frac{R}^\times$. However, since the map $\Cl{V} \to \Cl{X}$ is a surjection, the class of $\q$ must be in the image of $\Cl{V} \to \Cl{X}$ and thus $\q \sim D$ where $D$ is a divisor pulled back from $V$ i.e. with each component passing though $\m$  (ASK IS $\q$ necessarily principal??)
\end{enumerate}

\subsubsection{6.4}

Let $k$ be a field with characteristic $\neq 2$ and $f \in k[x_1, \dots, x_n]$ a square-free nonconstant polynomial. Let $R = k[x_1, \dots, x_n]$ and $A = k[x_1, \dots, x_n, z]/(z^2 - f)$ which is a domain since $z^2 - f$ is irreducible when $f$ is squarefree and nonconstant with fraction field $L = K[z]/(z^2 - f)$ where $K = k(x_1, \dots, x_n)$. The extension $L / K$ is Galois with $\Gal{L/K} = \Z / 2 \Z$ acting via $\sigma : z \mapsto -z$. A generic element is $\alpha = g + h z \in L$ with $g,h \in K$ which has minimal polynomial,
\[ m_\alpha(x) = (x - \alpha)(x - \sigma(\alpha)) = (x - g - h z)(x - g + hz) = x^2 - 2 xg + (g^2 - h^2 f) \]
Suppose that $\alpha \in L$ is integral over $R$ then there exists a monic polynomial $p \in R[x] \subset L[x]$ such that $p(\alpha) = 0$ and therefore $m_\alpha \divides p$. Since $R$ is a UFD Gauss' lemma applies showing that every monic factor of $p$ must have coefficients in $R$. In particular $m_\alpha \in R[x]$ meaning that $g \in R$ and $h^2 f \in R$. However, since $f \in R$ is squarefree this implies that $h \in R$ as well since otherwise the square of its denominator would have to divide $f$ and thus $\alpha in A$. Conversely, if $\alpha \in A$ then $g,h \in R$ and thus $m_\alpha \in R[x]$ is monic which implies that $\alpha$ is integral. Therefore $A$ is the integral closure of $R$ in $L$ and thus $A$ is itself integrally closed in its field of fractions $L$.

\subsubsection{6.5}

Let $k$ be a field with characteristic $\neq 2$ and $X$ be the affine quadric hypersurface,
\[ X = \Spec{k[x_0, \dots, x_n]/(x_0^2 + x_1^2 + \cdots + x_r^2)} \]
\begin{enumerate}
\item Let $r \ge 2$. Then write $x_0^2 + x_1^2 + \cdots + x_r^2 = x_0^2 - f$ where $f = -(x_1^2 + \cdots + x_r^2) \in k[x_1, \dots, x_r]$. Since $r \ge 2$ the polynomial $f$ is squarefree (because each derivative $\partial_{x_i} f = - 2 x_i$ is coprime to $f$) and nonconstant so by the previous problem the ring,
\[ A = k[x_0, \dots, x_n]/(x_0^2 + x_1^2 + \cdots + x_r^2) \]
is an integrally closed domain and thus normal.

\item Let $x_0 = \tfrac{i}{2}(x_0' + x_1')$ and $x_1 = \tfrac{1}{2}(x_0' - x_1')$ then $x_0^2 = - \tfrac{1}{4}( x_0'^2 + 2 x_0' x_1' + x_1'^2 )$ and $x_1'^2 = \tfrac{1}{4} (x_0'^2 - 2 x_0' x_1' + x_1'^2)$ therefore,
\[ x_0^2 + x_1^2 + x_2^2 + \cdots + x_r^2 = - x_0' x_1' + x_2^2 + \cdots x_r^2 \]
so $X$ is cut out by the equation,
\[ x_0 x_1 = x_2^2 + \cdots + x_r^2 \]
In general, consider the closed subscheme $Z = V(x_1)$. This subscheme is not necessarily prime because $x_2^2 + \cdots + x_r^2$ is not irreducible unless $r \ge 4$ and thus $(x_1)$ is not prime because $x_0 x_1 = x_2^2 + \cdots + x_r^2$. Therefore, let $Y_i$ be the irreducible components of $Z$ given their reduced structure so the $Y_i$ are prime divisors. Then there is an exact sequence,
\begin{center}
\begin{tikzcd}
\Z Y_1 \oplus \cdots \oplus \Z Y_k \arrow[r] & \Cl{X} \arrow[r] & \Cl{X \setminus Z} \arrow[r] & 0
\end{tikzcd}
\end{center}
the first map sending $Y_i \mapsto [Y_i]$. However, $X \setminus Z = \Spec{A_{x_1}}$ and,
\begin{align*}
A_{x_1} & = k[x_0, x_1, x_1^{-1}, x_2 \dots, x_r]/(x_0 x_1 - (x_2^2 + \cdots + x_r^2)) 
\\
& = k[x_0, x_1, x_1^{-1}, x_2 \dots, x_r]/(x_0 - x_1^{-1} (x_2^2 + \cdots + x_r^2)) = k[x_1, x^{-1}, x_2, \dots, x_r] 
\end{align*}
is an open subspace of $\A^r$ and thus $A_{x_1}$ is a UFD and therefore we find that,
\[ \Cl{X \setminus Z} = \Cl{\Spec{A_{x_1}}} = 0 \]
Therefore $\Cl{X}$ is generated by the $Y_i$. It remains to compute the relations.
\bigskip\\
First let $r = 2$. Consider the divisor $Y \subset X$ cut out by $x_1 = x_2 = 0$ i.e. a ruling of the cone. Since $A/(x_1) = k[x_0, x_2]/(x_2^2)$ the divisor $Y$ is the unique component of $Z$ and $[Z] = 2 [Y]$ since $\div{x_1} = 2 Y$ because $x_1 = x_0^{-1} x_2^2$ and $x_2$ is a local parameter for $Y$. Now we need to show that $[Y]$ is not principal or equivalently the corresponding prime ideal $\p = (x_1, x_2)$ is not a principal ideal. Let $\m = (x_0, x_1, x_2)$ and notice that $\m / \m^2$ is a 3 dimensional $\kappa(\m)$-vectorspace generated by $\bar{x}_0, \bar{x}_1, \bar{x}_3$. Furthermore, the image of $\p \otimes \kappa(\m) \to \m \otimes \kappa(\m)$ contains $\bar{x}_1$ and $\bar{x}_2$ so $\p \otimes \kappa(\m)$ must have dimension at least $2$ showing that $\p$ cannot be principal because $\p \otimes \kappa(\m)$ is the number of generators of $\p_\m$. Therefore $[Y] \neq 0$ and $2 [Y] = 0$ showing that $\Cl{X} = \Z / 2 \Z$ generated by $[Y]$.
\bigskip\\
Let $r = 3$. Then $Z$ contains two prime divisors, $Y_1 = V(x_1, x_2 - i x_3)$ and $Y_2 = V(x_2, x_2 + i x_3)$ with local parameters $x_2 - i x_3$ and $x_2 + i x_3$. The only relation is given by $\div{x_1} = Y_1 + Y_2$ which follows from the fact that $x_1 = x_0^{-1} (x_2^3 + x_3^2) = x_0^{-1} (x_2 - i x_3)(x_2 + i x_3)$. Therefore, $\Cl{X} = (\Z Y_1 \oplus \Z Y_2)/ (Y_1 + Y_2) \cong \Z$. Alternatively, we see that $X$ is the affine cone over the projective quadric $Q = V(x_0 x_1 - x_2 x_3)$ which is isomorphic to $\P^1 \times \P^1$. By the previous problem there is a short exact sequence,
\begin{center}
\begin{tikzcd}
0 \arrow[r] & \Z \arrow[r] & \Cl{Q} \arrow[r] & \Cl{X} \arrow[r] & 0
\end{tikzcd}
\end{center}
where the map $\Z \to \Cl{Q}$ sends $1 \mapsto Q \cdot H$ the hyperplane class. However $\Cl{Q} = \Z \oplus \Z$ and $Q \cdot H = (1,1)$ so we see that $\Cl{X} = (\Z \oplus \Z)/(1,1) \Z \cong \Z$.
\bigskip\\
Finally, let $r \ge 4$. Then $Z$ is a prime divisor and $\div{x_1} = Z$ so it is trivial in the class group showing that $\Cl{X} = 0$.
\item Let $Q$ be the projective quadric hypersurface in $\P^r$ cut out by $x_0 x_1 = x_2^2 + \cdots + x_r^2$. We apply the exact sequence,
\begin{center}
\begin{tikzcd}
0 \arrow[r] & \Z \arrow[r] & \Cl{Q} \arrow[r] & \Cl{X} \arrow[r] & 0
\end{tikzcd}
\end{center}
where $X$ is the affine cone over $X$ considered previously. 
\bigskip\\
For $r = 2$ we know $\Cl{X} = \Z / 2 \Z$ and furthermore $Q \cong \P^1$ so $\Cl{Q} = \Z$ therefore the seqeunce shows that the hyperplane class $Q \cdot H$ is twice a generator.
\bigskip\\
For $r = 3$ we know $\Cl{X} = \Z$ and therefore the sequence splits giving $\Cl{Q} = \Z \oplus \Z$.
\bigskip\\
For $r \ge 4$ we know $\Cl{X} = 0$ and therefore the sequence gives $\Cl{Q} = \Z$ generated by the hyperplane class $Q \cdot H$.

\item Let $r \ge 4$ and $Y \subset Q$ be an irreducible subvariety of codimension one i.e. a prime divisor. The homogeneous coordinate ring,
\[ C(Q) = k[x_0, \dots, x_r]/(x_0 x_1 - (x_2^2 + \cdots x_r^2)) \]
is integrally closed by 2.6.4 and thus $Q$ is projectively normal. Furthermore, by the previous problem $\Cl{Q} = \Z$ generated by $Q \cdot H$ so by 2.6.3(c) we see that $S(Q)$ is a UFD. Therefore, the homogeneous prime $\p$ corresponding to $Y$ has height one and thus is principal $\p = (\bar{f})$ with a lift $f \in k[x_0, \dots, x_r]$. Therefore, let $V = V(f) \subset \P^n$ which is an irreducible hypersurface since $(\bar{f})$ is prime and thus $f$ is irreducible. Then we see that $V \cap Q = Y$ scheme theoretically showing that $Y$ is a complete intersection.
\bigskip\\
Alternatively, there is a cohomological argument. Because $Q$ is locally factorial ($Q$ is regular) prime divisors $Y$ correspond to nonzero sections $s \in H^0(X, \L)$ for a line bundle $\L \in \Pic{Q} = \Cl{Q}$. However, $\Cl{Q}$ is generated by the hyperplane class so $\L = \struct{Q}(d)$ where $\struct{Q}(d) = \iota^* \struct{\P^r}(d)$. We need to show that every section $s \in H^0(Q, \struct{Q}(d))$ is the pullback of a section $s' \in H^0(\P^r, \struct{\P^r}(d))$ since then $Y = V \cap Q$ where $V = V(s')$ is a hypersurface of degree $d$. There is an exact sequence,
\begin{center}
\begin{tikzcd}
0 \arrow[r] & \I(d) \arrow[r] & \struct{\P^r}(d) \arrow[r] & \struct{Q}(d) \arrow[r] & 0
\end{tikzcd}
\end{center}
where $\I = \struct{\P^r}(-2)$ is the ideal sheaf of $Q$. We get a long exact sequence,
\begin{center}
\begin{tikzcd}
0 \arrow[r] & H^0(\P^r, \struct{\P^r}(d-2)) \arrow[r] & H^0(\P^r, \struct{\P^r}(d)) \arrow[r, "\iota^*"] & H^0(Q, \struct{Q}(d)) \arrow[r] &  H^1(\P^r, \struct{\P^r}(d-2)) 
\end{tikzcd}
\end{center}
However, since $r > 1$ we know $H^1(\P^r, \struct{\P^r}(d-2)) = 0$ and thus,
\[ \iota^* : H^0(\P^r, \struct{\P^r}(d)) \to H^0(Q, \struct{Q}(d)) \]
is surjective. In fact, the kernel is $f \cdot H^0(\P^r, \struct{\P^r}(d-2))$ where $f = x_0 x_1 - (x_2^2 + \cdots + x_r^2)$ is the defining equation and thus irreducible hypersurfaces of degree $d$ besides $f$ pull back to nonzero sections.
\end{enumerate}

\subsubsection{6.6}

Let $X$ be the nonsingular plane cubic curve $y^2 z = x^3 - x z^2$. 
\begin{enumerate}
\item 
If $P, Q, R$ are colinear then let $\ell$ be the line in $\P^2$ passing through $P,Q,R$. Then $\ell \sim \ell_\infty$ the line at infinity and pulling back to $X$ we find $X \cap \ell \sim X \cap \ell_{\infty}$. Because $X$ is a cubic curve $\ell$ intersects $X$ at exactly $P,Q,R$ with multiplicity one and $\ell_{\infty}$ intersects $P_0$ with multiplicity three. Thus in $\Cl{X}$ we find $[P] + [Q] + [R] = 3 [P_0]$. Therefore,
\[ ([P] - [P_0]) + ([Q] - [P_0]) + ([R] - [R_0])  = 0 \]
Since the isomorphism $X \to \Cl{X}$ sends $P \mapsto [P] - [P_0]$ we see that $P + Q + R = 0$ in $X$. Conversely, suppose that $P + Q + R = 0$ then we know that $[P] + [Q] + [R] = 3 [P_0]$. The line passing through $P$ and $Q$ intersects $X$ at some additional point $R'$ (possibly equal to $P$ or $Q$) and from above we know that $[P] + [Q] + [R'] = 3 [P_0]$ and thus $[R] - [P_0] = [R'] - [P_0]$ so $R = R'$ by injectivity showing that $P,Q,R$ are colinear.

\item Let $P \in X$ be a point such that the tangent line $\ell$ passes through $P_0$. Then $\ell \cap X = 2 [P] + [P_0]$ (as divisors) so we see that $2 P + 0 = 0$ i.e. $2 P = 0$ so $P$ has order two. Let $P \in X$ have order two meaning that $2 P = 0$. Then if the tanget line $\ell$ intersects $X$ as $\ell \cap X = 2 [P] + [Q]$ (as divisors) for $[Q] \neq [P_0]$ we would have $2 P + Q = 0$ and thus $Q = 0$ implying that $Q = P_0$ as points so the tangent line passes through $P_0$. 

\item If $P$ is an inflection point then the tangent line $\ell$ intersects $X$ as $\ell \cap X = 3 [P]$ so we have $3 [P] = 3 [P_0]$ and thus $3 P = 0$ in the group law. Conversely if $3 P = 0$ then in $\Cl{X}$ we have $3 [P] = 3 [P_0]$. If $P$ is not an inflection point then its tangent line $\ell$ intersects $X$ with $\ell \cap X = 2 [P] + [Q]$ (as divisors). Then we find $2 P + Q = 0$ in the group law so $P = Q$ meaning that $\ell \cap X = 3 P$ so $P$ is an infection point.

\item Let $k = \C$ then $X(\Q)$ is a subgroup because if $P$ and $Q$ have rational coefficients then the line between them is defined over $\Q$ and thus the third intersection point $R$ with $X$ has rational coefficients. Furthermore, $P + Q + R = 0$ so $-R = P + Q$ also has rational coefficents because $-R$ is simply $R$ reflected along the $y$-axis (because $R + (-R) + P_0 = 0$ so $R$ and $-R$ must define a line passing through $P_0$ i.e. vertical in the $(x,y)$ patch). 
\bigskip\\
For $z = 0$ there is a unique point $P_0 = [0:1:0]$. For $z \neq 0$ we consider the affine patch cut out by $y^2 = x(x^2 - 1)$. If $y = 0$ then $x(x - 1)(x + 1) = 0$ so $x = -1, 0, 1$ giving three rational points $P = [-1 : 0 : 1]$ and $Q = [0 : 0 : 1]$ and $R = [1 : 0 : 1]$. If $x = 0$ then $y^2 = 0$ so $y = 0$ giving the point $[0 : 0 : 1]$. Fermat descent shows there are no others. By colinearity, $P + Q + R = 0$ and since the tangent lines pass through $P_0$ we also have $2 P = 2 Q = 2 R = 0$. Therefore, we have found the group is,
\[ X(\Q) = \Z / 2 \Z \oplus \Z / 2 \Z \]
generated by $P$ and $Q$ with $R = P + Q$.
\end{enumerate}

\subsubsection{6.7}

\newcommand{\CaClcirc}[1]{\mathbb{CaCl}^\circ\left( #1 \right)}

Let $X$ be the nodal cubic $y^2 z = x^3 + x^2 z$ in $\P^2$. Let $Z \in X$ denote the singular point and $U = X \setminus Z$. We want to compute $\CaCl{X}$. Notice that by multiplying by a suitable global function, any Cartier divisor is linearly equivalent to one which is invertible in a neighborhood of $Z$. Furthermore, there is an isomorphism $\CaCl{U} \iso \Cl{U}$ giving a degree map $\CaCl{X} \to \Z$ because I claim that any $f \in K(X)^\times$ which is invertible in a neighborhood of $Z$ has $\deg{\div{(f)}} = 0$. Indeed let $\tilde{X} \to X$ be the normalization then the rational function $f : X \rat \P^1$ extends uniquely to $\tilde{f} : \tilde{X} \to \P^1$ and $\div{(\tilde{f})} = [\tilde{f}^{-1}(0)] - [\tilde{f}^{-1}(\infty)]$ has degree zero since $[0] - [\infty]$ has degree zero. However, by assumption $Z$ is not in either fiber and $\tilde{X} \to X$ is an isomorphism away from $Z$ meaning that $\div{(f)} = [f^{-1}(0)] - [f^{-1}(\infty)]$ also has degree zero. 
\bigskip\\
Next, for each $P \in U$ there is an associated divisor $D_P$ which is $1$ on a neighborhood of $Z$ and equal to the Weil divisor $[P] - [P_0]$ on $U$ where $P_0 = [0:1:0]$. I claim that $U(k) \to \CaClcirc{X}$ is an isomorphism giving $U$ the structure of the group scheme $\Gm$. First suppose that $D_P \sim D_Q$ then there exists $f \in K(X)^\times$ such that $f$ is invertible in a neighborhood of $Z$ and $\div{(f)}|_U = [P] - [Q]$ therefore defines a birational morphism $f : X \to \P^1$. Note: $f$ is automatically defined on $U$ but since $f$ is already an element of $\struct{V}(V)^\times$ for some affine open $Z \in V \subset X$ we can extend $f : X \to \P^1$ via $k[t] \to \struct{V}(V)$ by sending $t \mapsto f$. Such a map would be an isomorphism because rational maps $\P^1 \rat X$ extend. This is impossible because $X$ is singular so we must have $P = Q$.
\bigskip\\
Fix points $P,Q \in U$. The line $\ell$ passing through $P$ and $Q$ intersects $X \subset \P^2$ at exactly one other point $R$. Notice that $R \in U$ because if $R = Z$ then $\ell$ intersects $R$ at multiplicity two since $Z \in X$ is a double point which is impossible because $\ell \cdot X$ has degree three. Furthermore, the line at infinity $\ell_\infty = [s : t : 0]$ intersects $X$ when $s^3 = 0$ so $\ell_\infty \cdot X = 3 \cdot P_0$. Therefore, $P + Q + R \sim 3 \cdot P_0$ in $\Cl{U}$ by a linear relation invertible at $Z$ since both lines intersect $X$ inside $U$ and thus,
\[ D_P + D_Q + D_R = 0 \]
in $\CaCl{X}$. Thus, taking the line through $P$ and $P_0$ gives a point $-P$ such that,
\[ D_P + D_{-P} = 0 \]
Therefore, the map $U(k) \to \CaClcirc{X}$ induces a group structure on $U$ via $P_0 = 0$ and $P + Q = -R$ and $P \mapsto -P$ satisfies $P + -P = 0$. Furthermore, the $D_P$ generate $\CaClcirc{X}$ because and any Cartier divisor is linearly equivalent to a divisor invertible near $Z$ and furthermore over $U$ Cartier divisors correspond to Weil divisors,
\[ D|_U = \sum_{P \in U} n_P \cdot [P] \]
but since $\deg{D|_U} = 0$ we have,
\[ D|_U = \sum_{P \in U} n_P \cdot ([P] - [P_0]) \]
and thus,
\[ D = \sum_{P \in U} n_P D_P \]
however by the sum and inversion relations we can reduce to $D \sim D_{P'}$ where,
\[ P' = \sum_{P \in U} n_P \cdot P \]
in the group law on $U$. Therefore $U(k) \to \CaClcirc{X}$ is an isomorphism of groups. 
\bigskip\\
Finally, I claim that $U \cong \Gm$ as a group scheme. Consider the normalization $\P^1 \to X$ given by $x = t^2  + 1$ and $y = t(t^2 + 1)$ (FINISH THIS!!)

$t^2 = (x - 1)$ 
\subsubsection{6.8}

Let $f : X \to Y$ be a morphism of schemes.

\begin{enumerate}
\item The map $f^* : \Pic{Y} \to \Pic{X}$ sending $\L \mapsto f^* \L$ is a group homomorphism because $f^* (\L_1 \otimes \L_2) = f^* \L_1 \otimes f^* \L_2$. 

\item If $f$ is a finite morphism of nonsingular curves then consider the square,
\begin{center}
\begin{tikzcd}
\Pic{Y} \arrow[d, "f^*"] \arrow[r, "c_1"] & \Cl{Y} \arrow[d, "f^*"] 
\\
\Pic{X} \arrow[r, "c_1"] & \Cl{X}
\end{tikzcd}
\end{center}
On prime divisors $Q \in Y$ pullback acts via,
\[ f^* [Q] = \sum_{P \in f^{-1}(Q)} \nu_P(t_Q) \cdot [P] \]
where $t_Q \in K(Y) \subset K(X)$ is a local parameter of $\stalk{Y}{Q}$. For $\L \in \Pic{Y}$, under $c_1$ we choose a nonzero meromorphic section $s \in \L_\eta$ then $c_1(\L) = \div{(s)}$. Furthermore, $f^* s$ is a nonvanishing meromorphic section of $f^* \L$. Thus $c_1(f^* \L) = \div{(f^* s)} = f^* \div{(s)}$. To see this, locally we choose a trivializations $s_{U} : \struct{U} \to \L|_U$ then,
\[ \div{(s)} = \sum_{Q \in Y} \nu_Q(s/s_U) \cdot [Q] \] 
Therefore,
\[ f^* \div{(s)} = \sum_{Q \in Y} \sum_{P \in f^{-1}(Q)} \nu_Q(s/s_U) \nu_P(t_Q) \cdot [P] \]
Furthermore, 
\[ \div{(f^* s)} = \sum_{P \in X} \nu_P(f^\#(s/s_U)) \cdot [P] \]
Since $f^\# : \stalk{Y}{Q} \to \stalk{X}{P}$ is a map of DVRs sending local parameters $t_Q \mapsto u \, t_P^{\nu_P(t_Q)}$ and thus $\nu_P(f^\#(s/s_U)) = \nu_Q(s/s_U) \nu_P(t_Q)$ showing that,
\[ \div{(f^* s)} = f^* \div{(s)} \]
and therefore $c_1(f^* \L) = f^* c_1(\L)$. Alternatively, the morphism $\CaCl{X} \to \Pic{X}$ is clearly natural so we can consider the diagram,
\begin{center}
\begin{tikzcd}
\CaCl{Y} \arrow[d, "f^*"] \arrow[r] & \Cl{Y} \arrow[d, "f^*"] 
\\
\CaCl{X} \arrow[r] & \Cl{X}
\end{tikzcd}
\end{center}
Let $D = \{(U_i, f_i)\}$ be a Cartier divisor on $Y$ then $f^*D = \{(f^{-1}(U_i), f^\#(f_i))\}$. Then the associated Weil divisor of $D$ is,
\[ \wt{D} = \sum_{\substack{Q \in Y \\ Q \in U_i}} \nu_Q(f_i) \cdot [Q] \]
Then we see,
\[ \wt{f^* D} = \sum_{\substack{P \in X \\ P \in f^{-1}(U_i)}} \nu_P(f^\#(f_i)) \cdot [P] \]
However, $\nu_P(f^\#(f_i)) = \nu_Q(f_i) \nu_P(t_Q)$ and therefore,
\[ f^* \wt{D} = \sum_{\substack{Q \in Y \\ Q \in U_i}} \sum_{P \in f^{-1}(Q)} \nu_Q(f_i) \nu_P(t_Q) \cdot [P] = \sum_{\substack{P \in X \\ P \in f^{-1}(U_i)}} \nu_Q(f_i) \nu_P(t_Q) \cdot [P] = \wt{f^* D} \]

\item Let $X \subset \P^n$ be a locally factorial closed subscheme and $\iota : X \embed \P^n$ the inclusion map. 
\end{enumerate}

\subsubsection{6.9}

Let $X$ be a (integral) projective curve over $k$ and $\pi : \tilde{X} \to X$ the normalization. For each point $p \in X$ let $\wt{\mathcal{O}}_{p}$ be the integral closure of $\stalk{X}{x}$ in $k(X)$. Consider the exact sequence of sheaves on $X$,
\begin{center}
\begin{tikzcd}
0 \arrow[r] & \pi_* \struct{\tilde{X}}^\times / \struct{X}^\times \arrow[r] & \K_X^\times / \struct{X}^\times \arrow[r] & \K_X^\times / \pi_* \struct{\tilde{X}}^\times \arrow[r] & 0
\end{tikzcd}
\end{center}
The sheaf $\pi_* \struct{\tilde{X}}^\times / \struct{X}^\times$ is supported on the singular points which is a finite set. Taking the cohomology of this exact sequence,
\begin{center}
\begin{tikzcd}
0 \arrow[r] & H^0(\pi_* \struct{\tilde{X}}^\times / \struct{X}^\times) \arrow[r] & H^0(\K_X^\times / \struct{X}^\times) \arrow[r] & H^0(\K_X^\times / \pi_* \struct{\tilde{X}}^\times) \arrow[r] & H^1(\pi_* \struct{\tilde{X}}^\times / \struct{X}^\times)
\end{tikzcd}
\end{center}
However, if $\F$ is a sheaf supported on a finite set of points $S$ then,
\[ H^q(X, \F) = 
\begin{cases}
\bigoplus\limits_{p \in S} \F_p & q = 0
\\
0 & q > 0
\end{cases} \]
Therefore, we get an exact sequence,
\begin{center}
\begin{tikzcd}
0 \arrow[r] & \bigoplus\limits_{p \in X} (\pi_* \struct{\tilde{X}}^\times / \struct{X}^\times)_p \arrow[r] & H^0(\K_X^\times / \struct{X}^\times) \arrow[r] & H^0(\K_X^\times / \pi_* \struct{\tilde{X}}^\times) \arrow[r] & 0
\end{tikzcd}
\end{center}
We need to compute $(\pi_* \struct{\tilde{X}}^\times)_p$ Locally $\pi : \tilde{X} \to X$ over $U = \Spec{A}$ is given by $A \to \wt{A}$ where $\wt{A}$ is the integral closure of $A$ in $K$ and thus $(\pi_* \struct{\tilde{X}}^\times)_p = \wt{A}_\p^\times = \wt{\mathcal{O}}^\times_{p}$ because localization and normalization commute. Furthermore, since $\pi : \tilde{X} \to X$ is birational $\pi_* \K_{\tilde{X}} = \K_X$ and since $\pi : \tilde{X} \to X$ is affine $\pi_*$ is exact so,
\[ \pi_*(\K_{\tilde{X}}^\times / \struct{\tilde{X}})^\times = \pi_* \K_{\tilde{X}}^\times / \pi_* \struct{\tilde{X}}^\times = \K_X^\times / \pi_* \struct{\tilde{X}}^\times \]
Furthermore, recall that the Cartier divisors are $\CaDiv{X} = H^0(\K_X^\times / \struct{X}^\times)$. Thus the exact sequence becomes,
\begin{center}
\begin{tikzcd}
0 \arrow[r] & \bigoplus\limits_{p \in X} \wt{\mathcal{O}}_p^\times / \stalk{X}{p}^\times \arrow[r] & \CaDiv{X} \arrow[r, "\pi^*"] & \CaDiv{\tilde{X}} \arrow[r] & 0
\end{tikzcd}
\end{center}
Finally, because on function fields, $\pi^* : K(X) \to K(\tilde{X})$ is an isomorphism ($\pi$ is birational), we can quotient by $K^\times$ and use the isomorphism $\CaCl{X} = \CaDiv{X} / K^\times \iso \Pic{X}$ (using that $X$ is integral) to get an exact sequence,
\begin{center}
\begin{tikzcd}
0 \arrow[r] & \bigoplus\limits_{p \in X} \wt{\mathcal{O}}_p^\times / \stalk{X}{p}^\times \arrow[r] & \Pic{X} \arrow[r, "\pi^*"] & \Pic{\tilde{X}} \arrow[r] & 0
\end{tikzcd}
\end{center}
Now, let $X$ be a plane cuspidal cubic curve. Then $\tilde{X} = \P^1_k$. There is one singularity at $p$ where locally $A = k[x, y]/(x^3 - y^2)$ with normalization $A \to \wt{A}$ equal to $k[x,y]/(x^3 - y^2) \to k[t]$ via $x \mapsto t^2$ and $y \mapsto t^3$. Then $\wt{\mathcal{O}}_p^\times / \stalk{X}{p}^\times = 1 + k \cdot t$ so there is an exact sequence,
\begin{center}
\begin{tikzcd}
0 \arrow[r] & \Ga \arrow[r] & \Pic{X} \arrow[r] & \Z \arrow[r] & 0
\end{tikzcd}
\end{center}
Likewise, let $X$ be a plane nodal cubic curve. Then $\tilde{X} = \P^1_k$. There is again one singularity at $p$ where locally $A = k[x, y]/(y^2 - x^2(x+1))$ with normalization $A \to \wt{A}$ equal to $k[x,y]/(y^2 - x^2(x+1)) \to k[t]$ via $x \mapsto t^2 - 1$ and $y \mapsto t(t^2 - 1)$. Therefore $t^{n+1} - t^n$ is in the image for each $n$. Then $\wt{\mathcal{O}}_p^\times / \stalk{X}{p}^\times = k^\times$ so there is an exact sequence,
\begin{center}
\begin{tikzcd}
0 \arrow[r] & \Gm \arrow[r] & \Pic{X} \arrow[r] & \Z \arrow[r] & 0
\end{tikzcd}
\end{center}

\subsubsection{6.10}

Let $X$ be a Noetherian scheme and $K(X)$ the Grothendieck group of coherent sheaves.

\begin{enumerate}
\item Let $X = \A^1_k$. Then $K(X)$ is the Grothendieck group for the category of finite $k[x]$-modules. Since $A = k[x]$ is a PID every finite $A$-module has a length two resolution by finite free $A$-modules because the kernel of $A^n \onto M$ is a submodule of a free module and thus free. Thus there is an exact sequence,
\begin{center}
\begin{tikzcd}
0 \arrow[r] & A^m \arrow[r] & A^n \arrow[r] & M \arrow[r] & 0
\end{tikzcd}
\end{center}
Therefore, $[M] = [A^n] - [A^m]$. Furthermore, it is clear that $[A^{n+m}] = [A^n] + [A^m]$ and thus $[M] = [A^{n-m}]$ so we see that $K(X) \cong \Z$ via rank. 
\item Let $X$ be an integral scheme with generic point $\xi \in X$ with function field $K = \stalk{X}{\xi}$. If $\F$ is a coherent sheaf we define the rank $\rank(\F) = \rank_\xi(\F) = \dim_K \F \otimes_{\stalk{X}{\xi}} K = \dim_K \F_\xi$. Given an exact sequence of coherent sheaves,
\begin{center}
\begin{tikzcd}
0 \arrow[r] & \F \arrow[r] & \G \arrow[r] & \H \arrow[r] & 0
\end{tikzcd}
\end{center}
then there is an exact sequence at the generic point,
\begin{center}
\begin{tikzcd}
0 \arrow[r] & \F_\xi \arrow[r] & \G_\xi \arrow[r] & \H_\xi \arrow[r] & 0
\end{tikzcd}
\end{center}
meaning that $\dim_K \G_\xi = \dim_K \F_\xi + \dim_K \H_\xi$. Therefore, $\rank(\G) - \rank(\F) - \rank(\H) = 0$ so rank is compatible with the relation $[\G] - [\F] - [\H] = 0$ in $K(X)$. Therefore, rank factors through the quotient to give a well-defined map $\rank : K(X) \to \Z$.
\bigskip\\
Taking $\F = \struct{X}^{\oplus n}$ we see $\rank(\F) = n$ and thus $[\struct{X}^{\oplus n}] \mapsto n$ so we see that $K(X) \to \Z$ is surjective.

\item Let $\iota : Z \embed X$ be the inclusion of a closed subscheme and $j : U \to X$ inclusion of the complement. Consider the maps $K(Z) \to K(X) \to K(U)$ defined via $[\F] \mapsto [\iota_* \F]$ and $[\F] \mapsto [\F|_U]$. Since $X$ is Noetherian, Exercise 5.15 the map $K(X) \to K(U)$ is surjective. Clearly $(\iota_* \F)|_U = 0$ so to show that the sequence,
\begin{center}
\begin{tikzcd}
K(Z) \arrow[r] & K(X) \arrow[r] & K(U) \arrow[r] & 0
\end{tikzcd}
\end{center}
is exact we simply need to check that if $\F$ is a coherent sheaf on $X$ supported on $Z$ then $[\F]$ is in the image of $K(Z) \to K(X)$. Let $\I = \ker{(\struct{X} \to \struct{Z})}$ and define $\F_k = \I^k \cdot \F$ which is a filtration,
\[ \F = \F_0 \supset \F_1 \supset \cdots \supset \F_n \supset \F_{n+1} = 0 \]
of coherent subsheaves. Note that $\F_k = \I^k \cdot \F = 0$ for sufficently large $k$ because $\F$ is supported on $Z$ meaning that $\I \subset \sqrt{\shAnn{\struct{X}}{\F}}$ and $\I$ is coherent so $\I^k \subset \shAnn{\struct{X}}{\F}$ for sufficiently large $k$.
\bigskip\\
Because $\iota : Z \embed X$ is a closed immersion, $\iota_*$ induces an equivalence of categories from coherent $\struct{Z}$-modules to coherent $\struct{X}$-modules killed by $\I$. Furthermore, $\I \cdot (\F_i/\F_{i+1}) = 0$ so $\F_i / \F_{i + 1} = \iota_* \G_i$ for a coherent $\struct{Z}$-module $\G_i$. From the exact sequence,
\begin{center}
\begin{tikzcd}
0 \arrow[r] & \F_{i+1} \arrow[r] & \F_i \arrow[r] & \iota_* \G_i \arrow[r] & 0
\end{tikzcd}
\end{center}
we see that $[\F_i] = [\F_{i+1}] + [\iota_* \G_i]$ and thus,
\[ [\F] = [\iota_* \G_0] + [\iota_* \G_1] + \cdots + [\iota_* \G_n] = \iota_* ([\G_0] + [\G_1] + \cdots + [\G_n]) \in \Im{\iota_* : K(Y) \to K(X)} \]
\end{enumerate}

\subsubsection{6.11}

Let $X$ be a nonsingular curve over an algebraically closed field $k$. 
\begin{enumerate}
\item We define a map $\psi : \Cl{X} \to K(X)$. Let $D = n_1 [P_1] + \cdots + n_k [P_k]$ be a divisor. Then we define,
\[ \phi(D) = \sum_{i = 1}^n n_i [(\iota_{P_i})_* \kappa(P_i)] \]
where $[\F]$ denotes the class of $\F$ in the Grothendieck group. Suppose that $D$ is an effective divisor. Then there is an exact sequence,
\begin{center}
\begin{tikzcd}
0 \arrow[r] & \struct{X}(-D) \arrow[r] & \struct{X} \arrow[r] & \struct{D} \arrow[r] & 0
\end{tikzcd}
\end{center}
showing that, explicitly,
\[ \struct{D} = \prod_{i = 1}^n (\iota_{P_i})_* \kappa(P_i)^{n_i} \]
and therefore $\psi(D) = [\struct{D}] = [\struct{X}] - [\struct{X}(-D)]$. For any divisor $D$ write $D = D_+ - D_-$ then,
\[ \psi(D) = \psi(D_+) - \psi(D_-) = [\struct{X}] - [\struct{X}(-D_+)] - [\struct{X}] + [\struct{X}(-D_-)] = [\struct{X}] - [\struct{X}(D_- - D_+)] \]
Therefore, since $\struct{X}(-D)$ depends only on the linear equivalence class of $D$ we see that $\psi(D)$ also descends to the quotient as  map $\psi : \Cl{X} \to K(X)$.

\item Let $\L$ be an ample line bundle on $X$ say given by an embedding into projective space. For any coherent sheaf $\F$ on $X$, we know $\F \otimes \L^{\otimes n}$ is generated by global sections so we get a surjection $\L^{\otimes -n} \onto \F$ and thus take $\E_0 = \L^{\otimes -n}$ which is locally free. Then let $\E_1 = \ker{(\E_0 \onto \F)}$. For each $x \in X$ the stalk $\stalk{X}{x}$ is either a field or a DVR and thus because $(\E_0)_x$ is free the submodule $(\E_1)_x \subset (\E_0)_x$ is also free. Since $\E_1$ is coherent we find that $\E_1$ is also locally free giving a resolution,
\begin{center}
\begin{tikzcd}
0 \arrow[r] & \E_1 \arrow[r] & \E_0 \arrow[r] & \F \arrow[r] & 0
\end{tikzcd}
\end{center}
Let $r_i = \rank{\E_i}$ and define,
\[ \det{\F} = \left( \bigwedge^{r_0} \E_0 \right) \otimes \left( \bigwedge^{r_1} \E_1 \right)^{-1} \in \Pic{X} \]
Now suppose we have a second locally free resolution,
\begin{center}
\begin{tikzcd}
0 \arrow[r] & \E_1' \arrow[r] & \E_0' \arrow[r] & \F \arrow[r] & 0
\end{tikzcd}
\end{center}
Consider the diagram,
\begin{center}
\begin{tikzcd}
& 0  \arrow[d] & 0  \arrow[d] & 0  \arrow[d]
\\
0 \arrow[r] & \E_1 \arrow[d] \arrow[r] & \E_0 \arrow[d] \arrow[r] & \F \arrow[d] \arrow[r] & 0
\\
0 \arrow[r] & \G \arrow[d] \arrow[r] & \E_0 \oplus \E_0' \arrow[d] \arrow[r] & \F \arrow[d] \arrow[r] & 0
\\
0 \arrow[r] & \E_0' \arrow[d] \arrow[r] & \E_0'  \arrow[d] \arrow[r] & 0  \arrow[d] \arrow[r] & 0
\\
& 0 & 0 & 0
\end{tikzcd}
\end{center}
where $\G = \ker{(\E_0 \oplus \E_0' \to \F \oplus \F)}$ is locally free since it is a submodule of a locally free module. Since the rows are exact and the second and third column are exact we get an exact sequence,
\begin{center}
\begin{tikzcd}
0 \arrow[r] & \E_1 \arrow[r] & \G \arrow[r] & \E_0' \arrow[r] & 0
\end{tikzcd}
\end{center} 
and swaping the resolutions gives an exact sequence,
\begin{center}
\begin{tikzcd}
0 \arrow[r] & \E_1' \arrow[r] & \G \arrow[r] & \E_0 \arrow[r] & 0
\end{tikzcd}
\end{center} 
Therefore, $r_1 + r_0' = r_1' + r_0$ and there are isomorphisms,
\[ \bigwedge^{\mathrm{top}} \G \cong \left( \bigwedge^{r_1} \E_1 \right) \otimes \left( \bigwedge^{r_0'} \E_0' \right) \cong \left( \bigwedge^{r_1'} \E_1' \right) \otimes \left( \bigwedge^{r_0} \E_0 \right)  \]
Rearanging gives,
\[ \left( \bigwedge^{r_0} \E_0 \right) \otimes \left( \bigwedge^{r_1} \E_1 \right)^{-1} \cong \left( \bigwedge^{r_0'} \E_0' \right) \otimes \left( \bigwedge^{r_1'} \E_1' \right)^{-1} \]
and therefore $\det{\F}$ is well-defined in $\Pic{X}$. Furthermore, given an exact sequence,
\begin{center}
\begin{tikzcd}
0 \arrow[r] & \F' \arrow[r] & \F \arrow[r] & \F'' \arrow[r] & 0
\end{tikzcd}
\end{center}
we can choose a resolution,
\begin{center}
\begin{tikzcd}
& 0 \arrow[d] & 0 \arrow[d] & 0 \arrow[d]
\\
0 \arrow[r] & \E_1' \arrow[d] \arrow[r] & \E_1 \arrow[d] \arrow[r] & \E_1'' \arrow[d] \arrow[r] & 0
\\
0 \arrow[r] & \E_0' \arrow[d] \arrow[r] & \E_0 \arrow[d] \arrow[r] & \E_0'' \arrow[d] \arrow[r] & 0
\\
0 \arrow[r] & \F' \arrow[d] \arrow[r] & \F \arrow[d] \arrow[r] & \F'' \arrow[d] \arrow[r] & 0
\\
& 0 & 0 & 0
\end{tikzcd}
\end{center}
with exact rows and columns. To see this is possible, first resolve $\F''$ then let $\E_0$ be a locally free surjecting onto $\F \oplus \E_0''$ and $\E_1 = \ker{(\E_0 \to \F)}$ which is locally free as well and there automatically exists a surjection $\E_1 \onto \E_1''$ because the columns are exact and the horizontal maps $\E_0 \to \E_0''$ and $\F \to \F''$ are surjective. Finally, let $\E_i' = \ker{(\E_i \to \E_i'')}$ which are locally free and by the nine lemma this final column is exact. Therefore, we see,
\begin{align*}
\det{\F} & = \left( \bigwedge^{\mathrm{top}} \E_0 \right) \otimes \left( \bigwedge^{\mathrm{top}} \E_1 \right)^{-1} 
\\
& = \left( \bigwedge^{\mathrm{top}} \E_0' \right) \otimes \left( \bigwedge^{\mathrm{top}} \E_0'' \right) \otimes \left( \bigwedge^{\mathrm{top}} \E_1' \right)^{-1} \otimes \left( \bigwedge^{\mathrm{top}} \E_1'' \right)^{-1} = (\det{\F'}) \otimes (\det{\F''}) 
\end{align*}
In particular $\det{(\F \oplus \F')} = (\det{\F}) \otimes (\det{\F'})$ so $\det$ is a group map and preserves the relation $[\F] = [\F'] + [\F'']$ thus factoring through the Grothendieck group to give a group homomorphism $\det : K(X) \to \Pic{X}$. Finally, if $D$ is a divisor, then there is an exact sequence,
\begin{center}
\begin{tikzcd}
0 \arrow[r] & \struct{X}(-D) \arrow[r] & \struct{X} \arrow[r] & \struct{D} \arrow[r] & 0
\end{tikzcd}
\end{center}
which is a locally free resolution of $\struct{D}$. Therefore,
\[ \det{\struct{D}} = \struct{X} \otimes \struct{X}(-d)^{\otimes - 1} = \struct{X}(D) \]
and thus since $\psi(D) = [\struct{D}]$ we find that $\det{\psi(D)} = \struct{X}(D)$.
\item Let $\F$ be a coherent sheaf of rank $r$. After twisting by $\L$ we know $\F \otimes \L^{\otimes n}$ is generated by global sections so we can choose global sections $s_1, \dots, s_r \in H^0(X, \F \otimes \L^{\otimes n})$ which give a basis of $\F_{\xi}$ over the function field $k(X) = \stalk{X}{\xi}$ where $\xi \in X$ is the generic point. Then, the map $\L^{\oplus -r} \to \F$ is an isomorphism at the generic point. Since $\L^{\oplus -r}$ is locally free and $X$ is integral, this implies that $\L^{\oplus -r} \to \F$ is injective. Let $\sT = \coker{(\L^{\oplus -r} \to \F)}$ giving an exact sequence,
\begin{center}
\begin{tikzcd}
0 \arrow[r] & \L^{\oplus -r} \arrow[r] & \F \arrow[r] & \sT \arrow[r] & 0
\end{tikzcd}
\end{center}
However, we can choose $\L = \struct{X}(-D)$ for an antiample divisor $D$ and furthermore, because the first map is an isomorphism at the generic point, $\sT_\xi = 0$ and $\sT$ is coherent so it is a torsion sheaf. Therefore, in $K(X)$ we see $[\F] = r \cdot [\struct{X}(D)] + [\sT]$. The support of $\sT$ is a closed and thus finite set $S \subset X$ and elsewhere $\sT_x = 0$ so,
\[ \sT = \bigoplus_{P \in S} \sT_P \]
Let $M = \sT_P$ and $A = \stalk{X}{P}$.
Then $M$ is a finite length $A$-module because $M \otimes_A \Frac{A} = 0$ meaning that $\m$ is a minimal prime of the support ($\dim{A} = 1$) so $\dim{A / \Ann{A}{M}} = 0$ and thus $A / \Ann{A}{M}$ is Artinian but $M$ is a finitely generated $A / \Ann{A}{M}$-module so $\length{A}{M}$ is finite. Since $\m$ is nilpotnent on $A/\Ann{A}{M}$ we see that $\m^n M = 0$ so consider,
\begin{center}
\begin{tikzcd}
0 \arrow[r] & \m^{i+1} M \arrow[r] & \m^i M \arrow[r] & \m^i M / \m^{i+1} M \arrow[r] & 0
\end{tikzcd}
\end{center}
and $\m^i M / \m^{i+1} M = \m^i M \otimes_A A / \m \cong (A / \m)^{\oplus n}$. Thus in $K(A)$ we see $[M] = \length{A}{M} \cdot [A/\m]$ and therefore,
\[ [\sT] = \sum_{P \in S} \length{\stalk{X}{P}}{\sT_P} \cdot [(\iota_P)_* \kappa(P)] = \psi(D') \quad \text{ where } \quad D' = \sum_{P \in S} \length{\stalk{X}{P}}{\sT_P} \cdot [P] \]
Then, because $[\struct{X}(D)] = [\struct{X}] + \psi(D)$ we have,
\[ [\F] = r [\struct{X}(D)] + \psi(D') = r [\struct{X}] + r \psi(D) + \psi(D') = r [\struct{X}] + \psi(r D + D') \]
Therefore $[\F] - r [\struct{X}] = \psi(rD + D')$ and furthermore, 
\[ \det{\F} = \struct{X}(D)^{\otimes r} \otimes \det{\sT} = \struct{X}(rD) \otimes \struct{X}(D) = \struct{X}(rD + D') \]
so we find that $[\F] = r [\struct{X}] + \psi(\det{\F})$.

\item We have maps $\psi : \Pic{X} \to K(X)$ and $[\struct{X}] : \Z \to K(X)$ and $\det : K(X) \to \Pic{X}$ and $\rank : K(X) \to \Z$. We have shown that given any $[\F] \in K(X)$ we have,
\[ [\F] = \rank{(\F)} \cdot [\struct{X}] + \psi(\det{\F}) \]
and furthermore $\det{\psi(D)} = \struct{D}$ and $\rank{(r [\struct{X}])} = r$. Therefore, these maps give the structure maps of a biproduct so $K(X) \cong \Pic{X} \oplus \Z$. 
\end{enumerate}

\subsubsection{6.12}

Let $X$ be a nonsingular curve. Then there is a degree function $\deg : K(X) \to \Z$ i.e. an assignment $\F \mapsto \deg{\F} \in \Z$ for any coherent $\struct{X}$-module satisfying,
\begin{enumerate}
\item $\deg{\struct{X}(D)} = \deg{D}$ for any divisor $D$
\item if $\F$ is a torsion sheaf ($\F_\xi = 0$ where $\xi \in X$ is the generic point) then,
\[ \deg{\F} = \sum_{x \in X} \length{\stalk{X}{x}}{\F_x} \]
\item for any exact sequence of coherent $\struct{X}$-modules,
\begin{center}
\begin{tikzcd}
0 \arrow[r] & \F' \arrow[r] & \F \arrow[r] & \F'' \arrow[r] & 0
\end{tikzcd}
\end{center}
then $\deg{\F} = \deg{\F'} + \deg{\F''}$.
\end{enumerate}
I first claim there is a unique such function. Any coherent sheaf $\F$ fits into an exact sequence,
\begin{center}
\begin{tikzcd}
0 \arrow[r] & \struct{X}(D)^{\oplus r} \arrow[r] & \F \arrow[r] & \sT \arrow[r] & 0
\end{tikzcd}
\end{center}
where $\sT$ is a torsion sheaf. Therefore, for any degree function
\[ \deg{\F} = \deg{\struct{X}(D)^{\oplus r}} + \deg{\sT} = r \deg{D} + \sum_{x \in X} \length{\stalk{X}{x}}{\sT_x} \]
where $\deg{\F^{\oplus n}} = n \deg{\F}$ by the third property. Therefore, there is a unique such degree function because the above number is fixed. 
\bigskip\\
To show such a degree function exists it suffices to check that $\deg{\F} := \deg{\det{\F}}$ satisfies the given properties. First, note that $\det{\struct{X}(D)} = \struct{X}(D)$ because it is locally free so $\deg{\struct{X}(D)} = \deg{D}$. Next, if $\F$ is a torsion sheaf then we have shown that $\F = \psi(D)$ and $\det{\psi(D)} = \struct{X}(D)$ and $\deg{\F} = \det{D}$ satisfies the formula. Finally, given an exact sequence,
\begin{center}
\begin{tikzcd}
0 \arrow[r] & \F' \arrow[r] & \F \arrow[r] & \F'' \arrow[r] & 0
\end{tikzcd}
\end{center}
we know $\det{\F} = (\det{\F'} \otimes \det{\F''})$ and therefore,
\[ \deg{\F} = \deg{\det{\F}} = \deg{\det{\F'}} + \deg{\det{\F''}} = \deg{\F'} + \deg{\F''} \]

\subsection{7}

\subsubsection{7.1}

Let $(X, \struct{X})$ be a locally ringed space and $f : \L \to \I$ a surjective map of invertible sheaves on $X$. Then for each $x \in X$ the map $f_x : \L_x \to \I_x$ is a surjective map of free rank one $\stalk{X}{x}$-modules. Then we have get a diagram of $\stalk{X}{x}$-module morphisms,
\begin{center}
\begin{tikzcd}
\L_x \arrow[r, "f_x", two heads] \arrow[d, "\sim"] & \I_x \arrow[d, "\sim"]
\\
\stalk{X}{x} \arrow[r, dashed, two heads] & \stalk{X}{x}
\end{tikzcd}
\end{center}
Therefore, it suffices to prove that if a $\stalk{X}{x}$-module map $\stalk{X}{x} \to \stalk{X}{x}$ is surjective then it is injective. Such a map satisfies $f(a) = a \cdot f(1)$ and since $f$ is surjective we must have $a \cdot f(1) = 1$ for some $a$ and therefore $f(1)$ is a unit so $f(a) = a \cdot f(1)$ is injective.
\bigskip\\
The fact that $\stalk{X}{x}$ is local in not necessary. In fact, we have the following more general fact.

\begin{theorem}
Let $(X, \struct{X})$ be a ringed space and $f : \F \to \K$ be a surjective map of finite locally free $\struct{X}$-modules of equal rank then $f : \F \to \K$ is an isomorphism. 
\end{theorem}

\begin{proof}
For each $x \in X$ the map $f_x : \F_x \to \K_x$ is surjective and since $\F$ and $\K$ are finite locally free both of rank $n$ we have a diagram,
\begin{center}
\begin{tikzcd}
\F_x \arrow[r, "f", two heads] \arrow[d, "\sim"] & \K_x \arrow[d, "\sim"]
\\
\stalk{X}{x}^{\oplus n} \arrow[r, dashed] & \stalk{X}{x}^{\oplus n}
\end{tikzcd}
\end{center}
Now since $\stalk{X}{x}^{\oplus n}$ as a finitely generated $\stalk{X}{x}$-module,  we use that any surjective endomorphism of a finitely generated module (Noetherian is not necessary) is injective to conclude that $f_x : \F_x \to \K_x$ is injective and thus $f : \F \to \K$ is an isomorphism. 
\end{proof}

\subsubsection{7.2}

Let $X$ be a scheme over a field $k$ and $\L$ an invertible sheaf on $X$. Let $\{ s_0, \dots, s_n \}$ and $\{ t_0, \dots, t_m \}$ be two sets of sections of $\L$ which generate the same subscpace $V \subset \Gamma(X, \L)$ and which generate the sjeaf $\L$ at every point. Suppose WLOG $n \le m$. Let $\varphi : X \to \P^n_k$ and $\psi : X \to \P^m_k$ be the corresponding morphisms. There is a matrix $a_{ij} \in k$ such that $t_j = s_i a_{ij}$. Then $a_{ij}$ defines a rational map $\P^m_k \rat \P^n_k$ which is projection from the linear subspace $L$ defined by $\ker{(a_{ij})}$ composed with an automorphism $\P^n_k \to \P^n_k$. This is because any surjective linear transformation $T : V \onto W$ can be factored as $T : V \onto V/\ker{T} \iso W$. Finally, becasue $t_j = s_i a_{ij}$ the following diagram commutes,
\begin{center}
\begin{tikzcd}
X \arrow[r, "\psi"] \arrow[dr, "\varphi"'] & \P^m_k \arrow[d, dashed]
\\
& \P^n_k
\end{tikzcd}
\end{center}

\subsubsection{7.3 DO!!}

Let $\varphi : \P^n_k \to \P^m_k$ be a morphism. Then $\varphi^* \struct{\P^m_k}(1) = \struct{\P^m_k}(n)$ for $n = \deg{\varphi}$ which is a nonegative integer. We put explicit coordinates on these spaces: $\P^n_k = \Proj{k[x_0, \dots, x_n]}$ and $\P^m_k = \Proj{k[y_0, \dots, y_m]}$. Suppose that $n = 0$ then $\varphi^* y_i = a_i \in k$ and thus $\varphi(\P^n_k) = [a_0 : \cdots : a_m]$ is a single point. Otherwise $n > 0$ and the image is not a single point (else clearly $\varphi^* \struct{\P^m_k}(1) = \struct{\P^n_k}$ since every line bundle over a point is trivial). 
\bigskip\\
Now if $n > 0$ we require that $\varphi^* y_i$ globally generate $\struct{\P^n_k}(n)$ 

(FINISH THIS!!!)

\subsubsection{7.4}

\begin{enumerate}
\item Let $X$ be finite type over a Noetherian ring $A$. Suppose $\L$ is an ample line bundle on $X$. Then we know that for some $n > 0$ the line bundle $\L^{\otimes n}$ is very ample. Therefore, there must be an immersion $i : X \to \P^r_A$ for some $r > 0$. Now the immersion $i$ can be factored as,
\begin{center}
\begin{tikzcd}
X \arrow[r, "j", hook] & U \arrow[r, "q", hook] & \P^r_A
\end{tikzcd}
\end{center} 
where $j : X \to U$ is a closed immersion and $q : U \to \P^r_A$ is an open immersion. Since $\P^r_A$ is a separated scheme over $A$ then $U \embed \P^r_A$ must be separated over $A$. Now we apply the following lemma. 

\begin{remark}
Hartshorne is wrong about the definition of an immersion which he defines to be a morphsism giving a homeomorphism onto an open subscheme of a closed subscheme and thus an open immersion followed by a closed immersion. This is backwards, an immersion is a closed immersion followed by an open immersion. The two definitions are incompatible (see Tag 01QW) and Hartshorne's definition is not even stable under composition so it is a bad notion.  
\end{remark}

\begin{lemma}
Let $f : X \to Y$ be a closed immersion of schemes over $S$. If $Y$ is separated then $X$ is separated.
\end{lemma}

\begin{proof}
Consider the compositions,
\begin{center}
\begin{tikzcd}
X \arrow[rd, "f", bend right] \arrow[r, "\Delta_X"] & X \times_S X \arrow[r, "f \times f"] & Y \times_S Y
\\
& Y \arrow[ru, "\Delta_Y", bend right]
\end{tikzcd}
\end{center}
Since $\Delta_Y$ and $f$ are both closed immersions, by separatedness of $Y$ and hypothesis respectily, the composition $\Delta_Y \circ f = (f \times f) \circ \Delta_X$ is a closed immersion. Furthermore, $f \times f : X \times_S X \to Y \times_S Y$ is a closed immersion which implies that $\Delta_X : X \to X \times_S X$ must also be a closed immersion since it must have surjective sheaf map if the composition does and since $f \times f$ and $(f \times f) \circ \Delta_X$ are homeomorphisms onto a closed image then for any closed $Z \subset X$ the image $\Delta_X(Z)$ must be closed because it maps to a closed set under $f \times f$ which is a homeomorphism onto its closed immage.
\bigskip\\
A better proof is to show that closed immersions are separated and composition of separated morphisms are separated so $X \to Y \to S$ gives $X \to S$ separated and thus $X$ is separated as a scheme over $S$.
\end{proof}

\item Consider the $X$ the affine line with a doubled point and let $U$ and $V$ the the open affine copies of $\A^1_k$ and $W$ the single glued open $U \setminus \{ P_1 \} = V \setminus \{ P_2 \}$. Since $X$ is not separated we cannot apply Weil divisors. However, $X$ is integral so the map $\CaCl{X} \to \Pic{X}$ is an isomorphism so we need only consider Cariter divisors. 
\bigskip\\
First we compute Cartier divisors on $\A^1_k$. We know that $k[x]$ is a UFD and thus its class group is trivial so $\CaCl{\A^1_k} = 0$. Now any Cartier diviosr $D \in \Gamma(X, \K^\times_X / \struct{X}^\times)$ its restriction to $U$ and $V$ must be principal i.e. rational functions $f, g \in K(\A^1_k) = k(x)$. Furthermore, on the overlap, we must have $f/g \in \struct{X}^\times(W)$ but $\struct{X}^\times(W) = k[x]_{(x)}$ then $f / g = a x^n$ for $a \in k^\times$ and $n \in \Z$. Then quotienting by principal divisors $k(x)$ we can set $g = 1$ and quoting by units on $U$ we can set $a = 1$ so we find $f = x^n$. This gives $\CaCl{X} = \Z$. 
\bigskip\\
Now we construct the line bundles defined by the Cartier divisor $D_n$ which is defined by $D|_U = x^n$ and $D|_V = 1$. Then, $\L_n = \L(D_n)$ satisfies $\L_n|_U = x^{-n} \struct{X}|_U$ and $\L_n|_V = \struct{X}|_V$.     
Now the diagram,
\begin{center}
\begin{tikzcd}
& \L_n(X) \arrow[ld] \arrow[rd]
\\
\L_n(U) \arrow[dr] & & \L_n(V) \arrow[dl]
\\
& \L_n(W)
\end{tikzcd}
\end{center}
is cartesian. So we have,
\begin{center}
\begin{tikzcd}
& \L_n(X) \arrow[ld] \arrow[rd]
\\
x^{-n} k[x] \arrow[dr] & & k[x] \arrow[dl]
\\
& k[x]_{(x)}
\end{tikzcd}
\end{center}
However, in $k[x]_{(x)}$ the intersection $x^{-n} k[x] \cap k[x] = k[x]_{\ge -n}$ meaning terms of at least degree $-n$. However, since $f$ and $g$ satisfy $f/g = x^n$ and therefore for effective divisors corresponding to the vanishing of sections $s \in \L_n$ we have $f,g \in k[x]$. Therefore, if $n > 0$ we see that $f$ vanishes at $x = 0$ and if $n < 0$ we see that $g$ vanishes at $x = 0$. Therefore, if $n > 0$ then $\L_n$ has base locus containing the first origin and if $n < 0$ the base locus of $\L_n$ contains the second origin. Therefore, the only line bundle on $X$ which is generated by global sections is $\L_0 = \struct{X}$. Therefore, there cannot be any ample line bundles on $X$ since any line bundle is of the form $\L_m$ but $\L_m^{\otimes n} = \L_{mn}$ is not globally generated $m \neq 0$ and furthermore, 
\[ \F \otimes_{\struct{X}} \L_0^{\otimes n} = \F \otimes_{\struct{X}} \struct{X}^{\otimes n} = \F \]
is not generated by global sections for $\F = \L_1$ so $\L_0 = \struct{X}$ is not ample. 
\end{enumerate}

\subsubsection{7.5}


Let $X$ be a Noetherian scheme and $\L$ and $\M$ be line bundles.

\begin{enumerate}
\item Suppose that $\L$ is ample and $\M$ is generated by global sections. For any coherent $\struct{X}$-module $\F$ there is some $n(\F)$ such that for all $n \ge n(\F)$ we have,
\[ \F \otimes_{\struct{X}} \L^{\otimes n} \]
is generated by global sections. By $\M$ is generated by global sections and thus so is $\M^{\otimes n}$ so,
\[  \F \otimes_{\struct{X}} (\L^{\otimes n} \otimes_{\struct{X}} \M^{\otimes n}) \]
is generated by global sections. Since this holds for any $n \ge n(\F)$ the sheaf $\L \otimes_{\struct{X}} \M$ is ample. (See Tag 01AO)

\item Let $\M$ be a line bundle. Since $\L$ is ample for sufficiently large $n$ the sheaf $\L^n \otimes_{\struct{X}} \M$ is generated by global sections and thus by the previous part,
\[ \L \otimes_{\struct{X}} (\L^{\otimes n} \otimes_{\struct{X}} \M) = \L^{\otimes n+1} \otimes_{\struct{X}} \M \]
is ample.
\item Now let $\L$ and $\M$ be ample. There must exist $n$ such that,
\[ \M \otimes_{\struct{X}} \L^{\otimes n} \]
is generated by global sections and thus, since $\M$ is ample, by (a) we have,
\[ (\M \otimes_{\struct{X}} \L^{\otimes n}) \otimes_{\struct{X}} \M^{\otimes n - 1} =  \L^{\otimes n} \otimes_{\struct{X}} \M^{\otimes n} = (\L \otimes_{\struct{X}} \M)^{\otimes n} \]
is ample. This implies that $\L \otimes_{\struct{X}} \M$ is ample. 

\item Now let $X$ be finite type over a noetherian ring $A$. Suppose that $\L$ is very ample and $\M$ is generated by global sections. Since $\L$ is very ample there must be an immersion $i : X \to \P^n_A$ such that $\L = i^* \struct{\P^n_A}(1)$. Furthermore a choice of sections generating $\M$ defines a map $j : X \to \P_A^m$ such that $\M = j^* \struct{\P^m_A}(1)$. Now consider the product under the Segre embedding,
\begin{center}
\begin{tikzcd}
X \arrow[r, "\Delta"] & X \times_A X \arrow[r, "i \times j"] & \P^n_A \times_A \P^m_A \arrow[r] & \P^{N}_A
\end{tikzcd}
\end{center} 
Thus it suffices to prove that $q : X \to \P^N_A$ is an immersion and,
\[ q^* \struct{\P^N_A}(1) = \L \otimes_{\struct{X}} \M \]
which implies that $\L \otimes_{\struct{X}} \M$ is very ample. 
\bigskip\\
Under the Segre embedding $s : \P_A^n \times_A \P^m_A \to \P_A^N$ we have,
\[ s^* \struct{\P^N_A}(1) = p_1^* \struct{\P^n_A}(1) \otimes_{\struct{}} p_2^* \struct{\P^m_A}(1) \]
Now, consider,
\begin{align*}
(i, j)^* s^* \struct{\P^N_A}(1) & = (i, j)^* [p_1^* \struct{\P^n_A}(1) \otimes_{\struct{}} p_2^* \struct{\P^m_A}(1)] 
\\
& = [(i, j)^* p_1^* \struct{\P^n_A}(1)] \otimes_{\struct{X}} [(i, j)^*  p_2^* \struct{\P^m_A}(1)] 
\\
& = [p_1 \circ (i, j)]^* \struct{\P^n_A}(1)] \otimes_{\struct{X}} [p_2 \circ (i, j)]^* \struct{\P^m_A}(1)
\\
& = i^* \struct{\P^n_A}(1) \otimes_{\struct{X}} j^* \struct{\P^m_A}(1)
\\
& = \L \otimes_{\struct{X}} \M 
\end{align*}
Now we need to show that $s \circ (i \times j) \circ \Delta$ is an embedding. Using the lemma, $(i, j) = (i \times j) \circ \Delta$ is an embedding since $i : X \to \P^n_A$ is an embedding. Furthermore, $s : \P^n_A \times_A \P^m_A \to \P^N_A$ is an embedding so $s \circ (i, j) \circ \Delta$ is an embedding. 

\item Let $X$ be finite type over a noetherian ring $A$ and $\L$ an ample sheaf on $X$. We know there exists some $n_0 > 0$ such that $\L^{\otimes n_0}$ is very ample and, as a consequence, generated by global sections. Furthermore, for $n \ge n_1$ we know $\L^{\otimes n}$ is generated by global sections since $\L$ is ample. Now, for any $n \ge n_0 + n_1$ the sheaf,
\[ \L^{\otimes n} = \L^{\otimes n_0} \otimes_{\struct{X}} \L^{\otimes (n - n_0)} \] 
is ample by the previous result since $\L^{\otimes n_0}$ is ample and $\L^{\otimes (n - n_0)}$ is generated by global sections because $n - n_0 \ge n_1$.   
\end{enumerate}

\begin{lemma}
Tensor product of sheaves commutes with pullback.
\end{lemma}

\begin{proof}
Let $f : X \to Y$ be a morphism and $\F$ and $\G$ be $\struct{Y}$-modules on $Y$ and $\K$ a $\struct{X}$-module on $X$ then,
\begin{align*}
\Hom{\struct{X}}{f^* \F \otimes_{\struct{X}} f^* \G}{\K} & = \Hom{\struct{X}}{f^* \F}{\shHom{\struct{X}}{f^* \G}{\K}}
\\
& = \Hom{\struct{Y}}{\F}{f_* \shHom{\struct{X}}{f^* \G}{\K}}
\\
& = \Hom{\struct{Y}}{\F}{\shHom{\struct{X}}{\G}{f_* \K}}
\\
& = \Hom{\struct{Y}}{\F \otimes_{\struct{Y}} \G}{f_* \K}
\\
& = \Hom{\struct{X}}{f^* (\F \otimes_{\struct{Y}} \G)}{\K}
\end{align*}
Therefore, by Yoneda, $f^* \F \otimes_{\struct{X}} f^* \G \cong f^* (\F \otimes_{\struct{Y}} \G)$. 
\end{proof}

\begin{lemma} \label{product_of_immersion}
Let $f : X \to Y$ be an immersion and $g : X \to Z$ is any morphism all over $S$ then $X \to Y \times_S Z$ is an immersion. 
\end{lemma}

\begin{proof}
The map $X \to Y \times_S Z$ can be factored into the graph morphism,
\[ \Gamma_g = (\id_X, g) : X \to X \times_S Z \]
and the product $f \times \id_Z : X \times_S Z \to Y \times_S Z$,
\begin{center}
\begin{tikzcd}
X \arrow[r, "\Gamma_g"] & X \times_S Z \arrow[r, "f \times \id"] & Y \times_S Y
\end{tikzcd}
\end{center}
It suffices to show that both maps are immersions. Since $f$ and $\id_Z$ are immersion then $f \times \id_Z : X \times_S Z \to Y \times_S Z$ is an immersion. Furthermore, the morphism $\Gamma_g$ can be obtained via a base extension of $\Delta : Z \to Z \times_S Z$ along the map $X \times_S Z \xrightarrow{g \times \id_Z} Z \times_S Z$ since,
\begin{center}
\begin{tikzcd}
X \arrow[r, "\Gamma_g"] \arrow[d, "g"] & X \times_S Z \arrow[d, "g \times \id_Z"]
\\
Z \arrow[r, "\Delta"] & Z \times_S Z
\end{tikzcd}
\end{center}
is cartesian because $(X \times_S Z) \times_{Z \times_S Z} Z = X$. (BE EXPLICIT) Since immersions are stable under base change, the morphism $\Gamma_g : X \to X \times_S Z$ is an immersion. Thus $(f, g) =  (f \times \id) \circ \Gamma_g$ is a composition of immersions and thus an immersion. 
\end{proof}

\begin{corollary}
If with the above data $f : X \to Y$ is a closed immersion and $Z$ is separated then $X \to Y \times_S Z$ is a closed immersion.
\end{corollary}

\begin{proof}
The above proof holds equally for closed immersions since they are stable under products and base extensions and composition. However, the map $\Delta : Z \to Z \times_S Z$ must be a closed immersion for the base extension to be a closed immersion so we must assume $Z$ is separated. 
\end{proof}

\subsubsection{7.6}

Let $X$ be a nonsingular projective varitety over an algebraically closed field $k$ (WHY??). Let $D$ be a divisor and $\L$ the associated line bundle. We consider $f(n) = \dim |n D| = \dim_k \Gamma(X, \L^n) - 1$.

\begin{enumerate}
\item Let $\L$ be very ample and $\iota : X \embed \P^n_k$ the corresponding embedding into projective space. Then $\L = \iota^* \struct{\P^n_k}(1)$ and therefore, by the projection formula,
\[ \iota_* \L^n = \iota_* \struct{X}(n) \]
Therefore, $\dim_k \Gamma(X, \L^n) = \dim_k \Gamma(\P^n_k, \iota_* \struct{X}(n)) = P_X(n)$ for sufficiently large $n$ where $P_X(n)$ is the Hilbert polynomial. Therefore, for $n \gg 0$,
\[ f(n) = P_X(n) - 1 \]

\item Let $D$ be torsion in $\Pic{X}$ of order $r$. Then we know $r D = 0$ or equivalently, $\L^{\otimes r} \cong \struct{X}$. Therefore, it is clear that,
\[ \dim |rkD| = \dim_k \Gamma(X, (\L^{\otimes r})^{\otimes k}) - 1 = \Gamma(X, \struct{X}) - 1 = 0 \]
and therefore $\dim |n D| = 0$ if $r \divides n$. Otherwise, for $n \neq rk$ I claim that $\Gamma(X, \L^{\otimes n}) = 0$. Indeed, suppose we have a global section $s : \struct{X} \to \L^{\otimes n}$ then $s^{\otimes r} : \struct{X} \to \L^{\otimes rn}$ must be constant because $\L^{\otimes rn} \cong \struct{X}$ i.e. $\div{(s^{\otimes r})} = 0$. Therefore, locally, $s^r_x \in k \subset \stalk{X}{x}$ so suppose that $s^r_x = 0$ then because $X$ is integral $s_x = 0$ and thus $s = 0$ because global sections $\struct{X} \to \struct{X}$ are constant. Otherwise, $s$ must be nonvanishing i.e. $s_x \in \kappa(x)$ is nonzero, because $s^r_x \in \kappa(x)$ is nonzero. Thus, $s_x : \stalk{X}{x} \to \stalk{X}{x}$ is an endomorphism such that $s_x \otimes \id : \kappa(x) \to \kappa(x)$ is an isomorphism so $s_x$ is an isomorphism showing that $s : \struct{X} \to \L^{\otimes n}$ is an isomorphism contradicting the fact that $n$ is not divisible by the order of $\L$. Therefore, if $r \ndivides n$ then $\Gamma(X, \L^{\otimes r}) = 0$ and thus $f(n) = \dim{|nD|} = -1$. 
\end{enumerate}

(WHERE WAS NONSINGULAR NEEDED AND WHERE WAS ALGEBRAICALLY CLOSED NEEDED??)

\subsubsection{7.7 DO!}

\subsubsection{7.8}

Let $X$ be a Noetherian scheme and $\E$ a coherent locally free sheaf on $X$ and $\pi : \P(\E) \to X$. We want to show that sections $\sigma : X \to \P(\E)$ of $\pi : \P(\E) \to X$ and quotient invertible sheaves $\E \to \L \to 0$. This follows immediately from Proposition 7.12.

\subsubsection{7.9}

Let $X$ be a regular noetherian scheme, and $\E$ a locally free coherent sheaf of rank $r \ge 2$ on $X$. Consider $\pi : \P_X(\E) \to X$.
\bigskip\\
We also need to assume that $X$ is connected, and thus integral, otherwise we could take a collection of points and the formula fails. 

\begin{enumerate}
\item There is a map $\Phi : \Pic{X} \oplus \Z \to \Pic{\P(\E)}$ via $(\L, n) \mapsto \pi^* \L \otimes \struct{\P(\E)}(n)$. First, we show that $\Phi$ is injective. If $\pi^* \L \otimes \struct{\P(\E)}(n)$ is trivial, by restricting to a fiber $\P^{r-1} \to \Spec{\kappa(x)}$ we see that $n = 0$ so it suffices to show that $\L \cong \struct{X}$. Because $\pi_* \struct{\P(\E)} = \struct{X}$ and $\pi^* \L$ is locally free we know that $\L \to \pi_* \pi^* \L$ is an isomorphism. Thus, if $\pi^* \L \cong \struct{\P(\E)}$ then,
\[ \L \to \pi_* \struct{\P(\E)} = \struct{X} \]
is an isomorphism showing that $\Phi$ is injective. 
\bigskip\\
To show 
Let $U \subset X$ be a dense open on which $\E|_U \cong \struct{U}^{\oplus r}$ and $Z = X \setminus U$ is a divisor with irreducible (prime) components $Z_1, \dots, Z_k$. Then $\pi^{-1}(U) \cong \P^{r-1} \times U$ and $\pi^{-1}(Z)$ is a divisor because $\pi^{-1}(Z_i)$ is a projective bundle over $Z_i$ and this a prime divisor of $\P(\E)$. Then consider the diagram,
\begin{center}
\begin{tikzcd}
\Z^{k} \arrow[r] & \Pic{\P(\E)} \arrow[r] & \Pic{U \times \P^{r-1}} \arrow[r] & 0
\\
\Z^{k} \oplus \Z \arrow[u, two heads] \arrow[r] & \Pic{X} \oplus \Z \arrow[r] \arrow[u] & \Pic{U} \oplus \Z \arrow[u, two heads] \arrow[r] & 0
\end{tikzcd}
\end{center}
We know that $\Pic{U} \oplus \Z \to \Pic{U \times \P^{r-1}}$ is an isomorphism by Ex. 6.1. Since the leftmost and rightmost maps are surjective, then $\Phi : \Pic{X} \oplus \Z \to \Pic{\P(\E)}$ is surjective by the four lemma. (WHY REGULARITY NEEDED?)

\item Let $\E$ and $\E'$ be locally free coherent sheaves on $X$. Then $\P(\E)$ represents the functor $F_\E$ taking a scheme $f : T \to X$ to isomorphism classes of rank one quotients of $f^* \E$. By Yoneda, isomorphisms $\P(\E) \iso \P(\E')$ correspond to natural isomorphisms $\eta : F_{\E} \to F_{\E'}$. Suppose that $\E' \cong \E \otimes \L$ for a line bundle $\L$. Then $\eta_\L : F_\E \to F_{\E'}$ via,
\[ (f^* \E \to \mathcal{Q} \to 0) \mapsto (f^* \E' \to \mathcal{Q} \otimes \L \to 0) \] 
is a natural isomorphism because $\L$ is invertible. Thus $\varphi_\L : \P(\E) \iso \P(\E')$ is an isomorphism. Conversely, a map $\varphi : \P(\E) \to \P(\E')$ is equivalent to the data of an exact sequence,
\[ \pi^* \E' \to \varphi^*\struct{\P(\E')}(1) \to 0 \]
If $\varphi$ is an isomorphism then $\varphi^* : \Pic{\P(\E)} \to \Pic{\P(\E')}$ is an isomorphism preserving $\Pic{X}$. Using the decomposition $\Pic{\P(\E)} = \Pic{X} \oplus \Z$, the map $\Z \to \Pic{\P(\E')} \to \Pic{\P(\E)} \to \Z$ must be the identity (sections must pull back so $1 \mapsto -1$ is impossible) meaning that $\varphi^* \struct{\P(\E')}(1) = \pi^* \L \otimes \struct{\P(\E)}(1)$. Likewise, by the same argument,
\[ (\varphi^{-1})^* \struct{\P(\E)}(1) = \pi'^* \L' \otimes \struct{\P(\E')}(1) \]
Composing these maps,
\[ \struct{\P(\E)}(1) = \varphi^* (\varphi^{-1})^* \struct{\P(E)}(1) = \pi^* \L \otimes \pi^* \L' \otimes \struct{\P(E)}(1) \]
Since the map $\Pic{X} \to \Pic{\P(\E)}$ is injective we see that $\L \otimes \L' \cong \struct{X}$ so $\L' \cong \L^\vee$. Furthermore, $(\varphi^{-1})^* = \varphi_*$ and recall that $\E = \pi_* \struct{\P(\E)}(1)$. Thus, using the projection formula,
\[ \E = \pi_* \struct{\P(\E)}(1) = \pi'_* \varphi_* \struct{\P(\E)} = \pi'_* (\pi'^* \L' \otimes \struct{\P(\E')}(1)) = \L' \otimes \E' \]
but $\L' \cong \L^{\otimes -1}$ so,
\[ \E' \cong \E \otimes \L \]
\end{enumerate}

\subsubsection{7.10}

\subsubsection{7.11}

\subsubsection{7.12}

Let $X$ be a Noetherian scheme and $Y,Z \subset X$ closed subschemes not contained in eachother. Let $\tilde{X} \to X$ be the blowing up of $X$ along $Y \cap Z$ i.e. $\I_Y + \I_Z$. Consider the strict transforms $\tilde{Y}$ and $\tilde{Z}$ 


\subsubsection{7.14 (CHECK!)}

\begin{enumerate}
\item Let $X = \P^1$ and $\E = \struct{X} \oplus \struct{X}(-n)$ for $n > 0$. Then $\pi : \P(\E) \to X$ is the Hirzebruch surface $H_n$ over $\P^1$. We know there is an isomorphism $\pi_* \struct{\P(\E)}(1) \cong \E$ and therefore,
\[ H^0(\P(\E), \struct{\P(\E)}(1)) = H^0(X, \pi_* \struct{\P(\E)}(1)) = H^0(X, \E) = 0 \]
because $\struct{X}(-n)$ has no nontrivial global sections on $X$. Therefore, $\struct{\P(\E)}$ cannot be very ample relative to $X$ since it does not have enough global sections and $\P(\E) \to X$ has positive dimensional fibers.

\item Let $f : X \to Y$ be a finite type morphism, $\L$ an ample invertible sheaf on $X$, and $\A$ a sheaf of graded $\struct{X}$-algebras satisfying ($\dagger$). Let $P = \rProj{X}{\A}$ and $\pi : P \to X$ and $\struct{P}(1)$ the relative antitautological bundle. By Prop. 7.10, for some $n$ the invertible sheaf $\struct{P}(1) \otimes \pi^* \L^n$ is very ample relative to $X$. Furthermore, since $\L$ is ample for sufficiently large $m$ we know $\L^{\otimes m}$ is very ample relative to $f : X \to Y$ (CHECK THIS!!). Therefore, by Ex. 5.12 we know that,
\[ (\struct{P}(1) \otimes \pi^* \L^{\otimes n}) \otimes \L^{\otimes m} = \struct{P}(1) \otimes \pi^* \L^{\otimes n + m} \]
is very ample relative to $P \to Y$. Furthermore, since $\L$ is ample we know that for $\ell > \ell_0$ the bundle $\L^{\otimes \ell}$ is generated by global sections and therefore using a previous exercise,
\[ (\struct{P}(1) \otimes \pi^* \L^{\otimes n + m}) \otimes \L^{\otimes \ell}) = \struct{P}(1) \otimes \pi^* \L^{\otimes n + m + \ell} \]
is very ample relative to $P \to Y$. Thus for $k > k_0 = n + m + \ell_0$ we have shown that, $\struct{P}(1) \otimes \pi^* \L^{\otimes k}$ is very ample relative to $P \to Y$.
\end{enumerate}

\subsection{8}

\subsubsection{8.1}

\begin{enumerate}
\item Let $b$ be a local ring containing a field $k \subset B$ such that $\kappa = B / \m$ is a separably generated extension of $k$. Consider the exact sequence,
\begin{center}
\begin{tikzcd}
\m / \m^2 \arrow[r, "\delta"] & \Omega_{B/k} \otimes_B \kappa \arrow[r] & \Omega_{\kappa/k} \arrow[r] & 0
\end{tikzcd}
\end{center}
we want to prove exactness on the left.
Consider the $\kappa$-dual,
\[ \delta^* : \Hom{\kappa}{\Omega_{B/k} \otimes_B \kappa}{\kappa} \to \Hom{\kappa}{\m/\m^2}{\kappa} \]
Let $A = B / \m^2$ then there is an exact sequence,
\begin{center}
\begin{tikzcd}
\m^2 / \m^4 \arrow[r] & \Omega_{B/k} \otimes_B A \arrow[r] & \Omega_{A/k} \arrow[r] & 0
\end{tikzcd}
\end{center}
and thus we get a surjection,
\begin{center}
\begin{tikzcd}
\Omega_{B/k} \otimes_B \kappa \arrow[r] & \Omega_{A/k} \otimes_A \kappa \arrow[r] & 0
\end{tikzcd}
\end{center}
Since $A$ is a complete local ring, by 8.25A, there is a subfield $K \subset A$ such that $K \to A \to \kappa$ is an isomorphism. Therefore, taking the $K$-dual gives an injection,
\[ \Hom{K}{\Omega_{A/k} \otimes_A \kappa}{\kappa} \embed \Hom{K}{\Omega_{B/k} \otimes_B \kappa }{\kappa} \]
It suffices to show that,
\[ \Hom{K}{\Omega_{A/k} \otimes_A \kappa}{\kappa} \embed \Hom{K}{\Omega_{B/k} \otimes_B \kappa }{\kappa} \xrightarrow{\delta^*} \Hom{\kappa}{\m/\m^2}{\kappa} \]
is surjective. However,
\[ \Hom{K}{\Omega_{A/k} \otimes_A K}{\kappa} = \Hom{K}{\Omega_{A/k}}{\kappa} = \Der{K}{A}{\kappa} \]
giving the natural map $\Der{K}{A}{\kappa} \to \Hom{\kappa}{\m/m^2}{\kappa}$ defined because any $K$-derivation $\varphi : A \to \kappa$ satisfies $\varphi(\m^2) = 0$.
\bigskip\\
Given any $\kappa$-linear map $\varphi : \m / \m^2 \to \kappa$ we construct a $K$-derivation $D : A \to \kappa$ as follows. Since $K \to A \to \kappa$ is an isomorphism, for any $a \in A$ we can write $a = \lambda + c$ with $\lambda \in K$ and $c \in \m$ uniquely. Define $D(a) = \varphi(c)$ then,
\[ D(aa') = D(\lambda \lambda' + \lambda c' + \lambda' c + c c') = \lambda \varphi(c') + \varphi(c) \lambda' = a D(a') + D(a) a' \]
since $\varphi$ is $\kappa$ linear it is also $K$-linear under $K \subset A \to \kappa$. Therefore, $\delta^*$ is surjective so $\delta$ is injective.

\item Assume furthermore $k$ is perfect and $B$ is a localization of a finite type $k$-algebra. 
\bigskip\\
The exact sequence,
\begin{center}
\begin{tikzcd}
0 \arrow[r] & \m / \m^2 \arrow[r] & \Omega_{B/k} \otimes_B \kappa \arrow[r] & \Omega_{\kappa/k} \arrow[r] & 0
\end{tikzcd}
\end{center}
shows that,
\[ \dim_\kappa \Omega_{B/k} \otimes_B \kappa = \dim_\kappa \m/\m^2 + \dim_\kappa \Omega_{\kappa/k} \]
Furthermore, $\dim_\kappa \Omega_{\kappa / k} = \trdeg{k}{\kappa}$ because $\kappa / k$ is separably generated. Therefore,
\[ \dim_\kappa \Omega_{B/k} \otimes_B \kappa = \dim_k \m/\m^2 + \trdeg{k}{\kappa} \]
and thus if $B$ is regular if and only if
\[ \dim_\kappa \Omega_{B/k} \otimes_B \kappa = \dim{B} + \trdeg{k}{\kappa} \]
In particular, if $\Omega_{B/k}$ is free of rank $\dim{B} + \trdeg{k}{\kappa}$ then $B$ is regular. 
\bigskip
Now suppose that $B$ is regular. Then $B$ is a domain so let $K = \Frac{A}$,
\[ \Omega_{B/k} \otimes_B K = \Omega_{K/k} \]
and $K / k$ is separably generated because $k$ is perfect so again,
\[ \dim_{K} \Omega_{B/k} \otimes_B K = \trdeg{k}{K} \]
We know $B$ is the localization of a finite type $k$-algebra domain $A$ and thus $K = \Frac{A}$ and,
\[ \trdeg{k}{K} = \dim{A} = \dim{B} + \trdeg{k}{\kappa} \]
Therefore, the rank of $\Omega_{B/k}$ is free of rank $\dim{B} + \trdeg{k}{\kappa}$ by Lemma 8.9.

\item Let $X$ be an irreducible scheme of finite type over a perfect field $k$. Let $\dim{X} = n$. For $x \in X$, we know $B = \stalk{X}{x}$ is a localization of a finite type $k$-algebra and $(\Omega_{X/k})_x = \Omega_{B/k}$. By the previous part, $B$ is a regular local ring iff $\Omega_{B/k}$ is free of rank $\dim{B} + \trdeg{k}{\kappa} = n$. Therefore, 
\[ \stalk{X}{x} \text{ is regular } \iff (\Omega_{X/k})_x \text{ is free of rank } n \]

\item Let $X$ be an irreducible scheme of finite type over a perfect field.
\bigskip\\
If $\stalk{X}{x}$ is regular then $(\Omega_{X/k})_x$ is free of rank $n$. Since $X$ is noetherian and $\Omega_{X/k}$ is coherent, then $\Omega_{X/k}$ is free of rank $n$ on a neighborhood on $x$ by (Ex. 5.7). Therefore,
\[ U = \{x \in X \mid \stalk{X}{x} \text{ is regular} \} \]
is open. 
\bigskip\\
Assume further that $X$ is integral. Then $\dim{X} = \trdeg{k}{K}$ where $K$ is the function field. Let $\xi \in X$ be the generic point. Because $k$ is perfect, $K / k$ is seperably generated and thus $(\Omega_{X/k})_\xi = \Omega_{K/k}$ is free of rank $\trdeg{K}$ so $\xi \in U$. Therefore $U$ is a dense open.
\end{enumerate}

\subsubsection{8.2}

Let $X$ be a variety over $k$ of dimension $n$ and let $\E$ be a locally free sheaf on $X$ of rank $r$ with $r > n$ and let $V \subset \Gamma(X, \E)$ be a vector space of global sections which globally generate $\E$. Then $V \otimes_k \struct{X} \onto \E$ is surjective so in particular $V \otimes_k \kappa(x) \onto \E_x \otimes_{\stalk{X}{x}} \kappa(x)$ meaning that $\dim_k{V} \ge \rank{(\E)} > \dim{X}$. Write $d = \dim_k{V}$ then $d \ge r > n$. 
Consider the bad locus $B \subset X \times \P(V)$ defined by,
\[ B = \{ (x, [s]) \mid x \in X \text{ and } s \in V \setminus \{ 0 \} \text{ and } s_x \in \m_x \E_x \} \]
The fibers of the map $B \to X$ correspond to $[s] \in \P(V)$ such that $s_x \in \m_x \E_x$ i.e. the kernel of the map $V \otimes_k \kappa(x) \onto \E_x \otimes \kappa(x)$. Thus the fibers of $B \to X$ are isomorphic to $\P^{d - r - 1}$ and thus $\dim{B} = n + d - r - 1$. Now consider the projection $B \to \P(V)$. However,  $n < r$ and thus $\dim{B} < d - 1 = \dim{\P(V)}$ meaning that $B \to \P(V)$ cannot be surjective. Therefore, there exists a nonzero $s \in V$ such that $s_x \notin \m_x \E_x$ for every $x \in X$.
\bigskip\\
Consider the map $s : \struct{X} \to \E$. On stalks, $s_x : \stalk{X}{x} \to \E_x$ is injective because $s_x \neq 0$ and $\E_x$ is free and $\stalk{X}{x}$ is a domain. Thus $s : \struct{X} \to \E$ is injective. Let $\E' = \coker{(s : \struct{X} \to \E)}$. At each $x \in X$ there is an exact sequence,
\begin{center}
\begin{tikzcd}
0 \arrow[r] & \stalk{X}{x} \arrow[r, "s_x"] & \E_x \arrow[r] & \E'_x \arrow[r] & 0
\end{tikzcd}
\end{center}
furthermore, tensoring by $\kappa(x)$ we get an exact sequence,
\begin{center}
\begin{tikzcd}
0 \arrow[r] & \kappa(x) \arrow[r] & \E_x \otimes_{\stalk{X}{x}} \kappa(x) \arrow[r] & \E'_x \otimes_{\stalk{X}{x}} \kappa(x) \arrow[r] & 0
\end{tikzcd}
\end{center}
which is exact on the left because $s_x \notin \m_x \E_x$ so the map $\kappa(x) \to \E_x / \m_x \E_x$ is nonzero and thus injective because these are vector spaces. Therefore $\rank_x(\E') = \rank_x(\E) - 1 = r - 1$ which is constant. Therefore, since $X$ is noetherian and reduced and $\E'$ is coherent we find that $\E'$ is locally free by II.5.8 (c).

\subsubsection{8.3}

\begin{enumerate}
\item Let $X$ and $Y$ be schemes over a base $S$. Consider the cartesian diagram,
\begin{center}
\begin{tikzcd}
X \times_S Y \arrow[r, "\pi_2"] \arrow[d,"\pi_1"'] & Y \arrow[d, "g"]
\\
X \arrow[r, "f"'] & S 
\end{tikzcd}
\end{center}
There are exact sequences,
\begin{center}
\begin{tikzcd}
\pi_1^* \Omega_{X/S} \arrow[r] & \Omega_{X \times_S Y/S} \arrow[r] & \Omega_{X \times_S Y / X} \arrow[r] & 0
\\
\pi_2^* \Omega_{Y/S} \arrow[r] & \Omega_{X \times_S Y/S} \arrow[r] & \Omega_{X \times_S Y / Y} \arrow[r] & 0
\end{tikzcd}
\end{center}
Furthermore, from the base change diagram and (8.10),
\[ \Omega_{X \times_S Y/X} = \pi_2^* \Omega_{Y/S} \text{ and likewise } \Omega_{X \times_S Y/Y} = \pi_1^* \Omega_{Y/S} \]
The natural map $q : \Omega_{X \times_S Y/S} \to \Omega_{X \times_S Y/Y} = \pi_1^* \Omega_{Y/S}$ satisfies $q \circ \pi_1^* = \id$ because locally these maps are $\Omega_{A \otimes_R B/R} \to \Omega_{A \otimes_R B/B} = \Omega_{A/R} \otimes_R B$ and $\Omega_{A/R} \otimes_R B \to \Omega_{A \otimes_R B/R}$.
Thus we get a split exact sequence (exactness on the left follows from $\pi_1^*$ having a left inverse),
\begin{center}
\begin{tikzcd}
0 \arrow[r] & \pi_1^* \Omega_{X/S} \arrow[r] & \Omega_{X \times_S Y/S} \arrow[l, bend right, "q"'] \arrow[r] & \pi_2^* \Omega_{Y/S} \arrow[r] & 0
\end{tikzcd}
\end{center} 
and therefore $\Omega_{X \times_S Y/S} = \pi_1^* \Omega_{X/S} \oplus \pi_2^* \Omega_{Y/S}$.

\item If $X$ and $Y$ are nonsingular varieties over $k$ then $\Omega_{X} = \Omega_{X/k}$ and $\Omega_{Y} = \Omega_{Y/k}$ are finite locally free and $X \times Y$ is nonsingular over $k$ so $\Omega_{X \times Y} = \Omega_{X \times Y / k}$ is also locally free. By (a) there is a split exact sequence,
\begin{center}
\begin{tikzcd}
0 \arrow[r] & \pi_1^* \Omega_X \arrow[r] & \Omega_{X \times Y} \arrow[r] & \pi_2^* \Omega_Y \arrow[r] & 0
\end{tikzcd}
\end{center}
Since these are vector bundles this sequence gives an isomorphism of top exterior powers,
\[ \det{\Omega_{X \times Y}} = \det{\pi_1^* \Omega_X} \otimes_{\struct{X}} \det{\pi_2^* \Omega_Y} = \pi_1^* \det{\Omega_X} \otimes_{\struct{X}} \pi_2^* \det{\Omega_Y} \]
However, since these varieties are nonsingular, the canonical sheaf is $\omega_X = \det{\Omega_X}$ so we get,
\[ \omega_{X \times Y} = \pi_1^* \omega_X \otimes_{\struct{X}} \pi_2^* \omega_Y \]

\item Let $Y$ be a nonsingular plane cubic curve, and let $X = Y \times Y$. Since $Y \embed \P^2_k$ as a hypersurface, we know $\omega_Y = \struct{Y}(d - n + 1) = \struct{Y}$ because $d = 3$ and $n = 2$.  Now $\omega_X = \pi_1^* \omega_Y \otimes_{\struct{X}} \pi_1^* \omega_Y = \struct{X}$. Since $X$ is proper, $H^0(X, \omega_X) = H^0(X, \struct{X}) = k$ and thus $p_g(X) = 1$. However, the arithmetic genus is,
\[ p_a(X) = (-1)^{\dim{X}} (\chi(\struct{X}) - 1) \]
Therefore, in our case,
\[ p_a(X) = \dim_k H^2(X, \struct{X}) - \dim_k H^1(X, \struct{X}) \]
By K\"{u}nneth,
\begin{align*}
H^1(X, \struct{X}) & = H^1(Y, \struct{Y}) \otimes_k H^0(Y, \struct{Y}) \oplus H^0(Y, \struct{Y}) \otimes_k H^1(Y, \struct{Y}) = k \oplus k 
\\
H^2(X, \struct{X}) & = H^1(Y, \struct{Y}) \otimes_k H^1(Y, \struct{Y}) = k 
\end{align*}
Therefore, $p_a(X) \neq p_g(X)$.
\end{enumerate}

\subsubsection{8.4 DO!!}

A closed subscheme $U \subset \P^n_k$ is called a (global) complete intersection if the homogeneous ideal $I \subset S = k[x_0, \dots, x_n]$ can be generated by $r = \codim{Y}{\P^n}$ elements.

\begin{enumerate}
\item Let $\codim{Y}{\P^n} = r$. If $Y$ is a complete intersection then let $I = (f_1, \dots, d_r)$. Therefore, 
\[ \I_Y = \wt{I} = \wt{(f_1)} + \cdots + \wt{(f_r)} \]
and clearly $\wt{(f_i)} = \I_{H_i}$ where $H_i = V(f_i)$ is a hypersurface.
\bigskip\\
Conversely, suppose that $Y$ is cut out by the sheaf of ideals,
\[ \I_Y = \I_{H_1} + \cdots + \I_{H_r} \]
where $H_i$ is a hypersurface. Since $H_i$ is a prime Cartier divisor and $\Pic{\P^n_k} = \Z$ we must have $\struct{\P^n_k}(H_i) = \struct{\P^n_k}(n)$ and $H_i = V(f_i)$ where $f_i \in H^0(\P^n_k, \struct{\P^n_k}(n))$ is an irreducible homogeneous polynomial of degree $n$. It suffices to show that $I(Y) = J$ where $J = (f_1, \dots, f_r)$. First, we show that $J$ cuts out the subscheme $Y$. It suffices to show that $\I_Y = \wt{J}$ and that $J$ is saturated. First, 
\[ \I_Y(D_+(x_i)) = \I_{H_1}(D_+(x_i)) + \cdots + \I_{H_r}(D_+(x_i)) = (f_1/x_i^{\deg{f_1}}, \dots, f_r/x_i^{\deg{f_r}}) = J_{(x_i)} = \wt{J}(D_+(x_i)) \]
so $\I_Y = \wt{J}$. Thus $\height{J} = r$ so $f_1, \dots, f_r$ form a regular sequence and thus $S/(f_1, \dots, f_r)$ is Cohen-Macaulay of dimension $n + 1 - r$. I claim $I/J \subset S$ is finite length (FINISH!!)

\item Let $Y$ be a complete intersection of dimension $r \ge 1$ in $\P^n$ and $Y$ is normal. Consider $S(Y)$ and $C(Y) = \Spec{S(Y)}$ the cone over $Y$. Then $C(Y)$ is a locally complete intersection in $\A^{n+1}$ and regular in codimension one (WHY?). Therefore, by Prop. 2.23 $C(Y)$ is normal so $S(Y)$ is an integrally closed domain and thus $Y$ is projectively normal.

\item 

\item Proceed by induction on $r$. For $r = 1$ a general hypersurface $H \subset \P^n$ is irreducible and nonsingular of codimension $1$. Now suppose that (DO!!)

\item Let $Y \subset \P^n_k$ be a nonsingular complete intersection. Consider the surjection, 
\[ \I_{H_1} \oplus \cdots \oplus \I_{H_r} \onto \I_Y \]
where $Y = H_1 \cap \cdots \cap H_r$ and $H_i = V(f_i)$ and $\struct{\P^n_k}(H_i) = \struct{\P^n_k}(d_i)$ with $\struct{\P^n_k}(-d_r) \to \I_Y$ via multiplication by $f_i$ giving a surjection,
\[ \bigoplus_{i = 1}^r \struct{\P^n_k}(-d_i) \onto \I_Y \]
Now the conormal sheaf is,
\[ \C_{Y/\P^n_k} = \I_Y / \I_Y^2 = \iota^* \I_Y \]
Consider the pullback,
\[ \bigoplus_{i = 1}^r \struct{Y}(-d_i) \onto \C_{Y/\P^n_k} \]
Furthermore, since $Y$ is nonsingular, at each $y \in Y$ we know $\stalk{Y}{y} = \stalk{\P^n_k}{y} / I$ where $I = (f_1, \dots, f_r)$ and since $\codim{Y}{\P^n_k} = r$ this is a regular sequence so the map $\stalk{Y}{y}^{\oplus n} \onto I/I^2$ is an isomorphism. Thus the map,
\[ \bigoplus_{i = 1}^r \struct{Y}(-d_i) \iso \C_{Y/\P^n_k} \]
is an isomorphism. From the exact sequence,
\begin{center}
\begin{tikzcd}
0 \arrow[r] & \C_{Y/\P^n_k} \arrow[r] & \iota^* \Omega_{\P^n_k} \arrow[r] & \Omega_Y \arrow[r] & 0
\end{tikzcd}
\end{center}
which are locally free we get,
\begin{align*}
\omega_Y \cong \iota^* \omega_{\P^n_k} \otimes \left( \bigwedge^r \C_{Y/\P^n_k} \right)^\vee & = \struct{Y}(-n-1) \otimes \left( \bigwedge^r \bigoplus_{i = 1}^r \struct{Y}(-d_i) \right)
\\
& = \struct{Y}(-n-1) \otimes \struct{Y}(\sum d_i) = \struct{Y}(\sum d_i - n - 1)
\end{align*}

\item Let $Y$ be a nonsingular hypersurface of degree $d$ in $\P^n$. Then,
\[ P_g(Y) = \dim_k \Gamma(Y, \omega_Y) = \dim_k \Gamma(Y, \struct{Y}(d - n - 1)) \]
Furthermore, $Y$ is normal so,
\[ \Gamma(Y, \struct{\P^n_k}(\ell) \onto \Gamma(Y, \struct{Y}(\ell)) \]
is surjective for all $\ell \ge 0$ and the kernel is $\Gamma(\P^n_k, \I_Y(\ell))$. Because $\struct{Y}(-\ell)$ is anti-very ample it cannot have any sections (else $\L$ and $\L^{\otimes -1}$ both have nonzero sections) so $\Gamma(Y, \struct{Y}(-\ell)) = 0$. Thus we may assume that $d - n - 1 \ge 0$. Then we have an exact sequence,
\begin{center}
\begin{tikzcd}
0 \arrow[r] & \Gamma(\P^n_k, \struct{\P^n_k}(-n-1)) \arrow[r] & \Gamma(\P^n_k, \struct{\P^n_k}(d - n - 1)) \arrow[r] & \Gamma(Y, \struct{Y}(d - n - 1)) \arrow[r] & 0
\end{tikzcd}
\end{center}
Therefore,
\[ p_g(Y) = \dim_k \Gamma(Y, \struct{Y}(d - n - 1)) = \dim_k \Gamma(\P^n_k, \struct{\P^n_k}(d - n - 1)) - \dim_k \Gamma(\P^n_k, \struct{\P^n_k}(- n - 1)) = { d - 1 \choose n }  \]
\item Let $Y$ be a nonsingular curve in $\P^3$ which is a complete intersection of nonsingular surfaces of degrees $d,e$. By (c) there is an exact sequence,
\begin{center}
\begin{tikzcd}
0 \arrow[r] & \Gamma(\P^n_k, \I_Y(\ell)) \arrow[r] & \Gamma(\P^n_k, \struct{\P^n_k}(\ell)) \arrow[r] & \Gamma(Y, \struct{Y}(\ell)) \arrow[r] & 0
\end{tikzcd}
\end{center}
Furthermore, there is an exact sequence,
\begin{center}
\begin{tikzcd}
0 \arrow[r] & \struct{\P^n_k}(-d - e + \ell) \arrow[r] & \struct{\P^n_k}(-d + \ell) \oplus \struct{\P^n_k}(-e + \ell) \arrow[r] & \I_Y \arrow[r] & 0
\end{tikzcd}
\end{center}
Since $n > 1$ this sequence is exact on global sections (vanishing of $H^1$ on line bundles) so we find,
\[ \dim_k \Gamma(\P^n_k, \I_Y(\ell)) = { \ell - d + n \choose n } + { \ell - e + n \choose n } - { \ell - d - e + n \choose n } \] Therefore, we get an exact sequence,
\begin{center}
\begin{tikzcd}[column sep = small]
0 \arrow[r] & \Gamma(\struct{\P^n_k}(\ell - d - e)) \arrow[r] & \Gamma(\struct{\P^n_k}(\ell - d) \oplus \struct{\P^n_k}(\ell - e)) \arrow[r] & \Gamma(\struct{\P^n_k}(\ell)) \arrow[r] & \Gamma(Y, \struct{Y}(\ell)) \arrow[r] & 0
\end{tikzcd}
\end{center}
Now take $\ell = d + e - n - 1$,
\begin{align*}
p_g(Y) & = \dim_k \Gamma(Y, \omega_Y) = \dim_k \Gamma(Y, \struct{Y}(d + e - n - 1)) 
\\
& = {d + e - 1 \choose n } - { e - 1 \choose n } - { d - 1 \choose n } 
\end{align*} 
Since $n = 3$ we get,
\[ p_g(Y) = \tfrac{1}{2} de (e + d - 4) + 1 \]
\end{enumerate}

\begin{rmk}
Notice that the following does not necessarily hold,
\[ \Gamma_*(\I_Y) \neq \Gamma_*(\I_{H_1}) + \cdots + \Gamma_*(\I_{H_1}) \]
so the reverse of (a) is not immediate. This is because $\I_Y = \I_{H_1} + \cdots + \I_{H_r}$ does not imply that,
\[ H^0(\P^n_k, \I_Y) = H^0(\P^n_k,\I_{H_1}) + \cdots + H^0(\P^n_k,\I_{H_r}) \]
For example, let $\I_1 = \wt{(xy)}$ and $\I_2 = \wt{(y^2)}$ on $\P^1_k = \Proj{k[x,y]}$. Then, I claim $\I_1 + \I_2 = \I = \wt{(y)}$ because on $D(x)$ we have $(y/x) + (y^2/x^2) = (y/x)$ and on $D(y)$ we have $(x/y) + (1) = (1)$. However,
\[ \Gamma_*(\I) = (y) \quad \text{and} \quad \Gamma_*(\I_1) + \Gamma_*(\I_2) = (xy) + (y^2) = (xy, y^2) \neq (y) \]
However, the saturation of these ideals are equal.
\end{rmk}

\subsubsection{8.5 DO!!}

\renewcommand{\Pic}[1]{\mathrm{Pic}(#1)}

Let $X$ be a nonsingular variety and $Y \subset X$ a nonsingular subvariety of codimension $r \ge 2$. Let $\pi : \tilde{X} \to X$ be the blowing up of $X$ along $Y$ and $E = \pi^{-1}(Y)$ the exceptional divisor.

\begin{enumerate}
\item Consider the diagram,
\begin{center}
\begin{tikzcd}
\Z \arrow[r] & \Pic{\tilde{X}} \arrow[r] & \Pic{\tilde{X} \setminus E} \arrow[r] & 0
\\
& \Pic{X} \arrow[u, "\pi^*"] \arrow[r, "\sim"] & \Pic{X \setminus Y} \arrow[u, "\sim"']
\end{tikzcd}
\end{center}
where $\Z \to \Pic{\tilde{X}}$ sends $n \mapsto n [E]$ where the exceptional divisor $E$ is Cartier. 
Because $\pi : \tilde{X} \setminus E \to X \setminus Y$ is an isomorphism, the map $\Pic{X \setminus Y} \iso \Pic{\tilde{X} \setminus E}$ is also an isomorphism. Therefore, the sequence is right split. Finally, $E$ corresponds to $\struct{\tilde{X}}(1)$ and under $\P(\C_{Y/X}) \iso E \to \tilde{X}$ this pulls back to $\struct{\P}(1)$ and therefore $\struct{\tilde{X}}(n)$ is nontrivial for all $n \neq 0$ so the map $\Z \to \Pic{\tilde{X}}$ is injective. Therefore, we get a right split exact sequence,
\begin{center}
\begin{tikzcd}
0 \arrow[r] & \Z \arrow[r] & \Pic{\tilde{X}} \arrow[r] & \Pic{X} \arrow[r] & 0
\end{tikzcd}
\end{center}
and thus $\Pic{\tilde{X}} \cong \Pic{X} \oplus \Z$.

\item Since $\Pic{\tilde{X}} = \Pic{X} \oplus \Z$ we know that $\omega_{\tilde{X}} = f^* \L \oplus \struct{\tilde{X}}(q E)$ for some integer $q \in \Z$. Furthermore, $\pi : \tilde{X} \setminus E \to X \setminus Y$ is an isomorphism so $\M = \omega_X$ because they agree on $X \setminus Y$ which and $\codim{Y}{X} \ge 2$. Furthermore, $\P_Y(\C_{Y/X}) \iso E \subset \tilde{X}$ and $E$ is Cartier with $\struct{\tilde{X}}(E) = \struct{\tilde{X}}(1)$.

\item Consider $X = \Spec{k[x,y,z]/(z^2 - xy)}$ and $I = (x,y,z)$. Then consider the graded ring,
\[ R = \bigoplus_{i = 0}^\infty I^i \]
Then,
\[ R = A[t, u, v]/(x u - y t, y v - z u, x v - z t, v^2 - tu) \]
Now consider $\tilde{X} = \Proj{R} \to X$. Consider the open patches,
\begin{align*}
D_+(t) & = \Spec{k[x, v]} 
\\
D_+(u) & = \Spec{k[y, v]} 
\\
D_+(v) & = \Spec{k[z, t, u]/(tu - 1)}
\end{align*}
Therefore $\tilde{X}$ is smooth. Likewise, consider the close subschemes,
\begin{align*}
V(t) & = \Proj{k[x, y, z, u, v]/(xu, yv - zu, xv, v^2)} = \Proj{k[y, z, u, v]/(yv - zu, v^2)}
\\
D_+(u) & = \Spec{k[y, v]} 
\\
D_+(v) & = \Spec{k[z, t, u]/(tu - 1)}
\end{align*}
\end{enumerate}

\subsubsection{8.6 DO!!}

\subsubsection{8.7 DO!!}

\subsubsection{8.8 DO!!}

(DO IN NOTES)

\subsection{9}

\subsubsection{9.1 CHECK!}

Let $X$ be a noetherian scheme, $Y \subset X$ a closed subscheme, and $\hat{X}$ the completion of $X$ along $Y$. Call $\Gamma(\hat{X}, \struct{\hat{X}})$ the ring of formal regular functions.
\bigskip\\
Let $Y$ be a connected, nonsingular, positive-dimensional subvariety of $X = \P^n_k$ and $k$ algebraically closed.

\begin{enumerate}
\item Let $\I$ be the sheaf of ideals of $Y$. since $Y$ is nonsingular there is an exact sequence,
\begin{center}
\begin{tikzcd}
0 \arrow[r] & \I/\I^2 \arrow[r] & \Omega_X |_Y \arrow[r] & \Omega_Y \arrow[r] & 0
\end{tikzcd}
\end{center}
Futhermore, there is an exact sequence,
\begin{center}
\begin{tikzcd}
0 \arrow[r] & \Omega_X \arrow[r] & \struct{X}(-1)^{\oplus n + 1} \arrow[r] & \struct{X} \arrow[r] & 0
\end{tikzcd}
\end{center}
Pulling this sequence back to $Y$, the sequence remains exact because $\struct{X}$ is locally free so we get an injection $\Omega_X |_Y \embed \struct{Y}(-1)^{\oplus n + 1}$. Putting these together gives an injection,
\[ \I/\I^2 \embed \struct{Y}(-1)^{\oplus n + 1} \]

\item Because $Y$ is nonsingular, $\I/\I^2$ is locally free and thus $\I^r / \I^{r+1} = \Sym{\struct{Y}}{\I/\I^2}$. First, $\I / \I^2 \embed \struct{Y}(-1)^{\oplus n + 1}$ and therefore,
\[ \Gamma(Y, \I/\I^2) \embed \Gamma(Y, \struct{Y}(-1)^{\oplus n + 1}) \]
Therefore we need to show that $\Gamma(Y, \struct{Y}(-1)^{\oplus n + 1}) = 0$. This follows from the fact that $Y$ is positive dimensional (SHOW THIS!!). 

\item Consider the exact sequence,
\begin{center}
\begin{tikzcd}
0 \arrow[r] & \I^r / \I^{r+1} \arrow[r] & \struct{X} / \I^{r+1} \arrow[r] & \struct{X} / \I^r \arrow[r] & 0
\end{tikzcd}
\end{center}
Taking global sections,
\begin{center}
\begin{tikzcd}
0 \arrow[r] & \Gamma(Y, \I^r / \I^{r+1}) \arrow[r] & \Gamma(Y, \struct{X}/\I^{r+1}) \arrow[r] & \Gamma(Y, \struct{X}/\I^r) 
\end{tikzcd}
\end{center}
but $\Gamma(Y, \I^r / \I^{r+1}) = 0$ and therefore $\Gamma(Y, \struct{X}/\I^{r+1}) \embed \Gamma(Y, \struct{X} / \I^{r})$. For induction we assume that $\Gamma(Y, \struct{X} / \I^{r}) = k$ and therefore the $k$-algebra $\Gamma(Y, \struct{X}/\I^{r+1})$ is either $0$ or $k$ but $1$ is a global section so we find $\Gamma(Y, \struct{X}/\I^{r+1}) = k$. Finally, the base case 
\[ \Gamma(Y, \struct{X}/\I) = \Gamma(Y, \struct{Y}) = k \]
follows because $Y$ is integral and proper over $Y$ which is algebraically closed.

\item We know,
\[ \Gamma(\hat{X}, \struct{\hat{X}}) = \varprojlim_n \Gamma(Y, \struct{X} / \I^n) = k \]
\end{enumerate}

\subsubsection{9.2 DO!!}

Let $Y \subset X = \P^n$ be a connected nonsingular positive dimensional closed subvariety. Suppose that $f : X \to Z$ is a map of $k$-varieties (where $k$ is algebraically closed) such that $f(Y) = P$ for a closed point $P \in Z$.
\bigskip\\
We may replace $Z$ by the scheme theoretic image of $f$ such that $\struct{Z} \to f_* \struct{X}$ is injective. Let $\m_P$ be the sheaf of ideals corresponding to $P \in Z$. Then $\m_P$ must map inside $f_* \I$ because $f(Y) = P$ and $Y$ is reduced. Therefore, we get a diagram,
\begin{center}
\begin{tikzcd}
\struct{Z} \arrow[d] \arrow[r] & f_* \struct{X} \arrow[d]
\\
\struct{Z}/ \m_p^n \arrow[r] & f_* (\struct{X} / \I^n)
\end{tikzcd}
\end{center} 
We get an injection (WHY??) $\struct{\hat{Z}} \embed f_* \struct{\hat{X}}$ and therefore $\Gamma(\hat{Z}, \struct{\hat{Z}}) \embed \Gamma(\hat{X}, \struct{\hat{X}})$. However, $\Gamma(\hat{Z}, \struct{\hat{Z}}) = \hat{\stalk{Z}{z}}$ and $\Gamma(\hat{X}, \struct{\hat{X}}) = k$ by (Ex. 9.1). Therefore, we must have $\hat{\stalk{Z}{z}} = k$ meaning that $\dim{Z} = 0$ so $Z$ is a single point because $X$ is connected.
\bigskip\\
Here is an alternative proof. Consider the map of locally ringed spaces $\hat{X} \to X \to Z$. For any affine open $V \subset Z$ containing $P$ we know $Y \subset f^{-1}(Z)$ and therefore $\hat{X} \to Z$ factors through $\hat{X} \to f^{-1}(Z) \to V \subset Z$. Furthermore, since $V$ is affine, the map $\hat{X} \to V$ is determined by $\Gamma(V, \struct{V}) \to \Gamma(\hat{X}, \struct{\hat{X}})$ but $\Gamma(\hat{X}, \struct{\hat{X}}) = k$ so this map is constant. Now if $U$ is an affine open of $f^{-1}(V)$ intersecting $Y$ then $\hat{U} \to \hat{X} \to V \to Z$ is constant. Therefore, consider $\hat{U} \to V$ where $U = \Spec{B}$ and $V = \Spec{A}$. We know that $A \to B \to \hat{B}$ factors through $k$ but $B \to \hat{B}$ is injective because $B$ is a domain ($X$ is integral) so $A \to B$ factors through $k$ and thus $U \to V$ is constant. Since $U \subset X$ is dense we see that $f : X \to Z$ is constant.

\subsubsection{9.3 DO!!}

\subsubsection{9.4 DO!!}

\subsubsection{9.5 DO!!}

\subsubsection{9.6 DO!!}


\section{III Cohomology}

\subsection{1}


\subsection{2}

\subsubsection{2.3}

Let $X$ be a topological space and $Y \subset X$ a closed subset. Let $\F$ be an abelian sheaf on $X$. Let $U = X \setminus Y$ and $j : U \to X$ be the inclusion.

\begin{enumerate}
\item Consider an exact sequence of sheaves on $X$,
\begin{center}
\begin{tikzcd}
0 \arrow[r] & \F \arrow[r] & \G \arrow[r] & \H \arrow[r] & 0
\end{tikzcd}
\end{center}
Then since the functor $(-)|_U$ is exact and $j_*$ is right-exact then we get a commutative diagram with exact rows,
\begin{center}
\begin{tikzcd}
0 \arrow[r] & \F \arrow[d] \arrow[r] & \G \arrow[r] \arrow[d] & \K \arrow[r] \arrow[d] & 0
\\
0 \arrow[r] & j_*(\F |_U) \arrow[r] & j_*(\G |_U) \arrow[r] & j_*(\K |_U)
\end{tikzcd}
\end{center}
Since taking kernels is left-exact (limits are right adjoints) we get an exact sequence,
\begin{center}
\begin{tikzcd}
0 \arrow[r] & \H^0_Y(\F) \arrow[r] & \H^0_Y(\G) \arrow[r] & \H^0_Y(\K)
\end{tikzcd}
\end{center}
Applying the left-exact functor $\Gamma(X, -)$ gives an exact sequence,
\begin{center}
\begin{tikzcd}
0 \arrow[r] & \Gamma_Y(X, \F) \arrow[r] & \Gamma_Y(X, \G) \arrow[r] & \Gamma_Y(X, \K)
\end{tikzcd}
\end{center}
Since $\Gamma(X, \H^0_Y(\F)) = \Gamma_Y(X, \F)$ by definition. 
\bigskip\\
We define the sheaf cohomology with supports in $Y$ to be the right-derived functors $H^n_Y(X, -) = R^n \Gamma_Y(X, -)$ of the left-exact functor $\Gamma_Y(X, -)$. 

\item Consider an exact sequence of sheaves,
\begin{center}
\begin{tikzcd}
0 \arrow[r] & \F \arrow[r] & \G \arrow[r] & \K \arrow[r] & 0
\end{tikzcd}
\end{center}
where the sheaf $\F$ is flasque. Then consider the diagram, with exact rows and columns,
\begin{center}
\begin{tikzcd}
& 0 \arrow[d] & 0 \arrow[d] & 0 \arrow[d]
\\
0 \arrow[r] & \H^0_Y(\F) \arrow[d] \arrow[r] & \H^0_Y(\G) \arrow[d] \arrow[r] & \H^0_Y(\K) \arrow[d]
\\
0 \arrow[r] & \F \arrow[d] \arrow[r] & \G \arrow[d] \arrow[r] & \K \arrow[d] \arrow[r] & 0
\\
0 \arrow[r] & j_*(\F |_U) \arrow[d] \arrow[r] & j_* (\G |_U) \arrow[r] & j_*(\K |_U) 
\\
& 0
\end{tikzcd}
\end{center}
where $\H^0_Z(\F)$ is the kernel of $\F \to j_*(\F |_U)$ and when $\F$ is flasque then we have the exact sequence,
\begin{center}
\begin{tikzcd}
0 \arrow[r] & \H^0_Z(\F) \arrow[r] & \F \arrow[r] & j_*(\F |_U) \arrow[r] & 0
\end{tikzcd}
\end{center}
Furthermore, the maps $\H_Y^0(\F) \to \H^0_Y(\G)$ and $j_*(\F |_U) \to j_*(\F |_U)$ are injective becase these functors are left-exact (taking kernels is left-exact and $j_*$ is left-exact, recall that $(-)|_U = j^{-1}$ is exact). Now apply the left-exact functor $\Gamma(X, -)$ to find a diagram with exact rows and columns,
\begin{center}
\begin{tikzcd}
& 0 \arrow[d] & 0 \arrow[d] & 0 \arrow[d]
\\
0 \arrow[r] & \Gamma_Y(X, \F) \arrow[d] \arrow[r] & \Gamma_Y(X, \G) \arrow[d] \arrow[r] & \Gamma_Y(X, \K) \arrow[d]
\\
0 \arrow[r] & \Gamma(X, \F) \arrow[d] \arrow[r] & \Gamma(X, \G) \arrow[d] \arrow[r] & \Gamma(X, \K) \arrow[d] \arrow[r] & 0
\\
0 \arrow[r] & \Gamma(U, \F |_U) \arrow[d] \arrow[r] & \Gamma(U, \G |_U) \arrow[r] & \Gamma(U, \K |_U) 
\\
& 0
\end{tikzcd}
\end{center}
Where $\Gamma(X, \F) \to \Gamma(U, \F |_U)$ remains surjective because $\F$ is a flasque sheaf so restriction is surjective. Furthermore, the sequence,
\begin{center}
\begin{tikzcd}
0 \arrow[r] & \Gamma(X, \F) \arrow[r] & \Gamma(X, \G) \arrow[r] & \Gamma(X, \K) \arrow[r] & 0
\end{tikzcd}
\end{center} 
remains exact because $\F$ is a flasque sheaf so $\Gamma(X, -)$ preserves the exact sequence since $H^1(X, \F) = 0$. Now, applying the snake lemma gives an exact sequence of the kernels to cokernels,
\begin{center}
\begin{tikzcd}
0 \arrow[r] & \Gamma_Y(X, \F)  \arrow[r] & \Gamma_Y(X, \G)  \arrow[r] & \Gamma_Y(X, \K) \arrow[r] & 0
\end{tikzcd}
\end{center}


\item Let $\F$ be a flasque sheaf. Now embedd $\F$ into an injective sheaf $\I$ to form an exact sequence,
\begin{center}
\begin{tikzcd}
0 \arrow[r] & \F \arrow[r] & \I \arrow[r] & \K \arrow[r] & 0 
\end{tikzcd}
\end{center}
Since $\F$ and $\I$ are flasque then $\K$ is also flasque.
Taking the long exact sequence of derived functors we find,
\begin{center}
\begin{tikzcd}[column sep = small]
0 \arrow[r] & H^0_Y(X, \F) \arrow[draw=none]{d}[name=Z, shape=coordinate]{} \arrow[r] & H^0_Y(X, \I) \arrow[r] & H^0_Y(X, \K) \arrow[r] & H^1_Y(X, \F) 
\arrow[dlll,
rounded corners, crossing over,
to path={ -- ([xshift=2ex]\tikztostart.east)
|- (Z) [near end]\tikztonodes
-| ([xshift=-2ex]\tikztotarget.west)
-- (\tikztotarget)}]
\\ 
& H^1_Y(X, \I) \arrow[r] & H^1_Y(X, \K) \arrow[r] & H^2_Y(X, \F) \arrow[r] & H^2_Y(X, \I) \arrow[r] & \cdots
\end{tikzcd}
\end{center}
Since $\I$ is an injective sheaf $H^n_Y(X, \I) = 0$ for $n > 0$ so this long exact sequence gives exact sequences,
\begin{center}
\begin{tikzcd}
0 \arrow[r] & H^0_Y(X, \F) \arrow[r] & H^0_Y(X, \I) \arrow[r] & H^0_Y(X, \K) \arrow[r] & H^1_Y(X, \F) \arrow[r] & 0
\end{tikzcd}
\end{center}
and,
\begin{center}
\begin{tikzcd}
0 \arrow[r] & H^n_Y(X, \K) \arrow[r] & H^{n+1}_Y(X, \F) \arrow[r] & 0
\end{tikzcd}
\end{center}
for $n \ge 1$. However, we have shown that the sequence,
\begin{center}
\begin{tikzcd}
0 \arrow[r] & \Gamma_Y(X, \F) \arrow[r] & \Gamma_Y(X, \I) \arrow[r] & \Gamma_Y(X, \K) \arrow[r] & 0
\end{tikzcd}
\end{center}
is exact when $\F$ is flasque and since $H_Y^1(X, \F)$ is the cokernel of this last map we have $H^1_Y(X, \F) = 0$ for any flasque sheaf. Now performing induction we find that $H^n_Y(X, \K) = 0$ for any flasque sheaf $\K$ implies $H^{n+1}_Y(X, \F) = 0$ via the second exact sequence. This proves that $H^n_Y(X, \F) = 0$ for all $n > 0$. 

\item If $\F$ is flasque then we have shown that the sequence,
\begin{center}
\begin{tikzcd}
0 \arrow[r] & \H^0_Y(\F) \arrow[r] & \F \arrow[r] & j_* (\F |_U) \arrow[r] & 0
\end{tikzcd}
\end{center}
is exact. Applying the functor $\Gamma(X, -)$ we get an exact sequence,
\begin{center}
\begin{tikzcd}
0 \arrow[r] & \Gamma_Y(X, \F) \arrow[r] & \Gamma(X, \F) \arrow[r] & \Gamma(U, \F) \arrow[r] & 0 
\end{tikzcd}
\end{center}
where the last map is restruction $\res_{U,X} : \F(X) \to \F(U)$ which is surjective by flasqueness. We have used $\Gamma(X, j_* (\F |_U)) = \Gamma(U, \F|_U) = \Gamma(U, \F)$. 


\item The above exact sequence for flasque sheaves implies that the left-exact functors $\Gamma_Y(X, -)$ and $\Gamma(X, -)$ and $\Gamma(U, (-)|_U)$ satisfy the hypothesis of Lemma \ref{lem:exact_seq_of_functors} giving the required long-exact sequence of cohomology,
\begin{center}
\begin{tikzcd}
0 \arrow[r] & H^0_{Y}(X, \F) \arrow[r] & H^0(X, \F)  \arrow[draw=none]{d}[name=Z, shape=coordinate]{} \arrow[r] & H^0(U, \F |_U)
\arrow[dll,
rounded corners, crossing over,
to path={ -- ([xshift=2ex]\tikztostart.east)
|- (Z) [near end]\tikztonodes
-| ([xshift=-2ex]\tikztotarget.west)
-- (\tikztotarget)}]
\\ 
& H^1_{Y}(X, \F) \arrow[r] & H^1(X, \F)  \arrow[draw=none]{d}[name=Z', shape=coordinate]{} \arrow[r] & H^1(U, \F |_U) \arrow[dll,
rounded corners, crossing over,
to path={ -- ([xshift=2ex]\tikztostart.east)
|- (Z') [near end]\tikztonodes
-| ([xshift=-2ex]\tikztotarget.west)
-- (\tikztotarget)}]
\\
& H^2_Y(X, \F) \arrow[r] & H^2(X, \F) \arrow[r] & H^2(U, \F |_U) \arrow[r] & \cdots
\end{tikzcd}
\end{center}

\item Let $Z \subset X$ be closed and $V \subset X$ be an open set such that $Z \subset V$ and let $\F$ be a sheaf on $X$. Then consider the restriction map $\res_{V, X} : \Gamma_Z(X, \F) \to \Gamma_Z(V, \F)$. Note that $V$ and $U = X \setminus Z$ form an open cover of $X$. If $s \mapsto 0$ then $s|_V = 0$ but also $s|_{X \setminus Z} = 0$ since $\Supp{\F}{s} \subset Z$ and thus $s_x = 0$ for each $x \in X \setminus Z$. Therefore, by the sheaf property of $\F$ we have $s = 0$ so $\res_{U,X}$ is injective. Furthermore, consider a section $s \in \Gamma_Z(V, \F)$. Since $\Supp{\F}{s} \subset Z$ we know that $s |_{V \cap U} = 0$ because $(V \cap U) \cap Z = \varnothing$. Therefore, $s$ and $0 \in \Gamma(U, \F)$ agree on the overlap and thus glue to a global section $s' \in \Gamma(X, \F)$ such that $s'|_V = s$. Furthermore, $s' |_U = 0$ and thus $\Supp{\F}{s'} \subset Z$ so the map $\res_{V, X} : \Gamma_Z(X, \F) \to \Gamma_Z(V, \F)$ is surjective and thus an isomorphism. Therefore, there is a natural isomorphism $\Gamma_Z(X, -) \cong \Gamma_Z(V, (-)|_V)$. Therefore, these functors give rise to the same derived functors so,
\[ H^p_Z(X, \F) \cong H^p_Z(V, \F|_V) \] 

\end{enumerate}

\subsubsection{2.4}

Let $Z_1, Z_2 \subset X$ be closed subsets. Let $\F$ be a flasque sheaf and consider the diagram,
\begin{center}
\begin{tikzcd}[column sep = small]
& 0 \arrow[d] & 0 \arrow[d] & 0 \arrow[d] &
\\
0 \arrow[r] & \Gamma_{Z_1 \cap Z_2}(X, \F) \arrow[r] \arrow[d] & \Gamma_{Z_1}(X, \F) \oplus \Gamma_{Z_2}(X, \F) \arrow[r] \arrow[d] & \Gamma_{Z_1 \cup Z_2}(X, \F) \arrow[r] \arrow[d] & 0
\\
0 \arrow[r] & \Gamma(X, \F) \arrow[r] \arrow[d] & \Gamma(X, \F) \oplus \Gamma(X, \F) \arrow[r] \arrow[d] & \Gamma(X, \F) \arrow[r] \arrow[d] & 0
\\
0 \arrow[r] & \Gamma(X \setminus (Z_1 \cap Z_2), \F) \arrow[r] \arrow[d] & \Gamma(X \setminus  Z_1, \F) \oplus \Gamma(X \setminus Z_2, \F) \arrow[r] \arrow[d] & \Gamma(X \setminus (Z_1 \cup Z_2), \F) \arrow[r] \arrow[d] & 0
\\
& 0 & 0 & 0 &
\end{tikzcd}
\end{center}
where the columns are the exact sequences of 2.3 (d), the last row is the exact sequence of Lemma \ref{lem:flasque_mayer_vietoris}, the middle row is the diagonal exact sequence associated to the direct sum ($s \mapsto (s, s)$ then $(s, t) \mapsto s - t$), and the top rwo is given first inclusion maps and second by the difference of the inclusion maps (including the group of sections with support in a smaller set into the group of section with support in a larger set). Since this diagram commutes, has exact columns, and the last two rows are exact, by the nine-lemma, the top row is exact as well. Therefore for any flasque sheaf, and in particular any injective sheaf, there is an exact sequence,
\begin{center}
\begin{tikzcd}
0 \arrow[r] & \Gamma_{Z_1 \cap Z_2}(X, \F) \arrow[r] & \Gamma_{Z_1}(X, \F) \oplus \Gamma_{Z_2}(X, \F) \arrow[r] & \Gamma_{Z_1 \cup Z_2}(X, \F) \arrow[r]  & 0
\end{tikzcd}
\end{center}
Therefore, the left-exact functors $\Gamma_{Z_1 \cap Z_2}(X, -)$ and $\Gamma_{Z_1}(X, -) \oplus \Gamma_{Z_2}(X, -)$ and $\Gamma(Z_1 \cup Z_2, -)$ satisfy the conditions of Lemma \ref{lem:exact_seq_of_functors} giving an exact sequence of their derived functors. Furthermore, because direct sum is exact it commutes with taking cohomology and thus direct sum commutes with taking derived functors. Thus Lemma \ref{lem:exact_seq_of_functors} gives the required long exact sequence, 
\begin{center}
\begin{tikzcd}
0 \arrow[r] & H^0_{Z_1 \cap Z_2}(X, \F) \arrow[r] & H^0_{Z_1}(X, \F) \oplus H^0_{Z_2}(X, \F) \arrow[draw=none]{d}[name=Z, shape=coordinate]{} \arrow[r] & H^0_{Z_1 \cup Z_2}(X, \F)
\arrow[dll,
rounded corners, crossing over,
to path={ -- ([xshift=2ex]\tikztostart.east)
|- (Z) [near end]\tikztonodes
-| ([xshift=-2ex]\tikztotarget.west)
-- (\tikztotarget)}]
\\ 
& H^1_{Z_1 \cap Z_2}(X, \F) \arrow[r] & H^1_{Z_1}(X, \F) \oplus H^1_{Z_2}(X, \F) \arrow[draw=none]{d}[name=Z', shape=coordinate]{} \arrow[r] & H^1_{Z_1 \cup Z_2}(X, \F) \arrow[dll,
rounded corners, crossing over,
to path={ -- ([xshift=2ex]\tikztostart.east)
|- (Z') [near end]\tikztonodes
-| ([xshift=-2ex]\tikztotarget.west)
-- (\tikztotarget)}]
\\
& H^2_{Z_1 \cap Z_2}(X, \F) \arrow[r] & H^2_{Z_1}(X, \F) \oplus H^2_{Z_2}(X, \F) \arrow[r] & H^2_{Z_1 \cup Z_2}(X, \F) \arrow[r] & \cdots
\end{tikzcd}
\end{center}

\subsubsection{2.5 DO!!}

\subsubsection{2.6 DO!!}

\subsubsection{2.7 DO!!}




\subsection{Section 4}

\subsubsection{4.1}


Let $f : X \to Y$ be an affine morphism of schemes and $\F$ be a quasi-coherent $\struct{X}$-module. We proved in class that $R^q f_* \F = 0$ for $q \ge 0$ when $f$ is affine and $\F$ quasi-coherent (note that this proof uses the vanishing of higher cohomology for quasi-coherent sheaves on affine schemes which is difficult to prove without the Noetherian assumption but still true). Consider the commutative diagram of functors,
\begin{center}
\begin{tikzcd}
\Ab(X) \arrow[dr, "\Gamma_X"'] \arrow[rr, "f_*"] & & \Ab(Y) \arrow[dl, "\Gamma_Y"]
\\
& \Ab
\end{tikzcd}
\end{center}
Take an injective resolution of sheaves over $X$,
\begin{center}
\begin{tikzcd}
0 \arrow[r] & \F \arrow[r] & \I^\bullet
\end{tikzcd}
\end{center}
Because $f_*$ is a right-adjoint to the exact functor $f^{-1}$ by Lemma \ref{right_adjoint_to_exact_preserves_injectives}, $f_*$ preserves injectives. I claim that,
\begin{center}
\begin{tikzcd}
0 \arrow[r] & f_* \F \arrow[r] & f_* \I^\bullet 
\end{tikzcd}
\end{center}
is an injective resolution of sheaves over $Y$.
To show exactness, split the long exact resolution into short exact sequences of sheaves,
\begin{center}
\begin{tikzcd}
0 \arrow[r] & \F \arrow[r] & \I^0 \arrow[r] & \K^0 \arrow[r] & 0 
\end{tikzcd}
\end{center}
\begin{center}
\begin{tikzcd}
0 \arrow[r] & \K^{p-1} \arrow[r] & \I^p \arrow[r] & \K^p \arrow[r] & 0 
\end{tikzcd}
\end{center}
Now applying the long exact sequences of cohomology from the derived functors of the left-exact functor $f_*$ we get,
\begin{center}
\begin{tikzcd}
0 \arrow[r] & f_* \F \arrow[r] & f_* \I^0 \arrow[r] & f_* \K^0 \arrow[r] & R^1 f_* \F 
\end{tikzcd}
\end{center}
but $R^1 f_* \F$ vanishes so the sequence remains short exact and 
\begin{center}
\begin{tikzcd}
R^q f_* \F \arrow[r] & R^q f_* \I^0 \arrow[r] & R^q f_* \K^0 \arrow[r] & R^{q + 1} f_* \F 
\end{tikzcd}
\end{center}
but $R^{q+1} f_* \F = 0$ and $R^q f_* \I^0 = 0$ because $\I^0$ is injective so we find $R^q f_* \K^0 = 0$ for all $q \ge 0$. Now assume for induction that $R^q f_* \K^{p - 1} = 0$ for all $q \ge 0$. The long exact sequence then gives,
\begin{center}
\begin{tikzcd}
0 \arrow[r] & f_* \K^{p-1} \arrow[r] & f_* \I^p \arrow[r] & f_* \K^p \arrow[r] & R^1 f_* \K^{p - 1}
\end{tikzcd}
\end{center}
by the induction hypothesis $R^1 f_* \K^{p - 1} = 0$ so the sequence remains short exact. Furthermore the long exact sequence gives,
\begin{center}
\begin{tikzcd}
R^q f_* \K^{p-1} \arrow[r] & R^q f_* \I^p \arrow[r] & f_* R^q \K^p \arrow[r] & R^{q+1} f_* \K^{p - 1}
\end{tikzcd}
\end{center}
but $R^{q+1} f_* \F = 0$ and $R^q f_* \I^p = 0$ because $\I^p$ is injective so we find that $R^q f_* \K^p = 0$ for all $q \ge 0$ so we may proceed by induction. Thus we have shown that $f_*$ preserves each short exact sequences which, laced together, shows that
\begin{center}
\begin{tikzcd}
0 \arrow[r] & f_* \F \arrow[r] & f_* \I^\bullet 
\end{tikzcd}
\end{center}
is exact and thus an injective resolution. Therefore, we may directly compute,
\[ H^q(Y, f_* \F) = H^q(\Gamma(Y, f_* \I^\bullet)) = H^q(\Gamma(X, \I^\bullet)) = H^q(X, \F) \]

\begin{remark}
What I have shown here is a special case of the convergence of the Grothendieck spectral sequence applied to the left-exact functors $\Gamma(Y, -)$ and $f_*$ where $f_*$ takes injectives to injectives. This spectral sequence is characterized by,
\[ E^{p, q}_2 = H^p(Y, R^q f_* \F) \implies H^{p+q}(X, \F) \] 
In the case of an affine morphism $f : X \to Y$ and quasi-coherent $\struct{X}$-module $\F$, we have $R^q f_* \F = 0$ and thus $E^{pq}_2$ collapses to $E_2^{p0} = H^p(Y, f_* \F)$ in which case we know that,
\[ H^p(X, \F) = E^{p, 0}_2 = H^p(Y, f_* \F) \]
\end{remark}

\subsubsection{4.2}

Let $f : X \to Y$ be an finite surjective morphism of noetherian separated schemes. We will prove that if $X$ is affine then $Y$ is affine.

\begin{enumerate}
\item Let $f : X \to Y$ be a finite surjective morphism of integral noetherian schemes. Restrict to affine opens $U \subset X$ and $V \subset Y$ such that $U = f^{-1}(V)$ and denote $U = \Spec{B}$ and $V = \Spec{A}$ where $A$ and $B$ are noetherian integral domains. Then the sheaf map $f^\# : \struct{Y} \to f_* \struct{X}$ on $V$ gives an inclusion $\varphi : A \embed B$ which makes $B$ a finitely generated $A$-module since $f : X \to Y$ is finite. Let $K = \Frac{A}$ and $L = \Frac{B}$. Now localizing this map gives $\varphi : K \to S^{-1} B$ which makes $S^{-1} B$ a finitely generated $K$-module. However, $S^{-1}B$ is a domain and a finite $K$-vectorspace and thus a field. Thus $S^{-1} B = L$ since $L$ is the smallest field containing $A$. Therefore, $K \embed L$ is a finite extension. Taking generators $x_1, \dots, x_r \in B$ for the field extension, perhapse after clearing denominators, gives a map $A^r \to B$ such that $A^r \otimes_A K \to B \otimes_A K = L$ is an isomorphism of $K$-modules.
\bigskip\\
Let $j : U \to X$ be the inclusion and let $\M$ be a coherent subsheaf of $j_* \struct{U}$ generated by global sections $x_1, \dots, x_n$ i.e. the image of $\struct{X}^{\oplus r} \to \iota_* \struct{U}$ via sections $x_1, \dots, x_n \in \Gamma(X, j_* \struct{U}) = \Gamma(U, \struct{U}) = B$.
Now define $\alpha : \struct{Y}^{\oplus r} \to f_* \M$ via the sections $x_1, \dots, x_n \in \Gamma(Y, f_* \M) = \Gamma(X, \M)$. On the affine open $V$ we see that $\alpha_U : A^r \to (x_1, \dots, x_n)$ is the previous map and thus at the generic point $\alpha_\eta : \stalk{Y}{\eta} \to (f_* \M)_\eta$ is the map $\alpha_U \otimes_K : K^r \iso (x_1, \dots, x_n) \otimes_A K = L$ is an isomorphism. 

\item Let $\F$ be any coherent sheaf on $Y$. Consider,
\[ \alpha^* : \shHom{\struct{Y}}{f_* \M}{\F} \to \shHom{\struct{Y}}{\struct{Y}^{\oplus n}}{\F} = \F^{\oplus r} \]
However, $\shHom{\struct{Y}}{f_* \M}{\F}$ is a coherent $\struct{Y}$-module and $f_* \M$ is a module over $\sA = f_* \struct{X}$ meaning that $\shHom{\struct{Y}}{f_* \M}{\F}$ is a $\sA$-module. Using (Ex. II.5.17 (e)) because $f : X \to Y$ is finite and thus affine, there is some coherent $\struct{X}$-module $\G$ such that $\shHom{\struct{Y}}{f_* \M}{\F} = f_* \G$. Then we have a morphism $\alpha^* : f_* \G \to \F^{\oplus r}$. Furthermore, since $\alpha$ is an isomorphism at the generic point and $f_* \M$ and $\F$ and $\struct{Y}$ are coherent then internal hom commutes with taking stalks meaning that,
\[ \alpha^*_\eta : \shHom{\struct{Y}}{f_* \M}{\F}_\eta = \Hom{\stalk{Y}{\eta}}{(f_* \M)_\eta}{\F_\eta} \to \Hom{\stalk{Y}{\eta}}{\stalk{Y}{\eta}^r}{\F_\eta} = \F^{\oplus r}_\eta \]
is an isomorphism because $\alpha_\eta : \stalk{Y}{\eta}^r \to (f_* \M)_\eta$ is an isomorphism.


\item Let $f : X \to Y$ be a finite surjective morphism of noetherian separated schemes and $X$ is affine. We now will show that $Y$ is affine by Noetherian induction. By (Ex. III.3.1) and (Ex. III.3.2) $Y$ is affine iff $Y_\red$ is affine iff each irreducible component $Z \subset Y_\red$ is affine. Let $\cP$ be the property of closed subsets $Z \subset Y$ that $Z$ with its reduced subscheme structure is affine. Then if $Y$ has $\cP$ meaning $Y_\red$ is affine then $Y$ is affine so, by Noetherian induction, it suffices to show that if $Z \subset Y$ is a closed subset such that every proper closed subset $C \subsetneq Z$ has $\cP$ then $Z$ has $\cP$. Notice if $Z$ is reducible this is automatic because $Z$ is affine iff each irreducible component is affine by (Ex. III.3.2) thus we need only consider the case that $Z$ is irreducible.
\bigskip\\
Base changing by $Z \embed Y$ we get a finite surjective map $X_Z \to Z$ where $X_Z \embed X$ is a closed immersion so $X_Z$ is affine. Since $X_Z \to Z$ is surjective, some $\xi \in X_Z$ must hit the generic point $\eta \in Z$. Give $W = \overline{\{ \xi \}}$ the reduced subscheme structure then composing with the closed immersion $W \embed X_Z$ gives a finite map $f' : W \to Z$ which is dominant because $\xi \mapsto \eta$ and thus surjective since $f' : W \to Z$ is closed. Since $W$ is affine and both $W$ and $Z$ are integral we have reduced to the previous case.
\bigskip\\
We will show that $Z$ is affine by using Serre's criterion. For any coherent $\struct{Z}$-module $\F$, by part (b) there is a coherent $\struct{W}$-module $\G$ and a morphism $\beta : f_* \G \to \F^{\oplus r}$ which is an isomorphism at the generic point $\eta \in Z$. Extend to an exact sequence,
\begin{center}
\begin{tikzcd}
0 \arrow[r] & \ker{\beta} \arrow[r] & f_* \G \arrow[r, "\beta"] & \F^{\oplus r} \arrow[r] & \coker{\beta} \arrow[r] & 0
\end{tikzcd}
\end{center} 
Taking the stalk at $\eta$ gives an exact sequence,
\begin{center}
\begin{tikzcd}
0 \arrow[r] & (\ker{\beta})_\eta \arrow[r] & (f_* \G)_\eta \arrow[r, "\beta"] & \F^{\oplus r}_\eta \arrow[r] & (\coker{\beta})_\eta \arrow[r] & 0
\end{tikzcd}
\end{center} 
but $\beta$ is an isomorphism at $\eta$ so $(\ker{\beta})_\eta = (\coker{\beta})_\eta = 0$ and thus their supports are proper closed subsets $C_1$ and $C_2$ of $Z$. In particular, $\ker{\beta}$ and $\coker{\beta}$ are extensions of coherent sheaves on $C_1$ and $C_2$ (with possible nonreduced structure) but by the induction hypothesis $(C_i)_\red$ is affine and thus $C_i$ is affine so for $q > 0$,
\[ H^q(X, \ker{\beta}) = H^q(X, \iota_* \iota^* \ker{\beta}) = H^q(C_1, \iota^* \ker{\beta}) = 0 \]
and likewise $H^q(X, \coker{\beta}) = 0$. Now split the exact sequence into short exact sequences,
\begin{center}
\begin{tikzcd}
0 \arrow[r] & \ker{\beta} \arrow[r] & f_* \G \arrow[r] & \sC \arrow[r] & 0
\\
0 \arrow[r] & \sC \arrow[r] & \F^{\oplus r} \arrow[r] & \coker{\beta} \arrow[r] & 0
\end{tikzcd}
\end{center}
and consider the long exact sequences,
\begin{center}
\begin{tikzcd}
H^q(Z, \ker{\beta}) \arrow[r] & H^q(Z, f_* \G) \arrow[r] & H^q(Z, \sC) \arrow[r] & H^{q+1}(Z, \ker{\beta})
\\
H^q(Z, \sC) \arrow[r] & H^q(Z, \F)^{\oplus r} \arrow[r] & H^q(Z, \coker{\beta}) \arrow[r] & H^{q+1}(Z, \sC)
\end{tikzcd}
\end{center}
For $q > 0$ we see that $H^q(Z, f_* \G) \iso H^q(Z, \sC)$ and $H^q(Z, \sC) \onto H^q(Z, \F)^{\oplus r}$ by the vanishing of cohomology of $\ker{\beta}$ and $\coker{\beta}$. Furthermore, since $f$ is affine,
\[ H^q(Z, f_* \G) = H^q(W, \G) = 0 \]
because $W$ is affine and $\G$ is coherent. Thus we see that $H^q(Z, \F) = 0$ for $q > 0$ proving that $Z$ is affine by Serre's criterion and thus $Z$ satisfies $\cP$.
\end{enumerate}

\subsubsection{4.3}


It will be convenient to label variables as,
\[ \A^d_k = \Spec{k[x_0, \cdots, x_{d-1}]} \]
and $n = d-1$ to line up with the definitions in projective space. Consider the projection morphism $\pi : \A^{n+1}_k \setminus \{ (x_1, \dots, x_n) \} \to \P^{n}_k$ and let $U = \A^d_k \setminus \{ (x_1, \dots, x_n) \}$ and $X = \P^n_k$. The schemes $D_{+}(X_i)$ for each variable $X_i$ constitute an affine open cover of $\P^{n}_k$. Furthermore, $\pi^{-1}(D_{+}(X_i)) = D(x_{i}) \subset k[x_1, \dots, x_d]$. Therefore, $\pi$ is an affine morphism and $\struct{U}$ is a quasi-coherent $\struct{U}$-module so we have shown that,
\[ H^q(\P^{n}_k, \pi_* \struct{U}) = H^q(U, \struct{U}) \] 
Furthermore, denote $S = k[x_0, \cdots, x_n]$, then,
\begin{align*}
\pi_* \struct{U} |_{D_{+}(X_i)} & = \struct{U} |_{D(x_{i})} = \struct{\A^{n+1}_k} |_{D(x_{i})} = \widetilde{S_{x_{i}}}  = \bigoplus_{k \in \Z} \widetilde{\left( S_{x_i} \right)_k} = \bigoplus_{k \in \Z} \widetilde{\left( S(k)_{x_i} \right)_{0}} = \bigoplus_{k \in \Z} \struct{X}(k)|_{D_+(X_i)}
\end{align*}
Thus, because the sheaves agree on an open affine cover, we can identify,
\[ \pi_* \struct{U} = \bigoplus_{k \in \Z} \struct{X}(k) \]
Hartshorne has computed the cohomology of the sum of twists (Hartshorne III.5, Theorem 5.1) to be,
\[  H^q \left(X, \bigoplus_{n \in k} \struct{X}(k) \right) = 
\begin{cases}
k[X_0, \cdots, X_n] & q = 0
\\
0 & 0 < q < n
\\
\frac{1}{X_0 \cdots X_n} k[X_0^{-1}, \dots, X_n^{-1}] & q = n 
\end{cases} \]
Reverting to our initial notation and using the isomorphism $H^q(X, \pi_* \struct{U}) = H^q(U, \struct{U})$ we arrive at,
\[ H^q(U, \struct{U}) = 
\begin{cases}
k[x_1, \cdots, x_d] & q = 0
\\
0 & 0 < q < n
\\
\frac{1}{x_1 \cdots x_d} k[x_1^{-1}, \dots, x_d^{-1}] & q = d-1 
\end{cases} \]
Therefore $U$ is not affine since $\struct{U}$ is coherent and yet has nontrivial cohomology on $U$.

\subsubsection{4.4}

Let $X$ be a topological space and $\F$ an abelian sheaf. 

\renewcommand{\U}{\mathfrak{U}}
\newcommand{\V}{\mathfrak{V}}

\begin{enumerate}
\item Let $\U = (U_i)_{i \in I}$ be an open cover of $X$ and $\V = (V_j)_{j \in J}$ a refinement i.e. an open cover of $X$ with a map $\lambda : J \to I$ of index sets such that $V_{j} \subset U_{\lambda(j)}$.  This refinement gives a morphism of Cech complexes $r : \check{C}^\bullet(\U, \F) \to \check{C}^\bullet(\V, \F)$ via the restriction map,
\[ \res : \prod_{i_0 < \cdots < i_r} \F(U_{i_0, \dots, i_r}) \to \prod_{j_0 < \cdots < j_r} \F(V_{j_0, \dots, j_r}) \quad \quad (\xi_{i_0, \dots, i_r}) \mapsto (\xi_{\lambda(j_0), \dots, \lambda(j_r)} |_{V_{j_0, \dots, j_r}} ) \]
making the diagram commute,
\begin{center}
\begin{tikzcd}
0 \arrow[r] & \prod\limits_{i_0} \F(U_{i_0}) \arrow[d] \arrow[r] & \prod\limits_{i_0 < i_1} \F(U_{i_0, i_1}) \arrow[d] \arrow[r] & \prod\limits_{i_0 < i_1 < i_2} \F(U_{i_0, i_1, i_2}) \arrow[d] \arrow[r] & \cdots
\\ 
0 \arrow[r] & \prod\limits_{j_0} \F(V_{j_0}) \arrow[r] & \prod\limits_{j_0 < j_1} \F(V_{j_0, j_1}) \arrow[r] & \prod\limits_{j_0 < j_1 < j_2} \F(V_{j_0, j_1, j_2}) \arrow[r] & \cdots
\end{tikzcd}
\end{center}
This induces a map of the cohomologies, 
\[ \lambda^q : \check{H}^q(\U, \F) \to \check{H}^q(\V, \F) \]

\item Take an injective resolution of abelian sheaves,
\begin{center}
\begin{tikzcd}
0 \arrow[r] & \F \arrow[r] & \I^\bullet
\end{tikzcd}
\end{center}
and consider the diagram of abelian sheaves lifting $\id_\F : \F \to \F$,
\begin{center}
\begin{tikzcd}[column sep={4em,between origins},row sep=1em]
& 0 \arrow[rr] & & \F \arrow[rr] \arrow[from = dd] & & \I^\bullet
\\
0 \arrow[rr] & & \F \arrow[rr, crossing over] \arrow[ru] & & \I^\bullet \arrow[ru, "\id_\I"]
\\
& 0 \arrow[rr] & & \F \arrow[rr]  & & \Cech^\bullet(\V, \F) \arrow[uu, dashed]
\\
0 \arrow[rr] & & \F \arrow[rr] \arrow[uu, crossing over] \arrow[ur] & & \Cech^\bullet(\U, \F) \arrow[uu, dashed, crossing over] \arrow[ur, "\lambda"']
\end{tikzcd}
\end{center}
Because the Cech resolutions are exact and $\I$ is an injective resolution $\id_\F$ lifts to morphisms of complexes $\Cech^\bullet(\U, \F) \to \I^\bullet$ and $\Cech^\bullet(\V, \F) \to \I^\bullet$. Since these lifts are unique up to homotopy and $\lambda$ is a chain map, the last square commutes up to homotopy. Taking the cohomology of the above complexes, these morphisms induce maps $\beta_\U^q : \check{H}^q(\U, \F) \to H^q(X, \F)$ and $\beta_\V^q : \check{H}^q(\V, \F) \to H^q(X, \F)$ making the following diagram commute,
\begin{center}
\begin{tikzcd}
\check{H}^q(\U, \F) \arrow[d, "\lambda^q"] \arrow[r, "\beta_\U^q"] & H^q(X, \F) \arrow[d, "\id"]
\\
\check{H}^q(\V, \F) \arrow[r, "\beta_\V^q"] & H^q(X, \F)
\end{tikzcd}
\end{center}
Consider the directed system of all open covers of $X$ partially ordered under refinement over which we define,
\[ \check{H}^q(X, \F) = \varinjlim_{\U} \check{H}(\U, \F) \]
with the given restricton maps $\lambda$. The morphisms $\beta^q_\U : \check{H}^q(\U, \F) \to H^q(X, \F)$ are compatible with the restrictions and thus define a natural morphism,
\[ \check{H}^q(X, \F) \to H(X, \F) \]  

\item Take the abelian sheaf $\F$ an inject it into a flasque sheaf $\G$ to give an exact sequence of sheaves,
\begin{center}
\begin{tikzcd}
0 \arrow[r] & \F \arrow[r] & \G \arrow[r] & \K \arrow[r] & 0
\end{tikzcd}
\end{center}
Given this injection we construct an exact sequence of complexes,
\begin{center}
\begin{tikzcd}
0 \arrow[r] & \check{C}^\bullet(\U, \F) \arrow[r] & \check{C}^\bullet(\U, \G) \arrow[r] & D^\bullet(\U) \arrow[r] & 0
\end{tikzcd}
\end{center}

\renewcommand{\Q}{\mathscr{Q}}

where $D^\bullet(\U)$ is the cokernel complex which is given by $D^\bullet(\U) = \check{C}^\bullet(\U, \Q)$ where $\Q$ is the presheaf $U \mapsto \G(U) / \F(U)$ and thus $\K$ is its sheafification. Therefore, there is a natural sheafification map $\Q \to \K$ which, as a morphism of presheaves, induces a map of Cech complexes, $D^\bullet(\U) \to \check{C}^\bullet(\U, \K)$. Furthermore, because the map $\Q \to \K$ is an isomorphism on the stalks, under refinement we have,
\[ \varinjlim_\U D^\bullet(\U) = \varinjlim_\U \check{C}^\bullet(\U, \Q) \xrightarrow{\sim} \varinjlim_\U \check{C}^\bullet(\U, \K) \]
Now the above exact sequence of sheaves and exact sequence of complexes give long exact sequences of sheaf and Cech cohomology respectivly,
\begin{center}
\begin{tikzcd}
0 \arrow[r] & \Gamma(X, \F) \arrow[r] & \Gamma(X, \G) \arrow[r] & \Gamma(X, \K) \arrow[r] & H^1(X, \F) \arrow[r] & 0
\end{tikzcd}
\end{center}
where I have used the fact that $H^1(X, \G) = 0$ because $\G$ is flasque and, 
\begin{center}
\begin{tikzcd}
0 \arrow[r] & \Gamma(X, \F) \arrow[r] & \Gamma(X, \G) \arrow[r] & H^0(D^\bullet(\U)) \arrow[r] & \check{H}^1(\U, \F) \arrow[r] & 0
\end{tikzcd}
\end{center}
where I have used the fact that $\check{H}^0(\U, \F) = H^0(\check{C}^\bullet(\U, \F)) = \Gamma(X, \F)$ for any sheaf $\F$ and that $\check{H}^1(\U, \G) = H^1(\check{C}^\bullet(\U, \G)) = 0$ because $\G$ is flasque. The morphism of complexes $D^\bullet(\U) \to \check{C}^\bullet(\U, \K)$ induces a map of cohomology,
\[ H^0(D^\bullet(\U)) \to H^0(\check{C}^\bullet(\U, \K)) = \check{H}^0(\U, \K) = \Gamma(X, \K) \]
and thus we get a morphism of exact sequences,
\begin{center}
\begin{tikzcd}
0 \arrow[r] & \Gamma(X, \F) \arrow[d] \arrow[r] & \Gamma(X, \G) \arrow[d] \arrow[r] & H^0(D^\bullet(\U)) \arrow[d] \arrow[r] & \check{H}^1(\U, \F) \arrow[d] \arrow[r] & 0
\\
0 \arrow[r] & \Gamma(X, \F) \arrow[r] & \Gamma(X, \G) \arrow[r] & \Gamma(X, \K) \arrow[r] & H^1(X, \F) \arrow[r] & 0
\end{tikzcd}
\end{center}
Because the poset of covers under refinement maps is filtered the direct limit functor is exact. Appling it to the second sequence gives an exact sequence,
\begin{center}
\begin{tikzcd}
0 \arrow[r] &  \Gamma(X, \F) \arrow[d] \arrow[r] & \Gamma(X, \G) \arrow[d] \arrow[r] & \varinjlim\limits_\U H^0(D^\bullet(\U)) \arrow[r] \arrow[d] & \varinjlim\limits_\U \check{H}^1(\U, \F) \arrow[r] \arrow[d] & 0
\\
0 \arrow[r] & \Gamma(X, \F) \arrow[r] & \Gamma(X, \G) \arrow[r] & \Gamma(X, \K) \arrow[r] & H^1(X, \F) \arrow[r] & 0
\end{tikzcd}
\end{center}
Furthermore, $\varinjlim\limits_\U$ is exact so it commutes with taking cohomology so the maps,
\begin{center}
\begin{tikzcd}
H^0(\varinjlim\limits_\U D^\bullet(\U)) \arrow[r, "\sim"] & H^0(\varinjlim\limits_\U \check{C}^\bullet(\U, \K))
\\
\varinjlim\limits_\U H^0(D^\bullet(\U)) \arrow[r] \arrow[u, equals] & \varinjlim\limits_\U H^0(\check{C}^\bullet(\U, \K)) \arrow[u, equals]  \arrow[r, equals] & \varinjlim\limits_\U \check{H}^0(\U, \K) = \Gamma(X, \K)
\end{tikzcd}
\end{center}
are isomorphisms. Therefore since cokernels are unique, the map,
\[ \varinjlim\limits_\U \check{H}^1(\U, \F) \to H^1(X, \F) \]
is an isomorphism. 


\renewcommand{\Q}{\mathbb{Q}}
\end{enumerate}


\subsubsection{4.5}


Let $X$ be a locally ringed space. Notate by $\struct{X}^\times$, the sheaf of abelian groups given by $U \mapsto \struct{X}(U)^\times$. Now let $\L$ be an invertable sheaf on $X$ meaning that there exists an open cover $\mathfrak{U}$ such that for each $U \in \mathfrak{U}$ we have isomorphisms $\varphi_U : \struct{X}|_U \to \L|_U$. Therefore, on the overlaps we have isomorphism,
\[ \varphi_{ij} = \varphi_{U_i}^{-1}|_{U_i \cap U_j} \circ \varphi_{U_j} |_{U_i \cap U_j} : \struct{X} |_{U_i \cap U_j} \to \struct{X} |_{U_i \cap U_j} \]
which, as $\struct{X}|_{U_i \cap U_j}$-module maps are determined uniquely by $e_{ij} \in \struct{X}(U_i \cap U_j)^\times$ which is a unit because the map it defines is an isomorphism. Thus, $e = (e_{ij})_{ij}$ is an element of the first Cech complex group, $C^1(\mathfrak{U}, \struct{X}^\times)$. Consider the Cech complex,
\begin{center}
\begin{tikzcd}
0 \arrow[r] & \prod\limits_{i_0} \struct{X}^\times(U_{i_0}) \arrow[r] & \prod\limits_{i_0 < i_1}  \struct{X}^\times(U_{i_0} \cap U_{i_1}) \arrow[r] & \prod\limits_{i_0 < i_1 < i_2} \struct{X}^\times(U_{i_0} \cap U_{i_1} \cap U_{i_2}) 
\end{tikzcd}
\end{center}
Furthermore, on triple overlaps,
\begin{align*}
\varphi_{ij}|_{ijk} \circ \varphi_{jk}|_{ijk} & = \varphi_{U_i}^{-1}|_{U_{ijk}} \circ \varphi_{U_j} |_{U_{ijk}} \circ \varphi_{U_j}^{-1}|_{U_{ijk}} \circ \varphi_{U_k} |_{U_{ijk}} 
\\
& = \varphi_{U_i}^{-1}|_{U_{ijk}} \circ \varphi_{U_k} |_{U_i \cap U_j \cap U_k} = \varphi_{ik} |_{ijk} 
\end{align*}
which clearly implies that $e_{ij} |_{U_{ijk}} \cdot e_{jk} |_{U_{ijk}} = e_{ik} |_{U_{ijk}}$. However, the Cech differential map $\mathrm{d} : C^1(\mathfrak{U}, \struct{X}^\times) \to \check{C}^2(\mathfrak{U}, \struct{X}^\times)$ acts via,
\[ (\d{\alpha})_{ijk} = \alpha_{jk} |_{U_{ijk}} \cdot \alpha_{ik}^{-1} |_{U_{ijk}} \cdot \alpha_{ij} |_{U_{ijk}} \]
Therefore, by the overlap identity,
\[ (\d{e})_{ijk} = e_{jk}|_{U_{ijk}} \cdot e_{ik}|_{U_{ijk}}^{-1} \cdot e_{ij} |_{U_{ijk}} = 1 \]
Thus $e$ is in the kernel of the Cech differential $\mathrm{d} : \check{C}^1(\mathfrak{U}, \struct{X}^\times) \to \check{C}^2(\mathfrak{U}, \struct{X}^\times)$ and thus $e$ represents a Cech cohomology class $[e] \in \check{H}^1(\mathfrak{U}, \struct{X}^\times)$. Furthermore, if $\tilde{\varphi}_{U_i} : \struct{X} |_{U_i} \to \L |_{U_i}$ is another choice of locally trivializing isomorphisms then denote $\tilde{e}_{ij} \in \struct{X}^\times(U_i \cap U_j)$ for the element determining the isomorphisms,
\[ \tilde{\varphi}_{ij} = \tilde{\varphi}_{U_i}^{-1} |_{U_{ij}} \circ \tilde{\varphi}_{U_j} |_{U_{ij}} : \struct{X} |_{U_i \cap U_j} \to \struct{X} |_{U_i \cap U_j} \]
Then we may consider the isomorphisms $t_i = \tilde{\varphi}_{U_i}^{-1} \circ \varphi_{U_i} : \struct{X} |_{U_i} \to \struct{X} |_{U_i}$ which are defined by an element $f_i \in \struct{X}^\times(U_i)$. Then we find that,
\begin{align*}
\tilde{\varphi}_{ij} & = \tilde{\varphi}_{U_i}^{-1} |_{U_{ij}} \circ \tilde{\varphi}_{U_j} |_{U_{ij}} = \tilde{\varphi}_{U_i}^{-1} |_{U_{ij}} \circ \varphi_{U_i} |_{U_{ijk}} \circ \varphi_{U_i}^{-1} |_{U_{ijk}} \circ \varphi_{U_j} |_{U_{ijk}} \circ  \varphi_{U_j} |_{U_{ijk}}^{-1} \circ \tilde{\varphi}_{U_j} |_{U_{ij}} 
\\
& = t_i |_{U_{ij}} \circ \varphi_{ij} \circ t_j^{-1} |_{U_{ij}}
\end{align*}
This shows that the elements must satisfy, $\tilde{e}_{ij} \cdot e_{ij}^{-1} = t_i |_{U_{ij}} \cdot t_j^{-1} |_{U_{ij}}$. Furthermore, the Cech differential map $\mathrm{d} : \check{C}^0(\mathfrak{U}, \struct{X}^\times) \to \check{C}^1(\mathfrak{U}, \struct{X}^\times)$ acts via,
\[ (\d{\alpha})_{ij} = \alpha_{i} |_{U_{ij}} \cdot \alpha_{j}^{-1} |_{U_{ij}}  \]
Therefore, let $f = (f_i)_i$ then $\d{f} = \tilde{e} \cdot e^{-1}$ which implies that $[\tilde{e}] = [e]$ in $\check{H}^1(\mathfrak{U}, \struct{X}^\times)$ so the cohomology class $[e]$ associated to the invertable sheaf $\L$ is well-defined. The map $\L \mapsto [e]$ is well-defined for sheaves which are locally trivialized on $\mathfrak{U}$. Therefore we get a well-defined map,
\[ \Pic{X} \to \check{H}^1(X, \struct{X}^\times) = \varinjlim_{\mathfrak{U}} \check{H}(\mathfrak{U}, \struct{X}^\times) \]
via decomposing,
\[ \Pic{X} = \bigcup_{\mathfrak{U}} \Pic{\mathfrak{U}, X} \quad \text{where} \quad \Pic{\mathfrak{U}, X} = \{ \mathcal{L} \in \Pic{X} \mid \forall U \in \mathfrak{U} : \mathcal{L}|_U \cong \struct{U} \} \]
and mapping,
\[ \Pic{\mathfrak{U}, X} \to \check{H}^1(\mathfrak{U}, \struct{X}^\times) \to \varinjlim_{\mathfrak{U}} \check{H}(\mathfrak{U}, \struct{X}^\times) = \check{H}^1(X, \struct{X}^\times) \]
using the constructed map. This map is an homomorphism because given invertable sheaves $\L_1$ and $\L_2$ and isomorphisms $\varphi^r_{U_i} : \struct{X} |_{U_i} \to \L_r$ corresponding to cohomology classes $[e^r]$ then there is a natural map,
\[ \varphi^1_{U_i} \otimes \varphi^2_{U_i} \struct{X}|_{U_i} \to \L_1 |_{U_i} \otimes_{\struct{X} |_{U_i}} \L_2 |_{U_i} \]
which therefore gives overlap maps,
\[ \varphi_{ij}^\otimes = ((\varphi^1_{U_i})^{-1} \circ \varphi^1_{U_j}) \otimes ((\varphi^2_{U_i})^{-1} \circ \varphi^2_{U_j}) = \varphi_{ij}^1 \otimes \varphi_{ij}^2 \]
and thus, $\varphi_{ij}^\otimes(1) = e_{ij}^1 \otimes e^2_{ij} \mapsto e_{ij}^1 e_{ij}^2$ under the natural identification,
\[ \struct{X}(U_{ij}) \otimes_{\struct{X}(U_{ij})} \struct{X}(U_{ij}) \to \struct{X}(U_{ij}) \]
Therefore, the invertable sheaf $\L_1 \otimes_{\struct{X}} \L_2$ maps to the cohomology class $[e^1 e^2] = [e^1] [e^2]$ so this map is a homomorphism. 
\bigskip\\
I claim that this map is, in fact, an isomorphism. Let $\L$ be an invertable sheaf represented by the cohomology class $[e] = [1]$ then we know that $e_{ij} = t_i |_{U_{ij}} \cdot t_j^{-1} |_{U_{ij}}$ for some set of invertable sections $t_i$. Therefore, modify the isomorphism $\varphi_{U_i} : \struct{X}|_{U_i} \to \L |_{U_i}$ which gave rise to this cohomology representative via $\tilde{\varphi}_{U_i} = t_i \varphi_{U_i}$ which are still isomorphism because $t_i \in \struct{X}(U_i)^\times$ is invertable. Therefore, 
\[ \tilde{\varphi}_{U_i}^{-1}|_{U_{ij}} \circ \tilde{\varphi}_{U_j}|_{U_{ij}} = (t_i |_{U_{ij}}^{-1} \cdot t_j |_{U_{ij}}) \varphi_{U_i}^{-1}|_{U_{ij}} \circ \varphi_{U_j}|_{U_{ij}} = \id_{\struct{X}(U_{ij})} \]
this map takes $1 \mapsto (t_i |_{U_{ij}}^{-1} \cdot t_j |_{U_{ij}}) e_{ij} = 1$ so as a morphism of $\struct{X}|_{U_{ij}}$-modules is the identity map. Thus $\tilde{\varphi}_{U_i} |_{U_{ij}} = \tilde{\varphi}_{U_j} |_{U_{ij}}$, so the isomorphisms $\tilde{\varphi}_{U_i} \in \shHom{}{\struct{X}|_{U_i}}{\L|_{U_i}}$ glue since they agree on this open cover to a global isomorphism $\tilde{\varphi} : \struct{X} \to \L$ so $\L$ is a trivial invertable sheaf. Thus $\Pic{X} \to \check{H}^1(X, \struct{X})$ is injective. It remains to prove that it is surjective. Given any cohomology class $[e] \in \check{H}^1(X, \struct{X}^\times)$ we may construct an invertable sheaf as follows. Define $\L$ via,
\[ \L(V) = \{ f_i \in \struct{X}(U_i \cap V) \mid f_i |_{U_{ij} \cap V} \cdot e_{ij} |_{U_{ij} \cap V} = f_j |_{U_{ij} \cap V} \} \]
It is clear that this is an invertable sheaf if $e_{ij}$ satisfies the transition property given by its Cech differential vanishing and that $\L \mapsto [e]$. 
\bigskip\\
Finally, we use the general fact that $H^1(X, \F) = \check{H}^1(X, \F)$ to conclude that,
\[ \Pic{X} \cong H^1(X, \struct{X}^\times) \]

\subsubsection{4.6}

Let $(X, \struct{X})$ be a ringed space and $\J$ be a sheaf of ideals of $\struct{X}$ such that $\J^2 = 0$. Let $X_0$ be the ringed space $(X, \struct{X} / \J)$. Now consider the sequence of sheaves over $X$,
\begin{center}
\begin{tikzcd}
0 \arrow[r] & \J \arrow[r] & \struct{X}^\times \arrow[r] & \struct{X_0}^\times \arrow[r] & 0
\end{tikzcd}
\end{center}
where $\J \to \struct{X}^\times$ is the map $a \mapsto 1 + a$ which is a unit because $(1 + a)(1 - a) = 1 - a^2 = 1$ since $a^2 \in \J^2 = 0$. This map is clearly injective. The map $\struct{X}^\times \to \struct{X_0}^\times$ is the projection. At the stalks $\stalk{X}{x}^\times \to \stalk{X_0}{x}^\times$ the map is simply the projection $\stalk{X}{x}^\times \to (\stalk{X}{x} / \J_x)^\times$. Now if $ab - 1 \in \J_x$ then $ab = 1 + z$ for some $z \in \J_x$. Thus,
\[ ab(1 - z) = (1 + z)(1 - z) = 1 - z^2 = 1 \]
so $a \in \stalk{X}{x}^\times$ is actually invertible i.e. the stalk maps are surjective so $\struct{X}^\times \to \struct{X_0}^\times$ is a surjective morphism of sheaves. Now if $a \in \J_x$ then $1 + a = 1$ in $\stalk{X}{x} / \J_x$ so the sequence is a complex. Futhermore, if $1 + a = 1$ in $\stalk{X}{x} / \J_x$ then $a \in \J_x$ so the sequence is exact. Therefore, appling the long exact sequence of cohomology we get,
\begin{center}
\begin{tikzcd}
H^1(X, \J) \arrow[r] & H^1(X, \struct{X}) \arrow[r] & H^1(X, \struct{X_0}) \arrow[r] & H^2(X, \J)
\end{tikzcd}
\end{center}
Furthermore, using the identification $H^1(X, \struct{X}) = \Pic{X}$ for any ringed space we find an exact sequence,
\begin{center}
\begin{tikzcd}
H^1(X, \J) \arrow[r] & \Pic{X} \arrow[r] & \Pic{X_0} \arrow[r] & H^2(X, \J)
\end{tikzcd}
\end{center}

\subsubsection{4.7}

Let $X$ be the closed subscheme of $\P^2_k$ defined by the homogeneous polynomial $f(x_0, x_1, x_2) = 0$ of degree $d$. Let $S = k[x_0, x_1, x_2]$ be the graded ring such that $\P^2_k = \Proj{S}$. Now consider the exact sequence of graded rings,
\begin{center}
\begin{tikzcd}
0 \arrow[r] & S(-d) \arrow[r, "\times f"] & S \arrow[r] & S / (f) \arrow[r] & 0 
\end{tikzcd}
\end{center}
which gives an exact sequence of $\struct{\P^2_k}$-modules,
\begin{center}
\begin{tikzcd}
0 \arrow[r] & \widetilde{S(-d)} \arrow[r, "\times f"] & \widetilde{S} \arrow[r] & \widetilde{S / (f)} \arrow[r] & 0 
\end{tikzcd}
\end{center}
Let $\iota : \Proj{S/(f)} \to \Proj{S}$ be the closed immersion of the closed subscheme $X = \Proj{S/(f)}$ which is the plane curve corresponding the vanishing of $f$. Then $\widetilde{S / (f)} = \iota_* \struct{X}$ so we may rewrite this exact sequence as,
\begin{center}
\begin{tikzcd}
0 \arrow[r] & \struct{\P^2_k}(-d) \arrow[r, "\times f"] & \struct{\P^2_k} \arrow[r] & \iota_* \struct{X} \arrow[r] & 0 
\end{tikzcd}
\end{center}
Taking the long exact sequence of cohmology we find,
\begin{center}
\begin{tikzcd}[column sep = small]
0 \arrow[r] & H^0(\P^2_k, \struct{\P^2_k}(-d)) \arrow[draw=none]{d}[name=Z, shape=coordinate]{} \arrow[r] & H^0(\P^2_k, \struct{\P^2_k}) \arrow[r] & H^0(\P^2_k, \iota_* \struct{X}) \arrow[r] & H^1(\P^2_k, \struct{\P^2_k}(-d)) 
\arrow[dlll,
rounded corners, crossing over,
to path={ -- ([xshift=2ex]\tikztostart.east)
|- (Z) [near end]\tikztonodes
-| ([xshift=-2ex]\tikztotarget.west)
-- (\tikztotarget)}]
\\ 
& H^1(\P^2_k, \struct{\P^2_k}) \arrow[r] & H^1(\P^2_k, \iota_* \struct{X}) \arrow[r] & H^2(\P^2_k, \struct{\P^2_k}(-d)) \arrow[r] & H^2(\P^2_k, \struct{\P^2_k}) \arrow[r] & 0
\end{tikzcd}
\end{center}
Since $\iota : X \to \P^2_k$ is a closed immersion it is affine and thus,
\[ H^q(\P^2_k, \iota_* \F) = H^q(X, \F) \]
for any quasi-coherent $\struct{X}$-module and $q \ge 0$. In partiuclar, $H^q(\P^2_k, \iota_* \struct{X}) = H^q(X, \struct{X})$ and also we know that $H^1(\P^2_k, \struct{\P^2_k}(n)) = 0$. Therefore, the long exact sequence becomes,
\begin{center}
\begin{tikzcd}[column sep = small]
0 \arrow[r] & H^0(\P^2_k, \struct{\P^2_k}(-d)) \arrow[draw=none]{d}[name=Z, shape=coordinate]{} \arrow[r] & H^0(\P^2_k, \struct{\P^2_k}) \arrow[r] & H^0(X, \struct{X}) \arrow[r] & 0
\arrow[dlll,
rounded corners, crossing over,
to path={ -- ([xshift=2ex]\tikztostart.east)
|- (Z) [near end]\tikztonodes
-| ([xshift=-2ex]\tikztotarget.west)
-- (\tikztotarget)}]
\\ 
& 0 \arrow[r] & H^1(X, \struct{X}) \arrow[r] & H^2(\P^2_k, \struct{\P^2_k}(-d)) \arrow[r] & H^2(\P^2_k, \struct{\P^2_k}) \arrow[r] & 0
\end{tikzcd}
\end{center}
Furthermore, since $-d < 0$ then $H^0(\P^2_k, \struct{\P^2_k}(-d)) = 0$ (because $S$ has no negative degree terms) and we know $H^0(\P^2_k, \struct{\P^2_k}) = k$. Therefore, $H^0(X, \struct{X}) = k$ and, in particular, it has dimension 1. Furthermore, 
\[ H^2(\P^2_k, \struct{\P^2_k}(n)) = \left( \frac{1}{x_0 x_1 x_2} k[x_0^{-1}, x_1^{-1}, x_2^{-1}] \right)_n \]
Thus, $\dim_k H^2(\P^2_k, \struct{\P^2_k}) = 0$ which implies that,
\[ H^1(X, \struct{X}) = H^2(\P^2_k, \struct{\P^2_k}(-d)) \]
Furthermore, a basis is given by $x_0^{-(a + 1)} x_1^{-(b + 1)} x_2^{-(c + 1)}$ where $a + b + c + 3 = d$. The number of solutions with $0 \le a,b,c \le d - 3$ is given as follows. There are $(d - 2)$ choices for $a$ in which case there are $d - a - 2$ choices for $b$ which fixes $c$. Then the number of solutions is thus,
\begin{align*}
\dim_k H^1(X, \struct{X}) & = \sum_{a = 0}^{d - 3} (d - a - 2) = (d - 2)^2 - \sum_{a = 0}^{d - 3} a 
\\
& = (d - 2)^2 - \tfrac{1}{2} (d - 3)(d - 2) = \tfrac{1}{2} (d - 1)(d - 2)
\end{align*}

\subsubsection{4.11}

Let $X$ be a topological space, $\F$ a sheaf of abelian groups on $X$, and $\U$ an open cover of $X$ such that on all finite intersections $V = U_{i_0} \cap \cdots \cap U_{i_p}$ the sheaf $\F|_V$ is acyclic i.e. $H^q(V, \F|_V) = 0$ for all $q > 0$. Now embedd $\F$ into an injective sheaf $\I$ and take its cokernel $\K$ to form a short exact sequence of sheaves on $X$,
\begin{center}
\begin{tikzcd}
0 \arrow[r] & \F \arrow[r] & \I \arrow[r] & \K \arrow[r] & 0
\end{tikzcd}
\end{center}
The long exact sequence of the cohomology of the left-exact functor $\Gamma(V, (-)|_V)$ gives,
\begin{center}
\begin{tikzcd}
0 \arrow[r] & H^0(V, \F|_V) \arrow[r] & H^0(V, \F_V) \arrow[r] & H^0(V, \K|_V) \arrow[r] & H^1(V, \F|_V) 
\end{tikzcd}
\end{center} 
However, by assumption, $H^1(V, \F|_V) = 0$ on each finite intersection $V = U_{i_0} \cap \cdots \cap U_{i_p}$. Therefore, there is an exact sequence of abelian groups,
\begin{center}
\begin{tikzcd}
0 \arrow[r] & \F(V) \arrow[r] & \I(V) \arrow[r] & \K(V) \arrow[r] & 0
\end{tikzcd}
\end{center} 
and thus taking products over possible intersections $V$ we find that the sequence of Cech complexes,
\begin{center}
\begin{tikzcd}
0 \arrow[r] & \check{C}^\bullet(\U, \F) \arrow[r] & \check{C}^\bullet(\U, \I) \arrow[r] & \check{C}^\bullet(\U, \K) \arrow[r] & 0
\end{tikzcd}
\end{center}
is exact. Taking the long exact sequence associated to this sequence of Cech complexes gives a long exact sequence of Cech cohomology. However, $\I$ is injective and thus flasque so the higher Cech and sheaf cohomology vanishes. Therefore, we have a morphism of exact sequences,
\begin{center}
\begin{tikzcd}
0 \arrow[r] & \check{H}^0(\U, \F) \arrow[r] \arrow[d] & \check{H}^0(\U, \F) \arrow[r] \arrow[d] & \check{H}^0(\U, \K) \arrow[r] \arrow[d] & \check{H}^1(\U, \F) \arrow[r] \arrow[d] & 0
\\
0 \arrow[r] & H^0(X, \F) \arrow[r] & H^0(X, \F) \arrow[r] & H^0(X, \K) \arrow[r] & H^1(X, \F) \arrow[r] & 0
\end{tikzcd}
\end{center}
However, for any abelian sheaf $\G$,
\[ \check{H}^0(\U, \G) = H^0(\check{C}^\bullet(\U, \G)) = \Gamma(X, \G) = H^0(\U, \G) \]
so the first three downwards maps are isomorphisms and thus, by the five lemma, the map,
\[ \check{H}^1(\U, \F) \xrightarrow{\sim} H^1(\U, \F) \]
is an isomorphism proving the theorem at $q = 1$. 
\bigskip\\
In proceed by induction on $q$, assume that the map,
\[ \check{H}^q(\U, \G) \xrightarrow{\sim} H^q(X, \G) \]
is an isomorphism for any abelian sheaf $\G$ satisfying $H^p(V, \G|_V) = 0$ for all $p \ge 1$ and finite intersection $V$. 
\bigskip\\
Now the long exact sequence of Cech cohomology gives an isomorphism,
\begin{center}
\begin{tikzcd}
0 \arrow[r] & \check{H}^q(\U, \K) \arrow[r,"\sim"] & \check{H}^{q+1}(\U, \F) \arrow[r] & 0
\end{tikzcd}
\end{center}
Furthermore, for $p \ge 1$ the long exact sequence of sheaf cohomology restricted to $V$ gives,
\begin{center}
\begin{tikzcd}
H^p(V, \F|_V) \arrow[r] & H^p(V, \I|_V) \arrow[r] & H^p(V, \K|_V) \arrow[r] & H^{p+1}(V, \F|_V)
\end{tikzcd}
\end{center}
By assumption, $H^p(V, \F|_V) = 0$ and since $\I|_V$ is injective $H^p(V, \I|_V) = 0$ and thus $H^p(V, \K|_V) = 0$ for all $p \ge 1$ and any finite intersection $V = U_{i_0} \cap \cdots \cap U_{i_p}$. Therefore, $\K$ is an abelian sheaf on $X$ satisfying the hypothesis. Thus the isomorphisms,
\begin{center}
\begin{tikzcd}
\check{H}^q(\U, \K) \arrow[d] \arrow[r, "\sim"] & \check{H}^{q+1}(\U, \F) \arrow[d]
\\
H^q(\U, \K) \arrow[r, "\sim"] & H^{q+1}(\U, \F) 
\end{tikzcd}
\end{center}
shift the isomorphism for $\K$ given by the induction hypothesis to an isomorphism for $\F$ in one degree higher,
\[ \check{H}^{q+1}(\U, \F) \xrightarrow{\sim} H^{q+1}(X, \F) \]
completing the proof by induction.
\bigskip\\
We can alternatively apply the Cech-to-derived spectral sequence,
\[ E_2^{p,q} = \check{H}^p(\U, \cH^q(\F)) \implies H^{p+q}(X, \F) \]
where $\cH^q(\F)$ is the presheaf $U \mapsto H^q(U, \F|_U)$. Since we assumed that $\F|_V$ is acyclic for each multiple intersection $V$ of the cover, we find that $\check{H}^p(\U, \cH^q(\F)) = 0$ for $q > 0$ and therefore the spectral sequence collapses at the $E_2$-page giving natural isomorphisms $H^n(X, \F) \iso \check{H}^n(\U, \F)$ since $\cH^0(\F) = \F$. 

\subsection{5}

\subsubsection{5.1}

Let $X$ be a proper scheme over $k$ and $\F$ a coherent $\struct{X}$-module. Then we know that $H^q(X, \F)$ are finite-dimensional $k$-vectorspaces and vanish for sufficiently large $q$. Therefore, the Euler-characteristic,
\[ \chi(\F) = \sum_{i = 0}^\infty (-1)^i \dim_k H^i(X, \F) \]
is a well-defined integer $\chi(\F) \in \Z$. 
Consider an exact sequence of coherent sheaves,  
\begin{center}
\begin{tikzcd}
0 \arrow[r] & \F \arrow[r] & \G \arrow[r] & \H \arrow[r] & 0
\end{tikzcd}
\end{center}
from which there is a long exact sequence of cohomology,
\begin{center}
\begin{tikzcd}[column sep = small]
0 \arrow[r] & H^0(X, \F) \arrow[r] & H^0(X, \G) \arrow[r] & H^0(X, \H) \arrow[r] & H^1(X, \F) \arrow[r] & H^1(X, \G) \arrow[draw=none]{d}[name=Z, shape=coordinate]{} \arrow[r] & H^1(X, \H)
\arrow[dlllll,
rounded corners, crossing over,
to path={ -- ([xshift=2ex]\tikztostart.east)
|- (Z) [near end]\tikztonodes
-| ([xshift=-2ex]\tikztotarget.west)
-- (\tikztotarget)}]
\\ 
& H^2(X, \F) \arrow[r] & H^2(X, \G)  \arrow[r] & H^2(X, \H) \arrow[r] & \cdots \arrow[r] & H^n(X, \H) \arrow[r] & 0
\end{tikzcd}
\end{center}
where the cohomology vanishes above the dimension of the scheme $X$. These groups are $k$-vectorspaces because $X$ is a scheme over $k$. By Lemma \ref{alt_sum_exact} we have the alternating sum,
\begin{align*}
\sum_{i = 0}^n (-1)^n \left[ \dim_k H^i(X, \F) - \dim_k H^i(X, \G) + \dim_k H^i(X, \H) \right] = \chi(\F) - \chi(\G) + \chi(\H) = 0  
\end{align*}
Therefore,
\[ \chi(\G) = \chi(\F) + \chi(\H) \]


\subsection{Section 6}

\subsubsection{6.1}

Let $(X, \struct{X})$ be a ringed space and $\F_1, \F_2$ be $\struct{X}$-modules. Given an exension,
\begin{center}
\begin{tikzcd}
0 \arrow[r] & \F_1 \arrow[r] & \F \arrow[r] & \F_2 \arrow[r] & 0
\end{tikzcd}
\end{center}
then the long exact sequence for $\Hom{\struct{X}}{\F_2}{-}$ gives an exact sequence,
\begin{center}
\begin{tikzcd}
\Hom{\struct{X}}{\F_2}{\F} \arrow[r] & \Hom{\struct{X}}{\F_2}{\F_2} \arrow[r, "\delta"] & \Ext{1}{\struct{X}}{\F_2}{\F_1} 
\end{tikzcd}
\end{center}
so we get a class $\delta(\id) \in \Ext{1}{\struct{X}}{\F_2}{\F_1}$. Furthermore, given a isomorphism of extensions,
\begin{center}
\begin{tikzcd}
0 \arrow[r] & \F_1 \arrow[d, equals] \arrow[r] & \F \arrow[d, "\sim"] \arrow[r] & \F_2 \arrow[r] \arrow[d, equals] & 0
\\
0 \arrow[r] & \F_1 \arrow[r] & \F' \arrow[r] & \F_2 \arrow[r] & 0
\end{tikzcd}
\end{center}
we get a morphism of exact sequences,
\begin{center}
\begin{tikzcd}
\Hom{\struct{X}}{\F_2}{\F_2} \arrow[r, "\delta"] \arrow[d, equals] & \Ext{1}{\struct{X}}{\F_2}{\F_1} \arrow[d, equals]
\\
\Hom{\struct{X}}{\F_2}{\F_2} \arrow[r, "\delta'"] & \Ext{1}{\struct{X}}{\F_2}{\F_1} 
\end{tikzcd}
\end{center}
and thus $\delta'(\id) = \delta(\id)$ so the ext class only depends on the isomorphism class of the extension. Suppose that the class $\delta(\id) = 0$ then by exactness $\id$ is in the image of,
\[ \Hom{\struct{X}}{\F_2}{\F} \to \Hom{\struct{X}}{\F_2}{\F_2} \]
so there exists a map $\pi : \F_2 \to \F$ such that $\F_2 \F \to \F_2$ is the identity meaning the sequence,
\begin{center}
\begin{tikzcd}
0 \arrow[r] & \F_1 \arrow[r] & \F \arrow[r] & \F_2 \arrow[r] \arrow[l, bend right] & 0
\end{tikzcd}
\end{center}
is split on the right and thus the extension is trivial and $\F = \F_1 \oplus \F_2$.
\bigskip\\
Now, suppose that $\xi \in \Ext{1}{\struct{X}}{\F_2}{\F_1}$ is a class. Choose an embedding $\F_1 \embed \I$ into an injective $\struct{X}$-module $\I$ and consider the exact sequence,
\begin{center}
\begin{tikzcd}
0 \arrow[r] & \F_1 \arrow[r] & \I^0 \arrow[r] & \K \arrow[r] & 0
\end{tikzcd}
\end{center}
where $\K$ is the cokernel. Then consider the long exact sequence for the functor $\Hom{\struct{X}}{\F_1}{-}$,
\begin{center}
\begin{tikzcd}
\Hom{\struct{X}}{\F_2}{\I} \arrow[r] & \Hom{\struct{X}}{\F_2}{\K} \arrow[r] & \Ext{1}{\struct{X}}{\F_2}{\F_1} \arrow[r] & \Ext{1}{\struct{X}}{\F_2}{\I}
\end{tikzcd}
\end{center}
Because $\I$ is injective, $\Ext{1}{\struct{X}}{\F_2}{\I} = 0$ so we see that,
\[ \Ext{1}{\struct{X}}{\F_2}{\F_1} = \coker{(\Hom{\struct{X}}{\F_2}{\I} \to \Hom{\struct{X}}{\F_2}{\K})} \]
Therefore we get a morphism $\varphi : \F_2 \to \K$ canonical up to a map $\F_2 \to \I$. Then we define a sheaf via the pullback,
\begin{center}
\begin{tikzcd}
& \F \arrow[r] \arrow[d] & \F_2 \arrow[d]
\\
\F_1 \arrow[r] \arrow[ru, dashed] & \I \arrow[r] & \K 
\end{tikzcd}
\end{center}
giving a map $\F_1 \to \F$ where the map $\F_1 \to \F_2$ is zero making the diagram commute because $\F_1 \to \I \to \K$ is zero. Then we get a morphism of exact sequences,
\begin{center}
\begin{tikzcd}
0 \arrow[r] & \F_1 \arrow[d, equals] \arrow[r] & \F \arrow[d] \arrow[r] & \F_2 \arrow[d] \arrow[r] & 0
\\
0 \arrow[r] & \F_1 \arrow[r] & \I \arrow[r] & \K \arrow[r] & 0
\end{tikzcd}
\end{center}
giving a morphism of long exact sequences for $\Hom{\struct{X}}{\F_2}{-}$,
\begin{center}
\begin{tikzcd}
\Hom{\struct{X}}{\F_2}{\F} \arrow[r] \arrow[d] & \Hom{\struct{X}}{\F_2}{\F_2} \arrow[r, "\delta"] \arrow[d] & \Ext{1}{\struct{X}}{\F_2}{\F_1} \arrow[r] \arrow[d, equals] & \Ext{1}{\struct{X}}{\F_2}{\F} \arrow[d]
\\
\Hom{\struct{X}}{\F_2}{\I} \arrow[r] & \Hom{\struct{X}}{\F_2}{\K} \arrow[r] & \Ext{1}{\struct{X}}{\F_2}{\F_1} \arrow[r] & 0
\end{tikzcd}
\end{center}
where $\id \in \Hom{\struct{X}}{\F_2}{\F_2}$ maps to $\varphi \in \Hom{\struct{X}}{\F_2}{\K}$ and under the map $\Hom{\struct{X}}{\F_2}{\K} \to \Ext{1}{\struct{X}}{\F_2}{\F_1}$  $\varphi \mapsto \xi$ proving that $\delta(\id) = \xi$. 

\subsubsection{6.2 DO!!}

Let $X = \P^1_k$ where $k$ is an infinite field.

\newcommand{\sP}{\mathscr{P}}

\begin{enumerate}
\item Suppose that $\sP$ is a projective $\struct{X}$-module with a surjective map $\P \onto \struct{X} \to 0$. For each $x \in X$, there is a surjection $\struct{X} \onto \iota_x \kappa(x)$ and therefore a surjection $\sP \onto \iota_* \kappa(x)$. Now, for any open $V \subset X$ containing $x$ we also have a surjection $j_! (\struct{X}|_V) \onto \iota_* \kappa(x)$. Therefore, we have a diagram,
\begin{center}
\begin{tikzcd}
\sP \arrow[d, two heads, dashed] \arrow[r, two heads] & \struct{X}  \arrow[d, two heads]
\\
j_!(\struct{X}|_V) \arrow[r, two heads] & \iota_* \kappa(x) 
\end{tikzcd}
\end{center}
Now suppose that $U \subset X$ is an open containing $x$ not contained in $V$. We get a diagram,
\begin{center}
\begin{tikzcd}
\sP(U) \arrow[d, dashed] \arrow[r] & \struct{X}(U) \arrow[d, two heads]
\\
j_!(\struct{X}|_V)(U) \arrow[r, two heads] & \kappa(x)
\end{tikzcd}
\end{center}
but $j_!(\struct{X}|_V)(U) = 0$ since $V \not\subset U$ and thus $\sP(U) \to \kappa(x)$ factors through zero. Thus every section of $\sP(U)$ vanishes at $x$. For any $U \subset X$ and $x \in U$ we can find an open $V$ and a point $x$ such that $x \in U \not\subset V$ by removing a hyperplane from $U$ not containing $x$, using that $k$ is infinite. Therefore every section in $\sP(U)$ vanishes at $x$ but $\sP \onto \struct{X}$ is surjective so $1$ must locally be in the image giving a contradiction.

\item Now suppose that $\sP$ is projective in $\Coh{X}$ or $\QCoh{X}$ and there is a surjection $\sP \onto \struct{X}$. For any invertible sheaf $\L$ and $x \in X$ there is a surjection $\L \onto \iota_x \kappa(x)$ giving a diagram,
\begin{center}
\begin{tikzcd}
\sP \arrow[d, two heads, dashed] \arrow[r, two heads] & \struct{X}  \arrow[d, two heads]
\\
\L \arrow[r, two heads] & \iota_* \kappa(x) 
\end{tikzcd}
\end{center}
Because $\struct{X}$ and $\L$ are coherent, we can find a coherent subsheaf $\sP' \subset \sP$ such that $\sP' \onto \struct{X}$ and $\sP' \onto \L$ are surjective. However, we can twist $\sP' \to \struct{X}$ by a sufficiently large $n$ such that,
\[ H^0(X, \sP'(n)) \onto H^0(X, \struct{X}(n)) \]
is surjective. Therefore, we get a diagram,
\begin{center}
\begin{tikzcd}
H^0(X, \sP'(n)) \arrow[d] \arrow[r, two heads] & H^0(X, \struct{X}(n)) \arrow[d]
\\
H^0(X, \L(n)) \arrow[r] & \kappa(x)
\end{tikzcd}
\end{center}
Since $\L$ is arbitrary, we can choose $\L = \struct{X}(-n-1)$ to get,
\begin{center}
\begin{tikzcd}
H^0(X, \sP'(n)) \arrow[r, two heads] \arrow[d] & H^0(X, \struct{X}(n)) \arrow[d]
\\
H^0(X, \struct{X}(-1)) \arrow[r] & \kappa(x)
\end{tikzcd}
\end{center}
but $H^0(X, \struct{X}(-1)) = 0$ so $H^0(X, \sP'(n)) \onto H^0(X, \struct{X}(n)) \onto \kappa(x)$ factors through zero which is impossible because $\struct{X}(n)$ has nonvanishing sections at $x$.

\end{enumerate}

\subsubsection{6.3}

Let $X$ be a Noetherian scheme and $\F, \G$ be $\struct{X}$-modules. Let $\F$ be coherent. We want to show that,
\begin{enumerate}
\item $\G$ is quasi-coherent then $\shExt{i}{\struct{X}}{\F}{\G}$ is quasi-coherent
\item $\G$ is coherent then $\shExt{i}{\struct{X}}{\F}{\G}$ is coherent.
\end{enumerate}
\noindent
First, because $(-)|_U$ is exact and preserves injectives and,
\[ \shHom{\struct{X}}{\F}{\G}|_U = \shHom{\struct{U}}{\F|_U}{\G|_U} \]
we see that,
\[ \shExt{i}{\struct{X}}{\F}{\G}|_U = \shExt{i}{\struct{U}}{\F|_U}{\G|_U} \] 
For each point $x \in X$ we can find a neighbrohood $x \in U$ on which $\F$ is finitely presented and thus there exists a sequence
\begin{center}
\begin{tikzcd}
\struct{U}^{\oplus m} \arrow[r] & \struct{U}^{\oplus n} \arrow[r] & \F|_U \arrow[r] & 0
\end{tikzcd}
\end{center}
and therefore we get an exact sequence,
\begin{center}
\begin{tikzcd}
\shHom{\struct{X}}{\struct{X}^{\oplus m}}{\G} |_U \arrow[r] & \shHom{\struct{X}}{\struct{U}^{\oplus n}}{\G} |_U \arrow[r] & \shHom{\struct{X}}{\F}{\G}|_U \arrow[r] & 0
\end{tikzcd}
\end{center}
However, $\shHom{\struct{X}}{\struct{X}^{\oplus n}}{\G} = \G^{\oplus n}$ giving an exact sequence,
\begin{center}
\begin{tikzcd}
\G|_U^{\oplus m} \arrow[r] & \G|_U^{\oplus n} \arrow[r] & \shHom{\struct{X}}{\F}{\G}|_U \arrow[r] & 0
\end{tikzcd}
\end{center}
and therefore if $\G$ is (quasi)-coherent then so is $\shHom{\struct{X}}{\F}{\G}|_U $ because it is the cokernel of (quasi)-coherent $\struct{U}$-modules proving the base case, $\shExt{0}{\struct{X}}{\F}{\G} = \shHom{\struct{X}}{\F}{\G}$ because (quasi)-coherence is a local property. Now we proceed by induction using the local exact sequence,
\begin{center}
\begin{tikzcd}
0 \arrow[r] & \K \arrow[r] & \struct{U}^{\oplus n} \arrow[r] & \F|_U \arrow[r] & 0
\end{tikzcd}
\end{center}
where $\K$ ic coherent because it is the kernel of coherent modules. The long exact sequence gives,
\begin{center}
\begin{tikzcd}
\shExt{i}{\struct{U}}{\struct{U}^{\oplus n}}{\G|_U} \arrow[r] & \shExt{i}{\struct{U}}{\K}{\G|_U} \arrow[r] & \shExt{i+1}{\struct{U}}{\F|_U}{\G|_U} \arrow[r] & \shExt{i+1}{\struct{U}}{\struct{U}^{\oplus n}}{\G|_U} 
\end{tikzcd}
\end{center}
but for $i > 0$ we know $\shExt{i}{\struct{U}}{\struct{U}}{\G|_U} = 0$ and thus when $i > 0$,
\[ \shExt{i}{\struct{U}}{\K}{\G|_U} \iso \shExt{i+1}{\struct{X}}{\F}{\G}|_U \]
so by the induction hypothesis we conclude that $\shExt{i+1}{\struct{X}}{\F}{\G}|_U$ is (quasi)-coherent and therefore so is $\shExt{i+1}{\struct{X}}{\F}{\G}$. Finally, for the case $i = 1$ we use the exact sequence,
\begin{center}
\begin{tikzcd}
\G|_U^{\oplus n} \arrow[r] & \shHom{\struct{U}}{\K}{\G|_U} \arrow[r] & \shExt{1}{\struct{X}}{\F}{\G}|_U \arrow[r] & 0
\end{tikzcd}
\end{center}
so $\shExt{1}{\struct{X}}{\F}{\G}|_U$ is the cokernel of (quasi)-coherent sheaves and thus is (quasi)-coherent.

\subsubsection{6.4}

Let $X$ be a noetherian scheme and suppose that $\Coh{X}$ has enough locally frees (i.e. for each $\F \in \Coh{X}$ there exists a locally free $\G \in \Coh{X}$ and a surjection $\G \to \F$ making every coherent sheaf a quotient of a locally free). Then for any $\G \in \shMod{X}$, consider the contravariant $\delta$-functor $\shExt{i}{\struct{X}}{-}{\H} : \Coh{X}^{\op} \to \shMod{X}$. To show that such a functor is universal is suffices to prove that this contravariant $\delta$-functor is coeffaceable (or equivalently is an effaceable $\delta$-functor on the opposite category $\Coh{X}^{\op}$) meaning that for each $\F \in \Coh{X}$ there exists an epimorphism $a : \G \to \F$ such that $\shExt{i}{\struct{X}}{a}{\H} = 0$ for all $i \ge 1$. Since we are given such maps from locally free sheaves $\G \to \F$, it suffices to prove that $\shExt{i}{\struct{X}}{\G}{\H} = 0$ for all $i \ge 1$ and locally free $\G$. 
\bigskip\\
However, we have shown that for locally free coherent $\G$,
\[ \shExt{i}{\struct{X}}{\G}{\H} = \shExt{i}{\struct{X}}{\struct{X}}{\H} \otimes \G^\vee = 0 \]
for $i > 0$ since $\shExt{i}{\struct{X}}{\struct{X}}{\G} = 0$ because $\shHomover{\struct{X}}{\struct{X}}{-}$ is the identity functor. 

\subsubsection{6.5}

\newcommand{\hd}[1]{\mathrm{hd}\left( #1 \right)}
\newcommand{\pd}[2]{\mathrm{pd}_{#1}\left(#2 \right)}

Let $X$ be a noetherian scheme such that $\Coh{X}$ has enough injectives i.e. for any coherent $\struct{X}$-module $\F$ there exists a finite locally free sheaf $\E$ and a surjection $\E \onto \F$. Then we define the homological dimension $\hd{\F}$ to be the length of the shortest locally free resolution of $\F$. 

\begin{enumerate}
\item Suppose $\F$ is locally free then $\shExt{1}{\struct{X}}{\F}{\G} = \shExt{1}{\struct{X}}{\struct{X}}{\G} \otimes \F^\vee = 0$ for $\struct{X}$-module $\G$. Now suppose that $\F$ is coherent and $\shExt{1}{\struct{X}}{\F}{\G} = 0$ for any $\struct{X}$-module $\G$. Take an exact sequence,
\begin{center}
\begin{tikzcd}
0 \arrow[r] & \K \arrow[r] & \E \arrow[r] & \F \arrow[r] & 0
\end{tikzcd}
\end{center}
where $\E$ is finite locally free. Because $\shExt{1}{\struct{X}}{\F}{\K} = 0$, the sequence remains exact after applying $\shHom{\struct{X}}{\F}{-}$ giving an exact sequence,
\begin{center}
\begin{tikzcd}
0 \arrow[r] & \shHom{\struct{X}}{\F}{\K} \arrow[r] & \shHom{\struct{X}}{\F}{\E} \arrow[r] & \shHom{\struct{X}}{\F}{\F} \arrow[r] & 0
\end{tikzcd}
\end{center}
Now $\id \in \Gamma(X, \Hom{\struct{X}}{\F}{\F})$ is a global section so locally at any $x \in X$ there must be an open neighborhood $U$ such that $\id$ is in the image of
\[ \Gamma(X, \shHom{\struct{X}}{\F}{\E}) \to \Gamma(X, \shHom{\struct{X}}{\F}{\F}) \]
In particular, there is a morphism $\varphi : \F|_U \to \E|_U$ such that $\F|_U \to \E|_U \to \F|_U$ is the identity meaning that,
\begin{center}
\begin{tikzcd}
0 \arrow[r] & \K|_U \arrow[r] & \E|_U \arrow[r] & \F|_U \arrow[r] & 0
\end{tikzcd}
\end{center}
is split. Therefore, $\F_x$ is is a finite projective $\stalk{X}{x}$-module and therefore free meaning that $\F$ is locally free because $\F$ is coherent and $X$ is noetherian.

\item If $\hd{\F} \le n$ then we can find a locally free resolution,
\begin{center}
\begin{tikzcd}
0 \arrow[r] & \E_n \arrow[r] & \cdots \arrow[r] & \E_0 \arrow[r] & \F \arrow[r] & 0
\end{tikzcd}
\end{center}
and we can use this resolution to compute ext,
\[ \shExt{i}{\struct{X}}{\F}{\G} = H^i(\shHom{\struct{X}}{\E_\bullet}{\G}) \]
which is therefore zero for $i > n$. Conversely, suppose that $\shExt{i}{\struct{X}}{\F}{\G} = 0$ for any $\struct{X}$-module $\G$ and all $i > n$. If $n = 0$ then $\shExt{1}{\struct{X}}{\F}{\G} = 0$ and thus $\F$ is locally free so $\hd{\F} = 0$. Now we proceed by induction on $n$. Consider a short exact sequence,
\begin{center}
\begin{tikzcd}
0 \arrow[r] & \K \arrow[r] & \E \arrow[r] & \F \arrow[r] & 0
\end{tikzcd}
\end{center}
where $\E$ is finite locally free and thus the kernel $\K$ is coherent. Applying the long exact sequence of $\shHom{\struct{X}}{-}{\G}$ we find,
\begin{center}
\begin{tikzcd}
\shExt{i}{\struct{X}}{\K}{\G} \arrow[r,"\sim"] & \shExt{i+1}{\struct{X}}{\F}{\G} 
\end{tikzcd}
\end{center}
for $i > 0$ because $\shExt{j}{\struct{X}}{\E}{\G} = 0$ since $\E$ is locally free. Therefore, for $i + 1 > n$ we see that,
\[ \shExt{i}{\struct{X}}{\K}{\G} = \shExt{i+1}{\struct{X}}{\F}{\G} = 0 \]
Therefore, by the induction hypothesis $\hd{\K} \le n - 1$ so $\K$ has a length $n-1$ locally free resolution,
\begin{center}
\begin{tikzcd}
0 \arrow[r] & \E_{n} \arrow[r] & \cdots \arrow[r] & \E_1 \arrow[r] & \K \arrow[r] & 0
\end{tikzcd}
\end{center}
and therefore,
\begin{center}
\begin{tikzcd}
0 \arrow[r] & \E_{n} \arrow[r] & \cdots \arrow[r] & \E_1 \arrow[r] & \E \arrow[r] & \F \arrow[r] & 0
\end{tikzcd}
\end{center}
is a length $n$ locally free resolution of $\F$ showing that $\hd{\F} \le n$ and proving the converse by induction.

\item Since the stalks of a locally free resolution of $\F$ gives a free resolution of $\F_x$ clearly,
\[ \hd{\F} \ge \sup_x \pd{\stalk{X}{x}}{\F_x} \]
Now suppose that $n \ge \pd{\stalk{X}{x}}{\F_x}$ for each $x \in X$. We know $\Ext{i}{\stalk{X}{x}}{\F_x}{B} = 0$ for all $i > n$. Furthermore, because $\F$ is coherent, $\shExt{i}{\struct{X}}{\F}{\G}_x = \Ext{i}{\stalk{X}{x}}{\F_x}{\G_x} = 0$ for $i > n$ and thus $\shExt{i}{\struct{X}}{\F}{\G} = 0$ showing that $n \ge \hd{\F}$. Therefore,
\[ \hd{\F} = \sup_x \pd{\stalk{X}{x}}{\F_x} \]
\end{enumerate}

\subsubsection{6.6}

Let $A$ be a regular local ring, and $M$ a finitely generated $A$-module. 

\begin{enumerate}
\item If $M$ is projective then clearly $\Ext{i}{A}{M}{A} = 0$ for all $i > 0$. Conversely, suppose that $\Ext{i}{A}{M}{A} = 0$ for all $i > 0$. Since $A$ is regular, $\pd{A}{M} \le \dim{A} = n$ is finite. We claim that $\Ext{i}{A}{M}{N} = 0$ for any finite $A$-module $N$. The base case, $\Ext{n+1}{A}{M}{N} = 0$ because $\pd{A}{M} \le n$. Now we proceed by descending induction. Consider an exact sequence,
\begin{center}
\begin{tikzcd}
0 \arrow[r] & K \arrow[r] & F \arrow[r] & N \arrow[r] & 0
\end{tikzcd}
\end{center}
where $F$ is finite free. Then the long exact sequence for $\Hom{A}{M}{-}$ gives,
\begin{center}
\begin{tikzcd}
\Ext{i}{A}{M}{F} \arrow[r] & \Ext{i}{A}{M}{N} \arrow[r] & \Ext{i+1}{A}{M}{K} \arrow[r] & \Ext{i+1}{A}{M}{F}
\end{tikzcd}
\end{center}
but $\Ext{i}{A}{M}{F} = 0$ for $i > 0$ by assumption. Therefore, we find $\Ext{i}{A}{M}{N} = \Ext{i+1}{A}{M}{K}$. For induction, we assume that $\Ext{i+1}{A}{M}{N} = 0$ for all finite $A$-module $N$. Since $F$ is finite and $A$ is noetherian, $K$ is finite and thus $\Ext{i}{A}{M}{N} = \Ext{i+1}{A}{M}{K} = 0$ for any finite $A$-module $N$. In particular, taking an exact sequence,
\begin{center}
\begin{tikzcd}
0 \arrow[r] & K \arrow[r] & F \arrow[r] & M \arrow[r] & 0
\end{tikzcd}
\end{center}
with $F$ finite free. Then taking $\Hom{A}{M}{-}$ we get,
\begin{center}
\begin{tikzcd}
0 \arrow[r] & \Hom{A}{M}{K} \arrow[r] & \Hom{A}{M}{F} \arrow[r] & \Hom{A}{M}{M} \arrow[r] & \Ext{1}{A}{M}{K} 
\end{tikzcd}
\end{center}
but $\Ext{1}{A}{M}{K} = 0$ because $K$ is finite so $\Hom{A}{M}{F} \to \Hom{A}{M}{M}$ is surjective. Therefore there is a map $M \to F$ such that $M \to F \to M$ is $\id$ meaning that the sequence,
\begin{center}
\begin{tikzcd}
0 \arrow[r] & K \arrow[r] & F \arrow[r] & M \arrow[l, bend right] \arrow[r] & 0
\end{tikzcd}
\end{center}
is split on the right. Therefore $M \oplus K = F$ so $M$ and $K$ are projective.

\item If $\pd{A}{M} \le n$ then clearly $\Ext{i}{A}{M}{A} = 0$ for $i > n$. Conversely suppose that $\Ext{i}{A}{M}{A} = 0$ for $i > n$. We prove by induction on $n$ that $\pd{A}{M} \le n$. We proved the base case $n = 0$ above. Now assume the theorem for $n$ and let $\Ext{i}{A}{M}{A} = 0$ for $i > n+1$. Take a sequence,
\begin{center}
\begin{tikzcd}
0 \arrow[r] & K \arrow[r] & F \arrow[r] & M \arrow[r] & 0
\end{tikzcd}
\end{center}
where $F$ is finite free. Applying $\Hom{A}{-}{A}$ we get,
\begin{center}
\begin{tikzcd}
\Ext{i}{A}{F}{A} \arrow[r] & \Ext{i}{A}{K}{A} \arrow[r] & \Ext{i+1}{A}{M}{A} \arrow[r] & \Ext{i+1}{A}{F}{A}
\end{tikzcd}
\end{center}
Since $F$ is free $\Ext{i}{A}{F}{A} = 0$ for all $i > 0$ and therefore we get $\Ext{i}{A}{K}{A} = \Ext{i+1}{A}{M}{A}$ for $i > 0$. Since $\Ext{i+1}{A}{M}{A} = 0$ for $i+1 > n+1$ we see that $\Ext{i}{A}{K}{A} = 0$ for $i > n$. Since $K$ is finite, by the induction hypothesis, $\pd{A}{K} \le n$ and thus $\pd{A}{M} \le n + 1$ by concatenating the resolution with the above sequence proving that $\pd{A}{M} \le n \iff \Ext{i}{A}{M}{A} = 0$ for $i > n$ by induction. 
\end{enumerate}

\subsubsection{6.7}

Let $X = \Spec{A}$ be a Noetherian affine scheme and $M, N$ be $A$-modules with $M$ finite. Choose a finite free resolution $F_\bullet \to M \to 0$. Then we get a free resolution, $\wt{F}_\bullet \to \wt{M} \to 0$ and then we can compute,
\begin{align*}
\Ext{i}{\struct{X}}{\wt{M}}{\wt{N}} & = H^i(\Hom{\struct{X}}{\wt{F}_\bullet}{\wt{N}} = H^i(\Hom{A}{F_\bullet}{N}) = \Ext{i}{A}{M}{N} 
\\
\shExt{i}{\struct{X}}{\wt{M}}{\wt{N}} & = H^i(\shHom{\struct{X}}{\wt{F}_\bullet}{\wt{N}} = H^i(\wt{\Hom{A}{F_\bullet}{N}}) = H^i(\Hom{A}{F_\bullet}{N})^\sim = \wt{\Ext{i}{A}{M}{N}} 
\end{align*}
because $\wt{(-)}$ is exact and thus commutes with taking cohomology. 

\subsubsection{6.8 DO!}

Let $X$ be a noetherian integral, separated, locally factiorial scheme.

\begin{enumerate}
\item Let $x \in X$ and $U \subset X$ an open neighborhood of $x$. Then $X \setminus Z$ is closed so decompose $Z$ into irreducible components $Z = Z_1 \cup \cdots \cup Z_r$ which are also closed (and finitely many because $X$ is noetherian). Suppose we can find $s_i \in \Gamma(X, \L_i)$ such that $x \in X_{s_i} \subset X \setminus Z_i$ then $s = s_1 \otimes \cdots \otimes s_r \in \Gamma(X, \L_1 \otimes \cdots \otimes \L_r)$ satisfies $x \in X_s \subset U$. Therefore, it suffices to assume that $Z$ is irreducible. Let $\zeta \in Z$ be the generic point 
\end{enumerate}

\subsubsection{6.9 DO!}

Let $X$ be a noetherian, integral, separated, regular scheme. Let $K_1(X)$ be the group generated by locally free coherent $\struct{X}$-modules with relations $[\E] - [\E_1] - [\E_2]$ for each exact sequence,
\begin{center}
\begin{tikzcd}
0 \arrow[r] & \E_1 \arrow[r] & \E \arrow[r] & \E_2 \arrow[r] & 0
\end{tikzcd}
\end{center}
Consider the homomorphism $\varepsilon : K_1(X) \to K(X)$

\begin{enumerate}
\item Let $\F$ be a coherent $\struct{X}$-module. By the previous exercises, there exists a locally free (coherent) resolution $\E_\bullet \to \F \to 0$ so $\Coh{X}$ has enough locally frees. Furthermore, we know that,
\[ \hd{\F} = \sup_x \pd{\stalk{X}{x}}{\F_x} \le \dim{\stalk{X}{x}} = \dim{X} \]
because $X$ is regular. Therefore, $\F$ must have a locally free resolution,
\begin{center}
\begin{tikzcd}
0 \arrow[r] & \E_r \arrow[r] & \cdots \arrow[r] & \E_0 \arrow[r] & \F \arrow[r] & 0
\end{tikzcd}
\end{center} 
of length at most $n$. Therefore, in $K(X)$,
\[ [\F] = \sum_{i = 0}^r (-1)^i [\E_i] \]
and therefore $[\F]$ is in the image of $\varepsilon : K_1(X) \to K(X)$.

\item Furthermore, for a finite length locally free coherent resolution,
\begin{center}
\begin{tikzcd}
0 \arrow[r] & \E_r \arrow[r] & \cdots \arrow[r] & \E_0 \arrow[r] & \F \arrow[r] & 0
\end{tikzcd}
\end{center} 
we define a class,
\[ \delta([\F]) = \sum_{i = 0}^{r} (-1)^i [\F_i] \]
in $K_1(X)$. We need to show this class does not depend on the choice of resolution. (HOW TO SHOW INDEP OF REPRESENTATION!!)
\end{enumerate}

\subsubsection{6.10 CHECK}

\begin{enumerate}
\item Let $f : X \to Y$ be  a finite morphism of noetherian schemes. Let $\G$ be a quasi-coherent $\struct{Y}$-module. Then $f_* \struct{X}$ is coherent because $f$ is finite. Therefore, $\shHom{\struct{Y}}{f_* \struct{X}}{\G}$ is a quasi-coherent $f_* \struct{X}$-module so because $f : X \to Y$ is affine we get a quasi-coherent $\struct{X}$-module $f^! \G$.
(FINITENESS CONDITION?)

\item Let $\F$ be a coherent $\struct{X}$-module and $\G$ a quasi-coherent $\struct{Y}$-module. We define a morphism,
\[ \alpha : f_* \shHom{\struct{X}}{\F}{f^! \G} \to \shHom{\struct{Y}}{f_* \F}{\G} \]
as follows. First,
\[ f_* \shHom{\struct{X}}{\F}{f^! \G} \to \shHom{\struct{Y}}{f_* \F}{f_* f^! \G} \]
and then $f_* f^! \G = \shHom{\struct{Y}}{f_* \struct{X}}{\G}$ giving a map $f_* f^! \G \to \shHom{\struct{Y}}{\struct{Y}}{\G} = \G$ using $\struct{Y} \to f_* \struct{X}$. Thus we get a map,
\[ f_* \shHom{\struct{X}}{\F}{f^! \G} \to \shHom{\struct{Y}}{f_* \F}{f_* f^! \G} \to \shHom{\struct{Y}}{f_* \F}{\G} \]
Now affine locally on $V \subset Y$ with $V = \Spec{A}$ we have $f^{-1}(V) = U = \Spec{B}$ and $\F|_U = \wt{M}$ and $\G|_U = \wt{N}$. Then the map becomes,
\[ \Hom{B}{M}{\Hom{A}{B}{N}} \to \Hom{A}{M_A}{\Hom{A}{B}{N}} \to \Hom{A}{M_A}{N} \] 
which is an isomorphism.

\item There is a map $\Hom{\struct{X}}{\F}{-} \to \Hom{\struct{Y}}{f_* \F}{f_*-}$. Since $f_*$ is exact $\Ext{i}{\struct{Y}}{f_* \F}{f_* -}$ form a $\delta$-functor and therefore we get compatible natural transformations
\[ \varphi_i : \Ext{i}{\struct{X}}{\F}{-} \to \Ext{i}{\struct{Y}}{f_* \F}{f_* -} \]
because $\Ext{i}{\struct{X}}{\F}{-}$ are a universal $\delta$-functor. Therefore, we get,
\[ \varphi_i : \Ext{i}{\struct{X}}{\F}{f^! \G} \to \Ext{i}{\struct{Y}}{f_* \F}{f_* f^! \G} \to \Ext{i}{\struct{Y}}{f_* \F}{\G} \]

\item Now assume that $X$ and $Y$ are separated (I THINK WE ONLY NEED THIS BECAUSE AFFINE PUSHFORWARD IS ONLY PROVED IN THIS GENERALITY), $\Coh{X}$ has enough locally frees, and $f_* \struct{X}$ is finite locally free on $Y$ (i.e. $f$ is finite flat). Let $\F$ be a coherent $\struct{X}$-module and $\G$ a quasi-coherent $\struct{Y}$-module. For the case $i = 0$,
\[ \Hom{\struct{X}}{\F}{f^! \G} \to \Hom{\struct{Y}}{f_* \F}{\G} \]
is the global sections of the preceeding isomorphism
\[ \alpha : f_* \shHom{\struct{X}}{\F}{f^! \G} \to \shHom{\struct{Y}}{f_* \F}{\G} \]
and thus is an isomorphism (in general). Now for the case $\F = \struct{X}$ we consider the morphism,
\begin{center}
\begin{tikzcd}
\Ext{i}{\struct{X}}{\struct{X}}{f^! \G} \arrow[r, "\varphi_i"] \arrow[d, equals] & \Ext{i}{\struct{Y}}{f_* \struct{X}}{\G} \arrow[d, equals]
\\
H^i(X, f^! \G) \arrow[r] & H^i(Y, \shHom{\struct{Y}}{f_* \struct{X}}{\G})
\end{tikzcd}
\end{center}
using that $f_* \struct{X}$ is locally free. But $\shHom{\struct{X}}{f_* \struct{X}}{\G} = f_* f^! \G$ and therefore the bottom map is an isomorphism because $f$ is affine so $\varphi_i$ is an isomorphism for $\F = \struct{X}$. Likewise, when $\F = \E$ is finite locally free we get a diagram,
\begin{center}
\begin{tikzcd}
\Ext{i}{\struct{X}}{\E}{f^! \G} \arrow[r, "\varphi_i"] \arrow[d, equals] & \Ext{i}{\struct{Y}}{f_* \E}{\G} \arrow[d, equals]
\\
H^i(X, \shHom{\struct{X}}{\E}{f^! \G}) \arrow[r] & H^i(Y, \shHom{\struct{Y}}{f_* \E}{\G})
\end{tikzcd}
\end{center}
because $\E$ and $f_* \E$ are locally free so the local-to-global Ext spectral sequence collapses giving the vertical isomorphisms.
However, we have shown there is an isomorphism,
\[ f_* \shHom{\struct{X}}{\E}{f^! \G} \iso \shHom{\struct{Y}}{f_* \E}{\G} \] 
and therefore the bottom arrow is an isomorphism proving that $\varphi_i$ is an isomorphism for $\F = \E$ finite locally free.
\bigskip\\
Now we proceed by induction on $i$. We have proved the base case $i = 0$ so suppose that $\varphi_i$ is an isomorphism. Now we fit $\F$ into an exact sequence,
\begin{center}
\begin{tikzcd}
0 \arrow[r] & \K \arrow[r] & \E \arrow[r] & \F \arrow[r] & 0
\end{tikzcd}
\end{center}
where $\E$ is locally free. We get a morphism of exact sequences,
\begin{center}
\begin{tikzcd}[column sep = small]
\Ext{i}{\struct{X}}{\E}{f^! \G} \arrow[d, "\varphi_i"] \arrow[r] & \Ext{i}{\struct{X}}{\K}{f^! \G} \arrow[d, "\varphi_i"] \arrow[r] & \Ext{i+1}{\struct{X}}{\F}{f^! \G} \arrow[d, "\varphi_{i+1}"]  \arrow[r] & \Ext{i+1}{\struct{X}}{\E}{f^! \G} \arrow[d, "\varphi_{i+1}"] \arrow[r] & \Ext{i+1}{\struct{X}}{\E}{f^! \G} \arrow[d, "\varphi_{i+1}"] 
\\
\Ext{i}{\struct{Y}}{f_* \E}{\G} \arrow[r] & \Ext{i}{\struct{Y}}{f_* \K}{\G} \arrow[r] & \Ext{i+1}{\struct{Y}}{f_* \F}{\G} \arrow[r] & \Ext{i+1}{\struct{Y}}{f_* \E}{\G} \arrow[r] & \Ext{i+1}{\struct{Y}}{f_* \E}{\G}
\end{tikzcd}
\end{center}
where the bottom row is exact because $f_*$ is exact since it is affine. The second and fourth is an isomorphism because $\E$ is locally free. The first is an isomorphism by the induction hypothesis. Therefore, $\varphi_{i+1}$ is injective for $\F$ by the five lemma. Since $\F$ is a arbitrary coherent $\struct{X}$-module, it applies to $\K$ also making the rightmost arrow injective and thus $\varphi_{i+1}$ for $\F$ is an isomorphism again by the five lemma.
\end{enumerate}

\subsection{Section 7}

\subsubsection{7.1}

Let $X$ be an integral projective scheme of dimension $\dim{X} \ge 1$ over a field $k$. Let $\L$ be an ample invertible sheaf on $X$. Since $\L$ is ample, for sufficiently large $n$ the sheaf $\L^{\otimes n}$ is globally generated and thus,
\[ \dim_k H^0(X, \L^{\otimes n}) \ge \dim{X} + 1 \]
Suppose that $s \in H^0(X, \L^{\otimes -1})$ is a nonzero section. Then under the map,
\[ H^0(X, \L^{\otimes -n}) \otimes H^0(X, \L^{\otimes n}) \to H^0(X, \struct{X}) \]
we can multiply $s^{\otimes n}$ by a nonzero section to get a global section $s' \in H^0(X, \struct{X}) = k$ which is nowhere vanishing and thus $s$ is nowhere vanishing so $s : \struct{X} \to \L^{\otimes -1}$ is an isomorphism contradicting the ampleness of $\L$. Therefore, $H^0(X, \L^{\otimes -1}) = 0$.  

\subsubsection{7.2}

\renewcommand{\tr}{\mathrm{tr}}

Let $f : X \to Y$ be a finite morphism of projective schemes of the same dimension $n$. Let $\omega_Y^\circ$ be a dualizing sheaf for $Y$.

\begin{enumerate}
\item Let $\F$ be a coherent $\struct{X}$-module. From (Ex. 6.10) and the definition of $\omega_Y^\circ$ there are canonical isomorphisms,
\[ \Hom{\struct{X}}{\F}{f^! \omega_Y^\circ} \iso \Hom{\struct{Y}}{f_* \F}{\omega_Y^\circ} = H^n(Y, f_* \F)^\vee = H^n(X, \F)^\vee \]
using the fact that $f$ is affine and $f_* \F$ is coherent. Thus $f^! \omega_Y^\circ$ is a dualizing sheaf for $X$

\item Setting $\omega_X^\circ  = f^! \omega_Y^\circ$ we see that,
\[ \Hom{\struct{X}}{\omega_X^\circ}{f^! \omega_Y^\circ} = \Hom{\struct{Y}}{f_* \omega_X^\circ}{\omega_Y^\circ} \]
and thus $\id \in \Hom{\struct{X}}{\omega_X^\circ}{f^! \omega_Y^\circ}$ corresponds to a canonical trace map $\tr : f_* \omega_X^\circ \to \omega_Y^\circ$. We can describe this as the counit of the adjunction,
\[ \tr : f_* f^! \omega_Y^\circ \to \omega_Y^\circ \]
\bigskip\\
If $X$ and $Y$ are both nonsingular, and $k$ is algebraically closed then the dualizing and canonical sheafs agree $\omega_X^\circ = \omega_X$ and $\omega_Y^\circ = \omega_Y$. Therefore, we immediately get a canonical trace map $\tr : f_* \omega_X \to \omega_Y$.
\end{enumerate}

\renewcommand{\tr}[1]{\mathrm{tr}\left( #1 \right)}

\subsubsection{7.3}

Let $X = \P^n$. Consider the Euler sequence,
\begin{center}
\begin{tikzcd}
0 \arrow[r] & \Omega_{X} \arrow[r] & \struct{X}(-1)^{\oplus (n+1)} \arrow[r] & \struct{X} \arrow[r] & 0 
\end{tikzcd}
\end{center}
Then we may apply exterior powers to get the following sequence,
\begin{center}
\begin{tikzcd}
0 \arrow[r] & \Omega^p_{X} \arrow[r] & \bigwedge^p \struct{X}(-1)^{\oplus (n+1)} \arrow[r] & \struct{X} \otimes_{\struct{X}} \Omega^{p-1}_{X} \arrow[r] & 0
\end{tikzcd}
\end{center}
However, 
\[ \bigwedge^p \struct{X}(-1)^{\oplus (n+1)} = \struct{X}(-p)^{\oplus { n + 1 \choose p }} \] 
and thus we have the sequence,
\begin{center}
\begin{tikzcd}
0 \arrow[r] & \Omega^p_{X} \arrow[r] & \struct{X}(-p)^{\oplus { n + 1 \choose p }} \arrow[r] & \Omega^{p-1}_{X} \arrow[r] & 0
\end{tikzcd}
\end{center}
Now applying the cohomology sequence we find,
\begin{center}
\begin{tikzcd}
H^{q-1}(X, \struct{X}(-p))^{\oplus {n + 1 \choose p }} \arrow[r] & H^{q-1}(X, \Omega_{X}^{p-1}) \arrow[r] & H^{q}(X, \Omega_{X}^p) \arrow[r] & H^{q}(X, \struct{X}(-p))^{\oplus {n + 1 \choose p }}
\end{tikzcd}
\end{center}
Therefore, if $0 < q < n$ and $p > 0$ then $H^{q-1}(X, \Omega^{p-1}_{X}) \iso H^q(X, \Omega^p_{X})$. Furthermore, if $q = 0$ and $p > 0$ then $H^0(X, \Omega^p_{X}) = 0$ because we get an exact sequence,
\begin{center}
\begin{tikzcd}
0 \arrow[r] & H^0(X, \Omega^p_X) \arrow[r] & H^0(X, \struct{X}(-p))^{\oplus {n + 1 \choose p }}
\end{tikzcd}
\end{center}
and $H^0(X, \struct{X}(-p)) = 0$. Finally, if $q = n$ and $p < n + 1$ (which it must) then $H^{q-1}(X, \Omega_X^{p-1}) \iso H^q(X, \Omega_X^p)$ because $H^n(X, \struct{X}(-p)) = H^0(X, \struct{X}(p - n - 1))$ by Serre duality.
\bigskip\\
To finish the base case $p = 0$ we know,
\[ H^q(X, \struct{X}) = 
\begin{cases}
k & q = 0
\\
0 & q > 0
\end{cases} \]
Therefore, by induction, $H^p(X, \Omega_X^p) = H^{p-1}(X, \Omega_X^{p-1}) = k$ for $p \le n$. Furthermore, if $p \neq q$ then reducing via $H^{q-1}(X, \Omega^{p-1}_{X}) \iso H^q(X, \Omega^p_{X})$ we get to either $H^q(X, \Omega^0_X) = 0$ with $q > 0$ or $H^0(X, \Omega^p_X) = 0$ with $p > 0$. Therefore,
\[ H^q(X, \Omega^p_X) = 
\begin{cases}
k & p = q \le n
\\
0 & p \neq q
\end{cases} \]

\subsubsection{7.4 DO!!}

\renewcommand{\tr}{\mathrm{tr}}

Let $X$ be a nonsingular projective variety with $\dim{X} = n$ over an algebraically closed field $k$. Let $Y \subset X$ be a nonsingular subvariety of codimension $p$. The nautral pullback map $\iota^* \Omega_{X}^{n-p} \to \Omega_{Y}^{n-p}$ gives a map $\Omega_{X}^{n - p} \to \iota_* \Omega_{Y}^{n-p}$ by adjunction. Furthermore, $\Omega_{Y}^{n-p} = \omega_Y$ and therefore we get a map,
\[ \varphi_Y : H^{n-p}(X, \Omega_X^{n-p}) \to H^{n-p}(X, \iota_* \omega_Y) = H^{n-p}(Y, \omega_Y) \xrightarrow{\mathrm{tr}_Y} k \]
Therefore, by Serre duality we get an element $\eta(Y) \in H^p(X, \Omega_X^p)$ via,
\[ \eta(Y) = \varphi_Y \in H^{n-p}(X, \Omega^{n-p})^\vee = H^p(X, \Omega^p) \]

\begin{enumerate}
\item Let $Y = P$ for some closed point $P \in X$. Then $p = 0$ so $\varphi_P$ is the map,
\[ H^0(X, \struct{X}) \iso H^0(Y, \struct{X}) \iso k \]
Then Serre duality gives a perfect pairing,
\[ H^0(X, \struct{X}) \times H^n(X, \omega_X) \to k \]
given by $(a, q) \mapsto a \cdot \tr_X(q)$. Since $\eta(P)$ corresponds to $\varphi_P$ under the perfect pairing, we know that $(a, \eta(P)) \mapsto \varphi_P(a) = a$ and therefore $\tr_X(\eta(P)) = 1$. 

\item Let $X = \P^n_k$ and $Y \subset \P^n_k$ a smooth closed subvariety. If $\dim{Y} = 0$ then $Y$ is a finite set of points so, by the previous part, under the identification $H^n(X, \Omega_X^n) \xrightarrow{\tr_X} k$ we see that $\tr_X(\eta(Y)) = \deg{Y}$ where $\deg{Y}$ is equal to the number of points. Now we proceed by induction on $\dim{Y} = n - p$. Given $Y$ with $\dim{Y} = k + 1$ we take a generic hyperplane section $Y \cap H$ to get a subvariety $Y' \subset \P^{n}_k$ of the same degree with $\dim{Y'} = k$. By Bertini's theorem, for a generic hyperplane section the new subvariety $Y'$ will be smooth. 

\item Let $X$ be a scheme of finite type over $k$. There is a homomorphism of sheaves $\d{\log} : \struct{X}^\times \to \Omega_X$ given by $f \mapsto f^{-1} \d{f}$. This is a morphism of abelian sheaves because,
\[\d{\log}(fg) = f^{-1} g^{-1} \d{(fg)} = f^{-1} g^{-1} (g \d{f} + f \d{g} ) = f^{-1} \d{f} + g^{-1} \d{g} = \d{\log}(f) + \d{\log}(g) \]
Therefore we get a map $c_1 : \Pic{X} \to H^1(X, \Omega^1)$ via,
\[ \Pic{X} \iso H^1(X, \struct{X}^\times) \xrightarrow{\d{\log}} H^1(X, \Omega_X^1) \]

\item Let $p = 1$ then $Y \subset X$ is an effective Cartier divisor. We want to show that $\eta(Y) = c_1(\struct{X}(Y))$. We investigate these maps explicitly in terms of Cech cohomology. First we have the map,
\[ \iota^* \Omega^{n-1}_X \to \Omega^{n-1}_Y = \omega_Y = \iota^* \omega_X \otimes \struct{Y}(Y) \]
where the isomorphism $\omega_Y \iso \iota^* \omega_X \otimes \struct{Y}(Y)$
derives from the conormal sequence,
\begin{center}
\begin{tikzcd}
0 \arrow[r] & \struct{Y}(-Y) \arrow[r] & \iota^* \Omega_X \arrow[r] & \Omega_Y \arrow[r] & 0
\end{tikzcd}
\end{center}
Therefore, if we trivialize $\struct{X}(-Y)$ on an affine open cover $U_i$ with generating sections $e_i \in \Gamma(U_i, \struct{X}(-Y)) \subset \Gamma(U_i, \struct{X})$ then locally over $U_i$ the map is explicitly,
\[ \omega \mapsto \iota^* \omega \mapsto \iota^* \omega \otimes \id = \iota^* \omega \otimes e_i \otimes e^i \mapsto (\d{e_i} \wedge \omega) \otimes e^i \]
This is well-defined because if we use $e'_i = f_i e_i$ instead then,
\[ (\d{e_i'} \wedge \omega) \otimes e'^i = (\d{e_i} \wedge \omega) \otimes e^i + (e_i \d{f} \wedge \omega )\otimes e^i = (\d{e_i} \wedge \omega) \otimes e^i \]
because $e_i$ vanishes on $Y$ and thus $(e_i \d{f} \wedge \omega) \otimes e^i = 0$ as sections of $\iota^* \omega_X \otimes \struct{X}(Y)$.
\bigskip\\
Now, by adjunction for dualizing sheaves the trace map $\tr_Y : H^{n-1}(Y, \omega_Y) \to k$ factors through $\tr_X : H^n(X, \omega_X) \to k$ via the connecting map $\delta : H^{n-1}(X, \iota_* \omega_Y) \to H^n(X, \omega_X)$ from the exact sequence,
\begin{center}
\begin{tikzcd}
0 \arrow[r] & \omega_X \arrow[r] & \omega_X(Y) \arrow[r] & \iota_* \omega_Y \arrow[r] & 0
\end{tikzcd}
\end{center}
In terms of Cech cohomology for the cover $\U$, we can describe $\delta$ explicitly for an exact sequence of sheaves,
\begin{center}
\begin{tikzcd}
0 \arrow[r] & \K \arrow[r] & \F \arrow[r, "\varphi"] & \G \arrow[r] & 0
\end{tikzcd}
\end{center} 
Given a cocycle $\alpha \in \check{H}^{q}(\U, \G)$ we compute $\delta(\alpha) \in \check{H}^{q}(\U, \K)$ by first lifting $\alpha$ to a cochain $\tilde{\alpha} \in C^{q}(\U, \F)$ and applying the boundary map to get a cochain $\d{\tilde{\alpha}} \in C^{q+1}(\U, \F)$. However, $\varphi(\d{\tilde{\alpha}}) = \d{\varphi(\tilde{\alpha})} = \d{\alpha} = 0$ because $\alpha$ is a cocycle. Therefore, $\d{\tilde{\alpha}}$ is in the image of $C^{q+1}(\U, \K) \to C^{q+1}(\U, \F)$ and thus defines a cocycle $\delta(\alpha) \in \check{H}^{q+1}(\U, \K)$ because $\d{\delta(\alpha)} = \d^2{\tilde{\alpha}} = 0$ in $\F$. This is independent of the lift since two lifts differ by an element of $C^{q}(\U, \K)$ which maps to zero under $\d$ inside $\check{H}^{q+1}(\U, \K)$. Furthermore, the image of $\delta$ is exactly the kernel of $\check{H}^{q+1}(\U, \K) \to \check{H}^{q+1}(\U, \F)$ as expected because the image of $\delta(\alpha)$ is $\d{\tilde{\alpha}} \in C^{q+1}(\U, \F)$ which is a coboundary and thus zero in $\check{H}^{q+1}(\U, \F)$.
\bigskip\\
Consider $\omega \in \check{H}^{n-1}(X, \Omega^{n-1}_X)$ then we get $\omega' \in \check{H}^{n-1}(X, \iota_* \omega_Y)$ via,
\[ \omega'_{i_0, \dots, i_{n-1}} = (\d{e_{i_0}} \wedge \omega_{i_0, \dots, i_{n-1}}) \otimes e^{i_0} \]
and therefore,
\[ \varphi_Y(\omega) = \tr_X(\delta(\omega')) \]
Furthermore,
\begin{align*}
\delta(\omega')_{i_0, \dots, i_n} & = (\d{e_{i_1}} \wedge \omega_{i_1, \dots, i_{n}}) \otimes e^{i_1} + \left( \d{e_{i_0}} \wedge \sum_{k = 1}^n (-1)^k \omega_{i_0, \dots, \hat{i}_k, \dots, i_n} \right) \otimes e^{i_0}
\end{align*}
Furthermore, because $\omega$ is a cocycle, 
\[ (\d{\omega})_{i_0, \dots, i_n} = \sum_{k = 0}^n (-1)^k \omega_{i_0, \dots, \hat{i}_k, \dots, i_n} = \omega_{i_1, \dots, i_n} + \sum_{k = 1}^n (-1)^k \omega_{i_0, \dots, \hat{i}_k, \dots, i_n} = 0 \]
and therefore,
\begin{align*}
\delta(\omega')_{i_0, \dots, i_n} & = (\d{e_{i_1}} \wedge \omega_{i_1, \dots, i_{n}}) \otimes e^{i_1} - (\d{e_{i_0}} \wedge \omega_{i_1, \dots, i_n}) \otimes e^{i_0} 
\end{align*}
On the intersections, let $e_j = f_{ij} e_i$ for transition functions $f_{ij} \in \struct{U_{ij}}^\times$. Then we find,
\[ \d{e_j} \otimes e^j = \d{e_i} \otimes e^i + e_i f_{ij}^{-1} \d{f_{ij}} \otimes e^i \]
Therefore,
\begin{align*}
\delta(\omega')_{i_0, \dots, i_n} & = ( \d{\log}(f_{i_0 i_1}) \, e_{i_0} \wedge \omega_{i_1, \dots, i_n} ) \otimes e^{i_0} = ( \d{\log}(f_{i_0 i_1})  \wedge \omega_{i_1, \dots, i_n} ) \otimes e_{i_0} \otimes e^{i_0} = \d{\log}(f_{i_0 i_1}) \wedge \omega_{i_1, \dots, i_n}
\end{align*}
as an element of $H^n(X, \omega_X)$ using the identification of the subbundle $\omega_X(-Y) \subset \omega_X$ of forms vanishing on $Y$ (which $\delta(\omega')$ is in because of the $e_{i_0}$) after tensoring by $\struct{X}(Y)$ with the subbundle $\omega_X \subset \omega_X(Y)$ which sends $e_i \otimes e^i$ to $1$ under $\L \otimes \L^\vee \iso \struct{X}$. Therefore,
\[ \varphi_Y(\omega) = \tr_X( \{ \d{\log}(f_{i_0 i_1}) \wedge \omega_{i_1, \dots, i_n} \}_{i_0, \dots, i_n} ) \]
\bigskip\\
Next, consider the perfect pairing,
\[ H^1(X, \Omega_X^1) \times H^{n-1}(X, \Omega_X^{n-1}) \xrightarrow{\smile} H^n(X, \Omega^n) \xrightarrow{\tr_X} k \]
Explicitly on Cech cocycles $\alpha \in \check{H}^p(\U, \F)$ and $\beta \in \check{H}^{q}(\U, \G)$ the cup product,
\[  \check{H}^p(\U, \F) \times \check{H}^q(\U, \G) \xrightarrow{\smile} \check{H}^{p+q}(\U, \F \otimes_{\struct{X}} \G) \] 
acts via,
\[ (\alpha, \beta) \mapsto \alpha \otimes \beta \quad \text{where} \quad (\alpha \otimes \beta)_{i_0, \dots, i_n} = \alpha_{i_0, \dots, i_p} \otimes \beta_{i_p, \dots, i_n} \]
In our situation, we use the isomorphism $\Omega^1_X \otimes_{\struct{X}} \Omega^{n-1}_X \iso \Omega^n_X$ via $\alpha \otimes \omega \mapsto \alpha \wedge \omega$ so that explicitly on Cech cocycles $\alpha \in \check{H}^1(X, \Omega^1_X)$ and $\omega \in \check{H}^{n-1}(X, \Omega^{n-1}_X)$ the perfect pairing is,
\[ (\alpha, \omega) \mapsto \alpha \otimes \omega \mapsto \alpha \wedge \omega \mapsto \tr_X(\alpha \wedge \omega) \quad \text{where} \quad (\alpha \wedge \omega)_{i_0, \dots, i_n} = \alpha_{i_0, i_1} \otimes \omega_{i_1, \dots, i_n} \]
The class $c_1(\struct{X}(Y)) \in H^1(X, \Omega^1_X)$ corresponding to the Cech cocycle $\{ \d{\log}(f_{i_0 i_1}) \}_{i_0,i_1} \in \check{H}^1(X, \Omega^1)$ then has a dual,
\[ \omega \mapsto \tr_X( \{ \d{\log(f_{i_0, i_1})} \wedge \omega_{i_1, \dots, i_n} \}_{i_0, \dots, i_n}) \]
equal to $\varphi_Y(\omega)$. Therefore, the classes $\eta(Y) = c_1(\struct{X}(Y))$ themselves are equal.
\end{enumerate}

\subsection{Section 8}

\subsubsection{8.1}

Let $f : X \to Y$ be a continuous map of topological spaces. Let $\F$ be an abelian sheaf on $X$ and assume that $R^i f_* \F = 0$ for all $i > 0$. Then there are natural isomorphism,
\[ H^i(X, \F) = H^i(Y, f_* \F) \]
which follows immediately from the Leray spectral sequence and is done explicitly in exercises for Johan's class. Basically, this implies that $f_*$ preserves exactness of an injective resolution of $\F$ and also $f_*$ preserves injectives proving the proposition.

\subsubsection{8.2}

Let $f : X \to Y$ be an affine morphism of schemes with $X$ noetherian. Let $\F$ be a quasi-coherent sheaf on $X$. Then $R^i f_* \F$ is the sheafification of $V \mapsto H^i(f^{-1}(V), \F|_{f^{-1}(V)})$ but $f^{-1}(V)$ is a noetherian affine scheme and $\F |_{f^{-1}(V)}$ is quasi-coherent so $H^i(f^{-1}(V), \F|_{f^{-1}(V)}) = 0$ proving that $R^i f_* \F = 0$. Therefore, $H^i(X, \F) = H^i(Y, f_* \F)$.

\subsubsection{8.3}

Let $f : X \to Y$ be a morphism of ringed spaces. Let $\F$ be a $\struct{X}$-module, and $\E$ be a finite locally free $\struct{Y}$-module. We know there is a natural isomorphism,
\[ f_* (\F \otimes_{\struct{X}} f^* \E) \iso f_* \F \otimes_{\struct{Y}} \E \]
Since $\E$ is finite locally free $- \otimes_{\struct{Y}} \E$ is exact and therefore $R^i f_* (\F) \otimes_{\struct{Y}} \E$ form a $\delta$-functor. Furthermore, it is effacable because $R^i f_* \I = 0$ for injective $\struct{X}$-modules $\I$ and thus $R^i f_* (-) \otimes_{\struct{Y}} \E$ is a universal $\delta$-functor. Therefore, the degree zero isomorphism canonically extends to a  natural isomorphism of $\delta$-functors,
\[ R^i f_* (\F \otimes_{\struct{X}} f^* \E) \iso R^i f_* \F \otimes_{\struct{Y}} \E \]

\subsubsection{8.4 DO!!}

Let $Y$ be a noetherian scheme and $\E$ a locally free $\struct{Y}$-module of rank $n + 1$ with $n \ge 1$. Let $X = \P(\E)$ and $\pi : X \to Y$ and $\struct{X}(1)$ be the anti-tautological line bundle.

\begin{enumerate}
\item 
\end{enumerate}

\subsection{Section 9}

\subsection{Section 10}


\subsubsection{10.1}

Let $k_0$ be a field of characteristic $p > 2$ and $k = k_0(t)$. Consider the plane curve,
\[ X = \Spec{k[x,y]/(y^2 - x^p + t)} \]
and let $f : X \to \Spec{k}$ be the structure map. Let $A = k[x,y]/(y^2 - x^p + t)$ and consider the naive cotangent complex,
\begin{center}
\begin{tikzcd}
0 \arrow[r] & A \arrow[r, "2 y \d{y}"] & A \d{x} \oplus A \d{y} \arrow[r] & 0
\end{tikzcd}
\end{center}
which is quasi-isomorphic to $\Omega_{A/k} \cong A \d{x} \oplus A/(2y) \d{y}$ which is not projective because it jumps rank at the origin $\p = (y)$. Notice that $\p = (y)$ is maximal because,
\[ A/\p = k[x]/(x^p - t) = k(t^{\frac{1}{p}}) \]
is a field. Therefore $f$ is smooth everywhere except for $\p$ and thus regular everywhere except maybe at $\p$. However, $\dim{X} = 1$ and $\p$ is principal already so clearly $A_\p$ is regular. Therefore $X$ is regular but $X \to \Spec{k}$ is not smooth at $\p$. 
\bigskip\\
Notice futhermore that after base changing to $k(t^{\frac{1}{p}})$ that $A$ is no longer regular. Consider,
\[ A' = A \otimes_k k(t^{\frac{1}{p}}) = k(t^{\frac{1}{p}})[x, y]/(y^2 - x^p + t) \]
which has maximal ideal $\p' = (y, x - t^{\frac{1}{p}})$. However, $A'_{\p'}$ is not regular because the relation,
\[ y \cdot y = (x - t^{\frac{1}{p}})(x - t^{\frac{1}{p}})^{p - 1} \]
involves only elements of the maximal ideal and thus cannot be inverted to isolate either generator.

\subsubsection{10.2}

Let $f : X \to Y$ be a proper, flat morphism of varities over $k$. For $y \in Y$ suppose that $X_y \to \Spec{\kappa(y)}$ is smooth and therefore $f$ is smooth at each point $x \in X_y$. Consider the smooth locus $S \subset X$ of $f$ which is open because smoothness is an open property. Since $f$ is proper we see that $f(X \setminus S)$ is closed so consider the open set $U = Y \setminus f(X \setminus S)$. Furthermore, since $f^{-1}(y) \subset S$ we know that $y \notin f(X \setminus S)$ because $f^{-1}(y) \subset S$ so is disjoint from $X \setminus S$ and thus $y \in U$. Then consider $f : f^{-1}(U) \to U$. If $x \in f^{-1}(U)$ then $f(x) \in U$ meaning that $f^{-1}(f(x))$ is disjoint from $X \setminus S$ so $X_{f(x)}$ is smooth. Therefore $f : f^{-1}(U) \to U$ is smooth.

\subsubsection{10.3 (DO!!)}

\subsubsection{10.4 (DO!!)}

\subsubsection{10.5}

Let $X$ be a scheme and $\F$ a coherent $\struct{X}$-module. If $\F$ is locally free then for each $x \in X$ there is a Zariski neighborhood $U$ of $x$ such that $\F|_U$ is a free $\struct{U}$-module and $U \to X$ is clearly an \etale neighborhood. 
\bigskip\\
Conversely, suppose that $\F$ is \etale locally free. Since we need to show that $\F$ is Zariski locally free, it suffices to shrink the target to the image of $U \to X$ so we consider the case that $f : U \to X$ is a surjective \etale cover of affine schemes. In particular, $f$ is faithfully flat. Therefore, we reduce to the situation that $A \to B$ is an \etale (and thus faithfully flat) ring map and $M$ is a finite $A$-module such that $M \otimes_A B$ is free. Thus, $M \otimes_A B$ is finite projective. Since $A \to B$ is faithfully flat we see that $M$ is finite projective and thus locally free by Tag 058R.


\subsubsection{10.6 DO!!}

Consider $A = k[x,y]/(y^2 - x^2(x + 1))$. Then the normalization in $K = \Frac{A}$ is,
\[ A' = k[x,y,\tfrac{y}{x}]/(Y^2 - x^2(x + 1)) = k[t] \]
where $t = \frac{y}{x}$ under $x = t^2 - 1$ and $y = t(t^2 - 1)$. Now consider two copies of $\A^1$ glued at $\pm 1$ as follows,
\[ B = \{ (p, q) \in k[t_1] \times k[t_2] \mid p(1) = q(-1) \text{ and } p(-1) = q(1) \} \subset k[t_1] \times k[t_2] \]

\subsection{Section 11}


\subsubsection{11.1}

\subsubsection{11.2}

Let $f : X \to Y$ be a projective morphism with finite fibers. Then the Stein factorization $f = g \circ f'$ where $g$ is finite and $f'$ has connected fibers. Therefore, it suffices to show that $f' : X \to Y'$ is finite where $f'$ is a projective morphism with finite and connected fibers and $f'_* \struct{X} = \struct{Y'}$.
\bigskip\\
Let $f : X \to Y$ be a projective morphism with finite fibers $f_* \struct{X} = \struct{Y}$ and thus connected fibers.
It is clear that $f$ has isolated fibers is injective. I claim that $f'$ is a closed and open immersion onto a component of $Y'$ and thus is finite. For each $x \in X$ any an open neighborhood $U \subset X$ we know $f(X \setminus U)$ is closed. Let $V = Y$


\subsubsection{11.3}

\subsubsection{11.4}

Let $T$ be an irreducible curve of finite type over $k$ and $f : X \to T$ a flat family of subschemes of $\P^n_k$ meaning there is a closed immersion $X \embed \P^n_k \times_k T$ over $T$. Supppose that $U \subset T$ is a nonempty open such that for each closed point $t \in U$ the fiber $X_t \subset \P^n_{\kappa(t)}$ is connected. Let $\nu : \tilde{T} \to T$ be the normalization and consider the base change,
\begin{center}
\begin{tikzcd}
\tilde{X} \arrow[r] \arrow[d,"\tilde{f}"'] & X \arrow[d, "f"]
\\
\tilde{T} \arrow[r, "\nu"] & T
\end{tikzcd}
\end{center}
and 


\subsection{Section 12}

\subsubsection{12.1}

Let $X$ be a scheme of finite type over $k$ with $k$ algebraically closed. Then the closed points of $X$ coincide with $X(k)$ at which points $x \in X$ we have $k \to \stalk{X}{x} \to \kappa(x)$ is an isomorphism. Therefore, at closed points,
\[ \m_x / \m_x^2 \cong \Omega_{\stalk{X}{x}/k} \otimes_{\stalk{X}{x}} k = (\Omega_X)_x \otimes_{\stalk{X}{x}} \kappa(x) \]
showing that,
\[ \varphi(x) = \dim_k \m_x / \m_x^2 = \dim_k(\Omega_X)_x \otimes_{\stalk{X}{x}} \kappa(x) = \rank_x{(\Omega_X)} \]
Furthermore, since $X$ is finite type over $k$ we know that $\Omega_X$ is coherent and thus the rank function is upper-semicontinuous so $\varphi(x)$ is upper-semicontinuous on the closed points.

\subsubsection{12.2}

Consider $X \subset \P^n_T$ a family of hypersurfaces of constant degree $d$ i.e. $\pi : X \to T$ is flat and for each $t \in T$ the fiber $X_t$ is a hypersurface of degree $d$ in $\P^n_{\kappa(t)}$. Now, a hypersurface $H \subset \P^n_k$ of degree $d$ is Cartier giving an exact sequence,
\begin{center}
\begin{tikzcd}
0 \arrow[r] & \struct{\P^n}(-d) \arrow[r] & \struct{\P^n} \arrow[r] & \struct{H} \arrow[r] & 0
\end{tikzcd}
\end{center}
taking cohomology gives exact sequences,
\begin{center}
\begin{tikzcd}
0 \arrow[r] & k \arrow[r] & H^0(H, \struct{H}) \arrow[r] & 0
\\
0 \arrow[r] & H^i(H, \struct{H}) \arrow[r] & H^{i+1}(\P^n, \struct{\P^n}(-d)) \arrow[r] & 0
\end{tikzcd}
\end{center}
Therefore,
\[ \dim_k H^{q}(H, \struct{H}) = \begin{cases}
1 & q = 0
\\
{d - 1 \choose n } & q = n-1
\\
0 & \text{else}
\end{cases} \] 
this follows independently on the choice of hypersurface and thus,
\[ \dim_k H^i(X_t, \struct{X_t}) \]
is constant.

\subsubsection{12.3 DO!!}


\subsubsection{12.4}

Let $Y$ be an integral scheme of finite type over an algebraically closed field $k$. Let $f : X \to Y$ be a flat projective morphism whose fibers are integral. Let $\L$ and $\M$ be invertible sheaves on $X$ such that for each $y \in Y$ we have $\L|_{X_y} \cong \M|_{X_y}$. 
\bigskip\\
Consider the cohernet sheaf $\F = \L \otimes \M^{\otimes -1}$ of $\struct{X}$-modules then $f_* \F$ is a coherent $\struct{Y}$-module since $f$ is proper. Because $\L$ and $\M$ are locally free and $f$ is flat, $\F$ is flat over $Y$. Now consider,
\[ \varphi^0(y) = \dim_{\kappa(y)} H^0(X_y, \F|_{X_y}) \]
Because $\L|_{X_y} \cong \M|_{X_y}$ we see that $\F|_{X_y} \cong \struct{X_y}$. 
Therefore, if $y \in Y$ is a closed point then,
\[ H^0(X_y, \F|_{X_y}) = H^0(X_y, \struct{X_y}) = k \] because $\kappa(y) = k$ ($Y$ is finite type over $k$ and $\bar{k} = k$)and $X_y$ is integral and projective over $\Spec{k}$ (and $k$ is algebraically closed). Therefore, $\varphi^0(y) = 1$ at closed points. By semicontinuity, 
\[ \{ y \in Y \mid \varphi^0(y) = 0 \} \]
is open but closed points are dense so $\varphi^0(y) \ge 1$. Furthermore, 
\[ \{ y \in Y \mid \varphi^0(y) > 1 \} \]
is closed so must contain a closed point and thus is empty. Thus $\varphi^0(y) = 1$ for all $y$. Since $Y$ is integral, by Grauert, $f_* \F$ is locally free of rank $\dim_k H^0(X_y, \struct{X_y}) = 1$. Let $\sN = f_* \F$. Then $f^* f_* \F \to \F$ is an isomorphism because it is on each fiber $X_y$ and thus on stalks. Therefore,
\[ \L \cong \M \otimes f^* \sN \]

\subsubsection{12.5 CHECK Ex 7.9}

Let $Y$ be an integral scheme of finite type over an algebraically closed field $k$. Let $\E$ be a locally free sheaf on $Y$, and let $X = \P_X(\E)$. 
\bigskip\\
Consider the map $\Phi : \Pic{Y} \oplus \Z \to \Pic{X}$ via $(\L, n) \mapsto \L \otimes \struct{X}(n)$ of Ex. 7.9 which we showed is injective generally.
\bigskip
Now $\pi : X \to Y$ is a flat projective morphism whose fibers are $\P^{r-1}_Y$ which are integral (since the product of $k$-varieties is a variety). Let $\L$ be an invertible sheaf on $X$ then the fiber $X_y = \P^{r-1}_{\kappa(y)}$ for any $y \in Y$ and thus $\L|_{X_y} = \struct{\P^{r-1}_{\kappa(y)}}(n_y)$ for some $n_y \in \Z$. However, since $\pi$ is flat, the hilbert polynomial $\chi(X_y, \L|_{X_y})$ is constant and thus $n_y = n$ is constant. Therefore, consider $\L$ and $\struct{X}(n)$ which are isomorphic on fibers $X_y$. Applying the previous problem, $\L = \pi^* \sN \otimes \struct{X}(n)$ for some invertible $\struct{Y}$-module $\sN$ showing that $\Phi$ is surjective.

\subsubsection{12.6 FINISH}

Let $X$ be an integral projective scheme over an algebraically closed field $k$ with $H^1(X, \struct{X}) = 0$. Let $T$ be a connected scheme finite type over $k$.

\begin{enumerate}
\item Let $\L$ be invertible on $X \times T$. Fix a closed point $t_0 \in T$ and let $\L_0 = \iota_0^* \L$ where $\iota : X = X \times \{t_0\} \embed X \times T$. Then for any closed point $t \in T$ let $\L_t = \iota^*_t \L$ where $\iota_t : X = X \times \{ t \} \embed X \times T$.
\bigskip\\
Consider $\pi : X \times T \to T$ which is flat, projective, with fibers $\pi^{-1}(t) = X$ which are integral. Let $\F = \L \otimes (\pi_X^* \L_0)^{\otimes -1}$ then $\F_0 = \struct{X}$. Consider,
\[ U = \{ t \in T^{\text{closed}} \mid \F_t \cong \struct{X} \} \subset T^{\text{closed}} \]
If $t \in U$ then,
\[ \varphi^i(t) : (R^i \pi_* \F)_t \otimes \kappa(t) \to H^i(X, \F_t) = H^i(X, \struct{X})  \]
Since $H^1(X, \struct{X}) = 0$ we see that if $t \in U$ and $i = 1$ then $\varphi^1(t)$ is surjective and thus an isomorphism so $(R^1 \pi_* \F)|_U = 0$ and thus $(R^1 \pi_* \F)$ is locally free at each $t \in U$ which implies that $\varphi^0$ is surjective as well. Therefore, $\varphi^0$ is an isomorphism so by cohomology and base change $\pi_* \F$ is locally free on $U$ which proves that $U$ is open. Now consider the complement,
\[ V = \{ t \in T^{\text{closed}} \mid \F_t \not\cong \struct{X} \} \subset T^{\text{closed}} \]
\end{enumerate}

\section{IV}

\begin{definition}
Here a curve is a regular integral scheme of dimension one which is finite type over an algebraically closed field $K$.
\end{definition}

\subsection{1}

\subsubsection{1.1}

Let $C$ be a curve of genus $g$ and a point $P \in C$. For $g = 0$ we know $C \cong \P^1$ in which case the desired functions are easily constructed. Thus we may assume $C$ has positive genus. Consider the divisor $(1 + g)[P]$ and the line bundle $\struct{C}((1 + g)[P])$. Then by Riemmann-Roch,
\[ \ell((1 + g)[P]) - \ell(K - (1 + g)[P]) = 1 - g + \deg{(1 + g) [P]} = 2 \]
Furthermore,
\[ \deg{(K - (1+g)[P])} = \deg{K} - \deg{(1 + g)[P]} = 2 - 2g - (1 + g) = 1 - 3 g < 0 \]
Therefore, $\ell(K - (1 + g)[P]) = 0$ so we find,
\[ \ell((1 + g)[P]) = 2 \]
and thus there must be nonconstant functions $f \in K(C)$ which are regular everywhere but $P$. 

\subsubsection{1.2}

Let $C$ be a curve and $P_1, \dots, P_n \in C$ points then using the above construction, we get a nonconstant function $f_i \in K(X)$ which has a pole of order $2$ at $P_i$ and is regular elsewhere. Then take $f = f_1^{e_1} \dots f_n^{e_n}$ has poles only at the points $P_1, \dots, P_n$ but it may not have a pole at each point if the $f_i$ have higher order zeros. There is a matrix $v_{ij} = \mathrm{ord}_{P_i} f_j$ which has $v_{ii} = -2$ and $0 \le v_{ij} \le 2$ for $i \neq j$ since $\deg{f_i} = 0$. We need to chose the vector $e_i$ sucht that $v \cdot e$ has negative entries. 

\subsection{2}

\subsection{3}

\subsubsection{3.1}

Let $X$ be a curve of genus 2.

\subsubsection{3.2}

\subsubsection{3.3}

Let $\iota : X \embed \P^n_k$ be a smooth curve of genus $g \ge 2$ embedded as a complete intersection. Then,
\[ \omega_X = \iota^* \struct{\P}(d_1 + \cdots + d_r - n - 1) \]
Since $\deg{\omega_X} = 2 g - 2 > 0$ we know that $\ell = (d_1 + \cdots + d_r - n - 1) > 0$. Therefore, under the Veronese embedding $v : \P^n_k \to \P^N_k$ of degree $\ell$ which is defined by $v^* \struct{\P^{N}}(1) = \struct{\P^n}(\ell)$. Therefore, embedding $v \circ \iota : X \to \P^N$ gives $\omega_X = (v \circ \iota)^* \struct{\P}(1)$ so $\omega_X$ is very ample. 



\section{V}

\section{Lemmata}


\begin{lemma} \label{open_in_irreducible_is_dense}
In an irreducible topological space every nonempty open set is dense.
\end{lemma}

\begin{proof}
Let $U \subset X$ be open with $X$ irreducible. Then take any closed set $C \supset U$. Then $C \cup U^C = X$ since if $x \notin U$ then $x \in U \subset C$. Therefore, since $X$ is irreducible either $U = \varnothing$ or $C = X$. If $U$ is nonempty then we must have $\overline{U} = X$.  
\end{proof}

\begin{lemma} \label{open_of_irreducible}
Let $X$ be an irreducible topological space and nonempty open $U \subset X$. Then $U$ is irreducible.
\end{lemma}

\begin{proof}
Suppose that there were closed sets $Z_1, Z_2 \subset X$ such that
 \[ (Z_1 \cap U) \cup (Z_1 \cap U) = U \]
 i.e. such that $Z_1 \cup Z_2 \supset U$. However, by Lemma \ref{open_in_irreducible_is_dense}, we have $Z_1 \cup Z_2 = X$ since $U$ is a nonempty open and $X$ is irreducible. Therefore, either $Z_1 = X$ or $Z_2 = X$ implying that $Z_1 \cap U = U$ or $Z_2 \cap U = U$ so $U$ is irreducible.   
\end{proof}

\begin{lemma} \label{irreducible_implies_connected}
Any irreducible topological space is connected.
\end{lemma}

\begin{proof}
Suppose that $X$ is irreducible. Suppose that $U \subset X$ is clopen. Then $U$ and $U^C$ are both closed but $U \cup U^C = X$ so either $U = X$ or $U = \varnothing$ proving that $X$ is connected. 
\end{proof}



\begin{lemma} \label{alt_sum_exact}
Consider the exact sequence of finite-dimensional $k$-vectorspaces,
\begin{center}
\begin{tikzcd}
0 \arrow[r] & V_0 \arrow[r, "T_0"] & V_1 \arrow[r, "T_1"] & V_2  \arrow[r] & \cdots \arrow[r] & V_n \arrow[r, "T_n"] & 0 
\end{tikzcd} 
\end{center}
Then we have the alternating sum,
\[ \sum_{i = 0}^n (-1)^i \dim_k V_i = 0 \]
\end{lemma}

\begin{proof}
The rank-nullty theorem gives,
\[ \dim_k{V_i} = \dim_k{\ker{T_i}} + \dim_k{\Im{T_i}} \]
However, by exactness, $\Im{T_i} = \ker{T_{i+1}}$ so consider,
\begin{align*}
\sum_{i = 0}^n (-1)^i \dim_k V_i & = \sum_{i = 0}^n (-1)^i \left[ \dim_k{\ker{T_i}} + \dim_k{\ker{T_{i + 1}}} \right]
\\
& = \sum_{i = 0}^n (-1)^i  \dim_k{\ker{T_i}} + \sum_{i = 0}^{n} (-1)^i  \dim_k{\ker{T_{i+1}}} 
\\
& = \sum_{i = 0}^n (-1)^i  \dim_k{\ker{T_i}} - \sum_{i = 1}^{n+1} (-1)^i  \dim_k{\ker{T_{i}}}
\\
& = \dim_k{\ker{T_0}} - (-1)^{n+1} \ker{T_{n+1}}
\end{align*}
However, $T_0$ is injective and $T_{n+1}$ is the map $0 \to 0$ so both kernels vanish. Therefore,
\[ \sum_{i = 0}^n (-1)^i \dim_k V_i = 0 \]
\end{proof}

\begin{definition}
We say a scheme is \textit{locally noetherian} if there is an open affine cover $U_i = \Spec{A_i}$ by spectra of noetherian rings $A_i$. Furthermore we say a scheme is \textit{noetherian} if it is locally noetherian and quasi-compact.
\end{definition}

\begin{definition}
A morphism $f : X \to Y$ of schemes is \textit{locally of finite type} if on each open $U \subset Y$ the ring map $f^\# : \struct{Y}(U) \to \struct{X}(f^{-1}(U))$ is finite type. Furthermore a the morphism $f :  X \to Y$ is \textit{finite type} if it is locally finite type and quasi-compact.
\end{definition}

\begin{lemma}
Let $f : X \to Y$ be a finite-type morphism of schemes and $Y$ noetherian. Then $X$ is noetherian.
\end{lemma}

\begin{proof}
Since $Y$ is noetherian, it is quasi-compact and has an open affine cover by spectra of noetherian rings $U_i = \Spec{A_i}$. Since $f$ is a finite-type morphism $f$ is quasi-compact so $X = f^{-1}(Y)$ is quasi-compact. Furthermore, the ring map $f^\# : \struct{Y}(U_i) \to \struct{X}(f^{-1}(U_i))$ is finite-type meaning that $\struct{X}(f^{-1}(U_i))$ is a finitely-generated $A_i$-algebra since $\struct{Y}(U_i) = A_i$. Since $A_i$ is noetherian and there is a surjection, \begin{center}
\begin{tikzcd}
A_i[x_1, \dots, x_n] \arrow[r, two heads] & \struct{X}(f^{-1}(U_i))
\end{tikzcd}
\end{center}
then by Hilbert's basis theorem $A_i[x_1, \dots, x_n]$ is noetherian and thus so is $\struct{X}(f^{-1}(U_i))$ proving that $X$ is a noetherian scheme. 
\end{proof}

\begin{corollary}
Any variety is noetherian.
\end{corollary}

\begin{proof}
By definition, a variety $X$ is a finite type scheme over $k$ i.e. the morphism $X \to \Spec{k}$ is finite type. However, $\Spec{k}$ is clearly noetherian thus so is $X$.
\end{proof}

\begin{lemma} \label{tensor_inverse}
Let $R$ be a local ring and let $M, N$ be $R$-modules such that $M \otimes_R N = R$ then $M \cong R$ and $N \cong R$.
\end{lemma}

\begin{proof}

\end{proof}

\begin{lemma} \label{projective_scheme_proper}
Any projective scheme over $k$ is proper over $k$.
\end{lemma}

\begin{proof}

\end{proof}

\begin{lemma} \label{global_sections_proper_scheme}
Let $X$ be an integral scheme proper over $k$. Then $H^0(X, \struct{X})$ is a field which is a finite extension of $k$. 
\end{lemma}

\begin{proof}
The sheaf $\struct{X}$ is coherent and $X$ is a proper scheme over $k$ so $H^0(X, \struct{X})$ is a finite-dimensional $k$-vectorspace. Furthermore, $X$ is integral so $H^0(X, \struct{X})$ is a field and thus a finite field extension of $k$.
\end{proof}

\begin{lemma}
Let $B$ be an $A$-algebra giving $f : \Spec{B} \to \Spec{A}$. Then as quasi-coherent $\struct{\Spec{A}}$-modules,
 \[ f_* \struct{\Spec{B}} = \widetilde{B} \]
\end{lemma}

\begin{proof}
Denote the algebra map $\iota : A \to B$ and $f = \Spec{\iota}$. We have, 
\[ f_* \struct{\Spec{B}}(D(x)) = \struct{\Spec{B}}(f^{-1}(D(x))) \] However, 
\[ f(\p) = \iota^{-1}(\p) \quad \text{thus} \quad x \in \iota^{-1}(\p) \iff \iota(x) \in \p \quad \text{i.e.} \quad f(\p) \in D(x) \iff \p \in D(\iota(x)) \] Thus $f^{-1}(D(x)) = D(\iota(x))$ so,
\[ f_* \struct{\Spec{B}}(D(x)) = \struct{\Spec{B}}(D(\iota(x))) = B_{\iota(x)}  = \widetilde{B}(D(x)) \]
since localizing $B$ at $x$ as an $A$-module is the same as localizing $B$ at $\iota(x)$ as a ring.
\end{proof}

\begin{lemma} \label{lem:exact_seq_of_functors}
Let $F, G, H : \mathcal{A} \to \mathcal{B}$ be additive functors between abelian categories and let $\mathcal{A}$ have enough injectives. Suppose there exists a sequence of natural transformations $F \xrightarrow{\alpha} G \xrightarrow{\beta} H$ such that for each injective object $I \in \mathcal{A}$ that the sequence,
\begin{center}
\begin{tikzcd}
0 \arrow[r] & F(I) \arrow[r, "\alpha_I"] & G(I) \arrow[r, "\beta_I"] & H(I) \arrow[r] & 0  
\end{tikzcd}
\end{center}
is exact. Then for any object $A \in \mathcal{A}$ there exists a long exact cohomology sequence relating the right-derived functors,
\begin{center}
\begin{tikzcd}[column sep = small]
0 \arrow[r] & R^0 F(A) \arrow[r] & R^0 G(A) \arrow[r] & R^0 H(A) \arrow[r] & R^1 F(A) \arrow[r] & R^1 G(A) \arrow[draw=none]{d}[name=Z, shape=coordinate]{} \arrow[r] & R^1 H(A)
\arrow[dlllll,
rounded corners, crossing over,
to path={ -- ([xshift=2ex]\tikztostart.east)
|- (Z) [near end]\tikztonodes
-| ([xshift=-2ex]\tikztotarget.west)
-- (\tikztotarget)}]
\\ 
& R^2 F(A) \arrow[r] & R^2 G(A) \arrow[r] & R^2 H(A) \arrow[r] & R^3 F(A) \arrow[r] & R^3 G(A) \arrow[r] & R^3 H(A) \arrow[r] & \cdots
\end{tikzcd}
\end{center}
\end{lemma}

\begin{proof}
Consider an injective resolution of $A$,
\begin{center}
\begin{tikzcd}
0 \arrow[r] & A \arrow[r] & I^0 \arrow[r] & I^1 \arrow[r] & I^2 \arrow[r] & I^3 \arrow[r] & \cdots
\end{tikzcd}
\end{center}
Now consider the complex,
\begin{center}
\begin{tikzcd}
& 0 \arrow[d] & 0 \arrow[d] & 0 \arrow[d] 
\\
& F(A) \arrow[r, "\alpha_A"] \arrow[d] & G(A) \arrow[r, "\beta_A"] \arrow[d] & H(A) \arrow[d] 
\\
0 \arrow[r] & F(I^0) \arrow[d] \arrow[r, "\alpha_I"] & G(I^0) \arrow[d] \arrow[r, "\beta_I"] & H(I^0) \arrow[d] \arrow[r] & 0  
\\
0 \arrow[r] & F(I^1) \arrow[d] \arrow[r, "\alpha_I"] & G(I^1) \arrow[d] \arrow[r, "\beta_I"] & H(I^1) \arrow[d] \arrow[r] & 0  
\\
0 \arrow[r] & F(I^2) \arrow[d] \arrow[r, "\alpha_I"] & G(I^2) \arrow[d] \arrow[r, "\beta_I"] & H(I^2) \arrow[d] \arrow[r] & 0  
\\
& \vdots & \vdots & \vdots
\end{tikzcd}
\end{center}
with (except for the first) exact rows. Thus, this is an exact sequence of complexes,
\begin{center}
\begin{tikzcd}
0 \arrow[r] & F(\bf{I}^\bullet) \arrow[r, "\alpha"] & G(\bf{I}^\bullet) \arrow[r, "\beta"] & H(\bf{I}^\bullet) \arrow[r] & 0
\end{tikzcd}
\end{center}
which gives rise to an exact sequence of cohomology coinciding with the required sequence since $R^p F(A) = H^p(F(\bf{I}^\bullet)$. 
\end{proof}

\begin{lemma} \label{lem:flasque_mayer_vietoris}
Let $\F$ be a flasque sheaf on $X$ and $U, V \subset X$ be open sets. Then the following sequence,
\begin{center}
\begin{tikzcd}
0 \arrow[r] & \Gamma(U \cup V, \F) \arrow[r] & \Gamma(U, \F) \oplus \Gamma(V, \F) \arrow[r] & \Gamma(U \cap V, \F) \arrow[r] & 0
\end{tikzcd}
\end{center}
with maps $s \mapsto (s|_U, s|_V)$ and $(s, t) \mapsto (s - t)|_{U \cap V}$ is exact.
\end{lemma}

\begin{proof}
The first map is the kernel of the second by the sheaf property of $\F$ i.e. the pair of sections $(s, t)$ is the image of a global section exactly when then agree on the overlap i.e. $s|_{U \cap V} = t|_{U \cap V} \iff (s - t)|_{U \cap V} = 0$. Finally, the map sending $(s, t) \mapsto (s - t)|_{U \cap V}$ is surjective because $\F$ is flasque so the restriction map $(s, 0) \mapsto s|_{U \cap V}$ is surjective. 
\end{proof}

\begin{theorem}[Mayer-Vietoris]
Let $\F$ be a sheaf on $X$ and $U, V \subset X$ be open sets. Then there is a long-exact sequence of cohomology, 
\begin{center}
\begin{tikzcd}
0 \arrow[r] & H^0(U \cup V, \F) \arrow[r] & H^0(U, \F) \oplus H^0(V, \F) \arrow[draw=none]{d}[name=Z, shape=coordinate]{} \arrow[r] & H^0(U \cap V, \F)
\arrow[dll,
rounded corners, crossing over,
to path={ -- ([xshift=2ex]\tikztostart.east)
|- (Z) [near end]\tikztonodes
-| ([xshift=-2ex]\tikztotarget.west)
-- (\tikztotarget)}]
\\ 
& H^1(U \cup V, \F) \arrow[r] & H^1(U, \F) \oplus H^1(V, \F) \arrow[draw=none]{d}[name=Z', shape=coordinate]{} \arrow[r] & H^1(U \cap V, \F) \arrow[dll,
rounded corners, crossing over,
to path={ -- ([xshift=2ex]\tikztostart.east)
|- (Z') [near end]\tikztonodes
-| ([xshift=-2ex]\tikztotarget.west)
-- (\tikztotarget)}]
\\
& H^2(U \cup V, \F) \arrow[r] & H^2(U, \F) \oplus H^2(V, \F) \arrow[r] & H^2(U \cap V, \F) \arrow[r] & \cdots
\end{tikzcd}
\end{center}
\end{theorem}

\begin{proof}
By the above lemma, the left-exact functors $\Gamma(U \cup V, -)$ and $\Gamma(U, -) \oplus \Gamma(V, -)$ and $\Gamma(U \cap V, -)$ satisfy the conditions of Lemma \ref{lem:exact_seq_of_functors} giving an exact sequence of their derived functors. Furthermore, because direct sum is exact it commutes with taking cohomology and thus direct sum commutes with taking derived functors. Thus Lemma \ref{lem:exact_seq_of_functors} gives the required long exact sequence. 
\end{proof}
\end{document}
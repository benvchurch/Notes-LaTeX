\documentclass[12pt]{article}
\usepackage{hyperref}
\hypersetup{
    colorlinks=true,
    linkcolor=blue,
    filecolor=magenta,      
    urlcolor=cyan,
}
 
\urlstyle{same}
\usepackage{import}
\import{./}{AlgGeoCommands}


\begin{document}

\tableofcontents

\newpage

\section{Chapter II Schemes}

\subsubsection{5.13}

Let $S$ be a graded ring, generated by $S_1$ as an $S_0$-algebra. For any integer $d > 0$ define $S^{(d)}$ to be the graded ring,
\[ S^{(d)} = \bigoplus_{n \ge 0} S^{(d)}_{n} \]
where $S^{(d)}_n = S_{nd}$. 
\bigskip\\
Consider the graded ring map $\psi : S \to S^{(d)}$ via sending $f \mapsto f^d$ and the (multipying grading) ring map $\iota : S^{(d)} \embed S$. Notice that if $\p \subset S$ is a homogeneous prime not containing $S_+$ then $\iota^{-1}(\p) \subset S^{(d)}$ is a homogeneous prime and if $S^{(d)}_+ \subset \iota^{-1}(\p)$ then for each $f \in S_+$ we would have $f^d \in \p$ and thus $f \in \p$ proving that $\p \supset S_+$. Therefore, $\iota$ defines a map $\Proj{S} \to \Proj{S^{(d)}}$ which becomes a map of schemes by seeing that locally on $D_+(f)$ for $f \in S^{(d)}_1$ it comes from a map,
\begin{center}
\begin{tikzcd}
\Proj{S} \arrow[r, "\iota"] & \Proj{S^{(d)}}
\\
\Spec{S_{(f)}} \arrow[u, hook] \arrow[r] & \Spec{S^{(d)}_{(f)}} \arrow[u, hook]
\end{tikzcd}
\end{center}
with $S^{(d)}_{(f)} \to S_{(f)}$ the inclusion which is actually an isomorphism because $\deg{f} = d$ (in $S$) so by degree reasons any fraction in $S_{(f)}$ has numerator of degree $n d$. Furthermore, since $S^{(d)}$ is generated in degree $1$ (because $S$ is generated in degree $1$ we see that $S_{nd}$ is generated by $S_d$) we see that $D_+(f)$ cover $\Proj{S^{(d)}}$ since a prime with $f \in \p$ for all $f$ would contain the irrelevant ideal. Furthermore, the $D_+(f) \subset \Proj{S}$ also cover $\Proj{S}$ because if for any $g \in S_1$ we have $g^d \in S^{(d)}_1$ and if $g^d \in \p$ then $g \in d$ so the only primes of $S$ containing $S^{(d)}_1$ contain the irrelevant ideal. Therefore, this map is locally on the target an isomorphism and therefore an isomorphism. 
\bigskip\\
Furthermore, under this map $S^{(d)}(1)$ is pulled back to $S^{(d)}(1) \otimes_{S^{(d)}} S = S(d)$ showing that this isomorphism identifies $\struct{\Proj{S^{(d)}}}(1)$ with $\struct{\Proj{S}}(1)$. 



\subsection{6}

\subsubsection{6.1 MAKE THIS ACTUALLY WORK}

Let $X$ be a Noetherian integral separated scheme regular in codimension one. Then $X \times \P^n$ is also Noetherian integral and separated and locally $X \times \P^n$ is isomorphic to $X \times \A^n$ which is regular in codimension one by Proposition 6.6. Fix a closed immersion $\P^{n-1} \embed \P^n$ (the line at infinity) and base change to $X$ giving $X \times \P^{n-1} \embed X \times \P^n$ and write $Z = X \times \P^{n-1}$ for this closed subscheme. Then there is an exact sequence,
\begin{center}
\begin{tikzcd}
\Z \arrow[r, "1 \mapsto Z"] & \Cl{X \times \P^n} \arrow[r] & \Cl{X \times \A^n} \arrow[r] & 0 
\end{tikzcd}
\end{center}
where $U = X \setminus Z = X \times \A^n$. First we need to show that $\Z \to \Cl{X \times \P^n}$ is injective i.e. that $n [Z]$ is never a principal divisor. Let $\xi : \Spec{K} \to X$ be the generic point of $X$ then consider the map $\P^n_K \to X \times \P^n$ pulling back gives a map $\Cl{X \times \P^n} \to \Cl{\P^n_K} = \Z$ giving a diagram,
\begin{center}
\begin{tikzcd}
\Z \arrow[r, "1 \mapsto Z"] & \Cl{X \times \P^n} \arrow[d] \arrow[r] &  \Cl{X \times \A^n} \arrow[r] & 0 
\\
& \Cl{\P^n_K} \arrow[ul, "\deg"]
\end{tikzcd}
\end{center}
I claim that $\Z \to \Cl{X \times \P^n} \to \Cl{\P^n_K} \xrightarrow{\deg} \Z$ is the identity. We send $1 \mapsto [Z] \mapsto [\xi^{-1}(Z)]$ where $[\xi^{-1}(Z)] = [\P^{n-1}_K \embed \P^n_K]$ is the divisor ``at infinity'' which generates $\Cl{\P^n_K} = \Z$ because $\P^{n-1}_K$ is a prime divisor of degree $1$. Therefore $\Z \to \Cl{X \times \P^n}$ has a section meaning that it is injective and the sequence,
\begin{center}
\begin{tikzcd}
0 \arrow[r] & \Z \arrow[r, "1 \mapsto Z"] & \Cl{X \times \P^n} \arrow[r] & \Cl{X \times \A^n} \arrow[r] & 0 
\end{tikzcd}
\end{center}
splits. Finally, $\Cl{X \times \A^n} \cong \Cl{X}$ and thus the split sequence provides an isomorphism, 
\[ \Cl{X \times \P^n} \cong \Cl{X} \times \Z \]

\subsubsection{6.2 DO!!}

Let $k$ be an algebraically closed field and $X \subset \P^n_k$ be a closed subvariety regular in codimension one.

\begin{enumerate}
\item Let $V \subset \P^n_k$ be an irreducible (reduced) hypersurface (i.e. a prime divisor of $\P^n_k$) not containing $X$. Now let $Y_i$ be the irreducible components of $V \cap X$. Let $U_i$ be an affine open of $\P^n_k$ intersecting $Y_i$ then $V \cap U_i = V(f_i)$ for some $f_i \in \struct{\P^n_k}(U_i)$ because $V$ is a hypersurface. Under $\struct{\P^n_k} \to \struct{X}$ this pulls back to $\bar{f}_i \in \struct{X}(U_i \cap X)$ and we define,
\[ V \cdot X = \sum \nu_{Y_i}(\bar{f}_i) [Y_i] \]
We need to show this is independent of choices. Firstly, $f_i$ is determined up to units and thus $\nu_{Y_i}(\bar{f}_i)$ are independent of the choice of local equation. If we choose a different open patch $U_i'$ the equations are equal up to units at the generic stalk of $Y_i$. Extending linearly corresponds to alowing irreducible $f_i$ because the sum of two divisors is cut out by the product of equations and whose valuations are again summed.

\item Let $D = \div{f}$ be a principal divisor on $\P^n_k$ such that each prime component does not contain $X$. We want to show that $D \cdot X$ is principal. Write $f = \frac{g_1^{r_1} \cdots g_k^{r_k}}{h_1^{s_1} \cdots h_r^{s_p}}$ where $g_i,h_i \in k[x_0, \dots, x_n]$ are irreducible homogeneous of degree $d_i, d_i'$ such that $r_1 d_1 + \cdots + r_k d_k = s_1 d_1' + \cdots + s_p d_p'$. Then we have,
\[ \div{(f)} = \sum_{i = 1}^k d_i \cdot [V(g_i)] - \sum_{i = 1}^p d_i' \cdot [V(h_i)] \]
where the $V(g_i)$ and $V(h_i)$ are prime divisors i.e. irreducible hypersurfaces corresponding to the homogeneous polynomials. Under the map $\struct{\P^n_k} \to \struct{X}$ locally over $D(g_i)$ the element $\frac{f_i}{g_j}$ is nonzero because $X$ is not contained in either $V(f_i)$ for $V(g_j)$ and therefore $f$ maps to an element $\bar{f} \in K(X)$ and we see that $D \cdot X = \div{(\bar{f})}$ because each $f_i$ and $g_i$ are local equations for the principal divisors and therefore by part (a),
\[ V(g_i) \cdot D = \sum \nu_{Y_j(f_i)}(\bar{f}_i) [Y_j(f_i)] \]
and therefore, 
\[ (\div{f}) \cdot X = \sum_{i = 1}^k d_i \sum \nu_{Y_j(g_i)}(g_i) [Y_j(g_i)]  - \sum_{i = 1}^p d_i' \sum \nu_{Y_j(f_i)}(f_i) [Y_j(f_i)] \]
Likewise,
\[ \div{\bar{g_i}} = \sum \nu_{Y_j}(\bar{g_i}) [Y_i]  \]
so we conclude that,
\[ (\div{f}) \cdot X = \div{\bar{f}} \]
Hence we get a map $\Cl{\P^n_k} \to \Cl{X}$ given by $D \mapsto D \cdot X$. 

\item Cover $\P^n_k$ by affine opens $U_i = \Spec{A_i}$ then locally $X \cap U_i = \Spec{A_i / I_i}$ (DOOOO!!!!)

\item Let $D$ be a principal divisor on $X$ so $D = \div{(f)}$ for $k \in K(X)$. Since locally $X \cap U = \Spec{A/I}$ we have $f \in K(A/I)$ so we can write $f = \frac{\bar{a}}{\bar{b}}$ for $a, b \in A$ and $\bar{a}, \bar{b} \in A$ the images. Therefore, we can consider $f' = \frac{a}{b} \in K(\P^n_k)$. Furthermore, let $D' = \div{(f')}$ then the local equations $a$ and $b$ reduce to $\bar{a}$ and $\bar{b}$ and therefore,
\[ D' \cdot X = D \]
proving that,
\[ \deg{D} = \deg{D' \cdot X} = (\deg{D'}) \cdot (\deg{X}) = 0 \]
because $D'$ is a principal divisor on $\P^n_k$ and thus $\deg{D'} = 0$. Therefore, we get a homomorphism $\deg : \Cl{X} \to \Z$ making the diagram,
\begin{center}
\begin{tikzcd}[row sep = large, column sep = large]
\Cl{\P^n_k} \arrow[r, "D \mapsto D \cdot X"] \arrow[d, "\deg"] & \Cl{X} \arrow[d, "\deg"]
\\
\Z \arrow[r, "\cdot \deg{X}"] & \Z
\end{tikzcd}
\end{center}
commute since $\deg{D \cdot X} = (\deg{D}) \cdot (\deg{X})$.
\end{enumerate}

\subsubsection{6.3 DO!!}

Let $V \subset \P^n_k$ be a projective variety with $\dim{V} \ge 1$ which is nonsingular in codimension $1$. Let $X = C(V)$ be the affine cone over $V$ in $\A^{n+1}_k$ and $\bar{X}$ be its projective closure in $\P^{n+1}_k$. Let $P \in X$ be the vertex of $X \subset \bar{X}$. 

\begin{enumerate}
\item Let $V$ be cut out by a homogeneous ideal $I \subset k[x_0, \dots, x_n]$. Then,
\[ V = \Proj{k[x_0, \dots, x_n]/I} \quad \text{ and } \quad X = C(V) = \Spec{k[x_0, \dots, x_n]/I} \]
Therefore,
\[ \overline{X} = \Proj{k[x_0, \dots, x_n, x_{n+1}]/\tilde{I}} \]
where $\tilde{I} = I \cdot k[x_0, \dots, x_n, x_{n+1}]$.
Let $\pi : \overline{X} \setminus \{ P \} \to V$ be the linear projection. Let $U_i = D_{+}(x_i)$ be standard affines on $\P^n$ and then,
\[ \tilde{U}_i = U_i \cap V = \Spec{(k[x_0, \dots, x_n]/I)_{(x_i)}} \]
Therefore we compute,
\begin{align*}
\pi^{-1}(U_i) & = \Spec{(k[x_0, \dots, x_n, x_{n+1}/\tilde{I})_{(x_i)}} = \Spec{(k[x_0, \dots, x_n]/I)_{(x_i)} \otimes_k k[\tfrac{x_{n+1}}{x_i}]} 
\\
& = \Spec{(k[x_0, \dots, x_n]/I)_{(x_i)}} \times_k \Spec{k[\tfrac{x_{n+1}}{x_i}]} = U_i \times_k \A^1_k
\end{align*}
Therefore, consider the map $\Cl{V} \to \Cl{\overline{X} \setminus \{ P \}}$ (DO THIS!!)


Finally, because $\codim{P, \overline{X}} \ge 2$ we see that $\Cl{\overline{X}} \to \Cl{\overline{X} \setminus \{ P \}}$ is an isomorphism. Putting everything together we find that,
\begin{center}
\begin{tikzcd}
& \Cl{\overline{X}} \arrow[d, "\sim"]
\\
\Cl{V} \arrow[ru, dashed, "\sim"] \arrow[r, "\sim"] & \Cl{\overline{X} \setminus \{ P \}}
\end{tikzcd}
\end{center}
giving an isomorphism $\Cl{V} \iso \Cl{\overline{X}}$.

\item We have $V \subset \overline{X}$ cut out as $V = V(x_{n+1})$. Choose a hyperplane $H$ not containing $V$. Then $H = V(f)$ is cut out by a linear form $f$. Since $H$ does not contain $V$, then $f' \frac{f}{x_{n+1}} \in K(C(V))$ is a nonzero rational function. Then $[V] + \div{(f)} = \pi^* [V \cdot H]$ because the local equation for $V \cdot H$ is $f$ which pulls back locally to $f' x_{n+1}$ inside $K(X)$ and $V(x_{n+1}) = V \subset \overline{X}$. Therefore, taking the map $X \setminus \{ P \} \embed \bar{X} \setminus \{ P \} \xrightarrow{\pi} V$, we get a map $\Cl{V} \to \Cl{X}$ and a sequence,
\begin{center}
\begin{tikzcd}
0 \arrow[r] & \Z \arrow[r, "1 \mapsto V \cdot H"] & \Cl{V} \arrow[r] & \Cl{X} \arrow[r] & 0
\end{tikzcd}
\end{center}
where $\Cl{X \setminus \{ P \}} = \Cl{X}$ because $\dim{X} \ge 2$.
The map $\Z \to \Cl{V}$ is injective because $\deg{n \cdot [V \cdot H]} = n \cdot (\deg{V}) \cdot  (\deg{H}) = n \cdot \deg{V}$ which is injective since $\deg{V} \neq 0$ meaning $\Z \to \Cl{V} \to \Z$ is injective and thus $\Z \to \Cl{V}$ is injective. Next, $X \setminus \{ P \} \embed \overline{X} \setminus \{ P \}$ is an open immersion with compliment $V$ which is a divisor and therefore the map $\Cl{\overline{X} \setminus \{ P \}} \to \Cl{X \setminus \{ P \}}$ is surjective with kernel exactly the subgroup generated by $[V]$ i.e. the image of $\Z \to \Cl{V} \to \Cl{\overline{X} \setminus \{ P \}}$. Furthermore, $\Cl{V} \to \Cl{\overline{X} \setminus \{ P \}}$ is an isomorphism implying that $\Cl{V} \to \Cl{X \setminus \{ P \}}$ is surjective with kernel exactly the image of $\Z \to \Cl{V}$ meaning the seqeunce is exact.

\item Let $S(V)$ be the homogeneous coordinate ring of $V$ and thus $X = \Spec{S(V)}$. Suppose that $S(V)$ is a UFD. Since $S(V)$ is a Noetherian domain, by Prop. 6.2 we see that $\Cl{X} = 0$ and $S(V)$ is integrally closed. Therefore by the exact sequence $\Cl{V} = \Z$ generated by the class of $V \cdot H$ and $S(V)$ being an integrally closed domain means that $V \embed \P^n_k$ is a projectively normal embedding. Conversely, if we assume that $V \embed \P^n_k$ is projectively normal, i.e. $S(V)$ is an integrally closed domain, and $\Cl{V} = \Z$ generated by the class of $V \cdot H$. From the exact sequence, $\Cl{X} = 0$ and since $S(V)$ is an integrally closed domain we see that $S(V)$ is a UFD by Prop. 6.2.  

\item Let $\stalk{X}{P}$ be the local ring at the cone point $P$ on $X$. Consider the natrual map,
\[ \Cl{X} \to \Cl{\Spec{\stalk{X}{P}}} \]
Let $R = S(V)$ then $X = \Spec{R}$ and $\stalk{X}{P} = R_\m$ where $\m = (x_0, \dots, x_n)$. Prime divisors $D$ of $\Spec{\stalk{X}{P}}$ correspond to height one primes $\q \subset R$ contained in $\m$ and thus the closure $\overline{D}$ in $X$ corresponds to $V(\q) \subset \Spec{R}$. Therefore $\Cl{X} \to \Cl{\Spec{\stalk{X}{P}}}$ is surjective. To establish that this map is injective we need to show for any divisor,
\[ D = \sum_{\height{\q} = 1} n_\q \cdot \q \]
such that for some $f \in \Frac{R}^\times$,
\[ D_P = \sum_{\q \subset \m} n_\q \cdot \q = \div{(f)}|_P \]
then we have $D = \div{(f')}$ for some $f, \in \Frac{R}^\times$ where $\div{(f)}|_P = \div_P{(f)}$ viewing $f \in \Frac{R_\m}$ and taking the associated divisor on $\Spec{R_\m}$. However, since the map $\Cl{V} \to \Cl{X}$ is a surjection, the class of $\q$ must be in the image of $\Cl{V} \to \Cl{X}$ and thus $D \sim D'$ where $D'$ is a divisor pulled back from $V$ and thus with each component passing though $\m$. Then,
\[ D - D' = \div{f'} \]
However, $D' = \div{f'}$ and 
\[ (D - D')_P = D_P - D'_P = (\div{f} - \div{f'})|_P \]
However, $D'_P = D'$ because each component of $D$ passes through $P$. Therefore,
\[ D' = \div{f'} +  \]
It is interesting to note that this implies the following. If $\p \subset R$ has $\height{\p} = 1$ and $\p \not\subset \m$ then $\p$ is principal. This is because $\p \mapsto 0$ under $\Cl{X} \to \Cl{\Spec{\stalk{X}{P}}}$ and therefore $\p = 0$ in $\Cl{X}$.  
(FUCK WE NEED TO SHOW THAT THE PRIMES NOT CONTAINING $\m$ ARE PRINCIPAL)
\end{enumerate}

\subsubsection{6.4}

Let $k$ be a field with characteristic $\neq 2$ and $f \in k[x_1, \dots, x_n]$ a square-free nonconstant polynomial. Let $R = k[x_1, \dots, x_n]$ and $A = k[x_1, \dots, x_n, z]/(z^2 - f)$ which is a domain since $z^2 - f$ is irreducible when $f$ is squarefree and nonconstant with fraction field $L = K[z]/(z^2 - f)$ where $K = k(x_1, \dots, x_n)$. The extension $L / K$ is Galois with $\Gal{L/K} = \Z / 2 \Z$ acting via $\sigma : z \mapsto -z$. A generic element is $\alpha = g + h z \in L$ with $g,h \in K$ which has minimal polynomial,
\[ m_\alpha(x) = (x - \alpha)(x - \sigma(\alpha)) = (x - g - h z)(x - g + hz) = x^2 - 2 xg + (g^2 - h^2 f) \]
Suppose that $\alpha \in L$ is integral over $R$ then there exists a monic polynomial $p \in R[x] \subset L[x]$ such that $p(\alpha) = 0$ and therefore $m_\alpha \divides p$. Since $R$ is a UFD Gauss' lemma applies showing that every monic factor of $p$ must have coefficients in $R$. In particular $m_\alpha \in R[x]$ meaning that $g \in R$ and $h^2 f \in R$. However, since $f \in R$ is squarefree this implies that $h \in R$ as well since otherwise the square of its denominator would have to divide $f$ and thus $\alpha \in A$. Conversely, if $\alpha \in A$ then $g,h \in R$ and thus $m_\alpha \in R[x]$ is monic which implies that $\alpha$ is integral. Therefore $A$ is the integral closure of $R$ in $L$ and thus $A$ is itself integrally closed in its field of fractions $L$.

\subsubsection{6.5}

Let $k$ be a field with characteristic $\neq 2$ and $X$ be the affine quadric hypersurface,
\[ X = \Spec{k[x_0, \dots, x_n]/(x_0^2 + x_1^2 + \cdots + x_r^2)} \]
\begin{enumerate}
\item Let $r \ge 2$. Then write $x_0^2 + x_1^2 + \cdots + x_r^2 = x_0^2 - f$ where $f = -(x_1^2 + \cdots + x_r^2) \in k[x_1, \dots, x_r]$. Since $r \ge 2$ the polynomial $f$ is squarefree (because each derivative $\partial_{x_i} f = - 2 x_i$ is coprime to $f$) and nonconstant so by the previous problem the ring,
\[ A = k[x_0, \dots, x_n]/(x_0^2 + x_1^2 + \cdots + x_r^2) \]
is an integrally closed domain and thus normal.

\item Let $x_0 = \tfrac{i}{2}(x_0' + x_1')$ and $x_1 = \tfrac{1}{2}(x_0' - x_1')$ then $x_0^2 = - \tfrac{1}{4}( x_0'^2 + 2 x_0' x_1' + x_1'^2 )$ and $x_1'^2 = \tfrac{1}{4} (x_0'^2 - 2 x_0' x_1' + x_1'^2)$ therefore,
\[ x_0^2 + x_1^2 + x_2^2 + \cdots + x_r^2 = - x_0' x_1' + x_2^2 + \cdots x_r^2 \]
so $X$ is cut out by the equation,
\[ x_0 x_1 = x_2^2 + \cdots + x_r^2 \]
In general, consider the closed subscheme $Z = V(x_1)$. This subscheme is not necessarily prime because $x_2^2 + \cdots + x_r^2$ is not irreducible unless $r \ge 4$ and thus $(x_1)$ is not prime because $x_0 x_1 = x_2^2 + \cdots + x_r^2$. Therefore, let $Y_i$ be the irreducible components of $Z$ given their reduced structure so the $Y_i$ are prime divisors. Then there is an exact sequence,
\begin{center}
\begin{tikzcd}
\Z Y_1 \oplus \cdots \oplus \Z Y_k \arrow[r] & \Cl{X} \arrow[r] & \Cl{X \setminus Z} \arrow[r] & 0
\end{tikzcd}
\end{center}
the first map sending $Y_i \mapsto [Y_i]$. However, $X \setminus Z = \Spec{A_{x_1}}$ and,
\begin{align*}
A_{x_1} & = k[x_0, x_1, x_1^{-1}, x_2 \dots, x_r]/(x_0 x_1 - (x_2^2 + \cdots + x_r^2)) 
\\
& = k[x_0, x_1, x_1^{-1}, x_2 \dots, x_r]/(x_0 - x_1^{-1} (x_2^2 + \cdots + x_r^2)) = k[x_1, x^{-1}, x_2, \dots, x_r] 
\end{align*}
is an open subspace of $\A^r$ and thus $A_{x_1}$ is a UFD and therefore we find that,
\[ \Cl{X \setminus Z} = \Cl{\Spec{A_{x_1}}} = 0 \]
Therefore $\Cl{X}$ is generated by the $Y_i$. It remains to compute the relations.
\bigskip\\
First let $r = 2$. Consider the divisor $Y \subset X$ cut out by $x_1 = x_2 = 0$ i.e. a ruling of the cone. Since $A/(x_1) = k[x_0, x_2]/(x_2^2)$ the divisor $Y$ is the unique component of $Z$ and $[Z] = 2 [Y]$ since $\div{x_1} = 2 Y$ because $x_1 = x_0^{-1} x_2^2$ and $x_2$ is a local parameter for $Y$. Now we need to show that $[Y]$ is not principal or equivalently the corresponding prime ideal $\p = (x_1, x_2)$ is not a principal ideal. Let $\m = (x_0, x_1, x_2)$ and notice that $\m / \m^2$ is a 3 dimensional $\kappa(\m)$-vectorspace generated by $\bar{x}_0, \bar{x}_1, \bar{x}_3$. Furthermore, the image of $\p \otimes \kappa(\m) \to \m \otimes \kappa(\m)$ contains $\bar{x}_1$ and $\bar{x}_2$ so $\p \otimes \kappa(\m)$ must have dimension at least $2$ showing that $\p$ cannot be principal because $\p \otimes \kappa(\m)$ is the number of generators of $\p_\m$. Therefore $[Y] \neq 0$ and $2 [Y] = 0$ showing that $\Cl{X} = \Z / 2 \Z$ generated by $[Y]$.
\bigskip\\
Let $r = 3$. Then $Z$ contains two prime divisors, $Y_1 = V(x_1, x_2 - i x_3)$ and $Y_2 = V(x_2, x_2 + i x_3)$ with local parameters $x_2 - i x_3$ and $x_2 + i x_3$. The only relation is given by $\div{(x_1)} = Y_1 + Y_2$ which follows from the fact that $x_1 = x_0^{-1} (x_2^3 + x_3^2) = x_0^{-1} (x_2 - i x_3)(x_2 + i x_3)$. Therefore, $\Cl{X} = (\Z Y_1 \oplus \Z Y_2)/ (Y_1 + Y_2) \cong \Z$. Alternatively, we see that $X$ is the affine cone over the projective quadric $Q = V(x_0 x_1 - x_2 x_3)$ which is isomorphic to $\P^1 \times \P^1$. By the previous problem there is a short exact sequence,
\begin{center}
\begin{tikzcd}
0 \arrow[r] & \Z \arrow[r] & \Cl{Q} \arrow[r] & \Cl{X} \arrow[r] & 0
\end{tikzcd}
\end{center}
where the map $\Z \to \Cl{Q}$ sends $1 \mapsto Q \cdot H$ the hyperplane class. However $\Cl{Q} = \Z \oplus \Z$ and $Q \cdot H = (1,1)$ so we see that $\Cl{X} = (\Z \oplus \Z)/(1,1) \Z \cong \Z$.
\bigskip\\
Finally, let $r \ge 4$. Then $Z$ is a prime divisor and $\div{(x_1)} = Z$ so it is trivial in the class group showing that $\Cl{X} = 0$.

\item Let $Q$ be the projective quadric hypersurface in $\P^r$ cut out by $x_0 x_1 = x_2^2 + \cdots + x_r^2$. We apply the exact sequence,
\begin{center}
\begin{tikzcd}
0 \arrow[r] & \Z \arrow[r] & \Cl{Q} \arrow[r] & \Cl{X} \arrow[r] & 0
\end{tikzcd}
\end{center}
where $X$ is the affine cone over $X$ considered previously. 
\bigskip\\
For $r = 2$ we know $\Cl{X} = \Z / 2 \Z$ and furthermore $Q \cong \P^1$ so $\Cl{Q} = \Z$ therefore the seqeunce shows that the hyperplane class $Q \cdot H$ is twice a generator.
\bigskip\\
For $r = 3$ we know $\Cl{X} = \Z$ and therefore the sequence splits giving $\Cl{Q} = \Z \oplus \Z$.
\bigskip\\
For $r \ge 4$ we know $\Cl{X} = 0$ and therefore the sequence gives $\Cl{Q} = \Z$ generated by the hyperplane class $Q \cdot H$.

\item Let $r \ge 4$ and $Y \subset Q$ be an irreducible subvariety of codimension one i.e. a prime divisor. The homogeneous coordinate ring,
\[ C(Q) = k[x_0, \dots, x_r]/(x_0 x_1 - (x_2^2 + \cdots x_r^2)) \]
is integrally closed by 2.6.4 and thus $Q$ is projectively normal. Furthermore, by the previous problem $\Cl{Q} = \Z$ generated by $Q \cdot H$ so by 2.6.3(c) we see that $S(Q)$ is a UFD. Therefore, the homogeneous prime $\p$ corresponding to $Y$ has height one and thus is principal $\p = (\bar{f})$ with a lift $f \in k[x_0, \dots, x_r]$. Therefore, let $V = V(f) \subset \P^n$ which is an irreducible hypersurface since $(\bar{f})$ is prime and thus $f$ is irreducible. Then we see that $V \cap Q = Y$ scheme theoretically showing that $Y$ is a complete intersection.
\bigskip\\
Alternatively, there is a cohomological argument. Because $Q$ is locally factorial ($Q$ is regular) prime divisors $Y$ correspond to nonzero sections $s \in H^0(X, \L)$ for a line bundle $\L \in \Pic{Q} = \Cl{Q}$. However, $\Cl{Q}$ is generated by the hyperplane class so $\L = \struct{Q}(d)$ where $\struct{Q}(d) = \iota^* \struct{\P^r}(d)$. We need to show that every section $s \in H^0(Q, \struct{Q}(d))$ is the pullback of a section $s' \in H^0(\P^r, \struct{\P^r}(d))$ since then $Y = V \cap Q$ where $V = V(s')$ is a hypersurface of degree $d$. There is an exact sequence,
\begin{center}
\begin{tikzcd}
0 \arrow[r] & \I(d) \arrow[r] & \struct{\P^r}(d) \arrow[r] & \struct{Q}(d) \arrow[r] & 0
\end{tikzcd}
\end{center}
where $\I = \struct{\P^r}(-2)$ is the ideal sheaf of $Q$. We get a long exact sequence,
\begin{center}
\begin{tikzcd}
0 \arrow[r] & H^0(\P^r, \struct{\P^r}(d-2)) \arrow[r] & H^0(\P^r, \struct{\P^r}(d)) \arrow[r, "\iota^*"] & H^0(Q, \struct{Q}(d)) \arrow[r] &  H^1(\P^r, \struct{\P^r}(d-2)) 
\end{tikzcd}
\end{center}
However, since $r > 1$ we know $H^1(\P^r, \struct{\P^r}(d-2)) = 0$ and thus,
\[ \iota^* : H^0(\P^r, \struct{\P^r}(d)) \to H^0(Q, \struct{Q}(d)) \]
is surjective. In fact, the kernel is $f \cdot H^0(\P^r, \struct{\P^r}(d-2))$ where $f = x_0 x_1 - (x_2^2 + \cdots + x_r^2)$ is the defining equation and thus irreducible hypersurfaces of degree $d$ besides $f$ pull back to nonzero sections.
\end{enumerate}

\subsubsection{6.6}

Let $X$ be the nonsingular plane cubic curve $y^2 z = x^3 - x z^2$. 
\begin{enumerate}
\item 
If $P, Q, R$ are colinear then let $\ell$ be the line in $\P^2$ passing through $P,Q,R$. Then $\ell \sim \ell_\infty$ the line at infinity and pulling back to $X$ we find $X \cap \ell \sim X \cap \ell_{\infty}$. Because $X$ is a cubic curve $\ell$ intersects $X$ at exactly $P,Q,R$ with multiplicity one and $\ell_{\infty}$ intersects $P_0$ with multiplicity three. Thus in $\Cl{X}$ we find $[P] + [Q] + [R] = 3 [P_0]$. Therefore,
\[ ([P] - [P_0]) + ([Q] - [P_0]) + ([R] - [R_0])  = 0 \]
Since the isomorphism $X \to \Cl{X}$ sends $P \mapsto [P] - [P_0]$ we see that $P + Q + R = 0$ in $X$. Conversely, suppose that $P + Q + R = 0$ then we know that $[P] + [Q] + [R] = 3 [P_0]$. The line passing through $P$ and $Q$ intersects $X$ at some additional point $R'$ (possibly equal to $P$ or $Q$) and from above we know that $[P] + [Q] + [R'] = 3 [P_0]$ and thus $[R] - [P_0] = [R'] - [P_0]$ so $R = R'$ by injectivity showing that $P,Q,R$ are colinear.

\item Let $P \in X$ be a point such that the tangent line $\ell$ passes through $P_0$. Then $\ell \cap X = 2 [P] + [P_0]$ (as divisors) so we see that $2 P + 0 = 0$ i.e. $2 P = 0$ so $P$ has order two. Let $P \in X$ have order two meaning that $2 P = 0$. Then if the tanget line $\ell$ intersects $X$ as $\ell \cap X = 2 [P] + [Q]$ (as divisors) for $[Q] \neq [P_0]$ we would have $2 P + Q = 0$ and thus $Q = 0$ implying that $Q = P_0$ as points so the tangent line passes through $P_0$. 

\item If $P$ is an inflection point then the tangent line $\ell$ intersects $X$ as $\ell \cap X = 3 [P]$ so we have $3 [P] = 3 [P_0]$ and thus $3 P = 0$ in the group law. Conversely if $3 P = 0$ then in $\Cl{X}$ we have $3 [P] = 3 [P_0]$. If $P$ is not an inflection point then its tangent line $\ell$ intersects $X$ with $\ell \cap X = 2 [P] + [Q]$ (as divisors). Then we find $2 P + Q = 0$ in the group law so $P = Q$ meaning that $\ell \cap X = 3 P$ so $P$ is an infection point.

\item Let $k = \CC$ then $X(\Q)$ is a subgroup because if $P$ and $Q$ have rational coefficients then the line between them is defined over $\Q$ and thus the third intersection point $R$ with $X$ has rational coefficients. Furthermore, $P + Q + R = 0$ so $-R = P + Q$ also has rational coefficents because $-R$ is simply $R$ reflected along the $y$-axis (because $R + (-R) + P_0 = 0$ so $R$ and $-R$ must define a line passing through $P_0$ i.e. vertical in the $(x,y)$ patch). 
\bigskip\\
For $z = 0$ there is a unique point $P_0 = [0:1:0]$. For $z \neq 0$ we consider the affine patch cut out by $y^2 = x(x^2 - 1)$. If $y = 0$ then $x(x - 1)(x + 1) = 0$ so $x = -1, 0, 1$ giving three rational points $P = [-1 : 0 : 1]$ and $Q = [0 : 0 : 1]$ and $R = [1 : 0 : 1]$. If $x = 0$ then $y^2 = 0$ so $y = 0$ giving the point $[0 : 0 : 1]$. Fermat descent shows there are no others. By colinearity, $P + Q + R = 0$ and since the tangent lines pass through $P_0$ we also have $2 P = 2 Q = 2 R = 0$. Therefore, we have found the group is,
\[ X(\Q) = \Z / 2 \Z \oplus \Z / 2 \Z \]
generated by $P$ and $Q$ with $R = P + Q$.
\end{enumerate}

\subsubsection{6.7 DO THIS!!}

Let $X$ be the nodal cubic $y^2 z = x^3 + x^2 z$ in $\P^2$. Let $Z \in X$ denote the singular point and $U = X \setminus Z$. We want to compute $\CaCl{X}$. Notice that by multiplying by a suitable global function, any Cartier divisor is linearly equivalent to one which is invertible in a neighborhood of $Z$. Furthermore, there is an isomorphism $\CaCl{U} \iso \Cl{U}$ giving a degree map $\CaCl{X} \to \Z$ because I claim that any $f \in K(X)^\times$ which is invertible in a neighborhood of $Z$ has $\deg{\div{(f)}} = 0$. Indeed let $\tilde{X} \to X$ be the normalization then the rational function $f : X \rat \P^1$ extends uniquely to $\tilde{f} : \tilde{X} \to \P^1$ and $\div{(\tilde{f})} = [\tilde{f}^{-1}(0)] - [\tilde{f}^{-1}(\infty)]$ has degree zero since $[0] - [\infty]$ has degree zero. However, by assumption $Z$ is not in either fiber and $\tilde{X} \to X$ is an isomorphism away from $Z$ meaning that $\div{(f)} = [f^{-1}(0)] - [f^{-1}(\infty)]$ also has degree zero. 
\bigskip\\
Next, for each $P \in U$ there is an associated divisor $D_P$ which is $1$ on a neighborhood of $Z$ and equal to the Weil divisor $[P] - [P_0]$ on $U$ where $P_0 = [0:1:0]$. I claim that $U(k) \to \CaClcirc{X}$ is an isomorphism giving $U$ the structure of the group scheme $\Gm$. First suppose that $D_P \sim D_Q$ then there exists $f \in K(X)^\times$ such that $f$ is invertible in a neighborhood of $Z$ and $\div{(f)}|_U = [P] - [Q]$ therefore defines a birational morphism $f : X \to \P^1$. Note: $f$ is automatically defined on $U$ but since $f$ is already an element of $\struct{V}(V)^\times$ for some affine open $Z \in V \subset X$ we can extend $f : X \to \P^1$ via $k[t] \to \struct{V}(V)$ by sending $t \mapsto f$. Such a map would be an isomorphism because rational maps $\P^1 \rat X$ extend. This is impossible because $X$ is singular so we must have $P = Q$.
\bigskip\\
Fix points $P,Q \in U$. The line $\ell$ passing through $P$ and $Q$ intersects $X \subset \P^2$ at exactly one other point $R$. Notice that $R \in U$ because if $R = Z$ then $\ell$ intersects $R$ at multiplicity two since $Z \in X$ is a double point which is impossible because $\ell \cdot X$ has degree three. Furthermore, the line at infinity $\ell_\infty = [s : t : 0]$ intersects $X$ when $s^3 = 0$ so $\ell_\infty \cdot X = 3 \cdot P_0$. Therefore, $P + Q + R \sim 3 \cdot P_0$ in $\Cl{U}$ by a linear relation invertible at $Z$ since both lines intersect $X$ inside $U$ and thus,
\[ D_P + D_Q + D_R = 0 \]
in $\CaCl{X}$. Thus, taking the line through $P$ and $P_0$ gives a point $-P$ such that,
\[ D_P + D_{-P} = 0 \]
Therefore, the map $U(k) \to \CaClcirc{X}$ induces a group structure on $U$ via $P_0 = 0$ and $P + Q = -R$ and $P \mapsto -P$ satisfies $P + -P = 0$. Furthermore, the $D_P$ generate $\CaClcirc{X}$ because and any Cartier divisor is linearly equivalent to a divisor invertible near $Z$ and furthermore over $U$ Cartier divisors correspond to Weil divisors,
\[ D|_U = \sum_{P \in U} n_P \cdot [P] \]
but since $\deg{D|_U} = 0$ we have,
\[ D|_U = \sum_{P \in U} n_P \cdot ([P] - [P_0]) \]
and thus,
\[ D = \sum_{P \in U} n_P D_P \]
however by the sum and inversion relations we can reduce to $D \sim D_{P'}$ where,
\[ P' = \sum_{P \in U} n_P \cdot P \]
in the group law on $U$. Therefore $U(k) \to \CaClcirc{X}$ is an isomorphism of groups. 
\bigskip\\
Finally, I claim that $U \cong \Gm$ as a group scheme. Consider the normalization $\P^1 \to X$ given by $x = t^2  + 1$ and $y = t(t^2 + 1)$ (FINISH THIS!!)

$t^2 = (x - 1)$ 
\subsubsection{6.8 DO THIS!!}

Let $f : X \to Y$ be a morphism of schemes.

\begin{enumerate}
\item The map $f^* : \Pic{Y} \to \Pic{X}$ sending $\L \mapsto f^* \L$ is a group homomorphism because $f^* (\L_1 \otimes \L_2) = f^* \L_1 \otimes f^* \L_2$. 

\item If $f$ is a finite morphism of nonsingular curves then consider the square,
\begin{center}
\begin{tikzcd}
\Pic{Y} \arrow[d, "f^*"] \arrow[r, "c_1"] & \Cl{Y} \arrow[d, "f^*"] 
\\
\Pic{X} \arrow[r, "c_1"] & \Cl{X}
\end{tikzcd}
\end{center}
On prime divisors $Q \in Y$ pullback acts via,
\[ f^* [Q] = \sum_{P \in f^{-1}(Q)} \nu_P(t_Q) \cdot [P] \]
where $t_Q \in K(Y) \subset K(X)$ is a local parameter of $\stalk{Y}{Q}$. For $\L \in \Pic{Y}$, under $c_1$ we choose a nonzero meromorphic section $s \in \L_\eta$ then $c_1(\L) = \div{(s)}$. Furthermore, $f^* s$ is a nonvanishing meromorphic section of $f^* \L$. Thus $c_1(f^* \L) = \div{(f^* s)} = f^* \div{(s)}$. To see this, locally we choose a trivializations $s_{U} : \struct{U} \to \L|_U$ then,
\[ \div{(s)} = \sum_{Q \in Y} \nu_Q(s/s_U) \cdot [Q] \] 
Therefore,
\[ f^* \div{(s)} = \sum_{Q \in Y} \sum_{P \in f^{-1}(Q)} \nu_Q(s/s_U) \nu_P(t_Q) \cdot [P] \]
Furthermore, 
\[ \div{(f^* s)} = \sum_{P \in X} \nu_P(f^\#(s/s_U)) \cdot [P] \]
Since $f^\# : \stalk{Y}{Q} \to \stalk{X}{P}$ is a map of DVRs sending local parameters $t_Q \mapsto u \, t_P^{\nu_P(t_Q)}$ and thus $\nu_P(f^\#(s/s_U)) = \nu_Q(s/s_U) \nu_P(t_Q)$ showing that,
\[ \div{(f^* s)} = f^* \div{(s)} \]
and therefore $c_1(f^* \L) = f^* c_1(\L)$. Alternatively, the morphism $\CaCl{X} \to \Pic{X}$ is clearly natural so we can consider the diagram,
\begin{center}
\begin{tikzcd}
\CaCl{Y} \arrow[d, "f^*"] \arrow[r] & \Cl{Y} \arrow[d, "f^*"] 
\\
\CaCl{X} \arrow[r] & \Cl{X}
\end{tikzcd}
\end{center}
Let $D = \{(U_i, f_i)\}$ be a Cartier divisor on $Y$ then $f^*D = \{(f^{-1}(U_i), f^\#(f_i))\}$. Then the associated Weil divisor of $D$ is,
\[ \wt{D} = \sum_{\substack{Q \in Y \\ Q \in U_i}} \nu_Q(f_i) \cdot [Q] \]
Then we see,
\[ \wt{f^* D} = \sum_{\substack{P \in X \\ P \in f^{-1}(U_i)}} \nu_P(f^\#(f_i)) \cdot [P] \]
However, $\nu_P(f^\#(f_i)) = \nu_Q(f_i) \nu_P(t_Q)$ and therefore,
\[ f^* \wt{D} = \sum_{\substack{Q \in Y \\ Q \in U_i}} \sum_{P \in f^{-1}(Q)} \nu_Q(f_i) \nu_P(t_Q) \cdot [P] = \sum_{\substack{P \in X \\ P \in f^{-1}(U_i)}} \nu_Q(f_i) \nu_P(t_Q) \cdot [P] = \wt{f^* D} \]

\item Let $X \subset \P^n$ be a locally factorial closed subscheme and $\iota : X \embed \P^n$ the inclusion map. 
\end{enumerate}

\subsubsection{6.9}

Let $X$ be a (integral) projective curve over $k$ and $\pi : \tilde{X} \to X$ the normalization. For each point $p \in X$ let $\wt{\mathcal{O}}_{p}$ be the integral closure of $\stalk{X}{x}$ in $k(X)$. Consider the exact sequence of sheaves on $X$,
\begin{center}
\begin{tikzcd}
0 \arrow[r] & \pi_* \struct{\tilde{X}}^\times / \struct{X}^\times \arrow[r] & \K_X^\times / \struct{X}^\times \arrow[r] & \K_X^\times / \pi_* \struct{\tilde{X}}^\times \arrow[r] & 0
\end{tikzcd}
\end{center}
The sheaf $\pi_* \struct{\tilde{X}}^\times / \struct{X}^\times$ is supported on the singular points which is a finite set. Taking the cohomology of this exact sequence,
\begin{center}
\begin{tikzcd}
0 \arrow[r] & H^0(\pi_* \struct{\tilde{X}}^\times / \struct{X}^\times) \arrow[r] & H^0(\K_X^\times / \struct{X}^\times) \arrow[r] & H^0(\K_X^\times / \pi_* \struct{\tilde{X}}^\times) \arrow[r] & H^1(\pi_* \struct{\tilde{X}}^\times / \struct{X}^\times)
\end{tikzcd}
\end{center}
However, if $\F$ is a sheaf supported on a finite set of points $S$ then,
\[ H^q(X, \F) = 
\begin{cases}
\bigoplus\limits_{p \in S} \F_p & q = 0
\\
0 & q > 0
\end{cases} \]
Therefore, we get an exact sequence,
\begin{center}
\begin{tikzcd}
0 \arrow[r] & \bigoplus\limits_{p \in X} (\pi_* \struct{\tilde{X}}^\times / \struct{X}^\times)_p \arrow[r] & H^0(\K_X^\times / \struct{X}^\times) \arrow[r] & H^0(\K_X^\times / \pi_* \struct{\tilde{X}}^\times) \arrow[r] & 0
\end{tikzcd}
\end{center}
We need to compute $(\pi_* \struct{\tilde{X}}^\times)_p$ Locally $\pi : \tilde{X} \to X$ over $U = \Spec{A}$ is given by $A \to \wt{A}$ where $\wt{A}$ is the integral closure of $A$ in $K$ and thus $(\pi_* \struct{\tilde{X}}^\times)_p = \wt{A}_\p^\times = \wt{\mathcal{O}}^\times_{p}$ because localization and normalization commute. Furthermore, since $\pi : \tilde{X} \to X$ is birational $\pi_* \K_{\tilde{X}} = \K_X$ and since $\pi : \tilde{X} \to X$ is affine $\pi_*$ is exact so,
\[ \pi_*(\K_{\tilde{X}}^\times / \struct{\tilde{X}})^\times = \pi_* \K_{\tilde{X}}^\times / \pi_* \struct{\tilde{X}}^\times = \K_X^\times / \pi_* \struct{\tilde{X}}^\times \]
Furthermore, recall that the Cartier divisors are $\CaDiv{X} = H^0(\K_X^\times / \struct{X}^\times)$. Thus the exact sequence becomes,
\begin{center}
\begin{tikzcd}
0 \arrow[r] & \bigoplus\limits_{p \in X} \wt{\mathcal{O}}_p^\times / \stalk{X}{p}^\times \arrow[r] & \CaDiv{X} \arrow[r, "\pi^*"] & \CaDiv{\tilde{X}} \arrow[r] & 0
\end{tikzcd}
\end{center}
Finally, because on function fields, $\pi^* : K(X) \to K(\tilde{X})$ is an isomorphism ($\pi$ is birational), we can quotient by $K^\times$ and use the isomorphism $\CaCl{X} = \CaDiv{X} / K^\times \iso \Pic{X}$ (using that $X$ is integral) to get an exact sequence,
\begin{center}
\begin{tikzcd}
0 \arrow[r] & \bigoplus\limits_{p \in X} \wt{\mathcal{O}}_p^\times / \stalk{X}{p}^\times \arrow[r] & \Pic{X} \arrow[r, "\pi^*"] & \Pic{\tilde{X}} \arrow[r] & 0
\end{tikzcd}
\end{center}
Now, let $X$ be a plane cuspidal cubic curve. Then $\tilde{X} = \P^1_k$. There is one singularity at $p$ where locally $A = k[x, y]/(x^3 - y^2)$ with normalization $A \to \wt{A}$ equal to $k[x,y]/(x^3 - y^2) \to k[t]$ via $x \mapsto t^2$ and $y \mapsto t^3$. Then $\wt{\mathcal{O}}_p^\times / \stalk{X}{p}^\times = 1 + k \cdot t$ so there is an exact sequence,
\begin{center}
\begin{tikzcd}
0 \arrow[r] & \Ga \arrow[r] & \Pic{X} \arrow[r] & \Z \arrow[r] & 0
\end{tikzcd}
\end{center}
Likewise, let $X$ be a plane nodal cubic curve. Then $\tilde{X} = \P^1_k$. There is again one singularity at $p$ where locally $A = k[x, y]/(y^2 - x^2(x+1))$ with normalization $A \to \wt{A}$ equal to $k[x,y]/(y^2 - x^2(x+1)) \to k[t]$ via $x \mapsto t^2 - 1$ and $y \mapsto t(t^2 - 1)$. Therefore $t^{n+1} - t^n$ is in the image for each $n$. Then $\wt{\mathcal{O}}_p^\times / \stalk{X}{p}^\times = k^\times$ so there is an exact sequence,
\begin{center}
\begin{tikzcd}
0 \arrow[r] & \Gm \arrow[r] & \Pic{X} \arrow[r] & \Z \arrow[r] & 0
\end{tikzcd}
\end{center}

\subsubsection{6.10}

Let $X$ be a Noetherian scheme and $K(X)$ the Grothendieck group of coherent sheaves.

\begin{enumerate}
\item Let $X = \A^1_k$. Then $K(X)$ is the Grothendieck group for the category of finite $k[x]$-modules. Since $A = k[x]$ is a PID every finite $A$-module has a length two resolution by finite free $A$-modules because the kernel of $A^n \onto M$ is a submodule of a free module and thus free. Thus there is an exact sequence,
\begin{center}
\begin{tikzcd}
0 \arrow[r] & A^m \arrow[r] & A^n \arrow[r] & M \arrow[r] & 0
\end{tikzcd}
\end{center}
Therefore, $[M] = [A^n] - [A^m]$. Furthermore, it is clear that $[A^{n+m}] = [A^n] + [A^m]$ and thus $[M] = [A^{n-m}]$ so we see that $K(X) \cong \Z$ via rank. 
\item Let $X$ be an integral scheme with generic point $\xi \in X$ with function field $K = \stalk{X}{\xi}$. If $\F$ is a coherent sheaf we define the rank $\rank(\F) = \rank_\xi(\F) = \dim_K \F \otimes_{\stalk{X}{\xi}} K = \dim_K \F_\xi$. Given an exact sequence of coherent sheaves,
\begin{center}
\begin{tikzcd}
0 \arrow[r] & \F \arrow[r] & \G \arrow[r] & \H \arrow[r] & 0
\end{tikzcd}
\end{center}
then there is an exact sequence at the generic point,
\begin{center}
\begin{tikzcd}
0 \arrow[r] & \F_\xi \arrow[r] & \G_\xi \arrow[r] & \H_\xi \arrow[r] & 0
\end{tikzcd}
\end{center}
meaning that $\dim_K \G_\xi = \dim_K \F_\xi + \dim_K \H_\xi$. Therefore, $\rank(\G) - \rank(\F) - \rank(\H) = 0$ so rank is compatible with the relation $[\G] - [\F] - [\H] = 0$ in $K(X)$. Therefore, rank factors through the quotient to give a well-defined map $\rank : K(X) \to \Z$.
\bigskip\\
Taking $\F = \struct{X}^{\oplus n}$ we see $\rank(\F) = n$ and thus $[\struct{X}^{\oplus n}] \mapsto n$ so we see that $K(X) \to \Z$ is surjective.

\item Let $\iota : Z \embed X$ be the inclusion of a closed subscheme and $j : U \to X$ inclusion of the complement. Consider the maps $K(Z) \to K(X) \to K(U)$ defined via $[\F] \mapsto [\iota_* \F]$ and $[\F] \mapsto [\F|_U]$. Since $X$ is Noetherian, Exercise 5.15 the map $K(X) \to K(U)$ is surjective. Clearly $(\iota_* \F)|_U = 0$ so to show that the sequence,
\begin{center}
\begin{tikzcd}
K(Z) \arrow[r] & K(X) \arrow[r] & K(U) \arrow[r] & 0
\end{tikzcd}
\end{center}
is exact we simply need to check that if $\F$ is a coherent sheaf on $X$ supported on $Z$ then $[\F]$ is in the image of $K(Z) \to K(X)$. Let $\I = \ker{(\struct{X} \to \struct{Z})}$ and define $\F_k = \I^k \cdot \F$ which is a filtration,
\[ \F = \F_0 \supset \F_1 \supset \cdots \supset \F_n \supset \F_{n+1} = 0 \]
of coherent subsheaves. Note that $\F_k = \I^k \cdot \F = 0$ for sufficently large $k$ because $\F$ is supported on $Z$ meaning that $\I \subset \sqrt{\shAnn{\struct{X}}{\F}}$ and $\I$ is coherent so $\I^k \subset \shAnn{\struct{X}}{\F}$ for sufficiently large $k$.
\bigskip\\
Because $\iota : Z \embed X$ is a closed immersion, $\iota_*$ induces an equivalence of categories from coherent $\struct{Z}$-modules to coherent $\struct{X}$-modules killed by $\I$. Furthermore, $\I \cdot (\F_i/\F_{i+1}) = 0$ so $\F_i / \F_{i + 1} = \iota_* \G_i$ for a coherent $\struct{Z}$-module $\G_i$. From the exact sequence,
\begin{center}
\begin{tikzcd}
0 \arrow[r] & \F_{i+1} \arrow[r] & \F_i \arrow[r] & \iota_* \G_i \arrow[r] & 0
\end{tikzcd}
\end{center}
we see that $[\F_i] = [\F_{i+1}] + [\iota_* \G_i]$ and thus,
\[ [\F] = [\iota_* \G_0] + [\iota_* \G_1] + \cdots + [\iota_* \G_n] = \iota_* ([\G_0] + [\G_1] + \cdots + [\G_n]) \in \Im{\iota_* : K(Y) \to K(X)} \]
\end{enumerate}

\subsubsection{6.11}

Let $X$ be a nonsingular curve over an algebraically closed field $k$. 
\begin{enumerate}
\item We define a map $\psi : \Cl{X} \to K(X)$. Let $D = n_1 [P_1] + \cdots + n_k [P_k]$ be a divisor. Then we define,
\[ \phi(D) = \sum_{i = 1}^n n_i [(\iota_{P_i})_* \kappa(P_i)] \]
where $[\F]$ denotes the class of $\F$ in the Grothendieck group. Suppose that $D$ is an effective divisor. Then there is an exact sequence,
\begin{center}
\begin{tikzcd}
0 \arrow[r] & \struct{X}(-D) \arrow[r] & \struct{X} \arrow[r] & \struct{D} \arrow[r] & 0
\end{tikzcd}
\end{center}
showing that, explicitly,
\[ \struct{D} = \prod_{i = 1}^n (\iota_{P_i})_* \kappa(P_i)^{n_i} \]
and therefore $\psi(D) = [\struct{D}] = [\struct{X}] - [\struct{X}(-D)]$. For any divisor $D$ write $D = D_+ - D_-$ then,
\[ \psi(D) = \psi(D_+) - \psi(D_-) = [\struct{X}] - [\struct{X}(-D_+)] - [\struct{X}] + [\struct{X}(-D_-)] = [\struct{X}] - [\struct{X}(D_- - D_+)] \]
Therefore, since $\struct{X}(-D)$ depends only on the linear equivalence class of $D$ we see that $\psi(D)$ also descends to the quotient as  map $\psi : \Cl{X} \to K(X)$.

\item Let $\L$ be an ample line bundle on $X$ say given by an embedding into projective space. For any coherent sheaf $\F$ on $X$, we know $\F \otimes \L^{\otimes n}$ is generated by global sections so we get a surjection $\L^{\otimes -n} \onto \F$ and thus take $\E_0 = \L^{\otimes -n}$ which is locally free. Then let $\E_1 = \ker{(\E_0 \onto \F)}$. For each $x \in X$ the stalk $\stalk{X}{x}$ is either a field or a DVR and thus because $(\E_0)_x$ is free the submodule $(\E_1)_x \subset (\E_0)_x$ is also free. Since $\E_1$ is coherent we find that $\E_1$ is also locally free giving a resolution,
\begin{center}
\begin{tikzcd}
0 \arrow[r] & \E_1 \arrow[r] & \E_0 \arrow[r] & \F \arrow[r] & 0
\end{tikzcd}
\end{center}
Let $r_i = \rank{\E_i}$ and define,
\[ \det{\F} = \left( \bigwedge^{r_0} \E_0 \right) \otimes \left( \bigwedge^{r_1} \E_1 \right)^{-1} \in \Pic{X} \]
Now suppose we have a second locally free resolution,
\begin{center}
\begin{tikzcd}
0 \arrow[r] & \E_1' \arrow[r] & \E_0' \arrow[r] & \F \arrow[r] & 0
\end{tikzcd}
\end{center}
Consider the diagram,
\begin{center}
\begin{tikzcd}
& 0  \arrow[d] & 0  \arrow[d] & 0  \arrow[d]
\\
0 \arrow[r] & \E_1 \arrow[d] \arrow[r] & \E_0 \arrow[d] \arrow[r] & \F \arrow[d] \arrow[r] & 0
\\
0 \arrow[r] & \G \arrow[d] \arrow[r] & \E_0 \oplus \E_0' \arrow[d] \arrow[r] & \F \arrow[d] \arrow[r] & 0
\\
0 \arrow[r] & \E_0' \arrow[d] \arrow[r] & \E_0'  \arrow[d] \arrow[r] & 0  \arrow[d] \arrow[r] & 0
\\
& 0 & 0 & 0
\end{tikzcd}
\end{center}
where $\G = \ker{(\E_0 \oplus \E_0' \to \F \oplus \F)}$ is locally free since it is a submodule of a locally free module. Since the rows are exact and the second and third column are exact we get an exact sequence,
\begin{center}
\begin{tikzcd}
0 \arrow[r] & \E_1 \arrow[r] & \G \arrow[r] & \E_0' \arrow[r] & 0
\end{tikzcd}
\end{center} 
and swaping the resolutions gives an exact sequence,
\begin{center}
\begin{tikzcd}
0 \arrow[r] & \E_1' \arrow[r] & \G \arrow[r] & \E_0 \arrow[r] & 0
\end{tikzcd}
\end{center} 
Therefore, $r_1 + r_0' = r_1' + r_0$ and there are isomorphisms,
\[ \bigwedge^{\mathrm{top}} \G \cong \left( \bigwedge^{r_1} \E_1 \right) \otimes \left( \bigwedge^{r_0'} \E_0' \right) \cong \left( \bigwedge^{r_1'} \E_1' \right) \otimes \left( \bigwedge^{r_0} \E_0 \right)  \]
Rearanging gives,
\[ \left( \bigwedge^{r_0} \E_0 \right) \otimes \left( \bigwedge^{r_1} \E_1 \right)^{-1} \cong \left( \bigwedge^{r_0'} \E_0' \right) \otimes \left( \bigwedge^{r_1'} \E_1' \right)^{-1} \]
and therefore $\det{\F}$ is well-defined in $\Pic{X}$. Furthermore, given an exact sequence,
\begin{center}
\begin{tikzcd}
0 \arrow[r] & \F' \arrow[r] & \F \arrow[r] & \F'' \arrow[r] & 0
\end{tikzcd}
\end{center}
we can choose a resolution,
\begin{center}
\begin{tikzcd}
& 0 \arrow[d] & 0 \arrow[d] & 0 \arrow[d]
\\
0 \arrow[r] & \E_1' \arrow[d] \arrow[r] & \E_1 \arrow[d] \arrow[r] & \E_1'' \arrow[d] \arrow[r] & 0
\\
0 \arrow[r] & \E_0' \arrow[d] \arrow[r] & \E_0 \arrow[d] \arrow[r] & \E_0'' \arrow[d] \arrow[r] & 0
\\
0 \arrow[r] & \F' \arrow[d] \arrow[r] & \F \arrow[d] \arrow[r] & \F'' \arrow[d] \arrow[r] & 0
\\
& 0 & 0 & 0
\end{tikzcd}
\end{center}
with exact rows and columns. To see this is possible, first resolve $\F''$ then let $\E_0$ be a locally free surjecting onto $\F \oplus \E_0''$ and $\E_1 = \ker{(\E_0 \to \F)}$ which is locally free as well and there automatically exists a surjection $\E_1 \onto \E_1''$ because the columns are exact and the horizontal maps $\E_0 \to \E_0''$ and $\F \to \F''$ are surjective. Finally, let $\E_i' = \ker{(\E_i \to \E_i'')}$ which are locally free and by the nine lemma this final column is exact. Therefore, we see,
\begin{align*}
\det{\F} & = \left( \bigwedge^{\mathrm{top}} \E_0 \right) \otimes \left( \bigwedge^{\mathrm{top}} \E_1 \right)^{-1} 
\\
& = \left( \bigwedge^{\mathrm{top}} \E_0' \right) \otimes \left( \bigwedge^{\mathrm{top}} \E_0'' \right) \otimes \left( \bigwedge^{\mathrm{top}} \E_1' \right)^{-1} \otimes \left( \bigwedge^{\mathrm{top}} \E_1'' \right)^{-1} = (\det{\F'}) \otimes (\det{\F''}) 
\end{align*}
In particular $\det{(\F \oplus \F')} = (\det{\F}) \otimes (\det{\F'})$ so $\det$ is a group map and preserves the relation $[\F] = [\F'] + [\F'']$ thus factoring through the Grothendieck group to give a group homomorphism $\det : K(X) \to \Pic{X}$. Finally, if $D$ is a divisor, then there is an exact sequence,
\begin{center}
\begin{tikzcd}
0 \arrow[r] & \struct{X}(-D) \arrow[r] & \struct{X} \arrow[r] & \struct{D} \arrow[r] & 0
\end{tikzcd}
\end{center}
which is a locally free resolution of $\struct{D}$. Therefore,
\[ \det{\struct{D}} = \struct{X} \otimes \struct{X}(-d)^{\otimes - 1} = \struct{X}(D) \]
and thus since $\psi(D) = [\struct{D}]$ we find that $\det{\psi(D)} = \struct{X}(D)$.
\item Let $\F$ be a coherent sheaf of rank $r$. After twisting by $\L$ we know $\F \otimes \L^{\otimes n}$ is generated by global sections so we can choose global sections $s_1, \dots, s_r \in H^0(X, \F \otimes \L^{\otimes n})$ which give a basis of $\F_{\xi}$ over the function field $k(X) = \stalk{X}{\xi}$ where $\xi \in X$ is the generic point. Then, the map $\L^{\oplus -r} \to \F$ is an isomorphism at the generic point. Since $\L^{\oplus -r}$ is locally free and $X$ is integral, this implies that $\L^{\oplus -r} \to \F$ is injective. Let $\sT = \coker{(\L^{\oplus -r} \to \F)}$ giving an exact sequence,
\begin{center}
\begin{tikzcd}
0 \arrow[r] & \L^{\oplus -r} \arrow[r] & \F \arrow[r] & \sT \arrow[r] & 0
\end{tikzcd}
\end{center}
However, we can choose $\L = \struct{X}(-D)$ for an antiample divisor $D$ and furthermore, because the first map is an isomorphism at the generic point, $\sT_\xi = 0$ and $\sT$ is coherent so it is a torsion sheaf. Therefore, in $K(X)$ we see $[\F] = r \cdot [\struct{X}(D)] + [\sT]$. The support of $\sT$ is a closed and thus finite set $S \subset X$ and elsewhere $\sT_x = 0$ so,
\[ \sT = \bigoplus_{P \in S} \sT_P \]
Let $M = \sT_P$ and $A = \stalk{X}{P}$.
Then $M$ is a finite length $A$-module because $M \otimes_A \Frac{A} = 0$ meaning that $\m$ is a minimal prime of the support ($\dim{A} = 1$) so $\dim{A / \Ann{A}{M}} = 0$ and thus $A / \Ann{A}{M}$ is Artinian but $M$ is a finitely generated $A / \Ann{A}{M}$-module so $\length{A}{M}$ is finite. Since $\m$ is nilpotnent on $A/\Ann{A}{M}$ we see that $\m^n M = 0$ so consider,
\begin{center}
\begin{tikzcd}
0 \arrow[r] & \m^{i+1} M \arrow[r] & \m^i M \arrow[r] & \m^i M / \m^{i+1} M \arrow[r] & 0
\end{tikzcd}
\end{center}
and $\m^i M / \m^{i+1} M = \m^i M \otimes_A A / \m \cong (A / \m)^{\oplus n}$. Thus in $K(A)$ we see $[M] = \length{A}{M} \cdot [A/\m]$ and therefore,
\[ [\sT] = \sum_{P \in S} \length{\stalk{X}{P}}{\sT_P} \cdot [(\iota_P)_* \kappa(P)] = \psi(D') \quad \text{ where } \quad D' = \sum_{P \in S} \length{\stalk{X}{P}}{\sT_P} \cdot [P] \]
Then, because $[\struct{X}(D)] = [\struct{X}] + \psi(D)$ we have,
\[ [\F] = r [\struct{X}(D)] + \psi(D') = r [\struct{X}] + r \psi(D) + \psi(D') = r [\struct{X}] + \psi(r D + D') \]
Therefore $[\F] - r [\struct{X}] = \psi(rD + D')$ and furthermore, 
\[ \det{\F} = \struct{X}(D)^{\otimes r} \otimes \det{\sT} = \struct{X}(rD) \otimes \struct{X}(D) = \struct{X}(rD + D') \]
so we find that $[\F] = r [\struct{X}] + \psi(\det{\F})$.

\item We have maps $\psi : \Pic{X} \to K(X)$ and $[\struct{X}] : \Z \to K(X)$ and $\det : K(X) \to \Pic{X}$ and $\rank : K(X) \to \Z$. We have shown that given any $[\F] \in K(X)$ we have,
\[ [\F] = \rank{(\F)} \cdot [\struct{X}] + \psi(\det{\F}) \]
and furthermore $\det{\psi(D)} = \struct{D}$ and $\rank{(r [\struct{X}])} = r$. Therefore, these maps give the structure maps of a biproduct so $K(X) \cong \Pic{X} \oplus \Z$. 
\end{enumerate}

\subsubsection{6.12}

Let $X$ be a nonsingular curve. Then there is a degree function $\deg : K(X) \to \Z$ i.e. an assignment $\F \mapsto \deg{\F} \in \Z$ for any coherent $\struct{X}$-module satisfying,
\begin{enumerate}
\item $\deg{\struct{X}(D)} = \deg{D}$ for any divisor $D$
\item if $\F$ is a torsion sheaf ($\F_\xi = 0$ where $\xi \in X$ is the generic point) then,
\[ \deg{\F} = \sum_{x \in X} \length{\stalk{X}{x}}{\F_x} \]
\item for any exact sequence of coherent $\struct{X}$-modules,
\begin{center}
\begin{tikzcd}
0 \arrow[r] & \F' \arrow[r] & \F \arrow[r] & \F'' \arrow[r] & 0
\end{tikzcd}
\end{center}
then $\deg{\F} = \deg{\F'} + \deg{\F''}$.
\end{enumerate}
I first claim there is a unique such function. Any coherent sheaf $\F$ fits into an exact sequence,
\begin{center}
\begin{tikzcd}
0 \arrow[r] & \struct{X}(D)^{\oplus r} \arrow[r] & \F \arrow[r] & \sT \arrow[r] & 0
\end{tikzcd}
\end{center}
where $\sT$ is a torsion sheaf. Therefore, for any degree function
\[ \deg{\F} = \deg{\struct{X}(D)^{\oplus r}} + \deg{\sT} = r \deg{D} + \sum_{x \in X} \length{\stalk{X}{x}}{\sT_x} \]
where $\deg{\F^{\oplus n}} = n \deg{\F}$ by the third property. Therefore, there is a unique such degree function because the above number is fixed. 
\bigskip\\
To show such a degree function exists it suffices to check that $\deg{\F} := \deg{\det{\F}}$ satisfies the given properties. First, note that $\det{\struct{X}(D)} = \struct{X}(D)$ because it is locally free so $\deg{\struct{X}(D)} = \deg{D}$. Next, if $\F$ is a torsion sheaf then we have shown that $\F = \psi(D)$ and $\det{\psi(D)} = \struct{X}(D)$ and $\deg{\F} = \det{D}$ satisfies the formula. Finally, given an exact sequence,
\begin{center}
\begin{tikzcd}
0 \arrow[r] & \F' \arrow[r] & \F \arrow[r] & \F'' \arrow[r] & 0
\end{tikzcd}
\end{center}
we know $\det{\F} = (\det{\F'} \otimes \det{\F''})$ and therefore,
\[ \deg{\F} = \deg{\det{\F}} = \deg{\det{\F'}} + \deg{\det{\F''}} = \deg{\F'} + \deg{\F''} \]

\subsection{8}

\subsubsection{8.1}

\begin{enumerate}
\item Let $B$ be a local ring containing a field $k \subset B$ such that $\kappa = B / \m$ is a separably generated extension of $k$. Consider the exact sequence,
\begin{center}
\begin{tikzcd}
\m / \m^2 \arrow[r, "\delta"] & \Omega_{B/k} \otimes_B \kappa \arrow[r] & \Omega_{\kappa/k} \arrow[r] & 0
\end{tikzcd}
\end{center}
we want to prove exactness on the left.
Consider the $\kappa$-dual,
\[ \delta^* : \Hom{\kappa}{\Omega_{B/k} \otimes_B \kappa}{\kappa} \to \Hom{\kappa}{\m/\m^2}{\kappa} \]
Let $A = B / \m^2$ then there is an exact sequence,
\begin{center}
\begin{tikzcd}
\m^2 / \m^4 \arrow[r] & \Omega_{B/k} \otimes_B A \arrow[r] & \Omega_{A/k} \arrow[r] & 0
\end{tikzcd}
\end{center}
and thus we get a surjection,
\begin{center}
\begin{tikzcd}
\Omega_{B/k} \otimes_B \kappa \arrow[r] & \Omega_{A/k} \otimes_A \kappa \arrow[r] & 0
\end{tikzcd}
\end{center}
Therefore, taking the $\kappa$-dual gives an injection,
\[ \Hom{\kappa}{\Omega_{A/k} \otimes_A \kappa}{\kappa} \embed \Hom{\kappa}{\Omega_{B/k} \otimes_B \kappa }{\kappa} \]
It suffices to show that,
\[ \Hom{\kappa}{\Omega_{A/k} \otimes_A \kappa}{\kappa} \embed \Hom{\kappa}{\Omega_{B/k} \otimes_B \kappa }{\kappa} \xrightarrow{\delta^*} \Hom{\kappa}{\m/\m^2}{\kappa} \]
is surjective. However,
\[ \Hom{\kappa}{\Omega_{A/k} \otimes_A \kappa}{\kappa} = \Hom{A}{\Omega_{A/k}}{\kappa} = \Der{k}{A}{\kappa} \]
giving the natural map $\Der{k}{A}{\kappa} \to \Hom{\kappa}{\m/\m^2}{\kappa}$ by restriction which is well-defined because any $k$-derivation $\varphi : A \to \kappa$ satisfies $\varphi(\m^2) = 0$.
\bigskip\\
Since $A$ is a complete local ring, by 8.25A, there is a subfield $K \subset A$ such that $K \to A \to \kappa$ is an isomorphism. Given any $\kappa$-linear map $\varphi : \m / \m^2 \to \kappa$ we construct a $K$-derivation $D : A \to \kappa$ as follows. Since $K \to A \to \kappa$ is an isomorphism, for any $a \in A$ we can write $a = \lambda + c$ with $\lambda \in K$ and $c \in \m$ uniquely. Define $D(a) = \varphi(c)$ then,
\[ D(aa') = D(\lambda \lambda' + \lambda c' + \lambda' c + c c') = \lambda \varphi(c') + \varphi(c) \lambda' = a D(a') + D(a) a' \]
since $\varphi$ is $\kappa$ linear it is also $K$-linear under $K \subset A \to \kappa$. Therefore, $\delta^*$ is surjective so $\delta$ is injective.

\item Assume furthermore $k$ is perfect and $B$ is a localization of a finite type $k$-algebra. 
\bigskip\\
The exact sequence,
\begin{center}
\begin{tikzcd}
0 \arrow[r] & \m / \m^2 \arrow[r] & \Omega_{B/k} \otimes_B \kappa \arrow[r] & \Omega_{\kappa/k} \arrow[r] & 0
\end{tikzcd}
\end{center}
shows that,
\[ \dim_\kappa \Omega_{B/k} \otimes_B \kappa = \dim_\kappa \m/\m^2 + \dim_\kappa \Omega_{\kappa/k} \]
Furthermore, $\dim_\kappa \Omega_{\kappa / k} = \trdeg{k}{\kappa}$ because $\kappa / k$ is separably generated. Therefore,
\[ \dim_\kappa \Omega_{B/k} \otimes_B \kappa = \dim_k \m/\m^2 + \trdeg{k}{\kappa} \]
and thus if $B$ is regular if and only if
\[ \dim_\kappa \Omega_{B/k} \otimes_B \kappa = \dim{B} + \trdeg{k}{\kappa} \]
In particular, if $\Omega_{B/k}$ is free of rank $\dim{B} + \trdeg{k}{\kappa}$ then $B$ is regular. 
\bigskip\\
Now suppose that $B$ is regular. Then $B$ is a domain so let $K = \Frac{A}$,
\[ \Omega_{B/k} \otimes_B K = \Omega_{K/k} \]
and $K / k$ is separably generated because $k$ is perfect so again,
\[ \dim_{K} \Omega_{B/k} \otimes_B K = \trdeg{k}{K} \]
We know $B$ is the localization of a finite type $k$-algebra domain $A$ so $K = \Frac{A}$ and,
\[ \trdeg{k}{K} = \dim{A} = \dim{B} + \trdeg{k}{\kappa} \]
Therefore, $\Omega_{B/k}$ is free of rank $\dim{B} + \trdeg{k}{\kappa}$ by Lemma 8.9.

\item Let $X$ be an irreducible scheme of finite type over a perfect field $k$. Let $\dim{X} = n$. For $x \in X$, we know $B = \stalk{X}{x}$ is a localization of a finite type $k$-algebra and $(\Omega_{X/k})_x = \Omega_{B/k}$. By the previous part, $B$ is a regular local ring iff $\Omega_{B/k}$ is free of rank $\dim{B} + \trdeg{k}{\kappa} = n$. Therefore, 
\[ \stalk{X}{x} \text{ is regular } \iff (\Omega_{X/k})_x \text{ is free of rank } n \]

\item Let $X$ be an irreducible scheme of finite type over a perfect field.
\bigskip\\
If $\stalk{X}{x}$ is regular then $(\Omega_{X/k})_x$ is free of rank $n$. Since $X$ is noetherian and $\Omega_{X/k}$ is coherent, then $\Omega_{X/k}$ is free of rank $n$ on a neighborhood on $x$ by [Ex. 5.7]. Therefore,
\[ U = \{x \in X \mid \stalk{X}{x} \text{ is regular} \} \]
is open. 
\bigskip\\
Assume further that $X$ is integral. Then $\dim{X} = \trdeg{k}{K}$ where $K$ is the function field. Let $\xi \in X$ be the generic point. Because $k$ is perfect, $K / k$ is seperably generated and thus $(\Omega_{X/k})_\xi = \Omega_{K/k}$ is free of rank $\trdeg{k}{K}$ so $\xi \in U$. Therefore $U$ is a dense open.
\end{enumerate}

\subsubsection{8.2}

Let $X$ be a variety over $k$ of dimension $n$ and let $\E$ be a locally free sheaf on $X$ of rank $r$ with $r > n$ and let $V \subset \Gamma(X, \E)$ be a vector space of global sections which globally generate $\E$. Then $V \otimes_k \struct{X} \onto \E$ is surjective so in particular $V \otimes_k \kappa(x) \onto \E_x \otimes_{\stalk{X}{x}} \kappa(x)$ meaning that $\dim_k{V} \ge \rank{(\E)} > \dim{X}$. Write $d = \dim_k{V}$ then $d \ge r > n$. 
Consider the bad locus $B \subset X \times \P(V)$ defined by,
\[ B = \{ (x, [s]) \mid x \in X \text{ and } s \in V \setminus \{ 0 \} \text{ and } s_x \in \m_x \E_x \} \]
The fibers of the map $B \to X$ correspond to $[s] \in \P(V)$ such that $s_x \in \m_x \E_x$ i.e. the kernel of the map $V \otimes_k \kappa(x) \onto \E_x \otimes \kappa(x)$. Thus the fibers of $B \to X$ are isomorphic to $\P^{d - r - 1}$ and thus $\dim{B} = n + d - r - 1$. Now consider the projection $B \to \P(V)$. However,  $n < r$ and thus $\dim{B} < d - 1 = \dim{\P(V)}$ meaning that $B \to \P(V)$ cannot be surjective. Therefore, there exists a nonzero $s \in V$ such that $s_x \notin \m_x \E_x$ for every $x \in X$.
\bigskip\\
Consider the map $s : \struct{X} \to \E$. On stalks, $s_x : \stalk{X}{x} \to \E_x$ is injective because $s_x \neq 0$ and $\E_x$ is free and $\stalk{X}{x}$ is a domain. Thus $s : \struct{X} \to \E$ is injective. Let $\E' = \coker{(s : \struct{X} \to \E)}$. At each $x \in X$ there is an exact sequence,
\begin{center}
\begin{tikzcd}
0 \arrow[r] & \stalk{X}{x} \arrow[r, "s_x"] & \E_x \arrow[r] & \E'_x \arrow[r] & 0
\end{tikzcd}
\end{center}
furthermore, tensoring by $\kappa(x)$ we get an exact sequence,
\begin{center}
\begin{tikzcd}
0 \arrow[r] & \kappa(x) \arrow[r] & \E_x \otimes_{\stalk{X}{x}} \kappa(x) \arrow[r] & \E'_x \otimes_{\stalk{X}{x}} \kappa(x) \arrow[r] & 0
\end{tikzcd}
\end{center}
which is exact on the left because $s_x \notin \m_x \E_x$ so the map $\kappa(x) \to \E_x / \m_x \E_x$ is nonzero and thus injective because these are vector spaces. Therefore $\rank_x(\E') = \rank_x(\E) - 1 = r - 1$ which is constant. Therefore, since $X$ is noetherian and reduced and $\E'$ is coherent we find that $\E'$ is locally free by II.5.8 (c).

\subsubsection{8.3}

\begin{enumerate}
\item Let $X$ and $Y$ be schemes over a base $S$. Consider the cartesian diagram,
\begin{center}
\begin{tikzcd}
X \times_S Y \arrow[r, "\pi_2"] \arrow[d,"\pi_1"'] & Y \arrow[d, "g"]
\\
X \arrow[r, "f"'] & S 
\end{tikzcd}
\end{center}
There are exact sequences,
\begin{center}
\begin{tikzcd}
\pi_1^* \Omega_{X/S} \arrow[r] & \Omega_{X \times_S Y/S} \arrow[r] & \Omega_{X \times_S Y / X} \arrow[r] & 0
\\
\pi_2^* \Omega_{Y/S} \arrow[r] & \Omega_{X \times_S Y/S} \arrow[r] & \Omega_{X \times_S Y / Y} \arrow[r] & 0
\end{tikzcd}
\end{center}
Furthermore, from the base change diagram and (8.10),
\[ \Omega_{X \times_S Y/X} = \pi_2^* \Omega_{Y/S} \text{ and likewise } \Omega_{X \times_S Y/Y} = \pi_1^* \Omega_{Y/S} \]
The natural map $q : \Omega_{X \times_S Y/S} \to \Omega_{X \times_S Y/Y} = \pi_1^* \Omega_{Y/S}$ satisfies $q \circ \pi_1^* = \id$ because locally these maps are $\Omega_{A \otimes_R B/R} \to \Omega_{A \otimes_R B/B} = \Omega_{A/R} \otimes_R B$ and $\Omega_{A/R} \otimes_R B \to \Omega_{A \otimes_R B/R}$.
Thus we get a split exact sequence (exactness on the left follows from $\pi_1^*$ having a left inverse),
\begin{center}
\begin{tikzcd}
0 \arrow[r] & \pi_1^* \Omega_{X/S} \arrow[r] & \Omega_{X \times_S Y/S} \arrow[l, bend right, "q"'] \arrow[r] & \pi_2^* \Omega_{Y/S} \arrow[r] & 0
\end{tikzcd}
\end{center} 
and therefore $\Omega_{X \times_S Y/S} = \pi_1^* \Omega_{X/S} \oplus \pi_2^* \Omega_{Y/S}$.

\item If $X$ and $Y$ are nonsingular varieties over $k$ then $\Omega_{X} = \Omega_{X/k}$ and $\Omega_{Y} = \Omega_{Y/k}$ are finite locally free and $X \times Y$ is nonsingular over $k$ so $\Omega_{X \times Y} = \Omega_{X \times Y / k}$ is also locally free. By (a) there is a split exact sequence,
\begin{center}
\begin{tikzcd}
0 \arrow[r] & \pi_1^* \Omega_X \arrow[r] & \Omega_{X \times Y} \arrow[r] & \pi_2^* \Omega_Y \arrow[r] & 0
\end{tikzcd}
\end{center}
Since these are vector bundles this sequence gives an isomorphism of top exterior powers,
\[ \det{\Omega_{X \times Y}} = \det{\pi_1^* \Omega_X} \otimes_{\struct{X}} \det{\pi_2^* \Omega_Y} = \pi_1^* \det{\Omega_X} \otimes_{\struct{X}} \pi_2^* \det{\Omega_Y} \]
However, since these varieties are nonsingular, the canonical sheaf is $\omega_X = \det{\Omega_X}$ so we get,
\[ \omega_{X \times Y} = \pi_1^* \omega_X \otimes_{\struct{X}} \pi_2^* \omega_Y \]

\item Let $Y$ be a nonsingular plane cubic curve, and let $X = Y \times Y$. Since $Y \embed \P^2_k$ as a hypersurface, we know $\omega_Y = \struct{Y}(d - n + 1) = \struct{Y}$ because $d = 3$ and $n = 2$.  Now $\omega_X = \pi_1^* \omega_Y \otimes_{\struct{X}} \pi_1^* \omega_Y = \struct{X}$. Since $X$ is proper, $H^0(X, \omega_X) = H^0(X, \struct{X}) = k$ and thus $p_g(X) = 1$. However, the arithmetic genus is,
\[ p_a(X) = (-1)^{\dim{X}} (\chi(\struct{X}) - 1) \]
Therefore, in our case,
\[ p_a(X) = \dim_k H^2(X, \struct{X}) - \dim_k H^1(X, \struct{X}) \]
By K\"{u}nneth,
\begin{align*}
H^1(X, \struct{X}) & = H^1(Y, \struct{Y}) \otimes_k H^0(Y, \struct{Y}) \oplus H^0(Y, \struct{Y}) \otimes_k H^1(Y, \struct{Y}) = k \oplus k 
\\
H^2(X, \struct{X}) & = H^1(Y, \struct{Y}) \otimes_k H^1(Y, \struct{Y}) = k 
\end{align*}
Therefore, $p_a(X) \neq p_g(X)$.
\end{enumerate}


\subsubsection{8.4}

A closed subscheme $U \subset \P^n_k$ is called a (global) complete intersection if the homogeneous ideal $I \subset S = k[x_0, \dots, x_n]$ can be generated by $r = \codim{Y, \P^n}$ elements. Let $\m = (x_0, \dots, x_n)$.

\begin{rmk}
It is equivalent that $I$ be generated by $r$ homogeneous elements by \href{https://mathoverflow.net/questions/15584/minimal-number-of-generators-of-a-homogeneous-ideal-exercise-in-hartshorne}{this} answer. I will prove this here.
\end{rmk}

\begin{lemma} \label{controlled_at_cone_point}
Let $I \subset S$ be a homogeneous ideal. Then the following numbers are equal,
\begin{enumerate}
\item $\dim_{k} (I_\m \otimes_{S_\m} k)$
\item the minimal number of generators of $I_\m$
\item the minimal number of generators of $I$
\item the minimal number of homogeneous generators of $I$
\end{enumerate}
\end{lemma}


\begin{proof}
Call these numbers $a,b,c,d$. Clearly $a \le b \le c \le d$ so we just need to show that $d \le a$. Let $I = (f_1, \dots, f_d)$ is a minimal set of homogeneous generators. Then consider,
\begin{center}
\begin{tikzcd}
0 \arrow[r] & K \arrow[r] & S^{\oplus d} \arrow[r] & I \arrow[r] & 0
\end{tikzcd}
\end{center}
The module $K$ consists of relations,
\[ \sum_{i = 1}^d f_i g_i = 0 \]
but if $g_i$ had a constant term for some $i$ then taking the homogeneous part of degree $\deg{f_i}$ we would get a formula for $f_i$ from the other $f_j$ (the next term coming from $f_i g_i$ would have larger degree) contradicting minimality. Thus $g_i \in \m$ so $K \subset \m S^{\oplus d}$. Therefore, applying $- \otimes_S S/\m$ gives $k^{\oplus d} \iso I \otimes_S S/\m$ because the image of $K \to S^{\oplus d}$ is zero after tensoring. Thus,
\[ d = \dim_{k} (I \otimes_S S/\m) = \dim_k I_\m \otimes_{S_\m} k = a \]
\end{proof}

\begin{lemma} 
Let $I \subset S$ be a homogeneous ideal. The following are equivalent,
\begin{enumerate}
\item $I$ is saturated
\item for any $f \in S$ if for each $i$ there is some $n_i$ such that $x_i^{n_i} f \in I$ then $f \in I$
\item for any $f \in S$ if there is some $N$ such that $x_0^N f, \dots, x_n^N f \in I$ then $f \in I$
\item for any $f \in S$ if $f \m \subset I$ then $f \in I$
\end{enumerate}
\end{lemma}

\begin{proof}
The first three are obviously equivalent. It is clear that saturated $\implies$ (d). Now suppose that for any $f \in S$ if $f \m \subset I$ then $f \in I$. Then suppose that $x_i^{N} f \in I$ for each $i$ then let $y = x_0^{e_0} \cdots x_n^{e_n} f$ where these powers are maximal such that $y \notin I$ meaning if any power is increased then $y \in I$. This is possible because we already know that $x_i^N f \in I$ for each $i$ so all chains have finite length. Then $x_0 y, \dots, x_n y \in I$ by minimality so $y \m \subset I$ and thus $y \in I$ contradicting the definition of $y$ unless there are no such $y \notin I$ meaning that $f \in I$ so $I$ is saturated. However, by the noetherian property, repeating this construction to give $I \subset I' \subset I'' \subset \cdots$ must terminate in finitely many steps and, by the lemma, produces a saturated ideal which is therefore the saturation.
\end{proof}

\begin{rmk}
Although these are equivalent for \textit{checking} that $I$ is \textit{saturated} notice that they are not equivalent for \textit{producing} the \textit{saturation} of an ideal. Indeed if $I'$ is the set of $f \in S$ with $f \m \subset I$ then $I'$ need not be saturated exactly because if $I$ is not saturated it is possible for $f \m^2 \subset I$ without $f \m \subset I$. For example, let $I = (x^3, xy)$ then its saturation should be $(x,y)$ but $x \m \not \subset I$. 
\end{rmk}

\begin{lemma}
Let $I \subset S$ be a homogeneous ideal. Then $I$ is saturated if and only if $\m/J \subset (S/I)$ is an embedded prime.
\end{lemma}

\begin{proof}
If $\m$ is not embedded then $\m/J$ does not consist only of zero divisors in $S/J$. Therefore,  if $f \m \subset J$ then $f \in J$ meaning that $J$ is saturated. Conversely, suppose that $\m$ is embedded then $\m = \Ann{S/J}{f}$ for some $f \in S$ so $f \m \subset J$ but if $f \in J$ then $\m = J$ would be minimal and hence not embedded so $f \notin J$. Thus $I$ is not saturated. 
\end{proof}


\begin{enumerate}
\item Let $\codim{Y, \P^n} = r$. If $Y$ is a complete intersection then let $I = (f_1, \dots, f_r)$ for homogeneous elements $f_i \in I$. Therefore, 
\[ \I_Y = \wt{I} = \wt{(f_1)} + \cdots + \wt{(f_r)} \]
and clearly $\wt{(f_i)} = \I_{H_i}$ where $H_i = V(f_i)$ is a hypersurface.
\bigskip\\
Conversely, suppose that $Y$ is cut out by the sheaf of ideals,
\[ \I_Y = \I_{H_1} + \cdots + \I_{H_r} \]
where $H_i$ is a hypersurface. Since $H_i$ is a prime Cartier divisor and $\Pic{\P^n_k} = \Z$ we must have $\struct{\P^n_k}(H_i) = \struct{\P^n_k}(d_i)$ and $H_i = V(f_i)$ where $f_i \in H^0(\P^n_k, \struct{\P^n_k}(d_i))$ is an irreducible homogeneous polynomial of degree $d_i$. We need to show that $I(Y) = J$ where $J = (f_1, \dots, f_r)$. Applying [Ex. 5.10] it suffices to show that $\I_Y = \wt{J}$ and that $J$ is saturated because $I = \Gamma_*(\I_Y)$ is the unique saturated ideal cutting out $Y$. First, 
\[ \I_Y(D_+(x_i)) = \I_{H_1}(D_+(x_i)) + \cdots + \I_{H_r}(D_+(x_i)) = (f_1/x_i^{\deg{f_1}}, \dots, f_r/x_i^{\deg{f_r}}) = J_{(x_i)} = \wt{J}(D_+(x_i)) \]
so $\I_Y = \wt{J}$. Thus $\height{J} = r$ and $J$ is generated by $r$ elements so by the unmixedness property of $S$, we see that $J$ is unmixed meaning that the only associated primes of $S/J$ are minimal. Therefore, since $\m \subset S$ is not minimal over $J$ (because $S/J$ has positive dimension since it is a cone over a nonempty variety) it cannot be associated meaning that $\m/J$ must contain some non zerodivisor. Therefore, for any $y \in S$ if $y \m \subset J$ then $y \in J$ (this is just a restatement of if $\bar{y} \in S/J$ has $\bar{y} \m = 0$ then $\bar{y} = 0$). (Or just use the lemma above). Therefore, by the lemma, $J$ is saturated.

\item Let $Y$ be a complete intersection of dimension $r \ge 1$ in $\P^n$ and $Y$ is normal. Consider $S(Y) = S/I$ and $C(Y) = \Spec{S(Y)}$ the cone over $Y$. Because $I$ is generated by $r$ elements, then $C(Y)$ is a locally complete intersection in $\A^{n+1}$. Therefore, by [Prop. 2.23] $C(Y)$ is normal if and only if $C(Y)$ is regular in codimension $1$. Since $\dim{Y} \ge 1$ we see that $\dim{C(Y)} \ge 2$ so $C(Y)$ is regular in codimension $1$ if and only if $C(Y) \setminus \{ 0 \}$ is regular in codimension $1$. However, $C(Y) \setminus \{ 0 \}$ is isomorphic to the total space of the line bundle $\struct{Y}(-1)$ and therefore is locally isomorphic to $U \times \A^1$ so by [Prop. 2.6.6] it is regular in codimension $1$. Therefore $C(Y)$ is normal so $S(Y)$ is an integrally closed domain and thus $Y$ is projectively normal.

\item Using [Ex. 5.14(d)], $Y$ being projectively normal implies that $\Gamma(\P^n, \struct{\P^n}(d)) \onto \Gamma(Y, \struct{Y}(d))$ is surjective.

\item Let $(d_1, \dots, d_r)$ be positive integers with $r < n$. Proceed by induction on $r$. For $r = 1$ a general hypersurface $H \subset \P^n$ of degree $d_1$ is irreducible and nonsingular of codimension $1$. Now given $Y = H_1 \cap \cdots \cap H_{k} \subset \P^n$ which is irreducible and nonsingular of codimension $k$. Consider the $d_k$-uple embedding $\P^n \embed \P^{N}$ pulls back hyperplanes to hypersurfaces of degree $d_k$. Consider the nonsingular irreducible closed subscheme $Y \embed \P^n \embed \P^N$. Since $k < r < n$ then $\dim{Y} \ge 2$. By Bertini's theorem, for a general hyperplane $H \subset \P^N$ we have $Y \cap H$ is nonsingular and irreducible of codimension $N - n + k + 1$ in $\P^N$. Then letting $H_{d_k} = H \cap \P^n$ which is a general hypersurface of degree $d_k$ we see that $Y \cap H_{d_k} \cong Y \cap H$ is nonsingular and irreducible of codimension $k + 1$ proving the claim by induction. 

\item Let $Y \subset \P^n_k$ be a nonsingular complete intersection. Consider the surjection, 
\[ \I_{H_1} \oplus \cdots \oplus \I_{H_r} \onto \I_Y \]
where $Y = H_1 \cap \cdots \cap H_r$ and $H_i = V(f_i)$ and $\struct{\P^n_k}(H_i) = \struct{\P^n_k}(d_i)$ with $\struct{\P^n_k}(-d_r) \to \I_Y$ via multiplication by $f_i$ giving a surjection,
\[ \bigoplus_{i = 1}^r \struct{\P^n_k}(-d_i) \onto \I_Y \]
Now the conormal sheaf is,
\[ \C_{Y/\P^n_k} = \I_Y / \I_Y^2 = \iota^* \I_Y \]
Consider the pullback,
\[ \bigoplus_{i = 1}^r \struct{Y}(-d_i) \onto \C_{Y/\P^n_k} \]
Furthermore, since $Y$ is nonsingular, at each $y \in Y$ we know $\stalk{Y}{y} = \stalk{\P^n_k}{y} / I$ where $I = (f_1, \dots, f_r)$ and since $\codim{Y, \P^n_k} = r$ this is a regular sequence so the map $\stalk{Y}{y}^{\oplus n} \onto I/I^2$ is an isomorphism. Thus the map,
\[ \bigoplus_{i = 1}^r \struct{Y}(-d_i) \iso \C_{Y/\P^n_k} \]
is an isomorphism. From the exact sequence,
\begin{center}
\begin{tikzcd}
0 \arrow[r] & \C_{Y/\P^n_k} \arrow[r] & \iota^* \Omega_{\P^n_k} \arrow[r] & \Omega_Y \arrow[r] & 0
\end{tikzcd}
\end{center}
which are locally free we get,
\begin{align*}
\omega_Y \cong \iota^* \omega_{\P^n_k} \otimes \left( \bigwedge^r \C_{Y/\P^n_k} \right)^\vee & = \struct{Y}(-n-1) \otimes \left( \bigwedge^r \bigoplus_{i = 1}^r \struct{Y}(-d_i) \right)
\\
& = \struct{Y}(-n-1) \otimes \struct{Y}(\sum d_i) = \struct{Y}(\sum d_i - n - 1)
\end{align*}

\item Let $Y$ be a nonsingular hypersurface of degree $d$ in $\P^n$. Then,
\[ p_g(Y) = \dim_k \Gamma(Y, \omega_Y) = \dim_k \Gamma(Y, \struct{Y}(d - n - 1)) \]
Furthermore, $Y$ is normal and a complete intersection so by (c),
\[ \Gamma(\P^n_k, \struct{\P^n_k}(\ell)) \onto \Gamma(Y, \struct{Y}(\ell)) \]
is surjective for all $\ell \ge 0$ and the kernel is $\Gamma(\P^n_k, \I_Y(\ell))$. Because $\struct{Y}(-\ell)$ is anti-very ample it cannot have any sections (else $\L$ and $\L^{\otimes -1}$ both have nonzero sections) so $\Gamma(Y, \struct{Y}(-\ell)) = 0$. Thus we may assume that $d - n - 1 \ge 0$. Then we have an exact sequence,
\begin{center}
\begin{tikzcd}
0 \arrow[r] & \Gamma(\P^n_k, \struct{\P^n_k}(-n-1)) \arrow[r] & \Gamma(\P^n_k, \struct{\P^n_k}(d - n - 1)) \arrow[r] & \Gamma(Y, \struct{Y}(d - n - 1)) \arrow[r] & 0
\end{tikzcd}
\end{center}
Therefore,
\[ p_g(Y) = \dim_k \Gamma(Y, \struct{Y}(d - n - 1)) = \dim_k \Gamma(\P^n_k, \struct{\P^n_k}(d - n - 1)) - \dim_k \Gamma(\P^n_k, \struct{\P^n_k}(- n - 1)) = { d - 1 \choose n }  \]
\item Let $Y$ be a nonsingular curve in $\P^3$ which is a complete intersection of nonsingular surfaces of degrees $d,e$. By (c) there is an exact sequence,
\begin{center}
\begin{tikzcd}
0 \arrow[r] & \Gamma(\P^n_k, \I_Y(\ell)) \arrow[r] & \Gamma(\P^n_k, \struct{\P^n_k}(\ell)) \arrow[r] & \Gamma(Y, \struct{Y}(\ell)) \arrow[r] & 0
\end{tikzcd}
\end{center}
Furthermore, there is an exact sequence,
\begin{center}
\begin{tikzcd}
0 \arrow[r] & \struct{\P^n_k}(-d - e + \ell) \arrow[r] & \struct{\P^n_k}(-d + \ell) \oplus \struct{\P^n_k}(-e + \ell) \arrow[r] & \I_Y \arrow[r] & 0
\end{tikzcd}
\end{center}
Since $n > 1$ this sequence is exact on global sections (vanishing of $H^1$ on line bundles) so we find,
\[ \dim_k \Gamma(\P^n_k, \I_Y(\ell)) = { \ell - d + n \choose n } + { \ell - e + n \choose n } - { \ell - d - e + n \choose n } \] Therefore, we get an exact sequence,
\begin{center}
\begin{tikzcd}[column sep = small]
0 \arrow[r] & \Gamma(\struct{\P^n_k}(\ell - d - e)) \arrow[r] & \Gamma(\struct{\P^n_k}(\ell - d) \oplus \struct{\P^n_k}(\ell - e)) \arrow[r] & \Gamma(\struct{\P^n_k}(\ell)) \arrow[r] & \Gamma(Y, \struct{Y}(\ell)) \arrow[r] & 0
\end{tikzcd}
\end{center}
Now take $\ell = d + e - n - 1$,
\begin{align*}
p_g(Y) & = \dim_k \Gamma(Y, \omega_Y) = \dim_k \Gamma(Y, \struct{Y}(d + e - n - 1)) 
\\
& = {d + e - 1 \choose n } - { e - 1 \choose n } - { d - 1 \choose n } 
\end{align*} 
Since $n = 3$ we get,
\[ p_g(Y) = \tfrac{1}{2} de (e + d - 4) + 1 \]
\end{enumerate}

\begin{rmk}
Notice that the following does not necessarily hold,
\[ \Gamma_*(\I_Y) \neq \Gamma_*(\I_{H_1}) + \cdots + \Gamma_*(\I_{H_1}) \]
so the reverse of (a) is not immediate. This is because $\I_Y = \I_{H_1} + \cdots + \I_{H_r}$ does not imply that,
\[ H^0(\P^n_k, \I_Y) = H^0(\P^n_k,\I_{H_1}) + \cdots + H^0(\P^n_k,\I_{H_r}) \]
For example, let $\I_1 = \wt{(xy)}$ and $\I_2 = \wt{(y^2)}$ on $\P^1_k = \Proj{k[x,y]}$. Then, I claim $\I_1 + \I_2 = \I = \wt{(y)}$ because on $D(x)$ we have $(y/x) + (y^2/x^2) = (y/x)$ and on $D(y)$ we have $(x/y) + (1) = (1)$. However,
\[ \Gamma_*(\I) = (y) \quad \text{and} \quad \Gamma_*(\I_1) + \Gamma_*(\I_2) = (xy) + (y^2) = (xy, y^2) \neq (y) \]
However, the saturation of these ideals are equal.
\end{rmk}

\subsubsection{8.5}

\renewcommand{\Pic}[1]{\mathrm{Pic}(#1)}

Let $X$ be a nonsingular variety and $Y \subset X$ a nonsingular subvariety of codimension $r \ge 2$. Let $\pi : \wt{X} \to X$ be the blowing up of $X$ along $Y$ and $E = \pi^{-1}(Y)$ the exceptional divisor.
\bigskip\\
First note that $\wt{X}$ is a nonsingular variety by [II, 7.16] and [II, 8.24]. 

\begin{enumerate}
\item Consider the diagram,
\begin{center}
\begin{tikzcd}
\Z \arrow[r] & \Pic{\wt{X}} \arrow[r] & \Pic{\wt{X} \setminus E} \arrow[r] & 0
\\
& \Pic{X} \arrow[u, "\pi^*"] \arrow[r, "\sim"] & \Pic{X \setminus Y} \arrow[u, "\sim"']
\end{tikzcd}
\end{center}
where $\Z \to \Pic{\wt{X}}$ sends $n \mapsto n [E]$ where the exceptional divisor $E$ is Cartier. 
Because $\pi : \wt{X} \setminus E \to X \setminus Y$ is an isomorphism, the map $\Pic{X \setminus Y} \iso \Pic{\wt{X} \setminus E}$ is also an isomorphism. Therefore, the sequence is right split. Finally, $E$ corresponds to $\struct{\wt{X}}(1)$ and under $\P(\C_{Y/X}) \iso E \to \wt{X}$ this pulls back to $\struct{\P}(-1)$ and therefore $\struct{\wt{X}}(n)$ is nontrivial for all $n \neq 0$ so the map $\Z \to \Pic{\wt{X}}$ is injective. Therefore, we get a right split exact sequence,
\begin{center}
\begin{tikzcd}
0 \arrow[r] & \Z \arrow[r] & \Pic{\wt{X}} \arrow[r] & \Pic{X} \arrow[r] & 0
\end{tikzcd}
\end{center}
and thus $\Pic{\wt{X}} \cong \Pic{X} \oplus \Z$.

\item Since $\Pic{\wt{X}} = \Pic{X} \oplus \Z$ we know that $\omega_{\wt{X}} \cong \pi^* \M \ot \struct{\wt{X}}(q E)$ for some integer $q \in \Z$. Furthermore, $\pi : \wt{X} \setminus E \to X \setminus Y$ is an isomorphism so $\M|_{X\setminus Y} \cong \omega_X |_{X \setminus Y}$ and therefore $\M \cong \omega_X$ because $\codim{Y, X} \ge 2$. Let $\P := \P_Y(\C_{Y/X})$ then $\P \iso E$ and $E \subset \wt{X}$ is Cartier with $\struct{\wt{X}}(E) = \struct{\wt{X}}(1)$. From the adjunction formula,
\[ \omega_{\P} = (\omega_{\wt{X}} \ot \struct{\wt{X}}(E))|_E = (\pi^* \omega_X \ot \struct{\wt{X}}((q+1)E))|_E = \pi^* (\omega_X |_Y) \ot \struct{\P}(-(q+1)) \]
However, again from the adjunction formula,
\[ \omega_Y = \omega_X |_Y \ot \bigwedge^{r} \sN_{Y/X} \]
and from the formula for projective bundles (recall that $\rank(\sN_{Y/X}) = r$ so $\P$ has relative dimension $r-1$),
\[ \omega_{\P} = \pi^* \omega_Y \ot \pi^* \left( \bigwedge^{r} \C_{Y/X} \right) \ot \struct{\P}(-r) \]
[III, Ex. 8.4]. Therefore, putting these togerther we find,
\[ \omega_{\P} = \pi^* (\omega_X |_Y) \ot \pi^* \left( \wedge^r \C_{Y/X} \right)^\vee \otimes \pi^* \left( \wedge^r \C_{Y/X} \right) \ot \struct{\P}(-r) \cong \pi^* (\omega_X|_Y) \ot \struct{\P}(-r) \]
Therefore, $q + 1 = r$ meaning that $q = r - 1$ and therefore,
\[ \omega_{\wt{X}} \cong \pi^* \omega_X \ot \struct{\wt{X}}((r-1)E) \]


(WHAT IS THIS LAST PART HMMMM)
\item Consider $X = \Spec{k[x,y,z]/(z^2 - xy)}$ and $I = (x,y,z)$. Then consider the graded ring,
\[ R = \bigoplus_{i = 0}^\infty I^i \]
Then,
\[ R = A[t, u, v]/(x u - y t, y v - z u, x v - z t, v^2 - tu) \]
Now consider $\wt{X} = \Proj{R} \to X$. Consider the open patches,
\begin{align*}
D_+(t) & = \Spec{k[x, v]} 
\\
D_+(u) & = \Spec{k[y, v]} 
\\
D_+(v) & = \Spec{k[z, t, u]/(tu - 1)}
\end{align*}
Therefore $\wt{X}$ is smooth. Likewise, consider the close subschemes,
\begin{align*}
V(t) & = \Proj{k[x, y, z, u, v]/(xu, yv - zu, xv, v^2)} = \Proj{k[y, z, u, v]/(yv - zu, v^2)}
\\
D_+(u) & = \Spec{k[y, v]} 
\\
D_+(v) & = \Spec{k[z, t, u]/(tu - 1)}
\end{align*}
\end{enumerate}


\begin{lemma}
Let $X$ be a noetherian normal ($S_2$ is probably enough) scheme and $Z \subset X$ a closed subscheme with $\codim{Z,X} \ge 2$ and $U = X \setminus Z$. Let $\F, \G$ be locally free $\struct{X}$-modules. Then,
\begin{enumerate}
\item the restriction map $\res : \struct{X}(X) \to \struct{X}(U)$ is an isomorphism
\item the restriction map $\res : \F(X) \to \F(U)$ is an isomorphism
\item the restriction map $\Hom{\struct{X}}{\F}{\G} \iso \Hom{\struct{U}}{\F|_U}{\G|_U}$
is an isomorphism.
\item if $\F|_U \cong \G|_U$ then $\F \cong \G$. 
\end{enumerate}
\end{lemma}

\begin{proof}
I claim that (a) $\implies$ (b) $\implies$ (c) $\implies$ (d). First, let $U_i$ be an open cover of $X$ trivializing the locally free $\struct{X}$-module $\F$. Then we have a diagram,
\begin{center}
\begin{tikzcd}
9 \arrow[r] & \F(X) \arrow[r] \arrow[d] & \prod_i \F(U_i) \arrow[d, equals] \arrow[r] & \prod_{i,j} \F(U_i \cap U_j) \arrow[d, equals]
\\
0 \arrow[r] & \F(U) \arrow[r] & \prod_i \F(U_i \cap U) \arrow[r] & \prod_{i,j} \F(U_i \cap U_j \cap U)
\end{tikzcd}
\end{center}
because $\F|_{U_i}$ is free and $U_i$ is a normal scheme and $\codim{Z \cap U_i, U_i} \ge 2$ (the codimension can only increase which happens when some component does not lie in $U_i$) so by (a) the downward maps are isomorphisms and therefore $\res : \F(X) \to \F(U)$ is an isomorphism. 
\bigskip\\
Next, the sheaf,
\[ \shHom{\struct{X}}{\F}{\G} \]
is locally free and therefore (c) follows from (b) applied to this sheaf. Now suppose that $\varphi : \F|_U \to \G|_U$ and $\psi : \G|_U \to \F|_U$ are inverse. Then by (c) they extend to $\tilde{\varphi} : \F \to \G$ and $\tilde{\psi} : \G \to \F$ and $(\tilde{\psi} \circ \tilde{\varphi})|_U = \id_{\F|_U}$ and $(\tilde{\varphi} \circ \tilde{\psi})|_U = \id_{\G|_U}$ so by the uniqueness (injectivity) part of (c) we conclude that $\tilde{\varphi}$ and $\tilde{\psi}$ are inverse isomorphisms. 
\bigskip\\
Now we prove (a). A noetherian normal scheme is a disjoint union of finitely many normal integral schemes so we may assume that $X$ is integral. Then let $\xi \in X$ be the generic point. On an integral scheme the restriction maps $\res : \struct{X}(U) \to \struct{X}(V)$ are always injective (as are restrictions to stalks) so we may view these as subrings of the function field $K = \stalk{X}{\xi}$. We see that,
\[ \struct{X}(U) = \bigcap_{x \in U} \struct{X}{x} \]
In particular, for every affine open $V = \Spec{A}$ of $X$ we have,
\[ \struct{X}(V \cap U) = \bigcap_{x \in U \cap V} \struct{X}{x} = \bigcap_{\p \notin V(I)} A_\p \]
taken inside $K = \Frac{A}$ where $U \cap V = \Spec{A} \setminus V(I)$ because $U$ is open. However, 
\[ \codim{V(I), V} \ge \codim{Z, X} \ge 2 \]
Furthermore,
\[ \codim{ \overline{ \{ x \}} , X} = \dim{\stalk{X}{x}} \]
and therefore if $x \in V$ then
\[ \codim{ \overline{ \{ x \} }, X} = \codim{ \overline{\{ \p \}}, V} = \height{\p} \]
where $\p \in \Spec{A}$ is the corresponding prime. Therefore, if $\height{\p} = 1$ then we cannot have $\overline{ \{ \p \} } = V(\p) \subset V(I)$ else the codimension of $V(I)$ would be too large. Therefore $\p \notin V(I)$ meaning that,
\[ \struct{X}(U \cap V) = \bigcap_{\p \in V(I)^C} A_\p \subset \bigcap_{\height{\p} = 1} A_\p = A = \struct{X}(V) \]
inside $K$ since $A$ is a normal domain. Finally, choosing an affine open cover $\{ V_i \}$ we find,
\[ \struct{X}(U) = \bigcap_{i} \struct{X}(U \cap V_i) = \bigcap_i \struct{X}(V_i) = \struct{X}(X) \]
because the $V_i$ cover $X$ proving the claim. 
\end{proof}

\subsubsection{8.6 CHECK!!}

Let $k$ be an algebraically closed field (I think this works over any ring). Let $A$ be a finitely denerated $k$-algebra such that $\Spec{A}$ is a nonsingular variety (i.e. $A$ is regular). Consider an exact sequence,
\begin{center}
\begin{tikzcd}
0 \arrow[r] & I \arrow[r] & B' \arrow[r] & B \arrow[r] & 0
\end{tikzcd}
\end{center}
where $B'$ is a $k$-algebra with an ideal $I \subset B'$ such that $I^2 = 0$. Notice that $I = I/I^2$ is naturally a $B = B'/I$-module. Suppose $f : A \to B$ is a $k$-algebra homomorphism. We are looking for maps $g : A \to B'$ such that the diagram,
\begin{center}
\begin{tikzcd}
& 0 \arrow[d]
\\
& I \arrow[d]
\\
& B' \arrow[d]
\\
A \arrow[r, "f"] \arrow[ru, dashed, "g"] & B \arrow[d]
\\
& 0 
\end{tikzcd}
\end{center}

\begin{enumerate}
\item Let $g, g' : A \to B'$ be two homomorphism lifting $f$. Then clearly $\theta = g' - g$ lands in $I$ since $g, g'$ agree when projected to $B$. Now I claim that $\theta$ is a derivation. Indeed,
\[ \theta(ab) = g'(ab) - g(ab) = g'(a) g'(b) - g(a) g(b) \]
However, notice that for multiplication on $I$ we see that $g(a), g'(a), f(a)$ act the same way because they agree on $B'/I$ and $I^2 = 0$ giving a unique $A$-module structure on $I$. Thus,
\begin{align*}
\theta(a) b  + a \theta(b) & = (g'(a) - g(a)) f(b) + f(a)(g'(b) - g(b)) 
\\
& = g'(a) g'(b) - g(a) g'(b) + g(a) g'(b) - g(a) g(b) = g'(a) g'(b) - g(a) g(b) = \theta(ab)
\end{align*}
Furthermore, given any derivation $\theta : A \to I$ there is a ring map $g' = g + \theta : A \to B'$. This is clearly a map of $k$-modules and the projection of $g'(a)$ and $g(a)$ are equal in $B$ because $\theta(a) \in I$ so we just need to show that $g'(ab) = g'(a) g'(b)$ in $B'$. Indeed,
\[ g'(ab) = g(a) g(b) + \theta(a) f(b) + f(a) \theta(b) = (g(a) + \theta(a))(g(b) + \theta(b)) \]
because $\theta(a) \theta(b) = 0$ since $I^2 = 0$. Therefore, maps over $B$ correspond to,
\[ \Der{k}{A}{I} = \Hom{A}{\Omega_{A/k}}{I} \]

\item Let $P = k[x_1, \dots, x_n]$ be a polynomial ring such that $A$ is a quotient $P \onto A$. By sending each generator to a lift in $B'$ of their image in $B$ we get a map $h : P \to B'$ such that the diagram,
\begin{center}
\begin{tikzcd}
0 \arrow[d] & 0 \arrow[d]
\\
J \arrow[d] \arrow[r, "\bar{h}", dashed] & I \arrow[d]
\\
P \arrow[r, "h"] \arrow[d] & B' \arrow[d]
\\
A \arrow[d] \arrow[r, "f"]  \arrow[ru, dashed, "g"] & B \arrow[d]
\\
0 & 0 
\end{tikzcd}
\end{center}
commutes. If $j \in J$ then $h(j) \in B'$ maps to zero in $B'$ so $h(j) \in I$ giving the map $\bar{h} : J \to I$ of $P$-modules. Furthermore, since $h$ is an algebra map, $\bar{h}$ is multiplicative so $\bar{h}(j_1 j_2) = \bar{h}(j_1) \bar{h}(j_2) = 0$ because $I^2 = 0$ and thus $\bar{h}(J^2) = 0$ so we get a $P$-linear map $\bar{h} : J/J^2 \to I$. Furthermore, the $P$-action factors through $A = P/J$ giving an $A$-linear map where $A$ acts on $I$ through $f : A \to B$ and the $B$-action on $I$ from the fact that $I = I/I^2$.

\item Since $\Spec{A}$ is nonsingular we have an exact sequence,
\begin{center}
\begin{tikzcd}
0 \arrow[r] & J / J^2 \arrow[r] & \Omega_{P/k} \otimes_P A \arrow[r] & \Omega_{A/k} \arrow[r] & 0
\end{tikzcd}
\end{center}
where $\Omega_{A/k}$ is a finite locally free and thus projective $A$-module.
Then applying $\Hom{A}{-}{I}$ we get an exact sequence,
\begin{center}
\begin{tikzcd}
0 \arrow[r] & \Hom{A}{\Omega_{A/k}}{I} \arrow[r] & \Hom{A}{\Omega_{P/k} \otimes_P A}{I} \arrow[r] & \Hom{A}{J/J^2}{I} \arrow[r] & 0
\end{tikzcd}
\end{center}
where this is right-exact because $\Omega_{A/k}$ is projective so the first sequence is split exact and thus remains split exact after applying any additive functor (equivalently $\Ext{1}{A}{\Omega_{A/k}}{I} = 0$ showing exactness on the right). Furthermore, by hom-tensor adjunction,
\[ \Hom{A}{\Omega_{P/k} \otimes_P A}{I} = \Hom{P}{\Omega_{P/k}}{I} \]
so we have a split exact sequence,
\begin{center}
\begin{tikzcd}
0 \arrow[r] & \Hom{A}{\Omega_{A/k}}{I} \arrow[r] & \Hom{P}{\Omega_{P/k}}{I} \arrow[r] & \Hom{A}{J/J^2}{I} \arrow[r] & 0
\end{tikzcd}
\end{center}
Let $\wt{\theta} \in \Hom{P}{\Omega_{P/k}}{I}$ be an element whose image gives the map $\bar{h} \in \Hom{A}{J/J^2}{I}$. Then let $\theta \in \Der{k}{P}{I}$ be the corresponding derivation and let $h' = h - \theta : P \to B'$. Then $h'$ is a ring map because,
\[ h'(ab) = h(a)h(b) - \theta(a) b - a \theta(b) = (h'(a) - \theta(a))(h'(b) - \theta(b)) \]
because $\theta(a) \theta(b) = 0$ since $I^2 = 0$. Furthermore, under the map $J/J^2 \to \Omega_{P/k} \otimes_P A$ sending $j \mapsto \d{j} \otimes 1$ we know that $\tilde{\theta}$ restricts to $\bar{h}$ and thus $\tilde{\theta}(\d{j}) = \bar{h}(j) = h(j)$ in $I$. Furthermore, $\theta = \tilde{\theta} \circ \d$ and therefore,
\[ \theta(j) = \tilde{\theta}(\d{j}) = h(j) \]
so we see that $h'(j) = h(j) - \theta(j) = 0$ meaning that $h'(J) = 0$. Thus $h'$ descends to a map $g : A = P/J \to B'$ lifting $f : A \to B$ proving the claim.
\end{enumerate}

\subsubsection{8.7}

Let $X = \Spec{A}$ be an affine nonsingular scheme of finite type over $k$ so $A$ is a regular finite type $k$-algebra. Let $I$ be an $A$-module. We want to classify extensions,
\begin{center}
\begin{tikzcd}
0 \arrow[r] & I \arrow[r] & A' \arrow[r] & A \arrow[r] & 0
\end{tikzcd}
\end{center}
where $I \to A'$ identifies $I$ with an ideal of $A'$ such that $I^2 = 0$ in $A'$. Then consider,
\begin{center}
\begin{tikzcd}
& 0 \arrow[d]
\\
& I \arrow[d]
\\
& A' \arrow[d]
\\
A \arrow[r, equals] \arrow[ru, dashed, "s"] & A \arrow[d]
\\
& 0 
\end{tikzcd}
\end{center}
so since $A$ is regular, by the infinitessimal lifting property we get a section $s : A \to A'$ showing that the extension is split. Indeed, we get a map $(s, \iota) : A \oplus I \to A'$ making the diagram commute,
\begin{center}
\begin{tikzcd}
0 \arrow[d] & 0 \arrow[d]
\\
I \arrow[r, equals] \arrow[d] & I \arrow[d]
\\
A \oplus I \arrow[d] \arrow[r] & A' \arrow[d]
\\
A \arrow[d] \arrow[r, equals] \arrow[ru, dashed, "s"] & A \arrow[d]
\\
0 & 0 
\end{tikzcd}
\end{center}
Then by the five lemma the map $s : A \oplus I \to A'$ is an isomorphism of $k$-modules. We need to show it is also a ring map where we put the algebra structue $(a,i) \cdot (a', i') = (aa', i a' + a i')$ on $A \oplus I$ so that $I^2 = 0$ inside $A \oplus I$. Then,
\[ (s, \iota)(aa', i a' + a i') = s(a)s(a') + i a' + a i' = (s(a) + i)(s(a') + i') \]
because $i i' = 0$ since $I^2 = 0$ and $i s(a') = i a'$ and $s(a) i' = a i'$ because $I$ is an $A = A'/I$ module and $s(a)$ projects to $a$ in $A$. Therefore, we see that every extension is isomorphic to the split extension.

\subsubsection{8.8 DO!!}

(DONE IN MY NOTES)

\subsection{9}


\newcommand{\fX}{\mathfrak{X}}
\newcommand{\fU}{\mathfrak{U}}
\newcommand{\fF}{\mathfrak{F}}
\newcommand{\fJ}{\mathfrak{J}}
\newcommand{\fL}{\mathfrak{L}}

\subsubsection{9.1}

Let $X$ be a noetherian scheme, $Y \subset X$ a closed subscheme, and $\hat{X}$ the completion of $X$ along $Y$. We call $\Gamma(\hat{X}, \struct{\hat{X}})$ the ring of formal regular functions.
\bigskip\\
Let $Y$ be a connected, nonsingular, positive-dimensional subvariety of $X = \P^n_k$ and $k$ algebraically closed.

\begin{enumerate}
\item Let $\I$ be the sheaf of ideals of $Y$. since $Y$ is nonsingular there is an exact sequence,
\begin{center}
\begin{tikzcd}
0 \arrow[r] & \I/\I^2 \arrow[r] & \Omega_X |_Y \arrow[r] & \Omega_Y \arrow[r] & 0
\end{tikzcd}
\end{center}
Futhermore, there is an exact sequence,
\begin{center}
\begin{tikzcd}
0 \arrow[r] & \Omega_X \arrow[r] & \struct{X}(-1)^{\oplus n + 1} \arrow[r] & \struct{X} \arrow[r] & 0
\end{tikzcd}
\end{center}
Pulling this sequence back to $Y$, the sequence remains exact because $\struct{X}$ is locally free so we get an injection $\Omega_X |_Y \embed \struct{Y}(-1)^{\oplus n + 1}$. Putting these together gives an injection,
\[ \I/\I^2 \embed \struct{Y}(-1)^{\oplus n + 1} \]

\item Because $Y$ is nonsingular, $\I/\I^2$ is locally free by [II, Theorem 2.8.17] and therefore applying and globalizing [II, Theorem 2.8.21A] we have an isomorphism,
\[ \I^r / \I^{r+1} \cong  \noSym{r}{\struct{Y}}{\I/\I^2} \]
First, $\I / \I^2 \embed \struct{Y}(-1)^{\oplus n + 1}$ and therefore,
\[ \I^r / \I^{r+1} \cong \noSym{r}{\struct{Y}}{\I / \I^2} \embed \noSym{r}{\struct{Y}}{\struct{Y}(-1)^{\oplus n+1}} = \struct{Y}(-r)^{\oplus N_r} \]
where $N_r = { n + r - 1 \choose r }$. Therefore,
\[ \Gamma(Y, \I^r/\I^{r+1}) \embed \Gamma(Y, \struct{Y}(-r)^{\oplus N_r}) \]
Therefore we need to show that $\Gamma(Y, \struct{Y}(-r)) = 0$. This follows from the fact that $Y$ is positive dimensional. Because $\struct{Y}(r)$ is very ample it is globally generated and thus has a nonzero section $s \in \Gamma(Y, \struct{Y}(r))$. Therefore, if $\struct{Y}(-r)$ had a nonzero section $s' \in \Gamma(Y, \struct{Y}(-r))$ then we would have $\struct{Y}(r) \cong \struct{Y}$ because $s \ot s' \in \Gamma(Y, \struct{Y})$ is a global section not everywhere zero (because it is nonzero at the generic point) but $Y$ is integral (it is connected and regular) and $k$ is algebraically closed so $H^0(Y, \struct{Y}) = k$ and therefore $s \ot s'$ is nonvanishing so both $s$ and $s'$ are nonvanishing. However, because $\dim{Y} > 0$ we cannot have $\struct{Y}$ be ample since $Y$ is projective (since then any projective curve on $Y$ would have an ample divisor of degree zero or alternatively because then $Y$ would be quasi-affine and projective and therefore finite). Therefore $H^0(Y, \I^r / \I^{r+1}) = 0$ for all $r > 0$. 

\item Consider the exact sequence,
\begin{center}
\begin{tikzcd}
0 \arrow[r] & \I^r / \I^{r+1} \arrow[r] & \struct{X} / \I^{r+1} \arrow[r] & \struct{X} / \I^r \arrow[r] & 0
\end{tikzcd}
\end{center}
Taking global sections,
\begin{center}
\begin{tikzcd}
0 \arrow[r] & \Gamma(Y, \I^r / \I^{r+1}) \arrow[r] & \Gamma(Y, \struct{X}/\I^{r+1}) \arrow[r] & \Gamma(Y, \struct{X}/\I^r) 
\end{tikzcd}
\end{center}
but $\Gamma(Y, \I^r / \I^{r+1}) = 0$ and therefore $\Gamma(Y, \struct{X}/\I^{r+1}) \embed \Gamma(Y, \struct{X} / \I^{r})$. For induction we assume that $\Gamma(Y, \struct{X} / \I^{r}) = k$ and therefore the $k$-algebra $\Gamma(Y, \struct{X}/\I^{r+1})$ is either $0$ or $k$ but $1$ is a global section so we find $\Gamma(Y, \struct{X}/\I^{r+1}) = k$. Finally, the base case 
\[ \Gamma(Y, \struct{X}/\I) = \Gamma(Y, \struct{Y}) = k \]
follows because $Y$ is integral and proper over $k$ which is algebraically closed.

\item By [II, Prop. 2.9.2],
\[ \Gamma(\hat{X}, \struct{\hat{X}}) = \Gamma(\hat{X}, \varprojlim_n \struct{X} / \I^n) = \varprojlim_n \Gamma(Y, \struct{X} / \I^n) = k \]
\end{enumerate}

\subsubsection{9.2 DO!!}

Let $Y \subset X = \P^n_k$ be a connected nonsingular positive dimensional closed subvariety with $k$ algebraically closed. Suppose that $f : X \to Z$ is a map of $k$-varieties (where $k$ is algebraically closed) such that $f(Y) = P$ for a closed point $P \in Z$.
\bigskip\\
We may replace $Z$ by the scheme theoretic image of $f$ such that $\struct{Z} \to f_* \struct{X}$ is injective. Let $\m_P$ be the sheaf of ideals corresponding to $P \in Z$. Then $\m_P$ must map inside $f_* \I$ because $f(Y) = P$ and $Y$ is reduced. Therefore, we get a diagram,
\begin{center}
\begin{tikzcd}
\struct{Z} \arrow[d] \arrow[r] & f_* \struct{X} \arrow[d]
\\
\struct{Z}/ \m_p^n \arrow[r] & f_* (\struct{X} / \I^n)
\end{tikzcd}
\end{center} 
We get an injection (WHY??) $\struct{\hat{Z}} \embed f_* \struct{\hat{X}}$ and therefore $\Gamma(\hat{Z}, \struct{\hat{Z}}) \embed \Gamma(\hat{X}, \struct{\hat{X}})$. However, $\Gamma(\hat{Z}, \struct{\hat{Z}}) = \widehat{\stalk{Z}{z}}$ and $\Gamma(\hat{X}, \struct{\hat{X}}) = k$ by (Ex. 9.1). Therefore, we must have $\hat{\stalk{Z}{z}} = k$ meaning that $\dim{Z} = 0$ so $Z$ is a single point because $X$ is connected.
\bigskip\\
Here is an alternative proof. Consider the map of locally ringed spaces $\hat{X} \to X \to Z$. For any affine open $V \subset Z$ containing $P$ we know $Y \subset f^{-1}(Z)$ and therefore $\hat{X} \to Z$ factors through $\hat{X} \to f^{-1}(Z) \to V \subset Z$. Furthermore, since $V$ is affine, the map $\hat{X} \to V$ is determined by $\Gamma(V, \struct{V}) \to \Gamma(\hat{X}, \struct{\hat{X}})$ but $\Gamma(\hat{X}, \struct{\hat{X}}) = k$ so this map is constant. Now if $U$ is an affine open of $f^{-1}(V)$ intersecting $Y$ then $\hat{U} \to \hat{X} \to V \to Z$ is constant. Therefore, consider $\hat{U} \to V$ where $U = \Spec{B}$ and $V = \Spec{A}$. We know that $A \to B \to \hat{B}$ factors through $k$ but $B \to \hat{B}$ is injective because $B$ is a domain ($X$ is integral) so $A \to B$ factors through $k$ and thus $U \to V$ is constant. Since $U \subset X$ is dense we see that $f : X \to Z$ is constant.

\subsubsection{9.3 DO!!}

Let $\fX$ be an affine formal scheme and,
\begin{center}
\begin{tikzcd}
0 \arrow[r] & \fF' \arrow[r] & \fF \arrow[r] & \fF'' \arrow[r] & 0
\end{tikzcd}
\end{center}
is an exact sequence of $\struct{X}$-modules with $\fF'$ coherent. By assumption $\fX = \hat{X}$ where $X = \Spec{A}$ with $A = \Gamma(\fX, \struct{\fX})$ and the completion is taken along a closed subsecheme $Y = V(I)$ for some ideal $I \subset A$. Then $A$ is $I$-adically complete (CHECK).

\begin{enumerate}
\item First, let $\fX$ be an noetherian formal scheme and $\fU \subset \fX$ be an affine open. Let $\fJ \subset \struct{\fX}$ be an ideal of definition. Then consider the exact sequence,
\begin{center}
\begin{tikzcd}
0 \arrow[r] & \fF'/ \fJ^n \fF' \arrow[r] & \fF / \fJ^n \fF' \arrow[r] & \fF'' \arrow[r] & 0 
\end{tikzcd}
\end{center}
of sheaves which exists because $\fJ^n \fF'$ maps to zero under $\fF \to \fF''$. However, by [II, Prop. 9.6] we see that $\fF'/\fJ^n \fF'$ is a coherent $\struct{Y_n}$-module where $Y_n = (\fX, \struct{\fX} / \fJ^n)$ is a scheme. Then we apply a generalized version of [II, Prop. 5.6] (see the following lemma) to conclude that,
\begin{center}
\begin{tikzcd}
0 \arrow[r] & \Gamma(\fU, \fF'/\fJ^n \fF') \arrow[r] & \Gamma(\fU, \fF/\fJ^n \fF') \arrow[r] & \Gamma(\fU, \fF'') \arrow[r] & 0
\end{tikzcd}
\end{center}
is exact because $\fF'/\fJ^n \fF'$ is a coherent $\struct{Y_n}$-module and $\fU \cap Y_n$ is an affine subscheme of $Y_n$.

\item We have shown that for every formal affine open $\fU \subset \fX$ there is an exact sequence,
\begin{center}
\begin{tikzcd}
0 \arrow[r] & \Gamma(\fU, \fF'/\fJ^n \fF') \arrow[r] & \Gamma(\fU, \fF/\fJ^n \fF') \arrow[r] & \Gamma(\fU, \fF'') \arrow[r] & 0
\end{tikzcd}
\end{center}
Furthermore, because $\fF'$ is coherent we see that it corresponds to a finitely generated $\Gamma(\fU, \struct{\fU})$-module which is noetherian and therefore the inverse system $\Gamma(\fU, \fF'/\fJ^n \fF')$ satisfies the ML condition. Then by [II, Prop. 9.1] we see that the sequence,
\begin{center}
\begin{tikzcd}
0 \arrow[r] & \varprojlim \Gamma(\fU, \fF'/\fJ^n \fF') \arrow[r] & \varprojlim \Gamma(\fU, \fF/\fJ^n \fF') \arrow[r] & \Gamma(\fU, \fF'') \arrow[r] & 0 
\end{tikzcd}
\end{center}
is exact. Then by [II, Prop. 9.2] we see that, 
\[ \Gamma(\fU, \fF') = \varprojlim \Gamma(\fU, \fF'/\fJ^n \fF') \]
because by [II, Prop. 9.6] we know that,
\[ \fF = \varprojlim \fF / \fJ^n \fF \]
because $\fF$ is a coherent $\struct{\fU}$-module and $\fJ$ is an ideal of definition. Therefore consider the diagram,
\begin{center}
\begin{tikzcd}
0 \arrow[r] & \fF' \arrow[d, equals] \arrow[r] & \fF \arrow[r] \arrow[d] & \fF'' \arrow[r] \arrow[d, equals] & 0
\\
0 \arrow[r] & \varprojlim \fF'/ \fJ^n \fF' \arrow[r] & \varprojlim \fF / \fJ^n \fF' \arrow[r] & \fF'' \arrow[r] & 0 
\end{tikzcd}
\end{center}
where the map $\fF \to \varprojlim \fF /\fJ^n \fF'$ is induced by the quotient maps $\fF \onto \fF / \fJ^n \fF'$ which are compatible with the transition maps of the system $\fF/\fJ^n \fF'$ by definition. Furthermore, the previous exact sequence shows that the bottom row gives an exact sequence of sections over affines and therefore is an exact sequence (we apply . Then the five lemma shows that,
\[ \fF \to \varprojlim \fF /\fJ^n \fF' \]
is an isomorphism. Then taking sections over an affine open $\fU \subset \fX$ gives the diagram,
\begin{center}
\begin{tikzcd}
0 \arrow[r] & \Gamma(\fU, \fF') \arrow[d, equals] \arrow[r] & \Gamma(\fU, \fF) \arrow[d, equals] \arrow[r] & \Gamma(\fU, \fF'') \arrow[d, equals]
\\
0 \arrow[r] & \varprojlim \Gamma(\fU, \fF'/\fJ^n \fF') \arrow[r] & \varprojlim \Gamma(\fU, \fF/\fJ^n \fF') \arrow[r] & \Gamma(\fU, \fF'') \arrow[r] & 0 
\end{tikzcd}
\end{center}
for all affine opens $\fU \subset \fX$. This diagram is an isomorphism of sequences and therefore each of the sequences,
\begin{center}
\begin{tikzcd}
0 \arrow[r] & \Gamma(\fU, \fF') \arrow[r] & \Gamma(\fU, \fF) \arrow[r] & \Gamma(\fU, \fF'') \arrow[r] & 0 
\end{tikzcd}
\end{center}
are exact. Now we assume that $\fX$ is an affine noetherian formal scheme so taking $\fU = \fX$ we conclude that,
\begin{center}
\begin{tikzcd}
0 \arrow[r] & \Gamma(\fX, \fF') \arrow[r] & \Gamma(\fX, \fF) \arrow[r] & \Gamma(\fX, \fF'') \arrow[r] & 0 
\end{tikzcd}
\end{center}
is exact.
\end{enumerate}

\begin{lemma}
Let $X$ be an affine scheme. Then let,
\begin{center}
\begin{tikzcd}
0 \arrow[r] & \F_1 \arrow[r] & \F_2 \arrow[r] & \F_3 \arrow[r] & 0 
\end{tikzcd}
\end{center}
be an exact sequence of abelian sheaves. Suppose that $\F_1$ is a quasi-coherent $\struct{X}$-module then the sequence of global sections,
\begin{center}
\begin{tikzcd}
0 \arrow[r] & \Gamma(X, \F_1) \arrow[r] & \Gamma(X, \F_2) \arrow[r] & \Gamma(X, \F_3) \arrow[r] & 0
\end{tikzcd}
\end{center}
\end{lemma}

\begin{proof}
I am sure we can give a direct proof but this follows directly from the fact that $H^1(X, \F_1) = 0$ since $X$ is affine and $\F_1$ is quasi-coherent. 
\end{proof}

\subsubsection{9.4 (CHECK!!!)}

Let $\fX$ be a noetherian formal scheme and,
\begin{center}
\begin{tikzcd}
0 \arrow[r] & \fF' \arrow[r] & \fF \arrow[r] & \fF'' \arrow[r] & 0
\end{tikzcd}
\end{center}
be an exact sequence of $\struct{\fX}$-modules. We want to show that if $\fF'$ and $\fF''$ are coherent then $\fF$ is also coherent. Since this question is local, we may assume that $\fX$ is an affine noetherian formal scheme and thus by [II, Prop. 9.4] the canonical adjunction maps $\Gamma(\fX, \fF')^\Delta \to \fF'$ and $\Gamma(\fX, \fF'')^\Delta \to \fF''$ are isomorphisms. By [II, Ex. 9.3] we know that,
\begin{center}
\begin{tikzcd}
0 \arrow[r] & \Gamma(\fX, \fF') \arrow[r] & \Gamma(\fX, \fF) \arrow[r] & \Gamma(\fX, \fF'') \arrow[r] & 0
\end{tikzcd}
\end{center}
is exact because $\fX$ is affine and $\fF'$ is coherent. Furthermore, because $\fF'$ and $\fF''$ are coherent we see that $\Gamma(\fX, \fF')$ and $\Gamma(\fX, \fF'')$ are finitely generated $\Gamma(\fX, \struct{\fX})$-modules thus from the exact sequence $\Gamma(\fX, \fF)$ is a finitely generated $\Gamma(\fX, \struct{\fX})$-module. Therefore, we get a morphism of exact sequences,
\begin{center}
\begin{tikzcd}
0 \arrow[r] & \Gamma(\fX, \fF')^\Delta \arrow[d, "\sim"] \arrow[r] & \Gamma(\fX, \fF)^\Delta \arrow[d] \arrow[r] & \Gamma(\fX, \fF'')^\Delta \arrow[d, "\sim"] \arrow[r] & 0
\\
0 \arrow[r] & \fF' \arrow[r] & \fF \arrow[r] & \fF'' \arrow[r] & 0
\end{tikzcd}
\end{center}
so the adjunction map $\Gamma(\fX, \fF)^\Delta \to \fF$ is an isomorphism by the five lemma proving that $\fF$ is coherent.

\subsubsection{9.5} 

Let $\fX$ be a noetherian formal scheme and $\fF$ be a coherent $\struct{\fX}$-module. Suppose that $\fF$ is generated by global sections meaning there is a surjection,
\[ \bigoplus_{s \in S} \struct{\fX} \onto \fF \]
where $S \subset \Gamma(\fX, \fF)$ is a subset of the global sections. Choose a finite affine cover $\{ \fU_i \}$ of $\fX$. Then by [II, Thm 9.7] we see that $\fF|_{\fU_i} \cong M_i^\Delta$ for some finite $A_i$-module $M_i$ where $A_i = \Gamma(\fU_i, \struct{\fU_i})$. Since when restricting to $\fU_i$ the above map is still surjective and by the equivalence of categories we get a surjection,
\[ \bigoplus_{s \in S} A_i \onto M_i \]
but $M_i$ is finitely generated since $\fF$ is coherent and therefore there is a finite subset $S_i$ such that,
\[ \bigoplus_{s \in S_i} \embed \bigoplus_{s \in S} A_i \onto M_i \]
is surjective. Therefore let $S' = S_1 \cup \dots \cup S_n$ which is finite. Then I claim that,
\[ \bigoplus_{s \in S'} \struct{\fX} \onto \fF \]
is surjective. This can be checked locally on the open affine cover $\fU_i$ on which we have seen that,
\[ \bigoplus_{s \in S_i} \embed \bigoplus_{s \in S'} \to M_i \]
is surjective and hence the second map is surjective proving the claim. 

\subsubsection{9.6 DO!!}

Let $\fX$ be a noetherian formal scheme, let $\fJ$ be a an ideal of definition. For each $n$ let $Y_n$ be the scheme $Y_n = (\fX, \struct{\fX} / \fJ^n)$ and we assume that the inverse system $( \Gamma(Y_n, \struct{Y_n}))$ satisfies the ML condition. We are going to prove as follows that,
\[ \Pic{\fX} \iso \varprojlim_n \Pic{Y_n} \]
is an isomorphism.

\begin{enumerate}
\item For $n' \ge n$ consider the exact sequence of sheaves,
\begin{center}
\begin{tikzcd}
0 \arrow[r] & \fJ^{n} / \fJ^{n'} \arrow[r] & \struct{\fX} / \fJ^{n'} \arrow[r] & \struct{\fX} / \fJ^{n} \arrow[r] & 0 
\end{tikzcd}
\end{center}
giving the sequence of groups,
\begin{center}
\begin{tikzcd}
0 \arrow[r] & \Gamma(\fX, \fJ^n / \fJ^{n'}) \arrow[r] & \Gamma(Y_{n'}, \struct{Y_{n'}}) \arrow[r] & \Gamma(Y_n, \struct{Y_n}) 
\end{tikzcd}
\end{center}
I claim that the kernel $\Gamma(\fX, \fJ^n / \fJ^{n'})$ is nilpotent of order $k = \lceil \frac{n'}{n} \rceil$. Because this is a sequence of coherent sheaves on $\fX$, by the previous exercises, for any affine $\fU \subset \fX$ we have an exact sequence,
\begin{center}
\begin{tikzcd}
0 \arrow[r] & \Gamma(\fU, \fJ^n) \arrow[r] & \Gamma(\fU, \fJ^{n'}) \arrow[r] & \Gamma(\fU, \fJ^{n}/\fJ^{n'}) \arrow[r] & 0
\end{tikzcd}
\end{center}
Therefore, for any $f \in \Gamma(\fX, \fJ^n / \fJ^{n'})$ consider $f^k$ then $f^k |_{\fU} = (f|_{\fU})^k = 0$ from the exact sequence. Therefore $f^k = 0$ because $\fJ^{n} / \fJ^{n'}$ is a sheaf. 
\bigskip\\
Now I claim that, if $\varphi : A \to B$ is a surjective map of rings with $I = \ker{(A \to B)}$ nilpotent then $\varphi : A^\times \to B^\times$ is surjective. Suppose that $y = \varphi(x)$ is a unit. Then there is $y^{-1} = \varphi(x')$ so that $\varphi(x x') = 1$ and therefore $1 - x x' \in I$. Because $I$ is nilpotent, this implies that $xx' = 1 + i$ for $i \in I$ is invertible and hence $x$ is invertble proving the claim. Now we apply this to show that if $\{ A_n \}$ is an inverse system of rings satisfying the ML condition such that $I_{n',n} = \ker{(A_{n'} \to A_n)}$ is nilpotent then $\{ A_n^\times \}$ satisfies the ML condition. Indeed, I claim that for sufficiently large $n'$,
\[ \im{(A^\times_{n'} \to A_n^\times)} = (A'_{n})^\times \]
where $A'_n$ is the stable image. This is because we know that for sufficiently large $n'$,
\[ \im{(A_{n'} \to A_n)} = A'_n \]
and therefore we apply the previous lemma to the surjective ring map $A_{n'} \to A_n'$ whose kernel $I_{n',n}$ is nilpotent proving the claim. 


\item Let $\fF$ be a coherent sheaf of $\struct{\fX}$-modules such that for each $n$, there is some isomorphism $\varphi_n : \fF / \fJ^n \fF \iso \struct{Y_n}$. To produce an isomorphism $\wt{\varphi} : \fF \iso \struct{\fX}$ we need to modify the isomorphisms $\varphi_n$ to produce isomorphisms that fit into the diagram,
\begin{center}
\begin{tikzcd}
\fF / \fJ^n \fF \arrow[d, "\wt{\varphi}_{n+1}"'] \arrow[r] & \fF / \fJ^n \fF
\arrow[d, "\wt{\varphi}_n"] 
\\
\struct{Y_{n+1}} \arrow[r] & \struct{Y_n}
\end{tikzcd}
\end{center}
To do this, we need the following result on inverse limits of abelian groups.

\begin{lemma}
Let $(A_n)$ be an inverse system of abelian groups satisfiying the ML condition. Then there is an exact sequence,
\begin{center}
\begin{tikzcd}
0 \arrow[r] & \varprojlim_n A_n \arrow[r] & \prod\limits_{n \ge 0} A_n \arrow[r] & \prod\limits_{n \ge 0} A_n \arrow[r] & 0
\end{tikzcd}
\end{center}
where the map on products sends $(s_n) \mapsto (s_n - r_{n+1}(s_{n+1}))$ where $r_{n+1} : A_{n+1} \to A_n$ is the restriction map. 
\end{lemma}

\begin{proof}
Define the invere system,
\[ B_n = \prod_{i = 0}^n A_i \]
where the maps are the natural projections. Then clearly,
\[ \varprojlim_n B_n = \prod_{n \ge 0} A_n \]
Now consider the exact sequence of inverse systems,
\begin{center}
\begin{tikzcd}
0 \arrow[r] & (A_n) \arrow[r] & (B_n) \arrow[r] & (B_n)[-1] \arrow[r] & 0
\end{tikzcd}
\end{center}
defined on each level by,
\begin{center}
\begin{tikzcd}
0 \arrow[r] & A_n \arrow[r] & B_n \arrow[r] & B_{n-1} \arrow[r] & 0
\end{tikzcd}
\end{center}
where the first map sends $s \mapsto (r_{n,k}(s))_k$ and the second sends $(s_k)_k \mapsto (s_k - r_{k+1}(s_{k+1}))_k$. It is clear that $A_n$ is the kernel of the second map. To see that this map is surjective we use descending induction on $k$. For any $(t_k)_{k}$ for $k \in \{1, \dots, n-1 \}$ we set $s_{n+1} = 0$ and then inductively let $s_k = r_{k+1}(s_{k+1}) + t_k$ which satisfies $(s_k)_k \mapsto (t_k)_k$. Then there is an exact sequence,
\begin{center}
\begin{tikzcd}
0 \arrow[r] & \varprojlim_n A_n \arrow[r] & \prod\limits_{n \ge 0} A_n \arrow[r] & \prod\limits_{n \ge 0} A_n \arrow[r] & \varprojlim_n^{1} A_n
\end{tikzcd}
\end{center}
but $(A_n)$ satisfies the ML condition so by [II, Prop. 9.1(b)] $\varprojlim_n^{1} A_n = 0$ and therefore the required sequence,
\begin{center}
\begin{tikzcd}
0 \arrow[r] & \varprojlim_n A_n \arrow[r] & \prod\limits_{n \ge 0} A_n \arrow[r] & \prod\limits_{n \ge 0} A_n \arrow[r] & 0
\end{tikzcd}
\end{center}
is exact on the right (left exactness does not require ML and is automatic from the construction of $\varprojlim$ or we can just say this is a left exact functor).
\end{proof}

Now we apply this as follows. We write $\wt{\varphi}_n = s_n \cdot \varphi_n$ for $s_n \in \Gamma(Y_n, \struct{Y_n}^\times)$ and require that,
\[ r_{n+1}(\wt{\varphi}_{n+1}) = \wt{\varphi}_n \]
where,
\[ r_{n+1} : \Hom{\struct{Y_{n+1}}}{\fF / \fJ^{n+1} \fF}{\struct{Y_{n+1}}} \to \Hom{\struct{Y_{n}}}{\fF / \fJ^{n} \fF}{\struct{Y_{n}}} \]
are the maps in the inverse system induced by the functor $- \ot \struct{Y_n}$. Therefore, 
\[ r_{n+1}(s_{n+1}) \cdot r_{n+1}(\varphi_{n+1}) = s_n \cdot \varphi_n \]
which means that we require,
\[ s_n \cdot r_{n+1}(s_{n+1})^{-1} = t_n \quad \text{ where } \quad t_n = r_{n+1}(\varphi_{n+1}) \circ \varphi_n^{-1} \in \mathrm{Iso}_{\struct{Y_n}} \left( \struct{Y_n}, \struct{Y_n} \right) = \Gamma(Y_n, \struct{Y_n}^\times) \]
Because the inverse system $(\Gamma(Y_n, \struct{Y_n}))$ satisfies ML by the lemma there exists an element $(s_n) \mapsto (t_n)$ so the required sequence of maps $\wt{\varphi}_n$ exists. Thus we get an isomorphism of inverse systems $\wt{\varphi} : (\fF / \fJ^n \fF) \to (\struct{Y_n})$ and therefore applying [II, Prop. 9.6] we get an isomorphism,
\[ \fF = \varprojlim_n \fF / \fJ^n \fF \xrightarrow{\wt{\varphi}} \varprojlim_n \struct{Y_n} = \struct{\fX} \] 
Therefore, the map,
\[ \Pic{\fX} \to \varprojlim_n \Pic{Y_n} \]
is injective because if $\fF / \fJ^n \fF$ is a trivial line bundle for each $n$ then $\fF$ is a trivial line bundle.

\item Now given a sequence of line bundles $\L_n \in \Pic{Y_n}$ and isomorphisms,
\[ \psi_{n+1} : \L_{n+1} \ot_{\struct{Y_{n+1}}} \struct{Y_{n}} \iso \L_n \]
By tensor-hom adjunction this gives maps $\psi_{n+1} : \L_{n+1} \to \L_n$ of $\struct{Y_{n+1}}$-modules. Therefore, by [II, Prop. 9.6] the $\struct{\fX}$-module,
\[ \fL = \varprojlim_n \L_n \]
is coherent and satisfies,
\[ \fL / \fJ^n \fL \cong \L_n \]
Therefore, it suffices to show that $\fL$ is locally-free of rank $1$ since then $\fL \in \Pic{\fX}$ and $\fL \mapsto (\L_n)$ because $\L_n \cong \fL / \fJ^n \fL$ thus proving that,
\[ \Pic{\fX} \to \varprojlim_n \Pic{Y_n} \]
is surjective.
\bigskip\\ 
This question is local on $\fX$ so we can take an affine open $\fU \subset \fX$ and thus $\fL|_{\fU} = M^\Delta$ for the finitely-genrated $A = \Gamma(\fU, \struct{\fU})$-module $M = \Gamma(\fU, \fL)$. Then for $I = \Gamma(\fU, \fJ)$ the ring $A$ is $I$-adically complete and,
\[ \L_n |_{Y_n \cap \fU} = \fF / \fJ^n \fF = \wt{M / I^n M} \]
Since $\L_1$ is locally-free of rank $1$ by shrinking $\fU$ we may assume that $M / I M$ is a locally-free $A / I$-module. Choose a lift $s \in M$ of a section $\bar{s} : A/I \iso M/IM$ giving an isomorphism. Then I claim that the map $\bar{s} : A/I^n \to M / I^n M$ is an isomorphism for each $n \ge 1$. Since $M / I^n M$ is a locally-free $A/I^n$-module of rank $1$ it suffices to show that the map is surjective. Let $m_1, \dots, m_r \in M$ be a generating set. We proceed by induction on $n$. Suppose that $\bar{s} : A / I^n \to M / I^n M$ is surjective. Then we can write $m_i - t_i s = q_i \in I^{n} M$. However, because $\bar{s} : A/I \onto M/IM$ we can write,
\[ q_i = \sum a_{ij} m_j = \sum a_{ij} (s t_j + q_j) = \sum_j s a_{ij} t_j + \sum_j a_{ij} q_j \]
for $a_{ij} \in I^n$ and thus,
\[ m_i = t_i s + s \left( \sum_j a_{ij} t_j \right) + \sum_j a_{ij} q_j \]
and the last term is in $I^{n+1} M$ so we have proven that $\bar{s} : A/I^{n+1} \to M / I^{n+1} M$ is surjective thus proving the claim by induction.
\bigskip\\
Therefore, $s : A \to M$ is an isomorphism because these are $I$-adically complete. Thus $s : \struct{\fU} \iso \fL|_{\fU}$ proving that $\fL$ is locally free of rank $1$. 

\item Suppose that $\fX$ is an affine noetherian formal schemee. Then $\fX = \hat{X}$ for $X = \Spec{A}$ completed along some ideal $I$. Then we have $\Gamma(Y_n, \struct{Y_n}) = A / I^n$ and the maps $A/I^{n+1} \onto A/I^n$ are all surjective so the inverse system $(\Gamma(Y_n, \struct{Y_n}))$ satisfies the ML condition. On the other hand, if each $Y_n$ is projective over a field $k$ then each $\Gamma(Y_n, \struct{Y_n})$ is a finite $k$-module and therefore a noetherian module so the inverse system $(\Gamma(Y_n, \struct{Y_n})$ automatically satisfies the ML condition. 
\end{enumerate}

\section{Chapter III Cohomology}

\subsection{1}


\subsection{2}

\subsubsection{2.3}

Let $X$ be a topological space and $Y \subset X$ a closed subset. Let $\F$ be an abelian sheaf on $X$. Let $U = X \setminus Y$ and $j : U \to X$ be the inclusion.

\begin{enumerate}
\item Consider an exact sequence of sheaves on $X$,
\begin{center}
\begin{tikzcd}
0 \arrow[r] & \F \arrow[r] & \G \arrow[r] & \H \arrow[r] & 0
\end{tikzcd}
\end{center}
Then since the functor $(-)|_U$ is exact and $j_*$ is right-exact then we get a commutative diagram with exact rows,
\begin{center}
\begin{tikzcd}
0 \arrow[r] & \F \arrow[d] \arrow[r] & \G \arrow[r] \arrow[d] & \K \arrow[r] \arrow[d] & 0
\\
0 \arrow[r] & j_*(\F |_U) \arrow[r] & j_*(\G |_U) \arrow[r] & j_*(\K |_U)
\end{tikzcd}
\end{center}
Since taking kernels is left-exact (limits are right adjoints) we get an exact sequence,
\begin{center}
\begin{tikzcd}
0 \arrow[r] & \H^0_Y(\F) \arrow[r] & \H^0_Y(\G) \arrow[r] & \H^0_Y(\K)
\end{tikzcd}
\end{center}
Applying the left-exact functor $\Gamma(X, -)$ gives an exact sequence,
\begin{center}
\begin{tikzcd}
0 \arrow[r] & \Gamma_Y(X, \F) \arrow[r] & \Gamma_Y(X, \G) \arrow[r] & \Gamma_Y(X, \K)
\end{tikzcd}
\end{center}
Since $\Gamma(X, \H^0_Y(\F)) = \Gamma_Y(X, \F)$ by definition. 
\bigskip\\
We define the sheaf cohomology with supports in $Y$ to be the right-derived functors $H^n_Y(X, -) = R^n \Gamma_Y(X, -)$ of the left-exact functor $\Gamma_Y(X, -)$. 

\item Consider an exact sequence of sheaves,
\begin{center}
\begin{tikzcd}
0 \arrow[r] & \F \arrow[r] & \G \arrow[r] & \K \arrow[r] & 0
\end{tikzcd}
\end{center}
where the sheaf $\F$ is flasque. Then consider the diagram, with exact rows and columns,
\begin{center}
\begin{tikzcd}
& 0 \arrow[d] & 0 \arrow[d] & 0 \arrow[d]
\\
0 \arrow[r] & \H^0_Y(\F) \arrow[d] \arrow[r] & \H^0_Y(\G) \arrow[d] \arrow[r] & \H^0_Y(\K) \arrow[d]
\\
0 \arrow[r] & \F \arrow[d] \arrow[r] & \G \arrow[d] \arrow[r] & \K \arrow[d] \arrow[r] & 0
\\
0 \arrow[r] & j_*(\F |_U) \arrow[d] \arrow[r] & j_* (\G |_U) \arrow[r] & j_*(\K |_U) 
\\
& 0
\end{tikzcd}
\end{center}
where $\H^0_Z(\F)$ is the kernel of $\F \to j_*(\F |_U)$ and when $\F$ is flasque then we have the exact sequence,
\begin{center}
\begin{tikzcd}
0 \arrow[r] & \H^0_Z(\F) \arrow[r] & \F \arrow[r] & j_*(\F |_U) \arrow[r] & 0
\end{tikzcd}
\end{center}
Furthermore, the maps $\H_Y^0(\F) \to \H^0_Y(\G)$ and $j_*(\F |_U) \to j_*(\F |_U)$ are injective becase these functors are left-exact (taking kernels is left-exact and $j_*$ is left-exact, recall that $(-)|_U = j^{-1}$ is exact). Now apply the left-exact functor $\Gamma(X, -)$ to find a diagram with exact rows and columns,
\begin{center}
\begin{tikzcd}
& 0 \arrow[d] & 0 \arrow[d] & 0 \arrow[d]
\\
0 \arrow[r] & \Gamma_Y(X, \F) \arrow[d] \arrow[r] & \Gamma_Y(X, \G) \arrow[d] \arrow[r] & \Gamma_Y(X, \K) \arrow[d]
\\
0 \arrow[r] & \Gamma(X, \F) \arrow[d] \arrow[r] & \Gamma(X, \G) \arrow[d] \arrow[r] & \Gamma(X, \K) \arrow[d] \arrow[r] & 0
\\
0 \arrow[r] & \Gamma(U, \F |_U) \arrow[d] \arrow[r] & \Gamma(U, \G |_U) \arrow[r] & \Gamma(U, \K |_U) 
\\
& 0
\end{tikzcd}
\end{center}
Where $\Gamma(X, \F) \to \Gamma(U, \F |_U)$ remains surjective because $\F$ is a flasque sheaf so restriction is surjective. Furthermore, the sequence,
\begin{center}
\begin{tikzcd}
0 \arrow[r] & \Gamma(X, \F) \arrow[r] & \Gamma(X, \G) \arrow[r] & \Gamma(X, \K) \arrow[r] & 0
\end{tikzcd}
\end{center} 
remains exact because $\F$ is a flasque sheaf so $\Gamma(X, -)$ preserves the exact sequence since $H^1(X, \F) = 0$. Now, applying the snake lemma gives an exact sequence of the kernels to cokernels,
\begin{center}
\begin{tikzcd}
0 \arrow[r] & \Gamma_Y(X, \F)  \arrow[r] & \Gamma_Y(X, \G)  \arrow[r] & \Gamma_Y(X, \K) \arrow[r] & 0
\end{tikzcd}
\end{center}


\item Let $\F$ be a flasque sheaf. Now embedd $\F$ into an injective sheaf $\I$ to form an exact sequence,
\begin{center}
\begin{tikzcd}
0 \arrow[r] & \F \arrow[r] & \I \arrow[r] & \K \arrow[r] & 0 
\end{tikzcd}
\end{center}
Since $\F$ and $\I$ are flasque then $\K$ is also flasque.
Taking the long exact sequence of derived functors we find,
\begin{center}
\begin{tikzcd}[column sep = small]
0 \arrow[r] & H^0_Y(X, \F) \arrow[draw=none]{d}[name=Z, shape=coordinate]{} \arrow[r] & H^0_Y(X, \I) \arrow[r] & H^0_Y(X, \K) \arrow[r] & H^1_Y(X, \F) 
\arrow[dlll,
rounded corners, crossing over,
to path={ -- ([xshift=2ex]\tikztostart.east)
|- (Z) [near end]\tikztonodes
-| ([xshift=-2ex]\tikztotarget.west)
-- (\tikztotarget)}]
\\ 
& H^1_Y(X, \I) \arrow[r] & H^1_Y(X, \K) \arrow[r] & H^2_Y(X, \F) \arrow[r] & H^2_Y(X, \I) \arrow[r] & \cdots
\end{tikzcd}
\end{center}
Since $\I$ is an injective sheaf $H^n_Y(X, \I) = 0$ for $n > 0$ so this long exact sequence gives exact sequences,
\begin{center}
\begin{tikzcd}
0 \arrow[r] & H^0_Y(X, \F) \arrow[r] & H^0_Y(X, \I) \arrow[r] & H^0_Y(X, \K) \arrow[r] & H^1_Y(X, \F) \arrow[r] & 0
\end{tikzcd}
\end{center}
and,
\begin{center}
\begin{tikzcd}
0 \arrow[r] & H^n_Y(X, \K) \arrow[r] & H^{n+1}_Y(X, \F) \arrow[r] & 0
\end{tikzcd}
\end{center}
for $n \ge 1$. However, we have shown that the sequence,
\begin{center}
\begin{tikzcd}
0 \arrow[r] & \Gamma_Y(X, \F) \arrow[r] & \Gamma_Y(X, \I) \arrow[r] & \Gamma_Y(X, \K) \arrow[r] & 0
\end{tikzcd}
\end{center}
is exact when $\F$ is flasque and since $H_Y^1(X, \F)$ is the cokernel of this last map we have $H^1_Y(X, \F) = 0$ for any flasque sheaf. Now performing induction we find that $H^n_Y(X, \K) = 0$ for any flasque sheaf $\K$ implies $H^{n+1}_Y(X, \F) = 0$ via the second exact sequence. This proves that $H^n_Y(X, \F) = 0$ for all $n > 0$. 

\item If $\F$ is flasque then we have shown that the sequence,
\begin{center}
\begin{tikzcd}
0 \arrow[r] & \H^0_Y(\F) \arrow[r] & \F \arrow[r] & j_* (\F |_U) \arrow[r] & 0
\end{tikzcd}
\end{center}
is exact. Applying the functor $\Gamma(X, -)$ we get an exact sequence,
\begin{center}
\begin{tikzcd}
0 \arrow[r] & \Gamma_Y(X, \F) \arrow[r] & \Gamma(X, \F) \arrow[r] & \Gamma(U, \F) \arrow[r] & 0 
\end{tikzcd}
\end{center}
where the last map is restruction $\res_{U,X} : \F(X) \to \F(U)$ which is surjective by flasqueness. We have used $\Gamma(X, j_* (\F |_U)) = \Gamma(U, \F|_U) = \Gamma(U, \F)$. 


\item The above exact sequence for flasque sheaves implies that the left-exact functors $\Gamma_Y(X, -)$ and $\Gamma(X, -)$ and $\Gamma(U, (-)|_U)$ satisfy the hypothesis of Lemma \ref{lem:exact_seq_of_functors} giving the required long-exact sequence of cohomology,
\begin{center}
\begin{tikzcd}
0 \arrow[r] & H^0_{Y}(X, \F) \arrow[r] & H^0(X, \F)  \arrow[draw=none]{d}[name=Z, shape=coordinate]{} \arrow[r] & H^0(U, \F |_U)
\arrow[dll,
rounded corners, crossing over,
to path={ -- ([xshift=2ex]\tikztostart.east)
|- (Z) [near end]\tikztonodes
-| ([xshift=-2ex]\tikztotarget.west)
-- (\tikztotarget)}]
\\ 
& H^1_{Y}(X, \F) \arrow[r] & H^1(X, \F)  \arrow[draw=none]{d}[name=Z', shape=coordinate]{} \arrow[r] & H^1(U, \F |_U) \arrow[dll,
rounded corners, crossing over,
to path={ -- ([xshift=2ex]\tikztostart.east)
|- (Z') [near end]\tikztonodes
-| ([xshift=-2ex]\tikztotarget.west)
-- (\tikztotarget)}]
\\
& H^2_Y(X, \F) \arrow[r] & H^2(X, \F) \arrow[r] & H^2(U, \F |_U) \arrow[r] & \cdots
\end{tikzcd}
\end{center}

\item Let $Z \subset X$ be closed and $V \subset X$ be an open set such that $Z \subset V$ and let $\F$ be a sheaf on $X$. Then consider the restriction map $\res_{V, X} : \Gamma_Z(X, \F) \to \Gamma_Z(V, \F)$. Note that $V$ and $U = X \setminus Z$ form an open cover of $X$. If $s \mapsto 0$ then $s|_V = 0$ but also $s|_{X \setminus Z} = 0$ since $\Supp{\F}{s} \subset Z$ and thus $s_x = 0$ for each $x \in X \setminus Z$. Therefore, by the sheaf property of $\F$ we have $s = 0$ so $\res_{U,X}$ is injective. Furthermore, consider a section $s \in \Gamma_Z(V, \F)$. Since $\Supp{\F}{s} \subset Z$ we know that $s |_{V \cap U} = 0$ because $(V \cap U) \cap Z = \varnothing$. Therefore, $s$ and $0 \in \Gamma(U, \F)$ agree on the overlap and thus glue to a global section $s' \in \Gamma(X, \F)$ such that $s'|_V = s$. Furthermore, $s' |_U = 0$ and thus $\Supp{\F}{s'} \subset Z$ so the map $\res_{V, X} : \Gamma_Z(X, \F) \to \Gamma_Z(V, \F)$ is surjective and thus an isomorphism. Therefore, there is a natural isomorphism $\Gamma_Z(X, -) \cong \Gamma_Z(V, (-)|_V)$. Therefore, these functors give rise to the same derived functors so,
\[ H^p_Z(X, \F) \cong H^p_Z(V, \F|_V) \] 
where I have used that $(-)|_V$ preserves injectives and is exact so it commutes with taking derived functors.

\end{enumerate}

\subsubsection{2.4}

Let $Z_1, Z_2 \subset X$ be closed subsets. Let $\F$ be a flasque sheaf and consider the diagram,
\begin{center}
\begin{tikzcd}[column sep = small]
& 0 \arrow[d] & 0 \arrow[d] & 0 \arrow[d] &
\\
0 \arrow[r] & \Gamma_{Z_1 \cap Z_2}(X, \F) \arrow[r] \arrow[d] & \Gamma_{Z_1}(X, \F) \oplus \Gamma_{Z_2}(X, \F) \arrow[r] \arrow[d] & \Gamma_{Z_1 \cup Z_2}(X, \F) \arrow[r] \arrow[d] & 0
\\
0 \arrow[r] & \Gamma(X, \F) \arrow[r] \arrow[d] & \Gamma(X, \F) \oplus \Gamma(X, \F) \arrow[r] \arrow[d] & \Gamma(X, \F) \arrow[r] \arrow[d] & 0
\\
0 \arrow[r] & \Gamma(X \setminus (Z_1 \cap Z_2), \F) \arrow[r] \arrow[d] & \Gamma(X \setminus  Z_1, \F) \oplus \Gamma(X \setminus Z_2, \F) \arrow[r] \arrow[d] & \Gamma(X \setminus (Z_1 \cup Z_2), \F) \arrow[r] \arrow[d] & 0
\\
& 0 & 0 & 0 &
\end{tikzcd}
\end{center}
where the columns are the exact sequences of 2.3 (d), the last row is the exact sequence of Lemma \ref{lem:flasque_mayer_vietoris}, the middle row is the diagonal exact sequence associated to the direct sum ($s \mapsto (s, s)$ then $(s, t) \mapsto s - t$), and the top rwo is given first inclusion maps and second by the difference of the inclusion maps (including the group of sections with support in a smaller set into the group of section with support in a larger set). Since this diagram commutes, has exact columns, and the last two rows are exact, by the nine-lemma, the top row is exact as well. Therefore for any flasque sheaf, and in particular any injective sheaf, there is an exact sequence,
\begin{center}
\begin{tikzcd}
0 \arrow[r] & \Gamma_{Z_1 \cap Z_2}(X, \F) \arrow[r] & \Gamma_{Z_1}(X, \F) \oplus \Gamma_{Z_2}(X, \F) \arrow[r] & \Gamma_{Z_1 \cup Z_2}(X, \F) \arrow[r]  & 0
\end{tikzcd}
\end{center}
Therefore, the left-exact functors $\Gamma_{Z_1 \cap Z_2}(X, -)$ and $\Gamma_{Z_1}(X, -) \oplus \Gamma_{Z_2}(X, -)$ and $\Gamma(Z_1 \cup Z_2, -)$ satisfy the conditions of Lemma \ref{lem:exact_seq_of_functors} giving an exact sequence of their derived functors. Furthermore, because direct sum is exact it commutes with taking cohomology and thus direct sum commutes with taking derived functors. Thus Lemma \ref{lem:exact_seq_of_functors} gives the required long exact sequence, 
\begin{center}
\begin{tikzcd}
0 \arrow[r] & H^0_{Z_1 \cap Z_2}(X, \F) \arrow[r] & H^0_{Z_1}(X, \F) \oplus H^0_{Z_2}(X, \F) \arrow[draw=none]{d}[name=Z, shape=coordinate]{} \arrow[r] & H^0_{Z_1 \cup Z_2}(X, \F)
\arrow[dll,
rounded corners, crossing over,
to path={ -- ([xshift=2ex]\tikztostart.east)
|- (Z) [near end]\tikztonodes
-| ([xshift=-2ex]\tikztotarget.west)
-- (\tikztotarget)}]
\\ 
& H^1_{Z_1 \cap Z_2}(X, \F) \arrow[r] & H^1_{Z_1}(X, \F) \oplus H^1_{Z_2}(X, \F) \arrow[draw=none]{d}[name=Z', shape=coordinate]{} \arrow[r] & H^1_{Z_1 \cup Z_2}(X, \F) \arrow[dll,
rounded corners, crossing over,
to path={ -- ([xshift=2ex]\tikztostart.east)
|- (Z') [near end]\tikztonodes
-| ([xshift=-2ex]\tikztotarget.west)
-- (\tikztotarget)}]
\\
& H^2_{Z_1 \cap Z_2}(X, \F) \arrow[r] & H^2_{Z_1}(X, \F) \oplus H^2_{Z_2}(X, \F) \arrow[r] & H^2_{Z_1 \cup Z_2}(X, \F) \arrow[r] & \cdots
\end{tikzcd}
\end{center}



\subsection{Section 4}

\subsubsection{4.1}


Let $f : X \to Y$ be an affine morphism of schemes and $\F$ be a quasi-coherent $\struct{X}$-module. We proved in class that $R^q f_* \F = 0$ for $q \ge 0$ when $f$ is affine and $\F$ quasi-coherent (note that this proof uses the vanishing of higher cohomology for quasi-coherent sheaves on affine schemes which is difficult to prove without the Noetherian assumption but still true). Consider the commutative diagram of functors,
\begin{center}
\begin{tikzcd}
\Ab(X) \arrow[dr, "\Gamma_X"'] \arrow[rr, "f_*"] & & \Ab(Y) \arrow[dl, "\Gamma_Y"]
\\
& \Ab
\end{tikzcd}
\end{center}
Take an injective resolution of sheaves over $X$,
\begin{center}
\begin{tikzcd}
0 \arrow[r] & \F \arrow[r] & \I^\bullet
\end{tikzcd}
\end{center}
Because $f_*$ is a right-adjoint to the exact functor $f^{-1}$ by Lemma \ref{right_adjoint_to_exact_preserves_injectives}, $f_*$ preserves injectives. I claim that,
\begin{center}
\begin{tikzcd}
0 \arrow[r] & f_* \F \arrow[r] & f_* \I^\bullet 
\end{tikzcd}
\end{center}
is an injective resolution of sheaves over $Y$.
To show exactness, split the long exact resolution into short exact sequences of sheaves,
\begin{center}
\begin{tikzcd}
0 \arrow[r] & \F \arrow[r] & \I^0 \arrow[r] & \K^0 \arrow[r] & 0 
\end{tikzcd}
\end{center}
\begin{center}
\begin{tikzcd}
0 \arrow[r] & \K^{p-1} \arrow[r] & \I^p \arrow[r] & \K^p \arrow[r] & 0 
\end{tikzcd}
\end{center}
Now applying the long exact sequences of cohomology from the derived functors of the left-exact functor $f_*$ we get,
\begin{center}
\begin{tikzcd}
0 \arrow[r] & f_* \F \arrow[r] & f_* \I^0 \arrow[r] & f_* \K^0 \arrow[r] & R^1 f_* \F 
\end{tikzcd}
\end{center}
but $R^1 f_* \F$ vanishes so the sequence remains short exact and 
\begin{center}
\begin{tikzcd}
R^q f_* \F \arrow[r] & R^q f_* \I^0 \arrow[r] & R^q f_* \K^0 \arrow[r] & R^{q + 1} f_* \F 
\end{tikzcd}
\end{center}
but $R^{q+1} f_* \F = 0$ and $R^q f_* \I^0 = 0$ because $\I^0$ is injective so we find $R^q f_* \K^0 = 0$ for all $q \ge 0$. Now assume for induction that $R^q f_* \K^{p - 1} = 0$ for all $q \ge 0$. The long exact sequence then gives,
\begin{center}
\begin{tikzcd}
0 \arrow[r] & f_* \K^{p-1} \arrow[r] & f_* \I^p \arrow[r] & f_* \K^p \arrow[r] & R^1 f_* \K^{p - 1}
\end{tikzcd}
\end{center}
by the induction hypothesis $R^1 f_* \K^{p - 1} = 0$ so the sequence remains short exact. Furthermore the long exact sequence gives,
\begin{center}
\begin{tikzcd}
R^q f_* \K^{p-1} \arrow[r] & R^q f_* \I^p \arrow[r] & f_* R^q \K^p \arrow[r] & R^{q+1} f_* \K^{p - 1}
\end{tikzcd}
\end{center}
but $R^{q+1} f_* \F = 0$ and $R^q f_* \I^p = 0$ because $\I^p$ is injective so we find that $R^q f_* \K^p = 0$ for all $q \ge 0$ so we may proceed by induction. Thus we have shown that $f_*$ preserves each short exact sequences which, laced together, shows that
\begin{center}
\begin{tikzcd}
0 \arrow[r] & f_* \F \arrow[r] & f_* \I^\bullet 
\end{tikzcd}
\end{center}
is exact and thus an injective resolution. Therefore, we may directly compute,
\[ H^q(Y, f_* \F) = H^q(\Gamma(Y, f_* \I^\bullet)) = H^q(\Gamma(X, \I^\bullet)) = H^q(X, \F) \]

\begin{remark}
What I have shown here is a special case of the convergence of the Grothendieck spectral sequence applied to the left-exact functors $\Gamma(Y, -)$ and $f_*$ where $f_*$ takes injectives to injectives. This spectral sequence is characterized by,
\[ E^{p, q}_2 = H^p(Y, R^q f_* \F) \implies H^{p+q}(X, \F) \] 
In the case of an affine morphism $f : X \to Y$ and quasi-coherent $\struct{X}$-module $\F$, we have $R^q f_* \F = 0$ and thus $E^{pq}_2$ collapses to $E_2^{p0} = H^p(Y, f_* \F)$ in which case we know that,
\[ H^p(X, \F) = E^{p, 0}_2 = H^p(Y, f_* \F) \]
\end{remark}

\subsubsection{4.2}

Let $f : X \to Y$ be an finite surjective morphism of noetherian separated schemes. We will prove that if $X$ is affine then $Y$ is affine.

\begin{enumerate}
\item Let $f : X \to Y$ be a finite surjective morphism of integral noetherian schemes. Restrict to affine opens $U \subset X$ and $V \subset Y$ such that $U = f^{-1}(V)$ and denote $U = \Spec{B}$ and $V = \Spec{A}$ where $A$ and $B$ are noetherian integral domains. Then the sheaf map $f^\# : \struct{Y} \to f_* \struct{X}$ on $V$ gives an inclusion $\varphi : A \embed B$ which makes $B$ a finitely generated $A$-module since $f : X \to Y$ is finite. Let $K = \Frac{A}$ and $L = \Frac{B}$. Now localizing this map gives $\varphi : K \to S^{-1} B$ which makes $S^{-1} B$ a finitely generated $K$-module. However, $S^{-1}B$ is a domain and a finite $K$-vectorspace and thus a field. Thus $S^{-1} B = L$ since $L$ is the smallest field containing $A$. Therefore, $K \embed L$ is a finite extension. Taking generators $x_1, \dots, x_r \in B$ for the field extension, perhapse after clearing denominators, gives a map $A^r \to B$ such that $A^r \otimes_A K \to B \otimes_A K = L$ is an isomorphism of $K$-modules.
\bigskip\\
Let $j : U \to X$ be the inclusion and let $\M$ be a coherent subsheaf of $j_* \struct{U}$ generated by global sections $x_1, \dots, x_n$ i.e. the image of $\struct{X}^{\oplus r} \to \iota_* \struct{U}$ via sections $x_1, \dots, x_n \in \Gamma(X, j_* \struct{U}) = \Gamma(U, \struct{U}) = B$.
Now define $\alpha : \struct{Y}^{\oplus r} \to f_* \M$ via the sections $x_1, \dots, x_n \in \Gamma(Y, f_* \M) = \Gamma(X, \M)$. On the affine open $V$ we see that $\alpha_U : A^r \to (x_1, \dots, x_n)$ is the previous map and thus at the generic point $\alpha_\eta : \stalk{Y}{\eta} \to (f_* \M)_\eta$ is the map $\alpha_U \otimes_K : K^r \iso (x_1, \dots, x_n) \otimes_A K = L$ is an isomorphism. 

\item Let $\F$ be any coherent sheaf on $Y$. Consider,
\[ \alpha^* : \shHom{\struct{Y}}{f_* \M}{\F} \to \shHom{\struct{Y}}{\struct{Y}^{\oplus n}}{\F} = \F^{\oplus r} \]
However, $\shHom{\struct{Y}}{f_* \M}{\F}$ is a coherent $\struct{Y}$-module and $f_* \M$ is a module over $\sA = f_* \struct{X}$ meaning that $\shHom{\struct{Y}}{f_* \M}{\F}$ is a $\sA$-module. Using (Ex. II.5.17 (e)) because $f : X \to Y$ is finite and thus affine, there is some coherent $\struct{X}$-module $\G$ such that $\shHom{\struct{Y}}{f_* \M}{\F} = f_* \G$. Then we have a morphism $\alpha^* : f_* \G \to \F^{\oplus r}$. Furthermore, since $\alpha$ is an isomorphism at the generic point and $f_* \M$ and $\F$ and $\struct{Y}$ are coherent then internal hom commutes with taking stalks meaning that,
\[ \alpha^*_\eta : \shHom{\struct{Y}}{f_* \M}{\F}_\eta = \Hom{\stalk{Y}{\eta}}{(f_* \M)_\eta}{\F_\eta} \to \Hom{\stalk{Y}{\eta}}{\stalk{Y}{\eta}^r}{\F_\eta} = \F^{\oplus r}_\eta \]
is an isomorphism because $\alpha_\eta : \stalk{Y}{\eta}^r \to (f_* \M)_\eta$ is an isomorphism.


\item Let $f : X \to Y$ be a finite surjective morphism of noetherian separated schemes and $X$ is affine. We now will show that $Y$ is affine by Noetherian induction. By (Ex. III.3.1) and (Ex. III.3.2) $Y$ is affine iff $Y_\red$ is affine iff each irreducible component $Z \subset Y_\red$ is affine. Let $\cP$ be the property of closed subsets $Z \subset Y$ that $Z$ with its reduced subscheme structure is affine. Then if $Y$ has $\cP$ meaning $Y_\red$ is affine then $Y$ is affine so, by Noetherian induction, it suffices to show that if $Z \subset Y$ is a closed subset such that every proper closed subset $C \subsetneq Z$ has $\cP$ then $Z$ has $\cP$. Notice if $Z$ is reducible this is automatic because $Z$ is affine iff each irreducible component is affine by (Ex. III.3.2) thus we need only consider the case that $Z$ is irreducible.
\bigskip\\
Base changing by $Z \embed Y$ we get a finite surjective map $X_Z \to Z$ where $X_Z \embed X$ is a closed immersion so $X_Z$ is affine. Since $X_Z \to Z$ is surjective, some $\xi \in X_Z$ must hit the generic point $\eta \in Z$. Give $W = \overline{\{ \xi \}}$ the reduced subscheme structure then composing with the closed immersion $W \embed X_Z$ gives a finite map $f' : W \to Z$ which is dominant because $\xi \mapsto \eta$ and thus surjective since $f' : W \to Z$ is closed. Since $W$ is affine and both $W$ and $Z$ are integral we have reduced to the previous case.
\bigskip\\
We will show that $Z$ is affine by using Serre's criterion. For any coherent $\struct{Z}$-module $\F$, by part (b) there is a coherent $\struct{W}$-module $\G$ and a morphism $\beta : f_* \G \to \F^{\oplus r}$ which is an isomorphism at the generic point $\eta \in Z$. Extend to an exact sequence,
\begin{center}
\begin{tikzcd}
0 \arrow[r] & \ker{\beta} \arrow[r] & f_* \G \arrow[r, "\beta"] & \F^{\oplus r} \arrow[r] & \coker{\beta} \arrow[r] & 0
\end{tikzcd}
\end{center} 
Taking the stalk at $\eta$ gives an exact sequence,
\begin{center}
\begin{tikzcd}
0 \arrow[r] & (\ker{\beta})_\eta \arrow[r] & (f_* \G)_\eta \arrow[r, "\beta"] & \F^{\oplus r}_\eta \arrow[r] & (\coker{\beta})_\eta \arrow[r] & 0
\end{tikzcd}
\end{center} 
but $\beta$ is an isomorphism at $\eta$ so $(\ker{\beta})_\eta = (\coker{\beta})_\eta = 0$ and thus their supports are proper closed subsets $C_1$ and $C_2$ of $Z$. In particular, $\ker{\beta}$ and $\coker{\beta}$ are extensions of coherent sheaves on $C_1$ and $C_2$ (with possible nonreduced structure) but by the induction hypothesis $(C_i)_\red$ is affine and thus $C_i$ is affine so for $q > 0$,
\[ H^q(X, \ker{\beta}) = H^q(X, \iota_* \iota^* \ker{\beta}) = H^q(C_1, \iota^* \ker{\beta}) = 0 \]
and likewise $H^q(X, \coker{\beta}) = 0$. Now split the exact sequence into short exact sequences,
\begin{center}
\begin{tikzcd}
0 \arrow[r] & \ker{\beta} \arrow[r] & f_* \G \arrow[r] & \sC \arrow[r] & 0
\\
0 \arrow[r] & \sC \arrow[r] & \F^{\oplus r} \arrow[r] & \coker{\beta} \arrow[r] & 0
\end{tikzcd}
\end{center}
and consider the long exact sequences,
\begin{center}
\begin{tikzcd}
H^q(Z, \ker{\beta}) \arrow[r] & H^q(Z, f_* \G) \arrow[r] & H^q(Z, \sC) \arrow[r] & H^{q+1}(Z, \ker{\beta})
\\
H^q(Z, \sC) \arrow[r] & H^q(Z, \F)^{\oplus r} \arrow[r] & H^q(Z, \coker{\beta}) \arrow[r] & H^{q+1}(Z, \sC)
\end{tikzcd}
\end{center}
For $q > 0$ we see that $H^q(Z, f_* \G) \iso H^q(Z, \sC)$ and $H^q(Z, \sC) \onto H^q(Z, \F)^{\oplus r}$ by the vanishing of cohomology of $\ker{\beta}$ and $\coker{\beta}$. Furthermore, since $f$ is affine,
\[ H^q(Z, f_* \G) = H^q(W, \G) = 0 \]
because $W$ is affine and $\G$ is coherent. Thus we see that $H^q(Z, \F) = 0$ for $q > 0$ proving that $Z$ is affine by Serre's criterion and thus $Z$ satisfies $\cP$.
\end{enumerate}

\subsubsection{4.3}


It will be convenient to label variables as,
\[ \A^d_k = \Spec{k[x_0, \cdots, x_{d-1}]} \]
and $n = d-1$ to line up with the definitions in projective space. Consider the projection morphism $\pi : \A^{n+1}_k \setminus \{ (x_1, \dots, x_n) \} \to \P^{n}_k$ and let $U = \A^d_k \setminus \{ (x_1, \dots, x_n) \}$ and $X = \P^n_k$. The schemes $D_{+}(X_i)$ for each variable $X_i$ constitute an affine open cover of $\P^{n}_k$. Furthermore, $\pi^{-1}(D_{+}(X_i)) = D(x_{i}) \subset k[x_1, \dots, x_d]$. Therefore, $\pi$ is an affine morphism and $\struct{U}$ is a quasi-coherent $\struct{U}$-module so we have shown that,
\[ H^q(\P^{n}_k, \pi_* \struct{U}) = H^q(U, \struct{U}) \] 
Furthermore, denote $S = k[x_0, \cdots, x_n]$, then,
\begin{align*}
\pi_* \struct{U} |_{D_{+}(X_i)} & = \struct{U} |_{D(x_{i})} = \struct{\A^{n+1}_k} |_{D(x_{i})} = \widetilde{S_{x_{i}}}  = \bigoplus_{k \in \Z} \widetilde{\left( S_{x_i} \right)_k} = \bigoplus_{k \in \Z} \widetilde{\left( S(k)_{x_i} \right)_{0}} = \bigoplus_{k \in \Z} \struct{X}(k)|_{D_+(X_i)}
\end{align*}
Thus, because the sheaves agree on an open affine cover, we can identify,
\[ \pi_* \struct{U} = \bigoplus_{k \in \Z} \struct{X}(k) \]
Hartshorne has computed the cohomology of the sum of twists (Hartshorne III.5, Theorem 5.1) to be,
\[  H^q \left(X, \bigoplus_{n \in k} \struct{X}(k) \right) = 
\begin{cases}
k[X_0, \cdots, X_n] & q = 0
\\
0 & 0 < q < n
\\
\frac{1}{X_0 \cdots X_n} k[X_0^{-1}, \dots, X_n^{-1}] & q = n 
\end{cases} \]
Reverting to our initial notation and using the isomorphism $H^q(X, \pi_* \struct{U}) = H^q(U, \struct{U})$ we arrive at,
\[ H^q(U, \struct{U}) = 
\begin{cases}
k[x_1, \cdots, x_d] & q = 0
\\
0 & 0 < q < n
\\
\frac{1}{x_1 \cdots x_d} k[x_1^{-1}, \dots, x_d^{-1}] & q = d-1 
\end{cases} \]
Therefore $U$ is not affine since $\struct{U}$ is coherent and yet has nontrivial cohomology on $U$.

\subsubsection{4.4}

Let $X$ be a topological space and $\F$ an abelian sheaf. 

\renewcommand{\U}{\mathfrak{U}}
\newcommand{\V}{\mathfrak{V}}

\begin{enumerate}
\item Let $\U = (U_i)_{i \in I}$ be an open cover of $X$ and $\V = (V_j)_{j \in J}$ a refinement i.e. an open cover of $X$ with a map $\lambda : J \to I$ of index sets such that $V_{j} \subset U_{\lambda(j)}$.  This refinement gives a morphism of Cech complexes $r : \check{C}^\bullet(\U, \F) \to \check{C}^\bullet(\V, \F)$ via the restriction map,
\[ \res : \prod_{i_0 < \cdots < i_r} \F(U_{i_0, \dots, i_r}) \to \prod_{j_0 < \cdots < j_r} \F(V_{j_0, \dots, j_r}) \quad \quad (\xi_{i_0, \dots, i_r}) \mapsto (\xi_{\lambda(j_0), \dots, \lambda(j_r)} |_{V_{j_0, \dots, j_r}} ) \]
making the diagram commute,
\begin{center}
\begin{tikzcd}
0 \arrow[r] & \prod\limits_{i_0} \F(U_{i_0}) \arrow[d] \arrow[r] & \prod\limits_{i_0 < i_1} \F(U_{i_0, i_1}) \arrow[d] \arrow[r] & \prod\limits_{i_0 < i_1 < i_2} \F(U_{i_0, i_1, i_2}) \arrow[d] \arrow[r] & \cdots
\\ 
0 \arrow[r] & \prod\limits_{j_0} \F(V_{j_0}) \arrow[r] & \prod\limits_{j_0 < j_1} \F(V_{j_0, j_1}) \arrow[r] & \prod\limits_{j_0 < j_1 < j_2} \F(V_{j_0, j_1, j_2}) \arrow[r] & \cdots
\end{tikzcd}
\end{center}
This induces a map of the cohomologies, 
\[ \lambda^q : \check{H}^q(\U, \F) \to \check{H}^q(\V, \F) \]

\item Take an injective resolution of abelian sheaves,
\begin{center}
\begin{tikzcd}
0 \arrow[r] & \F \arrow[r] & \I^\bullet
\end{tikzcd}
\end{center}
and consider the diagram of abelian sheaves lifting $\id_\F : \F \to \F$,
\begin{center}
\begin{tikzcd}[column sep={4em,between origins},row sep=1em]
& 0 \arrow[rr] & & \F \arrow[rr] \arrow[from = dd] & & \I^\bullet
\\
0 \arrow[rr] & & \F \arrow[rr, crossing over] \arrow[ru] & & \I^\bullet \arrow[ru, "\id_\I"]
\\
& 0 \arrow[rr] & & \F \arrow[rr]  & & \Cech^\bullet(\V, \F) \arrow[uu, dashed]
\\
0 \arrow[rr] & & \F \arrow[rr] \arrow[uu, crossing over] \arrow[ur] & & \Cech^\bullet(\U, \F) \arrow[uu, dashed, crossing over] \arrow[ur, "\lambda"']
\end{tikzcd}
\end{center}
Because the Cech resolutions are exact and $\I$ is an injective resolution $\id_\F$ lifts to morphisms of complexes $\Cech^\bullet(\U, \F) \to \I^\bullet$ and $\Cech^\bullet(\V, \F) \to \I^\bullet$. Since these lifts are unique up to homotopy and $\lambda$ is a chain map, the last square commutes up to homotopy. Taking the cohomology of the above complexes, these morphisms induce maps $\beta_\U^q : \check{H}^q(\U, \F) \to H^q(X, \F)$ and $\beta_\V^q : \check{H}^q(\V, \F) \to H^q(X, \F)$ making the following diagram commute,
\begin{center}
\begin{tikzcd}
\check{H}^q(\U, \F) \arrow[d, "\lambda^q"] \arrow[r, "\beta_\U^q"] & H^q(X, \F) \arrow[d, "\id"]
\\
\check{H}^q(\V, \F) \arrow[r, "\beta_\V^q"] & H^q(X, \F)
\end{tikzcd}
\end{center}
Consider the directed system of all open covers of $X$ partially ordered under refinement over which we define,
\[ \check{H}^q(X, \F) = \varinjlim_{\U} \check{H}(\U, \F) \]
with the given restricton maps $\lambda$. The morphisms $\beta^q_\U : \check{H}^q(\U, \F) \to H^q(X, \F)$ are compatible with the restrictions and thus define a natural morphism,
\[ \check{H}^q(X, \F) \to H(X, \F) \]  

\item Take the abelian sheaf $\F$ an inject it into a flasque sheaf $\G$ to give an exact sequence of sheaves,
\begin{center}
\begin{tikzcd}
0 \arrow[r] & \F \arrow[r] & \G \arrow[r] & \K \arrow[r] & 0
\end{tikzcd}
\end{center}
Given this injection we construct an exact sequence of complexes,
\begin{center}
\begin{tikzcd}
0 \arrow[r] & \check{C}^\bullet(\U, \F) \arrow[r] & \check{C}^\bullet(\U, \G) \arrow[r] & D^\bullet(\U) \arrow[r] & 0
\end{tikzcd}
\end{center}

\renewcommand{\Q}{\mathscr{Q}}

where $D^\bullet(\U)$ is the cokernel complex which is given by $D^\bullet(\U) = \check{C}^\bullet(\U, \Q)$ where $\Q$ is the presheaf $U \mapsto \G(U) / \F(U)$ and thus $\K$ is its sheafification. Therefore, there is a natural sheafification map $\Q \to \K$ which, as a morphism of presheaves, induces a map of Cech complexes, $D^\bullet(\U) \to \check{C}^\bullet(\U, \K)$. Furthermore, because the map $\Q \to \K$ is an isomorphism on the stalks, under refinement we have,
\[ \varinjlim_\U D^\bullet(\U) = \varinjlim_\U \check{C}^\bullet(\U, \Q) \xrightarrow{\sim} \varinjlim_\U \check{C}^\bullet(\U, \K) \]
Now the above exact sequence of sheaves and exact sequence of complexes give long exact sequences of sheaf and Cech cohomology respectivly,
\begin{center}
\begin{tikzcd}
0 \arrow[r] & \Gamma(X, \F) \arrow[r] & \Gamma(X, \G) \arrow[r] & \Gamma(X, \K) \arrow[r] & H^1(X, \F) \arrow[r] & 0
\end{tikzcd}
\end{center}
where I have used the fact that $H^1(X, \G) = 0$ because $\G$ is flasque and, 
\begin{center}
\begin{tikzcd}
0 \arrow[r] & \Gamma(X, \F) \arrow[r] & \Gamma(X, \G) \arrow[r] & H^0(D^\bullet(\U)) \arrow[r] & \check{H}^1(\U, \F) \arrow[r] & 0
\end{tikzcd}
\end{center}
where I have used the fact that $\check{H}^0(\U, \F) = H^0(\check{C}^\bullet(\U, \F)) = \Gamma(X, \F)$ for any sheaf $\F$ and that $\check{H}^1(\U, \G) = H^1(\check{C}^\bullet(\U, \G)) = 0$ because $\G$ is flasque. The morphism of complexes $D^\bullet(\U) \to \check{C}^\bullet(\U, \K)$ induces a map of cohomology,
\[ H^0(D^\bullet(\U)) \to H^0(\check{C}^\bullet(\U, \K)) = \check{H}^0(\U, \K) = \Gamma(X, \K) \]
and thus we get a morphism of exact sequences,
\begin{center}
\begin{tikzcd}
0 \arrow[r] & \Gamma(X, \F) \arrow[d] \arrow[r] & \Gamma(X, \G) \arrow[d] \arrow[r] & H^0(D^\bullet(\U)) \arrow[d] \arrow[r] & \check{H}^1(\U, \F) \arrow[d] \arrow[r] & 0
\\
0 \arrow[r] & \Gamma(X, \F) \arrow[r] & \Gamma(X, \G) \arrow[r] & \Gamma(X, \K) \arrow[r] & H^1(X, \F) \arrow[r] & 0
\end{tikzcd}
\end{center}
Because the poset of covers under refinement maps is filtered the direct limit functor is exact. Appling it to the second sequence gives an exact sequence,
\begin{center}
\begin{tikzcd}
0 \arrow[r] &  \Gamma(X, \F) \arrow[d] \arrow[r] & \Gamma(X, \G) \arrow[d] \arrow[r] & \varinjlim\limits_\U H^0(D^\bullet(\U)) \arrow[r] \arrow[d] & \varinjlim\limits_\U \check{H}^1(\U, \F) \arrow[r] \arrow[d] & 0
\\
0 \arrow[r] & \Gamma(X, \F) \arrow[r] & \Gamma(X, \G) \arrow[r] & \Gamma(X, \K) \arrow[r] & H^1(X, \F) \arrow[r] & 0
\end{tikzcd}
\end{center}
Furthermore, $\varinjlim\limits_\U$ is exact so it commutes with taking cohomology so the maps,
\begin{center}
\begin{tikzcd}
H^0(\varinjlim\limits_\U D^\bullet(\U)) \arrow[r, "\sim"] & H^0(\varinjlim\limits_\U \check{C}^\bullet(\U, \K))
\\
\varinjlim\limits_\U H^0(D^\bullet(\U)) \arrow[r] \arrow[u, equals] & \varinjlim\limits_\U H^0(\check{C}^\bullet(\U, \K)) \arrow[u, equals]  \arrow[r, equals] & \varinjlim\limits_\U \check{H}^0(\U, \K) = \Gamma(X, \K)
\end{tikzcd}
\end{center}
are isomorphisms. Therefore since cokernels are unique, the map,
\[ \varinjlim\limits_\U \check{H}^1(\U, \F) \to H^1(X, \F) \]
is an isomorphism. 


\renewcommand{\Q}{\mathbb{Q}}
\end{enumerate}


\subsubsection{4.5}


Let $X$ be a locally ringed space. Notate by $\struct{X}^\times$, the sheaf of abelian groups given by $U \mapsto \struct{X}(U)^\times$. Now let $\L$ be an invertable sheaf on $X$ meaning that there exists an open cover $\mathfrak{U}$ such that for each $U \in \mathfrak{U}$ we have isomorphisms $\varphi_U : \struct{X}|_U \to \L|_U$. Therefore, on the overlaps we have isomorphism,
\[ \varphi_{ij} = \varphi_{U_i}^{-1}|_{U_i \cap U_j} \circ \varphi_{U_j} |_{U_i \cap U_j} : \struct{X} |_{U_i \cap U_j} \to \struct{X} |_{U_i \cap U_j} \]
which, as $\struct{X}|_{U_i \cap U_j}$-module maps are determined uniquely by $e_{ij} \in \struct{X}(U_i \cap U_j)^\times$ which is a unit because the map it defines is an isomorphism. Thus, $e = (e_{ij})_{ij}$ is an element of the first Cech complex group, $C^1(\mathfrak{U}, \struct{X}^\times)$. Consider the Cech complex,
\begin{center}
\begin{tikzcd}
0 \arrow[r] & \prod\limits_{i_0} \struct{X}^\times(U_{i_0}) \arrow[r] & \prod\limits_{i_0 < i_1}  \struct{X}^\times(U_{i_0} \cap U_{i_1}) \arrow[r] & \prod\limits_{i_0 < i_1 < i_2} \struct{X}^\times(U_{i_0} \cap U_{i_1} \cap U_{i_2}) 
\end{tikzcd}
\end{center}
Furthermore, on triple overlaps,
\begin{align*}
\varphi_{ij}|_{ijk} \circ \varphi_{jk}|_{ijk} & = \varphi_{U_i}^{-1}|_{U_{ijk}} \circ \varphi_{U_j} |_{U_{ijk}} \circ \varphi_{U_j}^{-1}|_{U_{ijk}} \circ \varphi_{U_k} |_{U_{ijk}} 
\\
& = \varphi_{U_i}^{-1}|_{U_{ijk}} \circ \varphi_{U_k} |_{U_i \cap U_j \cap U_k} = \varphi_{ik} |_{ijk} 
\end{align*}
which clearly implies that $e_{ij} |_{U_{ijk}} \cdot e_{jk} |_{U_{ijk}} = e_{ik} |_{U_{ijk}}$. However, the Cech differential map $\mathrm{d} : C^1(\mathfrak{U}, \struct{X}^\times) \to \check{C}^2(\mathfrak{U}, \struct{X}^\times)$ acts via,
\[ (\d{\alpha})_{ijk} = \alpha_{jk} |_{U_{ijk}} \cdot \alpha_{ik}^{-1} |_{U_{ijk}} \cdot \alpha_{ij} |_{U_{ijk}} \]
Therefore, by the overlap identity,
\[ (\d{e})_{ijk} = e_{jk}|_{U_{ijk}} \cdot e_{ik}|_{U_{ijk}}^{-1} \cdot e_{ij} |_{U_{ijk}} = 1 \]
Thus $e$ is in the kernel of the Cech differential $\mathrm{d} : \check{C}^1(\mathfrak{U}, \struct{X}^\times) \to \check{C}^2(\mathfrak{U}, \struct{X}^\times)$ and thus $e$ represents a Cech cohomology class $[e] \in \check{H}^1(\mathfrak{U}, \struct{X}^\times)$. Furthermore, if $\tilde{\varphi}_{U_i} : \struct{X} |_{U_i} \to \L |_{U_i}$ is another choice of locally trivializing isomorphisms then denote $\tilde{e}_{ij} \in \struct{X}^\times(U_i \cap U_j)$ for the element determining the isomorphisms,
\[ \tilde{\varphi}_{ij} = \tilde{\varphi}_{U_i}^{-1} |_{U_{ij}} \circ \tilde{\varphi}_{U_j} |_{U_{ij}} : \struct{X} |_{U_i \cap U_j} \to \struct{X} |_{U_i \cap U_j} \]
Then we may consider the isomorphisms $t_i = \tilde{\varphi}_{U_i}^{-1} \circ \varphi_{U_i} : \struct{X} |_{U_i} \to \struct{X} |_{U_i}$ which are defined by an element $f_i \in \struct{X}^\times(U_i)$. Then we find that,
\begin{align*}
\tilde{\varphi}_{ij} & = \tilde{\varphi}_{U_i}^{-1} |_{U_{ij}} \circ \tilde{\varphi}_{U_j} |_{U_{ij}} = \tilde{\varphi}_{U_i}^{-1} |_{U_{ij}} \circ \varphi_{U_i} |_{U_{ijk}} \circ \varphi_{U_i}^{-1} |_{U_{ijk}} \circ \varphi_{U_j} |_{U_{ijk}} \circ  \varphi_{U_j} |_{U_{ijk}}^{-1} \circ \tilde{\varphi}_{U_j} |_{U_{ij}} 
\\
& = t_i |_{U_{ij}} \circ \varphi_{ij} \circ t_j^{-1} |_{U_{ij}}
\end{align*}
This shows that the elements must satisfy, $\tilde{e}_{ij} \cdot e_{ij}^{-1} = t_i |_{U_{ij}} \cdot t_j^{-1} |_{U_{ij}}$. Furthermore, the Cech differential map $\mathrm{d} : \check{C}^0(\mathfrak{U}, \struct{X}^\times) \to \check{C}^1(\mathfrak{U}, \struct{X}^\times)$ acts via,
\[ (\d{\alpha})_{ij} = \alpha_{i} |_{U_{ij}} \cdot \alpha_{j}^{-1} |_{U_{ij}}  \]
Therefore, let $f = (f_i)_i$ then $\d{f} = \tilde{e} \cdot e^{-1}$ which implies that $[\tilde{e}] = [e]$ in $\check{H}^1(\mathfrak{U}, \struct{X}^\times)$ so the cohomology class $[e]$ associated to the invertable sheaf $\L$ is well-defined. The map $\L \mapsto [e]$ is well-defined for sheaves which are locally trivialized on $\mathfrak{U}$. Therefore we get a well-defined map,
\[ \Pic{X} \to \check{H}^1(X, \struct{X}^\times) = \varinjlim_{\mathfrak{U}} \check{H}(\mathfrak{U}, \struct{X}^\times) \]
via decomposing,
\[ \Pic{X} = \bigcup_{\mathfrak{U}} \Pic{\mathfrak{U}, X} \quad \text{where} \quad \Pic{\mathfrak{U}, X} = \{ \mathcal{L} \in \Pic{X} \mid \forall U \in \mathfrak{U} : \mathcal{L}|_U \cong \struct{U} \} \]
and mapping,
\[ \Pic{\mathfrak{U}, X} \to \check{H}^1(\mathfrak{U}, \struct{X}^\times) \to \varinjlim_{\mathfrak{U}} \check{H}(\mathfrak{U}, \struct{X}^\times) = \check{H}^1(X, \struct{X}^\times) \]
using the constructed map. This map is an homomorphism because given invertable sheaves $\L_1$ and $\L_2$ and isomorphisms $\varphi^r_{U_i} : \struct{X} |_{U_i} \to \L_r$ corresponding to cohomology classes $[e^r]$ then there is a natural map,
\[ \varphi^1_{U_i} \otimes \varphi^2_{U_i} \struct{X}|_{U_i} \to \L_1 |_{U_i} \otimes_{\struct{X} |_{U_i}} \L_2 |_{U_i} \]
which therefore gives overlap maps,
\[ \varphi_{ij}^\otimes = ((\varphi^1_{U_i})^{-1} \circ \varphi^1_{U_j}) \otimes ((\varphi^2_{U_i})^{-1} \circ \varphi^2_{U_j}) = \varphi_{ij}^1 \otimes \varphi_{ij}^2 \]
and thus, $\varphi_{ij}^\otimes(1) = e_{ij}^1 \otimes e^2_{ij} \mapsto e_{ij}^1 e_{ij}^2$ under the natural identification,
\[ \struct{X}(U_{ij}) \otimes_{\struct{X}(U_{ij})} \struct{X}(U_{ij}) \to \struct{X}(U_{ij}) \]
Therefore, the invertable sheaf $\L_1 \otimes_{\struct{X}} \L_2$ maps to the cohomology class $[e^1 e^2] = [e^1] [e^2]$ so this map is a homomorphism. 
\bigskip\\
I claim that this map is, in fact, an isomorphism. Let $\L$ be an invertable sheaf represented by the cohomology class $[e] = [1]$ then we know that $e_{ij} = t_i |_{U_{ij}} \cdot t_j^{-1} |_{U_{ij}}$ for some set of invertable sections $t_i$. Therefore, modify the isomorphism $\varphi_{U_i} : \struct{X}|_{U_i} \to \L |_{U_i}$ which gave rise to this cohomology representative via $\tilde{\varphi}_{U_i} = t_i \varphi_{U_i}$ which are still isomorphism because $t_i \in \struct{X}(U_i)^\times$ is invertable. Therefore, 
\[ \tilde{\varphi}_{U_i}^{-1}|_{U_{ij}} \circ \tilde{\varphi}_{U_j}|_{U_{ij}} = (t_i |_{U_{ij}}^{-1} \cdot t_j |_{U_{ij}}) \varphi_{U_i}^{-1}|_{U_{ij}} \circ \varphi_{U_j}|_{U_{ij}} = \id_{\struct{X}(U_{ij})} \]
this map takes $1 \mapsto (t_i |_{U_{ij}}^{-1} \cdot t_j |_{U_{ij}}) e_{ij} = 1$ so as a morphism of $\struct{X}|_{U_{ij}}$-modules is the identity map. Thus $\tilde{\varphi}_{U_i} |_{U_{ij}} = \tilde{\varphi}_{U_j} |_{U_{ij}}$, so the isomorphisms $\tilde{\varphi}_{U_i} \in \shHom{}{\struct{X}|_{U_i}}{\L|_{U_i}}$ glue since they agree on this open cover to a global isomorphism $\tilde{\varphi} : \struct{X} \to \L$ so $\L$ is a trivial invertable sheaf. Thus $\Pic{X} \to \check{H}^1(X, \struct{X})$ is injective. It remains to prove that it is surjective. Given any cohomology class $[e] \in \check{H}^1(X, \struct{X}^\times)$ we may construct an invertable sheaf as follows. Define $\L$ via,
\[ \L(V) = \{ f_i \in \struct{X}(U_i \cap V) \mid f_i |_{U_{ij} \cap V} \cdot e_{ij} |_{U_{ij} \cap V} = f_j |_{U_{ij} \cap V} \} \]
It is clear that this is an invertable sheaf if $e_{ij}$ satisfies the transition property given by its Cech differential vanishing and that $\L \mapsto [e]$. 
\bigskip\\
Finally, we use the general fact that $H^1(X, \F) = \check{H}^1(X, \F)$ to conclude that,
\[ \Pic{X} \cong H^1(X, \struct{X}^\times) \]

\subsubsection{4.6}

Let $(X, \struct{X})$ be a ringed space and $\I$ be a sheaf of ideals of $\struct{X}$ such that $\I^2 = 0$. Let $X_0$ be the ringed space $(X, \struct{X} / \I)$. Now consider the sequence of sheaves over $X$,
\begin{center}
\begin{tikzcd}
0 \arrow[r] & \I \arrow[r] & \struct{X}^\times \arrow[r] & \struct{X_0}^\times \arrow[r] & 0
\end{tikzcd}
\end{center}
where $\I \to \struct{X}^\times$ is the map $a \mapsto 1 + a$ which is a unit because $(1 + a)(1 - a) = 1 - a^2 = 1$ since $a^2 \in \I^2 = 0$. This map is clearly injective. The map $\struct{X}^\times \to \struct{X_0}^\times$ is the projection. At the stalks $\stalk{X}{x}^\times \to \stalk{X_0}{x}^\times$ the map is simply the projection $\stalk{X}{x}^\times \to (\stalk{X}{x} / \I_x)^\times$. Now if $ab - 1 \in \I_x$ then $ab = 1 + z$ for some $z \in \I_x$. Thus,
\[ ab(1 - z) = (1 + z)(1 - z) = 1 - z^2 = 1 \]
so $a \in \stalk{X}{x}^\times$ is actually invertible i.e. the stalk maps are surjective so $\struct{X}^\times \to \struct{X_0}^\times$ is a surjective morphism of sheaves. Now if $a \in \I_x$ then $1 + a = 1$ in $\stalk{X}{x} / \I_x$ so the sequence is a complex. Futhermore, if $1 + a = 1$ in $\stalk{X}{x} / \I_x$ then $a \in \I_x$ so the sequence is exact. Therefore, appling the long exact sequence of cohomology we get,
\begin{center}
\begin{tikzcd}
H^1(X, \I) \arrow[r] & H^1(X, \struct{X}) \arrow[r] & H^1(X, \struct{X_0}) \arrow[r] & H^2(X, \I)
\end{tikzcd}
\end{center}
Furthermore, using the identification $H^1(X, \struct{X}) = \Pic{X}$ for any ringed space and since topologically $X = X_0$ so $H^1(X, \struct{X_0}) = H^1(X_0, \struct{X_0}) = \Pic{X_0}$ we find an exact sequence,
\begin{center}
\begin{tikzcd}
H^1(X, \I) \arrow[r] & \Pic{X} \arrow[r] & \Pic{X_0} \arrow[r] & H^2(X, \I)
\end{tikzcd}
\end{center}

\subsubsection{4.7}

Let $X$ be the closed subscheme of $\P^2_k$ defined by the homogeneous polynomial $f(x_0, x_1, x_2) = 0$ of degree $d$. Let $S = k[x_0, x_1, x_2]$ be the graded ring such that $\P^2_k = \Proj{S}$. Now consider the exact sequence of graded rings,
\begin{center}
\begin{tikzcd}
0 \arrow[r] & S(-d) \arrow[r, "\times f"] & S \arrow[r] & S / (f) \arrow[r] & 0 
\end{tikzcd}
\end{center}
which gives an exact sequence of $\struct{\P^2_k}$-modules,
\begin{center}
\begin{tikzcd}
0 \arrow[r] & \widetilde{S(-d)} \arrow[r, "\times f"] & \widetilde{S} \arrow[r] & \widetilde{S / (f)} \arrow[r] & 0 
\end{tikzcd}
\end{center}
Let $\iota : \Proj{S/(f)} \to \Proj{S}$ be the closed immersion of the closed subscheme $X = \Proj{S/(f)}$ which is the plane curve corresponding the vanishing of $f$. Then $\widetilde{S / (f)} = \iota_* \struct{X}$ so we may rewrite this exact sequence as,
\begin{center}
\begin{tikzcd}
0 \arrow[r] & \struct{\P^2_k}(-d) \arrow[r, "\times f"] & \struct{\P^2_k} \arrow[r] & \iota_* \struct{X} \arrow[r] & 0 
\end{tikzcd}
\end{center}
Taking the long exact sequence of cohmology we find,
\begin{center}
\begin{tikzcd}[column sep = small]
0 \arrow[r] & H^0(\P^2_k, \struct{\P^2_k}(-d)) \arrow[draw=none]{d}[name=Z, shape=coordinate]{} \arrow[r] & H^0(\P^2_k, \struct{\P^2_k}) \arrow[r] & H^0(\P^2_k, \iota_* \struct{X}) \arrow[r] & H^1(\P^2_k, \struct{\P^2_k}(-d)) 
\arrow[dlll,
rounded corners, crossing over,
to path={ -- ([xshift=2ex]\tikztostart.east)
|- (Z) [near end]\tikztonodes
-| ([xshift=-2ex]\tikztotarget.west)
-- (\tikztotarget)}]
\\ 
& H^1(\P^2_k, \struct{\P^2_k}) \arrow[r] & H^1(\P^2_k, \iota_* \struct{X}) \arrow[r] & H^2(\P^2_k, \struct{\P^2_k}(-d)) \arrow[r] & H^2(\P^2_k, \struct{\P^2_k}) \arrow[r] & 0
\end{tikzcd}
\end{center}
Since $\iota : X \to \P^2_k$ is a closed immersion it is affine and thus,
\[ H^q(\P^2_k, \iota_* \F) = H^q(X, \F) \]
for any quasi-coherent $\struct{X}$-module and $q \ge 0$. In partiuclar, $H^q(\P^2_k, \iota_* \struct{X}) = H^q(X, \struct{X})$ and also we know that $H^1(\P^2_k, \struct{\P^2_k}(n)) = 0$. Therefore, the long exact sequence becomes,
\begin{center}
\begin{tikzcd}[column sep = small]
0 \arrow[r] & H^0(\P^2_k, \struct{\P^2_k}(-d)) \arrow[draw=none]{d}[name=Z, shape=coordinate]{} \arrow[r] & H^0(\P^2_k, \struct{\P^2_k}) \arrow[r] & H^0(X, \struct{X}) \arrow[r] & 0
\arrow[dlll,
rounded corners, crossing over,
to path={ -- ([xshift=2ex]\tikztostart.east)
|- (Z) [near end]\tikztonodes
-| ([xshift=-2ex]\tikztotarget.west)
-- (\tikztotarget)}]
\\ 
& 0 \arrow[r] & H^1(X, \struct{X}) \arrow[r] & H^2(\P^2_k, \struct{\P^2_k}(-d)) \arrow[r] & H^2(\P^2_k, \struct{\P^2_k}) \arrow[r] & 0
\end{tikzcd}
\end{center}
Furthermore, since $-d < 0$ then $H^0(\P^2_k, \struct{\P^2_k}(-d)) = 0$ (because $S$ has no negative degree terms) and we know $H^0(\P^2_k, \struct{\P^2_k}) = k$. Therefore, $H^0(X, \struct{X}) = k$ and, in particular, it has dimension 1. Furthermore, 
\[ H^2(\P^2_k, \struct{\P^2_k}(n)) = \left( \frac{1}{x_0 x_1 x_2} k[x_0^{-1}, x_1^{-1}, x_2^{-1}] \right)_n \]
Thus, $\dim_k H^2(\P^2_k, \struct{\P^2_k}) = 0$ which implies that,
\[ H^1(X, \struct{X}) = H^2(\P^2_k, \struct{\P^2_k}(-d)) \]
Furthermore, a basis is given by $x_0^{-(a + 1)} x_1^{-(b + 1)} x_2^{-(c + 1)}$ where $a + b + c + 3 = d$. The number of solutions with $0 \le a,b,c \le d - 3$ is given as follows. There are $(d - 2)$ choices for $a$ in which case there are $d - a - 2$ choices for $b$ which fixes $c$. Then the number of solutions is thus,
\begin{align*}
\dim_k H^1(X, \struct{X}) & = \sum_{a = 0}^{d - 3} (d - a - 2) = (d - 2)^2 - \sum_{a = 0}^{d - 3} a 
\\
& = (d - 2)^2 - \tfrac{1}{2} (d - 3)(d - 2) = \tfrac{1}{2} (d - 1)(d - 2)
\end{align*}

\subsubsection{4.8}

\newcommand{\cd}[1]{\mathrm{cd}\left( #1 \right)}
\renewcommand{\U}{\mathfrak{U}}

Let $X$ be a noetherian separated scheme. Define the cohomological dimension $\cd{X}$ of $X$ as the minimal integer $n$ such that $H^i(X, \F) = 0$ for all quasi-coherent sheaves $\F$ and all $i  > n$. 

\begin{enumerate}
\item To show we can replace quasi-coherent with coherent in the definition, it suffices to show that fixing $i$ if $H^i(X, \F) = 0$ for all coherent sheaves $\F$ then $H^i(X, \G) = 0$ for all quasi-coherent sheaves $\G$. However, by (Ex. II.5.15(e)) we can write any quasi-coherent sheaf $\G$ as a direct limit over coherent subsheaves,
\[ \G = \varinjlim \F_\alpha \]
and then by III.2.9 we have,
\[ H^q(X, \G) = H^q(X, \varinjlim \F_\alpha) = \varinjlim H^q(X, \F_\alpha) = 0 \]

\item Let $X$ be quasi-projective over a field $k$ so there is an ample line bundle $\L$ on $X$. Clearly for any finite locally free $\struct{X}$-module $\E$ we know $H^i(X, \E) = 0$ for all $i > \cd{X}$. Therefore, it suffices to assume $H^i(X, \E)$ for all finite locally free $\E$ and all $i > n$ and conclude that $n \ge \cd{X}$. We need to show that for each coherent sheaf $\F$ that $H^i(X, \F) = 0$ for $i > n$. We proceed by descending induction on $i$. For $i > \cd{X}$ this is obvious. Now assume for $i$ and use the ampleness of $\L$ to choose a surjection from a finite locally free module $\E$ which is a sum of twists of $\L$. Extending to an exact sequence,
\begin{center}
\begin{tikzcd}
0 \arrow[r] & \G \arrow[r] & \E \arrow[r] & \F \arrow[r] & 0
\end{tikzcd}
\end{center}
Therefore, we get a long exact sequence,
\begin{center}
\begin{tikzcd}
H^i(X, \E) \arrow[r] & H^i(X, \F) \arrow[r] & H^{i+1}(X, \G) \arrow[r] & H^{i+1}(X, \E)
\end{tikzcd}
\end{center}
For $i > n$ we have $H^i(X, \E) = H^{i+1}(X, \E) = 0$ and thus $H^i(X, \F) \iso H^{i+1}(X, \G)$ and by the induction hypothesis $H^{i+1}(X, \G) = 0$ so $H^i(X, \F) = 0$ and thus by induction $n \ge \cd{X}$.
 
\item Suppose that $X$ has a covering by $r+1$ affine open subsets $\U = \{ U_i \}$. On a Noetherian separated scheme, Cech cohomology on affine covers computes derived cohomology for quasi-coherent sheaves and thus,
\[ H^i(X, \F) = \check{H}^i(\U, \F) = H^i(\check{C}^\bullet(\U, \F)) \]
However, for $i > r$ we have $\check{C}^i(\U, \F) = 0$ because there are only $r+1$ values for the $i+1$ indices and repetition is not allowed. Therefore, for $i > r$ we find $H^i(X, \F) = 0$ for all quasi-coherent sheaves and thus $\cd{X} \le r$.

\item Let $X$ be quasi-projective over dimension $r$ over a field $k$. We need to show that $X$ has a cover by $\dim{X} + 1$ affine open subsets. Given this, by (c) we immediately see that $\cd{X} \le \dim{X}$. 
\bigskip\\
Now we prove the claim by induction on $r = \dim{X}$. We can take the projective closure of $X$ under an immersion $j : X \to \P^n$ to reduce to the case that $X$ is projective. This suffices because an affine open cover of $\overline{X}$ intersects to an affine open cover of $X$ because $\overline{X}$ is separated. First, projective schemes of dimension $0$ are affine since they are a finite discrete set of (possibly nonreduced) points and thus lie in the complement of a suitable hyperplane not passing through the finitely many points. Given a projective scheme $X \subset \P^n_k$ of dimension $r+1$ take a general hyperplane section $X \cap H \subset \P^{n-1}_k$ such that $\dim{X \cap H} = r$. Then by induction, $X \cap H$ can be covered by $r+1$ affine opens $U_0, \dots, U_{r}$ which are the complements of hyperplane sections in $H$. Thus, these extend to opens $U'_0, \dots, U'_r$ of $X$ which are the complements of hyperplane sections in $\P^n_k$ because we can always choose a hyperplane intersecting $H$ at a given hyperplane of $H$. These cover $X \cap H$ and $U_{r+1} = X \cap (\P^n \setminus H)$ is affine because $X \embed \P^n_k$ is affine and the complement of a hyperplane is affine. Thus $U_0', \dots, U_r', U_{r+1}$ is an affine open cover of $X$ proving the claim by induction.

\item Suppose that $Y$ is the set-theoretic intersection of hypersurfaces $H_1, \dots, H_r$ of codimension $r$ in $X = \P^n_k$. Then $U_i = X \setminus H_i$ are affine opens and because $Y = H_1 \cap \cdots \cap H_r$ set-theoretically we have $U_1 \cup \cdots \cup U_r = X \setminus Y$. Therefore, pulling back to $X \setminus Y$ the open cover $U_1, \dots, U_r$ is affine (because $X$ is separated) and therefore $\cd{X \setminus Y} \le r-1$.
\bigskip\\
Notice this argument works in the more general situation that $X$ is a quasi-projective scheme, $Y \subset X$ is a set-theoretic complete intersection $D_1 \cap \cdots \cap D_r$ for ample divisors $D_i \subset X$ then $\cd{X \setminus Y} \le r-1$. This is because $U_i = X \setminus D_i$ is an affine open and,
\[ U_1 \cup \cdots U_r = X \setminus (D_1 \cap \cdots D_r) = X \setminus Y \]
since $Y = D_1 \cap \cdots \cap D_r$ set-theoretically. Then $U_1, \dots, U_r$ forms an affine open cover of $X \setminus Y$ showing that $\cd{X \setminus Y} \le r-1$. 

\begin{rmk}
For a projective scheme $X$ the complement of an ample divisor $D$ is always affine. This is because we can find an embedding $X \embed \P^n$ such that $D = X \cap H$ set-theoretically and thus $X \setminus D = X \cap (\P^n \setminus H)$ is ample since $X \embed \P^n$ is affine. However, if $X$ is merely quasi-projective this may not be true because $j : X \embed \P^n$ may not be affine so the pullback $X \cap (\P^n \setminus H)$ need not be affine. This happens when the inclusion $j : X \embed \overline{X}$ into the projective closure is not an affine map. For example, let $X = \A^2 \setminus \{ (0,0) \}$. Then $\struct{X}$ is ample but the divisor $V(1 + x) = \A^2 \setminus \{ x = 1 \text{ or } (x,y) = (0,0) \}$ is not affine. This is because $j : X \to \overline{X} = \P^2$ is not affine. 
\end{rmk}
\end{enumerate}

\subsubsection{4.9}

Let $X = \Spec{k[x_1,x_2,x_3,x_4]}$ be affine four-space over $k$. Let $Y = Y_1 \cup Y_2$ where $Y_1 = V(x_1, x_2)$ and $Y_2 = V(x_3, x_4)$. If we suppose that $Y$ is a set theoretic complete intersection of dimension $2$ in $X$ then $\cd{X \setminus Y} \le 1$ by the extended version of Ex. III.4.8(e). Let $U = X \setminus Y$. To reach a contradiction we will show that $H^2(U, \struct{U}) \neq 0$.
\bigskip\\
Consider the cohomology with supports sequence,
\begin{center}
\begin{tikzcd}
H^2(X, \struct{X}) \arrow[r] & H^2(U, \struct{U}) \arrow[r] & H^3_Y(X, \struct{X}) \arrow[r] & H^3(X, \struct{X})
\end{tikzcd}
\end{center}
Since $H^q(X, \struct{X}) = 0$ for $q > 0$ there is an isomorphism $H^2(U, \struct{U}) \iso H^3_Y(X, \struct{X})$ so it suffices to show that $H^3_Y(X, \struct{X}) \neq 0$. Furthermore, by Mayer-Vietoris for cohomology with supports,
\begin{center}
\begin{tikzcd}
H^3_{Y_1}(X, \struct{X}) \oplus H^3_{Y_2}(X, \struct{X}) \arrow[r] & H^3_Y(X, \struct{X}) \arrow[r] & H^4_{Y_1 \cap Y_2}(X, \struct{X}) \arrow[r] & H^4_{Y_1}(X, \struct{X}) \oplus H^4_{Y_2}(X, \struct{X})
\end{tikzcd}
\end{center}
Furthermore, consider the cohomology with supports sequences,
\begin{center}
\begin{tikzcd}
H^q_{Y_i}(X, \struct{X}) \arrow[r] & H^q(X, \struct{X}) \arrow[r] & H^q(X \setminus Y_i, \struct{X}) \arrow[r] & H^{q+1}_{Y_i}(X, \struct{X}) \arrow[r] & H^{q+1}(X, \struct{X})
\end{tikzcd}
\end{center}
But $H^q(X, \struct{X}) = 0$ for $q > 0$ and $H^q(X \setminus Y_i, \struct{X}) = 0$ for $q > 1$ because $Y_i$ is the complete intersection of $V(x_1) \cap V(x_2)$ (or $V(x_3) \cap V(x_4)$) so $\cd{X \setminus Y_i} \le 1$. Therefore, $H^q_{Y_i}(X, \struct{X}) = 0$ for $q > 2$. Thus, returning to the Mayer-Vietoris sequence,
\begin{center}
\begin{tikzcd}
0 \arrow[r] & H^3_Y(X, \struct{X}) \arrow[r] & H^4_{Y_1 \cap Y_2}(X, \struct{X}) \arrow[r] & 0
\end{tikzcd}
\end{center}
gives an isomorphism $H^3_Y(X, \struct{X}) \iso H^4_{Y_1 \cap Y_2}(X, \struct{X})$ so it suffices to show that $H^4_{Y_1 \cap Y_2}(X, \struct{X}) \neq 0$. Applying the cohomology with supports in $P = Y_1 \cap Y_2$ sequence,
\begin{center}
\begin{tikzcd}
H^3(X, \struct{X}) \arrow[r] & H^3(X \setminus P, \struct{X}) \arrow[r] & H^4_P(X, \struct{X}) \arrow[r] & H^4(X, \struct{X}) 
\end{tikzcd}
\end{center}
using that $H^q(X, \struct{X}) = 0$ for $q > 0$ we get an isomorphism $H^3(X \setminus P, \struct{X}) \iso H^4_P(X, \struct{X})$ so, in total we have, 
\[ H^2(U, \struct{U}) \iso H^3_Y(X, \struct{X}) \iso H^4_{Y_1 \cap Y_2}(X, \struct{X}) \iso H^3(X \setminus P, \struct{X}) \]
and it suffices to show that $H^3(X \setminus P, \struct{X}) \neq 0$.
\bigskip\\
Now we take the cover $U_i = D(x_i)$ of $X \setminus P$ and consider the Cech complex begining in degree $3$,
\begin{center}
\begin{tikzcd}
\bigoplus\limits_{i = 1}^4 k[x_1,x_2,x_3,x_4]_{x_1 \cdots \hat{x_i} \cdots x_4} \arrow[r] & k[x_1^{\pm 1}, x_2^{\pm 1}, x_3^{\pm 1}, x_4^{\pm 1}] \arrow[r] & 0
\end{tikzcd}
\end{center}
where the map is the alternating sum. Notice that $x_1^{i_1} x_2^{i_2} x_3^{i_3} x_4^{i_4}$ cannot be in the image if all $i_j < 0$ because each term in the image comes from a ring with not every $x_i$ inverted. Therefore this is not surjective so $H^3(X \setminus P, \struct{X}) \neq 0$ proving that $H^2(U, \struct{U}) \neq 0$ so $Y$ cannot be a set-theoretic complete intersection. 

\subsubsection{4.10 (DO THE REVERSE DIRECTION)}

Let $X$ be a nonsingular variety over an algebraicalyl clsoed field $k$, and $\F$ be a coherent sheaf of $\struct{X}$-modules on $X$. By Ex. 8.7 we showed that if $X$ is affine, there is a unique extension of $X$ by $\F$ up to isomorphism. Suppose that $X_1'$ and $X_2'$ are extensions of $X$ by $\F$. Choose an open affine cover $\{ U_i \}$ of $X$. Then $X_1' \times_{X} U_i$ and $X_2' \times_{X} U_i$ are extensions of $X$ by $\F|_U$ and therefore must be isomorphic via some $\varphi_i :  X_1' \times_{X} U_i \to X_2' \times_{X} U_i$ over $U_i$ fixing $\F|_U$ as the sheaf of ideals. Then on the overlaps, $U_{ij}$ which are open because $Y$ is separated, we see that,
\[ \alpha_{ij} = \varphi_i^{-1} |_{U_{ij}} \circ \varphi_j|_{U_{ij}} \]
is an automorphism of $X_1' \times_{X} U_{ij}$. Furthermore, $(\alpha_{ij})$ defines a $1$-cocycle because on triple overlaps, 
\begin{align*}
\alpha_{jk} |_{U_{ijk}} \circ \alpha_{ik}^{-1} |_{U_{ijk}} \circ \alpha_{ij} |_{U_{ijk}} &= (\varphi_j^{-1} |_{U_{ijk}} \circ \varphi_k |_{U_{ijk}}) \circ (\varphi_k^{-1} |_{U_{ijk}} \circ \varphi_i |_{U_{ijk}}) \circ (\varphi_i^{-1}|_{U_{ijk}} \circ \varphi_j|_{U_{ijk}})
\\
& = \varphi_j^{-1} |_{U_{ijk}} \circ (\varphi_k |_{U_{ijk}} \circ \varphi_k^{-1} |_{U_{ijk}}) \circ (\varphi_i |_{U_{ijk}} \circ \varphi_i^{-1}|_{U_{ijk}}) \circ \varphi_j|_{U_{ijk}}
\\
& = \varphi_j^{-1} |_{U_{ijk}} \circ \varphi_j|_{U_{ijk}} = \id
\end{align*}
We can classify automorphisms of a fixed extension $X'$ of $X$ by $\F$ as follows when $X'$ is affine.
\begin{center}
\begin{tikzcd}
0 \arrow[r] & I \arrow[r] & A' \arrow[r] & A \arrow[r] & 0
\end{tikzcd}
\end{center}
If $\varphi : A' \to A'$ is a ring automorphism then $\varphi - \id : A' \to I'$ is a $k$-module map because $\varphi - \id$ projects to zero in $A$. Because $I^2 = 0$ it is easy to check that $\tilde{D} = \varphi - \id$ is a derivation. Moreover, any $k$-derivation kills $I'$ because $I' = IA'$ and $D(\iota a) = \iota D(a) \in I^2 A = 0$ so it factors through a $B$-derivation $D : A \to I$ (since $A'/I' = A$). Conversely, given a $k$-derivation $D : A \to I'$ we produce a $k$-map $\tilde{D} : A' \to A \xrightarrow{D} I' \to A'$ and a direct calculation shows that $\varphi = \id + \tilde{D}$ is a $B$-algebra automorphism over the identity on $A$ (because $D$ lands in $I' = \ker{(A' \to A)}$). Therefore,
\[ \mathrm{Aut}_k(A'/A) = \Der{k}{A}{I'} = \Hom{A}{\Omega_{A/k}}{I \otimes_k A} \]
Furthermore this is a ring map because,
\[ \id - \varphi \circ \varphi' = (\id - \varphi) + (\id - \varphi') - (\id - \varphi)(\id - \varphi') = (\id - \varphi) + (\id - \varphi') \]
since $(\id - \varphi)(\id - \varphi')$ has image inside $I^2 = 0$.
Therefore $c_{ij} = \alpha_{ij} - \id$ defines a 1-cocycle $(c_{ij})$ for the sheaf $\shHom{\struct{X}}{\Omega_{X/k}}{\F} = \T_{X/k} \otimes_{\struct{X}} \F$ and therefore an element,
\[ [c] \in H^1(X, \T_{X/k} \otimes_{\struct{X}} \F) \]
Conversely, given a class,
\[ [c] \in H^1(X, \T_{X/k} \otimes_{\struct{X}} \F) \]
represented by some cocycle $(c_{ij})$ we construct $X'$ by gluing the standard split extensions,
\begin{center}
\begin{tikzcd}
0 \arrow[r] & \F|_{U_i} \arrow[r] & \struct{U_i}' \arrow[r] & \struct{U_i} \arrow[r] & 0
\end{tikzcd}
\end{center}
via the gluing data of automorphisms on the overlaps given by $(c_{ij})$.
(FINISH THIS!!!)

\subsubsection{4.11}

Let $X$ be a topological space, $\F$ a sheaf of abelian groups on $X$, and $\U$ an open cover of $X$ such that on all finite intersections $V = U_{i_0} \cap \cdots \cap U_{i_p}$ the sheaf $\F|_V$ is acyclic i.e. $H^q(V, \F|_V) = 0$ for all $q > 0$. Now embedd $\F$ into an injective sheaf $\I$ and take its cokernel $\K$ to form a short exact sequence of sheaves on $X$,
\begin{center}
\begin{tikzcd}
0 \arrow[r] & \F \arrow[r] & \I \arrow[r] & \K \arrow[r] & 0
\end{tikzcd}
\end{center}
The long exact sequence of the cohomology of the left-exact functor $\Gamma(V, (-)|_V)$ gives,
\begin{center}
\begin{tikzcd}
0 \arrow[r] & H^0(V, \F|_V) \arrow[r] & H^0(V, \F_V) \arrow[r] & H^0(V, \K|_V) \arrow[r] & H^1(V, \F|_V) 
\end{tikzcd}
\end{center} 
However, by assumption, $H^1(V, \F|_V) = 0$ on each finite intersection $V = U_{i_0} \cap \cdots \cap U_{i_p}$. Therefore, there is an exact sequence of abelian groups,
\begin{center}
\begin{tikzcd}
0 \arrow[r] & \F(V) \arrow[r] & \I(V) \arrow[r] & \K(V) \arrow[r] & 0
\end{tikzcd}
\end{center} 
and thus taking products over possible intersections $V$ we find that the sequence of Cech complexes,
\begin{center}
\begin{tikzcd}
0 \arrow[r] & \check{C}^\bullet(\U, \F) \arrow[r] & \check{C}^\bullet(\U, \I) \arrow[r] & \check{C}^\bullet(\U, \K) \arrow[r] & 0
\end{tikzcd}
\end{center}
is exact. Taking the long exact sequence associated to this sequence of Cech complexes gives a long exact sequence of Cech cohomology. However, $\I$ is injective and thus flasque so the higher Cech and sheaf cohomology vanishes. Therefore, we have a morphism of exact sequences,
\begin{center}
\begin{tikzcd}
0 \arrow[r] & \check{H}^0(\U, \F) \arrow[r] \arrow[d] & \check{H}^0(\U, \F) \arrow[r] \arrow[d] & \check{H}^0(\U, \K) \arrow[r] \arrow[d] & \check{H}^1(\U, \F) \arrow[r] \arrow[d] & 0
\\
0 \arrow[r] & H^0(X, \F) \arrow[r] & H^0(X, \F) \arrow[r] & H^0(X, \K) \arrow[r] & H^1(X, \F) \arrow[r] & 0
\end{tikzcd}
\end{center}
However, for any abelian sheaf $\G$,
\[ \check{H}^0(\U, \G) = H^0(\check{C}^\bullet(\U, \G)) = \Gamma(X, \G) = H^0(\U, \G) \]
so the first three downwards maps are isomorphisms and thus, by the five lemma, the map,
\[ \check{H}^1(\U, \F) \xrightarrow{\sim} H^1(\U, \F) \]
is an isomorphism proving the theorem at $q = 1$. 
\bigskip\\
In proceed by induction on $q$, assume that the map,
\[ \check{H}^q(\U, \G) \xrightarrow{\sim} H^q(X, \G) \]
is an isomorphism for any abelian sheaf $\G$ satisfying $H^p(V, \G|_V) = 0$ for all $p \ge 1$ and finite intersection $V$. 
\bigskip\\
Now the long exact sequence of Cech cohomology gives an isomorphism,
\begin{center}
\begin{tikzcd}
0 \arrow[r] & \check{H}^q(\U, \K) \arrow[r,"\sim"] & \check{H}^{q+1}(\U, \F) \arrow[r] & 0
\end{tikzcd}
\end{center}
Furthermore, for $p \ge 1$ the long exact sequence of sheaf cohomology restricted to $V$ gives,
\begin{center}
\begin{tikzcd}
H^p(V, \F|_V) \arrow[r] & H^p(V, \I|_V) \arrow[r] & H^p(V, \K|_V) \arrow[r] & H^{p+1}(V, \F|_V)
\end{tikzcd}
\end{center}
By assumption, $H^p(V, \F|_V) = 0$ and since $\I|_V$ is injective $H^p(V, \I|_V) = 0$ and thus $H^p(V, \K|_V) = 0$ for all $p \ge 1$ and any finite intersection $V = U_{i_0} \cap \cdots \cap U_{i_p}$. Therefore, $\K$ is an abelian sheaf on $X$ satisfying the hypothesis. Thus the isomorphisms,
\begin{center}
\begin{tikzcd}
\check{H}^q(\U, \K) \arrow[d] \arrow[r, "\sim"] & \check{H}^{q+1}(\U, \F) \arrow[d]
\\
H^q(\U, \K) \arrow[r, "\sim"] & H^{q+1}(\U, \F) 
\end{tikzcd}
\end{center}
shift the isomorphism for $\K$ given by the induction hypothesis to an isomorphism for $\F$ in one degree higher,
\[ \check{H}^{q+1}(\U, \F) \xrightarrow{\sim} H^{q+1}(X, \F) \]
completing the proof by induction.
\bigskip\\
We can alternatively apply the Cech-to-derived spectral sequence,
\[ E_2^{p,q} = \check{H}^p(\U, \cH^q(\F)) \implies H^{p+q}(X, \F) \]
where $\cH^q(\F)$ is the presheaf $U \mapsto H^q(U, \F|_U)$. Since we assumed that $\F|_V$ is acyclic for each multiple intersection $V$ of the cover, we find that $\check{H}^p(\U, \cH^q(\F)) = 0$ for $q > 0$ and therefore the spectral sequence collapses at the $E_2$-page giving natural isomorphisms $H^n(X, \F) \iso \check{H}^n(\U, \F)$ since $\cH^0(\F) = \F$. 

\subsection{Section 5}

\subsubsection{5.1}

Let $X$ be a proper scheme over $k$ and $\F$ a coherent $\struct{X}$-module. Then we know that $H^q(X, \F)$ are finite-dimensional $k$-vectorspaces and vanish for sufficiently large $q$. Therefore, the Euler-characteristic,
\[ \chi(\F) = \sum_{i = 0}^\infty (-1)^i \dim_k H^i(X, \F) \]
is a well-defined integer $\chi(\F) \in \Z$. 
Consider an exact sequence of coherent sheaves,  
\begin{center}
\begin{tikzcd}
0 \arrow[r] & \F \arrow[r] & \G \arrow[r] & \H \arrow[r] & 0
\end{tikzcd}
\end{center}
from which there is a long exact sequence of cohomology,
\begin{center}
\begin{tikzcd}[column sep = small]
0 \arrow[r] & H^0(X, \F) \arrow[r] & H^0(X, \G) \arrow[r] & H^0(X, \H) \arrow[r] & H^1(X, \F) \arrow[r] & H^1(X, \G) \arrow[draw=none]{d}[name=Z, shape=coordinate]{} \arrow[r] & H^1(X, \H)
\arrow[dlllll,
rounded corners, crossing over,
to path={ -- ([xshift=2ex]\tikztostart.east)
|- (Z) [near end]\tikztonodes
-| ([xshift=-2ex]\tikztotarget.west)
-- (\tikztotarget)}]
\\ 
& H^2(X, \F) \arrow[r] & H^2(X, \G)  \arrow[r] & H^2(X, \H) \arrow[r] & \cdots \arrow[r] & H^n(X, \H) \arrow[r] & 0
\end{tikzcd}
\end{center}
where the cohomology vanishes above the dimension of the scheme $X$. These groups are $k$-vectorspaces because $X$ is a scheme over $k$. By Lemma \ref{alt_sum_exact} we have the alternating sum,
\begin{align*}
\sum_{i = 0}^n (-1)^n \left[ \dim_k H^i(X, \F) - \dim_k H^i(X, \G) + \dim_k H^i(X, \H) \right] = \chi(\F) - \chi(\G) + \chi(\H) = 0  
\end{align*}
Therefore,
\[ \chi(\G) = \chi(\F) + \chi(\H) \]


\subsubsection{5.2}

\begin{enumerate}
\item Let $X$ be a projective scheme over $k$ and $\struct{X}(1)$ be a very ample invertible sheaf on $X$ over $k$. Let $\F$ be a coherent $\struct{X}$-module. We will prove that $P(n) = \chi(\F(n))$ is a rational polynomial by induction on $\dim{\Supp{\struct{X}}{\F}}$. First, notice that under the embedding $\iota : X \embed \P^r_k$ associated to $\struct{X}(1)$ we have,
\begin{align*}
H^q(X, \F(n)) & = H^q(X, \F \otimes \struct{X}(n)) = H^q(X, \F \otimes \iota^* \struct{\P}(n)) = H^q(\P^r_k, \iota_* (\F \otimes \iota^* \struct{\P}(n)) 
\\
& = H^q(\P^r_k, \iota_* \F \otimes \struct{\P}(n)) = H^q(\P^r_k, (\iota_* \F)(n))
\end{align*}
using the projection formula and thus $\chi(X, \F(n)) = \chi(\P^r_k, \iota_* \F(n))$ and $\iota_* \F$ is a coherent sheaf on $\P^r_k$ with the same support (under the embedding $\iota : X \embed \P^r_k$). Thus we reduce to the case of coherent sheaves on $X = \P^r_k$.
\bigskip\\
Consider the base case $\dim{\Supp{\struct{X}}{\F}} = 0$ then the support is a discrete set of points and thus $\F(n) \cong \F$ so $\chi(\F(n))$ is a constant integrer and thus $P_\F \in \Q[z]$. 
\bigskip\\
Now proceed by induction. We want to choose a section $\ell \in \Gamma(\P^r_k, \struct{\P}(1))$ such that $\F(-1) \xrightarrow{\cdot \ell} \F$ is injective. To check that $\F(-1) \to \F$ is injective it suffices to on the stalks at the associated points $x \in \Ass{\struct{X}}{\F}$ of which there are finitely many (since $\F$ is coherent and $\P^r_k$ is Noetherian). Thus we may choose such an $\ell \in \Gamma(\P^r_k, \struct{\P}(1))$ by ensuring that $\ell_x \notin \m_x$ for $x \in \Ass{\struct{X}}{\F}$ then $\F_x \to \F_x$ via multiplication by $\ell_x$ is an isomorphism because $\stalk{X}{x}$ is local and $\F_x \to \F_x$ becomes an isomorphism after tensoring by $\kappa(x)$ since the image $\ell(x) \in \kappa(x)$ is nonzero. Therefore, we get an exact sequence,
\begin{center}
\begin{tikzcd}
0 \arrow[r] & \F(-1) \arrow[r] & \F \arrow[r] & \F \otimes \struct{H} \arrow[r] & 0
\end{tikzcd}
\end{center}
where $H = V(\ell)$ is a hyperplane and $\coker{(\F(-1) \to \F)} = \F \otimes \struct{H}$ via right exactness of $\F \otimes -$. Notice, if we only ensured that $\ell$ not vanish at the generic points of the componetns of $\Supp{\struct{X}}{\F}$ then $\F(-1) \to \F$ would have a nonzero kernel but one with strictly smaller dimensional support. Indeed, let $\G = \F \otimes \struct{H}$, then from the previous calculation, we see that $\G_x = 0$ for $x \in \Ass{\struct{X}}{\F}$ and $\Supp{\struct{X}}{\G} \subset \Supp{\struct{X}}{\F}$ so we must have, 
\[ \dim \Supp{\struct{X}}{\G} \le \Supp{\struct{X}}{\F} - 1 \]
In fact, we have equality because $s|_Z$ is a regular section of $\struct{Z}(1)$ where $Z = \Supp{\struct{X}}{\F}$ and thus $Z \cap H \subset Z$ is Cartier so the equality follows from Krull. Anyway, from the exact sequence twisted by $\struct{\P}(n)$,
\[ \chi(\F(n)) - \chi(\F(n-1)) = \chi(\G(n)) \]
However, by the induction hypothesis $P_\G(n) = \chi(\G(n))$ for a polynomial $P_\G \in \Q[z]$ and therefore since $P_\F(n) - P_\F(n-1) = P_\G(n)$ is a polynomial it implies that $P_\F \in \Q[z]$ proving the claim by induction.

\item Let $S = k[x_0, \dots, x_r]$. Recall that for a graded $S$-module $M$ we define the Hilbert function $\varphi_M(n) = \dim_k M_n$ and the Hilbert polynomial $P_M \in \Q[z]$ is the unique polynomial agreeing with $\varphi_M$ for $n \gg 1$. Now let $M = \Gamma_*(\F)$ so $M_n = H^0(\P^r_k, \F(n))$. For $n \gg 0$ we know that $\chi(\F(n)) = H^0(\P^r_k, \F(n))$ by vanishing of cohomology. Therefore $P_\F(n) = \varphi_M(n)$ for $n \gg 0$ and $P_\F \in \Q[z]$ proving that $P_\F = P_M$ by uniqueness. 
\end{enumerate}

\subsubsection{5.3}

Let $X$ be a projective scheme of dimension $r$ over a field $k$. The \textit{arithmetic genus} of $X$ is defined by,
\[ p_a(X) = (-1)^r \left( \chi(\struct{X}) - 1 \right) \]
Note that being projective is equivalent to being quasi-projective and proper so $\chi$ is defined for any coherent $\struct{X}$-module so, in particular, for $\struct{X}$ itself. 
 
\begin{enumerate}
\item Let $X$ be a projective integral scheme over an algebraically closed field $k$. By Lemma \ref{projective_scheme_proper} the scheme $X$ is proper over $k$ so by Lemma \ref{global_sections_proper_scheme}, $\struct{X}(X) = H^0(X, \struct{X})$ is a finite and thus algebraic extension of $k$. Since $k$ is algebraically closed, $\struct{X}(X) = k$
 and thus \[ \dim_k H^0(X, \struct{X}) = 1 \]
Therefore,
\begin{align*}
p_a(X) & = (-1)^{r + 1} + (-1)^r \sum_{i = 0}^r (-1)^i \dim_k H^i(X, \struct{X})
\\
& = (-1)^{r + 1} + (-1)^r + (-1)^r \sum_{i = 1}^r (-1)^i \dim_k H^i(X, \struct{X})
\\
& = \sum_{i = 1}^r (-1)^{i + r} \dim_k H^i(X, \struct{X}) = \sum_{i = 0}^{r-1} (-1)^i \dim_k H^{r - i}(X, \struct{X})
\end{align*}
In particular, when $X$ is a projective curve,
\[ p_a(X) = \dim_k H^1(X, \struct{X}) \]

\item In section I, we defined $p_a(Y) := (-1)^r (P_Y(0) - 1)$ where $P_Y$ is the Hilbert polynomial of the embedding $\iota : Y \embed \P^N_k$. However, in the previous exercise we showed that $P_Y(n)$ agrees with $\chi(\struct{Y}(n))$ where $\struct{Y}(n) = \iota^* \struct{\P^N_k}(n)$ and therefore $P_Y(0) = \chi(\struct{Y})$ so the two definitions agree.

\item We want to show that $p_a$ is a birational invariant for nonsingular projective curves over an algebraically closed field $k$. This is simply because each birational class of curves has a single nonsingular projective model (MAYBE GIVE A BETTER PROOF?).
\bigskip\\
In particular, a degree $3$ plane curve has $p_a(X) = 1$ and thus cannot be birational to $\P^1$.
\end{enumerate}

\subsubsection{5.4}

Let $X$ be a projective scheme over a field $k$ and let $\struct{X}(1)$ be a very ample line bundle on $X$. Consider the map,
\[ P : K(X) \to \Q[z] \]
sending the class of the coherent sheaf $\F$ to its Hilbert polynomial: $[\F] \mapsto P_\F$ where $P_\F(n) := \chi(\F(n))$ is the Hilbert polynomial. This is well-defined because given an exact sequence,
\begin{center}
\begin{tikzcd}
0 \arrow[r] & \F_1 \arrow[r] & \F_2 \arrow[r] & \F_3 \arrow[r] & 0
\end{tikzcd}
\end{center}
of coherent sheaves, then $[\F_2] = [\F_1] + [\F_3]$ but we also know $P_{F_2} = P_{\F_1} + P_{\F_2}$ and therefore $P([\F_2]) = P([\F_1] + [\F_3])$. Furthermore, this map is unique for the condition that $P([\F]) = P_\F$ since $K(X)$ is generated by these classes.
\bigskip\\
Now let $X = \P^r_k$ and let $L_i \subset \P^r_k$ be a linear space of dimension $i$ for each $i = 0,1, \dots, r$. Then notice,
\[ \chi(\struct{L_i}(n)) = {n + i \choose i} = \tfrac{1}{i!} (n + i)(n + i - 1) \cdots (n + 1) \]
We want to show that,
\begin{enumerate}
\item $K(X)$ is free abelian generated by $[\struct{L_i}]$ for $i = 0,1,\dots,r$
\item the map $P : K(X) \to \Q[z]$ is injective.
\end{enumerate}
First notice that (a) $\implies$ (b) because the polynomials $P_{L_i}$ are $\Q$-linearly independent. To show this, suppose that,
\[ \sum_{i = 0}^r a_i P_{L_i} = 0 \]
Since the leading order term $n^r$ only appears in $P_{L_r}$ so we must have $a_r = 0$ and thus,
\[ \sum_{i = 0}^{r-1} a_i P_{L_i} = 0 \]
reducing to the $r - 1$ case proving the linear independence by induction. 
\bigskip\\
Now we prove (a) and (b) by induction on $r$. The caes $r = 0$ is trivial because the Grothendieck group of finite $k$-modules is clearly free abelian on one generator $[k]$. Now for $X = \P^{r+1}_k$ consider a hyperplane $H \subset X$ so $H \cong \P^r_k$ and we may take $L_r = H$. In fact, we may take a flag on linear spaces,
\[ L_0 \subsetneq L_1 \subsetneq \cdots \subsetneq L_r = H \subsetneq L_{r+1} = X \] 
so that $\struct{L_i}$ have support contained in $H$.
Let $U = X \setminus H \cong \A^{r+1}_k$. Now by Exercise (II.6.10c) there is an exact sequence,
\begin{center}
\begin{tikzcd}
K(H) \arrow[r] & K(X) \arrow[r] & K(U) \arrow[r] & 0
\end{tikzcd}
\end{center}
Where the map $K(H) \to K(X)$ sends $[\F] \mapsto [\iota_* \F]$. Notice that $P_{\iota_* \F}(n) = \chi(X, \iota_* \F(n)) = \chi(H, \F(n)) = P_\F(n)$ because $\iota^* \struct{X}(1) = \struct{H}(1)$ and $H^q(X, \iota_* \F) = H^q(H, \F)$ and using the projection formula. Therefore, there is a commutative diagram,
\begin{center}
\begin{tikzcd}
K(H) \arrow[rd, "P"] \arrow[r, "\iota_*"] & K(X) \arrow[d, "P"]
\\
& \Q[z]
\end{tikzcd}
\end{center}
However, by the induction hypothesis, $P : K(H) \to \Q(z)$ is injective and therefore $K(H) \to K(X)$ is injective. Furthermore, $K(U) \cong \Z \cdot [\struct{U}]$ because $U \cong \A^{r+1}_k$ and thus every finite module has a finite free resolution by Hilbert's theorem on syzygies\footnote{The fact that $U$ is regular and affine is not enough as this only shows there is a finite locally free resolution but we need aditionally that on affine space finite projective modules are free.} and thus $K(U)$ is generated by $[\struct{U}]$. Since $\Z$ is projective, the sequence splits giving,
\[ K(X) = K(H) \oplus K(U) \]
Furthermore, because we assumed the linear spaces $L_i$ form a flag inside $H$ for $i \le r$ we see that $K(H)$ is a free abelian group generated by $[\struct{L_i}]$ for $i = 0,1, \dots, r$ by the induction hypothesis. Additionally, the coherent sheaves $\struct{L_i}$ have support inside $H$ and thus map to zero under $K(X) \to K(U)$ whereas $[\struct{L_{r+1}}] = [\struct{X}] \mapsto [\struct{U}]$ which is the generator and therefore we can choose a section $K(U) \to K(X)$ via $[\struct{U}] \to [\struct{X}]$. Thus, from the splitting $K(X) = K(H) \oplus K(U)$ we see that $K(X)$ is a free $\Z$-module generated by $[\struct{L_i}]$ for $i = 0,1, \dots, r, r+1$ proving (a) and thus also (b) for $r + 1$ and thus for all $r$ by induction.

\subsubsection{5.5}

Let $X = \P^r_k$ and $Y \subset X$ be a closed subscheme of dimension $q \ge 1$ which is a complete intersection. We want to prove the following,
\begin{enumerate}
\item for all $n \in \Z$ the natural map,
\[ H^0(X, \struct{X}(n)) \onto H^0(Y, \struct{Y}(n)) \]
is surjective

\item $Y$ is connected

\item $H^i(Y, \struct{Y}(n)) = 0$ for $0 < i < q$ and all $n \in \Z$

\item $p_a(Y) = \dim_k H^q(Y, \struct{Y})$
\end{enumerate}
First (a) $\implies$ (b) because $H^0(X, \struct{X}) \onto H^0(Y, \struct{Y})$ is thus one dimensional so $Y$ is connected. Furthermore, (a) and (c) $\implies$ (d) because $\dim_k H^0(Y, \struct{Y}) = 1$ and $H^i(Y, \struct{Y}) = 0$ for $0 < i < q$ and therefore,
\[ p_a(Y) = (-1)^{q} (\chi(\struct{Y}) - 1) = \sum_{i = 1}^q (-1)^{q - i} \dim_k H^i(Y, \struct{Y}) = \dim_k H^q(Y, \struct{Y}) \]
Thus it suffices to prove (a) and (c).
\bigskip\\
We proceed by descending induction on $q$. For $q = r$ we consider the case $Y = X$ for which (a) is obvious and we know $H^i(X, \struct{X}) = 0$ for $0 < i < r$. Now assume (a) and (c) for dimension $q + 1$. Let $Y$ be a complete intersection of dimension $q$ then $Y$ is the intersection of a hypersurface of degree $d$ and a complete intersection $W$ of dimension $q + 1$. Therefore, $Y \subset W$ is a closed subscheme cut out by a section of $\struct{W}(d)$ so there is an exact sequence,
\begin{center}
\begin{tikzcd}
0 \arrow[r] & \struct{W}(n-d) \arrow[r] & \struct{W}(n) \arrow[r] & \struct{Y}(n) \arrow[r] & 0
\end{tikzcd}
\end{center}
Therefore, we get an exact sequence,
\begin{center}
\begin{tikzcd}
H^0(W, \struct{W}(n)) \arrow[r] & H^0(Y, \struct{Y}(n)) \arrow[r] & H^1(W, \struct{W}(n-d))
\end{tikzcd}
\end{center}
However, by assumption (c) of the induction hypothesis $H^1(W, \struct{W}(n-d)) = 0$ because $1 < q + 1$ so $H^0(W, \struct{W}(n)) \onto H^0(Y, \struct{Y}(n))$ is surjective. By assumption (a), the map $H^0(X, \struct{X}(n)) \onto H^0(W, \struct{W}(n))$ is surjective and therefore,
\[ H^0(X, \struct{X}(n)) \onto H^0(W, \struct{W}(n)) \onto H^0(Y, \struct{Y}(n)) \]
is surjective. Furthermore, the long exact sequence contains,
\begin{center}
\begin{tikzcd}
H^i(W, \struct{W}(n)) \arrow[r] & H^i(Y, \struct{Y}(n)) \arrow[r] & H^{i+1}(W, \struct{W}(n-d))
\end{tikzcd}
\end{center}
By assumption (c), when $i > 0$ and $i+1 < q+1$ we know that $H^i(W, \struct{W}(n)) = H^{i+1}(W, \struct{W}(n)) = 0$ and therefore $H^i(Y, \struct{Y}(n)) = 0$ for $0 < i < q$. This proves (a) and (c) by induction for all complete intersections of dimension $q \ge 1$.  

\subsubsection{5.6}

Let $Q$ be the nonsingular quadric surface $xy = zw$ in $X = \P^3_k$ over a field $k$. Since $\Pic{Q} = \Z \oplus \Z$ so effective Cartier divisors correspond to nonzero sections of $\struct{Q}(a,b)$ so divisors on $Q$ are bigraded in degree $(a,b)$.

\begin{enumerate}
\item Because the sheaves $\struct{Q}(q,0)$ and $\struct{Q}(0,q)$ correspond to $q$ disjoint lines $\P^1$ on $Q$ which are the two ruling of $Q$. We call these two curves $C_1, C_2 \subset Q$ to distiguish the two modules structues on $\struct{C_1} \cong \struct{C_2} \cong \struct{\P^1}$ \textit{as sheaves of rings} but not as $\struct{Q}$-modules. Then we get exact sequences,
\begin{center}
\begin{tikzcd}
0 \arrow[r] & \struct{Q}(-q,0) \arrow[r] & \struct{Q} \arrow[r] & \struct{C_1}^{\oplus q} \arrow[r] & 0
\\
0 \arrow[r] & \struct{Q}(0,-q) \arrow[r] & \struct{Q} \arrow[r] & \struct{C_2}^{\oplus q} \arrow[r] & 0
\end{tikzcd}
\end{center}
and twising by $\struct{Q}(a+q,b)$ and $\struct{Q}(a,b+q)$ respectively we get,
\begin{center}
\begin{tikzcd}
0 \arrow[r] & \struct{Q}(a,b) \arrow[r] & \struct{Q}(a+q,b) \arrow[r] & \struct{C_1}^{\oplus q}(b) \arrow[r] & 0
\\
0 \arrow[r] & \struct{Q}(a,b) \arrow[r] & \struct{Q}(a,b+q) \arrow[r] & \struct{C_2}^{\oplus q}(a) \arrow[r] & 0
\end{tikzcd}
\end{center}
where because $Q \cong C_1 \times C_2$ we see that,
\[ \struct{Q}(1,0) \otimes \struct{C_1} = \struct{C_1} \quad \text{and} \quad \struct{Q}(0,1) \ot \struct{C_1} = \struct{C_1}(1) \cong \struct{\P^1}(1) \]
and likewise for $C_2$. Therefore, we get long exact sequences,
\begin{center}
\begin{tikzcd}
H^0(\P^1, \struct{\P^1}(b))^{\oplus q} \arrow[r] & H^1(Q, \struct{Q}(a,b)) \arrow[r] & H^1(Q, \struct{Q}(a+q,b)) \arrow[r] & H^1(\P^1, \struct{\P^1}(b))^{\oplus q}
\\
H^0(\P^1, \struct{\P^1}(a))^{\oplus q} \arrow[r] & H^1(Q, \struct{Q}(a,b)) \arrow[r] & H^1(Q, \struct{Q}(a,b+q)) \arrow[r] & H^1(\P^1, \struct{\P^1}(a))^{\oplus q}
\end{tikzcd}
\end{center}
Furthermore, $\struct{Q}(n,n) = \struct{Q}(n)$ therefore because $Q$ is a quadric surface, there is an exact sequence,
\begin{center}
\begin{tikzcd}
0 \arrow[r] & \struct{\P^3}(n-2) \arrow[r] & \struct{\P^3}(n) \arrow[r] & \struct{Q}(n) \arrow[r] & 0
\end{tikzcd}
\end{center}
and therefore, 
\begin{center}
\begin{tikzcd}
H^1(\P^3, \struct{\P^3}(n)) \arrow[r] & H^1(Q, \struct{Q}(n)) \arrow[r] & H^{2}(\P^3, \struct{\P^3}(n-2)) 
\end{tikzcd}
\end{center}
and therefore $H^1(Q, \struct{Q}(n)) = 0$ for all $n$. Then applying the above sequences for $q = 1$ and $a = b$ we see that
\[ H^1(Q, \struct{Q}(n+1, n)) = H^1(Q, \struct{Q}(n,n+1)) = 0 \] because $H^1(Q, \struct{Q}(n,n)) = 0$ and $H^1(\P^1, \struct{\P^1}(1)) = 0$.
\bigskip\\
Now if $a,b < 0$ without loss of generality let $a < b$ and $q = b - a$ so we have 
\[ H^1(Q, \struct{Q}(a+q,b)) = H^1(Q, \struct{Q}(b,b)) = 0 \]
and $H^0(\P^1, \struct{\P^1}(b))^{\oplus q} = 0$ because $b < 0$ and therefore from the long exact sequence,
\[ H^1(Q, \struct{Q}(a,b)) = 0 \]
Now let $a \le -2$ and let $b = 0$ and $q = -a$ then from the first exact sequence we get,
\begin{center}
\begin{tikzcd}
H^0(Q, \struct{Q}) \arrow[r] & H^0(\P^1, \struct{\P^1})^{\oplus q} \arrow[r] & H^1(Q, \struct{Q}(a,0)) \arrow[r] & H^1(Q, \struct{Q}(0,0)) 
\end{tikzcd}
\end{center}
but $H^1(Q, \struct{Q}(0,0)) = 0$ and $\dim H^0(Q, \struct{Q}) = 1$ from the third exact sequence,
\begin{center}
\begin{tikzcd}
H^0(\P^3, \struct{\P^3}(-2)) \arrow[r] & H^0(\P^3, \struct{\P^3}) \arrow[r] & H^0(Q, \struct{Q}) \arrow[r] & H^1(\P^3, \struct{\P^3}(-2))
\end{tikzcd}
\end{center} 
and $H^i(\P^3, \struct{\P^3}(-2)) = 0$ and $\dim H^0(\P^3, \struct{\P^3}) = 1$ so we conclude (we could alternatively use use [Ex. 5.5(b)]). Furthermore, $\dim H^0(\P^1, \struct{\P^1})^{\oplus q} = q = -a$ and thus $\dim H^1(Q, \struct{Q}(a,0)) \ge -(a+1)$ so if $a \le -2$ we see that $\dim H^1(Q, \struct{Q}(a,0)) > 0$.

\item Now we consider the exact sequence for a type $(a,b)$ divisor $C$ (a locally principal subscheme),
\begin{center}
\begin{tikzcd}
0 \arrow[r] & \struct{Q}(-a,-b) \arrow[r] & \struct{Q} \arrow[r] & \struct{C} \arrow[r] & 0
\end{tikzcd}
\end{center} 
and therefore,
\begin{center}
\begin{tikzcd}
H^0(Q, \struct{Q}(-a,-b)) \arrow[r] & H^0(Q, \struct{Q}) \arrow[r] H^0(C, \struct{C}) \arrow[r] & H^1(Q, \struct{Q}(-a,-b)) 
\end{tikzcd}
\end{center}
Now if $a, b > 0$ then we have seen that $H^1(Q, \struct{Q}(-a,-b)) = 0$. Furthermore, $H^0(Q, \struct{Q}(-a,-b)) = 0$ and therefore $H^0(Q, \struct{Q}) \iso H^0(C, \struct{C})$ so $\dim H^0(C, \struct{C}) = 1$ and thus $C$ is connected. 
\bigskip\\
Now assume that $k$ is algebraically closed. Notice that $Q = \P^1 \times \P^1$. Using the $a$-uple and $b$-uple Veronese embeddins on the two factors we get $Q \embed \P^{n_1} \times \P^{n_2}$ such that $\struct{}(1,1)$ pulls back to $\struct{Q}(a,b)$. Then apply the Segre embedding we get $Q \embed \P^{n_1} \times \P^{n_2} \embed \P^{N}$ such that $\struct{\P^N}(1)$ pulls back to $\struct{Q}(a,b)$. Therefore divisors on $Q$ of type $(a, b)$ correspond to hyperplane sections of $Q \embed \P^N$. Then we apply Bertini's theorem to show that the general such hyperplane section is smooth and of dimension $1$. By the previous part, $C = Q \cap H$ is connected and smooth and thus irreducible so $C$ is a complete nonsingular irreducible curve of type $(a,b)$.
\bigskip\\
Let $C$ be a smooth divisor of type $(a,b)$ on $Q$. Without loss of generality assume that $a \ge b$. Because $C$ is smooth it is normal. Therefore by [Ex. II 5.14(d)] $C$ is projectively normal if and only if,
\[ \Gamma(\P^3, \P^3(n)) \to \Gamma(C, \struct{C}(n)) \]
is surjective for all $n \ge 0$. Because $C \embed Q \embed \P^3$ the map factors as,
\[ \Gamma(\P^3, \P^3(n)) \to \Gamma(Q, \struct{Q}(n)) \to \Gamma(C, \struct{C}(n)) \]
We compute,
\begin{center}
\begin{tikzcd}
0 \arrow[r] & H^0(Q, \struct{Q}(n-a,n-b)) \arrow[r] & H^0(Q, \struct{Q}(n)) \arrow[r] & H^0(C, \struct{C}(n)) \connectingmap{dll}
\\
& H^1(Q, \struct{C}(n-a,n-b)) \arrow[r] & H^1(Q, \struct{Q}(n)) \arrow[r] & \cdots
\end{tikzcd}
\end{center}
However, $H^1(Q, \struct{Q}(n)) = 0$ and therefore if we set $n = b$ then we see that,
\begin{center}
\begin{tikzcd}
H^0(Q, \struct{Q}(n)) \arrow[r] & H^0(C, \struct{C}(n)) \arrow[r] & H^1(Q, \struct{C}(b-a,0)) \arrow[r] & 0
\end{tikzcd}
\end{center}
is exact. Because for $a - b \ge 2$ we showed that $H^1(Q, \struct{C}(b-a,0)) \neq 0$ this means that $H^0(Q, \struct{Q}(b)) \to H^0(C, \struct{C}(b))$ cannot be surjective. Therefore, the composite map $\Gamma(\P^3, \struct{\P^3}(b)) \to \Gamma(C, \struct{C}(b))$ cannot be surjective. Therefore, if $|a - b| \ge 2$ then $C$ is not projectively normal. 
\bigskip\\
Conversely, suppose that $|a - b| \le 1$. First notice that,
\begin{center}
\begin{tikzcd}
H^0(\P^3, \struct{\P^3}(n)) \arrow[r] & H^0(Q, \struct{Q}(n)) \arrow[r] & H^1(\P^3, \struct{\P^3}(n-2))
\end{tikzcd}
\end{center}
is exact but $H^1(\P^3, \struct{\P^3}(n-2)) = 0$ so $\Gamma(\P^3, \struct{\P^3}(n)) \onto \Gamma(Q, \struct{Q}(n))$ is surjective for all $n \ge 0$ (or we can use [Ex. III 5.5] and that $Q$ is a complete intersection in $\P^3$). Therefore, it suffices to show that $\Gamma(Q, \struct{Q}(n)) \to \Gamma(C, \struct{C}(n))$ is surjective. However, from the exact sequence,
\begin{center}
\begin{tikzcd}
H^0(Q, \struct{Q}(n)) \arrow[r] & H^0(C, \struct{C}(n)) \arrow[r] & H^1(Q, \struct{C}(n-a,n-b)) \arrow[r] & 0
\end{tikzcd}
\end{center}
we see that $\Gamma(Q, \struct{Q}(n)) \onto \Gamma(C, \struct{C}(n))$ is surjective because $H^1(Q, \struct{C}(n-a,n-b)) = 0$ when $| (n - a) - (n - b)| = |a - b| \le 1$. Therefore, $C \embed \P^3$ is projectively normal for $|a - b| \le 1$.

\item Let $Y$ be a locally principal subscheme of type $(a,b)$. Then the sequence,
\begin{center}
\begin{tikzcd}
0 \arrow[r] & \struct{Q}(-a,-b) \arrow[r] & \struct{Q} \arrow[r] & \struct{C} \arrow[r] & 0
\end{tikzcd}
\end{center} 
shows that
\[ p_a(C) = 1 - \chi(\struct{C}) = 1 + \chi(\struct{Q}(-a,-b)) - \chi(\struct{Q}) \] 
Furthermore, the two exact sequences give,
\begin{align*}
\chi(\struct{Q}(a+q,b)) - \chi(\struct{Q}(a,b)) & = q \, \chi(\struct{\P^1}(b)) = q (b+1) 
\\
\chi(\struct{Q}(a,b+q)) - \chi(\struct{Q}(a,b)) & = q \, \chi(\struct{\P^1}(a)) = q (a+1) 
\end{align*}
for $q \ge 0$. Therefore,
\begin{align*}
\chi(\struct{Q}(0,-b)) - \chi(\struct{Q}(-a,-b)) & = a \, \chi(\struct{\P^1}(-b)) = a (-b+1) 
\\
\chi(\struct{Q}) - \chi(\struct{Q}(0,-b)) & = b \, \chi(\struct{\P^1}) = b 
\end{align*}
and thus,
\begin{align*}
\chi(\struct{Q}) - \chi(\struct{Q}(-a,-b)) & = \chi(\struct{Q}) - \chi(\struct{Q}(0,-b)) + \chi(\struct{Q}(0,-b)) - \chi(\struct{Q}(-a,-b))
\\
& = b + a(-b+1) = -ab + a + b
\end{align*}
Therefore,
\[ p_a(C) = 1 - [\chi(\struct{Q}) - \chi(\struct{Q}(-a,-b))] = ab - a - b + 1 = (a - 1)(b - 1) \]
\end{enumerate}

\subsubsection{5.7 (CHECK!!!)}

Let $X, Y, Z$ be proper schemes over a noetherian ring $A$ and $\L$ and invertible sheaf.

\begin{enumerate}
\item If $\L$ is ample on $X$ and $\iota : Z \embed X$ is a closed embedding then consider $\iota^* \L$. For any coherent $\struct{Z}$-module $\F$ consider $\F \otimes \iota^* \L^{\otimes n}$. We know that,
\[ H^0(Z, \F \otimes \iota^* \L^{\otimes n}) = H^0(X, \iota_* (\F \otimes \iota^* L^{\otimes n})) \]
but by the projection formula,
\[ \iota_* (\F \otimes \iota^* L^{\otimes n}) = \iota_* \F \otimes \L^{\otimes n} \]
which is generated by global sections for $n \gg 0$ because $\iota_* \F$ is coherent and $\L$ is ample. Therefore, we get a surjection,
\[ \bigoplus_{i \in I} \struct{X} \onto \iota_* \F \otimes \L^{\otimes n} \]
and pulling back gives a surjection,
\[ \bigoplus_{i \in I} \struct{Z} \onto \F \otimes \iota^* \L^{\otimes n} \]
so $\F \otimes \iota^* \L^{\otimes n}$ is globally generated for $n \gg 0$ and thus $\iota^* \L$ is ample.

\item If $\L$ is ample on $X$ then $\L \otimes \struct{X_{\red}}$ is ample on $X_{\red}$ by (a) using the closed immersion $X_{\red} \embed X$. Conversely suppose that $\L \otimes \struct{X_{\red}}$ is ample on $X_{\red}$. To show that $\L$ is ample, it suffices to show that for each coherent sheaf $\F$ there exists a constant $n_\F$ such that for all $n \ge n_\F$ and $q > 0$ that $H^q(X, \F \otimes \L^{\otimes n}) = 0$. Consider the filtration,
\[ \F \supset \sN \cdot \F \supset \sN^2 \cdot \F \supset \cdots \supset \sN^n \cdot \F \supset \sN^{n+1} \cdot \F = 0 \]
let $\F_i = \sN^i \cdot \F$ then $\G_i = \F_i / \F_{i+1}$ satisfies $\sN \cdot \G_i = 0$. Since $\iota : X_\red \to X$ is a closed immersion $\iota_*$ induces an equivalence of categories between quasi-coherent $\struct{X_\red}$-modules and quasi-coherent $\struct{X}$-modules killed by $\sN$. Thus $\G_i = \iota_* \G_i'$ where $\G_i'$ is a $\struct{X_\red}$-module. The twisted exact sequence,
\begin{center}
\begin{tikzcd}
0 \arrow[r] & \F_{i+1} \otimes \L^{\otimes n} \arrow[r] & \F_i \otimes \L^{\otimes n} \arrow[r] & \G_i \otimes \L^{\otimes n} \arrow[r] & 0
\end{tikzcd}
\end{center}
gives an exact sequence,
\begin{center}
\begin{tikzcd}
H^q(X, \F_{i+1} \otimes \L^{\otimes n}) \arrow[r] & H^q(X, \F_i \otimes \L^{\otimes n}) \arrow[r] & H^q(X, \G_i \otimes \L^{\otimes n})
\end{tikzcd}
\end{center}
Using the projection formula, $\G_i \otimes \L^{\otimes n} = \iota_* \G_i' \otimes \L^n = \iota_* (\G_i' \otimes (\iota^* \L)^{\otimes n})$ and thus,
\[ H^q(X, \G_i \otimes \L^{\otimes n}) = H^q(X_{\red}, \G_i' \otimes (\L \otimes \struct{X_\red})^{\otimes n}) \]
which vanishes for $q > 0$ and $n \ge n_{\G_i'}$. Because $\F_{n+1} = 0$ vanishing holds for $i = n+1$. Thus we proceed by descending induction by assuming that $H^q(X, \F_{i+1} \otimes \L^{\otimes n}) = 0$ for $q > 0$ and $n \ge n_{i+1}$. Then if $n \ge n_i = \max\{(n_i, n_{\G_i})\}$ and $q > 0$ we see that $H^q(X, \F_i \otimes \L^{\otimes n})$ from the exact sequence. Thus, by induction, vanishing holds for $\F = \F_0$ and $n \ge n_0$ meaning that $\L$ is ample on $X$.

\item If $\L$ is ample on $X$ then any irreducible component $Z \embed X$ is included via a closed immersion and thus $\L|_Z$ is ample on $Z$. 
\bigskip\\
Conversely, suppose that $X$ is reduced and $\L|_Z$ is ample for each irreducible component $Z \subset X$. Because $X$ is Noetherian, there are finitely many irreducible components $Z_i$. We proceed by induction on the number of irreducible components so assume the theorem for $r$ components and let $X$ have irreducible components $Z_1, \dots, Z_{r+1}$. 
If there is only one irreducible component then because $X$ is reduced $X = Z$ and thus the statement is trivial. Now proceed by induction. Take any coherent $\struct{X}$-module $\F$ and consider the exact sequence,
\begin{center}
\begin{tikzcd}
0 \arrow[r] & \I_{Z} \cdot \F \arrow[r] & \F \arrow[r] & \F / \I_{Z} \F \arrow[r] & 0
\end{tikzcd}
\end{center}
where $Z \subset X$ is an irreducible component. By Lemma \ref{support_component_sheaf_ideal}, 
\[ \Supp{\struct{X}}{\I_Z \otimes \F} \subset X' = Z_1 \cup \cdots \cup Z_r \]
where $Z_1, \dots, Z_r \subset X$ are the irreducible components besides $Z$ so $X'$ has $r$ components and $\I_{Z} \cdot \F$ is the pushforward of a $\struct{X'}$-module $\F'$ (possibly with nonreduced structure but ampleness is preserved under reduction). Likewise, $\G = \F / \I_Z \F$ is anhilated by $\I_Z$ and thus $\F / \I_Z \F = \iota_* \iota^* \G$. Twisting by $\L^{\otimes n}$ and applying the projection formula gives an exact sequence,
\begin{center}
\begin{tikzcd}
0 \arrow[r] & j_* (\F' \otimes \L^{\otimes n}|_{X'}) \arrow[r] & \F \otimes \L^{\otimes n} \arrow[r] & \iota_* (\G \otimes \L^{\otimes n}|_Z) \arrow[r] & 0
\end{tikzcd}
\end{center}
Then taking the cohomology sequence,
\begin{center}
\begin{tikzcd}
H^q(X', \F' \otimes \L|_{X'}^{\otimes n}) \arrow[r] & H^q(X, \F \otimes \L^{\otimes n}) \arrow[r] & H^q(Z, \G \otimes \L|_Z^{\otimes n}) 
\end{tikzcd}
\end{center}
By assumption, $\L|_Z$ is ample and $\L|_{X'}$ is ample when restricted to the $r$ irreducible components of $X'$ so (perhaps after reducing $X'$) by the induction hypothesis $\L|_{X'}$ is ample. Since $\F'$ and $\G$ are coherent there exist integers $n_0'$ and $n_Z$ such that for all $q > 0$,
\[ n \ge n_0' \implies H^q(X', \F' \otimes \L|_{X'}^{\otimes n}) = 0 \quad \text{ and } \quad n \ge n_Z \implies H^q(Z, \G \otimes \L|_{Z}^{\otimes n}) = 0 \]
Therefore, for $n \ge n_0 = \max\{n_0', n_Z\}$ and $q > 0$ the exact sequence gives that $H^q(X, \F \otimes \L^{\otimes n}) = 0$ proving that $\L$ is ample on $X$. Thus the result holds for any number of irreducible components by induction.

\item First, let $f : X \to Y$ be a finite morphism and $\L$ ample on $Y$. Then I claim that $f^* \L$ is ample on $X$. Let $\F$ be any coherent $\struct{X}$-module then by the projection formula $f_* (\F \otimes f^* \L^{\otimes n}) = f_* \F \otimes \L^{\otimes n}$. Furthermore, $f$ is affine so $f_*$ preserves cohomology showing that,
\[ H^q(X, \F \otimes f^* \L^{\otimes n}) = H^q(Y, f_*(\F \otimes f^* \L^{\otimes n})) = H^q(Y, f_* \F \otimes \L^{\otimes n}) \]
Because $\F$ is coherent and $f : X \to Y$ is proper then $f_* \F$ is coherent so there exists an integer $n_{f_* \F}$ such that for all $n \ge n_{f_* \F}$ and $q > 0$ we have,
\[ H^q(X, \F \otimes f^* \L^{\otimes n}) = H^q(Y, f_* \F \otimes \L^{\otimes n}) = 0 \]
and therefore $f^* \L$ is ample on $X$. 
\bigskip\\
Now suppose that $f : X \to Y$ is finite and surjective and $f^* \L$ is ample. We now will show that $\L$ is ample by by Noetherian induction on $Y$. By (b) and (c) $\L$ is ample iff $\L|_{Y_{\red}}$ is ample iff $\L|_{Z}$ is ample for each irreducible component $Z \subset Y_\red$. Let $\cP$ be the property of closed subsets $Z \subset Y$ that $\L|_Z$ is ample. Then if $Y$ has $\cP$ meaning $\L|_{Y_\red}$ is ample then $\L$ is ample proving the claim. Thus, towards Noetherian induction, it suffices to show that if $Z \subset Y$ is a closed subset such that every proper closed subset $C \subsetneq Z$ has $\cP$ then $Z$ has $\cP$. Notice if $Z$ is reducible this is automatic because $\L|_Z$ is ample iff $\L|_Z$ restricted to irreducible component is ample by (c) thus we need only consider the case that $Z$ is irreducible.
\bigskip\\
Base changing by $Z \embed Y$ we get a finite surjective map $X_Z \to Z$ where $X_Z \embed X$ is a closed immersion so $(f^* \L)|_{X_Z}$ is ample. Since $X_Z \to Z$ is surjective, some $\xi \in X_Z$ must hit the generic point $\eta \in Z$. Give $W = \overline{\{ \xi \}}$ the reduced subscheme structure then composing with the closed immersion $W \embed X_Z$ gives a finite map $f' : W \to Z$ which is dominant because $\xi \mapsto \eta$ and thus surjective since $f' : W \to Z$ is closed. Since $(f')^* \L = (f^* \L)|_W$ is ample using the closed immerison $W \embed X$ and both $W$ and $Z$ are integral we have reduced to the integral case.
\bigskip\\
We will show that $\L|_Z$ is ample by using Serre's criterion. For any coherent $\struct{Z}$-module $\F$, by Ex. III.4.2(b) there is a coherent $\struct{W}$-module $\G$ and a morphism $\beta : f_* \G \to \F^{\oplus r}$ which is an isomorphism at the generic point $\eta \in Z$. Extend to an exact sequence,
\begin{center}
\begin{tikzcd}
0 \arrow[r] & \ker{\beta} \arrow[r] & f_* \G \arrow[r, "\beta"] & \F^{\oplus r} \arrow[r] & \coker{\beta} \arrow[r] & 0
\end{tikzcd}
\end{center} 
Taking the stalk at $\eta$ gives an exact sequence,
\begin{center}
\begin{tikzcd}
0 \arrow[r] & (\ker{\beta})_\eta \arrow[r] & (f_* \G)_\eta \arrow[r, "\beta"] & \F^{\oplus r}_\eta \arrow[r] & (\coker{\beta})_\eta \arrow[r] & 0
\end{tikzcd}
\end{center} 
but $\beta$ is an isomorphism at $\eta$ so $(\ker{\beta})_\eta = (\coker{\beta})_\eta = 0$ and thus their supports are proper closed subsets $C_1$ and $C_2$ of $Z$. In particular, $\ker{\beta}$ and $\coker{\beta}$ are extensions of coherent sheaves on $C_1$ and $C_2$ (with possibly nonreduced structure) but by the induction hypothesis $\L|_{(C_i)_\red}$ is ample and thus $\L|_{C_i}$ is ample. Since $\ker{\beta}$ and $\coker{\beta}$ are coherent there exists $n_0'$ such that for $n \ge n_0'$ and $q > 0$,
\[ H^q(X, \ker{\beta} \otimes \L^{\otimes n}) = H^q(X, \iota_* \iota^* \ker{\beta} \otimes \L|_{C_1}^{\otimes n}) = H^q(C_1, \iota^* \ker{\beta} \otimes \L|_{C_1}^{\otimes n}) = 0 \]
and likewise $H^q(X, \coker{\beta} \otimes \L^{\otimes n}) = 0$. Now split the exact sequence into short exact sequences,
\begin{center}
\begin{tikzcd}
0 \arrow[r] & \ker{\beta} \arrow[r] & f_* \G \arrow[r] & \sC \arrow[r] & 0
\\
0 \arrow[r] & \sC \arrow[r] & \F^{\oplus r} \arrow[r] & \coker{\beta} \arrow[r] & 0
\end{tikzcd}
\end{center}
and consider the long exact sequences after twisting,
\begin{center}
\begin{tikzcd}[column sep = small]
H^q(Z, \ker{\beta} \otimes \L^{\otimes n}) \arrow[r] & H^q(Z, f_* \G \otimes \L^{\otimes n}) \arrow[r] & H^q(Z, \sC \otimes \L^{\otimes n}) \arrow[r] & H^{q+1}(Z, \ker{\beta} \otimes \L^{\otimes n})
\\
H^q(Z, \sC \otimes \L^{\otimes n}) \arrow[r] & H^q(Z, \F \otimes \L^{\otimes n})^{\oplus r} \arrow[r] & H^q(Z, \coker{\beta} \otimes \L^{\otimes n}) \arrow[r] & H^{q+1}(Z, \sC \otimes \L^{\otimes n})
\end{tikzcd}
\end{center}
giving $H^q(Z, f_* \G \otimes \L^{\otimes n}) \iso H^q(Z, \sC \otimes \L^{\otimes n})$ and $H^q(Z, \sC \otimes \L^{\otimes n}) \onto H^q(Z, \F \otimes \L^{\otimes n})^{\oplus r}$ for $q > 0$ and $n \ge n_0'$  by the vanishing of cohomology for $\ker{\beta}$ and $\coker{\beta}$. Furthermore, using that $f$ is affine and the projection formula,
\[ H^q(Z, f_* \G \otimes \L^{\otimes n}) = H^q(Z, f_* (\G \otimes f^* \L^{\otimes n})) = H^q(W, \G \otimes f^* \L^{\otimes n}) \]
By assumption, $f^* \L$ is ample so because $\G$ is coherent there exists an integer $n_1$ such that for $n \ge n_1$ and $q > 0$ we have $H^q(Z, \G \otimes f^* \L^{\otimes n}) = 0$. Thus, the exact sequence shows that $H^q(Z, \F \otimes \L^{\otimes n}) = 0$ for $q > 0$ and $n \ge n_0 = \max\{n_0', n_1\}$ proving that $\L$ is affine by Serre's criterion and thus showing that $Z$ satisfies $\cP$.
\end{enumerate}

\subsubsection{5.8 (CHECK!)}

We prove that one-dimensional proper schemes $X$ over an algebraically closed field $k$ are projective.

\begin{enumerate}
\item Let $X$ be irreducible and nonsingular. Then $X$ is a nonsingular complete curve over $k$ and thus projective by [II 6.7]. 

\item Let $X$ be integral and $\nu : \wt{X} \to X$ be its normalization. Since $\wt{X}$ is normal and integral and $\dim{X} = 1$ (because $\nu$ is finite it does not change dimension) then we see that $\struct{\wt{X}}{x}$ is a $1$-dimensional integrally closed domain and thus a DVR and thus regular so $X$ is regular. Therefore, by the previous part $\wt{X}$ is projective. Let $\L$ be a very ample invertible sheaf on $\wt{X}$. This defined a closed embedding $\iota : \wt{X} \embed \P^N$. Since $\wt{X} \to X$ is birational, there is some closed, and therefore finite, set $Z \subset \wt{X}$ such that $\nu : \wt{X} \setminus Z \to X \setminus \nu(Z)$ is an isomorphism (we can set $Z = \nu^{-1}(X^{\text{sing}})$ and then $X \setminus \nu(Z) = X^{\text{smooth}}$). Because $k$ is infinite, there exists some hyperplane $H \subset \P^N$ that avoids the finite set $Z \subset \P^N$. Let,
\[ D = \wt{X} \cap H = \sum P_i \]
be the corresponding divisor with $P_i \notin Z$. Therefore, $\L \cong \iota^* \struct{\P^H}(1) = \iota^* \struct{\P^N}(H) = \struct{\wt{X}}(D)$ and $f(P_i) \in X$ is nonsingular. Therefore, consider the divisor,
\[ D_0 = \sum f(P_i) \]
on $X$ which is Cartier because each $f(P_i)$ is nonsingular and thus $\stalk{X}{P_i}$ is a UFD. Therefore, $\L_0 \cong \struct{X}(D_0)$ is a line bundle and we have $\nu^* \L_0 = \struct{\wt{X}}(\nu^* D_0) = \struct{\wt{X}}(D) = \L$. Then by [Ex. III 5.7(d)] we see that $\L_0$ is ample because $\nu^* \L_0 = \L$ is very ample and in particular ample. Then by [II 7.6] we see that $\L_0^{\ot n}$ is very ample over $k$ for $n \gg 0$ and by [II 5.16.1] this means there is an immersion $\iota : X \to \P^N$ defined by $\L_0^{\ot n}$ but $X$ is proper so $\iota$ is closed and therefore a closed embedding proving that $X$ is proper.

\item Now let $X$ be reduced by not necessarily irreducible. Because $X$ is noetherian, it has a decomposition,
\[ X = X_1 \cup \cdots \cup X_r \]
into finite many irreducible components $X_i$. By the previous discussion each $X_i$ is projective and therefore admits a very ample invertible sheaf $\L_i$. Now consider the map,
\[ \Pic{X} \to \bigoplus_{i = 1}^r \Pic{X_i} \]
This map corresponds via the natural isomorphism $\Pic{X} \iso H^1(X, \struct{X})$ to the map,
\[ \struct{X} \tolabel{\varphi} \bigoplus_{i = 1}^r (\iota_i)_* \struct{X_i} \]
induced by the exact sequences,
\begin{center}
\begin{tikzcd}
0 \arrow[r] & \I_{X_i} \arrow[r] & \struct{X} \arrow[r] & (\iota_i)_* \struct{X_i} \arrow[r] & 0
\end{tikzcd}
\end{center}
The map $\varphi$ is injective because $X$ is reduced so $\bigcap_{i = 1}^r \I_{X_i} = 0$. Notice further that on $Z_i \setminus \bigcup_{j \neq i} Z_j$ that $\varphi$ is an isomorphism the union of these form a dense open $U$ (since it contains the generic point of each $Z_i$) with complement $Z$. Thus $\dim{Z} = 0$ because it does not contain any irreducible component and $\dim{X} = 1$. Therefore, we get an exact sequence of sheaves,
\begin{center}
\begin{tikzcd}
0 \arrow[r] & \struct{X}^\times \arrow[r, "\varphi^\times"] & \bigoplus_{i = 1}^r (\iota_i)_* \struct{X_i}^\times \arrow[r] & \sC \arrow[r] & 0 
\end{tikzcd}
\end{center}
where $\varphi$ and thus $\varphi^\times$ is an isomorphism on $U$ and thus $\sC$ is supported on $Z$. Then taking cohomology,
\begin{center}
\begin{tikzcd}
H^1(X, \struct{X}^\times) \arrow[r, "\varphi"] & \bigoplus_{i = 1}^r H^1(X_i, \struct{X_i}^\times) \arrow[r] & H^1(X, \sC)
\end{tikzcd}
\end{center}
but $\sC$ is supported in dimension $1$ and thus $H^1(X, \sC) = 0$ by Grothendieck's theorem proving that,
\[ \Pic{X} \onto \bigoplus_{i = 1}^r \Pic{X_i} \]
is surjective. Therefore, there exists an invertible module $\L$ on $X$ such that $\L \ot \struct{X_i} = \L|_{X_i} \cong \L_i$ which is ample and therefore by [Ex. III 5.7(c)] $\L$ is ample. Thus, as before, $\L^{\ot n}$ is very ample over $k$ for $n \gg 0$ and thus $X$ is projective.

\item Let $X$ be any projective one dimensional scheme over $k$. By the previous part, $X_{\red}$ is projective and therefore admits a very ample invertible sheaf $\L'$. Consider the map,
\[ \Pic{X} \to \Pic{X_{\red}} \]
By [Ex. III 4.6] with $\I = \sN$ the sheaf of nilpotents, there exists an exact sequence,
\begin{center}
\begin{tikzcd}
\cdots \arrow[r] & H^1(X, \sN) \arrow[r] & \Pic{X} \arrow[r] & \Pic{X_{\red}} \arrow[r] & H^2(X, \sN) 
\end{tikzcd}
\end{center}
however, because $X$ is a noetherian topological space of dimension $1$ by Grothendieck's theorem $H^2(X, \sN) = 0$ and thus $\Pic{X} \onto \Pic{X_{\red}}$ is surjective. Therefore, there exists an invertible module $\L$ on $X$ such that $\L \ot \struct{X_{\red}} = \L|_{X_{\red}} = \L'$ which is ample and therefore by [Ex. III 5.7(b)] $\L$ is ample. Thus, as before, $\L^{\ot n}$ is very ample over $k$ for $n \gg 0$ and thus $X$ is projective.
\end{enumerate}

\subsubsection{5.9}

Note: I have inverted Hartshorne's sign convention to fit more nicely with my choice of isomorphism in [Ex. III 4.6]. Therefore (depending on your sign choice in the isomorphism $\Omega \iso \omega \ot \T$ corresponding to which induced map you choose from the alternating pairing $\wedge : \Omega \times \Omega \to \omega$ since $\wedge f = -f \wedge$ there is a real choice here) the ``intendend'' result may be off by a sign but of course it is still nonzero regardless of any sign choices.
\bigskip\\
Let $k$ be an algebraically closed field of characteristic $0$ and let $X = \P^2_k$. Let $\omega_X$ be the dualizing sheaf i.e. the sheaf of $2$-forms. We define a cohmology class $\xi \in H^1(X, \omega \ot \T_X)$ which classifies an infinitessimal extension $X'$ of $X$ by $\omega$. To define $\xi$, let $U_i = D_+(x_i)$ be the standard open affine covering with respect to homogeneous coordinates $x_1, x_2, x_3 \in H^0(X, \struct{X}(1))$. Let,
\[ \xi_{ij} = (x_i/x_j) \, \d{(x_j / x_i)} = \d{\log{(x_j/x_i)}} \]
then $\xi = (\xi_{ij}) \in \check{C}^1(\U, \Omega^1)$ defines a cocycle because,
\begin{align*}
\xi_{23}|_{U_{123}} - \xi_{13}|_{U_{123}} + \xi_{12} |_{U_{123}} & = \d{\log{(x_3/x_2)}} - \d{\log{(x_3/x_1)}} + \d{\log{x_2/x_1)}} = \d{\log{(x_3 x_1 x_2 / x_2 x_3 x_1)}} 
\\
& \d{\log{1}} = 0
\end{align*}
We can also desribe this by noting that $\eta = (\eta_{ij})$ where $\eta_{ij} = x_j / x_i$ is a cocycle $\eta \in \check{C}^1(\U, \struct{X}^\times)$ representing $\struct{X}(1) \in \Pic{X} \cong H^1(X, \struct{X}^\times)$ and therefore $\xi = \d{\log{\eta}}$ corresponding to the image of $\eta$ under the induced map $H^1(X, \struct{X}^\times) \tolabel{\d{\log}} H^1(X, \Omega^1_X)$ and thus the image of $\struct{X}(1)$ under $\Pic{X} \to H^1(X, \Omega^1_X)$.
Then notice that $\omega_X \ot \T_X = \Omega^1_X$ so we indeed get a class $\xi \in H^1(X, \omega \ot \T_X$).
\bigskip\\
By the definition of the deformation $X'$ there is an exact sequence,
\begin{center}
\begin{tikzcd}
0 \arrow[r] & \omega_X \arrow[r] & \struct{X'} \arrow[r] & \struct{X} \arrow[r] & 0 
\end{tikzcd}
\end{center}
where $\omega_X$ is identified with an ideal of square zero. Therefore, by [Ex. II 4.6] there is an exact sequence,
\begin{center}
\begin{tikzcd}
0 \arrow[r] & \omega_X \arrow[r] & \struct{X'}^\times \arrow[r] & \struct{X}^\times \arrow[r] & 0
\end{tikzcd}
\end{center}
and therefore there is an exact sequence,
\begin{center}
\begin{tikzcd}
H^1(X, \omega_X) \arrow[r] & \Pic{X'} \arrow[r] & \Pic{X} \arrow[r, "\delta"] & H^2(X, \omega_X)
\end{tikzcd}
\end{center}
However, since $\omega_X \cong \struct{X}(-3)$ and $X = \P^2$ we see that $H^1(X, \omega_X) = H^1(\P^2, \struct{\P^2}(-3)) = 0$. Furthermore, $\Pic{X} = \Z$ generated by $\struct{X}(1)$ so we need to show that $\delta(\struct{X}(1)) \neq 0$. In this case, $\delta$ is injective and $H^1(X, \omega_X) = 0$ so we see that $\Pic{X'} = 0$. In particular, $X'$ cannot be projective because otherwise it would admit a very ample line bundle $\L$ but $\L \cong \struct{X'}$ and since $X'$ is proper we see that $\L$ cannot be globally generated so there cannot exist such a very ample $\L$.
\bigskip\\
We compute $\delta(\struct{X}(1))$ in terms of \v{C}ech comology. Consider the corresponding class $\eta \in H^1(X, \struct{X}^\times)$ then we compute $\delta(\eta)$ by locally lifting to an element $\tilde{\eta} \in \check{C}(\U, \struct{X'}^\times)$ and taking the boundary map to give the element $\delta(\eta) = \d{\tilde{\eta}} \in H^2(X, \omega_X)$. Affine locally,
\begin{center}
\begin{tikzcd}
0 \arrow[r] & \omega_X|_{U} \arrow[r] & \struct{X'}|_{U} \arrow[r] & \struct{X}|_{U} \arrow[r] & 0
\end{tikzcd}
\end{center}
is a split extension by [Ex. II.8.7] and therefore,
\begin{center}
\begin{tikzcd}
0 \arrow[r] & \omega_X|_{U_{i}} \arrow[r] & \struct{X'}|_{U_{i}} \arrow[r] & \struct{X}|_{U_{i}} \arrow[r] & 0
\\
0 \arrow[r] & \omega_X|_{U_{i}} \arrow[u, equals] \arrow[r] & \struct{U_{i}} \oplus \omega_X|_{U_{i}} \arrow[u, "\varphi_i"'] \arrow[r] & \struct{U_{i}} \arrow[u, equals] \arrow[r] & 0
\end{tikzcd}
\end{center}
where on the double overlaps these sequences are identified through the automorphisms of the split extension classified by $\xi$ meaning $\varphi_i^{-1} \circ \varphi_j = 1 + \xi_{ij} \wedge \d$ viewing $\xi_{ij} \wedge \d \in \Der{\struct{U_{ij}}}{\struct{U_{ij}}}{\omega_X}$.
Therefore, we can choose a lift, 
\begin{align*}
\tilde{\eta}_{12} & = \varphi_1(\eta_{12}, 0)
\\
\tilde{\eta}_{23} & = \varphi_2(\eta_{23}, 0)
\\
\tilde{\eta}_{13} & = \varphi_1(\eta_{13}, 0)
\end{align*}
note that the choice of $\varphi_i$ for $\eta_{ij}$ with $i < j$ rather than $\varphi_j$ is arbitrary as is the choice of $0$ in the second component. However, $\delta$ is well-defined so our result will be independent of these arbitrary choices. Now we compute,
\[ (\d{\tilde{\eta}})_{123} = \tilde{\eta}_{23}|_{U_{123}} - \tilde{\eta}_{13} |_{U_{123}} + \tilde{\eta}_{12}|_{U_{123}} = \varphi_2(x_3/x_2, 0) \cdot \varphi_1(x_3/x_1, 0)^{-1} \cdot \varphi_1(x_2/x_1, 0) \]
We consider this as an element of $\struct{U_{123}} \oplus \omega_X|_{U_{123}}$ by pulling back via $\varphi_1$ (again this is arbitrary) and since multiplication in a split extension is given by $(a,b) \cdot (a', b') = (aa', ba' + a b')$ we compute,
\[ \varphi_{1}^{-1}((\d{\tilde{\eta}})_{123}) = (x_3/x_2, \xi_{12} \wedge \d{(x_3/x_2)}) \cdot (x_1/x_3, 0) \cdot (x_2/x_1, 0) = (1, x_2/ x_3 \cdot \xi_{12} \wedge \d{(x_3/x_2)}) \]
Therefore, $\d{\eta}_{123}$ is in the image of $\omega_X|_{U_{123}} \to \struct{X'}|_{U_{123}}$ (as it must be) and is exactly the image of
\[ \delta(\eta)_{123} = x_2/x_3 \cdot \xi_{12} \wedge \d{(x_3/x_2)} = \frac{x_1}{x_3} \, \d{\left( \frac{x_2}{x_1} \right)} \wedge \d{\left( \frac{x_3}{x_2} \right)} = \frac{x_1^2}{x_2 x_3} \, \d{\left( \frac{x_2}{x_1} \right)} \wedge \d{\left( \frac{x_3}{x_1} \right)} \in \check{C}^2(\U, \omega_X) \]
This is the canonical generator of $H^2(X, \omega_X)$ and hence not zero (see [III 7.1.1]).
We can check explicitly that $\delta(\eta)$ is a cocycle and not a coboundary. Because the Cech complex vanishes in degrees $n > 2$ we see that $\d{\delta(\eta)} = 0$ automatically. We can write down an explicit isomorphism $\omega_X \to \struct{X}(-3)$ from the Euler sequence,
\begin{center}
\begin{tikzcd}
0 \arrow[r] & \Omega_X \arrow[r, "\varphi"] & \struct{X}(-1)^{\oplus 3} \arrow[r, "\psi"] & \struct{X} \arrow[r] & 0
\end{tikzcd}
\end{center}
by sending $\d{f_1} \wedge \d{f_2} \mapsto \varphi(\d{f_1}) \wedge \varphi(\d{f_2}) \wedge g$ where $\psi(g) = 1$. Explicitly we can take $g = e_3 x_3^{-1}$ and,
\[ \varphi(\d{(x_j/x_i)}) = (1/x_i)^2 (x_i e_j - x_j e_i) \]
Therefore we send,
\[ \eta(\eta) \mapsto \frac{x_1^2}{x_2 x_3 x_1^4 x_3} (x_1 e_2 - x_2 e_1) \wedge (x_1 e_3 - x_3 e_1) \wedge e_3 = \frac{1}{x_1 x_2 x_3} e_1 \wedge e_2 \wedge e_3 \] 
This is the canonical generator of $H^2(X, \struct{X}(-3))$ and thus is not a coboundary (see [II 5.1(c)]).

\subsubsection{5.10}

Let $X$ be a projective scheme over a noetherian ring $A$. First, notice that if $\F \onto \G$ is a surjection of coherent sheaves then we may extend to an exact sequence,
\begin{center}
\begin{tikzcd}
0 \arrow[r] & \K \arrow[r] & \F \arrow[r] & \G \arrow[r] & 0
\end{tikzcd}
\end{center} 
Twisting by $\struct{X}(n)$ and taking the long exact sequence gives,
\begin{center}
\begin{tikzcd}
0 \arrow[r] & \Gamma(X, \K(n)) \arrow[r] & \Gamma(X, \F(n)) \arrow[r] & \Gamma(X, \G(n)) \arrow[r] & H^1(X, \K(n))
\end{tikzcd}
\end{center}
Since $\K$ is coherent, there exists a $n_\K$ such that for all $n \ge n_\K$ we have $H^1(X, \K(n)) = 0$ and thus $\Gamma(X, \F(n)) \onto \Gamma(X, \G(n))$ is surjective.
\bigskip\\
Now, we will prove the proposition by induction on $r$. The cases $r = 0,1,2$ are trivial. Now suppose the result holds for $r$ and let 
\begin{center}
\begin{tikzcd}
\F_1 \arrow[r] & \F_2 \arrow[r] & \cdots \arrow[r] & \F_r \arrow[r] & \F_{r+1}
\end{tikzcd}
\end{center}
be an exact sequence of coherent sheaves on $X$.
Then we can split this into sequences,
\begin{center}
\begin{tikzcd}
\F_1 \arrow[r] & \F_2 \arrow[r] & \cdots \arrow[r] & \F_{r-1} \arrow[r] & \K_r \arrow[r] & 0
\\
0 \arrow[r] & \K \arrow[r] & \F_r \arrow[r] & \sC \arrow[r] & 0
\end{tikzcd}
\end{center}
for subsheaves $\K \subset \F_r$ and $\sC \subset \F_{r+1}$. By the induction hypothesis there is an integer $n_1$ such that for all $n \ge n_1$ we have,
\begin{center}
\begin{tikzcd}
\Gamma(X, \F_1(n)) \arrow[r] & \Gamma(X, \F_2(n)) \arrow[r] & \cdots \arrow[r] & \Gamma(X, \F_{r-1}(n)) \arrow[r] & \Gamma(X, \K(n)) 
\end{tikzcd}
\end{center}
and from the long exact sequence of the twist of the second short exact sequence,
\begin{center}
\begin{tikzcd}
0 \arrow[r] & \Gamma(X, \K(n)) \arrow[r] & \Gamma(X, \F_r(n)) \arrow[r] & \Gamma(X, \sC(n)) \arrow[r] & H^1(X, \K(n))
\end{tikzcd}
\end{center}
and because $\K$ is coherent for $n \ge n_2$ we have $H^1(X, \K(n)) = 0$ and thus the sequence
\begin{center}
\begin{tikzcd}
0 \arrow[r] & \Gamma(X, \K(n)) \arrow[r] & \Gamma(X, \F_r(n)) \arrow[r] & \Gamma(X, \sC(n)) \arrow[r] & 0
\end{tikzcd}
\end{center}
is exact. 
Furthermore, for $n \ge n_3$ we know that $\Gamma(X, \F_{r-1}(n)) \onto \Gamma(X, \K(n))$ is surjecitve. Lastly, $\Gamma(X, \sC(n)) \embed \Gamma(X, \F_{r+1}(n))$ is injective because $\Gamma$ is right exact. Thus, for $n \ge n_0 = \max{(n_1, n_2, n_3)}$, we can patch these together to get a long exact sequence
\begin{center}
\begin{tikzcd}
\Gamma(X, \F_1(n)) \arrow[r] & \Gamma(X, \F_2(n)) \arrow[r] & \cdots \arrow[r] & \Gamma(X, \F_r(n)) \arrow[r] & \Gamma(X, \F_{r+1}(n))
\end{tikzcd}
\end{center}
proving the claim by induction.

\subsection{Section 6}

\subsubsection{6.1}

Let $(X, \struct{X})$ be a ringed space and $\F_1, \F_2$ be $\struct{X}$-modules. Given an exension,
\begin{center}
\begin{tikzcd}
0 \arrow[r] & \F_1 \arrow[r] & \F \arrow[r] & \F_2 \arrow[r] & 0
\end{tikzcd}
\end{center}
then the long exact sequence for $\Hom{\struct{X}}{\F_2}{-}$ gives an exact sequence,
\begin{center}
\begin{tikzcd}
\Hom{\struct{X}}{\F_2}{\F} \arrow[r] & \Hom{\struct{X}}{\F_2}{\F_2} \arrow[r, "\delta"] & \Ext{1}{\struct{X}}{\F_2}{\F_1} 
\end{tikzcd}
\end{center}
so we get a class $\delta(\id) \in \Ext{1}{\struct{X}}{\F_2}{\F_1}$. Furthermore, given a isomorphism of extensions,
\begin{center}
\begin{tikzcd}
0 \arrow[r] & \F_1 \arrow[d, equals] \arrow[r] & \F \arrow[d, "\sim"] \arrow[r] & \F_2 \arrow[r] \arrow[d, equals] & 0
\\
0 \arrow[r] & \F_1 \arrow[r] & \F' \arrow[r] & \F_2 \arrow[r] & 0
\end{tikzcd}
\end{center}
we get a morphism of exact sequences,
\begin{center}
\begin{tikzcd}
\Hom{\struct{X}}{\F_2}{\F_2} \arrow[r, "\delta"] \arrow[d, equals] & \Ext{1}{\struct{X}}{\F_2}{\F_1} \arrow[d, equals]
\\
\Hom{\struct{X}}{\F_2}{\F_2} \arrow[r, "\delta'"] & \Ext{1}{\struct{X}}{\F_2}{\F_1} 
\end{tikzcd}
\end{center}
and thus $\delta'(\id) = \delta(\id)$ so the ext class only depends on the isomorphism class of the extension. Suppose that the class $\delta(\id) = 0$ then by exactness $\id$ is in the image of,
\[ \Hom{\struct{X}}{\F_2}{\F} \to \Hom{\struct{X}}{\F_2}{\F_2} \]
so there exists a map $\pi : \F_2 \to \F$ such that $\F_2 \to \F \to \F_2$ is the identity meaning the sequence,
\begin{center}
\begin{tikzcd}
0 \arrow[r] & \F_1 \arrow[r] & \F \arrow[r] & \F_2 \arrow[r] \arrow[l, bend right, "\pi"'] & 0
\end{tikzcd}
\end{center}
is split on the right and thus the extension is trivial and $\F = \F_1 \oplus \F_2$.
\bigskip\\
Now, suppose that $\xi \in \Ext{1}{\struct{X}}{\F_2}{\F_1}$ is a class. Choose an embedding $\F_1 \embed \I$ into an injective $\struct{X}$-module $\I$ and consider the exact sequence,
\begin{center}
\begin{tikzcd}
0 \arrow[r] & \F_1 \arrow[r] & \I \arrow[r] & \K \arrow[r] & 0
\end{tikzcd}
\end{center}
where $\K$ is the cokernel. Then consider the long exact sequence for the functor $\Hom{\struct{X}}{\F_2}{-}$,
\begin{center}
\begin{tikzcd}
\Hom{\struct{X}}{\F_2}{\I} \arrow[r] & \Hom{\struct{X}}{\F_2}{\K} \arrow[r] & \Ext{1}{\struct{X}}{\F_2}{\F_1} \arrow[r] & \Ext{1}{\struct{X}}{\F_2}{\I}
\end{tikzcd}
\end{center}
Because $\I$ is injective, $\Ext{1}{\struct{X}}{\F_2}{\I} = 0$ so we see that,
\[ \Ext{1}{\struct{X}}{\F_2}{\F_1} = \coker{(\Hom{\struct{X}}{\F_2}{\I} \to \Hom{\struct{X}}{\F_2}{\K})} \]
Therefore we get a morphism $\varphi : \F_2 \to \K$ canonical up to a map $\F_2 \to \I$. Then we define a sheaf via the pullback,
\begin{center}
\begin{tikzcd}
& \F \pullback \arrow[r] \arrow[d] & \F_2 \arrow[d]
\\
\F_1 \arrow[r] \arrow[ru, dashed] & \I \arrow[r] & \K 
\end{tikzcd}
\end{center}
giving a map $\F_1 \to \F$ where the map $\F_1 \to \F_2$ is zero making the diagram commute because $\F_1 \to \I \to \K$ is zero. Then we get a morphism of exact sequences,
\begin{center}
\begin{tikzcd}
0 \arrow[r] & \F_1 \arrow[d, equals] \arrow[r] & \F \arrow[d] \arrow[r] & \F_2 \arrow[d] \arrow[r] & 0
\\
0 \arrow[r] & \F_1 \arrow[r] & \I \arrow[r] & \K \arrow[r] & 0
\end{tikzcd}
\end{center}
giving a morphism of long exact sequences for $\Hom{\struct{X}}{\F_2}{-}$,
\begin{center}
\begin{tikzcd}
\Hom{\struct{X}}{\F_2}{\F} \arrow[r] \arrow[d] & \Hom{\struct{X}}{\F_2}{\F_2} \arrow[r, "\delta"] \arrow[d] & \Ext{1}{\struct{X}}{\F_2}{\F_1} \arrow[r] \arrow[d, equals] & \Ext{1}{\struct{X}}{\F_2}{\F} \arrow[d]
\\
\Hom{\struct{X}}{\F_2}{\I} \arrow[r] & \Hom{\struct{X}}{\F_2}{\K} \arrow[r] & \Ext{1}{\struct{X}}{\F_2}{\F_1} \arrow[r] & 0
\end{tikzcd}
\end{center}
where $\id \in \Hom{\struct{X}}{\F_2}{\F_2}$ maps to $\varphi \in \Hom{\struct{X}}{\F_2}{\K}$ and the map 
\[ \Hom{\struct{X}}{\F_2}{\K} \to \Ext{1}{\struct{X}}{\F_2}{\F_1} \]
sends $\varphi \mapsto \xi$ proving that $\delta(\id) = \xi$. 



\subsubsection{6.2 (CHECK!!)}

Let $X = \P^1_k$ where $k$ is an infinite field.


\begin{enumerate}
\item Suppose that $\sP$ is a projective $\struct{X}$-module with a surjective map $\P \onto \struct{X} \to 0$. For each $x \in X$, there is a surjection $\struct{X} \onto \iota_x \kappa(x)$ and therefore a surjection $\sP \onto \iota_* \kappa(x)$. Now, for any open $V \subset X$ containing $x$ we also have a surjection $j_! (\struct{X}|_V) \onto \iota_* \kappa(x)$. Therefore, we have a diagram,
\begin{center}
\begin{tikzcd}
\sP \arrow[d, two heads, dashed] \arrow[r, two heads] & \struct{X}  \arrow[d, two heads]
\\
j_!(\struct{X}|_V) \arrow[r, two heads] & \iota_* \kappa(x) 
\end{tikzcd}
\end{center}
Now suppose that $U \subset X$ is an open containing $x$ not contained in $V$. We get a diagram,
\begin{center}
\begin{tikzcd}
\sP(U) \arrow[d, dashed] \arrow[r] & \struct{X}(U) \arrow[d, two heads]
\\
j_!(\struct{X}|_V)(U) \arrow[r, two heads] & \kappa(x)
\end{tikzcd}
\end{center}
but $j_!(\struct{X}|_V)(U) = 0$ since $V \not\subset U$ and thus $\sP(U) \to \kappa(x)$ factors through zero. Thus every section of $\sP(U)$ vanishes at $x$. For any $U \subset X$ and $x \in U$ we can find an open $V$ and a point $x$ such that $x \in U \not\subset V$ by removing a hyperplane from $U$ not containing $x$, using that $k$ is infinite. Therefore every section in $\sP(U)$ vanishes at $x$ but $\sP \onto \struct{X}$ is surjective so $1$ must locally be in the image giving a contradiction.

\item Now suppose that $\sP$ is projective in $\Coh{X}$ or $\QCoh{X}$ and there is a surjection $\sP \onto \struct{X}$. For any invertible sheaf $\L$ and $x \in X$ there is a surjection $\L \onto \iota_x \kappa(x)$ giving a diagram,
\begin{center}
\begin{tikzcd}
\sP \arrow[d, two heads, dashed] \arrow[r, two heads] & \struct{X}  \arrow[d, two heads]
\\
\L \arrow[r, two heads] & \iota_* \kappa(x) 
\end{tikzcd}
\end{center}
Because $\struct{X}$ and $\L$ are coherent, we can find a coherent subsheaf $\sP' \subset \sP$ such that $\sP' \onto \struct{X}$ and $\sP' \onto \L$ are surjective. However, we can twist $\sP' \to \struct{X}$ by a sufficiently large $n$ such that,
\[ H^0(X, \sP'(n)) \onto H^0(X, \struct{X}(n)) \]
is surjective. Therefore, we get a diagram,
\begin{center}
\begin{tikzcd}
H^0(X, \sP'(n)) \arrow[d] \arrow[r, two heads] & H^0(X, \struct{X}(n)) \arrow[d]
\\
H^0(X, \L(n)) \arrow[r] & \kappa(x)
\end{tikzcd}
\end{center}
Since $\L$ is arbitrary, we can choose $\L = \struct{X}(-n-1)$ to get,
\begin{center}
\begin{tikzcd}
H^0(X, \sP'(n)) \arrow[r, two heads] \arrow[d] & H^0(X, \struct{X}(n)) \arrow[d]
\\
H^0(X, \struct{X}(-1)) \arrow[r] & \kappa(x)
\end{tikzcd}
\end{center}
but $H^0(X, \struct{X}(-1)) = 0$ so $H^0(X, \sP'(n)) \onto H^0(X, \struct{X}(n)) \onto \kappa(x)$ factors through zero which is impossible because $\struct{X}(n)$ has nonvanishing sections at $x$.

\end{enumerate}

\subsubsection{6.3}

Let $X$ be a Noetherian scheme and $\F, \G$ be $\struct{X}$-modules. Let $\F$ be coherent. We want to show that,
\begin{enumerate}
\item $\G$ is quasi-coherent then $\shExt{i}{\struct{X}}{\F}{\G}$ is quasi-coherent
\item $\G$ is coherent then $\shExt{i}{\struct{X}}{\F}{\G}$ is coherent.
\end{enumerate}
\noindent
First, because $(-)|_U$ is exact and preserves injectives and,
\[ \shHom{\struct{X}}{\F}{\G}|_U = \shHom{\struct{U}}{\F|_U}{\G|_U} \]
we see that,
\[ \shExt{i}{\struct{X}}{\F}{\G}|_U = \shExt{i}{\struct{U}}{\F|_U}{\G|_U} \] 
For each point $x \in X$ we can find a neighbrohood $x \in U$ on which $\F$ is finitely presented and thus there exists a sequence
\begin{center}
\begin{tikzcd}
\struct{U}^{\oplus m} \arrow[r] & \struct{U}^{\oplus n} \arrow[r] & \F|_U \arrow[r] & 0
\end{tikzcd}
\end{center}
and therefore we get an exact sequence,
\begin{center}
\begin{tikzcd}
\shHom{\struct{X}}{\struct{X}^{\oplus m}}{\G} |_U \arrow[r] & \shHom{\struct{X}}{\struct{U}^{\oplus n}}{\G} |_U \arrow[r] & \shHom{\struct{X}}{\F}{\G}|_U \arrow[r] & 0
\end{tikzcd}
\end{center}
However, $\shHom{\struct{X}}{\struct{X}^{\oplus n}}{\G} = \G^{\oplus n}$ giving an exact sequence,
\begin{center}
\begin{tikzcd}
\G|_U^{\oplus m} \arrow[r] & \G|_U^{\oplus n} \arrow[r] & \shHom{\struct{X}}{\F}{\G}|_U \arrow[r] & 0
\end{tikzcd}
\end{center}
and therefore if $\G$ is (quasi)-coherent then so is $\shHom{\struct{X}}{\F}{\G}|_U $ because it is the cokernel of (quasi)-coherent $\struct{U}$-modules proving the base case, $\shExt{0}{\struct{X}}{\F}{\G} = \shHom{\struct{X}}{\F}{\G}$ because (quasi)-coherence is a local property. Now we proceed by induction using the local exact sequence,
\begin{center}
\begin{tikzcd}
0 \arrow[r] & \K \arrow[r] & \struct{U}^{\oplus n} \arrow[r] & \F|_U \arrow[r] & 0
\end{tikzcd}
\end{center}
where $\K$ ic coherent because it is the kernel of coherent modules. The long exact sequence gives,
\begin{center}
\begin{tikzcd}
\shExt{i}{\struct{U}}{\struct{U}^{\oplus n}}{\G|_U} \arrow[r] & \shExt{i}{\struct{U}}{\K}{\G|_U} \arrow[r] & \shExt{i+1}{\struct{U}}{\F|_U}{\G|_U} \arrow[r] & \shExt{i+1}{\struct{U}}{\struct{U}^{\oplus n}}{\G|_U} 
\end{tikzcd}
\end{center}
but for $i > 0$ we know $\shExt{i}{\struct{U}}{\struct{U}}{\G|_U} = 0$ and thus when $i > 0$,
\[ \shExt{i}{\struct{U}}{\K}{\G|_U} \iso \shExt{i+1}{\struct{X}}{\F}{\G}|_U \]
so by the induction hypothesis we conclude that $\shExt{i+1}{\struct{X}}{\F}{\G}|_U$ is (quasi)-coherent and therefore so is $\shExt{i+1}{\struct{X}}{\F}{\G}$. Finally, for the case $i = 1$ we use the exact sequence,
\begin{center}
\begin{tikzcd}
\G|_U^{\oplus n} \arrow[r] & \shHom{\struct{U}}{\K}{\G|_U} \arrow[r] & \shExt{1}{\struct{X}}{\F}{\G}|_U \arrow[r] & 0
\end{tikzcd}
\end{center}
so $\shExt{1}{\struct{X}}{\F}{\G}|_U$ is the cokernel of (quasi)-coherent sheaves and thus is (quasi)-coherent.

\subsubsection{6.4}

Let $X$ be a noetherian scheme and suppose that $\Coh{X}$ has enough locally frees (i.e. for each $\F \in \Coh{X}$ there exists a locally free $\G \in \Coh{X}$ and a surjection $\G \to \F$ making every coherent sheaf a quotient of a locally free). Then for any $\G \in \shMod{X}$, consider the contravariant $\delta$-functor $\shExt{i}{\struct{X}}{-}{\H} : \Coh{X}^{\op} \to \shMod{X}$. To show that such a functor is universal is suffices to prove that this contravariant $\delta$-functor is coeffaceable (or equivalently is an effaceable $\delta$-functor on the opposite category $\Coh{X}^{\op}$) meaning that for each $\F \in \Coh{X}$ there exists an epimorphism $a : \G \to \F$ such that $\shExt{i}{\struct{X}}{a}{\H} = 0$ for all $i \ge 1$. Since we are given such maps from locally free sheaves $\G \to \F$, it suffices to prove that $\shExt{i}{\struct{X}}{\G}{\H} = 0$ for all $i \ge 1$ and locally free $\G$. 
\bigskip\\
However, we have shown that for locally free coherent $\G$,
\[ \shExt{i}{\struct{X}}{\G}{\H} = \shExt{i}{\struct{X}}{\struct{X}}{\H} \otimes \G^\vee = 0 \]
for $i > 0$ since $\shExt{i}{\struct{X}}{\struct{X}}{\G} = 0$ because $\shHomover{\struct{X}}{\struct{X}}{-}$ is the identity functor. 

\subsubsection{6.5}

\newcommand{\hd}[1]{\mathrm{hd}\left( #1 \right)}
\newcommand{\pd}[2]{\mathrm{pd}_{#1}\left(#2 \right)}

Let $X$ be a noetherian scheme such that $\Coh{X}$ has enough injectives i.e. for any coherent $\struct{X}$-module $\F$ there exists a finite locally free sheaf $\E$ and a surjection $\E \onto \F$. Then we define the homological dimension $\hd{\F}$ to be the length of the shortest locally free resolution of $\F$. 

\begin{enumerate}
\item Suppose $\F$ is locally free then $\shExt{1}{\struct{X}}{\F}{\G} = \shExt{1}{\struct{X}}{\struct{X}}{\G} \otimes \F^\vee = 0$ for $\struct{X}$-module $\G$. Now suppose that $\F$ is coherent and $\shExt{1}{\struct{X}}{\F}{\G} = 0$ for any $\struct{X}$-module $\G$. Take an exact sequence,
\begin{center}
\begin{tikzcd}
0 \arrow[r] & \K \arrow[r] & \E \arrow[r] & \F \arrow[r] & 0
\end{tikzcd}
\end{center}
where $\E$ is finite locally free. Because $\shExt{1}{\struct{X}}{\F}{\K} = 0$, the sequence remains exact after applying $\shHom{\struct{X}}{\F}{-}$ giving an exact sequence,
\begin{center}
\begin{tikzcd}
0 \arrow[r] & \shHom{\struct{X}}{\F}{\K} \arrow[r] & \shHom{\struct{X}}{\F}{\E} \arrow[r] & \shHom{\struct{X}}{\F}{\F} \arrow[r] & 0
\end{tikzcd}
\end{center}
Now $\id \in \Gamma(X, \Hom{\struct{X}}{\F}{\F})$ is a global section so locally at any $x \in X$ there must be an open neighborhood $U$ such that $\id$ is in the image of
\[ \Gamma(X, \shHom{\struct{X}}{\F}{\E}) \to \Gamma(X, \shHom{\struct{X}}{\F}{\F}) \]
In particular, there is a morphism $\varphi : \F|_U \to \E|_U$ such that $\F|_U \to \E|_U \to \F|_U$ is the identity meaning that,
\begin{center}
\begin{tikzcd}
0 \arrow[r] & \K|_U \arrow[r] & \E|_U \arrow[r] & \F|_U \arrow[r] & 0
\end{tikzcd}
\end{center}
is split. Therefore, $\F_x$ is is a finite projective $\stalk{X}{x}$-module and therefore free meaning that $\F$ is locally free because $\F$ is coherent and $X$ is noetherian.

\item If $\hd{\F} \le n$ then we can find a locally free resolution,
\begin{center}
\begin{tikzcd}
0 \arrow[r] & \E_n \arrow[r] & \cdots \arrow[r] & \E_0 \arrow[r] & \F \arrow[r] & 0
\end{tikzcd}
\end{center}
and we can use this resolution to compute ext,
\[ \shExt{i}{\struct{X}}{\F}{\G} = H^i(\shHom{\struct{X}}{\E_\bullet}{\G}) \]
which is therefore zero for $i > n$. Conversely, suppose that $\shExt{i}{\struct{X}}{\F}{\G} = 0$ for any $\struct{X}$-module $\G$ and all $i > n$. If $n = 0$ then $\shExt{1}{\struct{X}}{\F}{\G} = 0$ and thus $\F$ is locally free so $\hd{\F} = 0$. Now we proceed by induction on $n$. Consider a short exact sequence,
\begin{center}
\begin{tikzcd}
0 \arrow[r] & \K \arrow[r] & \E \arrow[r] & \F \arrow[r] & 0
\end{tikzcd}
\end{center}
where $\E$ is finite locally free and thus the kernel $\K$ is coherent. Applying the long exact sequence of $\shHom{\struct{X}}{-}{\G}$ we find,
\begin{center}
\begin{tikzcd}
\shExt{i}{\struct{X}}{\K}{\G} \arrow[r,"\sim"] & \shExt{i+1}{\struct{X}}{\F}{\G} 
\end{tikzcd}
\end{center}
for $i > 0$ because $\shExt{j}{\struct{X}}{\E}{\G} = 0$ since $\E$ is locally free. Therefore, for $i + 1 > n$ we see that,
\[ \shExt{i}{\struct{X}}{\K}{\G} = \shExt{i+1}{\struct{X}}{\F}{\G} = 0 \]
Therefore, by the induction hypothesis $\hd{\K} \le n - 1$ so $\K$ has a length $n-1$ locally free resolution,
\begin{center}
\begin{tikzcd}
0 \arrow[r] & \E_{n} \arrow[r] & \cdots \arrow[r] & \E_1 \arrow[r] & \K \arrow[r] & 0
\end{tikzcd}
\end{center}
and therefore,
\begin{center}
\begin{tikzcd}
0 \arrow[r] & \E_{n} \arrow[r] & \cdots \arrow[r] & \E_1 \arrow[r] & \E \arrow[r] & \F \arrow[r] & 0
\end{tikzcd}
\end{center}
is a length $n$ locally free resolution of $\F$ showing that $\hd{\F} \le n$ and proving the converse by induction.

\item Since the stalks of a locally free resolution of $\F$ gives a free resolution of $\F_x$ clearly,
\[ \hd{\F} \ge \sup_x \pd{\stalk{X}{x}}{\F_x} \]
Now suppose that $n \ge \pd{\stalk{X}{x}}{\F_x}$ for each $x \in X$. We know $\Ext{i}{\stalk{X}{x}}{\F_x}{B} = 0$ for all $i > n$. Furthermore, because $\F$ is coherent, $\shExt{i}{\struct{X}}{\F}{\G}_x = \Ext{i}{\stalk{X}{x}}{\F_x}{\G_x} = 0$ for $i > n$ and thus $\shExt{i}{\struct{X}}{\F}{\G} = 0$ showing that $n \ge \hd{\F}$. Therefore,
\[ \hd{\F} = \sup_x \pd{\stalk{X}{x}}{\F_x} \]
\end{enumerate}

\subsubsection{6.6}

Let $A$ be a regular local ring, and $M$ a finitely generated $A$-module. 

\begin{enumerate}
\item If $M$ is projective then clearly $\Ext{i}{A}{M}{A} = 0$ for all $i > 0$. Conversely, suppose that $\Ext{i}{A}{M}{A} = 0$ for all $i > 0$. Since $A$ is regular, $\pd{A}{M} \le \dim{A} = n$ is finite. We claim that $\Ext{i}{A}{M}{N} = 0$ for any finite $A$-module $N$. The base case, $\Ext{n+1}{A}{M}{N} = 0$ because $\pd{A}{M} \le n$. Now we proceed by descending induction. Consider an exact sequence,
\begin{center}
\begin{tikzcd}
0 \arrow[r] & K \arrow[r] & F \arrow[r] & N \arrow[r] & 0
\end{tikzcd}
\end{center}
where $F$ is finite free. Then the long exact sequence for $\Hom{A}{M}{-}$ gives,
\begin{center}
\begin{tikzcd}
\Ext{i}{A}{M}{F} \arrow[r] & \Ext{i}{A}{M}{N} \arrow[r] & \Ext{i+1}{A}{M}{K} \arrow[r] & \Ext{i+1}{A}{M}{F}
\end{tikzcd}
\end{center}
but $\Ext{i}{A}{M}{F} = 0$ for $i > 0$ by assumption. Therefore, we find $\Ext{i}{A}{M}{N} = \Ext{i+1}{A}{M}{K}$. For induction, we assume that $\Ext{i+1}{A}{M}{N} = 0$ for all finite $A$-module $N$. Since $F$ is finite and $A$ is noetherian, $K$ is finite and thus $\Ext{i}{A}{M}{N} = \Ext{i+1}{A}{M}{K} = 0$ for any finite $A$-module $N$. In particular, taking an exact sequence,
\begin{center}
\begin{tikzcd}
0 \arrow[r] & K \arrow[r] & F \arrow[r] & M \arrow[r] & 0
\end{tikzcd}
\end{center}
with $F$ finite free. Then taking $\Hom{A}{M}{-}$ we get,
\begin{center}
\begin{tikzcd}
0 \arrow[r] & \Hom{A}{M}{K} \arrow[r] & \Hom{A}{M}{F} \arrow[r] & \Hom{A}{M}{M} \arrow[r] & \Ext{1}{A}{M}{K} 
\end{tikzcd}
\end{center}
but $\Ext{1}{A}{M}{K} = 0$ because $K$ is finite so $\Hom{A}{M}{F} \to \Hom{A}{M}{M}$ is surjective. Therefore there is a map $M \to F$ such that $M \to F \to M$ is $\id$ meaning that the sequence,
\begin{center}
\begin{tikzcd}
0 \arrow[r] & K \arrow[r] & F \arrow[r] & M \arrow[l, bend right] \arrow[r] & 0
\end{tikzcd}
\end{center}
is split on the right. Therefore $M \oplus K = F$ so $M$ and $K$ are projective.

\item If $\pd{A}{M} \le n$ then clearly $\Ext{i}{A}{M}{A} = 0$ for $i > n$. Conversely suppose that $\Ext{i}{A}{M}{A} = 0$ for $i > n$. We prove by induction on $n$ that $\pd{A}{M} \le n$. We proved the base case $n = 0$ above. Now assume the theorem for $n$ and let $\Ext{i}{A}{M}{A} = 0$ for $i > n+1$. Take a sequence,
\begin{center}
\begin{tikzcd}
0 \arrow[r] & K \arrow[r] & F \arrow[r] & M \arrow[r] & 0
\end{tikzcd}
\end{center}
where $F$ is finite free. Applying $\Hom{A}{-}{A}$ we get,
\begin{center}
\begin{tikzcd}
\Ext{i}{A}{F}{A} \arrow[r] & \Ext{i}{A}{K}{A} \arrow[r] & \Ext{i+1}{A}{M}{A} \arrow[r] & \Ext{i+1}{A}{F}{A}
\end{tikzcd}
\end{center}
Since $F$ is free $\Ext{i}{A}{F}{A} = 0$ for all $i > 0$ and therefore we get $\Ext{i}{A}{K}{A} = \Ext{i+1}{A}{M}{A}$ for $i > 0$. Since $\Ext{i+1}{A}{M}{A} = 0$ for $i+1 > n+1$ we see that $\Ext{i}{A}{K}{A} = 0$ for $i > n$. Since $K$ is finite, by the induction hypothesis, $\pd{A}{K} \le n$ and thus $\pd{A}{M} \le n + 1$ by concatenating the resolution with the above sequence proving that $\pd{A}{M} \le n \iff \Ext{i}{A}{M}{A} = 0$ for $i > n$ by induction. 
\end{enumerate}

\subsubsection{6.7}

Let $X = \Spec{A}$ be a Noetherian affine scheme and $M, N$ be $A$-modules with $M$ finite. Choose a finite free resolution $F_\bullet \to M \to 0$. Then we get a free resolution, $\wt{F}_\bullet \to \wt{M} \to 0$ and then we can compute,
\begin{align*}
\Ext{i}{\struct{X}}{\wt{M}}{\wt{N}} & = H^i(\Hom{\struct{X}}{\wt{F}_\bullet}{\wt{N}} = H^i(\Hom{A}{F_\bullet}{N}) = \Ext{i}{A}{M}{N} 
\\
\shExt{i}{\struct{X}}{\wt{M}}{\wt{N}} & = H^i(\shHom{\struct{X}}{\wt{F}_\bullet}{\wt{N}} = H^i(\wt{\Hom{A}{F_\bullet}{N}}) = H^i(\Hom{A}{F_\bullet}{N})^\sim = \wt{\Ext{i}{A}{M}{N}} 
\end{align*}
because $\wt{(-)}$ is exact and thus commutes with taking cohomology. 

\subsubsection{6.8 (CHECK THIS ASK D ABOUT HIS SOLUTION!!)}

Let $X$ be a noetherian integral, separated, locally factiorial scheme.

\begin{enumerate}
\item We want to show that open sets of the form $X_s$ for $s \in \Gamma(X, \L)$ form a base for the topology on $X$. Let $x \in X$ and $U \subset X$ an open neighborhood of $x$. Then $X \setminus Z$ is closed so decompose $Z$ into irreducible components $Z = Z_1 \cup \cdots \cup Z_r$ which are also closed (there are finitely many because $X$ is noetherian). Suppose we can find $s_i \in \Gamma(X, \L_i)$ such that $x \in X_{s_i} \subset X \setminus Z_i$ then $s = s_1 \otimes \cdots \otimes s_r \in \Gamma(X, \L_1 \otimes \cdots \otimes \L_r)$ satisfies $x \in X_s \subset U$. Therefore, it suffices to assume that $Z$ is irreducible. Let $\zeta \in Z$ be the generic point.
\bigskip\\
Now we show that points of $X$ are separated by rational functions. Indeed, I claim that if $\stalk{X}{x} = \stalk{X}{y}$ inside the function field $K(X)$ then $x = y$. In fact, I will show that neither can dominate the other. Suppose that $\stalk{X}{x} \subset \stalk{X}{y}$ with $\m_y \cap \stalk{X}{x} = \m_x$ (we say $\stalk{X}{y}$ dominates $\stalk{X}{x}$) then there is a valuation ring $A$ dominating $\stalk{X}{y}$ inside $K(X)$ giving maps,
\begin{center}
\begin{tikzcd}
\Spec{K(X)} \arrow[d] \arrow[r] & X \arrow[d]
\\
\Spec{A} \arrow[ru, dashed] \arrow[r] & \Spec{\Z} 
\end{tikzcd}
\end{center}
via sending $\m_A \mapsto x$ and the local ring map $\stalk{X}{x} \embed A$ and by $\m_A \mapsto y$ and the local ring map $\stalk{X}{y} \embed A$. By the valuative criterion of separatedness, there is at most one such dotted map and thus $x = y$ and $\stalk{X}{x} = \stalk{X}{y}$.
\bigskip\\
Furthermore, I claim that if $\stalk{X}{x} \subset \stalk{X}{y}$ then $x,y$ are contained in the same affine open and $y \leadsto x$. Indeed, the prime ideal $\p = \m_y \cap \stalk{X}{x}$ corresponds to some point $y' \in \Spec{\stalk{X}{x}}$ which lies in every affine open containing $x$. Then $\stalk{X}{x} \subset \stalk{X}{y'} \subset \stalk{X}{y}$ because $\stalk{X}{y'} = (\stalk{X}{x})_\p$ which are units in $\stalk{X}{y}$. Then $\stalk{X}{y'} \embed \stalk{X}{y}$ is local so by our previous result $y = y'$.
\bigskip\\
Therefore, since $x \in X \setminus Z$ and thus $\zeta$ does not specialize to $x$ thus $\stalk{X}{x} \not\subset \stalk{X}{\zeta}$ so we can choose $f \in \stalk{X}{x}$ and $f \notin \stalk{X}{\zeta}$. Using the fact that $X$ is locally factorial, the Weil divisor $D = (f)_{\infty}$ is Cartier and thus $\L = \struct{X}(D)$ is a line bundle with a canonical regular section $s \in \Gamma(X, \L)$ such that $X_s = X \setminus D$. Since $f \in \stalk{X}{x}$ we know $x \in X \setminus D$ and $f$ has a pole at $\zeta$ so $\zeta \in D$ and therefore $Z \subset D$ because $D$ is closed. Thus $X \setminus D \subset X \setminus Z$ so we see that $x \in X_s \subset X \setminus Z$ proving the claim.

\item Let $\F$ be a coherent sheaf on $X$. Choose a finite (using $X$ is noetherian) affine open cover $U_i$ of $X$ such that $\F|_{U_i} \cong \wt{M_i}$. Then by the previous part for each $x \in U_i$ we can choose a line bundle $\L_x$ and a section $s_x \in \Gamma(X, \L_x)$ such that $x \in X_{s_x} \subset U_i$. Since $X$ is noetherian, $U_i$ is quasi-compact so we can choose a finite set $\L_{ij}$ and $s_{ij}$ such that $X_{s_{ij}}$ cover $U_i$. Let $t_{ik} \in M_i$ be a finite generating set. Since $X$ is noetherian we can find a finite affine open cover trivializing $\L$ such that the intersections are quasi-compact (in fact they are affine because $X$ is separated) we can apply [II 5.14(b)] to conclude that we can lift $ s_{ij}^{n_i} t_{ik}|_{X_{s_{ij}}} \in \Gamma(X_{s_{ij}}, \F \ot \L_i^{\ot n_i})$ (using finiteness of the generating set to choose large enough $n$ for all $t_{ik}$) to global sections $t_{ijk} \in \Gamma(X, \F \ot \L_i^{\otimes n_i})$. These sections give a map,
\[ \bigoplus_{i, j, k} \L_i^{\otimes -n_i} \to \F \]
which over $X_{s_{ij}}$ are given by sending $s_{ij}^{-n_i} \mapsto s_{ij}^{-n_i} t_{ijk}|_{X_{s_{ij}}} = t_{ik}|_{X_{s_{ij}}}$ which form a generating set for $\F|_{X_{s_{ij}}}$ and therefore this map is surjective proving the claim because the sums are finite and $\L_i$ is locally free (and $n_i$ are finite) so we see that $\F$ is the quotient of a locally free sheaf.
\end{enumerate}

\subsubsection{6.9 (ASK ABOUT THIS!!)}

Let $X$ be a noetherian, integral, separated, regular scheme. Let $K_1(X)$ be the group generated by locally free coherent $\struct{X}$-modules with relations $[\E] - [\E_1] - [\E_2]$ for each exact sequence,
\begin{center}
\begin{tikzcd}
0 \arrow[r] & \E_1 \arrow[r] & \E \arrow[r] & \E_2 \arrow[r] & 0
\end{tikzcd}
\end{center}
Consider the homomorphism $\varepsilon : K_1(X) \to K(X)$

\begin{enumerate}
\item Let $\F$ be a coherent $\struct{X}$-module. By the previous exercises, there exists a locally free (coherent) resolution $\E_\bullet \to \F \to 0$ so $\Coh{X}$ has enough locally frees. Furthermore, we know that,
\[ \hd{\F} = \sup_x \pd{\stalk{X}{x}}{\F_x} \le \dim{\stalk{X}{x}} = \dim{X} \]
because $X$ is regular. Therefore, $\F$ must have a locally free resolution,
\begin{center}
\begin{tikzcd}
0 \arrow[r] & \E_r \arrow[r] & \cdots \arrow[r] & \E_0 \arrow[r] & \F \arrow[r] & 0
\end{tikzcd}
\end{center} 
of length at most $n$. Therefore, in $K(X)$,
\[ [\F] = \sum_{i = 0}^r (-1)^i [\E_i] \]
and therefore $[\F]$ is in the image of $\varepsilon : K_1(X) \to K(X)$.

\item Furthermore, for a finite length locally free coherent resolution,
\begin{center}
\begin{tikzcd}
0 \arrow[r] & \E_r \arrow[r] & \cdots \arrow[r] & \E_0 \arrow[r] & \F \arrow[r] & 0
\end{tikzcd}
\end{center} 
we define a class,
\[ \delta([\F]) = \sum_{i = 0}^{r} (-1)^i [\F_i] \]
in $K_1(X)$. We need to show this class does not depend on the choice of resolution. (HOW TO SHOW INDEP OF REPRESENTATION!!)
\end{enumerate}

\subsubsection{6.10 CHECK}

\begin{enumerate}
\item Let $f : X \to Y$ be  a finite morphism of noetherian schemes. Let $\G$ be a quasi-coherent $\struct{Y}$-module. Then $f_* \struct{X}$ is coherent because $f$ is finite. Therefore, $\shHom{\struct{Y}}{f_* \struct{X}}{\G}$ is a quasi-coherent $f_* \struct{X}$-module so because $f : X \to Y$ is affine we get a quasi-coherent $\struct{X}$-module $f^! \G$.
(FINITENESS CONDITION?)

\item Let $\F$ be a coherent $\struct{X}$-module and $\G$ a quasi-coherent $\struct{Y}$-module. We define a morphism,
\[ \alpha : f_* \shHom{\struct{X}}{\F}{f^! \G} \to \shHom{\struct{Y}}{f_* \F}{\G} \]
as follows. First,
\[ f_* \shHom{\struct{X}}{\F}{f^! \G} \to \shHom{\struct{Y}}{f_* \F}{f_* f^! \G} \]
and then $f_* f^! \G = \shHom{\struct{Y}}{f_* \struct{X}}{\G}$ giving a map $f_* f^! \G \to \shHom{\struct{Y}}{\struct{Y}}{\G} = \G$ using $\struct{Y} \to f_* \struct{X}$. Thus we get a map,
\[ f_* \shHom{\struct{X}}{\F}{f^! \G} \to \shHom{\struct{Y}}{f_* \F}{f_* f^! \G} \to \shHom{\struct{Y}}{f_* \F}{\G} \]
Now affine locally on $V \subset Y$ with $V = \Spec{A}$ we have $f^{-1}(V) = U = \Spec{B}$ and $\F|_U = \wt{M}$ and $\G|_U = \wt{N}$. Then the map becomes,
\[ \Hom{B}{M}{\Hom{A}{B}{N}} \to \Hom{A}{M_A}{\Hom{A}{B}{N}} \to \Hom{A}{M_A}{N} \] 
which is an isomorphism.

\item There is a map $\Hom{\struct{X}}{\F}{-} \to \Hom{\struct{Y}}{f_* \F}{f_*-}$. Since $f_*$ is exact $\Ext{i}{\struct{Y}}{f_* \F}{f_* -}$ form a $\delta$-functor and therefore we get compatible natural transformations
\[ \varphi_i : \Ext{i}{\struct{X}}{\F}{-} \to \Ext{i}{\struct{Y}}{f_* \F}{f_* -} \]
because $\Ext{i}{\struct{X}}{\F}{-}$ are a universal $\delta$-functor. Therefore, we get,
\[ \varphi_i : \Ext{i}{\struct{X}}{\F}{f^! \G} \to \Ext{i}{\struct{Y}}{f_* \F}{f_* f^! \G} \to \Ext{i}{\struct{Y}}{f_* \F}{\G} \]

\item Now assume that $X$ and $Y$ are separated (I THINK WE ONLY NEED THIS BECAUSE AFFINE PUSHFORWARD IS ONLY PROVED IN THIS GENERALITY), $\Coh{X}$ has enough locally frees, and $f_* \struct{X}$ is finite locally free on $Y$ (i.e. $f$ is finite flat). Let $\F$ be a coherent $\struct{X}$-module and $\G$ a quasi-coherent $\struct{Y}$-module. For the case $i = 0$,
\[ \Hom{\struct{X}}{\F}{f^! \G} \to \Hom{\struct{Y}}{f_* \F}{\G} \]
is the global sections of the preceeding isomorphism
\[ \alpha : f_* \shHom{\struct{X}}{\F}{f^! \G} \to \shHom{\struct{Y}}{f_* \F}{\G} \]
and thus is an isomorphism (in general). Now for the case $\F = \struct{X}$ we consider the morphism,
\begin{center}
\begin{tikzcd}
\Ext{i}{\struct{X}}{\struct{X}}{f^! \G} \arrow[r, "\varphi_i"] \arrow[d, equals] & \Ext{i}{\struct{Y}}{f_* \struct{X}}{\G} \arrow[d, equals]
\\
H^i(X, f^! \G) \arrow[r] & H^i(Y, \shHom{\struct{Y}}{f_* \struct{X}}{\G})
\end{tikzcd}
\end{center}
using that $f_* \struct{X}$ is locally free. But $\shHom{\struct{X}}{f_* \struct{X}}{\G} = f_* f^! \G$ and therefore the bottom map is an isomorphism because $f$ is affine so $\varphi_i$ is an isomorphism for $\F = \struct{X}$. Likewise, when $\F = \E$ is finite locally free we get a diagram,
\begin{center}
\begin{tikzcd}
\Ext{i}{\struct{X}}{\E}{f^! \G} \arrow[r, "\varphi_i"] \arrow[d, equals] & \Ext{i}{\struct{Y}}{f_* \E}{\G} \arrow[d, equals]
\\
H^i(X, \shHom{\struct{X}}{\E}{f^! \G}) \arrow[r] & H^i(Y, \shHom{\struct{Y}}{f_* \E}{\G})
\end{tikzcd}
\end{center}
because $\E$ and $f_* \E$ are locally free so the local-to-global Ext spectral sequence collapses giving the vertical isomorphisms.
However, we have shown there is an isomorphism,
\[ f_* \shHom{\struct{X}}{\E}{f^! \G} \iso \shHom{\struct{Y}}{f_* \E}{\G} \] 
and therefore the bottom arrow is an isomorphism proving that $\varphi_i$ is an isomorphism for $\F = \E$ finite locally free.
\bigskip\\
Now we proceed by induction on $i$. We have proved the base case $i = 0$ so suppose that $\varphi_i$ is an isomorphism. Now we fit $\F$ into an exact sequence,
\begin{center}
\begin{tikzcd}
0 \arrow[r] & \K \arrow[r] & \E \arrow[r] & \F \arrow[r] & 0
\end{tikzcd}
\end{center}
where $\E$ is locally free. We get a morphism of exact sequences,
\begin{center}
\begin{tikzcd}[column sep = small]
\Ext{i}{\struct{X}}{\E}{f^! \G} \arrow[d, "\varphi_i"] \arrow[r] & \Ext{i}{\struct{X}}{\K}{f^! \G} \arrow[d, "\varphi_i"] \arrow[r] & \Ext{i+1}{\struct{X}}{\F}{f^! \G} \arrow[d, "\varphi_{i+1}"]  \arrow[r] & \Ext{i+1}{\struct{X}}{\E}{f^! \G} \arrow[d, "\varphi_{i+1}"] \arrow[r] & \Ext{i+1}{\struct{X}}{\E}{f^! \G} \arrow[d, "\varphi_{i+1}"] 
\\
\Ext{i}{\struct{Y}}{f_* \E}{\G} \arrow[r] & \Ext{i}{\struct{Y}}{f_* \K}{\G} \arrow[r] & \Ext{i+1}{\struct{Y}}{f_* \F}{\G} \arrow[r] & \Ext{i+1}{\struct{Y}}{f_* \E}{\G} \arrow[r] & \Ext{i+1}{\struct{Y}}{f_* \E}{\G}
\end{tikzcd}
\end{center}
where the bottom row is exact because $f_*$ is exact since it is affine. The second and fourth is an isomorphism because $\E$ is locally free. The first is an isomorphism by the induction hypothesis. Therefore, $\varphi_{i+1}$ is injective for $\F$ by the five lemma. Since $\F$ is a arbitrary coherent $\struct{X}$-module, it applies to $\K$ also making the rightmost arrow injective and thus $\varphi_{i+1}$ for $\F$ is an isomorphism again by the five lemma.
\end{enumerate}

\subsection{Section 7}

\subsubsection{7.1}

Let $X$ be an integral projective scheme of dimension $\dim{X} \ge 1$ over a field $k$. Let $\L$ be an ample invertible sheaf on $X$. Since $\L$ is ample, for sufficiently large $n$ the sheaf $\L^{\otimes n}$ is globally generated and thus,
\[ \dim_k H^0(X, \L^{\otimes n}) \ge \dim{X} + 1 \]
Suppose that $s \in H^0(X, \L^{\otimes -1})$ is a nonzero section. Then under the map,
\[ H^0(X, \L^{\otimes -n}) \otimes H^0(X, \L^{\otimes n}) \to H^0(X, \struct{X}) \]
we can multiply $s^{\otimes n}$ by a nonzero section to get a global section $s' \in H^0(X, \struct{X}) = k$ which is nowhere vanishing and thus $s$ is nowhere vanishing so $s : \struct{X} \to \L^{\otimes -1}$ is an isomorphism contradicting the ampleness of $\L$. Therefore, $H^0(X, \L^{\otimes -1}) = 0$.  

\subsubsection{7.2}

\renewcommand{\tr}{\mathrm{tr}}

Let $f : X \to Y$ be a finite morphism of projective schemes of the same dimension $n$. Let $\omega_Y^\circ$ be a dualizing sheaf for $Y$.

\begin{enumerate}
\item Let $\F$ be a coherent $\struct{X}$-module. From (Ex. 6.10) and the definition of $\omega_Y^\circ$ there are canonical isomorphisms,
\[ \Hom{\struct{X}}{\F}{f^! \omega_Y^\circ} \iso \Hom{\struct{Y}}{f_* \F}{\omega_Y^\circ} = H^n(Y, f_* \F)^\vee = H^n(X, \F)^\vee \]
using the fact that $f$ is affine and $f_* \F$ is coherent. Thus $f^! \omega_Y^\circ$ is a dualizing sheaf for $X$

\item Setting $\omega_X^\circ  = f^! \omega_Y^\circ$ we see that,
\[ \Hom{\struct{X}}{\omega_X^\circ}{f^! \omega_Y^\circ} = \Hom{\struct{Y}}{f_* \omega_X^\circ}{\omega_Y^\circ} \]
and thus $\id \in \Hom{\struct{X}}{\omega_X^\circ}{f^! \omega_Y^\circ}$ corresponds to a canonical trace map $\tr : f_* \omega_X^\circ \to \omega_Y^\circ$. We can describe this as the counit of the adjunction,
\[ \tr : f_* f^! \omega_Y^\circ \to \omega_Y^\circ \]
\bigskip\\
If $X$ and $Y$ are both nonsingular, and $k$ is algebraically closed then the dualizing and canonical sheafs agree $\omega_X^\circ = \omega_X$ and $\omega_Y^\circ = \omega_Y$. Therefore, we immediately get a canonical trace map $\tr : f_* \omega_X \to \omega_Y$.
\end{enumerate}

\renewcommand{\tr}[1]{\mathrm{tr}\left( #1 \right)}

\subsubsection{7.3}

Let $X = \P^n$. Consider the Euler sequence,
\begin{center}
\begin{tikzcd}
0 \arrow[r] & \Omega_{X} \arrow[r] & \struct{X}(-1)^{\oplus (n+1)} \arrow[r] & \struct{X} \arrow[r] & 0 
\end{tikzcd}
\end{center}
Then we may apply exterior powers to get the following sequence,
\begin{center}
\begin{tikzcd}
0 \arrow[r] & \Omega^p_{X} \arrow[r] & \bigwedge^p \struct{X}(-1)^{\oplus (n+1)} \arrow[r] & \struct{X} \otimes_{\struct{X}} \Omega^{p-1}_{X} \arrow[r] & 0
\end{tikzcd}
\end{center}
However, 
\[ \bigwedge^p \struct{X}(-1)^{\oplus (n+1)} = \struct{X}(-p)^{\oplus { n + 1 \choose p }} \] 
and thus we have the sequence,
\begin{center}
\begin{tikzcd}
0 \arrow[r] & \Omega^p_{X} \arrow[r] & \struct{X}(-p)^{\oplus { n + 1 \choose p }} \arrow[r] & \Omega^{p-1}_{X} \arrow[r] & 0
\end{tikzcd}
\end{center}
Now applying the cohomology sequence we find,
\begin{center}
\begin{tikzcd}
H^{q-1}(X, \struct{X}(-p))^{\oplus {n + 1 \choose p }} \arrow[r] & H^{q-1}(X, \Omega_{X}^{p-1}) \arrow[r] & H^{q}(X, \Omega_{X}^p) \arrow[r] & H^{q}(X, \struct{X}(-p))^{\oplus {n + 1 \choose p }}
\end{tikzcd}
\end{center}
Therefore, if $0 < q < n$ and $p > 0$ then $H^{q-1}(X, \Omega^{p-1}_{X}) \iso H^q(X, \Omega^p_{X})$. Furthermore, if $q = 0$ and $p > 0$ then $H^0(X, \Omega^p_{X}) = 0$ because we get an exact sequence,
\begin{center}
\begin{tikzcd}
0 \arrow[r] & H^0(X, \Omega^p_X) \arrow[r] & H^0(X, \struct{X}(-p))^{\oplus {n + 1 \choose p }}
\end{tikzcd}
\end{center}
and $H^0(X, \struct{X}(-p)) = 0$. Finally, if $q = n$ and $p < n + 1$ (which it must) then $H^{q-1}(X, \Omega_X^{p-1}) \iso H^q(X, \Omega_X^p)$ because $H^n(X, \struct{X}(-p)) = H^0(X, \struct{X}(p - n - 1))$ by Serre duality.
\bigskip\\
To finish the base case $p = 0$ we know,
\[ H^q(X, \struct{X}) = 
\begin{cases}
k & q = 0
\\
0 & q > 0
\end{cases} \]
Therefore, by induction, $H^p(X, \Omega_X^p) = H^{p-1}(X, \Omega_X^{p-1}) = k$ for $p \le n$. Furthermore, if $p \neq q$ then reducing via $H^{q-1}(X, \Omega^{p-1}_{X}) \iso H^q(X, \Omega^p_{X})$ we get to either $H^q(X, \Omega^0_X) = 0$ with $q > 0$ or $H^0(X, \Omega^p_X) = 0$ with $p > 0$. Therefore,
\[ H^q(X, \Omega^p_X) = 
\begin{cases}
k & p = q \le n
\\
0 & p \neq q
\end{cases} \]

\subsubsection{7.4 (CHECK THIS !!!)}

\renewcommand{\tr}{\mathrm{tr}}

Let $X$ be a nonsingular projective variety with $\dim{X} = n$ over an algebraically closed field $k$. Let $Y \subset X$ be a nonsingular subvariety of codimension $p$. The nautral pullback map $\iota^* \Omega_{X}^{n-p} \to \Omega_{Y}^{n-p}$ gives a map $\Omega_{X}^{n - p} \to \iota_* \Omega_{Y}^{n-p}$ by adjunction. Furthermore, $\Omega_{Y}^{n-p} = \omega_Y$ and therefore we get a map,
\[ \varphi_Y : H^{n-p}(X, \Omega_X^{n-p}) \to H^{n-p}(X, \iota_* \omega_Y) = H^{n-p}(Y, \omega_Y) \xrightarrow{\mathrm{tr}_Y} k \]
Therefore, by Serre duality we get an element $\eta(Y) \in H^p(X, \Omega_X^p)$ via,
\[ \eta(Y) = \varphi_Y \in H^{n-p}(X, \Omega^{n-p})^\vee = H^p(X, \Omega^p) \]

\begin{enumerate}
\item Let $Y = P$ for some closed point $P \in X$. Then $p = 0$ so $\varphi_P$ is the map,
\[ H^0(X, \struct{X}) \iso H^0(Y, \struct{X}) \iso k \]
Then Serre duality gives a perfect pairing,
\[ H^0(X, \struct{X}) \times H^n(X, \omega_X) \to k \]
given by $(a, q) \mapsto a \cdot \tr_X(q)$. Since $\eta(P)$ corresponds to $\varphi_P$ under the perfect pairing, we know that $(a, \eta(P)) \mapsto \varphi_P(a) = a$ and therefore $\tr_X(\eta(P)) = 1$. 

\item Let $X = \P^n_k$ and $Y \subset \P^n_k$ a smooth closed subvariety. If $\dim{Y} = 0$ then $Y$ is a finite set of points so, by the previous part, under the identification $H^n(X, \Omega_X^n) \xrightarrow{\tr_X} k$ we see that $\tr_X(\eta(Y)) = \deg{Y}$ where $\deg{Y}$ is equal to the number of points. Now we proceed by induction on $\dim{Y} = n - p$. Given $Y$ with $\dim{Y} = k + 1$ we take a generic hyperplane section $Y \cap H$ to get a subvariety $Y' \subset \P^{n}_k$ of the same degree with $\dim{Y'} = k$. By Bertini's theorem, for a generic hyperplane section the new subvariety $Y'$ will be smooth. 

\item Let $X$ be a scheme of finite type over $k$. There is a homomorphism of sheaves $\d{\log} : \struct{X}^\times \to \Omega_X$ given by $f \mapsto f^{-1} \d{f}$. This is a morphism of abelian sheaves because,
\[\d{\log}(fg) = f^{-1} g^{-1} \d{(fg)} = f^{-1} g^{-1} (g \d{f} + f \d{g} ) = f^{-1} \d{f} + g^{-1} \d{g} = \d{\log}(f) + \d{\log}(g) \]
Therefore we get a map $c_1 : \Pic{X} \to H^1(X, \Omega^1)$ via,
\[ \Pic{X} \iso H^1(X, \struct{X}^\times) \xrightarrow{\d{\log}} H^1(X, \Omega_X^1) \]

\item Let $p = 1$ then $Y \subset X$ is an effective Cartier divisor. We want to show that $\eta(Y) = c_1(\struct{X}(Y))$. We investigate these maps explicitly in terms of Cech cohomology. First we have the map,
\[ \iota^* \Omega^{n-1}_X \to \Omega^{n-1}_Y = \omega_Y = \iota^* \omega_X \otimes \struct{Y}(Y) \]
where the isomorphism $\omega_Y \iso \iota^* \omega_X \otimes \struct{Y}(Y)$
derives from the conormal sequence,
\begin{center}
\begin{tikzcd}
0 \arrow[r] & \struct{Y}(-Y) \arrow[r] & \iota^* \Omega_X \arrow[r] & \Omega_Y \arrow[r] & 0
\end{tikzcd}
\end{center}
Therefore, if we trivialize $\struct{X}(-Y)$ on an affine open cover $U_i$ with generating sections $e_i \in \Gamma(U_i, \struct{X}(-Y)) \subset \Gamma(U_i, \struct{X})$ then locally over $U_i$ the map is explicitly,
\[ \omega \mapsto \iota^* \omega \mapsto \iota^* \omega \otimes \id = \iota^* \omega \otimes e_i \otimes e^i \mapsto (\d{e_i} \wedge \omega) \otimes e^i \]
This is well-defined because if we use $e'_i = f_i e_i$ instead then,
\[ (\d{e_i'} \wedge \omega) \otimes e'^i = (\d{e_i} \wedge \omega) \otimes e^i + (e_i \d{f} \wedge \omega )\otimes e^i = (\d{e_i} \wedge \omega) \otimes e^i \]
because $e_i$ vanishes on $Y$ and thus $(e_i \d{f} \wedge \omega) \otimes e^i = 0$ as sections of $\iota^* \omega_X \otimes \struct{X}(Y)$.
\bigskip\\
Now, by adjunction for dualizing sheaves the trace map $\tr_Y : H^{n-1}(Y, \omega_Y) \to k$ factors through $\tr_X : H^n(X, \omega_X) \to k$ via the connecting map $\delta : H^{n-1}(X, \iota_* \omega_Y) \to H^n(X, \omega_X)$ from the exact sequence,
\begin{center}
\begin{tikzcd}
0 \arrow[r] & \omega_X \arrow[r] & \omega_X(Y) \arrow[r] & \iota_* \omega_Y \arrow[r] & 0
\end{tikzcd}
\end{center}
In terms of Cech cohomology for the cover $\U$, we can describe $\delta$ explicitly for an exact sequence of sheaves,
\begin{center}
\begin{tikzcd}
0 \arrow[r] & \K \arrow[r] & \F \arrow[r, "\varphi"] & \G \arrow[r] & 0
\end{tikzcd}
\end{center} 
Given a cocycle $\alpha \in \check{H}^{q}(\U, \G)$ we compute $\delta(\alpha) \in \check{H}^{q}(\U, \K)$ by first lifting $\alpha$ to a cochain $\tilde{\alpha} \in C^{q}(\U, \F)$ and applying the boundary map to get a cochain $\d{\tilde{\alpha}} \in C^{q+1}(\U, \F)$. However, $\varphi(\d{\tilde{\alpha}}) = \d{\varphi(\tilde{\alpha})} = \d{\alpha} = 0$ because $\alpha$ is a cocycle. Therefore, $\d{\tilde{\alpha}}$ is in the image of $C^{q+1}(\U, \K) \to C^{q+1}(\U, \F)$ and thus defines a cocycle $\delta(\alpha) \in \check{H}^{q+1}(\U, \K)$ because $\d{\delta(\alpha)} = \d^2{\tilde{\alpha}} = 0$ in $\F$. This is independent of the lift since two lifts differ by an element of $C^{q}(\U, \K)$ which maps to zero under $\d$ inside $\check{H}^{q+1}(\U, \K)$. Furthermore, the image of $\delta$ is exactly the kernel of $\check{H}^{q+1}(\U, \K) \to \check{H}^{q+1}(\U, \F)$ as expected because the image of $\delta(\alpha)$ is $\d{\tilde{\alpha}} \in C^{q+1}(\U, \F)$ which is a coboundary and thus zero in $\check{H}^{q+1}(\U, \F)$.
\bigskip\\
Consider $\omega \in \check{H}^{n-1}(X, \Omega^{n-1}_X)$ then we get $\omega' \in \check{H}^{n-1}(X, \iota_* \omega_Y)$ via,
\[ \omega'_{i_0, \dots, i_{n-1}} = (\d{e_{i_0}} \wedge \omega_{i_0, \dots, i_{n-1}}) \otimes e^{i_0} \]
and therefore,
\[ \varphi_Y(\omega) = \tr_X(\delta(\omega')) \]
Furthermore,
\begin{align*}
\delta(\omega')_{i_0, \dots, i_n} & = (\d{e_{i_1}} \wedge \omega_{i_1, \dots, i_{n}}) \otimes e^{i_1} + \left( \d{e_{i_0}} \wedge \sum_{k = 1}^n (-1)^k \omega_{i_0, \dots, \hat{i}_k, \dots, i_n} \right) \otimes e^{i_0}
\end{align*}
Furthermore, because $\omega$ is a cocycle, 
\[ (\d{\omega})_{i_0, \dots, i_n} = \sum_{k = 0}^n (-1)^k \omega_{i_0, \dots, \hat{i}_k, \dots, i_n} = \omega_{i_1, \dots, i_n} + \sum_{k = 1}^n (-1)^k \omega_{i_0, \dots, \hat{i}_k, \dots, i_n} = 0 \]
and therefore,
\begin{align*}
\delta(\omega')_{i_0, \dots, i_n} & = (\d{e_{i_1}} \wedge \omega_{i_1, \dots, i_{n}}) \otimes e^{i_1} - (\d{e_{i_0}} \wedge \omega_{i_1, \dots, i_n}) \otimes e^{i_0} 
\end{align*}
On the intersections, let $e_j = f_{ij} e_i$ for transition functions $f_{ij} \in \struct{U_{ij}}^\times$. Then we find,
\[ \d{e_j} \otimes e^j = \d{e_i} \otimes e^i + e_i f_{ij}^{-1} \d{f_{ij}} \otimes e^i \]
Therefore,
\begin{align*}
\delta(\omega')_{i_0, \dots, i_n} & = ( \d{\log}(f_{i_0 i_1}) \, e_{i_0} \wedge \omega_{i_1, \dots, i_n} ) \otimes e^{i_0} = ( \d{\log}(f_{i_0 i_1})  \wedge \omega_{i_1, \dots, i_n} ) \otimes e_{i_0} \otimes e^{i_0} = \d{\log}(f_{i_0 i_1}) \wedge \omega_{i_1, \dots, i_n}
\end{align*}
as an element of $H^n(X, \omega_X)$ using the identification of the subbundle $\omega_X(-Y) \subset \omega_X$ of forms vanishing on $Y$ (which $\delta(\omega')$ is in because of the $e_{i_0}$) after tensoring by $\struct{X}(Y)$ with the subbundle $\omega_X \subset \omega_X(Y)$ which sends $e_i \otimes e^i$ to $1$ under $\L \otimes \L^\vee \iso \struct{X}$. Therefore,
\[ \varphi_Y(\omega) = \tr_X( \{ \d{\log}(f_{i_0 i_1}) \wedge \omega_{i_1, \dots, i_n} \}_{i_0, \dots, i_n} ) \]
\bigskip\\
Next, consider the perfect pairing,
\[ H^1(X, \Omega_X^1) \times H^{n-1}(X, \Omega_X^{n-1}) \xrightarrow{\smile} H^n(X, \Omega^n) \xrightarrow{\tr_X} k \]
Explicitly on Cech cocycles $\alpha \in \check{H}^p(\U, \F)$ and $\beta \in \check{H}^{q}(\U, \G)$ the cup product,
\[  \check{H}^p(\U, \F) \times \check{H}^q(\U, \G) \xrightarrow{\smile} \check{H}^{p+q}(\U, \F \otimes_{\struct{X}} \G) \] 
acts via,
\[ (\alpha, \beta) \mapsto \alpha \otimes \beta \quad \text{where} \quad (\alpha \otimes \beta)_{i_0, \dots, i_n} = \alpha_{i_0, \dots, i_p} \otimes \beta_{i_p, \dots, i_n} \]
In our situation, we use the isomorphism $\Omega^1_X \otimes_{\struct{X}} \Omega^{n-1}_X \iso \Omega^n_X$ via $\alpha \otimes \omega \mapsto \alpha \wedge \omega$ so that explicitly on Cech cocycles $\alpha \in \check{H}^1(X, \Omega^1_X)$ and $\omega \in \check{H}^{n-1}(X, \Omega^{n-1}_X)$ the perfect pairing is,
\[ (\alpha, \omega) \mapsto \alpha \otimes \omega \mapsto \alpha \wedge \omega \mapsto \tr_X(\alpha \wedge \omega) \quad \text{where} \quad (\alpha \wedge \omega)_{i_0, \dots, i_n} = \alpha_{i_0, i_1} \otimes \omega_{i_1, \dots, i_n} \]
The class $c_1(\struct{X}(Y)) \in H^1(X, \Omega^1_X)$ corresponding to the Cech cocycle $\{ \d{\log}(f_{i_0 i_1}) \}_{i_0,i_1} \in \check{H}^1(X, \Omega^1)$ then has a dual,
\[ \omega \mapsto \tr_X( \{ \d{\log(f_{i_0, i_1})} \wedge \omega_{i_1, \dots, i_n} \}_{i_0, \dots, i_n}) \]
equal to $\varphi_Y(\omega)$. Therefore, the classes $\eta(Y) = c_1(\struct{X}(Y))$ themselves are equal.
\end{enumerate}

\subsection{Section 8}

\subsubsection{8.1}

Let $f : X \to Y$ be a continuous map of topological spaces. Let $\F$ be an abelian sheaf on $X$ and assume that $R^i f_* \F = 0$ for all $i > 0$. Then there are natural isomorphism,
\[ H^i(X, \F) = H^i(Y, f_* \F) \]
which follows immediately from the Leray spectral sequence and is done explicitly in exercises for Johan's class. Basically, this implies that $f_*$ preserves exactness of an injective resolution of $\F$ and also $f_*$ preserves injectives proving the proposition.

\subsubsection{8.2}

Let $f : X \to Y$ be an affine morphism of schemes with $X$ noetherian. Let $\F$ be a quasi-coherent sheaf on $X$. Then $R^i f_* \F$ is the sheafification of $V \mapsto H^i(f^{-1}(V), \F|_{f^{-1}(V)})$ but $f^{-1}(V)$ is a noetherian affine scheme and $\F |_{f^{-1}(V)}$ is quasi-coherent so $H^i(f^{-1}(V), \F|_{f^{-1}(V)}) = 0$ proving that $R^i f_* \F = 0$. Therefore, $H^i(X, \F) = H^i(Y, f_* \F)$.

\subsubsection{8.3}

Let $f : X \to Y$ be a morphism of ringed spaces. Let $\F$ be a $\struct{X}$-module, and $\E$ be a finite locally free $\struct{Y}$-module. We know there is a natural isomorphism,
\[ f_* (\F \otimes_{\struct{X}} f^* \E) \iso f_* \F \otimes_{\struct{Y}} \E \]
Since $\E$ is finite locally free $- \otimes_{\struct{Y}} \E$ is exact and therefore $R^i f_* (\F) \otimes_{\struct{Y}} \E$ form a $\delta$-functor. Furthermore, it is effacable because $R^i f_* \I = 0$ for injective $\struct{X}$-modules $\I$ and thus $R^i f_* (-) \otimes_{\struct{Y}} \E$ is a universal $\delta$-functor. Therefore, the degree zero isomorphism canonically extends to a  natural isomorphism of $\delta$-functors,
\[ R^i f_* (\F \otimes_{\struct{X}} f^* \E) \iso R^i f_* \F \otimes_{\struct{Y}} \E \]

\subsubsection{8.4}

Let $Y$ be a noetherian scheme and $\E$ a locally free $\struct{Y}$-module of rank $n + 1$ with $n \ge 1$. Let $X = \P(\E)$ and $\pi : X \to Y$ and $\struct{X}(1)$ be the anti-tautological line bundle.

\begin{enumerate}
\item By [II, Prop. 7.11] we know that $\pi_* \struct{X}(\ell) = \mathrm{Sym}^n_{\struct{Y}}\left( \E \right)$. Now we use [III, Prop. 8.5] which shows that for a map $f : X \to \Spec{A}$ and a coherent sheaf $\F$ on $X$ we have $R^q f_* \F = \wt{H^q(X, \F)}$. Therefore, choosing an affine open $\Spec{A} = U \subset Y$ trivializing $\E$ then,
\[ R^q \pi_* \struct{X}(\ell) |_U = \wt{H^q(X_U, \struct{X_U}(\ell))}  \cong \wt{H^q(\P^n_A, \struct{\P^n_A}(\ell))} = 0 \]
for $0 < q < n$ and all $\ell$ and for $q = n$ and $\ell > -(n+1)$. Therefore, $R^q \pi_* \struct{X}(\ell) = 0$ in these cases because these affine opens trivializing $\E$ cover $Y$. 

\item There is an exact sequence,
\begin{center}
\begin{tikzcd}
0 \arrow[r] & \G \arrow[r] & (\pi^* \E)(-1) \arrow[r] & \struct{X} \arrow[r] & 0
\end{tikzcd}
\end{center}
arising from the canonical surjection $\pi^* \E \onto \strut{X}(1)$ of [II, Prop. 7.11(b)]. Then I claim that $\G \cong \Omega_{X/Y}$. Over an affine open $\Spec{A} = U \subset Y$ trivializing $\E$ there is a morphism of exact sequences,
\begin{center}
\begin{tikzcd}
0 \arrow[r] & \G|_{X_U} \arrow[d, dashed] \arrow[r] & (\pi^* \E|_U)(-1) \arrow[d, equals] \arrow[r] & \struct{X_U} \arrow[r] \arrow[d, equals] & 0
\\
0 \arrow[r] & \Omega_{X_U/U} \arrow[d, "\sim"] \arrow[r] & (\pi^* \E|_U)(-1) \arrow[d, "\sim"] \arrow[r] & \struct{X_U} \arrow[r] \arrow[d, "\sim"] & 0
\\
0 \arrow[r] & \Omega_{\P^n_A/A} \arrow[r] & \struct{\P^n_A}(-1)^{\oplus (n+1)} \arrow[r] & \struct{\P^n_A} \arrow[r] & 0
\end{tikzcd}
\end{center}
where the bottow row is the Euler sequence and therefore the middle row is exact. That the above diagram commutes for any choice of isomorphism $X_U \iso \P^n_A$ over $A$ is equvalent to the fact that the map $\Omega_{\P_A(V)/A} \to V \ot \struct{\P_A(V)}(-1)$ is basis independent. This induces an isomorphism,
\[ \varphi_U : \G |_{X_U} \iso \Omega_{X_U/U} = \Omega_{X/Y}|_{X_U} \]
We need to show that these morphisms glue. Because of the naturality of the kernel, this map is unique and restricts correctly along inclusions of opens on the base. Therefore, for affine opens $U, V \subset Y$ and an affine open $W \subset U \cap V$ we see that,
\[ \varphi_U |_W = \varphi_W = \varphi_V |_W \]
showing that the local morphisms glue to an isomorphism $\varphi : \G \iso \Omega_{X/Y}$ giving a global Euler sequence,
\begin{center}
\begin{tikzcd}
0 \arrow[r] & \Omega_{X/Y} \arrow[r] & (\pi^* \E)(-1) \arrow[r] & \struct{X} \arrow[r] & 0
\end{tikzcd}
\end{center}
Therefore, the relative canonical sheaf is,
\[ \omega_{X/Y} = \wedge^n \Omega_{X/Y} \cong \wedge^{n+1} (\pi^* \E)(-1) \cong (\pi^* \wedge^{n+1} \E) \ot \struct{X}(-n-1) \]
Furthermore, there are canonical isomorphisms $t : H^n(\P^n_A, \omega_{\P^n_A/A}) \iso A$ and therefore the isomorphisms,
\[ R^n \pi_* \omega_{X/Y} |_U = \wt{H^n(X_U, \omega_{X_U/U})} \cong \wt{H^n(\P^n_A, \omega_{\P^n_A/A})} \cong \wt{A} = \struct{U} \]
are canonical and do not depend on the choice of isomorphism $X_U \cong \P^n_A$ over $A$. Therefore, these isomorphism glue as above to give a canonical isomorphism,
\[ R^n \pi_* \omega_{X/Y} \iso \struct{Y} \]
Notice however that on $\P^n_A$ the isomorphism $\omega_{X/Y} \cong \struct{X}(-n-1)$ and therefore also the isomorphism $H^n(\P^n_A, \struct{X}(-n-1)) \iso A$ do depend on a choice of basis. 

\item There is an natural $A$-linear perfect pairing,
\[ \Hom{\struct{\P^n_A}}{\struct{\P^n_A}(\ell)}{\omega_{\P^n_A/A}} \times H^n(\P^n_A, \struct{\P^n_A}(\ell)) \to H^n(\P^n_A, \omega_{\P^n_A/A}) \]
Therefore, the local pairings,
\[ \Hom{\struct{X_U}}{\struct{X_U}(\ell)}{\omega_{X_U/U}} \times H^n(X_U, \struct{X_U}(\ell)) \to H^n(X_U, \omega_{X_U/U}) \]
are well-defined and glue via naturality to a $\struct{Y}$-linear global perfect pairing,
\[ \pi_* \shHom{\struct{X}}{\struct{X}(\ell)}{\omega_{X/Y}} \times R^n \pi_* \struct{X}(\ell) \to R^n \pi_* \omega_{X/Y} \iso \struct{Y} \]
Therefore, we see that,
\[ R^n \pi_* \struct{X}(\ell) \cong \shHom{\struct{Y}}{\pi_* \shHom{\struct{X}}{\struct{X}(\ell)}{\omega_{X/Y}}}{R^n \pi_* \omega_{X/Y}} \cong \left( \pi_* \shHom{\struct{X}}{\struct{X}(\ell)}{\omega_{X/Y}} \right)^\vee \]
However, by the previous problem,
\[ \shHom{\struct{X}}{\struct{X}(\ell)}{\omega_{X/Y}} \cong \omega_{X/Y}(-\ell) \cong \struct{X}(-\ell - n - 1) \ot \pi^* \left( \wedge^{n+1} \E \right)  \]
Therefore, by the projection formual,
\[ \pi_* \shHom{\struct{X}}{\struct{X}(\ell)}{\omega_{X/Y}} \cong \pi_* \struct{X}(-\ell - n - 1) \ot \left( \wedge^{n+1} \E \right) \]
Finally, plugging into the previous formula we find,
\[ R^n \pi_* \struct{X}(\ell) \cong \pi_* \struct{X}(-\ell - n - 1)^\vee \ot \left( \wedge^{n+1} \E \right)^\vee \]

\item Some extra assumptions on $Y$ here are necessary for the invariants $p_a$ and $p_g$ to make sense. Suppose that $Y$ is a proper scheme over a field $\Spec{k}$ so that its cohomology groups are finite $k$-modules. The arithmetic genus is defined as,
\[ p_a(X) = [ \chi(X, \struct{X}) - 1 ] (-1)^{\dim{X}} \]
Notice that $R^q \pi_* \struct{X} = 0$ for all $q > 0$ and therefore the Leray spectral sequence degenerates giving $H^q(X, \struct{X}) = H^q(Y, \pi_* \struct{X}) \cong H^q(Y, \struct{Y})$ for all $q$. Therefore, 
\[ \chi(X, \struct{X}) = \sum_{i} (-1)^i \dim_k H^0(X, \struct{X}) = \sum_i (-1)^i \dim_k H^0(Y, \struct{Y}) = \chi(Y, \struct{Y}) \]
Thus,
\[ p_a(X) = [ \chi(X, \struct{X}) - 1 ] (-1)^{\dim{X}} = [ \chi(Y, \struct{Y}) - 1](-1)^{\dim{Y} + n} = (-1)^n p_a(Y) \]
Now furthermore, $\pi : X \to Y$ is smooth because it is locally on $X$ isomorphic to $\A^n_A \to \Spec{A}$ which is smooth. Therefore we get an exact sequence,
\begin{center}
\begin{tikzcd}
0 \arrow[r] & \pi^* \Omega_Y \arrow[r] & \Omega_X \arrow[r] & \Omega_{X/Y} \arrow[r] & 0
\end{tikzcd}
\end{center}
and as we have seen previously $\Omega_{X/Y}$ is locally free. Therefore, if $X \to \Spec{k}$ is smooth then $\Omega_Y$ is locally free and therefore $\Omega_X$ is also locally free from the exact sequence (or since $X \to Y \to \Spec{k}$ is then smooth) and therefore there is an isomorphism,
\[ \omega_X \cong \wedge^{\dim{X}} \Omega_X \cong \pi^* \wedge^{\dim{Y}} \Omega_Y \ot \wedge^{n} \Omega_{X/Y} \cong \pi^* \omega_Y \ot \omega_{X/Y} \]
Then we define,
\[ p_g(X) = \dim_k H^0(X, \omega_X) \]
However, 
\[ H^0(X, \omega_X) = H^0(Y, \pi_* \omega_X) \]
By the projection formula,
\[ \pi_* \omega_X \cong \omega_Y \ot \pi_* \omega_{X/Y} \]
and furthermore,
\[ \omega_{X/Y} \cong \left( \wedge^{n+1} \E \right) \ot \struct{X}(-n-1) \]
so that, again by the projection formula,
\[ \pi_* \omega_{X/Y} \cong \left( \wedge^{n+1} \E \right) \ot \pi_* \struct{X}(-n-1) = 0 \]
because $-n -1 < 0$. Therefore, $\pi_* \omega_X = 0$ and therefore,
\[ p_g(X) = \dim_k H^0(X, \omega_X) = \dim_k H^0(Y, \pi_* \omega_X) = 0 \]

\item In particular, let $Y$ be a nonsingular projective curve of genus $g$, and $\E$ a locally free sheaf of rank $2$, then $X$ is a projective (by [II, Ex. 7.14(b)]) surface with $p_a(X) = -p_a(Y) = -g$ and $p_g(X) = 0$. The irregulaity $q$ is defined as,
\[ q = p_g(X) - p_a(X) = g \]
This is known as a projectively ruled surface.
\end{enumerate}

\subsection{Section 9}

\subsection{Section 10}


\subsubsection{10.1}

Let $k_0$ be a field of characteristic $p > 2$ and $k = k_0(t)$. Consider the plane curve,
\[ X = \Spec{k[x,y]/(y^2 - x^p + t)} \]
and let $f : X \to \Spec{k}$ be the structure map. Let $A = k[x,y]/(y^2 - x^p + t)$ and consider the naive cotangent complex,
\begin{center}
\begin{tikzcd}
0 \arrow[r] & A \arrow[r, "2 y \d{y}"] & A \d{x} \oplus A \d{y} \arrow[r] & 0
\end{tikzcd}
\end{center}
which is quasi-isomorphic to $\Omega_{A/k} \cong A \d{x} \oplus A/(2y) \d{y}$ which is not projective because it jumps rank at the origin $\p = (y)$. Notice that $\p = (y)$ is maximal because,
\[ A/\p = k[x]/(x^p - t) = k(t^{\frac{1}{p}}) \]
is a field. Therefore $f$ is smooth everywhere except for $\p$ and thus regular everywhere except maybe at $\p$. However, $\dim{X} = 1$ and $\p$ is principal already so clearly $A_\p$ is regular. Therefore $X$ is regular but $X \to \Spec{k}$ is not smooth at $\p$. 
\bigskip\\
Notice futhermore that after base changing to $k(t^{\frac{1}{p}})$ that $A$ is no longer regular. Consider,
\[ A' = A \otimes_k k(t^{\frac{1}{p}}) = k(t^{\frac{1}{p}})[x, y]/(y^2 - x^p + t) \]
which has maximal ideal $\p' = (y, x - t^{\frac{1}{p}})$. However, $A'_{\p'}$ is not regular because the relation,
\[ y \cdot y = (x - t^{\frac{1}{p}})(x - t^{\frac{1}{p}})^{p - 1} \]
involves only elements of the maximal ideal and thus cannot be inverted to isolate either generator.

\subsubsection{10.2}

Let $f : X \to Y$ be a proper, flat morphism of varities over $k$. For $y \in Y$ suppose that $X_y \to \Spec{\kappa(y)}$ is smooth and therefore $f$ is smooth at each point $x \in X_y$. Consider the smooth locus $S \subset X$ of $f$ which is open because smoothness is an open property. Since $f$ is proper we see that $f(X \setminus S)$ is closed so consider the open set $U = Y \setminus f(X \setminus S)$. Furthermore, since $f^{-1}(y) \subset S$ we know that $y \notin f(X \setminus S)$ because $f^{-1}(y) \subset S$ so is disjoint from $X \setminus S$ and thus $y \in U$. Then consider $f : f^{-1}(U) \to U$. If $x \in f^{-1}(U)$ then $f(x) \in U$ meaning that $f^{-1}(f(x))$ is disjoint from $X \setminus S$ so $X_{f(x)}$ is smooth. Therefore $f : f^{-1}(U) \to U$ is smooth.

\subsubsection{10.3}

\begin{defn}
Let $f : X \to Y$ be a morphism of finite type schemes over a field $k$. We say $f$ is \etale if it is smooth of relative dimension $0$.
\end{defn}

\begin{defn}
We say that $f : X \to Y$ is unramified at $x \in X$ if for $y = f(x)$ we have $\m_y \stalk{X}{x} = \m_x$ and $\kappa(y) \embed \kappa(x)$ is a separable algebraic extension.
\end{defn}

Now we show the following:

\begin{prop}
Let $f : X \to Y$ be a finite type schemes over $k$ then the following are equivalent,
\begin{enumerate}
\item $f$ is \etale
\item $f$ is flat and $\Omega_{X/Y} = 0$
\item $f$ is flat and unramified.
\end{enumerate}
\end{prop}

\begin{proof}
If $f$ is \etale by definition $f$ is flat and,
\[ \dim_{\kappa(x)}((\Omega_{X/Y})_x \ot \kappa(x)) = 0 \]
for all $x \in X$ and thus $\Omega_{X/Y} = 0$.
\bigskip\\
Then assume $f$ is flat and $\Omega_{X/Y} = 0$. Then we need to prove that $f$ is unramified. We have $\Omega_{X_y / \kappa(y)}$ is the pullback of $\Omega_{X/Y}$ over $\Spec{\kappa(y)} \to Y$ and therefore $\Omega_{X_y / \kappa(y)} = 0$. For each $x \in X_y$ let $(A, \m)$ be the local ring at $x \in X_y$ then there is an exact sequence,
\begin{center}
\begin{tikzcd}
\m / \m^2 \arrow[r] & \Omega_{A/\kappa(y)} \ot \kappa(x) \arrow[r] & \Omega_{\kappa(x)/\kappa(y)} \arrow[r] & 0
\end{tikzcd}
\end{center}
However, $\Omega_{A/\kappa(y)} = (\Omega_{X_y / \kappa(y)})_x = 0$ and therefore we conclude that $\Omega_{\kappa(x)/\kappa(y)} = 0$ meaning that $\kappa(y) \embed \kappa(x)$ is a seperable algebraic extension by [II, Theorem 8.6A]. Furthermore, in that situation, by [III, Ex. 8.1] we know that the sequence is exact on the left as well proving that $\m / \m^2 = 0$ so $\m = \m^2$. Therefore, by Nakayama's lemma, $\m = 0$ proving that $A = \stalk{X_y}{x} = \stalk{X}{x} / \m_y \stalk{X}{x}$ is a field and therefore $\m_y \stalk{X}{x} = \m_x$ so $f : X \to Y$ is unramified and also, by hypothesis, flat. 
\bigskip\\
Suppose that $f$ is flat an unramified. We need to show that $f$ is \etale. Choose $x \in X$ with $y = f(x)$ then we know,
\[ \stalk{X_y}{x} = \stalk{X}{x} / \m_y \stalk{X}{x} = \stalk{X}{x} / \m_x = \kappa(x) \]
which is an algebraic separable extension of $\kappa(y)$ and hence geometrically regular showing that $X_y$ is geometrically regular over $\kappa(y)$. Furthermore, since $\stalk{X_y}{x}$ is a field for all $x$ we see that $\dim{X_y} = 0$ because every point is closed. Therefore, by [III, Theorem 10.2] $f$ is smooth of relative dimension $0$ and thus \etale. 
\end{proof}

\subsubsection{10.4}

Let $f : X \to Y$ be a morphism of schemes of finite type over a field $k$. We will show that $f$ is \etale if and only if the following condition is satisfied. For each $x \in X$ let $y = f(x)$ then $\kappa(y) \embed \kappa(x)$ is a separable algebraic extension and,
\[ \widehat{\stalk{Y}{y}} \ot_{\kappa(y)} \kappa(x) \to \widehat{\stalk{X}{x}} \]
is an isomorphism where we choose fields of representatives $\kappa(x) \subset \widehat{\stalk{X}{x}}$ and $\kappa(y) \subset \widehat{\stalk{Y}{y}}$ so that the above makes sense.
\bigskip\\
First suppose that $f$ is \etale. Then by the previous problem we know that $f$ is flat and unramified. Therefore $\kappa(y) \embed \kappa(x)$ is an algebraic separable extension. Furthermore, consider the diagram, 
\begin{center}
\begin{tikzcd}
0 \arrow[r] & \m_x^n / \m_x^{n+1} \arrow[r] & \stalk{X}{x} / \m_x^{n+1} \arrow[r] & \stalk{X}{x} / \m_x^n \arrow[r] & 0
\\
0 \arrow[r] & (\m_y^n / \m_y^{n+1}) \ot_{\kappa(y)} \kappa(x) \arrow[u] \arrow[r] & (\stalk{Y}{y} / \m_y^{n+1}) \ot_{\kappa(y)} \kappa(x) \arrow[u] \arrow[r] & (\stalk{Y}{y} / \m_y^n) \ot_{\kappa(y)} \kappa(x) \arrow[u] \arrow[r] & 0
\end{tikzcd}
\end{center}
where the tensor product is taken over $\kappa(y)$ and thus is exact. The rightmost morphism in the $n = 0$ case is the isomorphism $\kappa(y) \ot_{\kappa(y)} \kappa(x) \to \kappa(x)$, We will show these are isomorphisms for all $n$. Since $\m_y \stalk{X}{x} = \m_x$ we see that likewise $\m_y^n \stalk{X}{x} = \m_x^n$ and therefore,
\[ (\m_y^n / \m_y^{n+1}) \ot_{\kappa(y)} \kappa(x) \to \m_x^n / \m_x^{n+1} \]
is surjective for each $n$. Furthermore, $f$ is flat so $\stalk{Y}{y} \to \stalk{X}{x}$ is flat meaning that,
\[ \m_y^n \ot_{\stalk{Y}{y}} \stalk{X}{x} \to \struct{Y}{y} \ot_{\struct{Y}{y}} \struct{X}{x} = \struct{X}{x} \]
is injective and therefore it is an isomorphism onto its image,
\[ m_y^n \ot_{\stalk{Y}{y}} \stalk{X}{x} \iso \m_x^n \]
Therefore applying $- \otimes_{\stalk{X}{x}} \kappa(x)$ gives an isomorphism,
\[ (\m_y^n / \m_y^{n+1}) \ot_{\kappa(y)} \kappa(x) \iso \m_x^n / \m_x^{n+1} \]
for each $n$ because,
\[ \m_y^n \ot_{\stalk{Y}{y}} \stalk{X}{x} \ot_{\stalk{X}{x}} \kappa(x) = \m_y^n \ot_{\stalk{Y}{y}} \kappa(x) = \m_y^n \ot_{\stalk{Y}{y}} \kappa(y) \ot_{\kappa(y)} \kappa(x) = \m_y^n / \m_y^{n+1} \ot_{\kappa(y)} \kappa(x) \]
Therefore, by induction the maps,
\[ (\stalk{Y}{y} / \m_y^{n+1}) \ot_{\kappa(y)} \kappa(x) \iso \stalk{X}{x} / \m_x^{n+1} \] 
are all isomorphisms showing the the compatible system inducing the map on complete local rings actually induced an isomorphism,
\[ \widehat{\stalk{Y}{y}} \ot_{\kappa(y)} \kappa(x) \iso \widehat{\stalk{X}{x}} \]
where tensor product over the field $\kappa(x)$ commutes with taking inverse limits because $- \ot_{\kappa(y)} \kappa(x)$ has a left adjoint. Indeed, $\Res{\kappa(x)}{\kappa(y)}{-}$ is canonically a right adjoint but because $\kappa(x) / \kappa(y)$ is finite, since it is an finitely-generated (since $f : X \to Y$ is finite type) algebraic extension, we also have that $\Res{\kappa(x)}{\kappa(y)}{-}$ is left adjoint to $- \ot_{\kappa(y)} \kappa(x) = \Hom{\kappa(y)}{\kappa(x)^*}{-}$.
\bigskip\\
Conversely, if the canonical map,
\[ \widehat{\stalk{Y}{y}} \ot_{\kappa(y)} \kappa(x) \iso \widehat{\stalk{X}{x}} \]
is an isomorphism then $\m_y \widehat{\stalk{X}{x}} = \m_x \widehat{\stalk{X}{x}}$. The inclusion $\m_y \widehat{\stalk{X}{x}} \subset \m_x \widehat{\stalk{X}{x}}$ is obvious and conversely we see that the maximal ideal $\m_x \widehat{\stalk{X}{x}}$ is surjected onto by the maximal ideal of $\widehat{\stalk{Y}{y}} \ot_{\kappa(y)} \kappa(x)$ which is generated by $\m_y$ since,
\[ (\widehat{\stalk{Y}{y}} \ot_{\kappa(y)} \kappa(x))/\m_y \widehat{\stalk{Y}{y}} \ot_{\kappa(y)} \kappa(x) = (\widehat{\stalk{Y}{y}} / \m_y \widehat{\stalk{Y}{y}}) \ot_{\kappa(y)} \kappa(x) = \kappa(y) \ot_{\kappa(y)} \kappa(x) = \kappa(x) \]
is a field. Therefore, because for any local ring $A$, the map $A \to \hat{A}$ is local and faithfully flat [Mat, Theorem 8.14],
\[ \m_y \stalk{X}{x} = \m_y \widehat{\stalk{X}{x}} \cap \stalk{X}{x} = \m_x \widehat{\stalk{X}{x}} \cap \stalk{X}{x} = \m_x \]
because for $A \to B$ faithfully flat we have $I B \cap A = I$ for any ideal $I \subset A$ [Mat, Theorem 7.5]. 
Therefore, $f$ is unramified since we also have $\kappa(y) \embed \kappa(x)$ separable algebraic by hypothesis.
Furthermore, there is a diagam, 
\begin{center}
\begin{tikzcd}
\stalk{Y}{y} \arrow[r] \arrow[d] & \stalk{X}{x} \arrow[d]
\\
\widehat{\stalk{Y}{y}} \ot_{\kappa(y)} \kappa(x) \arrow[r, "\sim"] & \widehat{\stalk{X}{x}}
\end{tikzcd}
\end{center}
and the downward maps are faithfully flat. Since the bottom map is an isomorphism and thus flat we see that $\stalk{Y}{y} \to \stalk{X}{x}$ is flat since ``flatness descends along faithfully flat maps'' by a direct argument. Therefore by [III, Ex. 10.3] $f$ is \etale because it is flat and unramified.

\subsubsection{10.5}

Let $X$ be a scheme and $\F$ a coherent $\struct{X}$-module. If $\F$ is locally free then for each $x \in X$ there is a Zariski neighborhood $U$ of $x$ such that $\F|_U$ is a free $\struct{U}$-module and $U \to X$ is clearly an \etale neighborhood. 
\bigskip\\
Conversely, suppose that $\F$ is \etale locally free. Since we need to show that $\F$ is Zariski locally free, it suffices to shrink the target to the image of $U \to X$ so we consider the case that $f : U \to X$ is a surjective \etale cover of affine schemes. In particular, $f$ is faithfully flat. Therefore, we reduce to the situation that $A \to B$ is an \etale (and thus faithfully flat) ring map and $M$ is a finite $A$-module such that $M \otimes_A B$ is free. Thus, $M \otimes_A B$ is finite projective. Since $A \to B$ is faithfully flat we see that $M$ is finite projective and thus locally free by Tag 058R.


\subsubsection{10.6}

Assume that $k$ is a field of characteristic not equal to $2$ (otherwise the curve is not nodal) and consider, $A = k[x,y]/(y^2 - x^2(x + 1))$. Then the normalization in $K = \Frac{A}$ is,
\[ \wt{A} = k[x,y,\tfrac{y}{x}]/(y^2 - x^2(x + 1)) = k[t] \]
where $t = \frac{y}{x}$ under $x = t^2 - 1$ and $y = t(t^2 - 1)$ with the normalization map $\nu : \Spec{\wt{A}} = \A^1 \to \Spec{A}$ via $x \mapsto t^2 - 1$ and $y \mapsto t(t^2 - 1)$. Now consider two copies of the normalization $\A^1$ glued at $\pm 1$ as follows,
\[ B = \{ (p, q) \in k[t_1] \times k[t_2] \mid p(1) = q(-1) \text{ and } p(-1) = q(1) \} \subset k[t_1] \times k[t_2] \]
Then the map $f : \Spec{B} \to \Spec{A}$ is given by gluing the normalization maps which agree at the glued points. I need to show that $f$ is finite \etale of degree $2$. It is clear that $B$ is a finite $A$-module by inspection since $\nu$ is finite. Furthermore, $\nu$ is generically an isomorphism so $f$ has generic degree $2$ because explicitly $B \ot_A K \cong K^2$. Furthermore, $\nu$ is unramified (its only nontrivial fiber is two reduced points). Therefore, if $f$ is flat then it is \etale and constant rank so it is finite \etale of degree $2$.
\bigskip\\
Away from the singular point $\p = (x,y) \subset \Spec{A}$ the map $\nu$ is an isomorphism and $f$ is a disjoint union of two copies of $\nu$ and thus is flat. Therefore, it suffices to show that $f$ is flat over the singular point. 
\bigskip\\
Therefore, we need to show that $A_\p \to B_\q$ is flat where $\q \mapsto \p$ and without loss of generality we can take $\q = (t_1 - 1, t_2 + 1)$ since the argument at the other point is identical. By the slicing criterion for local flatness, $A_\p \to B_\q$ is flat if there exists a non-zerodivisor $f \in A_\p$ such that $f \in B_\q$ is a non-zerodivisor and $A_\p / f \to B_\q / f B_\q$ is flat. Let $f = x$ which is a non-zerodivisor in $A_\p$ because $A_\p$ is a domain since the localization of a domain is a domain. Furthermore $x \mapsto (t_1^2 - 1, t_2^2 - 1) \in B$ is a non-zerodivisor because $B \subset k[t_1] \times k[t_2]$ and the image of $f$ in $k[t_1] \times k[t_2]$ is a non-zerodivisor because the image of $f$ in the domain $k[t_i]$ is a non-zerodivisor (by the lemma). Then the image of $f$ in $B$ remains a non-zerodivisor under $B \to B_\q$.
\bigskip\\
Now it suffices to show that, $A / (f) \to B / f B$ is flat because then it remains flat after localizing. This is the map,
\[ k[y]/(y^2) \to B / f B  \]
Furthermore, a $D = k[y]/(y^2)$-module $M$ is flat if and only if $M/(y) \xrightarrow{y} M$ is injective. Notice that,
\[ M/(y) = B / (x,y)B = \{ (p,q) \in k[t_1] / (t_1^2 - 1) \times k[t_2] / (t_2^2 - 1) \mid p(1) = q(-1) \text{ and } p(-1) = p(1) \} \]
which by CRT is,
\[ M/(y) \cong \{ (a,b,b,a) \mid a,b \in k \} \subset k^4 \]
where the isomorphism evaluates $p$ at $\pm 1$ and $q$ at $\pm 1$. 
Now I need to show that $M/(y) \to M$ is injective (you might think that $y = 0$ on $B$ because $y = t x$ in $k[t]$ but notice that $(t_1, t_2) \notin B$ and thus $y \notin f B$). To do this it suffices to prove injectivity on a basis. Consider $(1,0,0,1)$ corresponding to, 
\[ \tfrac{1}{2} (t_1 + 1, -t_2 + 1) \mapsto \tfrac{1}{2} y [(t_1, -t_2) + (1,1)] = \tfrac{1}{2} [ x (1,-1) + y ] \]
Likewise $(0, 1, 1, 0)$ corresponds to
\[ \tfrac{1}{2} (-t_1 + 1, t_2 + 1) \mapsto \tfrac{1}{2} y [(-t_2, t_2) + (1,1)] = \tfrac{1}{2} [x(-1, 1) + y] \]
Therefore, if $(a,b,b,a) \mapsto 0$ under $M/(y) \to M$ then,
\[ \tfrac{1}{2} [(a - b) x (1, -1) + (a + b) y ] \in f B \]
so there is some $(p,q) \in B$ such that $xp = \tfrac{1}{2} ((a - b) x + (a + b) y)$ and hence $p = \tfrac{1}{2} [(a - b) + (a + b) t_1]$ and likewise $xq = \tfrac{1}{2}((b-a) x + (a + b) y)$ and hence $q = \tfrac{1}{2} [(b - a) + (a + b) t_2]$ using that $x$ is a non-zerodivisor in $k[t_i]$. However, we require that $p(1) = q(-1)$ and $p(-1) = q(1)$ and thus,
\[ \tfrac{1}{2} [(a - b) + (a + b)] = \tfrac{1}{2} [-(a - b) - (a + b)] \quad \text{and} \quad \tfrac{1}{2}[(a - b) - (a - b)] = \tfrac{1}{2}[-(a - b) + (a + b)] \] 
which implies that,
\[ a = - a \quad \text{ and } \quad - b = b \]
so because $2 \in k$ is invertible we see that $a = b = 0$ proving injectivity and thus flatness.
\bigskip\\
Notice that without the gluing we would have $k[y]/(y^2) \to k[t]/(t^2 - 1)$ sending $y \mapsto t(t^2 - 1) = 0$ which is not flat.


\begin{lemma}
Let $f \in A$ and $g \in B$ be non-zerodivisors. Then $(f,g) \in A \times B$ is a non-zerodivisor.
\end{lemma}

\begin{proof}
Suppose that $(a,b) \in A \times B$ with $(a,b) \cdot (f,g) = (af, bg) = 0$ then $af = 0$ and $bg = 0$ so $a = b = 0$.
\end{proof}

\subsubsection{10.7 DO!!}

Let $k$ be an algebraically closed field of characteristic $2$. Let $P_1, \dots, P_7 \in \P^2_k$ be the $7$ rational points of $\P^2_{\FF_7}$ base changed to $k$. Then let $\mathfrak{d}$ be the linear system of all cubic curves in $X = \P^2_k$ passing through $P_1, \dots, P_7$.

\begin{enumerate}
\item There is a map $\Gamma(X, \struct{X}(3)) \to k^7$ determined by evaluation at the $7$ points. We are interested in the kernel $V$ of this map. Choosing the basis,
\[ z^3, y z^2, y^2 z, y^3, x z^2, x y z, x y^2, x^2 z, x^2 y, x^3 \]
for $\Gamma(X, \struct{X}(3))$ a straightforward computer algebra calculation shows that the matrix of this map has full rank
and therefore the map is surjective so $\dim{V} = 3$. It is clear that $P_1, \dots, P_7$ are basepoints of $V$ by construction. 
A basis of $V$ is given by,
\begin{align*}
c_1 & = x^2 y + x y^2
\\
c_2 & = x^2 z + x z^2
\\
c_3 & = y z^2 + z y^2
\end{align*} 
Solving the system of equations shows that there are no other base points.
\bigskip\\
Therefore we get a morphism $X \setminus \{ P_i \} \to \P^2_k$. At the generic point this map $k(s,t) \to k(x,y)$ is defined by sending,
\[ s \mapsto \frac{x^2 y + x y^2}{y + y^2} \quad \text{ and } \quad t \mapsto \frac{x^2 + x}{y^2 + y} \]
Now,
\[ s + y t = \frac{x y^2 + xy}{y + y^2} = x \]
and therefore $k(x,y) = k(s,t)(y)$ and furthermore, $y^2 \in k(s,t)$ because,
\[ (y^2 + y) t = x^2 + x = (s + y t)^2 + (s + t y) = s^2 + y^2 t^2 + s + t y \]
and therefore,
\[ y^2 = \frac{s^2 + s}{t^2 + t} \]
Therefore, $k(s,t) \embed k(x,t)$ is purely inseperable of degree $2$. 

\item We first need to show that every element of $V$ is singular. Notice,
\begin{center}
\begin{tabular}{c|c|c}
$\pderiv{c_1}{x} = y^2$ & $\pderiv{c_1}{y} = x^2$ & $\pderiv{c_1}{z} = 0$
\\
$\pderiv{c_2}{x} = z^2$ & $\pderiv{c_2}{y} = 0$ & $\pderiv{c_2}{z} = x^2$
\\
$\pderiv{c_3}{x} = 0$ & $\pderiv{c_3}{y} = z^2$ & $\pderiv{c_3}{z} = y^2$
\end{tabular}
\end{center}
Therefore, given $g = \alpha c_1 + \beta c_2 + \gamma c_3 = 0$ we require that,
\begin{align*}
\pderiv{g}{x} & = \alpha y^2 + \beta z^2 = 0 
\\
\pderiv{g}{y} & = \alpha x^2 + \gamma z^2 = 0 
\\
\pderiv{g}{z} & = \beta x^2 + \gamma y^2 = 0
\end{align*}
which occurs for all $\alpha, \beta, \gamma$ exactly when $[x:y:z] = [\sqrt{\gamma} : \sqrt{\beta} : \sqrt{\alpha}]$ and is the unique (because there is a unique square root in characteristic two) singular point on the conic determined by $g$. This shows there is a one-to-one correspondence between the singular point $P$ and the singular conic $C = V(g)$ determined by $g \in V$ where the map sends $P = [a : b : c] \mapsto [a^2 : b^2 : c^2]$ determining the equation $g = a x + b y + c z$.
\bigskip\\
A singular cubic is exactly one of:
\begin{enumerate}
\item a multiplicity 3 line
\item a line and a multiplicity 2 line
\item three lines
\item a conic together with a line
\item a nodal cubic with one node
\item a cuspidal cubic with one cusp
\end{enumerate}

However, exactly three $P_i$ line on every $\FF_2$-rational line and one $P_i$ on any other line and every pair of rational lines intersects at exactly one of $P_1, \dots, P_7$ (look at the Fano plane). Therefore if a union of 3 lines passes through all $P_1, \dots, P_7$ then all three must be rational and distinct (all multiplicites are $1$) because otherwise they could only pass through at most $5 + 1 = 6$ of the points (two distinct rational lines give $3 + 3 - 1 = 5$ and one irrational line can pass through at most $1$ additional point). Looking at the Fano plane, the only combination of $3$ rational lines that cover every point all intersect at one singular point $P$.
\bigskip\\
Otherwise, I claim that $C \in \mathfrak{d}$ is a conic along with a tangent line or a cuspidal cubic with cusp away from $\{ P_i \}$. Since $C \in \mathfrak{d}$ has a unique singular point if it has a component that is a conic then the line must be tangent. Finally, we need to consider the case that $C \in \mathfrak{d}$ is irreducible. (DO THIS PART!!!)
\end{enumerate}

\subsubsection{10.8}

Remark: need to assume in this problem that $\ch{k} \neq 2$. 
\bigskip\\
Consider in $\A^3 = \Spec{k[x,y,z]}$ the conic $C$ cut out by $(x-1)^2 + y^2 = 1$ in the $x$-$y$ plane, and let $P_t$ be the point $(0,0,t)$ on the $z$-axis. Let $Y_t$ be the closure in $\P^3$ of the cone over $C$ with vertex $P_t$. We can cut out this affine cone by the equation,
\[ (x + (z/t - 1))^2 + y^2 - (z/t - 1)^2 \]
which we homogenize and clear denominators of $t$ to find,
\[ (tx + (z + tw))^2 + t^2 y^2 - (z - t w)^2 = ((x^2 + y^2 - 2 x w)t + 2 x z)t \]
and therefore clearing $t$ we find that $Y_t$ has the homogenized equation,
\[ g_t = (x^2 + y^2 - 2 wx) t + 2 x z  \]
cut out of $\P^3 = \Proj{k[x,y,z,w]}$. Therefore, this is a one-dimensional linear system of divisors defined by the two conic equations,
\[ x^2 + y^2 - 2 w x \quad \text{ and } \quad 2 x z \] 
To check where the singularities of $Y_t$ live we compute the partial derivatives,
\begin{align*}
\pderiv{g_t}{x} & = 2 (x - w) t + 2 z 
\\
\pderiv{g_t}{y} & = 2 t y 
\\
\pderiv{g_t}{z} & = 2 x
\\
\pderiv{g_t}{w} & = -2 t x
\end{align*}
Therefore, we must have $x = 0$ and either $t = 0$ and $z = 0$ or we also have $y = 0$ and $z = w t$. Therefore, we get solutions,
\[ P_t = [0:0:t:1] \quad \text{ or } \quad t = 0 \text{ and } \{ [0:y:0:w] \} \]
so the singularities are $P_t$ for $t \neq 0$ and the $y$-axis for $t = 0$.  
\bigskip\\
Finally, we compute the base locus. Notice that $[0:0:z:w]$ satisfies the equation for all $t$ so the $z$-axis is contained in the base locus. Furthermore, $[x:y:0:1]$ such that $(x-1)^2 + y^2 = 1$ satisfies this equation for all $t$ so the base locus contains the conic $C$ and the $z$-axis. I need to show it does not contain any other point. The base locus is exactly the vanishing of both $x^2 + y^2 - 2 w x$ and $2 x z$. Thus either $x = 0$ or $z = 0$. If $x = 0$ then $y = 0$ so we get the $z$-axis. If $z = 0$ then we get $(x - w)^2 + y^2 - w^2$ giving the conic $C$.

\subsubsection{10.9}

Let $f : X \to Y$ be a morphism of varieties over $k$. Assume that $Y$ is regular, $X$ is Cohen-Macaulay and that every fiber of $f$ has dimension $\dim{X} - \dim{Y}$.
\bigskip\\
To show that $f : X \to Y$ is flat it suffices to show that $\stalk{Y}{y} \to \stalk{X}{x}$ with $f(x) = y$ is flat. Then by [II, Ex. 3.20] since $X,Y$ are finite type schemes over $k$ then,
 \[ \dim{X} = \dim{\stalk{X}{x}} + \trdeg{k}{\kappa(x)} \quad \text{and} \quad \dim{Y} = \dim{\stalk{Y}{y}} + \trdeg{k}{\kappa(y)} \] 
Furthermore, by [II, Ex. 3.22(b)] every component of $X_y$ has dimension $\ge \dim{X} - \dim{Y}$ but since $\dim{X_y} = \dim{X} - \dim{Y}$ by hypothesis this implies that the fibers are equidimensional. Therefore, 
\[ \dim{\stalk{X_y}{x}} + \trdeg{\kappa(y)}{\kappa(x)} = \dim{X_y} = \dim{X} - \dim{Y} \]
which implies that,
\begin{align*}
\dim{\stalk{X_y}{x}} & = \dim{X} - \dim{Y} - \trdeg{\kappa(y)}{\kappa(x)} = \dim{X} - \dim{Y} - \trdeg{k}{\kappa(x)}  + \trdeg{k}{\kappa(y)} 
\\
& = \dim{\stalk{X}{x}} - \dim{\stalk{Y}{y}}
\end{align*}
Furthermore, $\stalk{X_y}{y} = \stalk{X}{x} / \m_y \stalk{X}{x} = \stalk{X}{x} \ot_{\stalk{Y}{y}} \stalk{Y}{y} / \m_y$.
Therefore, it suffices to prove the following algebraic fact.

\begin{prop}
Let $\varphi : A \to B$ be a local map of local rings with $A$ regular and $B$ Cohen-Macaulay and,
\[ \dim{(B \ot_A A / \m_A)} = \dim{B} - \dim{A} \]
Then $\varphi$ is flat.
\end{prop}

\begin{proof}
We prove this by induction on $\dim{A}$. Since $A$ is regular, if $\dim{A} = 0$ then $A$ is a field so flatness is trivial. Suppose this holds for dimension $n-1$. Choose some $f \in \m_A \setminus \m_A^2$ then $A / (f)$ is regular and $\dim{A/(f)} = \dim{A} - 1$. Furthermore, $\dim{B/ fB} \ge \dim{B} - 1$ by Krull's principal ideal theorem. However by the dimension inequality (Lemma A), 
\begin{align*}
\dim{B/fB} - \dim{A} + 1 & = \dim{B/fB} - \dim{A/(f)} \le \dim{(B/fB \ot_A A/\m_A)}
\\
& = \dim{(B \ot_A A/\m_A)} = \dim{B} - \dim{A} 
\end{align*}
and therefore $\dim{B/fB} \le \dim{B} - 1$ so we see that $\dim{B/fB} = \dim{B} - 1$ and,
\[ \dim{(B/fB \ot_A A/\m_A)} = \dim{B/fB} - \dim{A/(f)} \]
Because $B$ is Cohen-Macaulay and $\dim{B/fB} = \dim{B} - 1$, we find that $f$ is not a zero divisor so $B/fB$ is also Cohen-Macaulay (Lemma B). Therefore, the map $A/(f) \to B/(f)$ satisfies the hypotheses of our induction hypothesis and therefore is flat. By the slicing criterion for flatness $A \to B$ is flat since $f \in A$ is a non-zerodivisor.
\end{proof}

\begin{lemma}[A]
Let $A \to B$ be a map of local rings. Then,
\[ \dim{B/\m_A B} \ge \dim{B} - \dim{A} \]
\end{lemma}

\begin{proof}
Let $I = (x_1, \dots, x_d) \subset A$ be an ideal of definition generated by $d = \dim{A}$ elements and $J = (\bar{y}_1, \dots,\bar{y}_c) \subset B/\m_A B$ be an ideal of definition generated by $c = \dim{B/\m_A B}$ elements. Then consider, 
\[ I' = (x_1, \dots, x_d, y_1, \dots, y_c) B \] 
Since $I \supset \m_A^N$ we have $I' \supset \m_A^N B$. Since $J \supset \m_B^M / \m_A B$, we see that $I' + \m_A B \supset \m_B^M$. Therefore,
\[ I' \supset I' + \m_A^N B \supset (I' + \m_A B)^N \supset \m_B^{NM} \]
proving that $I'$ is an ideal of definition of $B$ so $\dim{B} \le d + c$ proving the claim.
\end{proof}

\begin{lemma}[B]
Let $A$ be a Cohen-Macaulay local ring and $x \in A$. Then $x$ is a non zero-divisor if and only if $\dim{A/(x)} = \dim{A} - 1$. Furthermore, if $x$ is a non zero-divisor then $A/(x)$ is Cohen-Macaulay if and only if $A$ is Cohen-Macaulay.
\end{lemma}

\begin{proof}
We always have $\dim{A/(x)} \ge \dim{A} - 1$ with equality if $x$ is a non zero-divisor by Krull's principal ideal theorem. Thus suppose that $\dim{A/(x)} = \dim{A} - 1$ so $x$ is not contained in any minimal prime. Because $A$ is Cohen-Macaulay, there are no embedded primes and therefore $x$ is not contained in an associated prime so $x$ is a non zero-divisor.
\bigskip\\
Furthermore, if $x$ is a non zero-divisor then we know,
\[ \depth{}{A/(x)} = \depth{}{A} - 1 \quad \text{ and } \quad \dim{A/(x)} = \dim{A} - 1 \]
so we see that,
\[ \depth{}{A/(x)} = \dim{A/(x)} \iff \depth{}{A} = \dim{A} \]
\end{proof}

\subsection{Section 11}


\subsubsection{11.1}

Let $X = \A^n_k = \Spec{k[x_1, \dots, x_n]}$ and $P$ be the origin corresponding to $\m = (x_1, \dots, x_n)$. Let $U = X \setminus P$ and $f : U \to X$ the inclusion. Because $X$ is affine by [III 8.5],
\[ R^{n-1} f_* \struct{U} = \wt{H^{n-1}(U, \struct{U})} \] 
We can compute $H^{n-1}(U, \struct{U}) \neq 0$ directly using \v{C}ech cohomology, it turns out to be exactly the same computation as the cohomology of the sum of the twists on $\P^{n-1}_k$. This is not a coincidence. There is a morphism $\pi : U \onto \P^{n-1}_k$ making $U$  the total space of $\struct{\P^{n-1}}(1)$ minus the zero section and therefore,
\[ \pi_* \struct{U} = \bigoplus_{m \in \Z} \struct{\P^{n-1}}(m) \]
Furthermore, the affine open cover $\{ D_+(x_i) \}$ pulls back to $\{ D(x_i) \}$ giving an affine cover of $U$ so $\pi$ is affine. Therefore,
\[ H^{n-1}(U, \struct{U}) = H^{n-1}(\P^{n-1}, \pi_* \struct{U}) = \bigoplus_{m \in \Z} H^{n-1}(\P^{n-1}_k, \struct{\P^{n-1}}(m)) \neq 0 \]

\subsubsection{11.2 (CHECK!!)}

Let $f : X \to Y$ be a projective morphism with finite fibers. Then the Stein factorization $f = g \circ f'$ where $g$ is finite and $f'$ has connected fibers. Therefore, it suffices to show that $f' : X \to Y'$ is an isomorphism where $f'$ is a projective morphism with finite and connected fibers and $f'_* \struct{X} = \struct{Y'}$. Then we conclude that $f : X \to Y$ is isomorphic to $g : \rSpec{Y}{f_* \struct{X}} \to Y$ with $f_* \struct{X}$ coherent.
\bigskip\\
Let $f : X \to Y$ be a projective morphism with finite fibers $f_* \struct{X} = \struct{Y}$ and thus connected fibers. Because $f$ is finite type, the fibers $X_y \to \Spec{\kappa(y)}$ are finite type and $\Spec{X_y}$ is finite so $X_y \to \Spec{\kappa(y)}$ is finite (a finite type $k$-algebra has finitely many points iff it has finite dimension to prove this quotient by a minimal prime and apply noetherian normalization). Therefore $X_y = \Spec{A}$ for $A$ an Artin $\kappa(y)$-algebra and thus $X_y$ is discrete. Since $X_y$ is connected, we conclude that $A$ is local and $X_y$ consists of a single point. Therefore, $f$ is injective. Furthermore, since $f$ is projective we conclude that it is closed. Suppose that $f$ is not surjective. Choose $y \notin f(X)$ we have $(f_* \struct{X})_y = \stalk{Y}{y} \neq 0$. Since $f$ is closed $U = f(X)^C$ is an open neighborhood of $y$ but $(f_* \struct{X})(U) = \struct{X}(f^{-1}(U)) =  0$ and thus $(f_* \struct{X})_y = 0$ giving a contradiction. Thus $f$ is bijective and closed and therefore a homeomorphism. Furthermore, $f^\# : \struct{Y} \to f_* \struct{X}$ is an isomorphism by assumption. Therefore $f$ is an isomorphism.

\subsubsection{11.3 CHECK!!}

Let $X$ be a normal, projective variety over an algebraically closed field $k$. Let $\mathfrak{d}$ be a linear system of effective Cartier divisors without base points, and assume that $\mathfrak{d}$ is not composite with a pencil, meaning that if $f : X \to \P^n_k$ is the morphism determined by $\mathfrak{d}$ then $\dim{f(X)} \ge 2$. We want to show tha every divisor in $\mathfrak{d}$ is connected.
\bigskip\\
The goal is equivalent to showing that the preimages under  $f : X \to \P^n_k$ of hyperplanes $H \subset \P^n_k$ are all connected. Since $f$ is projective (because $X$ is projective and 2 out of three property), we can apply Stein factorization to write $f = g \circ f'$ where $f' : X \to X'$ is a projective morphism with connected fibers and $g : X' \to \P^n_k$ is a finite map. 
\bigskip\\
First, I claim that $f' : X \to X'$ is surjective. This is a general fact about Stein factorization of a closed map and $f$ is closed because it is projective. Given $X \to Y$ the fiber of $\rSpec{Y}{f_* \struct{X}} \to Y$ over $y$ is $\Spec{(f_* \struct{X})_y}$ but because $f$ is closed $(f_* \struct{Y})_y \neq 0$ if and only if $y \in f(X)$ which is a closed set. Otherwise $U = f(X)^C$ is open and $y \in U$ and $(f_* \struct{X})(U) = \struct{X}(f^{-1}(U)) = 0$ because $f^{-1}(U) = \empty$. Therefore $f' : X \to X'$ is surjective. Now, I claim that $X'$ is an integral normal scheme. Over any affine open $U \subset \P^n_k$ we have by definition that, 
\[ X'_U = \Spec{(f_* \struct{X})(U)} = \Spec{\struct{X}(f^{-1}(U))} \]
and from the following lemma $\struct{X}(f^{-1}(U))$ is an integrally closed domain which proves that $X'$ is normal and integral because it has an open cover by affine normal integral schemes and is connected (and noetherian so integral is the same as connected and locally integral) because $X \to X'$ is surjective and $X$ is connected. 

\begin{lemma}
Let $X$ be an integral normal scheme and $U \subset X$ a nonempty open. Then $\struct{X}(U)$ is an integrally closed domain.
\end{lemma}

\begin{proof}
Since $X$ is integral we know that $R = \struct{X}(U)$ is a domain. Let $f = \frac{a}{b} \in \Frac{R}$ be integral over $R$. Write,
\[ f^n + a_{1} f^{n-1} + \cdots + a_n = 0 \] 
Choose an open affine $V \subset U$ then $b \neq 0$ so $b|_V \neq 0$ by injectivity of restriction on an integral scheme. Then we have,
\[ (f|_V)^n + a_{1} |_V (f|_V)^{n-1} + \cdots + a_n |_V = 0 \] 
Therefore, $f|_V$ is integral over $\struct{X}(V)$ and hence $f|_V \in \struct{X}(V)$. Therefore, by uniqueness in the fraction field, the local sections $f|_V$ glue to a global section $f' \in \struct{X}(U)$ and,
\[ b|_V f'|_V = a|_V \]
for each $V$ proves that $b f = a$ so we see that $f' = f$ in $\Frac{R}$ proving the claim. 
\end{proof}
\noindent
Furthermore, since $g : X' \to \P^n_k$ is finite it is affine and thus $X'$ is separated and finite type over $k$. Therefore, we have that $X'$ is a normal $k$-variety. Now, by [II, Ex. 5.7(a,e)] we know that along a finite map $g : X' \to \P^n$ the pullback of the ample line bundle $\struct{\P^n}(1)$ gives an ample line bundle $\L = g^* \struct{\P^n}(1)$ (surjectivity of the map in part (e) is only used for the other direction). Since $\dim{f(X)} \ge 2$ we see that $\dim{X'} \ge 2$ because $X' \to f(X)$ is surjective. Therefore, we can apply [III, Cor. 7.9] to conclude that the effective ample divisors on $X'$ are connected. However, any hyperplane $H \subset \P^n_k$ is an effective divisor for $\struct{\P^n_k}(1)$ and therefore $g^{-1}(H)$ is an effective ample divisor (it must be an effective Cartier divisor because $f^{-1}(H) \in \mathfrak{d}$ is effective Cartier and thus cut out by a regular section and $f' : X \to X'$ is a dominant map of integral schemes so if the section $g^* h \in \Gamma(X', \L)$ cutting out $g^{-1}(H)$ is not regular then $f^* h = f'^* g^* h$ could not be a regular section since it would be zero at the generic point) for the ample line bundle $\L = g^* \struct{\P^n_k}(1)$ and thus $g^{-1}(H)$ is connected. 

Because $f' : X \to X'$ is projective it is closed has connected fibers and is surjective then we can apply the following lemma to the divisor $D \subset X'$ which we showed is connected to conclude that $f^{-1}(H) = f'^{-1}(D)$ is connected.

\begin{lemma}
Let $f : X \to Y$ be a surjective closed map of topological spaces with connected fibers and $C \subset Y$ is a closed connected subspace. Then $f^{-1}(C)$ is connected.
\end{lemma}

\begin{proof}
By considering $f : f^{-1}(C) \to C$ (which is clearly surjective with connected fibers and is closed because $C$ is closed) it suffices to consider the case that $C = Y$ is connected. 
Suppose that $C_1, C_2 \subset X$ are closed subsets such that $C_1 \cup C_2 = X$ and neither are empty. We want to show they intersect nontrivially. By surjectivity,
\[ f(C_1) \cup f(C_2) = f(C_1 \cup C_2) = f(X) = Y \]
and $f(C_i)$ are closed because $f$ is closed. Since neither are empty, by the connectedness of $Y$ we see that there is some $y \in f(C_1) \cap f(C_2)$. Therefore, $C_i \cap f^{-1}(y)$ are both nontrivial. Since $f^{-1}(y)$ is connected and $C_i \cap f^{-1}(y)$ are closed nonempty and cover $f^{-1}(y)$ there must exist,
\[ x \in (C_1 \cap f^{-1}(y)) \cap (C_2 \cap f^{-1}(y)) = (C_1 \cap C_2) \cap f^{-1}(y) \]
so $x \in C_1 \cap C_2$ proving the claim.
\end{proof}

\subsubsection{11.4 DO}

Let $T$ be an irreducible curve of finite type over $k$ and $f : X \to T$ a flat family of subschemes of $\P^n_k$ meaning there is a closed immersion $X \embed \P^n_k \times_k T$ over $T$. Supppose that $U \subset T$ is a nonempty open such that for each closed point $t \in U$ the fiber $X_t \subset \P^n_{\kappa(t)}$ is connected.
Let $\nu : \tilde{T} \to T$ be the normalization and consider the base change,
\begin{center}
\begin{tikzcd}
\wt{X} \arrow[r] \arrow[d,"\tilde{f}"'] & X \arrow[d, "f"]
\\
\wt{T} \arrow[r, "\nu"] & T
\end{tikzcd}
\end{center}
Since $T$ has finitely many singular points, we can assume that $U$ is contained in the nonsingular locus. Now we apply Stein factorization to $\tilde{f}$ to get $\tilde{f} : g \circ f'$ where $f' : \wt{X} \to \wt{T}'$ and $g : \wt{T}' \to \wt{T}$ and
\[ \wt{T}' = \rSpec{\wt{T}}{\tilde{f}_* \struct{\wt{X}}} \]
Now, by base change, $\tilde{f}$ is flat and proper and therefore is open and closed but $\wt{T}$ is irreducible so $\tilde{f}$ is surjective and $\wt{T}$ is integral so $\struct{\wt{T}} \embed \wt{f}_* \struct{\wt{X}}$ (this is because $\stalk{\wt{T}}{y} \to (f_* \struct{\wt{X}})_y \to \stalk{\wt{X}}{x}$ is injective by \href{https://stacks.math.columbia.edu/tag/01RI}{Tag 01RI}) (DO THIS OUT). However, $\wt{T}$ is a nonsingular curve so its local rings are DVRs and torsion-free modules over a DVR are flat meaning that $f_* \struct{\wt{X}}$ is a flat $\struct{\wt{T}}$-module. Therefore, $g$ is flat and finite meaning that it has constant degree. Suppose that $\deg{g} > 1$ then there are infinitely many split points of $\wt{T}$ meaning points where the fiber of $g$ is disconnected. Since $g$ has connected fibers over the dense open $U \subset \wt{T}$ (using that $U$ is contained in the nonsingular locus) which has finite complement we see that $\deg{g} = 1$. Thus, $g$ is an isomorphism and $f'$ has connected fibers by Stein factorization so $\wt{f} = g \circ f'$ has connected fibers. However, since $\nu : \wt{T} \to T$ is surjective we see that $\wt{X} \to X$ is surjective and thus $\wt{X}_t \to X_t$ is surjective so since $\wt{X}_t$ is connected then the image of a connected set is connected and therefore $X_t$ is connected for all $t \in T$.


\subsubsection{11.5 (CHECK)}

Let $Y \subset X = \P^N_k$ be a hypersurface with $N \ge 4$. Then the sheaf of ideals for $Y$ is $\I \cong \struct{X}(-d)$ for some integer $d \ge 1$. Let $\hat{X}$ be the formal completion of $X$ along $Y$. By [II, Ex. 9.6] we see that,
\[ \Pic{\hat{X}} = \varprojlim \Pic{X_n} \]
where $X_n = (Y, \struct{X} / \I^n)$ is the formal neighborhood. Now we consider the maps $\Pic{X_{n+1}} \to \Pic{X_n}$. By [III, Ex. 4.6] there is an exact sequence,
\begin{center}
\begin{tikzcd}
\cdots \arrow[r] & H^1(X_{n+1}, \I^n / \I^{n+1}) \arrow[r] & \Pic{X_{n+1}} \arrow[r] & \Pic{X_n} \arrow[r] & H^2(X_{n+1}, \I^n / \I^{n+1}) \arrow[r] & \cdots
\end{tikzcd}
\end{center}
First notice that $\I^n \cong \I^{\ot n} \cong \struct{X}(-nd)$ because $\I$ is locally-free. Furthermore, $\I^n / \I^{n+1} = \iota^* \I^n$ where $\iota : Y \embed X$ is the closed immersion. Therefore $\I^n / \I^{n+1} \cong \struct{Y}(-nd)$ as $\struct{Y}$-modules. Furthermore, because cohomology only depends on the underlying abelian sheaf and topological space we see that,
\[ H^i(X_{n+1}, \I^n / \I^{n+1}) = H^i(Y, \I^n / \I^{n+1}) \cong H^i(Y, \struct{Y}(-nd)) \]
Now applying [III, Ex. 5.5(c)] to the complete intersection $Y \subset X$ of dimension $N-1$ we see that,
\[  H^i(X_{n+1}, \I^n / \I^{n+1}) \cong H^i(Y, \struct{Y}(-nd)) = 0 \]
for all $n$ and all $0 < i < N - 1$ so in particular for $i = 1,2$ since $N \ge 4$. Therefore, by the above exact sequence,
\[ \Pic{X_{n+1}} \iso \Pic{X_n} \]
is an isomorphism for each $n$ and therefore we find,
\[ \Pic{\hat{X}} \iso \varprojlim_n \Pic{X_n} \iso \Pic{Y} \]
because we see that the canonical maps $\Pic{X_n} \to \Pic{Y}$ are all isomorphisms compatibly with the restrictions. 

\subsubsection{11.6 DO!!}

Let $Y \subset X = \P^N_k$ be a hypersurface with $N \ge 2$. 

\begin{enumerate}
\item Suppose that $\F$ is a locally-free sheaf on $X$ and consider the natural map,
\[ H^0(X, \F) \to H^0(\hat{X}, \hat{\F}) \]
By construction the inverse limit sheaf has sections given by the inverse limits of the sections and therefore,
\[ H^0(\hat{X}, \hat{\F}) = H^0(\hat{X}, \varprojlim \F \ot_{\struct{X}} \struct{X_n}) = \varprojlim H^0(X_n, \F \ot_{\struct{X}} \struct{X_n}) \]
Consider the exact sequence,
\begin{center}
\begin{tikzcd}
0 \arrow[r] & \I^n \arrow[r] & \struct{X} \arrow[r] & \struct{X_n} \arrow[r] & 0
\end{tikzcd}
\end{center}
and we have seen that $\I^n \cong \struct{X}(-nd)$. Since $\F$ is locally-free and hence flat we get an exact sequence,
\begin{center}
\begin{tikzcd}
0 \arrow[r] & \F(-nd) \arrow[r] & \F \arrow[r] & \F |_{X_n} \arrow[r] & 0 
\end{tikzcd}
\end{center}
and therefore we get an exact sequence,
\begin{center}
\begin{tikzcd}
0 \arrow[r] & H^0(X, \F(-nd)) \arrow[r] & H^0(X, \F) \arrow[r] & H^0(X_n, \F |_{X_n}) \arrow[r] & H^1(X, \F(-nd)) 
\end{tikzcd}
\end{center}
Furthermore, by [III, Theorem 7.6(b)] we see that for any locally free $\F$ on $X$ we have $H^i(X, \F(-q)) = 0$ for $i < N$ and $q \gg 0$ in particular for $i = 0,1$ since $N \ge 2$. Therefore, taking $n$ large enough in the above exact sequnece (since $d > 0$) we see that $H^0(X, \F(-nd)) = H^1(X, \F(-nd)) = 0$ and therefore, 
\[ H^0(X, \F) \iso H^0(X_n, \F |_{X_n}) \]
for all $n \gg 0$. This implies that,
\[ H^0(X, \F) \iso H^0(\hat{X}, \hat{\F}) = \varprojlim H^0(X_n, \F \ot_{\struct{X}} \struct{X_n}) \]
is an isomorphism because it is an isomorphism to each factor for all sufficiently large $n \gg 0$.

\item Suppose that all locally-free sheaves on $\hat{X}$ are algebraizable. Then for each locally-free sheaf $\fF$ on $\hat{X}$ there is ome coherent sheaf $\F$ on $X$ such that $\fF \cong \hat{\F}$. Then,
\[ \fF(n) \cong \hat{\F}(n) \]
and for $n \ge n_0$ the sheaf $\F(n)$ is globally generated because $\struct{X}(1)$ is ample which implies that $\fF(n)$ is globally generated because $\hat{\F} = \F \ot_{\struct{X}} \struct{\hat{X}}$ and tensor product is right exact. Conversely, suppose that for each locally-free sheaf $\fF$ on $\hat{X}$ there is some $n_0$ such that $\fF(n)$ is globally generated for all $n \ge n_0$. Because $\fF(n)$ is globally generated an coherent there is a surjection,
\[ \struct{\hat{X}}^N \onto \fF(n) \]
and therefore we get a surjection,
\[ \hat{\E}_0 = \struct{\hat{X}}^N(-n) \onto \fF \]
Then $\fF' = \ker{(\hat{\E}_0 \onto \fF)}$ is also coherent (WHY LOCALLY FREEEE!!!) so applying the same argument to $\fF'$ gives an exact sequence,
\begin{center}
\begin{tikzcd}
\hat{\E}_1 \arrow[r] & \hat{\E}_0 \arrow[r] & \fF \arrow[r] & 0 
\end{tikzcd}
\end{center}
where $\E_i$ is a direct sum of sheaves of the form $\struct{X}(-n_i)$. Now we apply (a) to the locally free sheaf $\shHom{\struct{X}}{\E_1}{\E_0}$ to see that,
\[ \Hom{\struct{X}}{\E_1}{\E_0} \iso \Hom{\struct{\hat{X}}}{\hat{\E}_1}{\hat{\E}_0} \]
is an isomorphism (where tensor product i.e. completion and sheaf hom commute for locally free sheaves). Therefore the above map $\hat{\E}_1 \to \hat{\E}_0$ arises from some map $\varphi : \E_1 \to \E_0$ of $\struct{X}$-modules. Then let $\F = \coker{\varphi}$. By [II, Cor. 9.8] the functor $\F \mapsto \hat{\F}$ is exact and therefore we see that,
\[ \fF = \coker{\hat{\varphi}} = \hat{\F} \]
because we have the exact sequence,
\begin{center}
\begin{tikzcd}
\hat{\E}_1 \arrow[r] & \hat{\E}_0 \arrow[r] & \hat{\F} \arrow[r] & 0
\end{tikzcd}
\end{center}
proving that $\fF$ is algebraizable. 

\item Now suppose that the equivalent conditions of (b) hold. Consider the map $\Pic{X} \to \Pic{\hat{X}}$. The restriction map $\Pic{X} \to \Pic{Y}$ factors as $\Pic{X} \to \Pic{\hat{X}} \to \Pic{Y}$. Since $\Pic{X} \to \Pic{Y}$ is injective because $\Pic{X} = \Z$ and $\struct{X}(1)$ restricts on each curve in $Y$ to a line bundle of positive degree (hence no power of it is trivial) thud $\Pic{X} \to \Pic{\hat{X}}$ is injective. 
\bigskip\\
Furthermore, for each $\fL \in \Pic{\hat{X}}$ there exists some coherent $\struct{X}$-module $\F$ such that $\fL = \hat{\F}$. Now I use the following result.

\begin{prop}
Let $X$ be a noetherian scheme and $Y \subset X$ a closed subscheme. Let $\hat{X}$ be the completion of $X$ along $Y$. Suppose that $\F$ is a coheret $\struct{X}$-module such that $\hat{\F}$ is locally free of rank $r$. Then there exists an open $V \subset X$ neighborhood of $Y$ meaning $V \supset Y$ such that $\F|_V$ is locally free of rank $r$.
\end{prop}

\begin{proof}
Because $\hat{\F}$ is locally free, there is an open cover by opens $\fU \subset \hat{X}$ trivializing $\hat{\F}$. Because $Y \embed X$ is a homeomorphism onto its image (with the subspace topology) each open of $Y$ is of the form $U \cap Y$ for some open $U \subset Y$. Therefore, we can cover $X$ by affine opens $U$ such that either $U \cap Y = \empty$ or $\hat{\F} |_{U \cap \hat{X}}$ is trivial. Then it suffices to prove the claim for each of these affine opens $U$ there is an principal affine $V_U \subset V$ with $V_U \supset U \cap Y$ such that $\F |_{V_U}$ is finite free because then,
\[ V = \bigcup_{U} V_U \]
satisfies $V \supset Y$ and $\F |_Y$ is locally free. 
\bigskip\\
Let $U = \Spec{A}$ be an affine open of $X$ trivializing $\hat{F}$ then $Y \cap V = V(I)$ for some ideal $I \subset A$. Then $\F = \wt{M}$ for some finite $A$-module $M$. Then $\hat{\F} = M^\Delta$ is trivial. Therefore, we get an isomorphism $\varphi : \hat{A}^r \iso \hat{M}$ defined by sections $s_1, \dots, s_r \in \hat{M}$. In particular, the reduction $\bar{\varphi} : A / I \to M / I M$ is an isomorphism. Lift these sections $\bar{s}_1, \dots, \bar{s}_r \in M / I M$ to some elements $m_1, \dots, m_r \in M$. I claim that $\bar{m}_1, \dots, \bar{m}_r \in M / I^n M$ define an isomorphism $A / I^n \to M / I^n M$ (we proved this in [II, Ex. 9.6(c)]). Therefore $m_1, \dots, m_r$ define an isomorphism $\hat{A} \to \hat{M}$ which is the completion of a map $\psi : A \to M$. 
\bigskip\\
Take the kernel-cokernel exact sequence for $\psi$,
\begin{center}
\begin{tikzcd}
0 \arrow[r] & K \arrow[r] & A^r \arrow[r, "\psi"] & M \arrow[r] & C \arrow[r] & 0 
\end{tikzcd}
\end{center} 
where $C$ is automatically finitely generated and $K$ is finitely generated because $A$ is noetherian. Then by [II, Theorem. 9.3A] or [II, Cor. 9.8] we get an exact sequence, 
\begin{center}
\begin{tikzcd}
0 \arrow[r] & \hat{K} \arrow[r] & \hat{A}^r \arrow[r] & \hat{M} \arrow[r] & \hat{C} \arrow[r] & 0 
\end{tikzcd}
\end{center} 
but $\hat{A}^r \iso \hat{M}$ is an isomorphism so $\hat{K} = 0$ and $\hat{C} = 0$. Then $K / IK \cong \hat{K} / I \hat{K} = 0$ so $I K = K$ and likewise $C = IC$. Since $K$ and $C$ are finite $A$-modules, by Nakayama there is some $g \in A$ with $g - 1 \in I$ such that $g K = 0$ and $g C = 0$ (take the product of the $g$ given by Nakayama for each of $K$ and $C$). Therefore $K_g = 0$ and $C_g = 0$ so localizing at $g$ we get an isomorphism $A^r_g \iso M_g$. Therefore $\F |_{D(g)} \cong \wt{M_g}$ is finite free. Furthermore, if $\p \in V(I)$ then $\p \supset I$ so we cannot have $g \in \p$ because $g - 1 \in I \subset \p$ so $g \notin \p$ meaning that $\p \in D(g)$ showing that $D(g) \supset V(I) = U \cap Y$ proving the claim.
\end{proof}

\noindent
Furthermore, $X \setminus V$ consists of a finite set of points. Indeed, if $X \setminus V$ contains a curve $C$ then $\struct{X}(Y) |_C \cong \struct{C}$ because the sheaf of ideals $\struct{X}(-Y) \embed \struct{X}$ consists of everything on $U = X \setminus Y$ but $V \subset Y$ so $C \subset X \setminus V \subset U$. However, $\struct{X}(Y) \cong \struct{X}(d)$ is very ample since $d > 0$ and $\struct{C}$ is not ample because $H^0(C, \struct{C}^{\ot n}) = k$ so $\struct{C}^{\ot n}$ is not globally generated for any $n$. Therefore, every curve $C$ intersects $Y$ proving that $V$ is the complement of a finite set of points (finite because the complement it closed). In particular, since $N \ge 2$ we see that $\codim{X \setminus V, X} \ge 2$ and therefore $\Pic{X} \to \Pic{V}$ is an isomorphism (because $X$ is locally factorial and applying [II, Prop. 2.6.5]) so there is a unique line bundle $\L \in \Pic{X}$ such that $\L|_V \cong \F|_V$. Since $Y \subset V$ we see that,
\[ \hat{\L} \cong \hat{\F} \cong \fF \]
proving that $\Pic{X} \to \Pic{\hat{X}}$ is surjective. 
\end{enumerate}

\subsubsection{11.7}

Let $Y \subset X = \P^2_k$ be a curve (I assume meaning a hypersurface). 

\begin{enumerate}
\item Using the notation of Ex. 11.5, consider the maps,
\[ \Pic{\hat{X}} \iso \varprojlim_n \Pic{X_n} \to \Pic{Y} \]
As before, there is an exact sequence,
\begin{center}
\begin{tikzcd}
H^1(X_{n+1}, \I^n / \I^{n+1}) \arrow[r] & \Pic{X_{n+1}} \arrow[r] & \Pic{X_n} \arrow[r] & H^2(X_{n+1}, \I^n / \I^{n+1})
\end{tikzcd}
\end{center}
However,
\[ H^2(X_{n+1}, \I^n / \I^{n+1}) = H^2(Y, \I^n / \I^{n+1}) = 0 \]
because $\dim{Y} = 1$. Therefore, each $\Pic{X_{n+1}} \onto \Pic{X_n}$ is surjecitve which implies that the map $\Pic{\hat{X}} \onto \Pic{Y}$ is surjective. Now we consider the kernel.  
\bigskip\\
Extending the above exact sequence,
\begin{center}
\begin{tikzcd}[column sep = small]
H^0(X_{n+1}, \struct{X_{n+1}}^\times) \arrow[r] & H^0(X_n, \struct{X_n}^\times) \arrow[r] & H^1(X_{n+1}, \I^n / \I^{n+1}) \arrow[r] & \Pic{X_{n+1}} \arrow[r] & \Pic{X_n} \arrow[r] & 0
\end{tikzcd}
\end{center}
However, each $X_n$ is a complete intersection (cut out by a power of the defining equation) and therefore $H^0(X_n, \struct{X_n}) = k$ for all $n$ meaning that $H^0(X_{n+1}, \struct{X_{n+1}}^\times) \to H^0(X_n, \struct{X_n}^\times)$ is an isomorphism and therefore we have an exact sequence,
\begin{center}
\begin{tikzcd}
0 \arrow[r] & H^1(X_{n+1}, \I^n / \I^{n+1}) \arrow[r] & \Pic{X_{n+1}} \arrow[r] & \Pic{X_n} \arrow[r] & 0
\end{tikzcd}
\end{center}
Furthermore, by the previous calculation,
\[ H^1(X_{n+1}, \I^n / \I^{n+1}) \cong H^1(Y, \struct{Y}(-nd)) \]
Since $d > 0$ the line bundle $\struct{Y}(-nd)$ which has degree $-nd^2$ has no global sections
and therefore by Riemann-Roch,
\[ \dim_k H^1(X_{n+1}, \I^n / \I^{n+1}) = nd^2 + g_a(Y) - 1 \] 
Since these all have positive (and linearly growing!) dimension we conclude that
\[ \ker{(\Pic{\hat{X}} \to \Pic{Y})} \]
is an infinite-dimensional $k$-vector space.

\item Choose nontrivial $\fL \in \ker{(\Pic{\hat{X}} \to \Pic{Y})}$. Then suppose that $\fL$ is algebraizable meaning $\fL \cong \hat{\L}$ for some $\L \in \Pic{X}$. However, then $\fL \cong \L \ot_{\struct{Y}} \struct{\hat{X}}$ and therefore $\fL \mapsto \fL \ot_{\struct{\hat{X}}} \struct{Y} \cong \L$ under $\Pic{\hat{X}} \to \Pic{Y}$ meaning that $\L \cong \struct{X}$ but we have $\fL \not\cong \struct{\hat{X}}$ by hypothesis contradicting the assumption that $\fL$ is algebraizable. 

\item By the previous problem [III, Ex. 11.6(c)] we see that if every coherent $\struct{\fX}$-module were algebraizable then we would have $\Pic{X} \to \Pic{\hat{X}}$ an isomorphism. However, $\Pic{X} \to \Pic{Y}$ is injective so factoring $\Pic{X} \to \Pic{\hat{X}} \to \Pic{Y}$ we would have $\Pic{\hat{X}} \to \Pic{Y}$ injective which we proved that it is not.  
\end{enumerate}

\subsubsection{11.8}

Here we assume that all schemes are locally noetherian (because this is the generality proved in Hartshorne for the theorem on formal functions).
\bigskip\\
Let $f : X \to Y$ be a projective morphism of locally noetherian schemes. Let $\F$ be a coherent sheaf on $X$ which is flat over $Y$ and suppose that $H^i(X_y, \F_y) = 0$ for some fixed $i$ and $y \in Y$.
\bigskip\\
First, I claim that $H^i(X_n, \F_n) = 0$ for all $n \ge 0$. I prove this by induction. The base case is given. Now we assume towards induction that $H^i(X_n, \F_n) = 0$. Then pulling back along the morphism $\Spec{\stalk{Y}{y} / \m_y^{n+1}} \to Y$ gives the thickened fibers $X_n \to \Spec{\stalk{Y}{y} / \m_y^{n+1}}$. Let $\iota : X_y \embed X$ be the inclusion of the fiber then $\struct{X_n} = \iota^{-1} \struct{X} \ot_{\struct{Y}} (\stalk{Y}{y} / \m_y^n)$.
From the exact sequence,
\begin{center}
\begin{tikzcd}
0 \arrow[r] & \m_y^n / \m_y^{n+1} \arrow[r] & \stalk{Y}{y} / \m_y^{n+1} \arrow[r] & \stalk{Y}{y} / \m_y^n \arrow[r] & 0
\end{tikzcd}
\end{center}
we get an exact sequence of sheaves,
\begin{center}
\begin{tikzcd}
0 \arrow[r] & \iota^{-1} \F \ot_{\struct{Y}} (\m_y^n / \m_y^{n+1}) \arrow[r] & \iota^{-1} \F \ot_{\struct{Y}} (\stalk{Y}{y} / \m_y^{n+1}) \arrow[r] & \iota^{-1} \F \ot_{\struct{Y}} (\stalk{Y}{y} / \m_y^n) \arrow[r] & 0
\end{tikzcd}
\end{center}
because $\F$ is flat over $Y$ and therefore $\iota^{-1} \F$ is a flat $\struct{Y}$-module. However, 
\[ \F_n := \iota_n^* \F = \iota^{-1} \F \ot_{\iota^{-1} \struct{X}} \struct{X_n} = \iota^{-1} \F \ot_{\iota^{-1} \struct{X}} (\iota^{-1} \struct{X} \ot_{\struct{Y}} (\stalk{Y}{y} / \m_y^n)) = \iota^{-1} \F \ot_{\struct{Y}} (\stalk{Y}{y} / \m_y^n) \]
and therefore we get an exact sequence,
\begin{center}
\begin{tikzcd}
0 \arrow[r] & \F_1 \ot_{\struct{X_y}} (\m_y^n / \m_y^{n+1}) \arrow[r] & \F_{n+1} \arrow[r] & \F_n \arrow[r] & 0
\end{tikzcd}
\end{center}
since $\m_y^n / \m_y^{n+1}$ is a $\kappa(y)$-module so $\F_y \ot_{\struct{X_y}} (\m_y^n / \m_y^{n+1}) \cong \F_y^{\oplus d_n}$ where $d_n = \dim_{\kappa(y)} (\m_y^n / \m_y^{n+1})$. Therefore, the exact sequence,
\begin{center}
\begin{tikzcd}
0 \arrow[r] & \F_y^{\oplus d_n} \arrow[r] & \F_{n+1} \arrow[r] & \F_n \arrow[r] & 0
\end{tikzcd}
\end{center}
gives in cohomology,
\begin{center}
\begin{tikzcd}
H^i(X_y, \F_y)^{\oplus d_n} \arrow[r] & H^i(X_{n+1}, \F_{n+1}) \arrow[r] & H^i(X_n, \F_n)
\end{tikzcd}
\end{center}
but by the induction hypothesis $H^i(X_n, \F_n) = 0$ and by the base case $H^i(X_y, \F_y) = 0$. Therefore, $H^i(X_{n+1}, \F_{n+1}) = 0$ proving that $H^i(X_n, \F_n) = 0$ for all $n$ by induction.
\bigskip\\
Therefore, by the theorem on formal functions,
\[ \widehat{R^i f_*(\F)_y} \iso \varprojlim H^i(X_n, \F_n) = 0 \]
Therefore, the completion of $R^i f_*(\F)_y$ is zero and $R^i f_*(\F)_y$ is a finite $\stalk{Y}{y}$-module since $R^i f_*(\F)$ is coherent by projective pushforward.
\bigskip\\
Now I claim that if $M$ is a finite $A$-module where $A$ is a local ring such that $\hat{M} = 0$ then $M = 0$. First I claim that $\hat{A} \to A / \m^n$ is surjective. Indeed, $A \to \hat{A} \to A / \m^n$ is the quotient map $A \to A / \m^n$ which is surjective by definition. Therefore,
\[ \hat{M} = M \ot_A \hat{A} \onto M \ot_A A / \m \]
is surjective so $M \ot_A (A / \m) = 0$ and thus $\m M = M$ so by Nakayama $M = 0$.
\bigskip\\
Therefore $R^i f_* (\F)_y = 0$. Because $\G = R^i f_*(\F)$ is coherent, there exists an open neighborhood $U$ of $y$ such that $R^i f_*(\F)|_U = 0$. Shrink to an affine open such that $\G = \wt{M}$ for a finite $A$-module $M$. Then $\G_y = 0$ so $M_\p = 0$ for $\p \in \Spec{A}$ the corresponding prime. Therefore, given a generating set $t_1, \dots, t_n \in M$ there is $s_1, \dots, t_n \in A \setminus \p$ such that $s_i t_i = 0$ in $M$ and thus letting $s = s_1 \dots s_n$ we have $M_s = 0$ and thus $\wt{M}|_{D(s)} = 0$ and $\p \in D(s)$ so we see that $\G|_U = 0$ where $U = D(s)$. 

\subsection{Section 12}

\subsubsection{12.1}

Let $X$ be a scheme of finite type over $k$ with $k$ algebraically closed. Then the closed points of $X$ coincide with $X(k)$ at which points $x \in X$ we have $k \to \stalk{X}{x} \to \kappa(x)$ is an isomorphism. Therefore, by [II, Prop. 8.6] at closed points,
\[ \m_x / \m_x^2 \cong \Omega_{\stalk{X}{x}/k} \otimes_{\stalk{X}{x}} k = (\Omega_X)_x \otimes_{\stalk{X}{x}} \kappa(x) \]
showing that,
\[ \varphi(x) = \dim_k \m_x / \m_x^2 = \dim_k(\Omega_X)_x \otimes_{\stalk{X}{x}} \kappa(x) = \rank_x{(\Omega_X)} \]
Furthermore, since $X$ is finite type over $k$ we know that $\Omega_X$ is coherent and thus the rank function is upper-semicontinuous so $\varphi(x)$ is upper-semicontinuous on the closed points.

\subsubsection{12.2}

Consider $X \subset \P^n_T$ a family of hypersurfaces of constant degree $d$ i.e. $\pi : X \to T$ is flat and for each $t \in T$ the fiber $X_t$ is a hypersurface of degree $d$ in $\P^n_{\kappa(t)}$. Now, a hypersurface $H \subset \P^n_k$ of degree $d$ is Cartier giving an exact sequence,
\begin{center}
\begin{tikzcd}
0 \arrow[r] & \struct{\P^n}(-d) \arrow[r] & \struct{\P^n} \arrow[r] & \struct{H} \arrow[r] & 0
\end{tikzcd}
\end{center}
taking cohomology gives exact sequences,
\begin{center}
\begin{tikzcd}
0 \arrow[r] & k \arrow[r] & H^0(H, \struct{H}) \arrow[r] & 0
\\
0 \arrow[r] & H^i(H, \struct{H}) \arrow[r] & H^{i+1}(\P^n, \struct{\P^n}(-d)) \arrow[r] & 0
\end{tikzcd}
\end{center}
Therefore,
\[ \dim_k H^{q}(H, \struct{H}) = \begin{cases}
1 & q = 0
\\
{d - 1 \choose n } & q = n-1
\\
0 & \text{else}
\end{cases} \] 
this follows independently on the choice of hypersurface and thus,
\[ \dim_k H^i(X_t, \struct{X_t}) \]
is constant in $t \in T$.

\subsubsection{12.3 (CHECK THIS WITH D!!!)!!}

Let $X_1 \subset \P^4_k$ be the \textit{rational normal quartic curve} meaning the $4$-upple embedding $\P^1_k \embed \P^4_k$. Let $X_0 \subset \P^3_k$ be a nonsingular rational quartic curve. We produe a flat family $\{ X_t \}$ of curves in $\P^4$ parametrized by $T = \A^1$ following the method of [III, 9.8.3] with fibers $X_1$ and $X_0$ for $t = 1$ and $t = 0$. This is the family given in [III, Ex. 9.5(a)] defined by
\[ \P^1_{\A^1} \to \P^4_{\A^1} \]
via $[s:t] \mapsto [s^4 : s^3 t : a s^2 t^2 : s t^3 : t^4]$. Let $X_a \subset \P^4$ be the image of this map. For $a = 1$ we know $\P^1 \embed \P^4$ is a rational normal curve so,
\[ H^0(\P^4, \struct{\P^4}(d)) \onto H^0(X_1, \struct{X_1}(d)) \]
is surjective for all $d$. Furthermore, $\P^1 \to \P^4$ has degree $4$ so we have,
\[ (S_1/I_1)_d =  H^0(X_1, \struct{X_1}(d)) = H^0(\P^1, \struct{\P^1}(4d)) \]
so we get,
\[ \dim_{k}(S_1/I_1)_d = \dim_{k} H^0(\P^1, \struct{\P^1}(4d)) = 4d + 1 \]
However, $X_0 \subset \P^4$ is the non-projectively normal twisted quadric rational curve $\P^1 \embed \P^3 \subset \P^4$. Since the ideal $I_0$ is the ideal of relations for the functions $s^4, s^3 t, s t^3, t^4$ we have an isomorphism,
\[ S_0/I_0 = k[x_0, x_1, x_3, x_4]/I' \cong k[s^4, s^3 t, s t^3, t^4] \] 
where we give $s,t$ degree $\frac{1}{4}$ (i.e. view it as a subring of $(k[s,t])^{(4)}$) to make this a graded isomorphism. We see that the ideals $I_a$ are homogeneous and prime since they are the kernel of a map of graded domains (the quotient of such a kernel is a subring of a domain and hence a domain) and therefore saturated (if $\p$ is a prime ideal not containing the irrelevant ideal and $x_i^{n} f \in \p$ then either $x_i \in \p$ for each $i$ or $f \in \p$). Therefore,
\[ \dim_{k} (S_0 / I_0)_d = \dim_k (k[s^4, s^3 t, s t^3, t^4])_d = 
\begin{cases}
1 & d = 0
\\
4 & d = 1
\\
4 d + 1 & d > 1
\end{cases} \]
so indeed we see that $S_d \to H^0(X_0, \struct{X_0}(d))$ is not surjective exactly for $d = 1$. Notice we can make the same argument for $X_1$ since,
\[ S_1 / I_1 \cong k[s^4, s^3 t, s^2 t^2, s t^3, t^4] = (k[s,t])^{(4)} \]
where we give $s,t$ degree $\frac{1}{4}$ (i.e. view it as a subring of $(k[s,t])^{(4)}$) to make this a graded isomorphism. Therefore,
\[ \dim_{k} (S_0 / I_0)_d = \dim_k (k[s^4, s^3 t, s t^3, t^4])_d = 
\begin{cases}
1 & d = 0
\\
4 & d = 1
\\
4 d + 1 & d > 1
\end{cases} \]
so indeed we see that $S_d \to H^0(X_0, \struct{X_0}(d))$ is not surjective exactly for $d = 1$. Notice we can make the same argument for $X_1$ since,
\[ S_1 / I_1 \cong k[s^4, s^3 t, s^2 t^2, s t^3, t^4] = (k[s,t])^{(4)} \]
and therefore we immediately see that,
\[ \dim_k (S_1 / I_1)_d = (k[s,t])^{(4)}_d = 4d + 1 \]
Notice that the Hilbert polynomials agree because these formulae are equal for $d \gg 0$. In fact, it is easy to check that $X_a$ for $a \neq 0$ has the same Hilbert function as $X_1$ (they are related by a diagonal $\mathrm{PGL}_5$ transformation) and therefore the family is flat by [III, Thm. 9.9].
\bigskip\\
Let $P = \P^4_{\A^1}$ and $\I = \I_X \subset \struct{P}$ be the sheaf of ideals of the family $X \subset P$. Therefore, we have an exact sequence of sheaves,
\begin{center}
\begin{tikzcd}
0 \arrow[r] & \I \arrow[r] & \struct{P} \arrow[r] & \struct{X} \arrow[r] & 0
\end{tikzcd}
\end{center}
and because $\struct{P}$ and $\struct{X}$ are flat over $\A^1$ because $P \to \A^1$ and $X \to \A^1$ are flat we see that $\I$ is flat over $\A^1$. Therefore, for each $t \in \A^1$ there is an exact sequence,
\begin{center}
\begin{tikzcd}
0 \arrow[r] & \I_t \arrow[r] & \struct{\P^4_{\kappa(t)}} \arrow[r] & \struct{X_t} \arrow[r] & 0
\end{tikzcd}
\end{center}
using that $\struct{X}$ is flat over $\A^1$. Taking the long exact sequence of cohomology,
\begin{center}
\begin{tikzcd}
0 \arrow[r] & H^0(\P^4, \I_t) \arrow[r] & H^0(\P^4, \struct{\P^4}) \arrow[r] & H^0(X_t, \struct{X_t}) \arrow[r] & H^1(\P^4, \I_t) \arrow[r] & H^1(\P^4, \struct{\P^4})
\end{tikzcd}
\end{center}
but $H^1(\P^4, \struct{\P^4}) = 0$ and $H^0(\P^4_{\kappa(t)}, \struct{\P^4}) = \kappa(t)$ and because $X_t$ is connected for each $t$ we have $H^0(X_t, \struct{X_t}) = \kappa(t)$ also. Therefore,
\begin{center}
\begin{tikzcd}
0 \arrow[r] & H^0(\P^4, \I_t) \arrow[r] & \kappa(t) \arrow[r] & \kappa(t) \arrow[r] & H^1(\P^4, \I_t) \arrow[r] & 0
\end{tikzcd}
\end{center}
However, $\kappa(t) \to \kappa(t)$ is a $\kappa(t)$-algebra map and thus an isomorphism. Therefore,
\[ H^0(\P^4, \I_t) = H^1(\P^4, \I_t) = 0 \]
for all $t \in \A^1$ so I think Hartshorne is wrong here.
\bigskip\\
Instead, let's look at $\I(d)$ on $\P^4_{\A^1}$. Since $\struct{\P^4}(d)$ is flat over $\A^1$ (it is locally free and $\P^4_{\A^1} \to \A^1$ is flat) we again get an exact sequence of sheaves,
\begin{center}
\begin{tikzcd}
0 \arrow[r] & \I_t(d) \arrow[r] & \struct{\P^4_{\kappa(t)}}(d) \arrow[r] & \struct{X_t}(d) \arrow[r] & 0
\end{tikzcd}
\end{center}
by twisting and then pulling back. Taking the long exact sequence on cohomology gives,
\begin{center}
\begin{tikzcd}
0 \arrow[r] & H^0(\P^4, \I_t(d)) \arrow[r] & H^0(\P^4, \struct{\P^4}(d)) \arrow[r] & H^0(X_t, \struct{X_t}(d)) \arrow[r] & H^1(\P^4, \I_t(d)) \arrow[r] & 0
\end{tikzcd}
\end{center}
using that $H^1(\P^4, \struct{\P^4}(d)) = 0$. However, $X_t \cong \P^1$ abstractly and $X_t \embed \P^4$ is degree $4$ meaning that,
\[ \dim_{\kappa(t)} H^0(X_t, \struct{X_t}(d)) = \dim_{\kappa(t)} H^0(\P^1, \struct{\P^1}(4d)) = 4 d + 1 \]
However, the map $H^0(\P^4_t, \struct{\P^4_t}(d)) \to H^0(X_t, \struct{X_t}(d))$ does not have constant rank in $t$. Indeed, because $X_t$ for $t \neq 0$ are projectively normal the map is surjective but for $t = 0$ it need not be surjective. Therefore, we see that for all closed points $t \neq 0$ that,
\begin{align*}
\dim_{k} H^0(\P^4, \I_t(d)) &= { d + 4 \choose 4} - (4 d + 1)
\\
\dim_{k} H^1(\P^4, \I_t(d)) & = 0
\end{align*} 
In the case $t = 0$, we give an explicit computation. The ideal $I_0 = \Gamma_*(\P^4, \I_t(d))$ is the saturated ideal cutting out $X_0$ and we earlier computed that,
\[ \dim_{k} (S_0 / I_0)_d = \dim_k (k[s^4, s^3 t, s t^3, t^4])_d = 
\begin{cases}
1 & d = 0
\\
4 & d = 1
\\
4 d + 1 & d > 1
\end{cases} \]
Therefore,
\[ \dim_{k} H^0(\P^4, \I_t(d)) = \dim_k (I_0)_d = \dim_k (S_0)_d - \dim_{k} (S_0 / I_0)_d = {d + 4 \choose 4} - \begin{cases}
1 & d = 0
\\
4 & d = 1
\\
4 d + 1 & d > 1
\end{cases} \] 
Then from the exact sequence we find,
\[ \dim_k H^1(\P^4, \I_t(d)) = \dim_k H^0(X_t, \struct{X_t}(d)) - \dim_k (S_0/I_0)_d = 
\begin{cases}
0 & d = 0
\\
1 & d = 1
\\
0 & d > 1
\end{cases} \]
Therefore, we should look at the case $d = 1$. Putting everything together,
\[
\dim_k H^0(\P^4, \I_t(1)) = 
\begin{cases}
0 & t \neq 0
\\
1 & t = 0
\end{cases}
\]
and also,
\[ \dim_k H^1(\P^4, \I_t(1)) = 
\begin{cases}
0 & t \neq 0
\\
1 & t = 0
\end{cases} \]
The first computation represents the fact that there is one linear form vanishing on $X_0$ (the hyperplane $\P^3 \subset \P^4$ containing it) but $X_t \subset \P^4$ is not contained in any hyperplane for $t \neq 0$. The second computation represents exactly the fact that $X_0$ is not projectively normal ``in degree $1$'' but $X_t$ is projectively normal for $t \neq 0$.

\subsubsection{12.4 CHECKE!?}

Let $Y$ be an integral scheme of finite type over an algebraically closed field $k$. Let $f : X \to Y$ be a flat projective morphism whose fibers are integral. Let $\L$ and $\M$ be invertible sheaves on $X$ such that for each $y \in Y$ we have $\L|_{X_y} \cong \M|_{X_y}$. 
\bigskip\\
Consider the cohernet sheaf $\F = \L \otimes \M^{\otimes -1}$ of $\struct{X}$-modules then $f_* \F$ is a coherent $\struct{Y}$-module since $f$ is proper. Because $\L$ and $\M$ are locally free and $f$ is flat, $\F$ is flat over $Y$. Now consider,
\[ \varphi^0(y) = \dim_{\kappa(y)} H^0(X_y, \F|_{X_y}) \]
Because $\L|_{X_y} \cong \M|_{X_y}$ we see that $\F|_{X_y} \cong \struct{X_y}$. 
Therefore, if $y \in Y$ is a closed point then,
\[ H^0(X_y, \F|_{X_y}) = H^0(X_y, \struct{X_y}) = k \] because $\kappa(y) = k$ ($Y$ is finite type over $k$ and $\bar{k} = k$)and $X_y$ is integral and projective over $\Spec{k}$ (and $k$ is algebraically closed). Therefore, $\varphi^0(y) = 1$ at closed points. By semicontinuity, 
\[ \{ y \in Y \mid \varphi^0(y) = 0 \} \]
is open but closed points are dense so $\varphi^0(y) \ge 1$. Furthermore, 
\[ \{ y \in Y \mid \varphi^0(y) > 1 \} \]
is closed so must contain a closed point and thus is empty. Thus $\varphi^0(y) = 1$ for all $y$. Since $Y$ is integral, by Grauert, $f_* \F$ is locally free of rank $\dim_k H^0(X_y, \struct{X_y}) = 1$. Let $\sN = f_* \F$. Then the adjunction map $f^* f_* \F \to \F$ is an isomorphism because both  are line bundles and it is an isomorphism on each fiber $X_y$ and thus $(f^* \sN)_x \ot \kappa(x) \to \F_x \ot \kappa(x)$ is an isomorphism for each $x \in X$. Therefore, $\F \cong f^* \sN$ and therefore,
\[ \L \cong \M \otimes f^* \sN \]

\subsubsection{12.5 FIX THIS AND CHECK Ex 7.9}

Let $Y$ be an integral scheme of finite type over an algebraically closed field $k$. Let $\E$ be a locally free sheaf of rank $r = n + 1$ on $Y$, and let $X = \P_Y(\E)$. We must have $r \ge 2$ or else $X \to Y$ is an isomorphism.
\bigskip\\
Consider the map $\Phi : \Pic{Y} \oplus \Z \to \Pic{X}$ via $(\L, n) \mapsto \L \otimes \struct{X}(n)$ of [II, Ex. 7.9] which we showed is injective in general. 
\bigskip\\
For surjectivity, let $\L$ be an invertible sheaf on $X$. Because the fibers $X_y \cong \P^{r-1}_{\kappa(y)}$ for any $y \in Y$ are isomorphic to projective space over a field there is some $n_y \in \Z$ such that,
 \[ \L|_{X_y} \cong \struct{\P^{n}_{\kappa(y)}}(n_y) \]
However, since $\pi$ is flat, the hilbert polynomial and hence the degree zero term, 
\[ \chi(X_y, \L|_{X_y}) = \chi(\P^{n}_{\kappa(y)}, \struct{\P^{n}_{\kappa(y)}}(n_y)) \]
is constant (FIND REFERENCE) and thus $n_y = n$ is constant. Therefore, the line bundles $\L$ and $\struct{X}(n)$ are isomorphic on fibers $X_y$. Applying [III, Ex. 12.4] we conclude that, $\L = \pi^* \sN \otimes \struct{X}(n)$ for some invertible $\struct{Y}$-module $\sN$ showing that $\Phi$ is surjective.

\subsubsection{12.6 CHECK!?!}

Let $X$ be an integral projective scheme over an algebraically closed field $k$ with $H^1(X, \struct{X}) = 0$. Let $T$ be a connected scheme finite type over $k$.

\begin{enumerate}
\item Let $\L$ be invertible on $X \times T$. Fix a closed point $t_0 \in T$ and let $\L_0 = \iota_0^* \L$ where $\iota : X = X \times \{t_0\} \embed X \times T$. Then for any closed point $t \in T$ let $\L_t = \iota^*_t \L$ where $\iota_t : X_t = X \times \{ t \} \embed X \times T$. Notice that for a closed point $t \in T$, we have $\kappa(t) = k$ through the structue map and thus $X_t = X$ canonically. Furthermore, since $k$ is algebrically closed $X_t = X \times \{ t \} = X \times_{\Spec{k}} \Spec{\kappa(t)}$ is integral by [II, Ex. 3.15].
\bigskip\\
Consider $\pi : X \times T \to T$ which is flat, projective, with fibers $\pi^{-1}(t) = X \times \kappa(t)$ which are integral since $k$ is algebraically closed. Let $\F = \L \otimes (\pi_X^* \L_0)^{\otimes -1}$ then $\F_0 = \struct{X}$. Consider,
\[ U = \{ t \in T \mid \F_t \cong \struct{X} \} \subset T \]
If $t \in U$ then,
\[ \varphi^i(t) : (R^i \pi_* \F)_t \otimes \kappa(t) \to H^i(X, \F_t) = H^i(X, \struct{X})  \]
Since $H^1(X, \struct{X}) = 0$ we see that if $t \in U$ and $i = 1$ then $\varphi^1(t)$ is surjective and thus an isomorphism so $(R^1 \pi_* \F)|_U = 0$ and thus $(R^1 \pi_* \F)$ is locally free at each $t \in U$ which implies that $\varphi^0$ is surjective as well. Therefore, $\varphi^0$ is an isomorphism so by cohomology and base change $\pi_* \F$ is locally free on a neighborhood of $t$ which proves that $U$ is open. Now consider the complement,
\[ V = \{ t \in T \mid \F_t \not\cong \struct{X_t} \} \subset T \]
I claim if a line bundle $\M$ on $X$ is nontrivial then either $H^0(X, \M) = 0$ or $H^0(X, \M^\vee) = 0$. Indeed, suppose $s \in H^0(X, \M)$ and $s' \in H^0(X, \M^\vee)$ are nonzero sections then $s \ot s' \in H^0(X, \struct{X})$ is a nonzero section. But $X$ is projective over an algebraically closed field so $H^0(X, \struct{X}) = 0$ and therefore $s \ot s'$ is everywhere nonvanishing so the associated maps $s : \struct{X} \to \M$ and $s' : \struct{X} \to \M^\vee$ are everywhere nonvanishing and hence isomorphisms.
\bigskip\\
Then choose $t \in V$. Therefore I may assume WLOG that $H^0(X, \F_t) = 0$ otherwise replace $\F_t \cong \struct{X} \iff (\F^\vee)_t \cong \struct{X}$. Then consider, 
\[ \varphi^i(t) : (R^i \pi_* \F)_t \otimes \kappa(t) \to H^i(X, \F_t)  \]
Since $H^0(X, \F_t) = 0$ we see that $\varphi^0(t)$ is surjective and thus an isomorphism by cohomology and base change and moreover since $\varphi^{-1}$ is always an isomorphism we see that $\pi_* \F$ is locally free at $t$ and hence $(\pi_* \F)|_V = 0$. Likewise, $\varphi^0$ is an isomorphism in a neighborhood of $t$ and hence $H^0(X, \F_{t'}) = 0$ for all $t'$ in some neighborhood of $t$ which implies that $V$ is open since if $H^0(X, \F_{t'}) = 0$ then $\F_{t'} \not\cong \struct{X}$. Therefore, since $T$ is connected and $t_0 \in U$ we see that $U = T$. Therefore $\F_t \cong \struct{X}$ for all $t \in T$ so we have $\L_t \cong \L_0$ for all closed points $t \in T$ (IS THERE MORE SIGNIFICANCE TO THE CLOSED POINTS)

\item Choosing closed points $t_0 \in T$ and $x_0 \in X$ give sections the projection maps and therefore we a left inverse of the canonical map,
\[ \Pic{X} \times \Pic{T} \to \Pic{X \times T} \]
so it is injective. To prove surjectivity, it suffices to show that,
\[ \Pic{T} \to \ker{(\Pic{X \times T} \to \Pic{X})} \]
is surjective. Let $\L \in \ker{(\Pic{X \times T} \to \Pic{X})}$ then $\L_{t_0} \cong \struct{X}$. By part (a), this implies that $\L_t \cong \struct{X_t}$ for all $t \in T$ (closed or not). Furthermore, in the proof of (a) we saw that $\pi_* \L$ is a locally-free $\struct{T}$-module and,
\[ (\pi_* \L)_t \iso H^0(X, \L_t) = H^0(X, \struct{X}) = k \]
is an isomorphism so $\pi_* \L$ is locally free of rank $1$. Let $\sN = f_* \F$. Then the adjunction map $\pi^* \pi_* \L \to \L$ is an isomorphism because both are line bundles and it is an isomorphism on each fiber $X_t$ and thus $(\pi^* \sN)_x \ot \kappa(x) \to \L_x \ot \kappa(x)$ is an isomorphism for each $x \in X$. Therefore, $\L \cong \pi^* \sN$ proving the surjectivity. 
\end{enumerate}

\section{IV}

\begin{definition}
Here a curve is a regular integral scheme of dimension one which is finite type over an algebraically closed field $K$.
\end{definition}

\subsection{1}

\subsubsection{1.1}

Let $C$ be a curve of genus $g$ and a point $P \in C$. For $g = 0$ we know $C \cong \P^1$ in which case the desired functions are easily constructed. Thus we may assume $C$ has positive genus. Consider the divisor $(1 + g)[P]$ and the line bundle $\struct{C}((1 + g)[P])$. Then by Riemmann-Roch,
\[ \ell((1 + g)[P]) - \ell(K - (1 + g)[P]) = 1 - g + \deg{(1 + g) [P]} = 2 \]
Furthermore,
\[ \deg{(K - (1+g)[P])} = \deg{K} - \deg{(1 + g)[P]} = 2 - 2g - (1 + g) = 1 - 3 g < 0 \]
Therefore, $\ell(K - (1 + g)[P]) = 0$ so we find,
\[ \ell((1 + g)[P]) = 2 \]
and thus there must be nonconstant functions $f \in K(C)$ which are regular everywhere but $P$. 

\subsubsection{1.2}

Let $C$ be a curve and $P_1, \dots, P_n \in C$ points then using the above construction, we get a nonconstant function $f_i \in K(X)$ which has a pole of order $2$ at $P_i$ and is regular elsewhere. Then take $f = f_1^{e_1} \dots f_n^{e_n}$ has poles only at the points $P_1, \dots, P_n$ but it may not have a pole at each point if the $f_i$ have higher order zeros. There is a matrix $v_{ij} = \mathrm{ord}_{P_i} f_j$ which has $v_{ii} = -2$ and $0 \le v_{ij} \le 2$ for $i \neq j$ since $\deg{f_i} = 0$. We need to chose the vector $e_i$ sucht that $v \cdot e$ has negative entries. 

\subsection{2}

\subsection{3}

\subsubsection{3.1}

Let $X$ be a curve of genus 2.

\subsubsection{3.2}

\subsubsection{3.3}

Let $\iota : X \embed \P^n_k$ be a smooth curve of genus $g \ge 2$ embedded as a complete intersection. Then,
\[ \omega_X = \iota^* \struct{\P}(d_1 + \cdots + d_r - n - 1) \]
Since $\deg{\omega_X} = 2 g - 2 > 0$ we know that $\ell = (d_1 + \cdots + d_r - n - 1) > 0$. Therefore, under the Veronese embedding $v : \P^n_k \to \P^N_k$ of degree $\ell$ which is defined by $v^* \struct{\P^{N}}(1) = \struct{\P^n}(\ell)$. Therefore, embedding $v \circ \iota : X \to \P^N$ gives $\omega_X = (v \circ \iota)^* \struct{\P}(1)$ so $\omega_X$ is very ample. 

\section{Chapter 3}

\subsection{3.5}

Let $X \subset \P^3$ be a smooth curve which does not lie in any hyperplane and $P \in \P^3 \sm X$ a point such that projection away from the point $\varphi : X \to \P^2$ induces a birational map onto its image. 

\begin{enumerate}
\item $\varphi : X \to \varphi(X)$ is an isomorphism if and only if $\varphi(X)$ is nonsingular since it is birational. Suppose that $\varphi$ is an isomorphism. Because $\varphi$ is a linear projection $\varphi^* \struct{\varphi(X)}(1) = \struct{X}(1)$. Therefore, 
\[ H^0(X, \struct{X}(1)) = H^0(\varphi(X), \struct{\varphi(X)}(1)) \]
However, since $X$ does not line in any hyperplane, $H^0(\P^3, \I_X(1)) = 0$ meaning that $H^0(\P^3, \struct{\P^3}(1)) \to H^0(X, \struct{X}(1))$ is injective so,
\[ \dim H^0(X, \struct{X}(1)) \ge 4 \]
Also, since $\varphi(X)$ is integral of dimension $1$ and hence a Cartier divisor (since $\P^2$ is locally factorial) therefore, 
\begin{center}
\begin{tikzcd}
0 \arrow[r] & \struct{\P^2}(1-d) \arrow[r] & \struct{\P^2}(1) \arrow[r] & \struct{\varphi(X)}(1) \arrow[r] & 0
\end{tikzcd}
\end{center}
so we see that,
\begin{center}
\begin{tikzcd}
H^0(\P^2, \struct{\P^2}(1)) \arrow[r] & H^0(\varphi(X), \struct{\varphi(X)}(1)) \arrow[r] & H^1(\P^2, \struct{\P^2}(1-d)) 
\end{tikzcd}
\end{center}
and the last term is zero so,
\[ \dim H^0(\varphi(X), \struct{\varphi(X)}(1)) \le \dim H^0(\P^2, \struct{\P^2}(1)) = 3 \]
which gives a contradiction.

\item Suppose that $X$ has degree $d$ and genus $g$. Then $\varphi(X)$ also has degree $d$ since we are projecting from a point not on $X$. Since $X \to \varphi(X)$ must be the normalization and is not an isomorphism we see that,
\[ g = p_a(X) < p_a(\varphi(X)) = \tfrac{1}{2} (d-1)(d-2) \]
which is a strict inequality (say using Ex. 1.8) because $\varphi(X)$ has singular points. 

\item Consider the flat family induced by the projection $\{ X_t \}$. Then $X_1 = X$ and $(X_0)_{\red} = \varphi(X)$. Since the Hilbert polynomial is constant we see that $\chi(X_0, \struct{X_0}) = 1 - g$ howeer, $\chi(\varphi(X), \struct{\varphi(X)}) = 1 - \tfrac{1}{2} (d - 1)(d-2)$ we see there cannot be an equality from the previous part. Therefore, $(X_0)_{\red} \neq X_0$ so $X_0$ is nonreduced.  
\end{enumerate}

\subsection{3.6}

\begin{enumerate}
\item Let $X$ be a curve of degree $4$ in $\P^n$. We assume $n$ is minimal i.e. $X$ does not lie in a hyperplane. Thus $H^0(X, \struct{X}(1)) \ge n+1$. However, $\deg{\struct{X}(1)} = 4$. Now, if $D$ is effective then $h^0(K-D) \le h^0(K) = g$ and thus,
\[ \deg{D} + 1 - g \le h^0(D) \le \deg{D} + 1 \]
Hence for $D = H$ the hyperplane class we see that,
\[ 5 - g \le h^0(H) \le 5 \]
and therefore $n \le 4$. If $g = 0$ then $h^0(H) = 5$ so we can take $X$ to be the rational normal quartic in $\P^4$ or the rational quartic in $\P^3$ for smaller dimension we check directly that the map cannot be an embedding.
\bigskip\\
Conversely, suppose that $h^0(H) = 5$. Then for arbitrary points $P_1, P_2, P_3, P_4 \in X$ we have $h^0(H - P_1 - P_2 - P_3 - P_4) \ge 1$ so we must have $H \sim P_1 + P_2 + P_3 + P_4$ for arbitrary points. If we choose $P_1, P_2, P_3$ such that $H = P_1 + P_2 + P_3 + Q$ then we see that $Q \sim P_4$ so all points on $X$ are equivalent and hence $X = \P^1$. Thus we may assume for the other cases that $h^0(X) \le 4$. 
\bigskip\\
We know that $g \le \frac{1}{2}(d - 1)(d - 2) = 3$ so we also need to consider the cases $g = 1,2,3$. 
\bigskip\\
In the case $g = 3$ because the genus-degree formula is an equality, by the previous problem $X$ must lie in a plane. Furthermore, $X \subset \P^2$ implies $g = 3$ since $X$ is smooth. 
\bigskip\\
In the case $g = 1$ we must have $h^0(H) = 4$ so $X \subset \P^3$ (it cannot lie in a plane else it would have genus $3$). 
\bigskip\\
Finally, we need to consider the case $g = 2$. Since $4 > 2 g - 2$ we have,
\[ h^0(H) = \deg{H} + 1 - g = 3 \]
and thus $X$ must be a plane curve but $g = 2$ and $X$ is singular giving a contradiction. 

\item In the case $g = 1$ we have $X \subset \P^3$. Consider the sequence,
\begin{center}
\begin{tikzcd}
0 \arrow[r] & \I_X(2) \arrow[r] & \struct{\P^3}(2) \arrow[r] & \struct{X}(2) \arrow[r] & 0
\end{tikzcd}
\end{center}
giving a long exact sequence,
\begin{center}
\begin{tikzcd}
H^1(\P^3, \struct{\P^3}(2)) \arrow[r] & H^1(X, \struct{X}(2)) \arrow[r] & H^2(\P^3, \I_X(2)) \arrow[r] & H^2(\P^3, \struct{\P^3}(2)) 
\end{tikzcd}
\end{center}
the edge terms are zero and thus $h^2(\I_X(2)) = h^1(\struct{X}(2))$. Furthermore, from the exact sequence,
\[ \chi(\I_X(n)) = \chi(\struct{\P^3}(n)) - \chi(\struct{X}(n)) = { 3 + n \choose 3 } - 4 n \]
For $n = 2$ we get,
\[ \chi(\I_X(2)) = 2 \]
Furthermore, $h^2(\I_X(2)) = h^1(\struct{X}(2)) = 0$ because $\deg{\struct{X}(2)} 8 > 2 g - 2$. Therefore, 
\[ h^0(\I_X(2)) \ge 2 \]
proving that $X$ lies on two distinct quadric surfaces. These surfaces must be smooth because $X$ does not lie in a plane. Now I claim that $X$ is the complete intersection of two of these sufaces. Let $Q \subset \P^2$ be one quadric. Then the other qudratic defines a divisor $D$ of type $(2,2)$ on $Q$ which is either a smooth curve of genus $1$ or a collection of lines and therefore since $X \subset D$ we must have $X = D$ and $D$ is a smooth curve of genus $1$. 
\end{enumerate}

\subsection{3.7}

Consider the plane curve $X \subset \P^2$ defined by $xy + x^4 + y^4 = 0$. Suppose that $\wt{X} \subset \P^3$ is a smooth curve such that $\varphi : \wt{X} \to X$ is a projection from a point. Then $\deg{\wt{X}} = 4$ and its genus is,
\[ g(\wt{X}) = \tfrac{1}{2}(d - 1)(d - 2) - r = 3 - 1 = 2 \]
since $X$ has exactly one node. However, we have shown in [Ex. 3.6] that there are no smooth curves of genus $2$ and degree $5$ in any $\P^n$. 


\subsection{3.12}

We do this by casework. First we remove the $d = 2,3,4,5$ and $r$ for which there exists a smooth curve in $\P^n$ of degree $d$ and genus $0 \le g \le p = \tfrac{1}{2}(d - 1)(d - 2)$ since then,
\[ r = p - g \]
and we get exactly $r$ nodes for a general projection into $\P^2$.
\bigskip\\
For $d = 2$ we have $p = 0$ and so we just take the plane curve. For $d = 3$ we have $p = 1$ so for $g = 0$ we have the twisted cubic in $\P^3$ (or we could just take the nodal cubic in $\P^2$ directly) and for $g = 1$ we take a smooth elliptic curve in $\P^2$. For $d = 4$ we have $p = 3$. For $g = 0$ we take the twisted quartic. For $g = 1$ we take the intersection of two quadric surfaces. For $g = 3$ we just take a smooth quartic curve. For $d = 5$ we have $p = 6$. Then for $g = 0$ we take the rational normal curve in $\P^5$. For $g = 6$ we just take a smooth plane curve. On an elliptic curve $E$ we can take $\struct{E}(5P)$ which gives a $g = 1$ curve with $d = 5$ in $\P^4$. Consider a quadric surface $Q \subset \P^3$ and a divisor of type $(3, 2)$ which has degree $(3, 2) \cdot (1,1) = 5$ and genus $g = (3-1)(2-1) = 2$. 
\bigskip\\
Therefore, we are missing $(d, r) = (4, 1), (5, 3), (5, 2), (5, 1)$. It is easy to produce a $(d, 1)$ via choosing a generic singular curve of degree $d$. Or we can use the equation,
\[ xy + x^d + y^d = 0 \]
Thus we just need $(5,3)$ and $(5,2)$. (HOW TO GET THESE)


\section{V}

\section{Lemmata}


\begin{lemma} \label{open_in_irreducible_is_dense}
In an irreducible topological space every nonempty open set is dense.
\end{lemma}

\begin{proof}
Let $U \subset X$ be open with $X$ irreducible. Then take any closed set $C \supset U$. Then $C \cup U^C = X$ since if $x \notin U$ then $x \in U \subset C$. Therefore, since $X$ is irreducible either $U = \varnothing$ or $C = X$. If $U$ is nonempty then we must have $\overline{U} = X$.  
\end{proof}

\begin{lemma} \label{open_of_irreducible}
Let $X$ be an irreducible topological space and nonempty open $U \subset X$. Then $U$ is irreducible.
\end{lemma}

\begin{proof}
Suppose that there were closed sets $Z_1, Z_2 \subset X$ such that
 \[ (Z_1 \cap U) \cup (Z_1 \cap U) = U \]
 i.e. such that $Z_1 \cup Z_2 \supset U$. However, by Lemma \ref{open_in_irreducible_is_dense}, we have $Z_1 \cup Z_2 = X$ since $U$ is a nonempty open and $X$ is irreducible. Therefore, either $Z_1 = X$ or $Z_2 = X$ implying that $Z_1 \cap U = U$ or $Z_2 \cap U = U$ so $U$ is irreducible.   
\end{proof}

\begin{lemma} \label{irreducible_implies_connected}
Any irreducible topological space is connected.
\end{lemma}

\begin{proof}
Suppose that $X$ is irreducible. Suppose that $U \subset X$ is clopen. Then $U$ and $U^C$ are both closed but $U \cup U^C = X$ so either $U = X$ or $U = \varnothing$ proving that $X$ is connected. 
\end{proof}



\begin{lemma} \label{alt_sum_exact}
Consider the exact sequence of finite-dimensional $k$-vectorspaces,
\begin{center}
\begin{tikzcd}
0 \arrow[r] & V_0 \arrow[r, "T_0"] & V_1 \arrow[r, "T_1"] & V_2  \arrow[r] & \cdots \arrow[r] & V_n \arrow[r, "T_n"] & 0 
\end{tikzcd} 
\end{center}
Then we have the alternating sum,
\[ \sum_{i = 0}^n (-1)^i \dim_k V_i = 0 \]
\end{lemma}

\begin{proof}
The rank-nullty theorem gives,
\[ \dim_k{V_i} = \dim_k{\ker{T_i}} + \dim_k{\Im{T_i}} \]
However, by exactness, $\Im{T_i} = \ker{T_{i+1}}$ so consider,
\begin{align*}
\sum_{i = 0}^n (-1)^i \dim_k V_i & = \sum_{i = 0}^n (-1)^i \left[ \dim_k{\ker{T_i}} + \dim_k{\ker{T_{i + 1}}} \right]
\\
& = \sum_{i = 0}^n (-1)^i  \dim_k{\ker{T_i}} + \sum_{i = 0}^{n} (-1)^i  \dim_k{\ker{T_{i+1}}} 
\\
& = \sum_{i = 0}^n (-1)^i  \dim_k{\ker{T_i}} - \sum_{i = 1}^{n+1} (-1)^i  \dim_k{\ker{T_{i}}}
\\
& = \dim_k{\ker{T_0}} - (-1)^{n+1} \ker{T_{n+1}}
\end{align*}
However, $T_0$ is injective and $T_{n+1}$ is the map $0 \to 0$ so both kernels vanish. Therefore,
\[ \sum_{i = 0}^n (-1)^i \dim_k V_i = 0 \]
\end{proof}

\begin{definition}
We say a scheme is \textit{locally noetherian} if there is an open affine cover $U_i = \Spec{A_i}$ by spectra of noetherian rings $A_i$. Furthermore we say a scheme is \textit{noetherian} if it is locally noetherian and quasi-compact.
\end{definition}

\begin{definition}
A morphism $f : X \to Y$ of schemes is \textit{locally of finite type} if on each open $U \subset Y$ the ring map $f^\# : \struct{Y}(U) \to \struct{X}(f^{-1}(U))$ is finite type. Furthermore a the morphism $f :  X \to Y$ is \textit{finite type} if it is locally finite type and quasi-compact.
\end{definition}

\begin{lemma}
Let $f : X \to Y$ be a finite-type morphism of schemes and $Y$ noetherian. Then $X$ is noetherian.
\end{lemma}

\begin{proof}
Since $Y$ is noetherian, it is quasi-compact and has an open affine cover by spectra of noetherian rings $U_i = \Spec{A_i}$. Since $f$ is a finite-type morphism $f$ is quasi-compact so $X = f^{-1}(Y)$ is quasi-compact. Furthermore, the ring map $f^\# : \struct{Y}(U_i) \to \struct{X}(f^{-1}(U_i))$ is finite-type meaning that $\struct{X}(f^{-1}(U_i))$ is a finitely-generated $A_i$-algebra since $\struct{Y}(U_i) = A_i$. Since $A_i$ is noetherian and there is a surjection, \begin{center}
\begin{tikzcd}
A_i[x_1, \dots, x_n] \arrow[r, two heads] & \struct{X}(f^{-1}(U_i))
\end{tikzcd}
\end{center}
then by Hilbert's basis theorem $A_i[x_1, \dots, x_n]$ is noetherian and thus so is $\struct{X}(f^{-1}(U_i))$ proving that $X$ is a noetherian scheme. 
\end{proof}

\begin{corollary}
Any variety is noetherian.
\end{corollary}

\begin{proof}
By definition, a variety $X$ is a finite type scheme over $k$ i.e. the morphism $X \to \Spec{k}$ is finite type. However, $\Spec{k}$ is clearly noetherian thus so is $X$.
\end{proof}

\begin{lemma} \label{tensor_inverse}
Let $R$ be a local ring and let $M, N$ be $R$-modules such that $M \otimes_R N \cong R^n$ then $M$ and $N$ are free.
\end{lemma}

\begin{proof}
Using the isomorphism $\varphi : M \ot_R N \iso R^n$ we get isomorphisms,
\[ M^n = R^n \ot_R M \iso (M \ot_R N) \ot_R M = M \ot_R (N \ot_R M) \iso M \ot_R R^n = M^n \]
Consider,
\[ \varphi^{-1}(e_i) = \sum_j m_{ij} \ot n_{ij} \]
The the first map sends,
\[ (x_i) \mapsto \sum_{ij} m_{ij} \ot n_{ij} \ot x_i \]
and the second maps sends,
\[ m \ot n \ot m' \mapsto m \ot \varphi(n \ot m') \]
Therefore, 
\[ (x_i) \mapsto \sum_{ij} m_{ij} \varphi(n_{ij} \ot x_i) \]
is an automorphism of $M^n$. This map factors as,
\[ M^n \to R^{nr} \to M^n \]
where the first map is,
\[ (x_i) \mapsto (\varphi(n_{ij} \ot x_i))_{ij} \]
and the second is given by the generating set $m_{ij}$. Therefore, $M^n$ is a direct factor of $R^{nr}$ and hence $M$ is a direct factor of $R^{nr}$ so $M$ is a finite projective. Since $R$ is a local ring then $M$ is free. Likewise for $N$ by symmetry. 
\end{proof}

\begin{lemma} \label{projective_scheme_proper}
Any projective scheme over $k$ is proper over $k$.
\end{lemma}

\begin{proof}

\end{proof}

\begin{lemma} \label{global_sections_proper_scheme}
Let $X$ be an integral scheme proper over $k$. Then $H^0(X, \struct{X})$ is a field which is a finite extension of $k$. 
\end{lemma}

\begin{proof}
The sheaf $\struct{X}$ is coherent and $X$ is a proper scheme over $k$ so $H^0(X, \struct{X})$ is a finite-dimensional $k$-vectorspace. Furthermore, $X$ is integral so $H^0(X, \struct{X})$ is a field and thus a finite field extension of $k$.
\end{proof}

\begin{lemma}
Let $B$ be an $A$-algebra giving $f : \Spec{B} \to \Spec{A}$. Then as quasi-coherent $\struct{\Spec{A}}$-modules,
 \[ f_* \struct{\Spec{B}} = \widetilde{B} \]
\end{lemma}

\begin{proof}
Denote the algebra map $\iota : A \to B$ and $f = \Spec{\iota}$. We have, 
\[ f_* \struct{\Spec{B}}(D(x)) = \struct{\Spec{B}}(f^{-1}(D(x))) \] However, 
\[ f(\p) = \iota^{-1}(\p) \quad \text{thus} \quad x \in \iota^{-1}(\p) \iff \iota(x) \in \p \quad \text{i.e.} \quad f(\p) \in D(x) \iff \p \in D(\iota(x)) \] Thus $f^{-1}(D(x)) = D(\iota(x))$ so,
\[ f_* \struct{\Spec{B}}(D(x)) = \struct{\Spec{B}}(D(\iota(x))) = B_{\iota(x)}  = \widetilde{B}(D(x)) \]
since localizing $B$ at $x$ as an $A$-module is the same as localizing $B$ at $\iota(x)$ as a ring.
\end{proof}

\begin{lemma} \label{lem:exact_seq_of_functors}
Let $F, G, H : \mathcal{A} \to \mathcal{B}$ be additive functors between abelian categories and let $\mathcal{A}$ have enough injectives. Suppose there exists a sequence of natural transformations $F \xrightarrow{\alpha} G \xrightarrow{\beta} H$ such that for each injective object $I \in \mathcal{A}$ that the sequence,
\begin{center}
\begin{tikzcd}
0 \arrow[r] & F(I) \arrow[r, "\alpha_I"] & G(I) \arrow[r, "\beta_I"] & H(I) \arrow[r] & 0  
\end{tikzcd}
\end{center}
is exact. Then for any object $A \in \mathcal{A}$ there exists a long exact cohomology sequence relating the right-derived functors,
\begin{center}
\begin{tikzcd}[column sep = small]
0 \arrow[r] & R^0 F(A) \arrow[r] & R^0 G(A) \arrow[r] & R^0 H(A) \arrow[r] & R^1 F(A) \arrow[r] & R^1 G(A) \arrow[draw=none]{d}[name=Z, shape=coordinate]{} \arrow[r] & R^1 H(A)
\arrow[dlllll,
rounded corners, crossing over,
to path={ -- ([xshift=2ex]\tikztostart.east)
|- (Z) [near end]\tikztonodes
-| ([xshift=-2ex]\tikztotarget.west)
-- (\tikztotarget)}]
\\ 
& R^2 F(A) \arrow[r] & R^2 G(A) \arrow[r] & R^2 H(A) \arrow[r] & R^3 F(A) \arrow[r] & R^3 G(A) \arrow[r] & R^3 H(A) \arrow[r] & \cdots
\end{tikzcd}
\end{center}
\end{lemma}

\begin{proof}
Consider an injective resolution of $A$,
\begin{center}
\begin{tikzcd}
0 \arrow[r] & A \arrow[r] & I^0 \arrow[r] & I^1 \arrow[r] & I^2 \arrow[r] & I^3 \arrow[r] & \cdots
\end{tikzcd}
\end{center}
Now consider the complex,
\begin{center}
\begin{tikzcd}
& 0 \arrow[d] & 0 \arrow[d] & 0 \arrow[d] 
\\
& F(A) \arrow[r, "\alpha_A"] \arrow[d] & G(A) \arrow[r, "\beta_A"] \arrow[d] & H(A) \arrow[d] 
\\
0 \arrow[r] & F(I^0) \arrow[d] \arrow[r, "\alpha_I"] & G(I^0) \arrow[d] \arrow[r, "\beta_I"] & H(I^0) \arrow[d] \arrow[r] & 0  
\\
0 \arrow[r] & F(I^1) \arrow[d] \arrow[r, "\alpha_I"] & G(I^1) \arrow[d] \arrow[r, "\beta_I"] & H(I^1) \arrow[d] \arrow[r] & 0  
\\
0 \arrow[r] & F(I^2) \arrow[d] \arrow[r, "\alpha_I"] & G(I^2) \arrow[d] \arrow[r, "\beta_I"] & H(I^2) \arrow[d] \arrow[r] & 0  
\\
& \vdots & \vdots & \vdots
\end{tikzcd}
\end{center}
with (except for the first) exact rows. Thus, this is an exact sequence of complexes,
\begin{center}
\begin{tikzcd}
0 \arrow[r] & F(\bf{I}^\bullet) \arrow[r, "\alpha"] & G(\bf{I}^\bullet) \arrow[r, "\beta"] & H(\bf{I}^\bullet) \arrow[r] & 0
\end{tikzcd}
\end{center}
which gives rise to an exact sequence of cohomology coinciding with the required sequence since $R^p F(A) = H^p(F(\bf{I}^\bullet)$. 
\end{proof}

\begin{lemma} \label{lem:flasque_mayer_vietoris}
Let $\F$ be a flasque sheaf on $X$ and $U, V \subset X$ be open sets. Then the following sequence,
\begin{center}
\begin{tikzcd}
0 \arrow[r] & \Gamma(U \cup V, \F) \arrow[r] & \Gamma(U, \F) \oplus \Gamma(V, \F) \arrow[r] & \Gamma(U \cap V, \F) \arrow[r] & 0
\end{tikzcd}
\end{center}
with maps $s \mapsto (s|_U, s|_V)$ and $(s, t) \mapsto (s - t)|_{U \cap V}$ is exact.
\end{lemma}

\begin{proof}
The first map is the kernel of the second by the sheaf property of $\F$ i.e. the pair of sections $(s, t)$ is the image of a global section exactly when then agree on the overlap i.e. $s|_{U \cap V} = t|_{U \cap V} \iff (s - t)|_{U \cap V} = 0$. Finally, the map sending $(s, t) \mapsto (s - t)|_{U \cap V}$ is surjective because $\F$ is flasque so the restriction map $(s, 0) \mapsto s|_{U \cap V}$ is surjective. 
\end{proof}

\begin{theorem}[Mayer-Vietoris]
Let $\F$ be a sheaf on $X$ and $U, V \subset X$ be open sets. Then there is a long-exact sequence of cohomology, 
\begin{center}
\begin{tikzcd}
0 \arrow[r] & H^0(U \cup V, \F) \arrow[r] & H^0(U, \F) \oplus H^0(V, \F) \arrow[draw=none]{d}[name=Z, shape=coordinate]{} \arrow[r] & H^0(U \cap V, \F)
\arrow[dll,
rounded corners, crossing over,
to path={ -- ([xshift=2ex]\tikztostart.east)
|- (Z) [near end]\tikztonodes
-| ([xshift=-2ex]\tikztotarget.west)
-- (\tikztotarget)}]
\\ 
& H^1(U \cup V, \F) \arrow[r] & H^1(U, \F) \oplus H^1(V, \F) \arrow[draw=none]{d}[name=Z', shape=coordinate]{} \arrow[r] & H^1(U \cap V, \F) \arrow[dll,
rounded corners, crossing over,
to path={ -- ([xshift=2ex]\tikztostart.east)
|- (Z') [near end]\tikztonodes
-| ([xshift=-2ex]\tikztotarget.west)
-- (\tikztotarget)}]
\\
& H^2(U \cup V, \F) \arrow[r] & H^2(U, \F) \oplus H^2(V, \F) \arrow[r] & H^2(U \cap V, \F) \arrow[r] & \cdots
\end{tikzcd}
\end{center}
\end{theorem}

\begin{proof}
By the above lemma, the left-exact functors $\Gamma(U \cup V, -)$ and $\Gamma(U, -) \oplus \Gamma(V, -)$ and $\Gamma(U \cap V, -)$ satisfy the conditions of Lemma \ref{lem:exact_seq_of_functors} giving an exact sequence of their derived functors. Furthermore, because direct sum is exact it commutes with taking cohomology and thus direct sum commutes with taking derived functors. Thus Lemma \ref{lem:exact_seq_of_functors} gives the required long exact sequence. 
\end{proof}
\end{document}
\documentclass[12pt]{article}
\usepackage{import}
\import{../}{AlgGeoCommands}


\begin{document}

\section{Sep. 20}

\subsection{History}

\begin{enumerate}
\item 19th century
\begin{enumerate}
\item $Z(f_1, \dots, f_n) \subset \CC^n$ using analytic tools
\item Riemann's idea of moduli of algebraic curves (1857)
\end{enumerate}
\item 20th century
\begin{enumerate}
\item $Z(f_1, \dots, f_n) \subset \CC^n$ using algebraic tools (commutative algebra)
\item replace $\CC$ with algebraically closed field $k$
\item number theory: want $k = \FF_p$ or $\Q$ not algebraically closed. Examples: Fermat's Last Theorem:
\[ u^n + v^n = 1 \]
want geometry for,
\[ \Q[u,v]/(u^2 + v^2 - 1) \]
but this only has finitely many points so how is there a ``geometry''. 
\end{enumerate}
\end{enumerate}

\subsubsection{Question: for any field $k$ is there a ``geometry'' for $k[X_1, \dots, X_n]/I$?}

First attempt (Weil and Zariski 1930s - 1940s) use Galois theory with algebraic sets in $\overline{k}^n$ for ideals $I \subset k[X_1, \dots, X_n]$. This only works for perfect fields (not $\F_p(t)$ which we want to consider generic families of equations over $\Z$). Weil's Foundations of Algebraic Geometry.

\subsubsection{The Weil Conjectures (1948)}

For $f_1, \dots, f_r \in \FF_p[x_1, \dots, x_n]$ then define,
\[ N_m = \{ x \in \FF_{p^m}^r \mid f_1(x) = \cdots = f_r(x) = 0 \} \]
Then we define a Zeta function,
\[ \zeta(s) = \exp{ \left( \sum_{m = 1}^\infty \frac{N_m}{m} p^{-sm} \right) } \]
which should control the behavior of $N_m$ as $m \to \infty$. Furthermore,
\[ Z_\CC(F_1, \dots, F_r) \subset \CC^n \quad \text{ and } \quad Z_p = \Z(F_1, \dots, F_r) \subset \overline{\FF_p}^n \]
are closely related where algebraic topology invariants of $Z_\CC$ gives formulas for counts of $Z_p$. 

\subsubsection{1950's: Chaos (K\"{a}hler, Shimura, Nagata, etc.)}

Proposal for algebraic geometry over Dedekind domains but chaotic and confusing. Then Schemes resolve all of these problems to give ``geometry over any commutative ring''. 

\subsection{Affine Algebraic Sets}

Let $k$ be an algebraically closed field. Let $\A^n = k^n$ and define a subset $Z \subset \A^n$ to be \textit{algebraic} if $Z = Z(\Sigma)$ where $\Sigma \subset k[X_1, \dots, X_n]$ is a set of polynomials. Then $Z(\Sigma) = Z(I)$ where $I$ is the ideal generated by $\Sigma$.

\begin{thm}
The algebraic sets form (the complement of) a topology on $\A^n$.
\end{thm}

\begin{rmk}
We call this the Zariski topology.
\end{rmk}
\noindent
There is a base of open sets given by,
\[ U_f = \{ x \in \A^n \mid f(x) \neq 0 \} \]

\subsubsection{Examples}

The Zariski topology on $\A^1$ has the cofinite topology. However, $\A^2 \neq \A^1 \times \A^1$ as a topological space. 

\begin{rmk}
Some weird properties of the Zariski topology,
\begin{enumerate}
\item In $\CC^n$ any nonzero open ball is Zariski dense.
\item $Z(f) = Z(f^n)$ and $Z(I) = Z(\sqrt{I})$.
\end{enumerate}
\end{rmk}

\begin{defn}
For any $Y \subset \A^n$ define,
\[ I(Y) = \{ f \in k[X_1, \dots, X_n] \mid \forall y \in Y : f(y) = 0 \} \]
which is a radical ideal.
\end{defn}

\begin{prop}[Nullstellensatz]
For a field $k$ and $\m \subset K[X_1, \dots, X_n]$ a maximal ideal. Then $K[X_1, \dots, X_n]/\m$ is a finite dimensional $K$-vector space.
\end{prop}

\begin{proof}
210B [Mat, Thm 5.3]
\end{proof}

\begin{cor}
If $K$ is algebraically closed then $K \to K[X_1, \dots, X_n]/\m$ is an isomorphism so we have $a_i \mapsto X_i$ and thus $X_i - a_i = 0$ in the quotient so,
\[ \m = (X_1 - a_1, \dots, X_n - a_n) \]
is the kernel of the map $K[X_1, \dots, X_n] \to K[X_1, \dots, X_n]/\m$. Therefore, points in $Z(J)$ correspond to $\m \supset J$. Therefore,
\[ I(Z(J)) = \bigcap_{\m \supset J} \m \]
\end{cor}

\begin{thm}
The following hold,
\begin{enumerate}
\item $I_1 \subset I_2 \implies Z(I_1) \supset Z(I_2)$
\item $Y_1 \subset Y_2 \implies I(Y_1) \supset I(Y_2)$
\item $I(Y_1 \cup Y_2) = I(Y_1) \cap I(Y_2)$
\item $Z(I_1 \cap I_2) = Z(Y_1) \cup Z(Y_2)$
\item $Z(I(Y)) = \overline{Y}$
\item $I(Z(J)) = \sqrt{J}$ (Hilbert's Nullstellensatz). 
\end{enumerate}
\end{thm}

\begin{proof}
The first three follow directly from definitions. Suppose that $x \notin Z(I_1)$ and $x \notin Z(I_2)$ then there is some $f_i \in I_i$ such that $f_i(x) \neq 0$ so $f_1(x) f_2(x) \neq 0$ but $f_1 f_2 \in I_1 \cap I_2$ so $x \notin Z(I_1 \cap I_2)$. 
\bigskip\\
Now $Y \subset Z(I(Y))$ so $\overline{Y} \subset Z(I(Y))$. Pick $x \notin \overline{Y}$ so there is some $x \in U_f$ such that $U_f \cap \overline{Y} = \empty$. Therefore, $U_f \cap Y = \empty$ so $f|_Y = 0$ since if $x \in Y$ then $x \notin U_f$. Thus $f \in I(Y)$ so $x \notin Z(I(Y))$ proving that $Z(I(Y)) \subset \overline{Y}$.
\bigskip\\
Since $J \subset I(Z(J))$ is radical we see that $\sqrt{J} \subset I(Z(J))$. The key is to apply the Nullstellensatz.
\end{proof}

\section{Sep. 22}

\begin{prop}
Let $K$ be a field and $A$ a finitely generated $K$-algebra, $J \subset A$ an ideal. Then,
\[ \sqrt{J} = \bigcap_{\m \supset J} \m = \Jac{J} \]
\end{prop}

\begin{proof}
Replace $A$ by $A / \sqrt{J}$ such that $J = (0)$ and $\nilrad{A} = (0)$. Choose $f \neq 0$ then $f$ is not nilpotent so $A_f$ is nonzero and $A_f = A[x]/(xf - 1)$ is a finitely generated $K$-algebra. Thus $A_f$ has a maximal ideal $\m \subset A_f$. Now under $\varphi : A \to A_f$ we see that $\varphi^{-1}(\m) \subset A$ is a prime. However, $A / \varphi^{-1}(\m) \embed A_f / \m$ but $A_f / \m$ is a finite field extension of $K$ so $A / \varphi^{-1}(\m)$ is a finite dimensional $K$-algebra and a domain so its a field and thus $\varphi^{-1}(\m)$ is maximal and $f \notin \varphi^{-1}(\m)$ and thus $f \notin \Jac{A}$.  
\end{proof}

\begin{rmk}
Any domain $D$ that is a finite dimensional $K$-algebra is a field because if $r \in D$ is nonzero then $D \xrightarrow{\times r} D$ is injective and thus surjective so $xr = 1$ for some $x \in D$ so $D$ is a field.
\end{rmk}

\begin{rmk}
Usually difficult to compute $\sqrt{J}$ given generators of $J$. 
\end{rmk}

\begin{defn}
Say that $f \in k[x_1, \dots, x_n]$ is \textit{radical} if $f \in k$ and no repeated irreducible factors. The hypersurface $Z(f)$ for non-constant $f$ is radical.
\end{defn}

\begin{defn}
A topological space $Y$ is \textit{irreducible} if $Y \neq \empty$ and $Y \neq Y_1 \cup Y_2$ for closed $Y_1, Y_2 \subsetneq Y$. Otherwise, $Y$ is \textit{reducible}.
\end{defn}

\begin{rmk}
If $Y$ is irreducible then every nonempty open $U \subset Y$ is dense. This is because $Y = (Y \setminus U) \cup \overline{U}$ but if $U$ is nonempty then $Y \setminus U$ is a proper subset so $\overline{U} = Y$.
\end{rmk}

\begin{defn}
A topological space $Y$ is \textit{Noetherian} if it satisfies the DCC for closed sets meaning if,
\[ Z_1 \supset Z_2 \supset Z_3 \supset \cdots \]
is a descending chain then it stabilizes meaning $Z_n = Z_{n + 1}$ for all sufficiently large $n$.
\end{defn}

\begin{example}
$\A^n$ is Noetherian because closed sets correspond to ideals and $k[x_1, \dots, x_n]$ is Noetherian.
\end{example}

\begin{prop}
Let $Z \subset \A^n$ be an algebraic set. Then $Z$ is irreducible if and only if $I(Z)$ is prime.
\end{prop}

\begin{proof}
Irreducible are nonempty and prime ideals $I$ are proper subsets. Thus consider the case that $I(Z) \neq (1)$ equivalently that $Z$ is nonempty. We see that,
\[ Z = Z_1 \cup Z_2 \iff I(Z) = I(Z_1) \cap I(Z_2) \]
and $Z_i \subsetneq Z$ iff $I(Z_1) \supsetneq I(Z)$. Therefore, irreducibility of $Z$ is equivalent to the condition that if $I(Z) = I_1 \cap I_2$ with $I_1$ and $I_2$ radical then $I_1 = I(Z)$ or $I_2 = I(Z)$ which is equivalent to in $A = k[x_1, \dots, x_n] / I(Z)$ the property that if $(0) = J_1 \cap J_2$ then either $J_1 = (0)$ or $J_2 = (0)$. Therefore, we reduce to showing the following: if $A$ is a nonzero reduced ring, then $A$ is a domain iff $J_1 \cap J_2 = (0)$ for radical ideals $J_1, J_2$ then either $J_1 = (0)$ or $J_2 = (0)$.
\bigskip\\
If $A$ is a domain then $J_1 J_2 \subset J_1 \cap J_2 = (0)$ so if $a_i \in J_i$ are nonzero then $a_1 a_2 \in J_1 J_2$ so $a_1 a_2 = 0$ contradicting the fact that $A$ is a domain. Now suppose that $A$ has this property. Choose $f,g \in A$ such that $fg = 0$ then let $Q = \sqrt{(f)} \cap \sqrt{(g)}$. If $a \in Q$ then $a^n = pf$ and $a^m = qg$ so $a^{n+m} = pq fg = 0$ and thus $a \in \nilrad{A}$ but $A$ is reduced so $Q = (0)$ and thus either $f = 0$ or $g = 0$ by the assumption.
\end{proof}

\begin{cor}
If $f$ is radical then $Z(f)$ is irreducible iff $f$ is irreducible.
\end{cor}

\begin{proof}
Both are equivalent to $(f)$ being prime.
\end{proof}

\begin{thm}
Every Noetherian topological space is a finite union of irreducible closed sets,
\[ Y = Y_1 \cup \cdots \cup Y_r \]
which is unique if we require the irredundancy,
\[ Y_i \not\subset \bigcup_{j \neq i} Y_i \]
Furthermore, in the irredundant case, the $Y_i$ are exactly the maximal irreducible subsets (i.e. irreducible components). 
\end{thm}

\begin{cor}
Every algebraic set $Z$ is a finite union of irreducible closed subsets.
\end{cor}

\begin{defn}
An \textit{affine variety} is a irreducible algebraic set.
\end{defn}

\section{Dimension and Regular Functions}

\begin{lemma}
If $Y \subset X$ is irreducible in the subspace topology and $Y \subset Z_1 \cup Z_2$ for closed $Z_j \subset X$ then $Z \subset Z_1$ or $Y \subset Z_2$.
\end{lemma}

\begin{proof}
Then $Y = (Y \cap Z_1) \cap (Y \cap Z_2)$.
\end{proof}

\begin{rmk}
This is why for an irredundant decomposition,
\[ X = Z_1 \cup \cdots \cup Z_n \]
into its irreducible components then every irreducible $Y \subset X$ lies inside some $Z_i$. Therefore, the $Z_i$ are indeed maximal irreducible subsets.
\end{rmk}

\begin{defn}
The (combinatorial) \textit{dimension} of a topological space $X$ is,
\[ \dim(X) = \sup \{ n \ge 0 \mid Z_0 \subsetneq Z_1 \subsetneq \cdots \subsetneq Z_n \subset X \text{ of irreducible closed } Z_j \subset X \} \]
Furthermore we set $\dim(\empty) = -\infty$.
\end{defn}

\begin{rmk}
We may have $\dim(X) = \infty$.
\end{rmk}

\begin{defn}
For a commutative ring $A$,
\[ \dim{A} = \sup\{ n \ge 0 \mid \p_0 \subsetneq \p_1 \subsetneq \cdots \subsetneq \p_n \subset A \text{ for prime } \p_j \subset A \} \]
and we set $\dim{(0)} = - \infty$.
\end{defn}

\begin{defn}
For $Y \subset \A^n$ an algebraic set, we define the coordinate ring $k[Y] := k[x_1, \dots, x_n]/I(Y)$. Notice this depends on the embedding into affine space not necessarily the intrinsic structure of $Y$.
\end{defn}

\begin{rmk}
We see that there are inclusion reversing equivalences,
\[ \{ \text{Radical ideals of } k[Y] \} \iff \{ \text{closed subsets } Z \subset Y \} \]
and likewise,
\[ \{ \text{Prime ideals of } k[Y] \} \iff \{ \text{irreducible closed subsets } Z \subset Y \} \]
Therefore,
\[ \dim{Y} = \dim{k[Y]} \]
\end{rmk}

\begin{rmk}
For irreducible closed $Z \subset X$ where $X$ is an affine algebraic set, does there exist a maximal length chain,
\[ Z_0 \subsetneq Z_1 \subsetneq \cdots \subsetneq Z_d \subset X \]
with some $Z_j = Z$? If $X$ is reducible then the answer is no because we can have irreducible components of different dimensions. However, for irreducible algebraic sets the answer is yes. 
\end{rmk}

\begin{thm}
Let $B$ be a domain finitely generated over a field $k$. Then,
\begin{enumerate}
\item $\dim{B} = \trdeg{k}{\Frac{B}}$ which is, in particular, finite
\item For any prime $\p \subset B$,
\[ \dim{B} = \dim{B/\p} + \dim{B_\p} \]
\end{enumerate}
\end{thm}

\begin{rmk}
We interpret the second part of this theorem as follows. The primes of $B / \p$ are exactly the primes containing $\p$ and thus we consider maximal chains,
\[ \p = \p_0 \subsetneq \cdots \subsetneq \p_n \]
and the primes of $B_\p$ are exactly the primes contained in $\p$ and thus we consider maximal chains,
\[ \p = \q_m \supsetneq \cdots \subsetneq \q_0 \]
and thus splicing them together, by the theorem, gives a maximal length chain in $B$ containing $\p$.
\end{rmk}

\begin{proof}
For (a) [Mat. Thm. 5.6] for (b) [Mat, Ex. 5.1] (the solution is in the back of the book and uses part (a) and induction for non-algebraically closed fields even if you only care about the case of algebraically closed fields).  
\end{proof}

\begin{theorem}[Krull]
For all local Noetherian rings, $\dim{A} < \infty$.
\end{theorem}

\begin{proof}
See [Mat, Thm. 13.5] and [AM, Cor. 11.11].
\end{proof}

\begin{cor}
$\dim{\A^n} = \dim{k[x_1, \dots, x_n]} = \trdeg{k}{k(x_1, \dots, x_n)} = n$
\end{cor}

\begin{cor}
Let $Z \subset \Z^n$ be irreducible and closed. Then,
\begin{enumerate}
\item For nonempty open $U \subset Z$ (so $\overline{U} = Z$ because $Z$ is irreducible) then $\dim{U} = \dim{Z}$
\item If $Z = Z(f)$ for irreducible $f$ then $\dim{Z} = n - 1$
\item For each $x \in Z$ we have $\dim{k[Z]_{\m_z}} = \dim{Z}$.
\end{enumerate}
\end{cor}

\begin{rmk}
$\dim{k[Z]_{\m_z}}$ corresponds to chains of irreducibles beginning at $Z_0 = \{ z \}$.
\end{rmk}

\begin{proof}
(c) is immediate. Now we do (a). We see that $U$ is irreducible because $\overline{U} = Z$ since $Y \subset U$ is closed then $\overline{Y} \cap U = Y$. Suppose that,
\[ Y_0 \subsetneq \cdots \subsetneq Y_n \subset U \]
is a chain of closed irreducible subsets of $U$ then,
\[ \overline{Y}_0 \subsetneq \cdots \subsetneq \overline{Y}_n \subset Z \]
is a chain of closed irreducible subsets of $Z$ since $\overline{Y}_i \cap U = Y_i$ so the containments are proper. Therefore, $\dim{U} \le \dim{Z}$. Now, by the previous theorem, we can choose a maximal chain such that $Z_0 = \{ z \}$ with $z \in U$ (the point can be chosen arbitrarily) and get,
\[ Z_0 \subsetneq \cdots \subsetneq Z_n = Z \]
so take $Y_j = Z_j \cap U$ which is clearly closed in $U$ and irreducible. Since $z \in U \cap Z_j$ we see that $Y_j$ is nonempty but open in $Z_j$ and thus $\overline{Y}_j = Z_j$ and thus the containments must be proper since $Z_j \subsetneq Z_{j+1}$. 
\bigskip\\
Finally to show (b) we apply the dimension formula,
\[ \dim{Z(f)} + \height{(f)} = n \]
so it suffices to prove that $(0) \subsetneq (f)$ is a minimal nonzero prime. However, for any nonzero $\p \subset (f)$ take an irreducible element $g \in \p$ (factor any element and by primality its irreducible factors are inside $\p$) and thus $(g) \subset \p \subset (f)$ so $g = fr$ but $g$ is irreducible and $f$ is not a unit so $r$ is a unit and thus $(g) = (f)$ so $\p = (f)$.
\end{proof}

\begin{prop}
Let $A$ be a Noetherian domain and suppose that $f \in A$ is nonzero and $(f)$ is prime. Then $(f)$ is a minimal nonzero prime.
\end{prop}

\begin{proof}
 then take $x \in \p$ so $x = fr$ but $f \notin \p$ so $r \in \p$ and thus $f \p = \p$. Thus if $\p$ is finitely generated (it is because we are in a Noetherian ring) then there is $r \in (f)$ such that $(r - 1) \p = 0$ by Nakayama but in a domain this implies $\p = 0$ because $r - 1 \neq 0$.
\end{proof}

\begin{rmk}
The closed sets $Z \subset \A^n$ whose irreducible components are all of dimension $n - 1$ are \textit{exactly} $Z = Z(f)$ for nonconstant $f \in k[x_1, \dots, x_n]$ (look at irreducible components $Z_j = Z(f_j)$).
\end{rmk}

\subsubsection{``Nice'' functions on algebraic sets}

We have $k[Z] = k[x_1, \dots, x_n] / I(Z) \embed \mathrm{Func}(Z, k)$ by sending $g \mapsto (z \mapsto g(z))$ because these functions by definition do not care about polynomials that vanish on $Z$. Consider $U = Z_f \subset Z$ then we get $\alpha_f : k[Z]_f \to \mathrm{Func}(Z_f, k)$ because $f$ is non-vanishing on $U$ and thus $f^{-1}$ makes sense as a function. 

\begin{defn}
For any open $U \subset Z$ nonempty we define,
\[ \struct{Z}(U) = \{ \varphi : U \to k \mid \forall u \in U : \exists u \in V \subset U : \varphi|_V = \tfrac{g}{h} \text{ for } g,h \in k[Z] \text{ and } h|_V \text{ nonvanishing } \} \]
\end{defn}

\begin{prop}
The map $\alpha_f : k[Z]_f \iso \struct{Z}(Y)$ is an isomorphism.
\end{prop}

\section{Sep. 27}

\subsection{Regular Functions and Coordinate Rings}

Let $Z \subset \A^n$ be an algebraic set. For nonempty open $U \subset Z$,
\[ \struct{Z}(U) = \{ \varphi : U \to k \mid \text{locally on } Z : \varphi \in \Frac{k[Z]} \} \]
furthermore $\struct{Z}(\empty) = (0)$. Last time: for $f \in k[Z]$ we said that $\alpha_f : k[Z]_f \to \struct{Z}(Z_f)$ via $\frac{g}{f^n} \mapsto \frac{g|_{Z_f}}{(f|_{Z_f})^r}$. We want to show that $\alpha_f$ is an isomorphism.

\begin{rmk}
$f = 0$ in $k[Z]$ iff $k[Z]_f = 0$ iff $Z_f = \empty$
\end{rmk}

\begin{example}

\end{example}
Let $Z = \{ xy = zw \} \subset \A^4$ then $k[Z] = k[x,y,z,w]/(xy - zw)$ is a domain because $xy - zw$ is irreducible. Therefore,
\[ \frac{x}{w} = \frac{z}{y} \in \Frac{k[Z]} \]
On $Z_w$ we have the regular function $\frac{x}{w}$ and on $Z_y$ we have the regular function $\frac{z}{y}$. Furthermore, they agree on $Z_w \cap Z_y = Z_{wy}$. Therefore, they glue to a regular function on $Z_w \cup Z_y$. However, we don't see $g,h \in k[Z]$ with $h|_U$ nonvanishing and $\varphi = \frac{g}{h}$.

\begin{example}
Consider $U = \A^2 \setminus \{ (0,0) \}$ which is Zariski open. We want to determine $\struct{\A^2}(U)$. However, $U = U_x \cup U_y$ where $U_f = Z_f^c$. Furthermore, $U_x \cap U_y = U_{xy}$. Granting the theorem of today, then $\struct{\A^2}(U_x) = k[x,y]_x$ and $\struct{\A^2}(U_y) = k[x,y]_y$ and $\struct{\A^2}(U_{xy}) = k[x,y]_{xy}$. These are domains so there are inclusions $k[x,y]_x \to k[x,y]_{xy}$ and $k[x,y]_y \to k[x,y]_y$. Therefore, inside $k[x,y]_{xy}$ we need to compute the intersection,
\[ k[x,y]_x \cap k[x,y]_y = k[x,y] \]
because $k[x,y]$ is a UFD with units $k^\times$.
Explicitly,
\begin{center}
\begin{tikzcd}
\struct{\A^2}(U_x) \arrow[rd] & & \struct{\A^2}(U_y) \arrow[ld]
\\
& \struct{\A^2}(U_{xy}) 
\end{tikzcd}
\end{center}
is the diagram,
\begin{center}
\begin{tikzcd}
k[x,y,x^{-1}] \arrow[rd] & & k[x,y,y^{-1}] \arrow[ld]
\\
& k[x,y,x^{-1},y^{-1}]
\end{tikzcd}
\end{center}
Equivalently, we can intersect inside the function field since $k[x,y,x^{-1},y^{-1}] \embed k(x,y)$. Therefore, $\struct{\A^2}(U) = k[x,y]$. 
\bigskip\\
This is the algebraic version of Harthog's theorem from several complex variables.
\end{example}

\begin{prop}
For $\varphi \in \struct{Z}(U)$ the map $\varphi : U \to \A^1$ is continuous.
\end{prop}

\begin{proof}
The only interesting closed sets in $\A^1$ are finite collections of points. Thus we only need to check that the preimage of a point is closed. Being closed is local (on all of $Z$ not just on the set that would be ``locally closed'') so we can choose an open cover $U_i$ such that $\varphi|_{U_i} = \frac{f}{g}$ with $g$ nonvanishing. Then,
\[ \varphi^{-1}(y) \cap U_i = \{ x \in U_i \mid f(x) - y g(x) = 0 \} = Z(f - y g) \]
is Zariski closed. Thus $\varphi^{-1}(y)$ is Zariski closed.
\end{proof}

\begin{prop}
The map $\alpha_f : k[Z]_f \to \struct{Z}(Z_f)$ is an isomorphism. 
\end{prop}

\begin{proof}
We can assume that $Z_f = \empty$ because in that case $f = 0$ and this is obvious. Now we reduce to the case $f = 1$ by the Rabinawitz trick. Let $Z \subset \A^n$ cut out by $I \subset k[x_1, \dots, x_n]$. Then consider,
\[ Y = \{ (\underline{x}, t) \in \A^{n+1} \mid \forall h \in I(Z) \, f(\underline{x}) t = 1 \} \]
I claim that $Z_f \cong Y$ and $Y$ is cut out by the ideal $I + (Tf - 1) \subset k[x_1, \dots, x_n, T]$. To see this, consider,
\[ k[x_1, \dots, x_n, T] / (I(Z), Tf - 1) = k[Z][T]/(T f - 1) = k[T]_f \]
which is reduced and thus $I + (T f - 1)$ is radical. 
\bigskip\\
Furthermore, there is a bijection $\xi$ between these sets,
\begin{align*}
(x \in Z_f) & \mapsto (\underline{x}, \tfrac{1}{f(x)}) 
\\
\underline{x} & \mapsfrom (\underline{x}, t) 
\end{align*}
This transforms the problem for $Z_f \subset Z$ into the problem for $Y$ with ``$f = 1$''. We need to check that,
\[ \struct{Z}(Z_f) \iso \struct{Y}(Y) \]
First we should check that $\xi$ is a homeomorphism. We need to check that $\xi^{-1}(Z)$ is closed where $Z$ is closed. Let $Z = (g_1, \dots, g_n)$ for $g_i \in k[\underline{x}, T]$. It suffices to show that $\xi^{-1}(Z(g))$ for one polynomial. Thus we need to show that,
\[ \xi^{-1}(Z(g)) = \{ x \in Z_f \mid g(\underline{x}, f(\underline{x})^{-1}) = 0 \} \]
is closed. However, there is a largest power of $f^{-1}$ in this expression so,
\[ g(\underline{x}, f(\underline{x})^{-1}) = \frac{p(\underline{x})}{f(\underline{x})^n} \]
but the left hand size is a polynomial $p \in k[x_1, \dots, x_n]$. Thus because $f$ is nonvanishing on $Z_f$ we see that $\xi^{-1}(Z(g)) = Z(p) \cap Z_f$ and thus is closed in $Z_f$. Now we need to show that $\xi$ is open. Now for $q \in k[x_1, \dots, x_n]$ we need to show that $\xi(Z(q) \cap Z_f)$ is closed but,
\[ \xi(Z(q) \cap Z_f) = \{ (\underline{x}, t) \in Y \mid q(\underline{x}) = 0 \} \]
which is closed. It remains to prove that the following diagram commutes,
\begin{center}
\begin{tikzcd}
k[Z]_f \arrow[r] & \struct{Z}(Z_f) \arrow[d,equals, "\xi"]
\\
k[Y] \arrow[u, "g \mapsto g \circ f^{-1}"] \arrow[r] & \struct{Y}(Y)
\end{tikzcd}
\end{center}
Therefore, we have reduced to showing that $\alpha : k[Y] \to \struct{Y}(Y)$ is an isomorphism. 
\bigskip\\
Therefore we do the case $f = 1$. Pick $\varphi \in \struct{Z}(Z)$ we want $\varphi \in k[Z]$. For all $P \in Z$ there is an open $V_P \subset Z$ such that $\varphi|_{V_p} = \frac{g_P}{h_P}$, for $g_P, h_P \in k[Z]$ and $h_P|_{V_P}$ is non-vanishing. Shrink $V_P$ to some $P \in Z_{H_P} = \{ H_p \neq 0 \}$ in $Z$. Shrink further to $\{ h_P H_P \neq 0 \}$ and $\varphi$ on here is,
\[ \varphi|_{Z_{H_P}} = \frac{g_P H_P}{h_P H_P} \]
which we rename to $\frac{g_P}{h_P}$. Thus $Z$ is covered by $Z_{h_P}$ with $\varphi|_{Z_{h_P}} = \frac{g_P}{h_P}$. By HW1, a Noetherian space is quasi-compact , so $Z$ is covered by finitely many,
\[ Z_{h_1}, \dots, Z_{h_m} \]
with $\varphi|_{Z_{h_j}} = \frac{g_j}{h_j}$ for $g_j, h_j \in k[Z]$. That these cover say that $\{ h_j \}$ have no common zero in $Z$, so $(h_1, \dots, h_m) = (1)$ in $k[Z]$ else it would contain some maximal ideal and thus by the Nullstellensatz it would vanish. Then we apply the following ring theory fact. If $M$ is an $R$ module and $(f_1, \dots, f_n) = (1)$ then the following sequence is exact,
\begin{center}
\begin{tikzcd}
0 \arrow[r] & M \arrow[r] & \prod\limits_{i = 1}^n M_{f_i} \arrow[r] & \prod\limits_{i,j} M_{f_i f_j} 
\end{tikzcd}
\end{center}
\end{proof}

\section{Sept. 29}

\subsection{Morphisms of Algebraic Sets}

For $C^\infty$-manifolds $M,N$ a $C^\infty$-morphism $\varphi : M \to N$ satisfies,
\begin{enumerate}
\item $\varphi$ is continuous
\item $\varphi^* : C^\infty(V) \to C^{\infty}(\varphi^{-1}(V))$ for all open $V \subset N$.
\end{enumerate}

\begin{prop}
Conversely, if $\varphi : M \to N$ satisfies (a) and (b) then it is $C^\infty$.
\end{prop}

\begin{proof}
Choose an open cover $\{ W_\alpha \}$ on $W$ such that $\varphi(W_\alpha) \subset V_\alpha$ which is a chart of $V_\alpha$ (using that $\varphi$ is continuous to choose open $W_\alpha \subset \varphi^{-1}(V_\alpha)$ around each point). If $V_\alpha \subset \RR^n$ is open. Choose coordinates $x_1, \dots, x_n$ then, $W_\alpha \xrightarrow{\varphi} V_\alpha \subset \RR^n$ is $w \mapsto (\varphi_1(w), \dots, \varphi_2(w))$ where $\varphi_j = x_j \circ \varphi = \varphi^*(x_j)$ for the local coordinate functions $x_j : V_\alpha \to \RR$. By (b) we know that $\varphi_j = x_j \circ \varphi = \varphi^*(x_j)$ is a smooth function on $V_\alpha$. Therefore, $\varphi$ is $C^\infty$ because $W_\alpha \to V_\alpha \subset \RR^n$ is smooth.
\end{proof}

\begin{defn}
For algebraic sets $Z_1 \subset \A^n$ and $\Z_2 \subset \A^m$ and open $U_j \subset Z_j$, a \textit{morphism} $\varphi : U_1 \to U_2$ is a continuous map such that for any open $V \subset U_2$ and $f$ a regular function on $V$ then $\varphi^* f = f \circ \varphi|_V$ is a regular function on $\varphi^{-1}(V)$.
\end{defn}

\begin{rmk}
Given a morphism $\varphi : U_1 \to U_2$ we get a map of $k$-algebras $\varphi^* : \struct{Z_2}(V) \to \struct{Z_1}(\varphi^{-1}(V))$  for each open $V \subset U_2$. This notion is Zariski-local meaning ``morphisms glue''. Furthermore, it is immediate that morphisms compose.
\end{rmk}

\begin{rmk}
There appears to be a defect in this definition: it seems to depend on an ambient $Z_j \supset U_j$ and how $Z_j$ are embedded in affine space.
\end{rmk}

\begin{example}
Given a map $\alpha : k[Z_2] \to k[Z_1]$ of coordinate rings consider coordinates,
\begin{center}
\begin{tikzcd}
k[Y_1, \dots, Y_m] \arrow[d, two heads] \arrow[r, dashed, "\Phi"] & k[X_1, \dots, X_n] \arrow[d, two heads]
\\
k[Z_2] \arrow[r, "\alpha"] & k[Z_1] 
\end{tikzcd}
\end{center}
corresponding to the embeddings $Z_1 \embed \A^n$ and $Z_2 \embed \A^m$. These maps have ideals $J_i = I(Z_i)$. Consider $\alpha : \bar{Y}_j \to \overline{h_j(X)}$. Define $\Phi$ by $Y_j \mapsto h_j(X)$ which is not unique but it is unique up to $J_1$. Then we get a corresponding diagram of algebraic sets,
\begin{center}
\begin{tikzcd}
\A^m & \A^n \arrow[l, "\tilde{\Phi}"]
\\
Z_2 \arrow[u, hook] & Z_1 \arrow[l, "\tilde{\alpha}"] \arrow[u, hook]
\end{tikzcd}
\end{center}
be defining,
\[ \tilde{\Phi}(x_1, \dots, x_n) = (h_1(x), \dots, h_m(x)) \]
Then I claim that $\tilde{\Phi}(Z_1) \subset Z_2$ and $Z_1 \to Z_2$ depends only on $\alpha$. Ambiguity in $h_j$ is $I(Z_1)$ which vanishes on $Z_1$ so $\tilde{\Phi}|_{Z_1}$ is independent of choices. Since $Z_2 = Z(J_2)$ it is enough to show for $g \in J_2$ that $g \circ \tilde{\Phi} = 0$. But $g \circ \tilde{\Phi} = \Phi(g)$, it is enough to check this on generators of the polynomial algebra $g \in k[Y_1, \dots, Y_m]$ but $Y_j \circ \tilde{\Phi} = h_j = \Phi(Y_j)$ by definition. Then for $g \in J_2$ we have $g \circ \tilde{\Phi} = \Phi(g) \in \Phi(J_2) \subset J_1$ by the diagram. 
\bigskip\\
Now we show that $\tilde{\alpha}$ is a morphism. To check that $\tilde{\alpha}$ is continuous, it suffices to show that $\tilde{\Phi}$ is continuous because the $Z_j$ have the subspace topology. However, 
\begin{align*}
\tilde{\Phi}(Z(I)) & = \{ x \in \A^n \mid \tilde{\Phi}(x) \in Z(I) \} = \{ x \in \A^n \mid \forall g \in I : g(\tilde{\Phi}(x)) = 0 \} = \{ x \in \A^n \mid \forall g \in I : \Phi(g)(x) = 0 \} 
\\
& = Z(\Phi(I))
\end{align*}
is closed. Regular functions on $Z_1$ are ``locally restrictions of reg. functions on $\A^n$'' and similarly for $Z_2$, so it is enough to show $\tilde{\Phi}$ pulls back regular functions to regular functions.
\bigskip\\
For open $V_2 \subset \A^m$ and $f \in \struct{\A^m}(V_2)$ want $f \circ \tilde{\Phi}$. This is local on $V_2$ because $\tilde{\Phi}$ is continuous. We can shrink $V_2$ so WLOG $V_2 = U_h = \{ h \neq 0 \}$ for some $h \in k[Y]$. We can choose sufficiently small $U_h$ (depending on $f$) such that $f|_{U_h} = \frac{g}{h}$ for $g,h \in k[Y]$. But, we computed that
\[ \tilde{\Phi}(Z(h)) = Z(\Phi(h)) \]
and thus by taking complements $\tilde{\Phi}^{-1}(U_h) = U_{\Phi(h)}$. Furthermore,
\[ \frac{g}{h} \circ \tilde{\Phi} = \frac{g \circ \tilde{\Phi}}{h \circ \tilde{\Phi}} = \frac{\Phi(g)}{\Phi(h)} \]
which is manifestly a regular function on $U_{\Phi(h)}$. Therefore $\tilde{\alpha}$ is a morphism.
\bigskip\\
Observe that,
\begin{center}
\begin{tikzcd}
\struct{Z_2}(Z_2) \arrow[r, "\tilde{\alpha}^*"] \arrow[d, equals] & \struct{Z_1}(Z_1) \arrow[d, equals]
\\
k[Z_2] \arrow[r, "\alpha"] & k[Z_1] 
\end{tikzcd}
\end{center}
commutes so $\tilde{\alpha}$ induces $\alpha$. Therefore we see that,
\[ \Hom{\text{alg}}{Z_1}{Z_2} \to \Hom{k\text{-alg}}{\struct{Z_2}(Z_2)}{\struct{Z_1}(Z_1)} \]
via $\varphi \mapsto (f \mapsto f \circ \varphi)$. We showed that this is surjective by given $\alpha$ constructing $\tilde{\alpha}$. We constructed a map,
\[  \Hom{k\text{-alg}}{\struct{Z_2}(Z_2)}{\struct{Z_1}(Z_1)} \to \Hom{\text{alg}}{Z_1}{Z_2} \]
such that $\alpha \mapsto \tilde{\alpha} \mapsto \alpha$ and thus a section. We next need to check that the map $\alpha \mapsto \tilde{\alpha}$ is injective or equivalently that $\tilde{\alpha} \mapsto \alpha \mapsto \tilde{\alpha}$. 
\end{example}

\begin{rmk}
If $A \to B$ is a map and $I \subset A$ and ideal then $(\varphi^*)^{-1}(V(I)) = V(\varphi(I))$ because,
\[ \varphi^{-1}(\p) \supset I \iff \p \supset \varphi(I) \]
Indeed, if $\varphi^{-1}(\p) \supset I$ then $\p \supset \varphi(\varphi^{-1}(\p)) \supset \varphi(I)$ and if $\p \supset \varphi(I)$ then $\varphi^{-1}(\p) \supset \varphi^{-1}(\varphi(I)) \supset I$. This was really an important aspect of the above proof.
\end{rmk}

\begin{rmk}
$\tilde{\Phi}:  \A^n \to \A^m$ then $\Phi^{-1}(\m_P) = \m_{\tilde{\Phi}(P)}$ because $\Phi(f)(P) = f(\tilde{\Phi}(P))$ and thus,
\[ f \in \Phi^{-1}(\m_P) \iff \Phi(f)(P) = 0 \iff f(\tilde{\Phi}(P)) = 0 \iff f \in \m_{\Phi(P)} \]
Therefore, this describes $\tilde{\Phi}$ in terms of $\Phi$ intrinsically. Consequently, $\tilde{\alpha} : Z_1 \to Z_2$ satisfies $\alpha^{-1}(\m_{Q_1}) = \m_{\tilde{\alpha}(Q_1)}$. 
\end{rmk}

\begin{example}
Let $Z_1 = Z(x - y^2)$ and $Z_2 = \A^1$. Consider $(x,y) \mapsto x$. Then $k[Z_1] = k[x,y]/(y^2 - x) \cong k[y]$ and $k[Z_2] = k[x]$ and the map $k[Z_2] \to k[Z_1]$ via $x \mapsto x = y^2$. The map $k[x] \to k[y]$ is finite free of rank $2$.
\end{example}

\begin{example}
Let $\varphi : \A^1 \to \A^3$ be given by $t \mapsto (t^3, t^4, t^5)$ then $k[t] \xleftarrow{\alpha} k[x,y,z]$ via $x \mapsto t^3$ and $y \mapsto t^4$ and $y \mapsto t^5$. Let $J = \ker{\alpha}$ which is prime since $k[x,y,z]/J \embed k[t]$ is thus a domain. The image of $\varphi$ is clearly contained in $Z(J)$ but $k[x,y,z]/J \embed k[t]$ is a finite extension and thus surjective on maxspec so $\varphi$ is surjective onto $Z(J)$. 
\end{example}

\section{Oct. 1}

\begin{thm}
For algebraic sets $Z_1 \subset \A^n$ and $Z_2 \subset \A^m$ the map,
\[ \Hom{\text{algset}}{Z_1}{Z_2} = \Hom{k\text{-alg}}{\struct{Z_2}(Z_2)}{\struct{Z_1}(Z_1)} \]
via $(\varphi : Z_1 \to Z_2) \mapsto (f \mapsto f \circ \varphi)$
is bijective.
\end{thm}

\begin{proof}
Last time, we proved surjectivity by, given a $k$-algebra map,
\[ \alpha : k[Z_2] \to k[Z_1] \]
we build $\tilde{\alpha} : Z_1 \to Z_2$ such that $f \circ \tilde{\alpha} = \alpha(f)$. For injectivity, given $\varphi, \psi : Z_1 \to Z_2$ that induce the same map $k[Z_2] \to k[Z_1]$ then we want to show that $\varphi = \psi$. Composing,
\[ Z_1 \to Z_2 \embed \A^m \]
then the compositions also act identically on coordinate functions. Therefore, we may assume we have maps $\varphi, \psi : Z_1 \to \A^m$. However, these are the maps $x \mapsto (h_1(x), \dots, h_m(x))$ where 
\[ h_j = Y_j \circ \varphi = Y_j \circ \psi \]
and therefore $\varphi = \psi$.
\end{proof}

\begin{rmk}
For $Z$ an affine variety and $U,V \subset Z$ non-empty open then,
\[ k[Z] \embed \struct{Z}(U), \struct{Z}(V) \subset \Frac{k[Z]} = k(Z) \]
since $k[Z]$ is a domain and,
\[ \struct{Z}(U) \cap \struct{Z}(V) = \struct{Z}(U \cup V) \]
\end{rmk}

\begin{cor}
For $U \subset Z \embed \A^n$ with $U$ open in $Z$ and $Z \embed \A^n$ closed. Then,
\[ \Hom{}{U}{\A^m} \xrightarrow{\dagger} \struct{Z}(U)^{\oplus m} \]
given by $\varphi \mapsto (Y_1 \circ \varphi, \dots, Y_m \circ \varphi)$ is bijective.
\end{cor}

\begin{proof}
If $U = Z_f$ is a basic affine open then by Rabinowitz trick we have that $U$ is an algebraic set so we use the previous result to get,
\[ \Hom{}{Z_f}{\A^m} = \Hom{k}{k[Y_1, \dots, Y_m]}{k[U_f]} = k[U_f]^{\oplus m} \]
Then, a general $U$ is covered by finitely many principal opens $Z_{f_1}, \dots, Z_{f_r}$ with $Z_{f_i} \cap Z_{f_j} = Z_{f_i f_j}$ so we can ``glue'' using the functoriality of $\dagger$.
\end{proof}

\begin{rmk}
The corollary says that,
\[ \Hom{}{-}{\A^m} = \Hom{}{-}{\A^1}^{\oplus m} \]
so we begin to see that $\A^m$ is the $m$-fold product of $\A^1$ in the category of algebraic sets.
\end{rmk}

\subsection{Local Rings and Smoothness}

For algebraic set $Z \subset \A^n$ and $P \in Z$, ``local'' properties of $Z$ at $P$ are encoded in the ``local ring'' at $P$.

\begin{defn}
$\stalk{Z}{P} = \varinjlim_{P \in U} \struct{Z}(U)$ where this is a directed system of $k$-algebras via $U_1 \ge U_2$ if $U_1 \subset U_2$ then $\struct{Z}(U_1) \leftarrow \struct{Z}(U_2)$.
\end{defn}

\begin{rmk}
For $C^\infty$ or $C^\omega$ manifolds there are corresponding analytic notions called ``germs'' furthermore there is a diagram,
\begin{center}
\begin{tikzcd}
\mathcal{C}^\omega_{\RR^n,0} \arrow[rr, hook] \arrow[rd, hook] & & \mathcal{C}^\infty_{\RR^n,0} \arrow[ld, two heads]
\\
& \RR[[x_1, \dots, x_n]]
\end{tikzcd}
\end{center}
where $\mathcal{C}^\infty_{\RR^n,0} \onto RR[[x_1, \dots, x_n]]$ a theorem of Borel.
\end{rmk}

\begin{example}
$\stalk{\A^n}{a} = \left\{ \frac{f(x)}{g(x)} \middle| g(a) \neq 0 \right\} \subset k(x_1, \dots, x_n) = k(\A^n)$
\end{example}

\begin{thm}
If $\m \subset k[Z]$ is a maximal ideal corresponding to $P \in Z_1$ then,
\begin{center}
\begin{tikzcd}
k[Z] \arrow[r] \arrow[rd] & \stalk{Z}{P} \arrow[d, dashed]
\\
& k[Z]_\m
\end{tikzcd}
\end{center}
there is a unique isomorphism making this diagram commute. Moreover, for open $U$ containing $P$ then the map $\struct{Z}(U) \to \stalk{Z}{P} \onto k$ is evaluation $f \mapsto f(P)$. 
\end{thm}

\begin{proof}
To compute $\stalk{Z}{P}$ it is enough to use the cofinal poset of principal opens $U = Z_h$ for $h \in k[Z]$ with $P \in Z_h$. Then we have,
\[ \struct{Z}(Z_h) = k[Z]_h \]
Now $Z_h \subset Z_{h'} \iff h' \divides h^n$ for some $n$ because $Z_h \subset Z_{h'}$ iff $Z(h) \supset Z(h')$ iff $\sqrt{(h)} \subset \sqrt{(h')}$ iff $h \in \sqrt{(h')}$ iff $h^n \in (h')$. Therefore, 
\[ \stalk{Z}{P} = \varinjlim_{h, h(P) \neq 0} k[Z]_h \]
with $h \ge h'$ iff $h' \divides h^n$ for some $n$. Therefore, we are localizing with increasing subsets of $k[Z] \setminus \m$ and thus,
\[ \stalk{Z}{P} = \varinjlim_{h, h(P) \neq 0} k[Z]_h = k[Z]_\m \]
To see this,
\begin{center}
\begin{tikzcd}
k[Z]_{h'} \arrow[r] \arrow[d, dashed] & k[Z]_\m
\\
k[Z]_h \arrow[ru]
\end{tikzcd}
\end{center}
where $h'$ is a unit in $k[Z]_h$ because $h' \divides h^n$ giving the existence of the downward map. 
\end{proof}

\begin{rmk}
Prime of $\stalk{Z}{P} = k[Z]_\m$ are the primes $\p$ of $k[Z]$ inside $\m$ equivalent to the irreducible closed $Y \subset Z$ through $P$. Therefore minimal primes of $\stalk{Z}{P}$ are the irreducible components passing through $P$. Now,
\[ \dim{\stalk{Z}{P}} = \max_{P \in Z} \dim{Z} \]
\end{rmk}

\begin{defn}
If $I(Z0 = (f_1, \dots, f_r) \subset k[X_1, \dots, X_n]$ for $Z \embed \A^n$ say $Z$ is \textit{smooth} at $P \in Z$ if,
\[ \mathrm{rank} \left( \pderiv{f_i}{x_j} (P) \right) = n - \dim{ \stalk{Z}{P}} \]
\end{defn}

\section{Oct. 4}

\subsection{Tangent Spaces and Presheaves}

For $P \in Z$ an algebraic set, say $P$ is \textit{smooth} on $Z$ if the inequality,
\[ \rank{(J(P))} \le n - \dim{\stalk{Z}{P}} \]
is an equality. This is equivalent to $\stalk{Z}{P}$ being an equality.

\begin{defn}
The Zariski tangent space is defined by,
\[ T_p(Z) = \Hom{k}{\m_P/\m_P^2}{k} \]
where $\m_P$ is the maximal ideal of $\stalk{Z}{P}$.
\end{defn}

\begin{rmk}
To give some motivation, consider a $C^\infty$-manifold $M$, and $a \in M$ then,
\[ T_a(M) = \Hom{\RR}{\m_a/\m_a^2}{\RR} \]
for the maximal ideal $\m_a$ of the local ring $\stalk{M}{a}$ of gems of $C^\infty$-functors near $a$. Then $\stalk{M}{a} \cong \RR \oplus \m_a$ and $T_a(M)$ is the space of point derivations at $a$ on $\stalk{M}{a}$.
\bigskip\\
Now we apply the Hadamard lemma: any $f \in \stalk{M}{a}$ can be written in the following form,
\[ f(x) = f(a) + \sum_{i = 1}^n (x - a)_i g_i(x) \]
for some smooth functions $g_i$.
\end{rmk}

\begin{example}
Let $Z \embed \A^n$ corresponding to $k[\A^n] \onto k[Z]$ then $M - (x_j - x_j(P))_j \onto \m_P$ and thus $M/M^2 \onto \m_P / \m_P^@$ so we get a map,
\[ T_P(Z) \embed T_P(\A^n) = \bigoplus_{i = 1}^n k (x_j - x_j(P)) \cong k^n \]
Then if $I(Z) = (f_1, \dots, f_r)$ then,
\[ T_P(Z) = \ker{J(P)} = \left\{ t \in k^n \, \middle| \, \forall i : \sum_{j} \pderiv{f_i}{x_j}(P) \, t_j = 0 \right\} \]
To see this,
\[ k[Z] = k[X_1, \dots, X_n]/(f_1, \dots, f_r) \]
so $\m_P = (X_1 - X_1(P), \dots, X_n - X_n(P)) / (f_1, \dots, f_r)$ where this makes sense because all $f_i$ vanish at $P$. Therefore,
\begin{align*}
\m_P / \m_P^2 & = (X_j - X_j(P))_j / ((X_i - X_i(P))(X_j - X_j(P)), f_1, \dots, f_r)
\\
& = \frac{\bigoplus_{j = 1}^r (X_j - X_j(P))}{\left< \sum_j \pderiv{f_i}{x_j}(P) (X_j - X_j(P)) \right>} \cong \frac{k^n}{\Im{\nabla f(P)}}
\end{align*}
Then,
\[ \Hom{k}{\m_P/\m_P^2}{k} \cong \ker{\nabla f(P)} \]
\end{example}

\begin{example}
If $Z = Z(f)$ for radical $f$ nonconstant so $\dim{Z} = n-1$ and $\dim{\stalk{Z}{P}} = n  -1$ (so all irreducible components of $Z$ have dimension $n-1$). Then,
\[ T_P(Z) = (\Delta f(P))^\perp \subset k^n \]
so $Z$ is \textit{smooth} at $P$ iff $(\Delta f)(P) \neq 0$ iff \textit{some},
\[ \pderiv{f}{x_j}(P) \neq 0 \]
For example, if,
\[ f = \sum X_j^2 \]
in characteristic not $2$ and $n \ge 2$ then $Z(f)$ is smooth except at the origin.
\end{example}

\begin{example}
For $n = 2$, radical $f \in k[x,y]$ non-constant then $C = Z(f)$ is a curve in $\A^2$. Now $C$ is smooth at $P$ iff $\pderiv{f}{x}(P)$ or $\pderiv{f}{y}(P)$ is nonzero. At a smooth point, $\stalk{C}{P}$ is a regular $1$-dimensional local Noetherian domain and thus a DVR [Mat, Thm.11.2]. 
\bigskip\\
We might ask how to detect when $h \in \m_{P}$ (e.g. $h \in k[x,y]$ with $h(P) = 0$) is a uniformizer. The valuation at a smooth point $P$ corresponds to the order of tangency of $C$ with the curve defined by $h$. Therefore, $h$ gives a uniformizer if it is \textit{not} tangent to $C$ i.e. its linear part is not a scalar multiple of $(\nabla f)(P)$. To see this,
\[ \m_P / \m_P^2 = \frac{k^2}{k (\nabla f)(P)} \]
so we need that $\bar{h} \in \m_P / \m_P^2$ is nonzero and thus its linear part should not be in the subspace $k (\nabla f)(P) \subset k^2$. Therefore, the line,
\[ \ell = V(c_1 (x-a) + c_2(y - b)) \]
gives a uniformizer when it cuts the curve at $P$ if and only if $(c_1, c_2) \notin k \left( \pderiv{f}{x}(P), \pderiv{f}{y}(P) \right)$.
\end{example}

\subsection{Schemes}

To build up schemes we want,
\begin{enumerate}
\item to allow any field $k$
\item to remove the ambient $\A^n$
\item allow rings not containing a field (e.g. ``geometry over $\Z$'').
\end{enumerate}
\noindent 
For manifolds there are two points of view:
\begin{enumerate}
\item charts and atlases
\item sheaves of rings on a topological space.
\end{enumerate}

\begin{defn}
Let $X$ be a topological space, $\C$ a category with a final object. Then a \underline{presheaf} on $X$ valued in $\C$ is an assignment,
\[ \F : U \mapsto \F(U) \]
for open subsets $U \subset X$ and for each containment $V \subset U$ a restriction map $\res_{V,U} : \F(U) \to \F(V)$ such that given $W \subset V \subset U$ we have $\res_{W,V} \circ \res_{V,U} = \res_{W,U}$ and $\res_{U,U} = \id$.
\end{defn}

\begin{example}
\begin{enumerate}
\item Let $X$ be a topological space. Take,
\[ \F(U) = \{ \varphi : U \to \RR \mid \varphi \text{ is continuous} \} \]
\item Let $X$ be a $C^\infty$-manifold. Take,
\[ \struct{X} : U \mapsto C^\infty(U) \]
or,
\[ \Omega^3_X : U \mapsto \{ \text{smooth 3-forms on } U \} \]
or also,
\[ \mathrm{Vec}_X : U \mapsto \{ C^\infty \text{ vector fields on } U \} \]
and also,
\[ \struct{X}^\times : U \mapsto \{ \varphi \in C^\infty(U) \mid \forall x \in U : \varphi(x) \neq 0 \} \]
\item Let $Z$ be an algebraic set, with the Zariski topology then,
\[ \struct{Z} : U \mapsto \{ \text{ regular functions on } U \} \]
is a presheaf of $k$-algebras.
\item If $A$ is an abelian group then I can form two presheaves,
\[ \F_A : U \mapsto A \text{ and } \res_{V,U} = \id \]
or you can form,
\[ \underline{A} : U \to \{ \varphi : U \to A \mid \varphi \text{ is continuous where } A \text{ has the discrete topology} \} \]
(Notice that $\underline{A}(\empty) = 0$ while $\F_A(\empty) = A$). 
\end{enumerate}
\end{example}

\begin{defn}
A \underline{morphism} $\varphi : \F \to \G$ of presheaves is a collection $f_U : \F(U) \to \G(U)$ which are compatible in the sense that,
\begin{center}
\begin{tikzcd}
\F(U) \arrow[r, "f_U"] \arrow[d, "\res_{V,U}^\F"'] & \G(U) \arrow[d, "\res_{V,U}^\G"]
\\
\F(V) \arrow[r, "f_V"] & \G(V)
\end{tikzcd}
\end{center}
If all $f_U$ are monic we say that $\F$ is a \underline{subpresheaf} of $\G$.
\end{defn}

\begin{example}
The map $\d : \Omega_X^2 \to \Omega_X^3$ is a map of presheaves on a $C^\infty$-manifold $X$. Furthermore, there is a sequence of presheaves,
\begin{center}
\begin{tikzcd}
\struct{X} \arrow[r, "\exp", hook] & \struct{X}^\times \arrow[r, "\mathrm{sign}"] & \underline{\{ \pm 1 \}} 
\end{tikzcd}
\end{center}
When $X$ is a complex manifold then $\struct{X}$ is the sheaf of holomorphic functions the exponential map has a kernel. There is a sequence,
\begin{center}
\begin{tikzcd}
\underline{2 \pi i \Z} \arrow[r] & \struct{X} \arrow[r, "\exp"] & \struct{X}^\times 
\end{tikzcd}
\end{center}
but $\exp$ is not surjective as presheaves because the global logarithm does not exist. Therefore we need to upgrade our notion to sheaves so capture this local data.
\end{example}

\section{Sheaves and Ringed Spaces}

Let $\C$ be a category with final object: (e.g. $\mathbf{Ab}$, $\mathbf{Ring}$, $\mathbf{Set}$). Assume $\C$ is a subcategory of $\mathbf{Set}$ compatibly with,
\begin{enumerate}
\item $\prod_{i \in I}$
\item $\varinjlim$
\item and the final object
\end{enumerate}

\begin{defn}
A $\C$-valued sheaf on a topological space $X$ is a $\C$-valued presheaf $\F$ on $X$ such that, for any open $U \subset X$ and any open cover $\{ U_\alpha \}_{\alpha \in A}$ of $U$, the diagram,
\begin{center}
\begin{tikzcd}
\F(U) \arrow[r] & \prod\limits_{\alpha} \F(\alpha)  \arrow[r,shift left=.75ex]  \arrow[r,shift right=.75ex] & \prod\limits_{(\beta, \gamma)} \F(U_\alpha \cap U_\beta) 
\end{tikzcd}
\end{center}
given by $s \mapsto (s |_{U_\alpha})$ and $(s_\alpha) \mapsto (s_\beta|_{U_{\beta \gamma}})$ or $(s_\gamma|_{U_{\beta \gamma}})$ via the two maps is an equalizer diagram. 
\end{defn}

\begin{rmk}
The says that for any $(s_\alpha)_{\alpha \in A}$ agreeing on overlaps, there exists a unique $s \in \F(U)$ s.t. $s|_{U_\alpha} = s_\alpha$.
\end{rmk}

\begin{rmk}
Some remarks on the definition,
\begin{enumerate}
\item If $\C = \mathbf{Ab}$, we can instead use the ``difference'' map,
\[ (s_\alpha) \mapsto (s_\beta |_{U_{\beta \gamma}} - s_\gamma |_{U_{\beta \gamma}}) \]
\item Often $U_{\beta \gamma} = \empty$
\item For any sheaf, $\F(\empty)$ is the final object. We could just insist on this but it is a consequence of the definition because the empty product is the final object.
\end{enumerate}
\end{rmk}

\begin{example}
For any abelian group $B$, we see that $\underline{B}$ is a sheaf but the ``very constant'' presheaf $\F_B$ is not a sheaf because it does not glue property on a disconnected cover. 
\end{example}

\begin{example}
Sheaves are the same notion of morphisms in the sense that they can be locally specified and glue.
\end{example}

\begin{example}
Non-example: let $M$ be a $C^\infty$-manifold,
\[ \F_k : U \mapsto \bigwedge^k_{\struct{M}(U)}(\Omega^1_M(U)) \]
is not a sheaf because locally wedges of $1$-forms glue to a global $k$-form but this need not be globally a wedge of global $1$-forms. However, $\Omega^k_M$ is the ``wedge product of sheaves'' but this only means that locally $k$-forms are wedges of $1$-forms.
\end{example}

\begin{defn}
The stalk of a presheaf $\F$ at $x \in X$ is,
\[ \F_x = \varinjlim_{U \ni x} \F(U) \]
\end{defn}

\begin{example}
Consider the following examples,
\begin{enumerate}
\item If $X$ is a manifold or an algebraic set, we have,
\[ \stalk{X}{x} = \varinjlim_{U \ni x} \struct{X}(U) \]
are the ``germs of [nice] functions at $x$''. 
\item
For any topological space, $A \iso (\underline{A})_x$ by sending $a \mapsto \text{constant function } a$. However, we also have $A \iso (\F_A)_x$. So even though $\underline{A}$ and $\F_A$ have the same stalks one is a sheaf and the other is not.
\item Consider a covering space $\pi : E \to X$. Introduce $\Sigma_{E/X}$ the sheaf of sections of $\pi$. Meaning,
\[ \Sigma_{E/X}(U) = \{ s : U \to E \mid \pi \circ s = \id_U \} \]
is the set of continuous sections. This is a sheaf of sets because continuous maps on opens glue. Now,
\[ (\Sigma_{E/X})_x \iso \pi^{-1}(x) \]
by evaluating the section at $x$. This is because $\pi : E \to X$ is locally a product $U \times \pi^{-1}(x)$ and the section condition completely determines $s : U \to U \times \pi^{-1}(x) \to U$ (the map to the first factor) to be the identity. For example, if $E = A \times X$ with $A$ discrete then $\underline{A} \cong \Sigma_{E/X}$ as sheaves of sets.
\item Consider $C^\infty$-map $\varphi : X \to Y$ for $C^\infty$-manifolds $X, Y$ we have maps,
\[ \struct{Y}(U) \to \struct{X}(f^{-1}(U)) \]
but NOT a map $\struct{Y} \to \struct{X}$. To make sense of a map of sheaves between two spaces we need to introduce a way of moving a sheaf on one space to another.
\end{enumerate}
\end{example}

\begin{defn}
For a continuous map $f : X \to Y$ of topological spaces, $\F$ a presheaf on $X$, the \textit{pushforward} $f_* \F$ on $Y$ is,
\[ (f_* \F)(U) = \F(f^{-1}(U)) \]
\end{defn}

\begin{rmk}
Easy to check that if $\F$ is a sheaf then $f_* \F$ is also a sheaf.
\end{rmk}

\begin{rmk}
In the preceding example: have ``pullback on functions'' $\struct{Y} \to f_* \struct{X}$.
\end{rmk}

\begin{rmk}
Given $\alpha : \F \to \G$ on $X$ then we get $f_* \alpha : f_* \F \to f_* \G$ via,
\[ \alpha_{f^{-1}(U)} : \F(f^{-1}(U)) \to \G(f^{-1}(U)) \]
Suppose we have continuous maps $X \xrightarrow{f} Y \xrightarrow{g} Z$ and a sheaf $\F$ on $X$ then $g_* (f_* \F) \iso (g \circ f)_* \F$.
\end{rmk}

\begin{example}
Let $f : X \to Y$ be a map of ringed spaces and $y = f(x)$. Then the sheaf map $\struct{Y} \to f_* \struct{X}$ gives a map on stalks,
\[ \stalk{Y}{y} \to (f_* \struct{X})_y = \varinjlim_{U \ni y} \struct{X}(f^{-1}(U)) \to \stalk{X}{x} \]
\end{example}

\begin{example}
Let $\iota : Y \embed X$ be a closed submanifold. Then,
\[ (\iota_* \struct{Y})_x = 
\begin{cases}
\stalk{Y}{x} & x \in Y 
\\
0 & x \notin Y
\end{cases} \]
\end{example}

\begin{defn}
Assume $\C$ is a category of rings (e.g. $k$-algebras, $A$-algebras, all CRings). A $\C$-\textit{ringed space} is a pair $(X, \struct{X})$ for a $\C$-valued sheaf of rings $\struct{X}$. A \textit{morphism} $(X, \struct{X}) \to (Y, \struct{Y})$ is a pair $(f, f^\#)$ where,
\begin{enumerate}
\item $f : X \to Y$ is continuous
\item $f^\# : \struct{Y} \to f_* \struct{X}$ is a map of sheaves of rings
\end{enumerate}
\end{defn}

\begin{rmk}
NOTE: $f^\#$ may not be determined by $f$.
\end{rmk}

\begin{rmk}
Composition is given as follows,
\[ (X, \struct{X}) \to (Y, \struct{Y}) \to (Z, \struct{Z}) \]
has composition $g \circ f : X \to Y$ and map of sheaves,
\begin{center}
\begin{tikzcd}
\struct{Z} \arrow[d, "g^\#"] \arrow[r, dashed] & (g \circ f_* \struct{X} \arrow[d, equals]
\\
g_* \struct{Y} \arrow[r, "g_* f^\#"] & g_* (f_* \struct{X})
\end{tikzcd}
\end{center}
\end{rmk}

\section{Oct. 8}

\subsection{Locally Ringed Spaces}

For ringed spaces $(X, \struct{X})$ and $(Y, \struct{Y})$, a \textit{morphism} $\varphi : (f, f^\#) : (X, \struct{X}) \to (Y, \struct{Y})$ such that,
\begin{enumerate}
\item $f$ is continuous
\item $f^\# : \struct{Y} \to f_* \struct{X}$ 
\end{enumerate}
We want a more refined notion of ``ringed space'' to make connection between $\struct{X}$ and ``functions'' on $X$ and to relate $f^\#$ more directly to to $f$. Let's see how $\struct{X}{x}$ is functorial in $(X, \struct{X}, x)$. 
\bigskip\\
Suppose we have, 
\[ (X, \struct{X}) \xrightarrow{\varphi = (f, f^\#)} (Y, \struct{Y}) \xrightarrow{\psi = (g, g^\#)} (Z, \struct{Z}) \]
then we have maps,
\[ \varphi_x : \stalk{Y}{f(x)} \xrightarrow{f^\#_{f(x)}} (f_* \struct{X})_{f(x)} = \varinjlim_{U \ni y} \struct{X}(f^{-1}(U)) \to \varinjlim_{V \ni x} \struct{X}(V) = \stalk{X}{x} \]
where the second map exists by the universal property of the direct limit. This is compatible with composition: if $x \mapsto y \mapsto z$ then the following diagram commutes,
\begin{center}
\begin{tikzcd}
\stalk{Z}{z} \arrow[rd, "\psi_y"] \arrow[r, "g^\#_z"] & (g_* \struct{Y})_z \arrow[d] \arrow[r, "g_* f^\#" ] & (g_* (f_* \struct{X}))_z \arrow[r, equals] & ((g \circ f)_* \struct{X})_z \arrow[d]
\\
& \stalk{Y}{y} \arrow[rr, "\varphi_x"] & & \stalk{X}{x}
\end{tikzcd}
\end{center}
and therefore $\varphi_x \circ \psi_y = (\psi \circ \varphi)_x$

\subsection{Examples of Ringed Spaces}

\begin{enumerate}
\item $X = \empty$ and $\struct{X}(\empty) = 0$
\item Smooth manifolds (over $\RR, \CC$) and usual notion of morphism (smooth or holomorphic map) and define $f^\# = (-) \circ f$ by pulling back smooth function. These are ringed spaces of $\RR$-algebras or $\CC$-algebras.
\item Algebraic sets and usual morphisms (ring spaces of $k$-algebras) and again $f^\# = (-) \circ f$.
\item $\Spec{\CC}$ is the ringed space where $X = *$ and $\struct{X}(*) = \CC$. We can view this in three ways:
\begin{enumerate}
\item as a ringed space of $\CC$-algebras and thus having a trivial automorphism group
\item as a ringed space of $\RR$-algebras and thus having a automorphism group $\{ \id, \sigma \}$
\item as a ringed space (of $\Z$-algebras) and thus having a massive automorphism group $\Aut{\CC}$.
\end{enumerate}
\item $\Spec{k[t]/(t^2)}$ is the ringed space where $X = *$ and $\struct{X}(*) = k[t]/(t^2)$ has a nonzero nilpotent. 
\item $\Spec{\Z}$ is the ringed space $X = \{ \text{primes} \} \cup \{ 0 \}$ where $\{ 0 \}$ is dense. A nonempty open $U$ is $X \setminus \{ p_1, \dots, p_r \}$ some finite set of closed points. Then define,
\[ \struct{X}(U) = \Z [ \tfrac{1}{p_1 \cdots p_r} ] = \{ q \in \Q \mid q = \tfrac{a}{b} \text{ where } p \divides b \implies p = p_i \text{ for some } i \} \]
with the obvious restriction maps $V \subset U$ gives $\struct{X}(U) \subset \struct{X}(V) \subset \Q$. Furthermore,
\[ \stalk{X}{(p)} = \varinjlim_{U \ni (p)} \struct{X}(U) = \bigcup_{U \ni (p)} \struct{X}(U) = \Z_{(p)} \]
and likewise,
\[ \stalk{X}{(0)} = \varinjlim_{U} \struct{X}(U) = \bigcup_{U} \struct{X}(U) = \Q = \Z_{(0)} \]
Therefore, $\stalk{X}{x} = \Z_{x}$ for each $x \in X$.
\end{enumerate}

\begin{rmk}
In the case of $\Spec{\Z}$ and also for manifolds, we saw that the stalks $\stalk{X}{x}$ are all local rings. This is the phenomenon we want to capture.
\end{rmk}

\subsection{Locally Ringed Spaces}

\begin{defn}
A \textit{locally $\C$-ringed space} is a $\C$-ringed space $(X, \struct{X})$ such that $\stalk{X}{x}$ is local $\forall x \in X$. A \textit{morphism} of locally $\C$-ringed spaces $(X, \struct{X}) \to (Y, \struct{Y})$ is a map of $\C$-ringed spaces $\varphi$ such that $\varphi_x : \stalk{Y}{f(x)} \to \stalk{X}{x}$ is local $\forall x \in X$ meaning that $\varphi_x^{-1}(\m_x) = \m_{f(x)}$ (or equivalently $\varphi_x(\m_{f(x)}) \subset \m_x$). 
\end{defn}

\begin{example}
Some examples of locally ringed spaces.
\begin{enumerate}
\item Manifolds (over $\RR$ or $\CC$) then $\m_x$ is the germs of functions vanishing at $x$ so $\varphi_x(\m_{f(x)}) = \m_{x}$ because $f^\# = (-) \circ f$ so if $g$ vanishes at $f(x)$ then $g \circ f$ vanishes at $x$.
\item Let $\C$ be $k$-algebras with $k = \overline{k}$. Then algebraic sets $(Z, \struct{Z})$ are examples. Then $\stalk{Z}{z}$ is a local $k$-algebra with $\struct{Z}{z} / \m_z = k$ (germs of regular functions $j$ near $z$ with $h(z) = 0$). We've seen for $k$-alg map $\alpha : k[Z_2] \to k[Z_1]$ and $\tilde{\alpha} : Z_1 \to Z_2$ that $\m_{\tilde{\alpha}(P)} = \alpha^{-1}(\m_P)$ and thus,
\begin{center}
\begin{tikzcd}
k[Z_2] \arrow[d] \arrow[r, "\alpha"] & k[Z_1] \arrow[d]
\\
k[Z_2]_{\m_{\tilde{\alpha}(P)}} \arrow[r, "\tilde{\alpha}_P"] & k[Z_1]_{\m_P} 
\end{tikzcd}
\end{center}
\end{enumerate}
\end{example}

\begin{exercise}
Let $\C$ be $k$-algebras for a field $k$. Consider locally $\C$-ringed spaces $(X, \struct{X})$ s.t. $k \iso \stalk{X}{x}/\m_x$ is an isomorphism $\forall : x \in X$. Then morphisms as $\C$-ringed spaces is automatically ``local'' meaning is actually a morphism of $\C$-locally ringed spaces.
\end{exercise}

\begin{rmk}
We call $\kappa(x) := \stalk{X}{x} / \m_x$. Call $\struct{X}$ the \textit{structure sheaf}.
\end{rmk}

\begin{example}
Non-example: $X$ is any topological space then $(X, \underline{\Z})$ is a ringed space but not a locally ringed space. 
\end{example}

\subsection{Evaluation}

For $U \ni x$ and $s \in \struct{X}(U)$ I get a germ $s_x \in \stalk{X}{x}$ and thus $s(x) \in \kappa(x)$ by taking its class in the quotient,
\[ \struct{X}(Y) \to \lim_{V \ni} \struct{X}(V) = \stalk{X}{x} \onto \kappa(x) \]
We call $s(x)$ the ``value'' of $s$ at $x$.

\begin{rmk}
Notice the field $\kappa(x)$ may vary from point to point so $s(x) \in \kappa(x)$ cannot be glued together to give a usual map to a field. 
\end{rmk}

\begin{example}
In familiar cases however, $s(x)$ does recover the ``value'' of $s$ at a point. For example,
\begin{enumerate}
\item For manifolds and algebraic sets under the isomorphism $k \iso \kappa(x)$ indeed $s(x) \in k$ is the usual value of the function.
\item In general if $\C$ is $k$-algebras and $k \to \stalk{X}{x} \onto \kappa(x)$ is an isomorphism $\forall x \in X$ then get a map of sheaves of $k$-algebras,
\[ \struct{X} \to \cA_{X,k} \]
where $\cA_{X,k}$ is the sheaf of $k$-valued functions on $X$. This sends $s \mapsto (x \mapsto s(x))$. In the previous example this is the subsheaf of smooth functions inside all functions. However, in general this is not injective: for example a variety over a non-algebraically closed field or a nonreduced scheme over an algebraically closed field. Lets give an example. For $X = \Spec{k[t]/(t^2)}$ and $t \in \struct{X}(X)$ gives the zero function on $X$ because $t \in \m_x$ at the unique point (indeed any nilpotent must be sent to zero in a field under a ring homomorphism and therefore any nilpotent $s \in \struct{X}(U)$ has $\forall x \in X : s(x) = 0$.
\item Let $X = \Spec{\Z}$ and $s = \frac{5}{3} \in \struct{X}(U)$ where $U = X \setminus \{ (3) \}$. Then, $x \in U$ is either $(p)$ for $p \neq 3$ or $(0)$. Then,
\[ \kappa(x) = \begin{cases}
\Z_{(p)} / p \Z_{(p)} = \FF_p & p \neq 0
\\
\Q & p = 0
\end{cases} \]
and $s(p) = 5 \cdot 3^{-1} \in \FF_p$ and $s(0) = \frac{5}{3} \in \Q$.
\end{enumerate}
\end{example}

\subsection{Functoriality of Evaluation} 

If $\varphi = (f, f^\#) : (X, \struct{X}) \to (Y, \struct{Y})$ and $x \mapsto f(x)$. Then there is a commutative diagram,
\begin{center}
\begin{tikzcd}
\struct{Y}{y} \arrow[r] \arrow[d, two heads] & \struct{X}{x} \arrow[d, two heads]
\\
\kappa(y) \arrow[r] & \kappa(x)
\end{tikzcd}
\end{center}
where the bottom map sends $s(y) \mapsto (f^\# s)(x)$ which must be the identity map if these are maps of $k$-algebras and $\kappa(x)$ is identified with $k$ because its a $k$-algebra map. Therefore on values, functions are pulled back as expected.


\section{Oct. 11}

By earlier result today, enough $\varphi_x^+ : \F_x^+ \to \G_x$ is an isometry for all $x$. Then the diagram,
\begin{center}
\begin{tikzcd}
& \F_x^+ \arrow[dd, "\varphi_x^+"]
\\ 
\F \arrow[ru, "\Theta_x"] \arrow[rd, "\varphi_x"'] 
\\
& \G_x
\end{tikzcd}
\end{center}
Need to construct $(\Phi, \F^+)$ with this universal property. Here we deviate from [H], for method applicable to Grothendieck topologies. There are two possible problems to fix,
\begin{enumerate}
\item $s \in \F(U)$ may not be determined by local data
\item compatible local data may not glue.
\end{enumerate}
Now we propose an iterated ``completion'' that fixes (a) in the first step and (b) in the second step.
\bigskip
\\
First we get $\Theta_0 : \F \to \F_0$ where $\F_0$ is a separated presheaf (sections are determined by local data) and $\F \xrightarrow{\Theta_0} \F_0 \xrightarrow{(\Theta_0)_0} (\F_0)_0$.
We are going to build, $\F_0(U)$ as $\varinjlim$ over open covers of $U$ in a manner similar to Cauchy sequences for metric space completion. Let,
\[ \Sigma_U  = \{ \mathfrak{V} = \{ V_i \}_{i \in I} \mid \text{ open cover } \} \]
is a poset under refinement where $\mathfrak{V}' \ge \mathfrak{V}$ if there exists a map $\tau : I' \to I$ such that $V'_{j} \subset V_{\tau{j}}$ for each $j \in I'$. Furthermore, $\Sigma_U$ is a directed poset because we can construct,
\[ \mathfrak{V} \cap \mathfrak{V}' = \{ V_i \cap V_j' \}_{(i,j) \in I \times I'} \]
Then we define,
\[ \F_0(U) = \varinjlim_{\mathfrak{V} \in \Sigma_U} \{ (s_i) \in \prod \F(V_i) \mid \forall i,j \in I : s_i |_{V_i \cap V_j} = s_j |_{V_i \cap V_j} \} \]
Furthermore, for $U' \subset U$ we get $\Sigma_U \to \Sigma_{U'}$ given by $\mathfrak{V} \mapsto \mathfrak{V} \cap U'$ giving a map $\F_0(U) \to \F_0(U')$ so $\F_0$ is a presheaf. It is easy to check that $\F_0$ is a separated presheaf. Furthermore, $\{ U \} \in \Sigma_U$ and this gives a map $\F(U) \xrightarrow{\Theta_0} \F_0(U)$.

\section{Properties of Sheafification}

Restriction to open subsets: for presheaf $\F$ on $X$, open $U \subset X$ define $\F|_U : (V \subset U) \mapsto \F(V)$.

\begin{proof}
Naturally, $(\F^+)|_U \cong (\F|_U)^+$. 
\end{proof}

\begin{proof}
It would be bogus to say ``they are equal because they have the same stalks'' because you need a global map that \textit{induces} an isomorphism on stalks. However, there is $\theta : \F \to \F^+$ then restricting gives,
\begin{center}
\begin{tikzcd}
& (\F|_U)^+ \arrow[d, "\alpha", dashed]
\\
\F|_U \arrow[r] \arrow[ru] & (\F^+)|_U
\end{tikzcd}
\end{center}
there exists a unique $\alpha : (\F|_U)^+ \to (\F^+)|_U$. The map $\alpha$ is an isomorphism because it is an isomorphism on stalks because the diagram commutes and the other two maps are both isomorphisms on stalks.
\end{proof}

\begin{example}
If $X$ is a $C^\infty$-manifold and $U \subset X$ then $\C^\infty_X|_U \cong \C^{\infty}_U$ and likewise $\Omega^1_X|_U \cong \Omega^1_U$.
\end{example}

``How much more pedantic than this can one get? Watch this!''

\begin{rmk}
For open $U' \subset U$ then the restrictions are compatible,
\begin{center}
\begin{tikzcd}
& (\F^+|_U)|_{U'} \arrow[rd, equals]
\\
(\F^+)|_{U'} \arrow[ru, equals] \arrow[d, equals] & & (\F|_U)^+|_{U'} \arrow[d, equals]
\\
(\F|_{U'})^+ \arrow[rr, equals] & & ((\F|_U)|_{U'})^+
\end{tikzcd}
\end{center}
You can check this commutes by looking at the diagram of stalk maps and noting that it commutes using the following lemma.
\end{rmk}

\begin{lemma}
If $\varphi_1, \varphi_2 : \F \to \G$ are maps of sheaves on $X$ then \[ \varphi_1 = \varphi_2 \iff \forall x \in X : (\varphi_1)_x = (\varphi_2)_x \]
\end{lemma}

\begin{defn}
$\Ab(X)$ is the category of abelian sheaves (sheaves of abelian groups) on $X$. For example $\struct{X}$ additively, $\struct{X}^\times$ multiplicatively. 
\end{defn}

\begin{defn}
For $\varphi : \F \to \G$ is a map in $\Ab(X)$. Define the following sheaves,
\begin{enumerate}
\item $\ker{\varphi} : U \mapsto \ker{\varphi|_U}$ is a subsheaf of $\F$
\item $\im{\varphi}$ is the sheafification of the presheaf $U \mapsto \im{\varphi|_U}$
\item $\coker{\varphi}$ is the sheafification of the presheaf $U \mapsto \coker{\varphi|_U}$.
\end{enumerate}
\end{defn}

\begin{rmk}
If we have subpresheaf $j : \H_1 \embed \H_2$ of a presheaf $\H_2$. Then,
\[ j^+ : \H_1^+ \to \H_2^+ \]
is a subsheaf meaning $(j^+)_U$ is injective for every open $U$ (in particular $\im{\varphi} \to \G = \G^+$ is a subsheaf). We can prove this in two ways,
\begin{enumerate}
\item look at the stalks (for sheaves, subsheaf iff injective on stalks and if $j$ is injective then its injective on stalks and $j^+$ and $j$ have the same stalk maps).
\item look at the construction which exhibits $\H_1^+$ as a subsheaf of $\H_2^+$ using exactness of $\varinjlim$.
\end{enumerate}
\end{rmk}

\begin{rmk}
Explicitly,
\[ (\im{\varphi})(U) = \{ s \in \G(U) \mid s \text{ is locally in the image of } \varphi \} \]
since for subsheaf $\H_1 \subset \H_2$ and $s \in \H_2(U)$ then $s \in \H_1(U)$ iff $s_x \in (\H_1)_x \subset (\H_2)_x$ for all $x \in U$. This is because an element locally is in the image iff its stalks are in the image because 
\end{rmk}

\begin{example}
Consider $\exp : \struct{X} \to \struct{X}^\times$ for $X$ a complex manifold. Then $\im{\exp} = \struct{X}^\times$ as a \textit{sheaf} since non-vanishing holomorphic functions \textit{locally} has a continuous logarithm. However, $\im{\varphi_U}$ is often strictly contained in $\struct{X}^\times(U)$ because global logarithms do not exist. Here, $\coker{\varphi} = \{ 1 \}$ but the presheaf cokernel is nontrivial. 
\end{example}

\begin{rmk}
In general, $\coker{\varphi} = 0 \iff \forall x \in X : \varphi_x : \F_x \onto \G_x$ NOT $\varphi_U : \F(U) \onto \G(U)$ for ``small'' open $U \subset X$. It means that sections are locally in the image but how locally you have to look depends on the section so there is not necessarily a cover on which it is surjective. 
\end{rmk}

\begin{thm}
Consider $\varphi : \F \to \G$ in $\mathrm{Ab}(X)$ then,
\begin{enumerate}
\item $\ker{\varphi} \subset \F$ and $\im{\varphi} \subset \G$ are categorical kernel and image respectively. Explicitly, this means that if $\psi : \H \to \G$ composes with $\varphi$ to give zero then there exists a unique $\H \to \ker{\varphi} \to \F$,
\begin{center}
\begin{tikzcd}
\H \arrow[r, "\psi"] \arrow[rd, dashed] \arrow[rr, bend left, "0"] & \F \arrow[r, "\varphi"] & \G
\\
& \ker{\varphi} \arrow[u, hook]
\end{tikzcd}
\end{center}
and likewise,
\begin{center}
\begin{tikzcd}
\F \arrow[r, "\varphi"] \arrow[rr, bend right, "0"] & \G \arrow[r, "\alpha"] & \H
\end{tikzcd}
\end{center}
if and only if $\alpha|_{\im{\alpha}} = 0$. (IS THIS RIGHT??)

\item $\im{\varphi} = \ker{(\G \to \coker{\varphi})}$ as subsheaves of $\G$

\item $\varphi : \F \to \G$ factors through a subsheaf $\G' \subset \G$ iff $\im{\varphi} \subset \G'$ as subsheaves meaning that $\im{\varphi}$ is the smallest subsheaf through which $\varphi$ factors.
\end{enumerate}
\end{thm}

\begin{proof}
For (c) just check on stalks since for subsheaves,
\[ \H_1, \H_2 \subset \H \]
and $\H$ is a sheaf then $\H_1 \subset \H_2 \iff (\H_1)_x \subset (\H_2)_x$ inside $\H_x$ for all $x \in X$.
\bigskip\\
Likewise for (b) also you can compare stalks inside $\G_x$ because these are both subsheaves so its enough to check that they have the same stalks inside $\G_x$.
\bigskip\\
For (a), it is easy to check the kernel property on $U$ sections since there is no sheafification involved. For the image, you use the fact that a map is zero iff it is zero on stalks so we again reduce to the case of abelian groups.
\end{proof}

\begin{defn}
Say $\varphi : \F \to \G$ in $\Ab(X)$ is \textit{surjective} if $\im{\varphi} = \G$ or equivalently $\forall x \in X : \im{\varphi_x} = (\im{\varphi})_x = \G_x$ iff $\forall x \in X : \varphi_x$ is surjective. 
\end{defn}

\begin{defn}
We say the $\varphi : \F \to \G$ in $\Ab(X)$ is \textit{injective} if $\ker{\varphi} = 0$. This is equivalent to $\ker{\varphi_x} = (\ker{\varphi})_x = 0$ if and only if $\varphi_x$ is injective $\forall x \in X$.
\end{defn}

\begin{prop}
$\varphi$ is injective and surjective iff $\varphi_x$ is an isomorphism for all $x \in X$ iff $\varphi$ is an isomorphism.
\end{prop}

\begin{rmk}
Notice that $\G \onto \coker{\varphi}$ is always surjective and $\ker{\varphi} \embed \F$ is always injective.
\end{rmk}

\begin{prop}
Given a map of sheaves $\varphi : \F \to \G$,
\begin{enumerate}
\item $\varphi$ is injective iff $\F \iso \im{\varphi}$ is an isomorphism
\item $\varphi$ is surjective iff $\im{\varphi} \iso \G$ is an isomorphism.
\end{enumerate}
\end{prop}

\begin{defn}
A complex in $\Ab(X)$,
\begin{center}
\begin{tikzcd}
\cdots \arrow[r] & \F^{i-1} \arrow[r, "\varphi^{i-1}"] & \F^i \arrow[r, "\varphi^i"] & \F^{i+1} \arrow[r] & \cdots
\end{tikzcd}
\end{center}
is \textit{exact} if $\im{\varphi^{i-1}} = \ker{\varphi^i}$ as subsheaves of $\F^i$.
\end{defn}

\begin{prop}
A sequence is exact iff it is exact of stalks for all $x \in X$.
\end{prop}

\section{Oct. 15}

\subsection{Some Loose Ends on Exactness}

\begin{example}
The exponential sequences.
\begin{enumerate}
\item for $\RR$-manifolds,
\begin{center}
\begin{tikzcd}
0 \arrow[r] & \struct{X} \arrow[r, "\exp"] & \struct{X}^\times \arrow[r, "\mathrm{sign}"] & \underline{\{\pm 1\}} \arrow[r] & 0
\end{tikzcd}
\end{center}
The long exact sequence shows that the first Stifel-Whitney class is an isomorphism,
\[ \Pic{X} = H^1(X, \struct{X}^\times) \iso H^1(X, \Z / 2 \Z) \]
(recall that $H^i(X, \struct{X}) = 0$ for $i > 0$ because $\struct{X}$ is soft). 
\item for $\CC$-manifolds,
\begin{center}
\begin{tikzcd}
0 \arrow[r] & \underline{2 \pi i\Z} \arrow[r] & \struct{X} \arrow[r, "\exp"] & \struct{X}^\times \arrow[r] & 0
\end{tikzcd}
\end{center}
\end{enumerate}
\end{example}

\begin{example}
For a subsheaf $j : \F \embed \F$ write $\G/\H = \coker{j}$ so $\G \onto \G/\F$ induces $\G_x \onto (\G / \F)_x$ with kernel $\F_x$ and thus $(\G / \F)_x = \G_x / \F_x$.
\end{example}

\begin{example}
Closed submanifolds of $X$ relate to quotients of $\struct{X}$ by (certain) ``ideal sheaves'' (subsheaves of $\struct{X}$ which are ideals on sections or equivalently ideals on stalks). 
\bigskip\\
Say $\iota : Y \embed X$ is a closed $C^\infty$-submanifold. Then topologically, $\iota$ is a closed embedding. Furthermore, on the sheaves of rings $\iota^\# : \struct{X} \onto \iota_* \struct{Y}$ is surjective because locally charts are subvectorspaces of the ambient chart (using the Immersion Theorem). 
\end{example}

\begin{exercise}
Check that any topological closed embedding with $\iota^\# : \struct{X} \to \iota_* \struct{Y}$ surjective then $\iota$ is a closed submanifold. 
\end{exercise}
\noindent
Use the fact that if $\iota^\# : \struct{X} \onto \iota_* \struct{Y}$ is surjective then $\stalk{X}{x} \onto \stalk{Y}{y}$ is surjective where $x = \iota(y)$ and local and thus the map $\m_x / \m_x^2 \onto \m_y / \m_y^2$ is surjective (because $\iota_x^{-1}(\m_y) = \m_x$) meaning that its dual $T_y \to T_x$ is injective proving that $\iota$ is an immersion. Furthermore, since $\iota$ is a topological closed embedding this shows that $\iota$ makes $Y$ a (closed) embedded submanifold.



\subsection{Sheaf Pullback}

For $h : M \to N$ of $C^\infty$-manifolds, for open $V \subset N$ and $\omega \in \Omega^k_N(V)$ get $h^* \omega \in \Omega_M^k(h^{-1} V)$ giving a pullback map $h^* : \Omega^k_N \to h_* \Omega^k_M$. 
\bigskip\\
Let $f : X \to Y$ be continuous then we defined the functor. $f_* : \Ab(X) \to \Ab(Y)$. Will will now define the functor $f^{-1} : \Ab(Y) \to \Ab(X)$. Our goal is to convert $\F \to f_* \G$ to a map $f^{-1} \F \to \G$.
\bigskip\\
Loose analogy: $\alpha : A \to B$ a ring map. Then $\p \mapsto \alpha^{-1}(\p)$ gives $\tilde{\alpha} : \Spec{B} \to \Spec{A}$. There is a forgetful map $\tilde{\alpha}_* : \Mod{B} \to \Mod{A}$ sending $M \mapsto _{A} M$. There is a functor going the other way $N \mapsto B \otimes_A N$. Furthermore,
\[ \Hom{A}{N}{_{A}{M}} = \Hom{B}{B \otimes_A N}{M} \]
This is supposed to be the analogue of,
\[ \Hom{Y}{\F}{f_* \G} = \Hom{X}{f^{-1} \F}{\G} \]

\begin{defn}
Let $f : X \to Y$ be a continuous map. For any presheaf of sets $\F$ on $Y$ define,
\[ f^{-1}_{\text{pre}} \F : U \mapsto \varinjlim_{U \subset f^{-1}(V)} \F(V) \]
For a sheaf of sets $\F$ on $X$ then we define $f^{-1} \F$ to be the sheafification of $f^{-1}_{\text{pre}} \F$. For $\F \in \Ab(Y)$ then $f^{-1} \F \in \Ab(X)$.
\end{defn}

\begin{prop}
$f^{-1} : \mathbf{Shv}(Y) \to \mathbf{Shv}(X)$ is a functor (also for $f^{-1} : \Ab(Y) \to \Ab(X)$).
\end{prop}

\begin{proof}
Given $\varphi : \F_1 \to \F_2$ on $Y$ then we get,
\begin{center}
\begin{tikzcd}
f^{-1}_{\text{pre}} \F_1 \arrow[r] \arrow[d, "\theta_1"] & f^{-1}_{\text{pre}} \F_2 \arrow[d, "\theta_2"]
\\
f^{-1} \F_1 \arrow[r, dashed] & f^{-1} \F_2
\end{tikzcd}
\end{center}
using universal property of $\theta_1$ there is a unique $f^{-1} \F_1 \to f^{-1} \F_2$ making the diagram commute. 
\end{proof}

\begin{example}
For $X = \{ y \} \embed Y$ then $\iota^{-1} \F = \F_y$ (as a sheaf on the one point space).
\end{example}

\begin{thm}
For $f : X \to Y$ continuous. 
\begin{enumerate}
\item For $x \in X$ then $(f^{-1} \F)_{x} = \F_{f(x)}$ naturally in $\F$.
\item For $\theta : \F \to \F^+$, then $f^{-1} \theta : f^{-1}_{\text{pre}} \to f^{-1} \F^+$ and therefore $(f^{-1}_{\text{pre}} \F)^+ \iso f^{-1} \F^+$ which we could write as saying $f^{-1} \theta : f^{-1} \F \iso f^{-1} \F^+$ (where $f^{-1} \F$ is the sheafification of $f^{-1}_{\text{pre}} \F$). 
\item For $X' \xrightarrow{g} \xrightarrow{f} Y$ there exists a natural isomorphism,
\[ \alpha_{f,g} : g^{-1} (f^{-1} \F) \cong (fg)^{-1} \F \]
which is associative (analogue for modules $B' \otimes_B (B \otimes_A M) \cong B' \otimes_A M)$. 
\end{enumerate}
\end{thm}

\begin{example}
For $f : X \to Y$ continuous and $\underline{A}_X = (\F_{A,X})^+$ the constant sheaf on $X$. We claim that naturally $f^{-1} \underline{A}_Y \iso \underline{A}_X$ (respecting composition in $f$). This is because,
\begin{center}
\begin{tikzcd}
\F_{A,X} \arrow[d] \arrow[r, equals] & f^{-1}_{\text{pre}} \F_{A,Y} \arrow[r] & f^{-1} \F_{A,Y} \arrow[r, equals] & f^{-1} \underline{A}_Y
\\
\underline{A}_X \arrow[rrru, dashed, bend right]
\end{tikzcd}
\end{center}
On $x$-stalks, check that,
\begin{center}
\begin{tikzcd}
(\underline{A}_X)_X \arrow[r] \arrow[d, equals] & (f^{-1} \underline{A}_Y)_x \arrow[d, equals]
\\
A \arrow[r, "\sim"] & (\underline{A}_Y)_{f(x)}
\end{tikzcd}
\end{center}
commutes.
\end{example}

\noindent 
Now we prove the theorem.

\begin{proof}
For (a) want $\F_{f(x)} \to (f^{-1} \F)_x$ and show isomorphism. Now,
\[ \F_{f(x)} = \varinjlim_{V \ni f(x)} \F(Y) \]
and,
\[ (f^{-1} \F)_x = (f^{-1}_{\text{pre}} \F)_x = \varinjlim_{U \ni x} f^{-1}_{\text{pre}}(U) = \varinjlim_{U \ni x} \left( \varinjlim_{W \supset f(U)} \F(W) \right) \]
Notice that all $W$ contain $f(x)$. Each $V$ contains $f(f^{-1}(V))$ so,
\[ \F(V) \to \F(V) \to \varinjlim_{W \supset f(f^{-1}(V))} \F(W) \to (f^{-1} \F)_x \]
because these maps respect inclusion we get,
\[ \F_{f(x)} \to (f^{-1} \F)_x \]
passing to the limit. In reverse,
\[ (f^{-1} \F)_x = (f^{-1}_{\text{pre}} \F)_x = \varinjlim_{W \ni x} (f^{-1}_{\text{pre}} \F)(W) = \varinjlim_{W \ni x} \varinjlim_{V \supset f(W)} \F(V) \to \varinjlim_{V \ni f(x)} \F(V) = \F_{f(x)} \]
Check these are inverse to each other.
\bigskip\\
For (b) consider $\theta : \F \to \F^+$ then we get,
\begin{center}
\begin{tikzcd}
f^{-1}_{\text{pre}} \F \arrow[d] \arrow[r, "f^{-1}_{\text{pre}} \theta"] & f^{-1}_{\text{pre}} (F^+) \arrow[d]
\\
f^{-1} \F \arrow[r, "f^{-1} \theta"] & f^{-1} (\F^+) 
\end{tikzcd}
\end{center}
then this diagram shows that on stalks $f^{-1} \theta$ induces an isomorphism on the stalks. 
\bigskip\\
For (c) build $g^{-1} f^{-1} \F \xrightarrow{\alpha} (fg)^{-1} \F$ using compatibility of $g^{-1}$ and $f^{-1}$ with sheafification (using (b)) and check that $\alpha_{x'}$ respects the identification of  both $x'$-stalks with $\F_{f(g(x'))}$ and thus is an isomorphism.
\end{proof}

\section{Oct. 18}

\subsection{More sheaf pullback and manifolds without charts}

Two instances of sheaf pullback:
\[ f : X \to Y \text{ continuous gives } f^{-1} \F = (f^{-1}_{\text{pre}} \F)^+ \]

\begin{enumerate}
\item for open subspace $j : U \to X$ and $\F \in \mathbf{Shv}(X)$ we get,
\[ \F|_U \cong j^{-1} \F \]
and we don't even need to sheafify.
\item For closed $\iota : Z \embed X$ if $\G \in Ab(X)$ then,
\[ (\iota_* \G)_x = 
\begin{cases}
\G_x & x \in Z
\\
0 & x \notin Z
\end{cases} \]
where the second follows from the fact that $X \setminus Z$ is \textit{open} this does not work for a general subset. For $\F \in \Ab(X)$ then $\F|_Z := \iota^{-1} \F \in \Ab(Z)$. If $\G \in \Ab(Z)$ then,
\[ \iota^{-1} (\iota_* \G) \cong \G \]
Build dmap $\iota^{-1} (\iota_* \G) \to \G$ using the universal property of sheafification. For open $U \subset Z$ and open $V \subset X$ with $V \cap Z = U$ then $(\iota_* \G)(V) = \G(U)$ which we use to build the map and then check on stalks that it is an isomorphism. 
\end{enumerate}

\begin{cor}
For a closed $\iota : Z \embed X$ then,
\[ \Ab(Z) \cong \{ \F \in \Ab(X) \mid \iota^{-1} \F = 0 \} \]
where the map is given by $\G \mapsto \iota_* \G$ whose inverse is $\F \mapsto \iota^{-1} \F$. 
\end{cor}

\begin{prop}
For $f : X \to Y$ continuous then,
\[ \Hom{X}{f^{-1} \F}{\G} = \Hom{Y}{\F}{\iota_* \G} \]
naturally (as sheaves of sets, then also as abelian sheaves).
\end{prop}

\begin{proof}
In Homework 4.
\end{proof}

\subsection{Universal property of open immersion}

Let $(X, \struct{X})$ be a $\C$-ringed space, $U \subset X$ an open set. Then $\struct{U} = \struct{X}|_U$ makes $(U, \struct{U})$ a $\C$-ringed space and have $j : (U, \struct{U}) \to (X, \struct{X})$ and if $f : (Y, \struct{Y}) \to (X, \struct{X})$ is a map of such with $f(Y) \subset U$ (as sets) then it factors,
\begin{center}
\begin{tikzcd}
(Y, \struct{Y}) \arrow[rd] \arrow[rr, "f"] & & (X, \struct{X})
\\
& (U, \struct{U}) \arrow[ur]
\end{tikzcd}
\end{center}
because $f^\# : \struct{X} \to f_* \struct{Y}$ factors.

\begin{defn}
Say $f : (Y, \struct{Y}) \to (X, \struct{X})$ is an open immersion if $f(Y) \subset X$ is open and $(Y, \struct{Y}) \iso (f(Y), \struct{f(Y)})$. 
\end{defn}

\begin{rmk}
I think its better to say that $f$ is a homeomorphism onto an open subspace and the adjoint sheaf map $f^{-1} \struct{X} \to \struct{Y}$ is an isomorphism but okay its the same.
\end{rmk}

\begin{rmk}
Beware: an open immersion is an open subspace \textit{with the correct sheaf of rings}. For example $\Spec{k} \to \Spec{k[t]/(t^2)}$ is \textit{not} an open immersion. Similarly, the map $\A^1_{\CC} \onto \{ x^3 = y^2 \} \subset \A^2_{\CC}$ sending $t \mapsto (t^2, t^3)$ given by the pullback map $\CC[x,y]/(x^3 - y^2) \to \CC[t]$ sending $x \mapsto t^3$ and $y \mapsto t^2$ is \textit{not} an open immersion. Away from the cusp it is an isomorphism but globally it has the wrong sheaf of rings even though it is a homeomorphism.
\end{rmk}

\begin{defn}
A \textit{complex manifold} is a $\CC$-ringed space $(X, \struct{X})$ with open cover $\{ U_i \}$ where each $U_\alpha$ is isomorphic (as a $\CC$-ringed space) to an open subspace of a finite dimensional $\CC$-vectorspace $V$ equipped with the sheaf of holomorphic functions.
\end{defn}

\begin{rmk}
These are automatically locally ringed and we saw that maps of $\CC$-algebras whose residue fields are all $\CC$ are automatically local.
\end{rmk}

\begin{rmk}
Convince yourself that this is equivalent to the usual definition (up to topological conditions on the underlying space). 
\end{rmk}

\subsection{Gluing of Ringed Spaces}

Let $(X_i, \struct{X_i})$ be (locally) $\C$-ringed spaces. Consider the idea of pasting together these spaces along identified open overlaps such that the identifications satisfy some coherence. Given opens $X_{ij} \subset X_i$ for each pair $i,j$ with isomorphisms of (locally) $\C$-ringed spaces,
\[ \varphi_{ij} : X_{ij} \iso X_{ji} \]
such that,
\begin{enumerate}
\item $X_{ii} = X_i$ and $\varphi_{ii} = \id$
\item $\varphi_{ij}(X_{ij} \cap X_{ik}) = X_{ji} \cap X_{jk}$ for all $i,j,k$
\item $\varphi_{jk}|_{X_{ji} \cap X_{jk}} \circ \varphi_{ij}|_{X_{ik} \cap X_{ij}} = \varphi_{ik}|_{X_{ik} \cap X_{ij}}$
\end{enumerate}

\begin{prop}
Given the above setup, there exists a unique (up to unique isomorphism) ringed space $X$ equipped with an open cover $\{ U_\alpha \}$ and isomorphisms $\varphi_i : X_i \iso U_i$ such that 
\begin{enumerate}
\item $\varphi_i(X_{ij}) = U_i \cap U_j$
\item $\varphi_{ij} = \varphi_j^{-1} |_{U_i \cap U_J} \circ \varphi_i |_{X_{ij}}$
\end{enumerate}
which together have the mapping property: for any (locally) $\C$-ringed space $Y$,
\[ \Hom{\C}{X}{Y} \iso \left\{ (f_i) \in \prod_i \Hom{\C}{X_i}{Y} \, \middle| \, f_i |_{X_{ij}} = f_j |_{X_{ji}} \circ \varphi_{ij} \right\} \]
given by sending $f \mapsto (f|_{U_i} \circ \varphi_i)$ is an isomorphism.
\end{prop}

\begin{proof}
Details in the handout. For topological space, use,
\[ X  = \left( \coprod X_i \right) / \sim \] 
where $x_{ij} \sim \varphi_{ij}(x_{ij})$ for $x_{ij} \in X_{ij} \subset X_i$ and $\varphi_{ij}(x_{[ij}) \in X_j$ with the quotient topology. The cocycle condition gives transitivity of this equivalence relation. Furthermore, we set,
\[ \struct{X}(U_i) := \left\{ (s_i) \in \prod_{i} \struct{X_i}(U \cap X_i) \, \middle| \, s_i |_{(U \cap X_i) \cap X_j} = s_j |_{(U \cap X_j) \cap X_i} \right\} \]
where for the equality we should stick in some $\varphi_{ij}$.
\end{proof}

\section{Oct. 20 Schemes}

Let $A$ be a commutative ring. Let $X = \Spec{A}$ with the topology whose closed sets are $V(I) = \{ \p \in \Spec{A} \mid \p \supset I \}$ for ideals $I \subset A$. Notice that many ideals give the same closed subset (e.g. $V(I) = V(\sqrt{I})$). For radical $J$,
\[ J = \bigcap_{\p \supset J} \p \]
then $A / J$ is a reduced ring. Therefore, closed sets are in bijective correspondence with radical ideals. This property has some interesting features:

\begin{enumerate}
\item $\overline{ \{ \p \} } = V(\p)$
so the closed points are exactly the maximal ideals
\item as topological spaces,
\[ \Spec{A/I} = V(I) \]
as \textit{topological spaces}.
\item $\Spec{A_a} = D(a) = X \setminus V(a) = \{ \p \in \Spec{A} \mid a \notin \p \}$ is also a homeomorphism (call these basic affine opens)
\end{enumerate}

\begin{prop}
The $D(a)$ are a basis of opens in $X$.
\end{prop}

\begin{proof}
Let $x \in U = X \setminus V(I)$. Then $x = \p$ for some prime ideal not containing $I$. Thus there is some $a \in I$ such that $a \notin \p$ and thus $\p \in D(a) \subset U$ because if $a \notin \q$ then $\q \not\supset I$ and thus $\q \in U$. 
\end{proof}

\begin{rmk}
$D(a) \cap D(a') = D(aa')$. Call such $D(a)$ basic affine opens. 
\end{rmk}

\begin{rmk}
Warning: $D(a)$ does not determine $a$ or even the set $\{ a, a^2, a^3, \dots \}$.
\end{rmk}

\subsection{Goal}

We want to build a sheaf of rings $\struct{A}$ on $X$ with $\struct{A}(D(a)) = A_a$ naturally. 

\begin{enumerate}
\item need to specify restriction maps for $D(a') \subset D(a)$
\item handle general open $U \subset X$
\item make sure $\struct{A}(D(a))$ only depends on subset $D(a) \subset X$, not $a \in A$.
\end{enumerate}

If $D(a) \subset D(a')$ then $a'$ is not in every prime of $A_a$ and thus $a'$ is a unit in $A_a$ meaning that by the universal property of localization,
\begin{center}
\begin{tikzcd}
A_{a'} \arrow[rr, dashed] & & A_a
\\
& A \arrow[ru] \arrow[lu]
\end{tikzcd}
\end{center}
This map is uniquely determined so we choose it a the restriction map $\res_{D(a), D(a')}$. 

\begin{prop}
As $A$-algebras, there is a unique isomorphism $A_a = S^{-1} A$ for the set
\[ S = \{ a' \in A \mid \forall \p \in D(a) : a' \notin \p \} \]
By definition, $S$ only depends on $A$ and $D(a)$ not on the element $a$.
\end{prop}

\begin{proof}
Clearly $a$ is a unit in $S^{-1} A$ so we get a diagram,
\begin{center}
\begin{tikzcd}
A_{a} \arrow[rr, dashed] & & S^{-1} A
\\
& A \arrow[ru] \arrow[lu]
\end{tikzcd}
\end{center}
Because maps over $A$ are unique by the universal property it suffices to produce a map the other way (because then they compose to be the unique automorphism over $A$ which is the identity). By the universal property, a map the other direction simply follows from $S \subset (A_a)^\times$ and $\Spec{A_a} = D(a)$ so every $s \in S$ is not contained in any prime in $D(a)$ so not contained in any prime of $\Spec{A_a}$ and thus $s$ is a unit in $A_a$.  
\end{proof}

\begin{rmk}
Because $S$ only depends on the open set $D(a)$ and not on the choice of presenting element $a \in A$ this means that $A_a$ depends up to isomorphism only on the open set $D(a)$ and therefore $\struct{A}(D(a)) = S^{-1}A \cong A_a$ is well-defined and restriction maps from the universal property. This forms a sheaf of $A$-algebras. The $A$-algebra structure is essential to make the restriction maps uniquely defined. 
\end{rmk}

\begin{example}
$\struct{A}(\empty) = (0)$ and $\struct{A}(X) = A$. 
\end{example}

\begin{rmk}
This is not yet a sheaf because we have only defined it on principal affine opens. However, let's first check that it satisfies the sheaf condition when it is defined.
\end{rmk}

\begin{prop}
If $\{ U_i = D(a_i) \}$ form an open cover of $U = D(a)$ then unique gluing holds in the sense that, given $s_i \in A_{a_i} = \struct{A}(D(a_i))$ such that,
\[ A_{a_i} \to A_{a_i a_j} \leftarrow A_{a_j} \]
agree meaning $s_i$ and $s_j$ map to the same element. Then there is a unique $s \in A_a$ so that $s \mapsto s_i$ under $A_a \to A_{a_i}$ for each $i$. 
\end{prop}

\begin{proof}
First we can set $B = A_a$ and $A_{a_i} = B_{b_i}$ where $b_i = \frac{a_i}{1} \in B$ since $U_i \subset U$ this means that $a$ is a unit in $A_{a_i}$ so you can factor $A \to A_a \to A_{a_i}$. and $A_a \to A_{a_i}$ is a localization. Thus we can replace $A$ by $B$ and forget about the first localization. First we show uniqueness. If $s,t \in A$ agree in all $A_{a_i}$ then want $s = t$. For each $\p$ we have $A_\p = (A_{a_i})_{\p_{a_i}}$ as $A$-algebras so $s = t$ in $A_\p$ at each prime and therefore $s = t$. Now for gluing. Given $s_i \in A_{a_i}$ with $s_i |_{U_i \cap U_j} = s_j |_{U_i \cap U_j}$ for all $i,j$ seek $s \in A$ so that $s \mapsto s_i$ in $A_{a_i}$ for each $i$. However, $X$ is quasi-compact, so some finite subcover $\{ U_{a_i}, \dots, U_{a_n} \}$ covers and by uniqueness a solution for the finite subcover will work for the whole cover because for each element of the cover the element is locally (in the overlaps with the finite cover) the correct element and thus correct by the uniqueness.
\bigskip\\
Because $\{ U_1, \dots, U_n \}$ covers $X$ is equivalent to $(a_1, \dots, a_n) = (1)$. However, we solved this finite gluing problem earlier when showing that $\struct{Z}$ is a sheaf. 
\end{proof}

\begin{lemma}
Let $X$ be a topological space, $B$ a base of open sets\footnote{Base means closed under finite overlaps here}, and $\F$ is a $B$-sheaf then there is a unique (up to unique isomorphism) sheaf $\F'$ on $X$ extending $\F$. This satisfies the mapping property, for sheaves $\G$ on $X$,
\[ \Hom{B}{\F}{\G|_B} = \Hom{X}{\F'}{\G} \]
\end{lemma}

\begin{proof}
For details [3.2.1-3.2.5 in EGA $\mathrm{O_I}$]
\[ \F'(U) = \varprojlim_{V \subset U} \F(V) = \left\{ (s_V) \in \prod_{V \subset U} \F(V) \, \middle| \, s_V|_{V \cap V'} = s_{V'} |_{V \cap V'} \right\}  \]
\end{proof}

\begin{rmk}
Now we have a sheaf $\struct{A}$ so we write $\Spec{A}$ to mean $(\Spec{A}, \struct{A})$. We want this to be functorial in $A$, enhancing the topological map meaning that,
\[ \varphi : A \to B \leadsto \tilde{\varphi} : \Spec{B} \to \Spec{A} \]
where $\tilde{\varphi}(\p) = \varphi^{-1}(\p)$ which is continuous because $\tilde{\varphi}^{-1}(D(a)) = D(\varphi(a))$. However, we want a map of sheaves of rings,
\[ \struct{A} \to \tilde{\varphi}_* \struct{B} \]
However, we have $A_a \to B_{\varphi(a)}$ and thus,
\[ \struct{A}(D(a)) \to \struct{B}(D(\varphi(a))) = (\tilde{\varphi}_* (\struct{B}))(D(a)) \]
\end{rmk}

\begin{lemma}
Let $f : X \to Y$ be continuous and bases of opens $B'$ for $X$ and $B$ for $Y$ such that $f^{-1}(U) \in B'$ for $U \in B$ then for $B'$-sheaf $\F$ on $X'$ then,
\[ f_* (\F') = (f_*\F)' \]
\end{lemma}

\begin{cor}
There is a functor $\mathbf{Ring} \to \mathbf{RingSp}$ via $A \mapsto \Spec{A}$ satisfying some properties,
\begin{enumerate}
\item $\Spec{A}$ is quasi-compact and $\Spec{A_a}$ gives a basis of ringed spaces because,
\[ \struct{A}|_{D(a)} = \struct{A_a} \]
\item $\stalk{A}{\p} = \varinjlim_{D(a) \ni x} \struct{A}(D(a)) = \varinjlim_{a \notin \p} A_a = A_\p$ so $\Spec{A}$ is a locally ringed space.
\end{enumerate}
\end{cor}

\begin{defn}
A scheme is a ringed space $X$ covered by \textit{affine} schemes as open subspaces. Thus $X$ is locally ringed. 
\end{defn}

\begin{rmk}
Open subspaces of schemes are schemes because $\Spec{A}$ has a base of $\Spec{A_a}$.
\end{rmk}

\section{Oct. 22}

\subsection{Some basic properties of schemes and morphisms}

\newcommand{\qc}{\text{q-c}\xspace}

First we start we a few remarks:

\begin{enumerate}
\item A scheme $X$ is \qc iff $X$ is covered by finitely many affine opens.
\item On schemes, usually for $x \neq y$ there are not isomorphic neighborhoods of $x$ and $y$ (even for smooth varieties and classical closed points).
\end{enumerate}
\noindent
Last time: given $\varphi : A \to B$ we made $(f, f^\#) : \Spec{B} \to \Spec{A}$ where $f(\p) = \varphi^{-1}(\p)$. Furthermore, $f^{-1}(D(a)) = D(\varphi(a))$ and $f^\# : \struct{A}(D(a)) \to \struct{B}(D(\varphi(a))$ is $\varphi_a : A_a \to B_{\varphi(a)}$.

\begin{prop}
For $y = \{ \p \} \in \Spec{B}$ then $x = f(y) = \{ \varphi^{-1}(\p) \} \in \Spec{A}$,
\begin{center}
\begin{tikzcd}
\stalk{A}{x} \arrow[r, "f^\#_x"] \arrow[d, equals] & (f_* \struct{B})_{x} \arrow[r] & \stalk{B}{y} \arrow[d, equals]
\\
A_\p \arrow[rr] & & A_{\varphi^{-1}(\p)}
\end{tikzcd}
\end{center}
commutes and is a \textit{local} map.
\end{prop}

\begin{proof}
Consider,
\begin{center}
\begin{tikzcd}
A \arrow[rd, "\text{can}"'] \arrow[r, equals] & \Gamma(X, \struct{X}) \arrow[d] \arrow[r, "f^\#"] & \Gamma(Y, \struct{Y}) \arrow[d] \arrow[r, equals] & B \arrow[dl, "\text{can}"]
\\
& A_{\varphi^{-1}(\p)} \arrow[r, "f^\#_y"] & B_\p
\end{tikzcd}
\end{center}
however, there is a unique map factoring $A \to B_\p$ through $A \to A_{\varphi^{-1}(\p)}$ which is local because it satisfies $\p \mapsto \varphi^{-1}(\p)$ on pullbacks.
\end{proof}

\begin{defn}
A \textit{morphism} of schemes $X \to Y$ is a map as locally ringed spaces.
\end{defn}

\begin{thm}
The map,
\[ \Hom{\mathbf{Ring}}{A}{B} \to \Hom{\mathbf{LRS}}{\Spec{B}}{\Spec{A}} \]
is bijective.
\end{thm}

\begin{proof}
Given $\varphi : A \to B$ for $f = \Spec{\varphi} : Y \to X$ we've seen that $\Gamma(X, f^\#) : A \to B$ is $\varphi$ so the map is injective.
\bigskip\\
Let $Y = \Spec{B}$ and $X = \Spec{A}$. Let $f : Y \to X$ be a map of locally ringed spaces. We get a ring map $\varphi : \Gamma(X, f^\#) : A \to B$ because $\struct{X}(X) = A$ and $(f_* \struct{Y})(X) = B$. Then we get $g = \Spec{\varphi} : Y \to X$. First we show that $f$ and $g$ agree topologically. For $y \in \{ \p \} \in Y$ we know that $g(y) = \varphi^{-1}(\p)$. Observe,
\begin{center}
\begin{tikzcd}
A \arrow[d, "\text{can}"'] \arrow[r, equals] & \Gamma(X, \struct{X}) \arrow[d] \arrow[r, "f^\#"] & \Gamma(Y, \struct{Y}) \arrow[d] \arrow[r, equals] & B \arrow[d, "\text{can}"]
\\
A_{f(y)} \arrow[r, equals] & \stalk{X}{f(y)} \arrow[r, "f^\#_y"'] & \stalk{Y}{y} \arrow[r, equals] & B_\p
\end{tikzcd}
\end{center}
However, since $f^\#_y$ is local, the maximal ideal of $A_{f(y)}$ is $(f^\#_y)^{-1}(\p B_\p)$ and the preimage of this maximal ideal in $A$ is $f(y)$. However, going around the top we get $\p B_\p$ pulls back to $\varphi^{-1}(\p) = g(y)$ so $f(y) = g(y)$. 
\bigskip\\
Now, to check that $f^\#, g^\# : \struct{X} \to f_* \struct{Y}$ (which are now maps between the same sheaves because the topological maps are the same) are the same it suffices to check on the base $D(a)$. Then,
\begin{center}
\begin{tikzcd}
A \arrow[d, "\text{can}"'] \arrow[r, equals] & \Gamma(X, \struct{X}) \arrow[d] \arrow[r, "f^\#"] & \Gamma(Y, \struct{Y}) \arrow[d] \arrow[r, equals] & B \arrow[d, "\text{can}"]
\\
A_{a} \arrow[r, equals] & \struct{X}(D(a)) \arrow[r, shift left=0.75ex, "f^\#"] \arrow[r, shift right=0.75ex, "g^\#"'] & \struct{Y}(f^{-1}(D(a))) \arrow[r, equals] & B_{\varphi(a)}
\end{tikzcd}
\end{center}
so both $f^\#$ and $g^\#$ both must equal $\varphi_a$ because they factor $A \to B \to B_{\varphi(a)}$ though $A \to A_a$ which is unique by the universal property when it exists.
\end{proof}

\begin{rmk}
For \underline{any} LRS $Y$,
\[ \Hom{\mathbf{LRS}}{Y}{\Spec{A}} \to \Hom{\mathbf{Ring}}{A}{\struct{Y}(Y)} \]
sending $f \mapsto \Gamma(f)$ is bijective. This is an observation of Tate. 
\end{rmk}

\begin{rmk}
In general, its very hard to characterize in terms of $\varphi : A \to B$ when $\Spec{\varphi}$ is an open immersion (except when its a principal open i.e. when $A \to B$ is the localization at an element).
\end{rmk}

\begin{example}
Some schemes:
\begin{enumerate}
\item $\A^n_R = \Spec{R[t_1, \dots, t_n]} \to \Spec{R}$ has the universal property: for a scheme $Y$,
\[ \Hom{}{Y}{\A^n_R} = \Hom{}{Y}{\Spec{R}} \times \{ g_1, \dots, g_n \in \struct{Y} \} \]
where $g_i = f^\#(t_i)$. In the category of $R$-schemes (schemes with a map to $\Spec{R}$) then,
\[ \Hom{R}{Y}{\A^n_R} = \struct{Y}(Y)^n \]
\end{enumerate}
\end{example}

\begin{defn}
For a scheme $S$, an $S$-scheme is a morphism $X \to S$. A morphism of $S$-schemes $X,Y$ is a scheme map $f : X \to Y$ such that,
\begin{center}
\begin{tikzcd}
X \arrow[rr, "f"] \arrow[rd] & & Y \arrow[dl]
\\
& S
\end{tikzcd}
\end{center}
For $S = \Spec{A}$ we call these $A$-schemes (generalization of $A$-algebras).
\end{defn}

\begin{rmk}
The classical setting is equivalent to studying $k$-schemes where $k$ is algebraically closed.
\end{rmk}

\begin{example}
$\Spec{\CC}$ as a $\CC$-scheme is ``just a point'' but $\Spec{\CC}$ as a $\Z$-scheme is ``huge'' it covers $\Spec{\Z[\pi, e]}$ which is a big scheme.
\end{example}

\section{Oct. 25}

Let $U_i = \Spec{R_i} \subset X$ be an open cover of a scheme. For $x \in U_1 \cap U_2$ wee a basis of open $V \subset U_1 \cap U_2$ and $x$ such that $V$ is base of affine in $U_1$ and $U_2$.

\begin{rmk}
$U_1 \cap U_2$ may fail tor be qc: 
\[ U = \A_k^\infty \setminus \{ 0 \} = \Spec{k[x_1, x_2, \dots]} \setminus \{ 0 \} = \bigcup_{i = 1}^\infty D(x_i) \]
Let $X$ be the gluing of $\A^\infty_k$ along $\id : U \to U$ (FINISH)
\end{rmk}

\begin{proof}
Let $U_1 = \Spec{R_1}$ and $U_2 = \Spec{R_2}$. 
Choose some principal affine open of $X$
\[ x \in W \subset U_1 \cap U_2 \]
then $x \in V_i \subset W$ with $V_1 = \Spec{(R_1)_{r_1}} \subset W$. Then we can find $x \in \Spec{(R_2)_{r_2}} = V_2 \subset V_1$. Let's check $V_2 = \{ r_2 \neq 0 \} \subset U$ is a basic affine open.
\end{proof}

\begin{thm}
For $f : X \to Y$ the following are equivalent,
\begin{enumerate}
\item there exists an affine open cover $\{ \Spec{B_i} \}$ of $Y$ such that $f^{-1}(\Spec{B_i})$ is covered by affine open $\{ \Spec{A_{ij}} \}$ with $A_{ij}$ a finitely generated $B_i$-algebra
\item for every affine open $\Spec{B}$ and affine open $\Spec{A} \subset f^{-1}(\Spec{B})$ then $B \to A$ makes $A$ a finitely generated $B$-algebra.
\end{enumerate}
\end{thm}

\begin{proof}
Clearly (b) $\implies$ (a). Now assume (a). Pick affine open $\Spec{B} \subset Y$. Then $\Spec{B_i} \cap \Spec{B}$ is covered by $U_{i \alpha}$ that are basic affine open in $\Spec{B_i}$ and $\Spec{B}$. Any $\Spec{(B_i)_{b_i}}$ has $f$-preimage covered by $\Spec{(A_ij)_{b_i}}$ and $(A_{ij})_{b_i}$ are finite type $(B_i)_{b_i}$-algebras. Thus, each $f^{-1}(U_{i\alpha})$ is covered by, $\Spec{(A_{ij})_{b_i}}$ we know these are also localizations of $B_{b, \alpha}$ where $U_{i\alpha} = \Spec{B_{b,\alpha}}$. Then $\Spec{B}$ is covered by $U_{i\alpha}$ and $B_{b,\alpha}$ are finite type $B$-algebras. Now we have shown that $f^{-1}(\Spec{B})$ is covered by $\Spec{R_\gamma}$ with $R_\gamma$ finitely generated over $B$.
\bigskip\\
Now, we can reduce to the case $Y = \Spec{B}$. Then consider $\Spec{A} \subset X$. Consider $\Spec{A} \cap \Spec{R_\gamma}$. Any $\Spec{(R_\gamma)_r}$ also has the same finitness property over $B$ (just include $r^{-1}$ in the generating set). Then by Nike trick, $\Spec{A} \cap \Spec{R_\gamma}$ is covered by affine opens that are simultaneously basic affine in $\Spec{A}$ and $\Spec{R_\gamma}$ which are finitely generated $B$-algebras (since they are principal affine inside $\Spec{R_\gamma}$) but also principal affine open in $\Spec{A}$. However, $\Spec{A}$ is qc so we can take a finite set $\{ \Spec{A_{f_i}} \}$ covering $\Spec{A}$ and thus $(f_1, \dots, f_n) = 1$. Since each $A_i$ is finitely generated as a $B$-algebra. 
\bigskip\\
Thus we reduce to a pure algebra problem. Let $B \to A$ with $f_i \in A$ and $\alpha_1, \dots, \alpha_n \in A$, and $\sum \alpha_i f_i = 1$ with each $A_{f_i}$ is finite type over $B$. Say $A_{f_i}$ is generated by $\frac{a_{ij}}{f_i^N}$ for common $N$ (by finiteness). Consider $A' = B[f_i, \alpha_i, a_{ij}] \subset A$. By design $f_i \in A'$ generate $1$. Furthermore,
\[ A_{f_i}' \embed A_{f_i} \]
is an equality for each $i$ and $\Spec{A'}$ is covered by these so each prime $\p' \subset A'$ does not contain $f_i$. For any $A'$-module $M$, then \[ M_{\p} = (M_{f_i})_{\p'_{f_i}} \]
Now $A' \to A$ yields an equality for 
\[ A'_{\p'} = (A'_{f_i})_{\p'_{f_i}} \embed (A_{f_i})_{\p'_{f_i}} = A_{\p'} \]
so therefore $A' \to A$ is an isomorphism of $A'$-modules.
\end{proof}

\begin{defn}
We call such maps locally of finite type.
\end{defn}

\begin{lemma}
For $f : X \to Y$ the following are equivalent:
\begin{enumerate}
\item for all affine open $U \subset Y$ the preimage $f^{-1}(U)$ is pc
\item For all qc open $U \subset Y$ the preimage $f^{-1}(U)$ is qc
\item there exists affine open cover $\{ U_i \}$ of $Y$ s.t. $f^{-1}(U_i)$ are qc.
\end{enumerate}
\end{lemma}

\begin{proof}
Clearly (b) $\implies$ (a) $\implies$ (c). Therefore we need to show (c) $\implies$ (b). For each $U_i = \Spec{B_i}$ also $f^{-1}(\Spec{(B_i)_{b_i}})$ is qc because $f^{-1}(U_i)$ is covered by finitely many $\Spec{A_{ij}}$ then $f^{-1}((U_i)_{b_i})$ is covered by $\Spec{(A_{ij})_{b_i}}$ which is quasi-compact. Notice that to prove (b) it suffices to consider affines because any qc open is covered by finitely many affines. Let $U = \Spec{B}$ be affine. Then $\Spec{B}$ is covered by $\{ \Spec{B} \cap U_i \}$ is open in $U_i$ hence covered by basis affine opens $U_{i\alpha}$ of $U_i$. Then finitely many $\{ U_{i \alpha} \}$ cover $\Spec{B}$ and each $f^{-1}(U_{i\alpha})$ is qc.
\end{proof}

\begin{defn}
We call such maps \textit{quasi-compact}. Say that $f$ is \textit{finite type} if it is qc and locally finite type.
\end{defn}

\begin{prop}
For a scheme $X$ the following are equivalent
\begin{enumerate}
\item all $\stalk{X}{x}$ are reduced
\item every affine open $\Spec{A} \subset X$ has reduced $A$
\item there exists an affine open cover $\{ \Spec{A_i} \}$ of $X$ by reduced $A_i$
\item every $\struct{X}(U)$ is reduced.
\end{enumerate}
\end{prop}

\begin{proof}
This holds just because $\stalk{X}{x} = A_\p = (A_i)_\p$ and $A$ is reduced iff $A_\p$ is reduced for all $\p$.
\end{proof}


\section{Oct. 27}

Let $k$ be an algebraically closed field. We want to define $\P^n$ as ``classical'' projective $n$-space over $k$ as a gluing of $n+1$ copied of $\A^n$. As a \textit{set}, for $V$ an $(n+1)$-dimensional $k$-vectorspace,
\[ \P(V) := 
\begin{cases}
(V - \{ 0 \})/k^\times = \{ \text{lines through the origin in } V \} & \ell = k \cdot (a_0, \dots, a_n) 
\\
(V^* - \{ 0 \})/k^\times = \{ \text{hyperplanes through the origin in } V \} & \sum_j a_j x_j = 0 
\end{cases} \]
We write $[a_0, \dots, a_n]$ for the equivalence class in $\P^n$ of $(a_0, \dots, a_n)$. The coordinate ``$a_j(p) \in k^n$'' is not well-defined because of scaling. However, the condition that ``$a_j(p) = 0$'' is well-defined. If $a_j(\varphi) \neq 0$ then the ratio $\frac{a_i(p)}{a_j(p)} \in k$ is well-defined. 

\begin{defn}
Let $U_j = \{ a_j \neq 0 \} \subset \P^n$. Let $U_j = \A^n$ as sets via,
\[ p \mapsto \left( \frac{a_i(p)}{a_j(p)} \right)_{i \neq j} \]
This is called dehomogenization with respect to $a_j$ or $x_j$. Then $\P^n$ is covered by $U_0, \dots, U_n$.
\end{defn}

\begin{rmk}
$U_i \cap U_j \subset U_j$ is the condition $\{ x_{ij} \neq 0 \} \subset \{ x_i \neq 0 \}$
\end{rmk}

\section{Oct. 29}

\subsection{Proj and More Morphism Properties}

\begin{rmk}
Read handout on irred. components and dimension and codimension for schemes (controlled by primes in $\stalk{X}{x}$ for $x \in X$).
\end{rmk}

\begin{example}
$\N$-graded rings,
\begin{enumerate}
\item $A$ a commutative ring, $S = A[x_0, \dots, x_n]$ with $\deg{x_i} = 1$ so we can write,
\[ S = \bigoplus_{d \ge 0} A[x_0, \dots, x_n]_{(d)} \]
where $A[x_0, \dots, x_n]_{(d)}$ are the homogeneous polynomials of degree $d$.
\item we can also take $S = A[x,y]$ where $\deg{x} = 2$ and $\deg{y} = 3$ then the degree decomposition becomes,
\begin{align*}
S_0 & = A
\\
S_1 & = 0
\\
S_2 & = A x
\\
S_3 & = A y
\\
S_4 & = A x^2
\\
S_5 & = A xy
\\
S_6 & = A x^3 \oplus A y^2
\end{align*}
so although the ring is the same, the graded structure is totally different.
\end{enumerate}
\end{example}

\begin{rmk}
We want to build a scheme $\Proj{S}$. To do so we need three ingredients,
\begin{enumerate}
\item the irrelevant ideal $S_+ = \bigoplus\limits_{m \ge 0} S_m$ (this corresponds to the origin that we are going to throw away).
\item For any $d > 0$, we can make,
\[ S^{(d)} = \bigoplus_{m \ge 0} S_{dm} \subset S \]
where we still give the $S_{dm}$ part degree $dm$ not degree $m$.
\item For $f \in S_d$ for $d > 0$ then,
\[ S_{(f)} = (S_f)_0 \]
where this means the degree zero part of the degree $0$ subring of the $\Z$-graded ring $S_f$ (which is graded because $f$ is homogeneous where we say $\deg{\left( \frac{a}{f^n} \right)} = \deg{a} - n\deg{f}$). Notice that $(S^{(d)})_{(f)} = S_{(f)}$ because the only terms have denominator multiples of $\deg{f} = d$ and thus the numerator in the degree zero part must also. 
\end{enumerate}
\end{rmk}

\begin{example}
For $A[x_0, \dots, x_n]_{(x_i)} = A[\frac{x_0}{x_i}, \dots, \frac{x_n}{x_i}]$ is the coordinate ring of $U_i$ the affine part of projective space.
\end{example}

\begin{rmk}
An $\N$-graded $A$-algebra is equipped with $A \to S_0 \subset S$ but this does \textit{not} need to be an equality.
\end{rmk}

\begin{rmk}
The most important case is $S$ generated by $S_1$ over $S_0$. For example the standard polynomial ring and any of its quotients. Going beyond this is useful but under Noetherian assumptions, often $S^{(d)}$ is generated by $S_d$ over $S_0$ for $d \gg 0$.
\end{rmk}

We're going to give a different definition from Hartshorne that turns out to be equivalent.


\begin{defn}
The set $\Proj{S} = \{ \text{homogeneous } \p \not\supset S_+ \}$ so that ($0$ is not in $\P^n$) and the closed sets are,
\[ V(I) = \{ \p \supset I \} \cap \Proj{S} \]
where $I$ is a homogeneous ideal. Check that $V(I) \to \Proj{S/I}$ via $\p \mapsto \p \mod I$ is an homeomorphism. 
\end{defn}

\begin{rmk}
If $\p \not\supset I$ then there is some $f \in I$ with $f \notin \p$ and if everything is homogeneous we can choose $f$ to be homogeneous as well. Then the handout shows that by Lemma 1.2 and Thm 1.3 a base of opens is,
\[ D_+(f) = \{ \p \in \Proj{S} \mid f \notin \p \} = \Proj{S} \setminus V(f) \]
for homogeneous $f \in S_+$ (the positive degree condition is important). 
\end{rmk}

\begin{prop}
$\Proj{S} = \empty \iff$ every element of $S_+$ is nilpotent
\end{prop}

\begin{thm}[2.1 + 2.2]
For homogeneous $f \in S_+$ the map,
\[ \varphi_f : D_+(f) \to \Spec{S_{(f)}} \quad \text{via} \quad \q \mapsto \q_{(f)} = \q_f \cap S_{(f)} \]
is a homeomorphism. Furthermore, let,
\[ T_f = \{ \text{homogeneous } f \in S \mid \forall \p \in D_+(f) : g \notin \p \} \]
Then,
\[ S_{(f)} \iso (T_f^{-1} S)_0 \]
so the ring $S_{(f)}$ does not depend on the choice of element representing $D_+(f)$. Moreover, suppose that $D_+(h) \subset D_+(f)$ then $T_f \subset T_h$ so get,
\[ S_{(f)} = (T_f^{-1} S)_0 \to (T_h^{-1} S)_0 = S_{(h)} \]
this identifies $S_{(h)}$ with $(S_{(f)})_{\frac{h^{\deg{f}}}{f^{\deg{h}}}}$. This enhances $D_+(h) \subset D_+(f)$ to a ``basic affine open'' 
\[ \Spec{S_{(h)}} \to \Spec{S_{(f)}} \]
\end{thm}

\begin{rmk}
$D_+(0) = \empty$ because $0 \in S_m$ for all $m$ this is okay (we don't say its degree zero).
\end{rmk}

\begin{prop}
Because $D_+(f)$ and $D_+(g)$ overlap to give $D_+(fg)$ we can define $\struct{\Proj{S}}$ via gluing the structure sheaves on $D_+(f) \iso \Spec{S_{(f)}}$.
\end{prop}

\begin{example}
$\P^n_A = \Proj{A[x_0, \dots, x_n]}$ is covered by the $n+1$ affine opens $D_+(x_i)$ where $D_+(x_i) \cong \A^n_A$ with coordinates $\frac{x_j}{x_i}$ for $j \neq i$ and $\P^n_A$ is the gluing as in the classical case.
\end{example}

\begin{example}
Let $X = \Proj{k[X,Y,Z]}$ where $\deg{X} = 2$ and $\deg{Y} = 3$ and $\deg{Z} = 4$. This is not smooth.
\end{example}

\subsection{Functoriality}

Given a graded map $\varphi : S' \to S$ we might hope to make $\Proj{S} \to \Proj{S'}$ via $\p \mapsto \varphi^{-1}(\p)$. However, what if $\varphi^{-1}(\p) \supset S'_+$. We have to remove some bad locus and get a scheme map,
\[ \Proj{S} \setminus V(\varphi(S'_+) S) \to \Proj{S'} \]
If $S'_m \to S$ is surjective for all $m \gg 0$ or even $m \gg 0$ and $d \divides m$ for fixed $d$ then $V(\varphi(S'_+)S) = \empty$.

\begin{example}
We can have,
\begin{enumerate}
\item $\varphi : S' \to S$ is surjective in all degrees $m \gg 0$
\item $S^{(d)} \embed S$
\end{enumerate}
give $\Proj{S} \embed \Proj{S'}$ a closed immersion in the first case and $\Proj{S} \iso \Proj{S^{(d)}}$ is an isomorphism in the first case.
\end{example}


\begin{example}
$\Gamma(\P^n_A, \struct{\P^n_A}) = A$ because this is,
\[ \bigcap_i A[\underline{X}]_{(X_i)} \subset \struct{\P^n_A}(D_+(X_0 \cdots X_n)) = A[\underline{X}]_{(X_0 \cdots X_n)} \]
\end{example}

\begin{rmk}
\[ \Proj{S} =  \Proj{\Z \oplus \left( \bigoplus\limits_{m \ge m_0} S_m \right)} \]
this retains its map to $\Spec{S_0}$ because each $S_m$ is an $S_0$-algebra and the $D_+(f)$ are thus affine schemes over $S_0$ (the degree zero localizations only involve positive degree stuff).
\end{rmk}

\section{Nov. 1}

\subsection{Some Examples}

\begin{example}
Two further examples of Proj ``functoriality``
\begin{enumerate}
\item $X = Z(I) \subset \P^N_k$ with ($k = \bar{k}$ and $I$ homogeneous ideal) $F_0, F_1, F_2 \in k[X_0, \dots, X_N]_d$ for $d > 0$
gives,
\[ X \rat \P^2_k \text{ via } [F_0(x) : F_1(x) : F_2(x)] \]
which is defined away from the common zeros so it is really defined on $U = \P^2_R \setminus Z(F_0, F_1, F_2)$. This exactly corresponds to $k[Y_0, Y_1, Y_2] \to k[X_0, \dots, X_n]/I$ via $Y_j \mapsto F_j$ (here we want $\deg{Y_j} = d$ for it to be a graded map but we can always multiply up the degrees). This corresponds to $\varphi : S' \to S$ has $\varphi(S_+')S = (F_0, F_1, F_2)$ so this is the locus we need to remove to make the map on Proj. 

\item Let $S = A[X_1, \dots, X_n]$ and
\[ A[T_I : I = (i_1, \dots, i_N) \text{ s.t } \sum i_j = d] \onto S^{(d)} \quad \text{via} \quad T_I \mapsto X^I = \prod X^{i_j}_j \]
gives a closed embedding,
\[ \P^N_A = \Proj{S} \cong \Proj{S^{(d)}} \embed \P_A^{{N + d \choose N} - 1} \]
called the $d$-uple or Segre embedding.

\item An example of the above is $\P^1 \embed \P^2$ given by $[x,y] \mapsto [x^2,xy,y^2]$.
\end{enumerate}
\end{example}

\subsection{Further Properties of Schemes and Morphisms}

\begin{defn}
Let $X$ be a scheme,
\begin{enumerate}
\item $X$ is \textit{integral} if $X \neq \empty$ and $\struct{X}(U)$ is a domain for all open $U \neq \empty$
\item $X$ is \textit{locally noetherian} if there is an open cover by $S_\alpha = \Spec{A_\alpha}$ for $A_\alpha$ noetherian
\item $X$ is \textit{noetherian} if it is locally noetherian and qc.
\item $f : X \to Y$ is \textit{finite} if there exists an open cover $\{ U_\alpha = \Spec{B_\alpha} \}$ of $Y$ such that $f^{-1}(U_\alpha) = \Spec{A_\alpha}$ is affine with $B_\alpha \to A_\alpha$ a finite ring map.
\end{enumerate}
\end{defn}

\begin{rmk}
The reason we need finite maps to be affine (c.f. finite type which does not) is that when we refine an affine open cover we are localizing which can mess up being finite (although it does not mess up being finite type).
\end{rmk}

\begin{thm}
$X$ is integral iff $X$ is reduced and $|X|$ is irreducible.
\end{thm}

\begin{proof}
[H, II, Prop. 3.1]. In affine case, analogue o what arose when we showed that affine algebraic set $Z \subset \A^n_k$ is irreducible iff $I(Z)$ is prime.
\end{proof}

\begin{rmk}
In [H] proof, implicit that if $U \neq \empty$ then $\struct{X}(U) \neq 0$ because $U$ contains a nonempty affine $\Spec{A}$ and thus $A \neq 0$ and $\struct{X}(U) \to A$ is a ring map so it must send $0,1$ to $0,1 \in A$ different elements so $\struct{X}(U) \neq 0$.
\end{rmk}

\begin{thm}
If $X$ is locally noetherian then all affine open $U = \Spec{A} \subset X$ have $A$ noetherian. In particular, $\Spec{B}$ is locally noetherian iff $B$ is noetherian.
\end{thm}

\begin{proof}
$X$ is covered by open $\Spec{R_\alpha}$ with $R_\alpha$ noetherian. For each $x \in \Spec{A}$ get open neighborhood $\Spec{A_f} \subset \Spec{A}$ with $A_f \cong (R_\alpha)_r$ for some $\alpha$ and $r \in R_\alpha$ by the Nike trick. Then $A_f$ is noetherian because $R_\alpha$ is noetherian. Furthermore, this is a cover so $(f_1, \dots, f_n) = A$ (finite because $\Spec{A}$ is qc) and each $A_{f_j}$ is Noetherian. 
\bigskip\\
We want to show that $A$ is noetherian. Use acc: for ideal $I_1 \subset I_2 \subset I_3 \subset \cdots$ in $A$. We have,
\[ (I_1)_{f_j} \subset (I_2)_{f_j} \subset \cdots \]
stabilizes in $A_{f_j}$ and since there are finitely many $f_j$ then there is some $M$ such that for all $m > M$,
\[ (I_m)_{f_j} = (I_{m+1})_{f_j} \]
for all $j$. We know that $\p \in \Spec{A_{f_{j_0}}}$ iff $f_{j_0} \notin \p$ but these cover so for each $\p \in \Spec{A}$ we have some $j_0$ with $f_{j_0} \notin \p$. Thus, for any $A$-module $Q$,
\[ Q_\p = (Q_{f_{j_0}})_{\p_{f_{j_0}}} \]
so $(I_m)_\p = (I_{m+1})_\p$ for all $\p$ for each $m \ge M$ inside $A_\p$ so $I_m = I_{m+1}$ inside $A$ for all $m \ge M$ and thus $A$ is Noetherian.
\end{proof}

\begin{thm}[Homework]
If $f : X \to Y$ is finite then for all affine open $\Spec{B} \subset Y$, then $f^{-1}(\Spec{B}) = \Spec{A}$ is affine and $B \to A$ is finite.
\end{thm}

\begin{proof}
Use Nike trick.
\end{proof}

\subsection{Comparison to Classical Varieties}

Consider schemes $X$ locally of finite type over $K$.

\begin{lemma}
$x \in X$ is closed iff $\kappa(x) / K$ is algebraic (and thus finite bc $X / K$ is finite type).
\end{lemma}

\begin{cor}
If $x \in U \subset X$ then $x \in U$ is closed iff $x \in X$ is closed.
\end{cor}

\begin{example}
For $Y = \Spec{R} = \{ \eta = (0), s = (\varpi) \}$ for $R$ a DVR. Let $U = \{ \eta \} = \Spec{R_{\varphi}}$ which is a basic affine open. Use $x = \eta$. Then $x \in U$ is closed by $x \in Y$ is not closed.
\end{example}

\begin{rmk}
For schemes locally of finite type over $\Z$ a point $x$ is closed iff $\kappa(x)$ is a finite field.
\end{rmk}

\begin{proof}[Proof of lemma]
For a topological space $T$, and subset $Z \subset T$ and open cover $\{U_\alpha\}$ of $T$, then $Z \subset T$ is closed if and only if $Z \cap U_\alpha \subset U_\alpha$ closed. Consider $Z = \{ z \}$ so $z \in T$ is closed iff $z$ is closed in each $U_\alpha$ containing $z$. We apply this to $T = X$ and $\{ U_\alpha = \Spec{A_\alpha} \}$ is an affine open cover and $Z = \{ x \}$. This reduces to the following: $A$ is a finite type $K$-algebra $\p \subset A$ is prime we want to show that $\p$ is maximal iff $\kappa(\p) / K$ is algebraic. 
\bigskip\\
Note that $\kappa(\p) = \Frac{A / \p}$ but $\trdeg{K}{\Frac{A/\p}} = \dim{A/\p}$ (see [Mat, 5]) so $\trdeg{K}{\Frac{A/\p}}$ iff $\dim{A/\p} = 0$ iff $A / \p$ is a field ($A/\p$ is a domain).
\end{proof}
\noindent
Now we discuss some consequences. 

\begin{defn}
$X(K) = \Hom{K}{\Spec{K}}{X} = \{ x \in X \mid K \iso \kappa(x) \}$ giving a section of $X \to \Spec{K}$.
\end{defn}

\begin{lemma}
For a morphism $f : X \to Y$ of schemes locally of finite type over $K$ the image of a closed point is a closed point. Furthermore, $f : X(K) \to Y(K)$.
\end{lemma}

\begin{proof}
That $f : X(K) \to Y(K)$ is an obvious general fact (for any scheme replacing $\Spec{K}$ even). Furthermore, for $x \in X$ closed,
\begin{center}
\begin{tikzcd}
\kappa(x) & &  \kappa(f(x)) \arrow[ll, hook]
\\
& K \arrow[lu,"K"] \arrow[ru]
\end{tikzcd}
\end{center}
But $K \to \kappa(x)$ is finite so $K \to \kappa(f(x))$ is also finite since $\kappa(f(x)) \embed \kappa(x)$ and thus $f(x) \in Y$ is closed.
\end{proof}

\subsection{Relation to the Classical Theory}

If $T(X) = \{ \text{closed points} \}$ with $\struct{T(X)} = \iota^{-1} \struct{X}$. then $(T(X), \struct{T(X)})$ is quite concrete for $\Spec{A}$ this is just $\mSpec{A}$ with the expected topology and sheaf. 
\bigskip\\
When $K = \overline{K}$ we get an equivalence of categories,
\[ \{ \text{reduced schemes locally finite type over } K \} \cong \{ \text{abstract algebraic sets over } K \} \]

\begin{defn}
If $L / K$ is any field extension then the $L$-valued points are,
\[ X(L) = \Hom{K}{\Spec{L}}{X} = \{ (x, \varphi) \mid x \in X \text{ and } \varphi : \kappa(x) \embed L \text{ over } K \} \]
these are something like the ``solutions in $L$ to the equations in $K$''.
\end{defn}

\subsection{Nov. 3}

\subsubsection{Fiber Products and Universal Structures}

\begin{example}
Consider $V(x - y^2) \mapsto \Spec{k[x]}$. Let $B = k[x,y]/(x-y^2) \cong k[y]$ and $A = k[x] \to B$ via $x \mapsto y^2$. Then we compute,
\[ B \otimes_A \kappa(P) = k[y] \otimes_{k[x]} k[x]/(x-a) = k[y]/(y^2 - a) = 
\begin{cases}
k \times k & a \neq 0
\\
k[y]/(y^2) & a = 0
\end{cases} \]
At $a = 0$ this is nonreduced and $\dim_k = 2$ showing that it remembers the degree of the point even though there is only one geometric point in the fiber.
\end{example}

\begin{example}
An elliptic curve family. Consider $k$ of characteristic not $2$ and $3$. Then $a(t), b(t) \in k[t]$ with $\Delta(t) = 4 a^3 - 27b^2 \neq 0$. Consider,
\[ E_{t_0} = \{ y^2z = x^3 + a(t_0) x z^2 + b(t_0) z^3 \} \subset \P^2_k \]
This is smooth when $\Delta(t_0) \neq 0$ for $t_0 \in k$. Let $A = k[t][\Delta^{-1}]$. Then, let
\[  \E = \{ y^2 z = x^3 + a(t) z^2 + b(t) z^3 \} \]
\begin{center}
\begin{tikzcd}
\E_{t_0} \arrow[d] \arrow[r] & \E \arrow[r, hook] \arrow[d] & \P^3_A \arrow[dl]
\\
\Spec{k} \arrow[r, "t_0"] & \Spec{A}
\end{tikzcd}
\end{center}
we want a way to put a natural scheme structure on the topological fiber $\E_{t_0}$ so it agrees with $E_{t_0}$. 
\end{example}

How do we find the correct scheme structure on $f^{-1}(s)$. We define it using the universal property,
\begin{center}
\begin{tikzcd}
T \arrow["\exists !", rd, dashed] \arrow["g_2", rrd, bend left] \arrow["g_1", rdd, bend right]
\\
& X_s \arrow[r] \arrow[d] & X \arrow[d]
\\
& \Spec{\kappa(s)} \arrow[r, "s"] & S
\end{tikzcd}
\end{center}
given $T$ and maps $g_1 : T \to \Spec{\kappa(s)}$ and $g_2 : T \to X$ that make the diagram commute then there exists a unique $T \to X_s$ making the diagram commute. 
\bigskip\\
I can get more, the infinitesimal fibers, say $S = \Spec{R}$, for $R$ is local and $s = \{ \m \}$ closed point $k = R / \m$ then,
\begin{center}
\begin{tikzcd}
X_s \pullback \arrow[r] \arrow[d] & X_{s, R} \pullback \arrow[r] \arrow[d] & X \arrow[d]
\\
\Spec{k} \arrow[r] & \Spec{R} \arrow[r, "s"] & S
\end{tikzcd}
\end{center}

\begin{example}
Direct product over a base. For $R$ a ring. Consider,
\begin{center}
\begin{tikzcd}
T \arrow[rd, "g_1"] \arrow[rr, dashed] & & \A^{n+m}_R \arrow[ld, "\pi_1"'] \arrow[rd, "\pi_2"] 
\\
& \A^n_R \arrow[rd] &  & \A^m_R \arrow[ld]
\\
& & \Spec{R}
\end{tikzcd}
\end{center}
Given maps $T \to \A^n_R$ and $R \to \A^m_R$ that agree over $\Spec{R}$ I claim there is a unique map $g : T \to \A^{n+m}_R$. The map $g_1$ is equivalent to $R[x_1, \dots, x_n] \to \struct{T}(T)$ as $R$-algebras and thus equivalent to an $n$-tuple $f_1, \dots, f_n \in \struct{T}(T)$. Thus, $(g_1, g_2) \iff T \to \A^{n+m}_R$ over $R$ naturally. 
\end{example}

\subsection{The Construction of Fiber Products}

For an $S$-scheme $X$, get a contravariant functor $h_{X/S} : \mathbf{Sch}_S \to \mathbf{Set}$ via $T \mapsto h_X(T) = \Hom{S}{T}{X}$. We think of this as $T$-valued solutions to $X$ over $S$ when $S = \Spec{R}$ is a Noetherian ring and $X = \Spec{A}$ is finitely presented over $S$ then it is literally given by polynomial equations with coefficients in $R$ and thus for $T = \Spec{B}$ maps $T \to X$ over $S$ are exactly $B$-valued solutions to the equations defining $X$ where the structure map $R \to B$ determines how the coefficients are interpreted in $B$.


Brian's oddball nonsequitor humor coupled with random profanity makes his class a very disorienting experience. 

\begin{example}
For $S$-scheme $T$, we see $\A^{n+m}_S(T) = \A^n_S(T) \times \A^m_S(T)$ (these are just sets). This tells us that $h_{\A^{n+m}_S}$ is naturally isomorphic to $h_{\A^n_S} \times h_{\A^m_S}$ so we do see a hint of a product structure. However, usually (including $k = \bar{k}$) there is no homeomorphism $|\A^{n+m}| \to |\A^n| \times |\A^m|$ it's not even bijective!
\bigskip\\
Let's look at the $n = m = 1$ case. For any irreducible curve in $\A^2$ that is not parallel to one of the axes defines a generic point in $\A^2$ which is not in $|\A^1| \times |\A^1|$.
\end{example}

The goal: given $S$-schemes $X,Y$ we see a ``relative direct product'' $X \times_S Y \to S$ with $s$-fiber equal to $X_s \times_{\kappa(s)} Y_s$ whatever that's supposed to be. 

\begin{defn}
Given, two $S$-schemes $X \times_S Y$ we construct,
\begin{center}
\begin{tikzcd}[row sep = small, column sep = small]
& P = X \times_S Y \arrow[rd, "p_1"'] \arrow[ld, "p_2"]
\\
X \arrow[dr] & & Y \arrow[dl]
\\
& S
\end{tikzcd}
\end{center}
with the universal property that given $g_1 : T \to X$ and $g_2 : T \to Y$ over $S$ there is a unique map $g : T \to P$ such that $g_i = p_i \circ g$. We can say that $h_P(T) \iso h_X(T) \times h_Y(T)$ naturally in $T$ via the maps $p_1, p_2$. The triple $(P, p_1, p_2)$ is called the \textit{fiber product} of $X$ and $Y$ if it exists.
\end{defn}

\begin{example}
Suppose that $S = \Spec{R}$ and $X = \Spec{A}$ and $Y = \Spec{B}$. Then,
\[ h_X(T) = \left \{ 
\begin{tikzcd}[row sep = small, column sep = small]
T \arrow[r] \arrow[rd] & \Spec{A} \arrow[d]
\\
& \Spec{R}
\end{tikzcd}
\right\} = \Hom{R}{A}{\struct{T}(T)} \]
Likewise $h_Y(T) = \Hom{R}{B}{\struct{T}(T)}$. However, notice that tensor product satisfies the opposite universal property,
\begin{center}
\begin{tikzcd}
B \arrow[rd, "j_2 : b \mapsto 1 \otimes b"'] \arrow[rrd, bend left]
\\
& A \otimes_R B \arrow[r, "\exists ! g", dashed] & \struct{T}(T)
\\
A \arrow[ur, "j_1 : a \mapsto a \otimes 1"] \arrow[rru, bend right]
\end{tikzcd}
\end{center}
over $R$. Therefore, in the affine case we get the fiber product,
\begin{center}
\begin{tikzcd}
& \Spec{A \otimes_R B} \arrow[rd, "\Spec{j_1}"] \arrow[ld, "\Spec{j_2}"]
\\
\Spec{A} & & \Spec{B}
\end{tikzcd}
\end{center}
over $\Spec{R}$.
\end{example}  

Next time we use the affine case and suitable affine covers to build $X \times_S Y$ in general. Furthermore, we Weil see that,
\[ X_S \times \Spec{\kappa(s)} \iso \text{s-fiber} \]
is a homeomorphism.

\subsection{Viewpoints on the Fiber Product}

Given $X, Y$ over $S$. Then $F : \mathbf{Sch}_S \to \mathbf{Set}$ is contravariant $T \mapsto X(T) \times Y(T)$ so $F = h_X \times h_Y$. Then the fiber product $(P, p_1, p_2) \in F(P) = X(P) \times Y(P)$ is a representing object for $F$ since,
\[ \theta_P : h_P \to F \]
given by sending $\Hom{S}{T}{P} \to F(T) = X(T) \times Y(T)$ sending $g \mapsto (p_1 \circ g, p_2 \circ g)$. The universal property is exactly that $\theta$ is a natural isomorphism of functors. 

\begin{lemma}
Let $\C$ be a category $F : \C \to \mathbf{Set}$ contravariant. Given a representing object $\xi \in \C$ and $\theta : F \iso \Hom{\C}{-}{\xi}$ then,
\[ \theta_\xi : \Hom{\C}{\xi}{\xi} \iso F(\xi) \]
sending $\id_\xi \mapsto \alpha_\xi \in F(\xi)$. Then $(\alpha_\xi, \theta)$ is the universal structure in the following sense meaning for any $T \in \C$ and any $t \in F(T)$ there is a unique map $T \to \xi$ such that $F(\xi) \to F(T)$ sending $\alpha_\xi \mapsto t$. 
\end{lemma}

\begin{proof}
For any $f : T \to \xi$ because $\theta$ is a natural isomorphism, there is a diagram,
\begin{center}
\begin{tikzcd}
F(T) \arrow[from=r, "\theta_T"] & \Hom{\C}{T}{\xi}
\\
F(\xi) \arrow[u, "F(f)"] \arrow[from=r, "\theta_\xi"] & \Hom{\C}{\xi}{\xi} \arrow[u, "- \circ f"] 
\end{tikzcd}
\end{center}
then $\theta_\xi(\id) = \alpha_\xi$ by definition. Then by commutativity, $F(f)(\alpha_\xi) = \theta_T(f)$.
\end{proof}

\subsection{Yoneda's Lemma and Fiber Products}

\begin{example}
Suppose that $S = \A^2$ and $X = \{ x = 0 \}$ and $Y = \{ x = 1 \}$ two parallel lines. Then $X \times_S Y = \empty$. Indeed,
\[ k[x,y]/(x) \otimes_{k[x,y]} k[x,y]/(x-1) = k[x,y]/(x,x-1) = 0 \]
In general,
\[ \Spec{A/I} \times_{\Spec{A}} \Spec{A/J} = \Spec{(A/I \otimes_A A/J)} = \Spec{A / (I + J)} \]
\end{example}

\begin{lemma}
Let $F : \C \to \mathbf{Set}$ be a contravariant functor and a representing object: $\xi \in \C$ with $\theta : F \iso h_\xi = \Hom{\C}{-}{\xi}$ then $\theta : F(\xi) \iso \Hom{\C}{\xi}{\xi}$ so there is $\alpha_\xi \in F(\xi)$ corresponding to $\id_\xi$. Then $(\alpha_\xi, \theta_\xi)$ is a universal structure: for any $t \in F(T)$ there is a unique map $f_t : T \to \xi$ (the classifying map) so that $f_t^* \alpha_\xi = t$.
\end{lemma}

\begin{example}
For $\C = \mathbf{Sch}_R$ for $F(T) = \struct{}(T)^n \cong \Hom{\C}{-}{\A^n_R}$ then the representing object is $\alpha_\xi = (x_1, \dots, x_n) \in F(\A^n_R)$.
\end{example}

\begin{example}
$\P^n_k$ ``classifies'' lines in $k^{n+1}$ or $L^{n+1}$ for all $L/k$ but this is not enough to actually define a moduli problem because we need to consider its $B$-points for all $k$-algebras $B$ or its $T$-points for all $k$-schemes $T$. Then $\P^n_k$ gets a ``universal line'' which will correspond to a line bundle with canonical generating sections.
\end{example}

\begin{rmk}
We say that such $F$ are representable and it represented by $\xi$ via $\alpha_\xi$. Furthermore, we can reconstruct $\theta$ from $\alpha_\xi$ via,
\[ \theta_T : F(T) \iso \Hom{\C}{T}{\xi} \]
sending $t \mapsto f_t$.
\end{rmk}

\begin{cor}
If $(\xi, \alpha_\xi)$ and $(\xi', \alpha_\xi')$ are two structures representing $F$ then,
\begin{center}
\begin{tikzcd}
& F
\\
h_{\xi'} \arrow[ru, "\theta'"] \arrow[rr, dashed] & & h_{\xi} \arrow[lu, "\theta"']
\end{tikzcd}
\end{center}
arises from a unique isomorphism $\xi' \iso \xi$ sending $\alpha_\xi \mapsto \alpha_{\xi'}$.
\end{cor}

\begin{proof}
We get $h_{\xi'}(\xi') \iso F(\xi') \iso h_{\xi}(\xi') = \Hom{\C}{\xi'}{\xi}$ then the image of $\id$ gives a map $f : \xi' \to \xi$ such that $\id_\xi' \mapsto \alpha_\xi' \mapsto f$ so $F(f) : \alpha_\xi \mapsto \alpha_{\xi'}$. Likewise we get $f' : \xi \to \xi'$ then we check via uniqueness that $f \circ f' = \id_{\xi}$ and $f' \circ f = \id_{\xi'}$ because they preserve $\alpha_\xi$ and $\alpha_\xi'$. 
\end{proof}

\begin{lemma}[Yoneda]
Let $X, Y \in \C$ then any natural transformation $\theta : h_X \to h_Y$ arises as $f \circ -$ for a unique map $f : X \to Y$.
\end{lemma}

\begin{proof}
Consider,
\begin{center}
\begin{tikzcd}
h_X(X) \arrow[r, "\theta_X"] & h_Y(X)
\end{tikzcd}
\end{center}
then $\id_X \mapsto f \in \Hom{\C}{X}{Y}$. This is the unique thing that can work because if $\theta(-) = \tilde{f} \circ -$ so $\theta_X(\id_X) = \tilde{f} \circ \id_X = \tilde{f}$ so $\tilde{f} = f$. Now we need to show that $\theta = f \circ -$. Let $T \in \C$ and $x \in h_X(T)$ meaning $x : T \to X$. Then consider,
\begin{center}
\begin{tikzcd}
h_X(T) \arrow[r, "\theta_T"] & h_Y(T)
\\
h_X(X) \arrow[u, "- \circ x"] \arrow[r, "\theta_X"] & h_Y(X) \arrow[u, "- \circ x"']
\end{tikzcd}
\end{center}
Therefore, $\id \mapsto x \mapsto \theta_T(x)$ and also $\id \mapsto \theta_X(\id) = f \mapsto f \circ x$ and therefore $\theta_T(x) = f \circ x$ so $\theta = f \circ -$ proving what we wanted.
\end{proof}

\begin{rmk}
How can this linguistic triviality be useful! The utility is we can often find a rich subcategory $\C' \subset \C$ (e.g. affine schemes inside schemes) to be like a ``base for the topology'' to build natural transformations $F_1 \to F_2$ between functors on $\C$ even if $\xi \notin \C'$. 
\end{rmk}

\subsection{Construction of Fiber Products}

We've done the affine case (universal on all schemes over $S$. Now we consider the general setting, $X, Y$ are $S$-schemes. We want to glue together the fiber products for compatible affine opens. 
\bigskip\\
First suppose that the fiber product $X \times_S Y$ exists. Then for any $U \subset X$ and $V \subset Y$ and $W \subset S$ compatibly meaning $U \to W$ and $V \to W$. Then I claim that $p_1^{-1}(U) \cap p_2^{-1}(V) \subset X \times_S Y$ is $U \times_W V$.  

\section{Nov. 8}

\subsection{Applications of Fiber Products}

Some loose ends:
\begin{enumerate}
\item $(X \times_S S') \times_{S'} S'' \cong X \times_S S''$ over $S''$. To see this, evaluate the functor of points on $S''$-schemes.
\item $|X \times_Z Y| \onto |X| \times_{|Z|} |Y|$ is always surjective although not usually injective. Given $x, y \mapsto z$ then we can form $\kappa(x) \otimes_{\kappa(z)} \kappa(y) \neq 0$ because these are nonzero vectorspaces over $\kappa(z)$. Thus, $\Spec{\kappa(x) \otimes_{\kappa(z)} \kappa(y)}$ is nonempty and thus there is a point mapping to $(x,y)$.
\item Think of $f : X \to Y$ as ``family of schemes'' $\{ X_y \}_{y \in Y}$ as $\kappa(y)$-schemes.
\item $X \ot_A B$ or $X_B$ denotes $X \times_{\Spec{A}} \Spec{B}$.
\end{enumerate}

\subsection{Applications of Fiber Products}

\subsubsection{Remarks}

\begin{enumerate}
\item $\A^n_R = \A^n_\Z \otimes_\Z R$ as $R$-schemes
\item $\P^n_R = \P^n_\Z \otimes_\Z R$ as $R$-schemes
\item For $A \to S_0$ (so the base change does not mess up the graded pieces) $\Proj{S} \otimes_A A' = \Proj{S \otimes_A A'}$.
\end{enumerate}

\subsubsection{Scheme Theoretic Intersection}

For $Z, Z' \embed X$ closed subschemes then $Z \times_X Z' \to X$ is a closed immersion with $|Z \times_X Z'| = |Z| \cap |Z'|$. Indeed, WLOG $X = \Spec{A}$ and $Z = \Spec{A/I}$ and $Z' = \Spec{A/I'}$ then,
\[ Z \cap Z' := Z \times_X Z' = \Spec{A/I \otimes_A A/I'} = \Spec{A/(I + I')} \]
This also has a good mapping property: $T \to X$ factors through $Z$ and $Z'$ iff it factors through $Z \times_X Z'$. However, if $T$ is nonreduced then it may not factor through the reduced structure on $(Z \cap Z')$ so this scheme structure is the ``right one''. 

\begin{rmk}
If $X, Z, Z'$ are smooth over $k = \bar{k}$ and $x \in Z \cap Z'$ with $T_x(Z)$ and $T_x(Z') \subset T_x(X)$ are transverse then $Z \cap Z'$ is smooth at $x$.
\end{rmk}

\subsubsection{Cartesian}

We say that a commutative square is Cartesian if,
\begin{center}
\begin{tikzcd}
P \arrow[r] \arrow[d] & X \arrow[d]
\\
Y \arrow[r] & S
\end{tikzcd}
\end{center}
the natural map $P \to X \times_S Y$ is an isomorphism. Fiber products interact well with base change in the sense that,
\begin{center}
\begin{tikzcd}
X' \times_{S'} Y' \arrow[d] \arrow[r] & X \times_S Y \arrow[d]
\\
S' \arrow[r] & S
\end{tikzcd}
\end{center}
is Cartesian where $X', Y' \to S'$ are the base changes of $X,Y \to S$. To see this, evaluate on an $S'$-scheme $T'$. Then,
\[ ((X \times_S Y) \times_S S')_{S'}(T') = (X \times_S Y)_S(T') = X_S(T') \times Y_S(T') = X'(T') \times Y'(T') = (X' \times_{S'} Y')(T') \]
where the natural map $X'(T') \to X_S(T')$ is given by ``forgetting'' the $S'$-structure or composing with $X' \to X$ are bijective by the universal property.
\bigskip\\
Key example, let $S' = \Spec{\kappa(s)} \to S$ then $(X \times_S Y)_s = X_s \otimes_{\kappa(s)} Y_s$. 

\subsubsection{Flatness}

Let $f : X \to Y$ be a flat morphism of locally noetherian schemes. Take $x \in X$ and $y = f(x)$. Then,
\[ \dim \stalk{X}{x} = \dim{\stalk{Y}{y}} + \dim{\stalk{X_y}{x}} \]
For a proof [Mat, Thm. 15.1] use that if $\varphi : A \to B$ is a flat map of noetherian rings and $\p \subset B$ is a prime and $\q = \varphi^{-1}(\p)$ then,
\[ \dim{B_\p} = \dim{A_\q} + \dim{B \otimes_A \kappa(\q)} \]
and $B \otimes_A \kappa(\q) = (B / \q B)_\q$.
\bigskip\\
In general, for non-flat maps there is an inequality,
\[ \dim{\stalk{X}{x}} \le \dim{\stalk{Y}{y}} + \dim{\stalk{X_y}{x}} \] 

\subsubsection{The Diagonal}

\begin{defn}
For $f : X \to S$ the \textit{relative diagonal} is $\Delta_f : X \to X \times_S X$ given by the identity maps $X \to X$.
\end{defn}

\begin{rmk}
In the affine case, $\Spec{B} \to \Spec{A}$ then $\Delta : \Spec{B} \to \Spec{B \ot_A B}$ is induced by the map $\delta : B \leftarrow B \otimes_A B$ given by sending $b_1 \otimes b_2 \mapsto b_1 b_2$. To see this, the projections $\Spec{B \ot_A B} \to \Spec{A}$ take $b \mapsto b \otimes 1$ and $b \mapsto 1 \otimes b$ and therefore since $\Delta$ is a ring map such that it sends $b_1 \otimes 1 \mapsto b_1$ and $1 \otimes b_2 \mapsto b_2$ because it is the identity when composed with the projections. Therefore, $b_1 \otimes b_2 \mapsto b_1 b_2$. Therefore, $\delta : B \otimes_A B \to B$ is surjective so $\Delta_{B/A}$ is a closed immersion.
\end{rmk}

\begin{rmk}
In general, $(\Delta_{X/S}) \times_S S' = \Delta_{X'/S'}$ which is easily checked on $S'$-pints. For $\{ S_i \}$ an open cover of $S$ then $\Delta_{X/S} : X \to X \times_S X$ restricts under base change to the $\Delta_{X_i/S_i}$ for $X_i = f^{-1}(S_i)$ with $f : X \to S$.
\end{rmk}
\noindent
Thus, we fix an open affine cover $\{ X_{ij} \}_{j \in J_i}$ of $X_i = f^{-1}(S_i)$. Then,
\begin{center}
\begin{tikzcd}
& X \arrow[r, "\Delta"] & X \times_S X
\\
\Delta^{-1}(V_{ij}) \arrow[r] & X_{ij} \arrow[u, hook] \arrow[r, "\Delta"] & X_{ij} \times_{S_i} X_{ij} = V_{ij} \arrow[u, hook]
\end{tikzcd}
\end{center}
Then this is actually Cartesian. Furthermore, let,
\[ V = \bigcup_{i,j} V_{ij} \subset X \otimes_S X \]
is an open subscheme. Then $\Delta(X) \subset V$ and because $X_{ij}$ is affine $\Delta : X_{ij} \to X_{ij} \times_{S_i} X_{ij}$ is a closed immersion. Since being a closed immersion is a local property on the target we get,
\begin{center}
\begin{tikzcd}
X \arrow[rr, bend left, "\Delta"] \arrow[r, hook, "\text{closed im}"'] & V \arrow[r, hook, "\text{open im}"'] & X \times_S X
\end{tikzcd}
\end{center}

\begin{defn}
Say that $g : Y \to Z$ is an \textit{immersion} if
\begin{enumerate}
\item $g$ is a homeomorphism onto $g(Y) \subset Z$ which is locally closed
\item for some (equivalently all)  open $V \subset Z$ with $g(Y) \subset V$ closed and $Y \to V$ is a closed immersion.
\end{enumerate}
\end{defn}

\begin{lemma}
An immersion is a closed immersion if and only if it has closed image.
\end{lemma}

\begin{defn}
Say $X$ is $S$-separated when $\Delta_{X/S}(X) \subset X \times_S X$ is closed or equivalently $\Delta_{X/S}$ is a closed immersion.
\end{defn}

\subsection{Nov. 10}

\subsection{Separatedness}

For $f : X \to S$ we factored the diagonal,
\begin{center}
\begin{tikzcd}
X \arrow[rr, bend left, "\Delta"] \arrow[r, hook, "\text{closed im}"'] & V \arrow[r, hook, "\text{open im}"'] & X \times_S X
\end{tikzcd}
\end{center}
and therefore say that $\Delta_{X/S}$ is an immersion. Immersions have reasonable properties: stable under base change and composition. Furthermore a closed immersion is exactly an immersion with closed image. However, notice that an immersion with open image is usually \underline{not} an open immersion e.g. 
\[ Y_{\red} \embed Y \]
If $\{ S_i \}$ is an open cover of $S$, and $X_i = f^{-1}(S_i) \subset X$, then pulling back $\Delta_f ; X \to X \times_S X$ over $S_i$ gives exactly,
\begin{center}
\begin{tikzcd}
X_i \arrow[rr, "\Delta_{X_i/S_i}"] \arrow[rd] & & X_i \times_{S_i} X_i \arrow[ld] \arrow[r, equals] & (X \times_S X)|_{S_i}
\\
& S_i
\end{tikzcd}
\end{center}
Since closedness can be checked locally \textit{on the target} then $\Delta_f$ is a closed immersion if and only if each $\Delta_{X_i / S_i}$ is a closed immersion iff each $X_i \to S_i$ is separated. Therefore, separatedness is ``local on the base''. 

\begin{prop}
Affine maps are separated.
\end{prop}

\begin{proof}
Work locally on the base then we reduce to the case of a map of affine schemes which we know is separated.
\end{proof}

\begin{example}
Both $\P^1_k$ and the line with a double origin are the gluing of $\A^1_k$ and $\A^1_k$ along $U_1 = \{ x \neq 0 \} = U_2$ via either the map $j : U_i \to U_i$ by $x \mapsto x$ or $x \mapsto x^{-1}$. Consider the graph,
\[ \Gamma_j \subset \A^1_k \times \A^1_k = \A^2_k \]
In the first case, $\Gamma_j = \{ x = y, x \neq 0 \}$ which is not closed but in the second case $\Gamma_j = \{ xy = 1 \}$ which is closed in $\A^2$. 
\end{example}

\begin{example}
We will see that $\Proj{S}$ is always separated over $\Spec{S_0}$. 
\end{example}

\begin{prop}
A morphism of schemes $f : X \to S$ is separated iff $f_{\red} : X_{\red} \to S_{\red}$ is separated.
\end{prop}

\begin{proof}
We just need to show that $\Delta_f$ has closed image. However, the following map is the same topologically,
\begin{center}
\begin{tikzcd}
X_{\red} \arrow[rdd, "\Delta_{f_{\red}}"', bend right] \arrow[r, "(\Delta_f)_{\red}"] & (X \times_S X)_{\red} \arrow[d, equals]
\\ 
& (X_{\red} \times_{S_{\red}} X_{\red})_{\red} \arrow[d, hook, "\text{homeo}"]
\\
& X_{\red} \times_{S_{\red}} X_{\red}
\end{tikzcd}
\end{center}
Therefore $\Delta_f$ has closed image if and only if $(\Delta_f)_\red$ has closed image if and only if $\Delta_{f_{\red}}$ has closed image.
\end{proof}

\subsection{Graph Construction}

Given an $S$-morphism,
\begin{center}
\begin{tikzcd}
X \arrow[rd] \arrow[rr, "f"] & & Y \arrow[ld]
\\
& S 
\end{tikzcd}
\end{center}
yields a morphism $\Gamma_f : X \to X \times_S Y$ given by the pair $(\id, f)$. Thus, on $T$ points it is given by $x \mapsto (x, f(x))$ where $f(x) = f \circ x$. Then the following diagram,
\begin{center}
\begin{tikzcd}
X \arrow[d, "f"] \arrow[r, "\Gamma_f"] & X \times_S Y \arrow[d, "f \times \id_Y"] 
\\
Y \arrow[r, "\Delta_{X/Y}"] & Y \times_S Y 
\end{tikzcd}
\end{center}
is Cartesian. By Yoneda, we can check the Cartesian property at the level of sets of $T$-points (basically this is just the definition of this diagram being a limit). On $T$-points, given $(x,y)$ and $y'$ such that $(f(x), y) = (y',y')$ then $y = y' = f(x)$ so $(x,y)$ and $y'$ come from the unique point $x \mapsto (x, f(x))$ and $x \mapsto f(x)$.
\bigskip\\
If $Y$ is separated over $S$ then $\Delta_{Y/S}$ is a closed immersion so by base change $\Gamma_f$ is also a closed immersion. In general, $\Delta_{Y/S}$ is always an immersion so $\Gamma_f$ is always an immersion. 
\bigskip\\
Therefore, for $Y \to S$ separated,
\[ \Hom{S}{X}{Y} \iso \{ Z \embed X \times_S Y \mid \pi_1 : Z \iso X \} \quad \text{via} f \mapsto \Gamma_f \]
Conversely, given $Z \embed X \times_S Y$ then consider,
\begin{center}
\begin{tikzcd}
Z \arrow[r, hook] \arrow[d, "\sim"] & X \times_S Y \arrow[d, "\pi_2"]
\\
X \arrow[r, dashed] & Y
\end{tikzcd}
\end{center}
Gives a morphism $X \to Y$ by inverting $Z \to X$. Furthermore, $Z = \Gamma_f$ because they have the same $T$-points.
\bigskip\\
A special case is $X = S$ then we are considering sections $\sigma : S \to Y$ of $\pi : Y \to S$ meaning $\pi \circ \sigma = \id$. Then we see that,
\[ Y(S) = \{ Z \embed Y \mid Z \iso S \} \]

\begin{thm}
We have the following,
\begin{enumerate}
\item any monomorphism $f : X \to S$ is separated
\item $X \xrightarrow{f} Y \xrightarrow{g} Z$ if $f, g$ are separated then $g \circ f$ is separated
\item if $f : X \to Y$ is separated then any base change $f' : X' \to Y'$ is separated
\item if $f : X \to Y$ and $f' : X' \to Y'$ are separated $S$-maps then $f \times f' : X \times X' \to Y \times Y'$ is also separated. (e.g. if $Y = S = Y'$ then $X \times_S X'$ is $S$-separated if $X$ and $X'$ are)
\item $X \xrightarrow{f} Y \xrightarrow{g} Z$ if $g \circ f$ is separated then $f$ is separated.
\end{enumerate}
\end{thm}

\begin{rmk}
The proof in Hartshorne is crazy and uses valuative criterion.
\end{rmk}

\begin{rmk}
Given a diagram,
\begin{center}
\begin{tikzcd}
X \arrow[rd] \arrow[rr, "f"] & & S \arrow[dl, "\text{separated}"]
\\
& \Spec{\Z}
\end{tikzcd}
\end{center}
therefore we see from the theorem that $X \to S$ is separated iff $X \to \Spec{\Z}$ is separated.
\end{rmk}

\begin{proof}
(a) $f$ is monic iff $\Delta_f$ is an isomorphism in which case $\Delta_f$ is a closed immersion. For (b) consider,
\begin{center}
\begin{tikzcd}
X \arrow[r, "\Delta_{g \circ f}"] & X \times_Z X
\end{tikzcd}
\end{center}

For (c) if you make a base change $\Delta_f \times_Y Y' = \Delta_{f'}$ and the base change of a closed immersion is a closed immersion.

For (d) then $X \times_S X' \xrightarrow{\id_X \times f'} X \times_S Y' \xrightarrow{f \times \id_{Y'}} Y \times_S Y'$ but the first is a base change of $f'$ and the second of $f$ so we conclude by the previous. 
\end{proof}

\begin{cor}
Then the following also follow,
\begin{enumerate}
\item if $f : X \to Y$ is separated and $Y$ is separated and $U \subset X$ affine open and $V \subset Y$ affine open then $U \cap f^{-1}(V) \subset X$ is affine.
\item If $X$ is separated and $U, V \subset X$ affine open then $U \cap V$ is affine.
\end{enumerate}
\end{cor}

\begin{proof}
It is clear that (a) $\implies$ (b) by taking $X = Y$. Then consider,
\begin{center}
\begin{tikzcd}
U \cap f^{-1}(V) \arrow[d] \arrow[r, hook] & U \times_{\Z} V \arrow[d, hook]
\\
X \arrow[r, "\Gamma_f"] & X \times_{\Z} Y 
\end{tikzcd}
\end{center}
Since $\Gamma_f$ is a closed immersion we see that $U \cap f^{-1}(V) \embed U \times_\Z V$ is a closed immersion. However, $U \times_{\Z} V$ is affine and thus any closed subscheme is affine.
\end{proof}

\subsection{Separatedness criterion}

\begin{lemma}
Let $\{ U_i \}$ be an affine open cover of $R$-scheme $X$. Then $X \to \Spec{R}$ is separated if and only if,
\begin{enumerate}
\item $U_i \cap U_j$ is affine 
\item $\struct{X}(U_i) \otimes_{\Z} \struct{X}(U_j) \onto \struct{X}(U_i \cap U_j)$ is surjective
\end{enumerate}
\end{lemma}

\begin{proof}
Consider the Cartesian diagram,
\begin{center}
\begin{tikzcd}
U_i \cap U_j \arrow[d, hook] \arrow[r, "\varphi_{ij}"] & U_i \times_R U_j \arrow[d, hook]
\\
X \arrow[r, "\Delta"] & X \times_R X
\end{tikzcd}
\end{center}
Therefore, $\Delta$ is a closed immersion if and only if each $\varphi_{ij}$ is a closed immersion. 
\end{proof}

\begin{prop}
$\Proj{S} \to \Spec{S_0}$ is separated.
\end{prop}

\begin{proof}
Consider the open affine cover $\{ U_i = D_+(f_i) \}$ for $f_i \in S_+$. Then $D_+(f) \cap D_+(g) = D_+(fg)$ which is affine. Thus we just need to consider,
\[ S_{(f)} \otimes_{S_0} S_{(g)} \to S_{(fg)} \]
However, $S_{(fg)} = (S_{(g)})_{\frac{f^n}{g^m}}$ so the map is given by,
\[ \left( \frac{g^m}{f^n} \right)^r \otimes \alpha \mapsto \frac{\alpha}{\left( \frac{f^n}{g^m} \right)^r} \]
which is surjective.
\end{proof}

\section{Nov. 12}

\subsection{Applications of Separatedness}

\begin{prop}
Let $f, g : X \to Y$ be morphisms with $Y$ separated. Let $j : U \embed X$ be an open subscheme that is schematically dense meaning $\struct{X} \embed j_* \struct{U}$ is injective. Then if $f|_U = g|_U$ then $f = g$.
\end{prop}

\begin{proof}
Consider,
\begin{center}
\begin{tikzcd}
U \arrow[rrd, bend left] \arrow[rdd, bend right] \arrow[rd, dashed]
\\
& Z \pullback \arrow[d, hook] \arrow[r] & Y \arrow[d, "\Delta", hook]
\\
& X \arrow[r] & Y \times_S Y 
\end{tikzcd}
\end{center}
because $\Delta_{Y/S} : Y \to Y \times_S Y$ is a closed immersion so $Z \to X$ is a closed immersion and therefore we get $U \embed Z \embed X$. We want to deduce that $Z = X$. This is equivalent to the condition that $\I_Z = (0)$. However, $\struct{X} \to \iota_* \struct{Z} \to j_* \struct{U}$ is injective and thus $\struct{X} \to \iota_* \struct{Z}$ is injective meaning that $\I_Z = \ker{(\struct{X} \to \iota_* \struct{Z})} = (0)$. Alternatively, $\I_Z \subset \struct{X} \to j_* \struct{U}$ is zero because $Z \cap U = U$ and this is the ideal sheaf of the closed $Z \cap U$ in $U$.
\end{proof}

\subsection{Triangular Implication of Properties}

Say that $\cP$ is some property of scheme maps. Consider conditions:
\begin{enumerate}
\item All closed immersions (e.g. all isomorphisms) are $\cP$
\item $\cP$ is stable under compositions
\item If $f : X \to Y$ and $f' : X' \to Y'$ over $S$ are $\cP$, then so it $f \times f' : X \times_S X' \to Y \times_S Y'$
\item $\cP$ is stable under base change
\item for $f : X \to Y$ and $g : Y \to Z$ if $g \circ f$ is $\cP$ and $g$ is separated then $f$ is $\cP$.
\item If $f$ is $\cP$ then $f_{\red}$ is $\cP$. 
\end{enumerate}

\begin{example}
finite, affine, finite type, closed immersion all satisfy (a) (b) (d) but open immersion violates (a). 
\end{example}

\begin{thm}
Assume that $\cP$ satisfies (a) and (b) then (c) $\iff$ (d) and when they hold then (d) and (e) hold. 
\end{thm}

\newcommand{\pr}{\mathrm{pr}}

\begin{proof}
For (d) $\implies$ (c) we just factor $X \times_S X' \xrightarrow{f \times \id_{Y'}} Y \times_S X' \xrightarrow{\id_Y \times f'} Y \times_S Y'$ which is $f \times_S f'$ so $f \times_S f'$ satisfies (d) by (b) and (d). Now suppose (c) then consider $X \to S$ then $X \times_S S' \to S \times_S S' \iso S$ is $\cP$ by (c) and (a) for $\id$ and isomorphism and (b). 
\bigskip\\
Now we grant (c) and (d). Consider the diagram,
\begin{center}
\begin{tikzcd}
X \pullback \arrow[rr, bend left, "f"] \arrow[d, "f"] \arrow[r, "\Gamma_f"'] & X \times_Z Y \arrow[d, "f \times \id_Y"] \arrow[r, "\pr_2"'] & Y
\\
Y \arrow[r, "\Delta_g"] & Y \times_Z Y 
\end{tikzcd}
\end{center}
since $f = \pr_2 \circ \Gamma_f$ and $\Gamma_f$ is the base change of $\Delta_g$ which is a closed immersion (because $g$ is separated) and thus is $\cP$ by (a) and (c) and $\pi_2$ is the base change of $f \circ g : X \to Z$ via $g : Y \to Z$ so by (c) it has $\cP$. Thus by (b) we see that $f = \pi_2 \circ \Gamma_f$ is $\cP$.
\bigskip\\
Finally, consider,
\begin{center}
\begin{tikzcd}
X_{\red} \arrow[r, hook] \arrow[d, "f_{\red}"'] & X \arrow[d, "f"]
\\
Y_{\red} \arrow[r, hook, "g"] & Y
\end{tikzcd}
\end{center}
Since $g \circ f_{\red}$ is the composition of $f$ and a closed immersion it is $\cP$ and $g$ is a closed immersion and this separated so by (v) we have $f_{\red}$ satisfies $\cP$.
\end{proof}

\subsection{Quasi-Separatedness}

In EGA ``scheme'' means ``separated scheme'' and ``pre-scheme'' means ``scheme''. A useful weakening of separatedness, discovered in EGA IV is the following.

\begin{defn}
A morphism $f : X \to S$ is \textit{quasi-separated} if $\Delta_f : X \to X \times_S X$ is quasi-compact.
\end{defn}

\begin{rmk}
For $X \to \Spec{A}$ and $U,V$ are affine open then consider,
\begin{center}
\begin{tikzcd}
U \cap V \pullback \arrow[d, hook] \arrow[r] & U \times_A V \arrow[d, hook]
\\
X \arrow[r, "\Delta_f"] & X \times_A X
\end{tikzcd}
\end{center}
and thus by base change $U \cap V \to U \times_A V$ is quasi-compact but $U \times_A V$ is affine so $U \cap V$ is quasi-compact.
\end{rmk}

\begin{rmk}
IF $\{ U_i \}$ is an \textit{affine} open cover of $X$ and $\{ U_i \times_A U_j \}$ is a cover for $X \times_A X$ then $f$ is quasi-separated iff all $U_i \cap U_j$ are quasi-compact.
\bigskip\\
If $\{ S_i \}$ is an open cover of $S$ and $X_i = f^{-1}(S_i)$ then $\Delta_f$ is quasi-compact iff all $X_i \to X_i \times_{S_i} X_i = (X \times_S X) |_{S_i}$ are quasi-compact for $f_i : X_i \to S_i$. Therefore quasi-separatedness is local on the base/target.
\end{rmk}

\begin{lemma}
quasi-separatedness is preserved by base change and composition.
\end{lemma}

\begin{proof}
Let $f : X \to Y$ and $g : Y \to Z$. Then,
\begin{center}
\begin{tikzcd}
X \arrow[rd, "\Delta_f"'] \arrow[r, "\Delta_{g \circ f}"] & X \times_Z X
\\
& X \times_Y X \arrow[u, "j"']
\end{tikzcd}
\end{center}
we say that $j$ is a base change of $\Delta_g$ which is quasi-compact and thus $j$ is quasi-compact and $f$ is quasi-compact so we see that $\Delta_{g \circ f}$ is quasi-compact so $g \circ f$ is quasi-separated. Likewise,
\begin{center}
\begin{tikzcd}
X' \arrow[d, "f'"] \arrow[r] & X \arrow[d, "f"]
\\
S' \arrow[r] & S
\end{tikzcd}
\end{center}
Then $\Delta_{f'}$ is the base change of $\Delta_f$ which is quasi-compact and thus $\Delta_f$ is quasi-compact so $f'$ is quasi-separated.
\end{proof}

\begin{prop}
If $f : X \to S$ for locally noetherian $X$ is quasi-separated.
\end{prop}

\begin{proof}
WLGO let $S = \Spec{A}$ so we want $U \cap V$ to be quasi-compact for affine opens $U, V \subset X$. But $U, V$ are noetherian so $U \cap V$  is open in $U \cap V$ and thus quasi-compact.
\end{proof}

\begin{cor}
Consider the diagram,
\begin{center}
\begin{tikzcd}
X' \arrow[r, "f'"] \arrow[d] & \Spec{R'} \arrow[d]
\\
X \arrow[r, "f"] & \Spec{R}
\end{tikzcd}
\end{center}
If $f$ is locally of finite type and $R$ is noetherian then $f'$ is quasi-separated.
\end{cor}

\begin{proof}
Since $f$ is locally finite type then $X$ is locally noetherian so by above $f$ is quasi-separated to by base change $f'$ is quasi-separated.
\end{proof}

\begin{rmk}
However, $R'$ could be some enormous non-noetherian ring.
\end{rmk}

\begin{rmk}
Consider,
\begin{center}
\begin{tikzcd}[row sep = small, column sep = small]
X \arrow[rr, "f"] \arrow[rd] & & \Spec{A} \arrow[dl]
\\
& \Spec{\Z}
\end{tikzcd}
\end{center}
Then $f$ is quasi-separated iff $X \to \Spec{\Z}$ is quasi-separated so we just say $X$ is quasi-separated in the case that it is quasi-separated over an affine.
\end{rmk}
\noindent
We want a notion of ``relative compactness'' not quasi-compactness because this doesn't give the correct properties for non-Hausdorff topologies. For locally compact Hausdorff spaces a continuous map $f :  X \to Y$ is \textit{proper} if $f^{-1}(C)$ is compact for each compact $C \subset Y$. We want an analogue for schemes.

\section{Nov. 15 Projectivity and Properness}

\subsection{Projectivity}

\begin{defn}
Let $S$ be a scheme. Let $\A^n_S = \A_\Z^n \times_\Z S$ and $\P^n_S = \P^n_\Z \times_\Z S$ (this is a gluing of $n+1$-copies of $\A^n_S$ in ``usual'' way). Over open affine $\Spec{R} \subset S$, this recovers $\A^n_R$ and $\P^n_R$.
\end{defn}

\begin{defn}
Say $f : X \to S$ is \textit{projective} if there exists.
\begin{center}
\begin{tikzcd}
X \arrow[rd, "f"] \arrow[r, "j", hook] & \P^n_S \arrow[d]
\\
& S
\end{tikzcd}
\end{center}
where $j$ is a closed immersion. If such diagram exists with $j$ a quasi-compact immersion then $X \to S$ is quasi-projective.
\end{defn}

\begin{rmk}
If $j$ is quasi-compact notice that quasi-protectiveness is equivalent to the existence of $j$ that is an immersion.
\end{rmk}

\begin{rmk}
In EGA, there is a more general version of projectivity but also fails to be local on the base.
\end{rmk}

\begin{prop}
The following hold:
\begin{enumerate}
\item quasi-projective $\implies$ separated and finite type
\item projective and quasi-projective are preserved under base change
\item closed immersions are projective
\item projective and quasi-projective are stable under composition. 
\end{enumerate}
\end{prop}

\begin{proof}
(a) through (c) are evident. Let's look at (d). Consider $f : X \to Y$ and $g : Y \to Z$. Then consider,
\begin{center}
\begin{tikzcd}
X \arrow[rd] \arrow[r, hook] & \P^N_Y \pullback \arrow[d] \arrow[r, hook] & \P^N_Y \times_Y \P^M_Z \arrow[d] \arrow[r, equals] & \P^N_Z \times_Z \P^M_Z  Z \arrow[r, "\text{segre}", hook] & \P^{NM + N + M}_Z \arrow[ddll]
\\
& Y \arrow[rd] \arrow[r, hook] & \P^M_Z \arrow[d]
\\
& & Z 
\end{tikzcd}
\end{center}
where heuristically the Segre embedding is given by $([x_i], [y_i]) \mapsto [x_i y_j]$.
\end{proof}

\begin{rmk}
Intersection theory gives a lot of geometric information on projective schemes $X \embed \P^n$ using ``hyperplane slices''. 
\end{rmk}

\subsection{Elimination Theory}

\begin{rmk}
Elimination theory is the study of taking a system of multivariable polynomial equations and producing polynomial conditions that decide if they have a common solution.
\end{rmk}

\begin{thm}[Fundamental Thm of Elimination Theory]
Given homogeneous $f_1, \dots, f_m \in k[\underline{a}][X_0, \dots, X_n]$ there exists $g_1, \dots, g_r \in k[\underline{a}]$ such that for $\underline{a}_0 \in k^N$, there exists nonzero $\bar{k}$-solution to $\{ f_i(\underline{a}_0, \underline{X}) \}$ iff all $g_j(\underline{a}_0) = 0$.
\end{thm}

\begin{rmk}
We reinterpret this fact as,
\[ \Proj{k[\underline{a}][\underline{X}]/(f_1, \dots, f_n)} \to \Spec{k[\underline{a}]} \]
has Zariski closed image. Equivalently, because the left is a closed subscheme of $\P^n_{k[\underline{a}]}$ we require that the map,
\[ \P^n_{k[\underline{a}]} \to \Spec{k[\underline{a}]} \]
is closed. This means that $\P^n_k \to \Spec{k}$ is closed after any base-change $S \to \Spec{k}$ for affine $S$ of finite type (because every finite type $k$-algebra is a quotient of a polynomial ring).
\end{rmk}

\begin{defn}
A morphism $f : X \to Y$ is \textit{universally closed} if for any $Z \to Y$, the base change $f' : X_Z \to Z$ is closed (topologically).
\end{defn}

\begin{defn}
A morphism $f : X \to Y$ is \textit{proper} if it is separated, finite type, and universally closed.
\end{defn}

\begin{rmk}
We will see that $\P^n_\Z \to \Spec{\Z}$ is proper. This is much stronger than standard elimination theory. In particular we recover that $\P^n_R \to \Spec{R}$ is closed giving ``elimination theory over any ring''. 
\end{rmk}

\subsubsection{Merits of Properness over Projectivity}

\begin{enumerate}
\item properness is insensitive to nilpotents, but not projectivity (properness is more topological and is simpler for deformation theory).
\item properness is local on the base (and indeed stability under base change)
\item properness has a functorial characterization
\item there are deep characterizations of \textit{other} properties of maps via properness [EGA, IV$_4$, 18.12].
\begin{enumerate}
\item finite $\iff$ proper and quasi-finite
\item integral $\iff$ affine and universally closed (Exercise 35, Ch.5 in [AM])
\item closed immersion $\iff$ proper monomorphism
\end{enumerate}
\end{enumerate}

\begin{rmk}
From the definitions we see that,
\begin{enumerate}
\item closed immersions are proper
\item composition of proper is proper
\item proper is stable under base change
\end{enumerate}
therefore we also see that (d) - (f) from last time also hold. In particular, given,
\begin{center}
\begin{tikzcd}
X \arrow[rr, "f"] \arrow[dr] & & Y \arrow[dl]
\\
& Z 
\end{tikzcd}
\end{center}
with $Y \to Z$ separated and finite type. Then $X \to Y$ is proper when $X \to Z$ is proper (like image of compact is compact). For example, if $f$ is quasi-projective,
\begin{center}
\begin{tikzcd}
X \arrow[rd, "f"] \arrow[r, "j", hook] & \P^n_S \arrow[d]
\\
& S
\end{tikzcd}
\end{center}
then $f$ is proper iff $f$ proper iff $j$ is a closed immersion iff $f$ is projective. 
\end{rmk}

\section{Valuative Criterion}

(Inspired by the notion of ``completeness'' in Weil's foundations for abstract varieties). 

\begin{thm}[Valuative Criterion]
Let $f : X \to Y$ be a quasi-separated finite type morphism. Consider valuation rings $V$ with $K = \Frac{V}$ and a diagram,
\begin{center}
\begin{tikzcd}
\Spec{K} \arrow[d] \arrow[r, "x_K"]  & X \arrow[d, "f"]
\\
\Spec{V} \arrow[ru, dashed] \arrow[r, "y"] & Y
\end{tikzcd}
\end{center}
Then,
\begin{enumerate}
\item $f$ is separated iff all such diagrams can be filled in at most one way (like uniqueness of limits in Hausdorff spaces)
\item $f$ is proper iff all such diagrams can be filled in a unique way (like existence of limits in a compact space)
\end{enumerate}
\end{thm}

\begin{rmk}
Consider,
\begin{enumerate}
\item $V = k[[t]]$ then $\Spec{V}$ is the AG version of the unit disc then $K = V[t^{-1}]$ and $\Spec{K}$ is the AG version of the punctured unit disk. Thus the valuative criterion asks about extending sections over $0$.
\item For noetherian $Y$ it suffices to use DVRs.
\item $f$ is separated $\iff \Delta_X$ is proper so it allows (a) to reduced to (b). 
\end{enumerate}
\end{rmk}

\subsection{Killer App}

Show that $\P^n_\Z \to \Spec{\Z}$ is proper (we know already that it is separated and finite type). 
\bigskip\\
For any DVR $V$ with $K = \Frac{V}$ we want $\P^n_\Z(V) \to \P^n_\Z(K)$ is bijective. For a local ring $A$, maps $\Spec{A} \to \P^n_\Z$ must land in one $\A^{n}_\Z$ because the only open containing the closed point is all of $\Spec{A}$ so if $\m$ lands in $\A^n_\Z$ then all of $\Spec{A}$ does (the preimage of $\A^n_\Z$ is an open containing $\m$). This yields,
\[ \P^n_\Z(A) = \frac{A^{n+1} \setminus \m^{n+1}}{A^\times} \]
we can interpret this as saying,
\begin{center}
\begin{tikzcd}
& \A^{n+1}_{\Z} \setminus \{ 0 \} \arrow[d]
\\
\Spec{A} \arrow[ru, dashed, "\exists"] \arrow[r] & \P^n_\Z
\end{tikzcd}
\end{center}
But by clearing denominators,
\[ \frac{V^{n+1} \setminus (\varpi)^{n+1}}{V^\times} \iso \frac{K^{n+1} \setminus \{ 0 \}}{K^\times} \]

\section{Nov. 17 Hilbert Functors and Quasi-Coherent Sheaves}

To get a handle on properness i n general settings, we use Chow's lemma.

\begin{lemma}
For $f : X \to S$ with $S$ noetherian and $f$ separated of finite type. Then there exists $\pi : X' \onto X$ projective surjective birational (there are dense opens $U \subset X$ and $U' \subset X'$ such that $\pi : U' \iso U$ is an isomorphism) such that $X' \to S$ is quasi-projective.
\end{lemma}

\begin{rmk}
We see that $X \to S$ is proper iff $X' \to S$ is surjective.
\end{rmk}

\begin{rmk}
For $S = \Spec{\CC}$. We can use Chow's lemma to show that a separated finite type $X \to \Spec{\CC}$ is proper iff $X(\C)$ is compact for the complex topology. We use Chow's lemma to reduce to showing that an immersion of finite type $\CC$-schemes is closed iff its image on $\CC$-points is closed for complex topology.
\end{rmk}


\subsection{Hilbert Functors}

\begin{rmk}
Grothendieck named these for Hilbert because the moduli space is stratified by Hilbert polynomial. Chow defined the ``Chow variety`` which encoded ``Chow forms`` in projective space that, up to some numerical invariants, classified certain subvarieties. However this was not a functorial construction and only worked for certain subvarieties not all subschemes. Grothendieck's Hilbert scheme actually represents the correct functor.
\end{rmk}

\newcommand{\uHilb}{\underline{\Hilb}}
\newcommand{\Set}{\mathbf{Set}}

\begin{defn}
Let $f : X \to S$ be proper with $S$ noetherian. Let $\uHilb_{X/S} : \Sch_S \to \Set$ be the functor,
\[ \uHilb_{X/S}(T) = \{ Z \embed X_T \mid \I_Z \subset \struct{X_I} \text{ is locally finitely generated and } Z \to T \text{ is flat} \} \]
and for $T' \to T$ we define $\uHilb_{X/S}(T) \to \uHilb_{X/S}(T')$ via $(Z \embed X_T) \mapsto (Z \times_T T' \embed X_{T'})$.
\end{defn}

\begin{rmk}
The locally finite generated assumption is redundant for $T$ locally noetherian and the flatness is redundant for $T = \Spec{k}$.
\end{rmk}

\begin{example}
For $T = S$ then,
\[ \uHilb_{X/S}(S) = \{ Z \embed X \mid \I_Z \subset \struct{X} \text{ locally finitely generated and } Z \to S \text{ flat } \} \]
so for $S = \Spec{k}$ we see that,
\[ \uHilb_{X/S}(S) = \{ Z \embed X \} \]
\end{example}

\begin{defn}
If $\uHilb_{X/S}$ is representable then there is a representing scheme $\Hilb_{X/S}$ called the \textit{Hilbert scheme of } $f : X \to S$ equipped with the ``universal closed subscheme'' of $X$,
\begin{center}
\begin{tikzcd}
\mathcal{Z} \arrow[rd] \arrow[r] & \Hilb_{X/S} \times_S X \arrow[d]
\\
& \Hilb_{X/S}
\end{tikzcd}
\end{center}
meaning for any $Z \embed X_T$ there is a unique map $T \to \Hilb_{X/S}$ with $\mathcal{Z} \times_{\Hilb_{X/S}} T = Z$ inside $X_T$.
\end{defn}

\begin{rmk}
For $s : \Spec{k} \to S$ the fibers of $\mathcal{Z}$ over $\Hilb_{X/S}(k)$ \textit{are} the closed subschemes of $X_s$.
\end{rmk}

\begin{thm}[G]
If $f$ is projective then $\Hilb_{X/S}$ exists and is a countable disjoint union of quasi-projective $S$-schemes.
\end{thm}

\begin{proof}
For a DVR $V$ and $K = \Frac{V}$. Consider $\varphi : \Spec{V} \to S$ we want 
\[ \Hilb_{X/S}(V) \to \Hilb_{X/S}(K) \]
bijective. This means for,
\[ X' = X \times_{\varphi} \Spec{V} \]
we want,
\[ \{ V\text{-flat closed } Z \embed X' \} \to \{ \text{closed } \wt{Z} \embed X'_X \} \]
sending $Z \mapsto Z_K$ to be bijective.
\end{proof}

\begin{lemma}
For any finite-type $V$-scheme $X'$,
\[ \{ V\text{-flat closed } Z \embed X' \} \to \{ \text{closed } \wt{Z} \embed X'_X \} \]
where $X = X'_K$.
\end{lemma}

\begin{proof}
By uniqueness aspect it suffices to prove this statement locally because then we get unique gluing. Thus let $X' = \Spec{A}$ be affine. Then we consider,
\[ \{ I \subset A \text{ with } A / I \text{ torsion-free } \} \to \{ \wt{I} \subset A_K \} \]
We define the inverse via $\wt{I} \mapsto I = A \cap \wt{I}$ and $A/I \embed A_K/\wt{I}$ so $A/I$ is torsion-free. Then we check that these are inverse constructions.
\end{proof}

\subsection{Quasi-Coherent Sheaves}

For a scheme $X$, seek special class of $\struct{X}$-modules $\F$ so that for affine opens $\Spec{A} \subset X$ then sheaf $\F|_{\Spec{A}}$ ``is an $A$-module''. 

\begin{rmk}
When building the structure sheaf on $\Spec{A}$ only the $A$-module structure was used in the construction. Thus given an $A$-module $M$ and a basic affine $U = D(f)$ then we can define $M_U = S_U^{-1} M$ where,
\[ S_U = \{ a \in A \mid \forall u \in U : a(u) \neq 0 \} \]
and $U \mapsto M_U$ is a $B$-sheaf.
\end{rmk}

\begin{defn}
Then $\wt{M}$ is the $\struct{A}$-module arising from $M$ so,
\[ \Gamma(D(f), \wt{M}) = M_f \]
as $A_f$-modules.
\end{defn}

\begin{prop}
Let $X = \Spec{A}$. For any $\struct{A}$-module $\G$ and $A$-module $M$ there is a natural isomorphism,
\[ \Hom{\struct{A}}{\wt{M}}{\G} = \Hom{A}{M}{\G(X)} \]
\end{prop}

\begin{proof}
Consider,
\begin{center}
\begin{tikzcd}
M \arrow[d] \arrow[r] & \Gamma(X, \G) \arrow[d]
\\
M_f \arrow[r, dashed] & \Gamma(D(f), \G) 
\end{tikzcd}
\end{center} 
and $\Gamma(D(f), \G)$ is an $A_f$-module so $M_f \to \Gamma(D(f), \G)$ is uniquely determined by $M \to \Gamma(X, \G)$.
\end{proof}

\begin{defn}
Let $f : X \to Y$ be a map of ringed spaces. We have $f_* : \Mod{\struct{X}} \to \Mod{\struct{Y}}$ given by $\F \mapsto f_* \F$ and $f_* \F$ is a $f_* \struct{X}$-module so given a $\struct{Y}$-module structure via $f^\# : \struct{Y} \to f_* \struct{X}$.
\bigskip\\
Then we define $f^* : \Mod{\struct{Y}} \to \Mod{\struct{X}}$ via,
\[ \G \mapsto f^* \G = \struct{X}  f^{-1} \otimes_{f^{-1} \struct{Y}} \G  \]
using the adjoint map $f^{-1} \struct{Y} \to \struct{X}$ which is a sheaf of rings. Then tensor product gives $f^* \G$ a $\struct{X}$-module structure.
\end{defn}

\begin{prop}
We have adjointness,
\[ \Hom{\struct{X}}{f^* \G}{\F} = \Hom{\struct{Y}}{\G}{f_* \F} \]
\end{prop}

\begin{proof}
This is just,
\[
\Hom{\struct{X}}{f^*\G}{\F} = \Hom{f^{-1} \struct{Y}}{f^{-1} \G}{\F} = \Hom{\struct{Y}}{\G}{f_* \F} \]
where we used the adjointness $\Hom{}{f^{-1} \G}{\F} = \Hom{}{\G}{f_* \F}$ and the fact that it preserves the module structure.
\end{proof}

\begin{thm}
For $f : X \to Y$ with $X = \Spec{B}$ and $Y = \Spec{A}$.
\begin{enumerate}
\item $M \mapsto \wt{M}$ for $A$-modules is fully faithfully and exact
\item $\wt{M_1} \ot_{\struct{Y}} \wt{M_2} \iso \wt{(M_1 \ot_A M_2)}$
\item $\wt{\varinjlim M_i} = \varinjlim \wt{M_i}$ (e.g. for direct sums)
\item $f_*(\wt{N}) = \wt{N_A}$ with $N_A$ being $N$ viewed as an $A$-module
\item $f^*(\wt{M}) = \wt{B \otimes_A M}$
\item $\wt{\Hom{A}{M'}{M}} = \shHom{\struct{Y}}{\wt{M'}}{\wt{M}}$ for finitely presented $M'$.
\end{enumerate}
\end{thm}

\begin{defn}
We say that $\F$ is quasi-coherent on a scheme $X$ if $\F|_U = \wt{M}$ on affine opens.
\end{defn}

\section{Nov. 19}

\begin{rmk}
\[ \Hilb^n_{X/k}(T) = \{ Z \embed X_T \mid \pi : Z \to T \text{ finite flat of degree } n \} \]
with $T$ locally noetherian is ``compactified configuration space''. For $X$ a smooth projective curve then $\Hilb^n_{X/k} = X^n / \Sigma_n$.
\end{rmk}

\subsection{Localization Criterion}

\begin{prop}
Let $X = \Spec{A}$ and $\F$ is an $\struct{X}$-module. For any $a \in A$ the restriction map $\Gamma(X, \F) \to \Gamma(D(a), \F)$ is linear over $A \to A_f$ and thus factors,
\[ \Gamma(X, \F) \to \Gamma(X, \F)_a \xrightarrow{\theta_a} \Gamma(D(a), \F) \]
Then $\F$ is quasi-coherent iff each $\theta_a$ is an isomorphism.
\end{prop}

\begin{proof}
Let $M = \Gamma(X, \F)$ so the map $\id_M : M \to \Gamma(X, \F)$ induces $\varphi : \wt{M} \to \F$. If $\F$ is quasi-coherent then $\varphi$ is an isomorphism. If $\varphi$ is an isomorphism then by definition $\F$ is quasi-coherent. By $\varphi$ is an isomorphism iff it is locally an isomorphism on $D(a)$ but this is the map $M_a \to \Gamma(X, \F)_a \xrightarrow{\theta_a} \Gamma(D(a), \F)$ and $M = \Gamma(X, \F)$ so the first map is an isomorphism and thus $\varphi$ is an isomorphism over $D(a)$ iff $\theta_a$ is an isomorphism.
\end{proof}

\begin{defn}
A $\struct{X}$-module $\F$ is \textit{finite type} if there exists an open cover $\{ U_i \}$ of $X$ and $\struct{U_i}^{\oplus n_i} \onto \F|_{U_i}$ for all $i$.
\end{defn}

\begin{rmk}
For open $U \embed X$ there is a crazy ideal sheaf $j_!(\struct{U}) \subset \struct{X}$ which is almost never quasi-coherent. But $\struct{X} / j_!( \struct{U} )$ is finite type and not (usually) quasi-coherent. 
\end{rmk}

\begin{prop}
For $X = \Spec{A}$ and $\F = \wt{M}$ quasi-coherent then $\F$ is finite type iff $M$ is finitely generated.
\end{prop}

\begin{proof}
If $M$ is finitely generated then $A^{\oplus n} \onto M$ gives $\struct{X}^{\oplus n} \onto \F$ and thus it is globally finite type and thus finite type.
\bigskip\\
Conversely, if $\F$ is finite then we can refine the open cover by basic affines which has a finite subcover since $X$ is quasi-compact. Thus we get $D(a_i)$ with,
\[ \struct{A_{a_i}}^{\oplus {n_i}} \onto \F|_{D(a_i)} = \wt{M_{a_i}} \]
Therefore, we get a map $A_a^{\oplus n} \onto M_{a_i}$ so $M_{a_i}$ is locally finitely generated and thus finitely generated. 
\end{proof}

\begin{defn}
For $X$ locally noetherian, an $\struct{X}$-module $\F$ is \textit{coherent} if $\F$ is quasi-coherent and finite type.
\end{defn}

\begin{rmk}
In the non-noetherian case, this is the wrong condition. Warning [H,p.111] does this without requiring locally noetherian: BAD.
\end{rmk}

\begin{thm}
\item quasi-coherence is stable under $\otimes, \varinjlim$ and affine pushforward
\item if $\varphi : \F \to \G$ is a $\struct{X}$-linear map of quasi-coherent sheaves then $\ker{\varphi}$ and $\coker{\varphi}$ and $\im{\varphi}$ are quasi-coherent.
\item If $X$ is locally noetherian then (b) also holds for coherent sheaves as does (a) for $\otimes$ and \textit{finite} pushforward. If $\F$ and $\G$ are coherent then $\shHom{\struct{X}}{\F}{\G}$ is coherent. 
\end{thm}

\begin{proof}
These are all of local nature (we assumed the maps are affine). Then we apply properties of $\wt{(-)}$. 
\end{proof}

\begin{example}
For closed immersion $\iota : Y \embed X$ with $X$ a scheme then $\I_Y = \ker{(\struct{X} \onto \iota_* \struct{Y})}$ is quasi-coherent ideal sheaf. In reverse, if $\I \subset \struct{X}$ is quasi-coherent, then $\I = \I_Y$ for a unique closed subscheme $Y \embed X$. 
\end{example}

\begin{rmk}
It is very fruitful to associate an ideal sheaf $\I_{\{x\}} \subset \struct{X}$ to a closed point $x \in X$ on a smooth curve.
\end{rmk}

\begin{prop}
Let $X$ be locally noetherian $\F, \G$ coherent on $X$ then,
\begin{enumerate}
\item Given $\varphi : \F \to \G$ with $\varphi_x : \F_x \iso \G_x$ then $\varphi|_U : \F|_U \to \G|_U$ is an isomorphism on some open neighborhood
\item Any $\stalk{X}{x}$-linear map $\varphi_x  : \F_x \to \G_x$ comes from $\varphi_U : \F|_U \to \G|_U$ for \textit{some} open $U$ and has a uniquely determined germ.
\end{enumerate}
These together mean that the natural map,
\[ \shHom{\struct{X}}{\F}{\G}_x \to \shHom{\stalk{X}{x}}{\F_x}{\G_x} \]
is an isomorphism.
\end{prop}

\begin{proof}
We see that $(\ker{\varphi})_x = \ker{\varphi_x} = 0$ and $(\coker{\varphi})_x = \coker{\varphi_x} = 0$ and $\coker{\varphi}$ and  $\ker{\varphi}$ are coherent. Now, if $\K$ is finite type and $\K_x = 0$ then $\K|_U = 0$ for some neighborhood $U$ (because we just need to kill all the generators. Thus $\ker{\varphi|_U} = \coker{\varphi|_U} = 0$ so $\varphi_U$ is an isomorphism.
\bigskip\\
We use a local finite presentation,
\begin{center}
\begin{tikzcd}
\struct{U}^{\oplus s} \arrow[r] & \struct{U}^{\oplus r} \arrow[r] & \F|_U \arrow[r] & 0
\end{tikzcd}
\end{center}
near $x$ to reduce to the case $\F = \struct{U}^{\oplus m}$. 
\end{proof}

\begin{thm}
For $f : X \to Y$ a morphism of schemes,
\begin{enumerate}
\item $f^*$ of a quasi-coherent is quasi-coherent and of a coherent is coherent (in loc. noetherian case)
\item if $f$ is qcqs then $f_*$ of a quasi-coherent is quasi-coherent
\end{enumerate}
\end{thm}

\begin{proof}
For (a) WLOG $X, Y$ are affine. For (b) use an exact sequence trick to reduce to affine case.
\end{proof}

\begin{prop}
Consider a short exact sequence of $\struct{X}$-modules,
\begin{center}
\begin{tikzcd}
0 \arrow[r] & \F' \arrow[r] & \F \arrow[r] & \F'' \arrow[r] & 0
\end{tikzcd}
\end{center}
if $\F'$ and $\F''$ are quasi-coherent then $\F$ is coherent.
\end{prop}

\section{Nov. 29}

\subsection{Quasi-Coherent Sheaves on Proj}

Just like how Proj is not a functor for general maps of graded rings we will see that we lose information when we pass from graded modules to quasi-coherent sheaves.
\bigskip\\
Fix some notation. Let,
\[ S = \bigoplus_{n \ge 0} S_n \]
an $\N$-graded ring, $\Proj{S}$ is covered by affine opens,
\[ D_+(f) = \Spec{S_{(f)}} \]
for homogeneous elements $f \in S_+$. 

\begin{rmk}
If $\{ f_i \} \subset S_+$ generate,
\[ \bigoplus_{n \ge n_0} S_n \]
then $\{ D_+(f_i) \}$ cover $\Proj{S}$ since we can rewrite any fraction with a large enough denominator. 
\end{rmk}

Let $M$ be a $\Z$-graded $S$-module,
\[ M = \bigoplus_{n \in \Z} M_n \]
meaning $S_a \times M_b \to M_{a+b}$. 

\begin{example}
Consider the following examples,
\begin{enumerate}
\item let $M = I \subset S$ a homogeneous ideal
\item $M = S(r)$ is the $S$-module with $M_d = S_{r+d}$ so $S(r)_0 = S_r$ for any $r \in \Z$
\end{enumerate}
\end{example}

\begin{example}
Let $S = A[x_0, \dots, x_n]$ then,
\begin{center}
\begin{tikzcd}
0 \arrow[r] & S(-1) \arrow[r, "x_0"] & S \arrow[r] & S/(x_0) \arrow[r] & 0
\end{tikzcd}
\end{center}
is an exact sequence of graded rings. Notice we needed to introduce the twist by $-1$ to make the map $S(-1) \to S$ a graded map since $1 \mapsto x_0$ so $1$ needs to have degree $1$ in $S(-1)$. 
\end{example}

\begin{defn}
Define $\wt{M}$ on $\Proj{S}$ by $\wt{M}|_{D_+(f)} = \wt{M_{(f)}}$ on $\Spec{S_{(f)}} = D_+(f)$ where $M_{(f)}$ is the degree zero submodule of the $\Z$-graded module $M_f$. The gluing are given by,
\[ (M_{(f)})_{\frac{g^{\deg{f}}}{f^{\deg{f}}}} = M_{(fg)} = (M_{(g)})_{\frac{f^{\deg{g}}}{g^{\deg{f}}}} \]
\end{defn}

\begin{rmk}
This process loses low degree information from $M$. Indeed, we can write,
\[ \frac{m}{f^r} = \frac{m f^n}{f^{r + n}} \]
then $m f^n$ can be made to have positive degrees. Therefore, if we get,
\[ M' = \bigoplus_{n \ge n_0} M_n \]
then $M' \embed M$ yields $\wt{M'} \iso \wt{M}$. Therefore, if $M_n = 0$ for $n \gg 0$ then $\wt{M} = 0$. 
\end{rmk}

\begin{rmk}
We define, $M \ot_S N = (M \ot_\Z N) / (s m \ot n - m \ot sn)$ for $m,n,s$ all homogeneous. This solves the problem of $M \ot_S N$ not having an obvious grading because the graded pieces of $M,N$ are not $S$-modules but rather $S_n$-modules which is not what the tensor-product is over. 
\end{rmk}

\begin{thm}
The following hold in the setting of associated sheaves on $\Proj{S}$,
\begin{enumerate}
\item the functor $M \mapsto \wt{M}$ is exact and preserves $\varinjlim$
\item for $\p \in \Proj{S}$ the stalk is $\wt{M}_\p = M_{(\p)}$ the degree zero part of the localization of $M$ at the homogeneous elements of $S_+ \setminus \p$.
\item $\wt{M}$ is quasi-coherent 
\item the canonical map $\wt{M} \ot_{\struct{X}} \wt{N} \to \wt{M \ot_S N}$ built from $M_{(f)} \ot_{S_{(f)}} N_{(f)} \to (M \ot_S N)_{(f)}$ (which is an isomorphism if $f \in S_1$ but otherwise not necessarily surjective because it is not clear how to split up fractions) is an isomorphism when $S$ is generated by $S_1$ over $S_0$
\end{enumerate}
\end{thm}

\begin{proof}
Compute over $D_+(f)$. 
\end{proof}

\begin{cor}
If $S$ is generated by $S_1$ over $S_0$ then for homogeneous ideal $I \subset S$,
\[ \wt{IM} = \wt{I} \cdot \wt{M} \]
considering the ideal sheaf $\wt{I} \subset \wt{S} = \struct{X}$. 
\end{cor}

\begin{proof}
$I \ot_S M \onto IM$ and under the hypothesis we see that $\wt{IM}$ is the image of $\wt{I} \ot_{\struct{X}} \wt{M} \to \wt{M}$. 
\end{proof}

\begin{rmk}
For the interaction of $\wt{-}$ with Hom $f_*$ and $f^*$ see section 1-2 of ``further properties of $\wt{M}$''. 
\end{rmk}

\begin{example}
Consider the exact sequence,
\begin{center}
\begin{tikzcd}
0 \arrow[r] & S(-1) \arrow[r, "x_0"] & S \arrow[r] & S/(x_0) \arrow[r] & 0
\end{tikzcd}
\end{center}
for $S = A[x_0, \dots, x_n]$. This yields the sequence,
\begin{center}
\begin{tikzcd}
0 \arrow[r] & \wt{S(-1)} \arrow[r] & \struct{\P^n_A} \arrow[r] j_* (\struct{\P^{n-1}}) \arrow[r] & 0
\end{tikzcd}
\end{center}
identifies $\wt{S(-1)}$ with $\I_{H}$ as $\struct{\P^n_A}$-modules where $H = V(x_0)$ is the hyperplane cut out by $x_0$. Now applying global sections,
\begin{center}
\begin{tikzcd}
0 \arrow[r] & \Gamma(\wt{S(-1)}) \arrow[r] & \Gamma(\struct{\P^n_A}) \arrow[r] & \Gamma(\struct{\P^{n-1}}) 
\end{tikzcd}
\end{center}
but $\Gamma(\P^n_A, \struct{\P^n_A}) = A$ and thus the second map is $\id : A \to A$ so $\Gamma(\wt{S(-1)}) = \ker{\id} = 0$. Thus $\wt{S(-1)}$ has no global sections. But,
\[ A X_0 \oplus \cdots \oplus A X_n = S(1)_0 \iso \Gamma(\wt{S(1)}) \]
\end{example}

\begin{rmk}
Usually, $M_0 \to \Gamma(\wt{M})$ is not an isomorphism. 
\end{rmk}

\begin{example}
When $S$ is not generated by $S_1$ over $S_0$, weird things can happen: $S = k[X_1, X_2, X_3]$ where $\deg{x_0} = i$. Let $I = (X_1)$. Then we get an ideal sheaf $\wt{I}$ on $\Proj{S} = X$. However, $\wt{I}$ is not locally principal. Look at $D_+(f)$ for $f = X_2$. Then,
\[ D_+(f) = \Spec{S_{(X_2)}} \iso k[u,v,w]/(uw - v^2) \]
where,
\begin{align*}
u & \mapsto \frac{X_1^2}{X_2}
\\
v & \mapsto \frac{X_1 X_3}{X_2^2}
\\
w & \mapsto \frac{X_3^2}{X_2^3}
\end{align*}
Then we see that $\Gamma(D_+(f), \wt{I}) = I_{(f)} = (u, v)$.
\end{example}

\begin{rmk}
Some questions,
\begin{enumerate}
\item If $S$ is noetherian, is $\Proj{S}$ noetherian?
\item If $S$ is noetherian and $M$ is finite, is $\wt{M}$ coherent?
\end{enumerate}
Notice that $S_+$ has a \textit{finite} generating set $\{f_1, \dots, f_r \}$ of homogeneous elements ($S_+$ is finitely generated by noetherian property and then each generator has a finite set of generators). For homogeneous $f \in S_+$ and $\deg{f} = d > 0$, is $S_{(f)}$ noetherian? Notice this is not immediate because noetherianness is not preserved under taking subrings (every domain lies in a field). However, let $S^{(d)} = \bigoplus_{d \divides n} S_n \subset S$ this is a $S_0$-subalgebra. Then,
\[ S_{(f)} \iso S^{(d)}/(f-1) \]
via $\frac{h}{f^m} \mapsto h$ in $S_{dm}$. Therefore, it is enough to show that $S^{(d)}$ is noetherian for all $d \ge 0$. Notice that,
\[ S_0 \iso S / S_+ \]
so $S_0$ is noetherian. If $d_i = \deg{f_i} > 0$ then $S_0[X_1, \dots, X_r] \onto S$ for $\deg{X_i} = d_i$. Therefore, $S$ is finitely generated as a $S_0$-algebra. Can leverage properties of $S_0[X_1, \dots, X_r]$ to deduce affirmative answers to all these questions. The key part is that for $d \gg 0$ we have $S^{(d)}$ generated by $S_d$ over $S_0$ and also $M^{(d)}$ is $S^{(d)}$-finite if $M$ is $S$-finite and $M^{(d)} = \wt{M}$. Then we see that,
\[ \Proj{S} \iso \Proj{S^{(d)}} \embed \P^{n-1}_{S_0} \]
\end{rmk}

\begin{prop}
If $S$ is generated by $S_1$ over $S_0$ and $M$ is $S$-finite,
\[ \wt{M} = 0 \iff M_n = 0 \text{ for all } n \gg 0 \]
\end{prop}

\begin{rmk}
Is $S_1$-generating hypothesis necessary. It turns out NO [EGA II, 2.7.3(ii)] removes it when $S_+$ is finitely generated over $S_0$. EXCEPT let,
\[ S = k[X^2] \]
and $M = S(1)$ then it is zero in even degrees so $\wt{M} = 0$ because $S$ is supported in even degrees. There is an error in EGA where they cite a result with these hypotheses but forget them.
\end{rmk}

\section{Dec. 1 The Serre Twist, $I$}

\begin{example}
$S = A[X_0, \dots, X_n]$ then,
\[ S(1)_0 = A X_0 \oplus \cdots \oplus A X_n \iso \Gamma(\P^n_A, \wt{S(1)}) \]
Each,
\[ \Gamma(\P^n_A, \wt{S}(1)) \to \Gamma(D_+(X_i), \wt{S(1)}) \to \Gamma(D_+(X_0 \cdots X_n), \wt{S(1)}) \subset A[X_0, \dots, X_n]_{(X_0 \cdots X_n)} \]
This equals,
\[ S(1)_{(x_i)} = S_{(x_i)} X_i = A[\tfrac{X_0}{X_i}, \dots, \tfrac{X_n}{X_i}] X_i \]
Then,
\[ \Gamma(\P^n_A, \wt{S(1)}) = \bigcap_{i} A[\tfrac{X_\bullet}{X_i}] X_i = A X_0 \oplus \cdots \oplus A X_n \]
\end{example}

For $S$ an $\N$-graded ring $X = \Proj{S}$ for $Z$-graded modules $M, N$ have,
\[ M_{(f)} \ot_{S_{(f)}} N_{(f)} = (M \ot_S N)_{(f)} \]
yields $\wt{M} \ot \wt{N} \to \wt{M \ot S}$.
\bigskip\\
For $T$ a $\Z$-graded $S$-algebra, $\wt{T}$ is a $\struct{X}$-module with maps $T \ot T \to T$ giving $\wt{T} \ot \wt{T} \to \wt{T \ot T} \to \wt{T}$.

\begin{defn}
For $n \in \Z$ then $\struct{X}(n) = \wt{S(n)}$ (depending on $S$ not just $X$).
\end{defn}

\begin{proof}
For $\struct{X}$-module $\F$, define the Serre twists $\F(n) = \F \ot_{\struct{X}} \struct{X}(n)$. 
\end{proof}

\begin{rmk}
Consider $\F \mapsto \F(n)$ as sheaf version of $M \mapsto M(n)$ and $\Gamma(X, \F(n))$ is the sheaf version of $M \mapsto M_n$.
\end{rmk}

\begin{thm}
Assume $S$ is generated by $S_1$ over $S_0$. Let $M$ be a $\Z$-graded module. Then,
\begin{enumerate}
\item $\wt{M}(n) \iso \wt{M(n)}$
\item $\struct{X}(n) \ot \struct{X}(m) \iso \struct{X}(n+m)$ making all the right diagrams commute 
\item $\struct{X}(n)$ are invertible (locally free of rank $1$) where we have seen,
\[ \struct{X}(1)^{\otimes n} \iso \struct{X}(n) \]
and,
\[ \struct{X}(n) \ot_{\struct{X}} \struct{X}(-n) \iso \struct{X}(0) = \struct{X} \]
meaning that $\struct{X}(-n) \cong \struct{X}(n)^\vee$. 
\end{enumerate}
\end{thm}

\begin{proof}
\[ \wt{M}(n) = \wt{M} \ot \struct{X}(n) = \wt{M} \ot \wt{S(n)} \iso \wt{M \ot_S S(n)} = \wt{M(n)} \]
where the tensor commuting with tilde map is an isomorphism since $S_1$ generates. Furthermore, 
\[ \struct{X}(n) \ot \struct{X}(m) = \wt{S(n)} \ot \wt{S(m)} \iso \wt{S(n)(m)} = \wt{S(n+m)} = \struct{X}(n+m) \]
\end{proof}

\begin{example}
When $X = \Proj{A[T_0, \dots, T_n]}$ then $\struct{X}(1) = \wt{S(1)}$ has global sections,
\[ \Gamma(X, \struct{X}(1)) = A T_0 \oplus \cdots \oplus A T_n \]
This is the universal data on $\P^n_A$ in sense of Exercise B(y) HW 9.  Furthermore,
\[ \struct{X}(1)|_{D_+(T_i)} = \struct{D_+(T_i)} \cdot T_i \]
and likewise,
\[ \struct{X}(n)|_{D_+(T_i)} = \struct{D_+(T_i)} \cdot T_i^n \]
\end{example}

\begin{example}
Ex. 2.2 in ``Further Properties of $\wt{M}$''). The $d$-uple embedding $\iota : \P^n_A \embed \P^N_A$ defined by $[t_0 : \dots : t_n] \mapsto [t^I]_{\sum t_i = d}$. This pulls back hyperplanes to degree $d$ hypersurfaces. Therefore we get,
\[ \iota^* \struct{\P^N_A}(1) \iso \struct{\P^n_A}(d) \]
\end{example}

\begin{example}
Let $k = \bar{k}$. We've seen that $\struct{\P^n_k}(-1)$ is (isomorphic to) the ideal sheaf of every hyperplane $H \subset \P^n_A$ as $\struct{\P^n}$-modules (since $\mathrm{GL}_{n+1}(k) \acts k[x_0, \dots, x_n]$ as a graded ring). Consider a smooth closed curve $j : X \embed \P^2_k$ which is not a line. Let $L \subset \P^2$ be a line. Then $\I_L \iso \struct{\P^2}(-1)$. Then applying $\iota^*$ we get,
\begin{center}
\begin{tikzcd}
j^* \struct{\P^2}(-1) \arrow[r] & j^* \struct{\P^2}(-1) \arrow[r, equals] & \struct{X}
\end{tikzcd}
\end{center}
has image exactly $\I_{X \cap L} \subset \struct{X}$ using a local calculation. But $\struct{X}$ is a sheaf of Dedekind domains and $\I_{X \cap L}$ is a nonzero ideal and thus invertible. Therefore,
\[ j^* \struct{\P^2}(-1) \iso \I_{X \cap L} \]
is an isomorphism. 
\end{example}

\begin{defn}
Let $S$ be an $\N$-graded ring and $X = \Proj{S}$ and $\F$ a $\struct{X}$-module. Define,
\[ \Gamma_*(\F) = \bigoplus_{n \ge 0} \Gamma(X, \F(n)) \]
is a graded module over,
\[ \Gamma_*(\struct{X}, \struct{X}(n)) \]
which is an $\N$-graded ring but may not be generated by degree $1$.
\end{defn}

\begin{rmk}
We have,
\[ S = \bigoplus_{n \ge 0} S_n = \bigoplus_{n \ge 0} (S(n))_0 \to \Gamma_*(\struct{X}) \]
as $\N$-graded rings. Therefore $\Gamma_*(\F)$ is a graded $S$-module.
\end{rmk}

\begin{rmk}
We saw that the above map is an isomorphism if $S = A[T_0, \dots, T_n]$.
\end{rmk}

Now we assume that $S$ is generated by $S_1$ over $S_0$. 
For an $\N$-graded modules $M$ then consider,
\[ \alpha_M : M = \bigoplus_{n \ge 0} M_n = \bigoplus_{n \ge 0} M(n)_0 \to \bigoplus_{n \ge 0} \Gamma(X, \wt{M(n)}) = \Gamma_*(\wt{M}) \]
which is linear over,
\[ \alpha_S : S \to \Gamma_*(\struct{X}) \]
For a $\struct{X}$-module $\F$ we get,
\[ \beta_\F : \wt{\Gamma_*(\F)} \to \F \]
using,
\[ \Gamma_*(\F)_{(f)} \to \Gamma(D_+(f), \F) \]
for $f \in S_d$ over $S_{(f)}$. This map gives,
\[ \Gamma_*(\F)_{(f)} \to \Gamma(D_+(f), \F(dr)) \ot_{S_{(f)}} S(-rd)_{(f)} \]
given by sending,
\[ \frac{m}{f^r} \mapsto m \ot \frac{1}{f^r} \]
and the right-hand side is equal to,
\begin{align*}
 \Gamma(D_+(f), \F(dr)) \ot_{S_{(f)}} S(-rd)_{(f)} & = \Gamma(D_+(f), \F(dr)) \ot \Gamma(D_+(f), \struct{X}(-rd)) \to \Gamma(D_+(f), \F(dr)(-dr))
\\
& = \Gamma(D_+(f), \F)
\end{align*}

\section{Dec. 3}

The ring $\Gamma_*(\F, \L)$ is a graded module over $\Gamma_*(\struct{X}, \L) = \mathfrak{S}$. Then we construct a canonical map,
\[ \beta_f : \Gamma_*(\F, \L)_{(f)} \to \Gamma(X_f, \F) \]
defined by sending,
\[ \frac{s_n}{f^n} \mapsto a_n \]
for $s_n \in \Gamma(X, \F \ot \L^{\ot n})$ where $s_n|_{X_f} = f^{\ot n} a_n$ because,
\[ s_n|_{X_f} \in \Gamma(X_f, \F \ot \L^{\ot n}) \quad \text{ and } \quad \L^{\ot n} |_{X_f} \cong f^{\ot n} \cdot \struct{X_f} \]
In the key example, $\Gamma_*(\F)$ is an $S$-module via,
\[ S \to \Gamma_*(\struct{X}) \]

\begin{lemma}
if $X$ is qcqs then $\beta_f$ is an isomorphism.
\end{lemma}

\begin{proof}
Focus on qc and separated. Let $\{ U_i \}$ be a finite affine open cover of $X$ trivializing $\L$ and such that $U_i \cap U_j$ affine because we assumed that $X$ is separated. We reduce the problem to the affine case. The $\beta_f$ construction is \underline{functorial} with respect to restriction along open immersions $U \embed X$ since $U_{(f|_U)} = U \cap X_f$. We have a diagram,
\begin{center}
\begin{tikzcd}
0 \arrow[r] & \Gamma_*(\F, \L)_{(f)} \arrow[d, "\beta_f"] \arrow[r] & \prod_{i} \Gamma(\F_i, \L_i) \arrow[d, "\prod \beta_{f_i}"] \arrow[r] & \prod_{i,j} \Gamma_*(\F_{ij}, \F_{ij}) \arrow[d, "\prod_{ij} \beta_{f_{ij}}"]
\\
0 \arrow[r] & \Gamma(X_f, \F) \arrow[r] & \prod_{i} \Gamma((U_{i})_{f_i}, \F_i) \arrow[r] & \Gamma((U_{ij})_{f_{ij}}, \F_{ij}) 
\end{tikzcd}
\end{center}
commutes and therefore we reduce to showing that the $\beta$ map is an isomorphism on affines because the diagram proves that $\beta_f$ is an isomorphism on $X$. 
\bigskip\\
Let $X = \Spec{A}$ and $\L = \struct{X}$ and $\F = \wt{A}$. Then,
\[ \Gamma_*(\F, \L) = \bigoplus_{n \ge 0} \Gamma(X, \F \ot \struct{X}^{\ot n}) = \bigoplus_{n \ge 0} M = M[T] \]
Then $f \in A$ is some element and,
\[ \mathfrak{S} = \bigoplus_{n \ge 0} \Gamma(X, \struct{X}^{\ot n}) = A[T] \]
Then we have the map,
\[ \beta_f : M[T]_{(fT)} \to \Gamma(X_f, \wt{M}) = M_f \]
but the degree $0$ fractions in $M[T]_{fT}$ have to be $M_f$ so this map is an isomorphism.
\end{proof}

\begin{cor}
For $Y \embed \P^n_A = X$ a closed subscheme. Then,
\[ I = \Gamma_*(\I_Y) \embed \Gamma_*(\struct{X}) = A[T_0, \dots, T_n] \]
is an ideal and $Y = \Proj{S/I} \embed \Proj{S}$ as schemes.
\end{cor}

\begin{cor}
If $S_1$ is $S_0$-finite and generates $S$ over $S_0$ and $\F$ is quasi-coherent, finite type on $X = \Proj{S}$ then $\F = \wt{m}$ for $S$-finite $M$. 
\end{cor}

\begin{proof}
Choose $M$ such that $\wt{M} = \F$. Consider $\{ M_i \}$ directed system of finitely generated graded $S$-submodule of $M$. Then,
\[ \varinjlim_{i} M_i = M \]
Because tilde interacts well with limits and submodules,
\[ \wt{M_i} \subset \wt{M} = \F \]
and,
\[ \varinjlim \wt{M_i} = \F \]
Enough to show that $\{ \wt{M_i} \}$ on $X$ stabilizes. Since $X$ is qc it is enough to check on finitely many affine opens $\Spec{A}$ covering $X$. But $\F|_{\Spec{A}} = \wt{N'}$ for finite $N'$ because $\F$ is finite type. Therefore,
\[ \wt{M_i} |_{\Spec{A}} \subset \wt{N'} \]
and $N'$ is finitely generated so eventually this hits each generator of $N'$ so we only need finitely many. 
\end{proof}

\begin{rmk}
For noetherian $S$, it is true that $\Gamma_*(\F)$ is $S$-finite but the proof uses sheaf cohomology.
\end{rmk}

\section{Jan. 7}

For work with $\Cl{X}$ we assume that $X$ is smooth in codimension $\le 1$ meaning the singular locus has codimension $\ge 2$. To connected $\Cl{X}$ to $\Pic{X}$ we need the following property.

\begin{defn}
Say $X$ is \textit{locally factorial} if $X$ is noetherian and all $\stalk{X}{x}$ are UFDs.
\end{defn}

\begin{example}
If $X = \Spec{A}$ for $A$ a UFD then all $S^{-1} A$ are UFDs and hence all local rings are UFDs.
\end{example}

\begin{example}
$X = \P^n_R$ is locally factorial for $R$ locally factorial.
\end{example}

\begin{defn}
A locally noetherian scheme $X$ is \textit{regular} if all $\stalk{X}{x}$ are regular.
\end{defn}

\begin{defn}
Say that a noetherian ring $A$ is \textit{regular} if all $A_\m$ are regular for $\m \subset A$ maximal.
\end{defn}

\begin{example}
If $k = \bar{k}$, and $A$ is a \textit{smooth} $k$-algebra (in the Jacobian sense) then $A$ is regular.
\end{example}

\begin{thm}[Serre]
Let $A$ be a regular noetherian local ring and $\p \subset A$ a prime. Then $A_\p$ is regular.
\end{thm}

\begin{cor}
If $A$ is a regular noetherian ring then all $A_\p$ are regular.
\end{cor}

\begin{proof}
Choose a maximal ideal $\m \supset \p$ then $A_\p = (A_\m)_{\p_\m}$ and $A_\m$ is regular by hypothesis so $A_\p$ is regular.
\end{proof}

\begin{example}
If $k = \bar{k}$ and $X$ is a finite type $k$-scheme then $k$ is smooth if and only if $X$ is regular.
\end{example}

\begin{thm}
Regular local rings are UFDs. Therefore regular schemes are locally factorial.
\end{thm}

\subsection{Equivalence of Picard Group and Class Group}

For the rest of today, $X$ is noetherian, integrally, and locally factorial. 

\begin{rmk}
[H] uses an intermediate notion of a Cartier divisor which we avoid. Furthermore, [H] assumes that $X$ is separated which is not needed [EGA IV$_4$, 21.6.10(ii)]. 
\end{rmk}

\begin{rmk}
The role of separateness: if $Y \neq Y'$ are prime divisors then $\Spec{\stalk{X}{\eta_Y}}, \Spec{\stalk{X}{\eta_Y'}} \to X$ have $\eta_Y \neq \eta_{Y'}$ and therefore the valuations of $Y$ and $Y'$ on $K(X)$ (using the valuative criterion of separatedness).
\end{rmk}

\begin{prop}
For $X$ as above, if $Y \subset X$ is a prime divisor then $\I_Y \subset \struct{X}$ is invertible.
\end{prop}

\begin{proof}
Since $\I_Y \subset \struct{X}$ is coherent it suffices to show that $(\I_Y)_x \subset \stalk{X}{x}$ is a principal ideal. For $x \notin Y$ we know that $(\I_Y)_x = \stalk{X}{x}$ so we are done. Otherwise, $(\I_Y)_x = \stalk{X}{x}$ is a height $1$ prime ideal and $\stalk{X}{x}$ is a UFD so $(\I_Y)_x$ is principal and therefore we are done. 
\end{proof}

\begin{rmk}
Now we build a map $\Div{X} \to \Pic{X}$ via $[Y] \mapsto (\I_Y)^{\otimes -1}$ and extending linearly with the tensor product. We need to show this map is surjective and has kernel exactly equal to the principal divisors.
\end{rmk}

\begin{rmk}
Consider the map,
\[ \bigotimes_j \I_{Y_j}^{\otimes m_j} \onto \prod \I_{Y_j}^{m_j} \]
for $m_g \ge 0$ where the right hand side is a product of ideals. However, this is a surjection of invertible sheaves and therefore an isomorphism.  
\end{rmk} 

\begin{rmk}
For an open $U \subset X$ we can write,
\[ \I_Y(U) = 
\begin{cases}
\struct{X}(U) & U \cap Y = \empty 
\\
\{ f \in \struct{X}(U) \mid \ord_{Y}(f) \ge 1 \} & U \cap Y \neq \empty 
\end{cases} \]
We need to make the dual of the ideal more concrete. Consider the generic point $\eta : \Spec{K(X)} \to X$. This is an affine map so $\eta_* (K(X)) = \underline{K(X)}$ (this is a fancy way of saying that $\Frac{A} = K(X)$ for all affine opens $\Spec{A} \subset X$). Then we have,
\[ \I_Y \embed \struct{X} \to \eta_* \eta^* \struct{X} = \eta_* (K(X)) = \underline{K(X)} \]
gives $(\I_Y)_{\eta} = K(X)$. Then,
\[ \I_Y^{\ot -1} \to \eta_* (\eta^* \I_Y^{\ot -1}) = \eta_* ((\I_Y)_{\eta}^{\ot -1}) = \eta_*(K(X)^\vee) \cong \eta_*(K(X)) = \underline{K(X)} \]
\end{rmk}

\begin{prop}
For $U \neq \empty$,
\[ \I_Y^{\otimes -1}(U) = 
\begin{cases}
\struct{X}(U) & U \cap Y = \empty 
\\
\{ 0 \} \cup f \in K(X)^\times \div{f}|_U \ge - [Y \cap U] \} & Y \cap Y \neq \empty 
\end{cases} \]
inside $K(X)$ where this condition on $f$ is equivalent by Hartongs to $f \in \struct{X}(U \setminus Y)$ and $\ord_Y(f) \ge -1$.
\end{prop}

\begin{defn}
For $D \in \Div{X}$ define,
\[ \struct{X}(D) \subset \underline{K(X)} \]
to be the sheaf,
\[ \Gamma(U, \struct{X}(D)) = \begin{cases}
\struct{X}(U) & U \cap \supp{}{D} = \empty 
\\
\{ 0 \} \cup \{ f \in K(X)^\times \mid \div{f} |_U + D |_U \ge 0 \} 
\end{cases} \]
\end{defn} 

\begin{example}
$\struct{X}(Y) = \I_Y^{-1}$ for prime $Y$.
\end{example}

\begin{prop}
Let,
\[ D = \sum n_j Y_j \]
then,
\[ \struct{X}(D) \cong \bigotimes_j \struct{X}(Y_j)^{\otimes n_j} \]
\end{prop}

\begin{rmk}
In [H] they write $\L(D)$ instead.
\end{rmk}

\begin{lemma}
For $f \in K(X)^\times$ then the map $\underline{K(X)} \xrightarrow{f} \underline{K(X)}$ carries,
\[ \struct{X}(\div{f}) \iso \struct{X} \]
\end{lemma}

\begin{proof}
Check on stalks and use that $\stalk{X}{x}$ is a UFD to write $f \in K(X)^\times$ as a reduced fraction. 
\end{proof}

\begin{cor}
We get a map $\Cl{X} \to \Pic{X}$ via $D \mapsto \struct{X}(D)$. 
\end{cor}

\begin{prop}
The above map is an isomorphism. 
\end{prop}

\begin{proof}
For $\L \in \Pic{X}$ consider,
\[ \L \embed \eta_* (\eta^* \L) = \eta_* (\L_\eta) \cong \eta_* (K(X)) = \underline{K(X)} \]
\end{proof}

\section{Jan. 10}

\subsection{Automorphisms of $\P^n$}

\renewcommand{\GL}{\mathrm{GL}}
\renewcommand{\PGL}{\mathrm{PGL}}

There is an action of $\GL_{n+1}(k)$ on $k[t_0, \dots, t_n]$ respecting the grading which gives an induced right action on $\P^n_k$. Then on $\P^n(\bar{k}) = (\bar{k}^{n+1} - \{ 0 \})/\bar{k}^\times$ this is effect of transpose on $M \in \PGL_{n+1}(\bar{k})$. Therefore we see that,
\[ \PGL_{n+1}(k) \subset \Aut{\P^n_k} \]
is an inclusion.

\begin{prop}
This is an equality for automorphisms over $k$.
\end{prop}

\begin{proof}
Say $\varphi : \P^n_k \iso \P^n_k$ is an automorphism. Consider $\varphi^* \struct{\P^n_k}(1) \in \Pic{\P^n_k}$. Furthermore, because $\varphi$ is an isomorphism $\varphi^*$ is an isomorphism on $\Pic{\P^n_k}$ and thus $\varphi^* \struct{\P^n_k}(1)$ is a generator. However, $\varphi^* \struct{\P^n_k}(1) \not\cong \struct{\P^n_k}(-1)$ because the left is generated by global sections (this property is preserved by pullback in general) but the right has no nonzero global sections. Therefore, there is an isomorphism $\alpha : \varphi^* \struct{\P^n_k}(1) \iso \struct{\P^n_k}(1)$. Furthermore, we pull back the canonical global sections,
\[ \varphi^* (t_j) \mapsto \sum_i M_{ij} t_i \]
where $M_{ij}$ is the matrix we want because the map $\varphi$ is determined by the generating global sections $\alpha(\varphi(t_j))$. Since $\varphi$ is an isomorphism $M$ must be invertible since $\{ \alpha(\varphi(t_j)) \}$ must be a $k$-basis of $\Gamma(\P^n_k, \struct{\P^n_k}(1))$.
\end{proof}

\begin{rmk}
Applications of $\Cl{X} \cong \Pic{X}$ in Ex. 6.10.2 and Remark. 6.10.3 in [H] (see FGA Explained for Picard schemes). 
\end{rmk}

\subsection{Ample Line Bundle}

\begin{defn}
For $f : X \to Y$ separated, finite type. We say that a line bundle $\L$ is \textit{very ample} over $Y$ if $\L \cong j^* \struct{\P^n_k}(1)$ for some $Y$-immersion $j : X \embed \P^n_Y$.
\end{defn}

\begin{defn}
Let $X$ be a noetherian scheme. Say $\L$ on $X$ is \textit{ample} if for all coherent $\F$ on $X$ there is an integer $n_\F$ such that for all $n \ge n_\F$,
\[ \Gamma(X, \F \ot \L^{\ot n}) \ot \struct{X} \onto \F \ot \L^{\ot n} \]
is a surjection meaning $\F \ot \L^{\ot n}$ is generated by global sections.
\end{defn}

\begin{rmk}
We can check this by global sections spanning $(\F \ot \L^{\ot n})_x$ over $\stalk{X}{x}$ for each $x \in X$ or equivalently by Nakayama that it spans $(\F \ot \L^{\ot n})_x \ot \kappa(x)$ over $\kappa(x)$ for all $x \in X$. By coherence, this shows that for sufficiently small opens $\Gamma(X, \F \ot \L^{\ot n})$ generates $(\F \ot \L^{\ot n})|_U$ over $\struct{U}$.
\end{rmk}

\begin{example}
Consider the following line bundles,
\begin{enumerate}
\item if $X$ is affine then $\struct{X}$ is ample
\item if $X = \P^n_k$ then $\struct{X}(-1)$ is not ample, because use $\F = \struct{X}$ then $\struct{X}(-1)^{\ot n} = \struct{X}(-n)$ has no global sections
\item for immersion $j : X \embed \P^n_A$ then $\L = j^* \struct{}(1)$ is ample (very ample over a ring implies ample).
\end{enumerate}
The last one needs a proof.
\end{example}

\begin{prop}
Let $j : X \embed \P^n_A$ be an immersion then $\L = j^* \struct{}(1)$ is ample (very ample over a ring implies ample).
\end{prop}

\begin{proof}
Let $\overline{X} \subset \P^N_A$ be the schematic closure. Then $X \embed \overline{X}$ is an open subscheme. By [Ex. 5.15(b)] given a coherent sheaf $\F$ on $X$ there is a coherent sheaf $\G$ on $\overline{X}$ with $\G |_X = \F$. Then we apply Serre's theorem to $\G$. Then for all $n \gg 0$ we have $\G(n)$ is generated by global sections. Therefore $\F(n) = \G(n) |_X$ is also generated by global sections proving that $\struct{X}(1)$ is ample. 
\end{proof}

\begin{lemma}
For $X$ noetherian, $\L$ invertible on $X$ then if $\L^{\ot m_0}$ is ample for some $m_0 > 0$ then $\L^{\ot m}$ is ample for all $m > 0$.
\end{lemma}

\begin{proof}
Fix $m$. Let $\F$ be coherent. Then,
\[ \F \ot (\L^{\ot m})^{\ot n} = (\F \ot \L^{\ot m r}) \ot \L^{\ot m_0 q m} \]
for $n = m_0 q + r$ with $0 \le r < m_0$. There are finitely many such coherent $(\F \ot \L^{\ot m r})$ and $\L^{\ot m_0}$ is ample so for sufficiently large $n$ we have sufficiently large $q$ such that all these sheaves are generated by global sections.
\end{proof}

\begin{rmk}
Using the Veronese embedding if $\L$ is very ample then $\L^{\ot m}$ is very ample for all $m > 0$.
\end{rmk}

\begin{rmk}
For smooth connected curve $X$ over $k = \bar{k}$ write $\L = \struct{X}(D)$ then Riemann-Roch shows that $\L$ is ample if and only if $\deg{D} > 0$. 
\end{rmk}

\begin{thm}
For $X$ of finite type over $A$ a noetherian ring. Then a line bundle $\L$ on $X$ is ample if and only if $\L^{\ot m}$ is very ample over $A$ for some $m > 0$.
\end{thm}

\begin{proof}
One direction is clear. If $\L^{\ot m}$ is very ample then $\L^{\ot m}$ is ample so $\L$ is ample. The other direction is in the notes.
\end{proof}

\section{Cech Cohomology}

\renewcommand{\U}{\mathfrak{U}}

Let $X$ be a topological space, $\U$ an open cover and $\F \in \Ab(X)$. There are three notions of ``\v{C}ech complex'' for $\F$ with respect to $\U$. However, they all have the same cohomology.

\begin{defn}
For $n \ge 0$ define \textit{the unrestricted Cech complex},
\[ \check{C}^n(\U, \F) = \prod_{\underline{i} \in I^{n+1}} \F(U_{\underline{i}}) \]
where $U_{(i_0, \dots, i_n)} = U_{i_0} \cap \dots \cap U_{i_n}$ with differential,
\[ \check{\d}^n : \check{C}^n(\U, \F) \to \check{C}^{n+1}(\U, \F) \]
is defined,
\[ \check{\d}(\xi)_{(i_0, \dots, i_{n+1})} = \sum_{\ell = 0}^{n+1} (-1)^\ell \res_\ell (\xi_{i_0, \dots, \hat{i_\ell}, \dots, i_{n+1}}) \]
where $\res_\ell : \F(U_{(i_0, \dots, \hat{i}_\ell, \dots, i_{n+1})}) \to \F(U_{\underline{i}})$ is the restriction map. 
\end{defn}

\begin{defn}
The \textit{alternating Cech complex},
\[ C^n_{\text{alt}}(\U, \F) = \{ (\xi_{\underline{i}}) \in \check{C}^n(\U, \F) \mid \xi_{\underline{i}} \text{ is alternating} \} \]
meaning $\xi_{\underline{i}} = 0$ if there are repeated indices and $\xi_{\sigma(\underline{i})} = \mathrm{sing}(\sigma) \xi_{\underline{i}}$. 
\end{defn}

\begin{defn}
Fix a well-ordering on $I$. Then there is a quotient complex,
\[ C^n(\U, \F) = \prod_{i_0 < \dots < i_n} \F(U_{\underline{i}}) \]
called the \textit{ordered Cech complex} which depends on $I$ and the ordering.
\end{defn}

\begin{prop}
The map $C_{\text{alt}}^\bullet \to \check{C}^\bullet \to C^\bullet$ is an isomorphism of complexes.
\end{prop}

\begin{proof}
Alternating chains have the same information. 
\end{proof}

\begin{example}
Let $\mathfrak{V} = \{ V_0, V_1 \}$. The ordered Cech complex is,
\[ \F(V_0) \times \F(V_1) \to \F(V_{01}) \to 0 \]
where the differential is,
\[ (s_0, s_1) \mapsto s_1 - s_0 \]
\end{example}

\begin{example}
Let $\U = \{ U_0, U_1, U_2 \}$ then $C^\bullet(\U, \F)$ is,
\[ \F(U_0) \times \F(U_1) \times \F(U_2) \to \F(U_{12}) \times \F(U_{02}) \times \F(U_{01}) \to \F(U_{012}) \to 0 \]
The differentials take,
\[ (s_0, s_1, s_2) \mapsto (s_2 - s_1, s_2 - s_0, s_1 - s_0) \quad (t_{12}, t_{02}, t_{01}) \mapsto (t_{12} - t_{02} + t_{01}) \]
\end{example}

\begin{defn}
We define the cohomology of this complex as,
\begin{enumerate}
\item $\check{H}^n(\U, \F) = H^n(\check{C}^\bullet(\U, \F))$
\item $H^n_{\text{alt}}(\U, \F) = H^n(C^\bullet_{\text{alt}}(\U, \F))$
\item $H^n(\U, \F) = H^n(C^\bullet(\U, \F))$.
\end{enumerate}
\end{defn}

\begin{rmk}
[H] denotes $H^n(\U, \F)$ as $\check{H}^n(\U, \F)$ unfortunately.
\end{rmk}

\begin{rmk}
If $\# I = N$ then $C^m(\U, \F) = 0$ for $n > N$ because there are no strictly ordered lists of size greater than $N$.
\end{rmk}

\begin{example}
Let $X = \P^1_A$ and $\F = \struct{X}$. Then $\U = \{ U_0 = D_+(t_0), U_1 = D_+(t_1) \}$. Therefore, the Cech complex is,
\[ A[x] \times A[x^{-1}] \xrightarrow{\d} A[x, x^{-1}] \]
\[ (f(x), g(x^{-1})) \mapsto g(x^{-1}) - f(x) \]
Then,
\[ H^0 = \ker{\d} = \{ (f,g) \mid f(x) = g(x^{-1}) \} = \{ (a,a) \mid a \in A \} \]
and likewise,
\[ H^1 = \coker{\d} = 0 \]
since $\d$ is extremely surjective because every element of $A[x,x^{-1}]$ can be split into positive and negative degrees.
\end{example}

\begin{thm}
The maps $C^\bullet_{\text{alt}} \embed \check{C}^\bullet \to C^\bullet$ are each chain homotopy equivalences.
\end{thm}

\begin{cor}
Hence, there are natural isomorphisms,
\[ H^n_{\text{alt}}(\U, \F) \iso \check{H}^n(\U, \F) \iso H^n(\U, \F) \]
\end{cor}

\subsection{Refinement}

\begin{defn}
Let $\U$ be an open cover. We say that $\mathfrak{V}$ is a refinement if there exists a map of index sets $\tau : J \to I$ such that $V_j \subset U_{\tau{i}}$ (note: there are usually many $\tau$ and the $\tau$ need not be injective). 
\end{defn}

\begin{rmk}
We want to relate $\check{C}^\bullet(\U, \F)$ to $\check{C}^\bullet(\mathfrak{V}, \F)$. This will not work for the ordered Cech complex.
\end{rmk}

\begin{defn}
Define $\tau : J^{n+1} \to I^{n+1}$ by $(j_0, \dots, j_n) \mapsto (\tau(j_0), \dots, \tau(j_n))$ and $\tau^\bullet : \check{C}^\bullet(\U, \F) \to \check{C}^\bullet(\mathfrak{V}, \F)$ by,
\[ \tau^n(\xi)_{\underline{j}} = \res^{U_{\tau(\underline{j})}}_{V_{\underline{j}}}(\xi_{\tau(\underline{j})}) \in \F(V_{\underline{j}}) \]
\end{defn}

\begin{thm}
The maps $\tau^\bullet : \check{H}^n(\U, \F) \to \check{H}^n(\mathfrak{V}, \F)$ is independent of $\tau$.
\end{thm}

\begin{defn}
The \textit{\v{C}ech cohomology} of $\F$ on $X$ is,
\[ \check{H}^n(X, \F) = \varinjlim_{\U} \check{H}^n(\U, \F) \]
\end{defn}

\section{Feb. 9}

Cech resolution of sheaves.

\begin{prop}
The complex,
\begin{center}
\begin{tikzcd}
0 \arrow[r] & \F \arrow[r] & \Cech^\bullet(\U, \F)
\end{tikzcd}
\end{center}
is exact so $\Cech^\bullet(\U, \F)$ forms a resolution of $\F$.
\end{prop}

\begin{proof}
Exactness at $\F$, and $\Cech^\bullet(\U, \F)$ has been seen so we just need to show exactness of $\Cech^\bullet(\U, \F)$ and above. This is local on $X$ so we may restrict to $V \subset X$ and $\Cech^\bullet(\U, \F)|_V = \Cech^\bullet(V \cap \U, \F|_V)$ so we take $V \in \U$ which gives a cover. Apply this to $V = U_\alpha$ for each $\alpha \in I$ and rename $U_\alpha \cap U_i$ as $U_i$ to reduce to case $U_\alpha = X$ for some $\alpha \in I$.
\bigskip\\
We build a homotopy from $\id$ to $0$ on $\Cech^\bullet(\U, \F)$. Meaning maps $k^n : \Cech^n(\U, \F) \to \Cech^{n-1}(\U, \F)$ such that $\d^{n-1} \circ k^n + k^{n+1} \circ \d^n = \id$ on $\Cech^n(\U, \F)$ for $n \ge 1$. We need,
\[ k^n : \prod_{\underline{i} \in I^{n+1}} \Gamma(U_{\underline{i}}, \F) \to \prod_{\underline{i} \in I^{n}} \Gamma(U_{\underline{i}}, \F) \]
defined by sending,
\[ \xi \mapsto (\xi_{(\alpha, i_0, \dots, i_n)})_{(i_0, \dots, i_{n})} \]
Then one can check this is a nulhomotopy. 
\end{proof}

\begin{rmk}
We really proved: $\check{C}^\bullet(\U, \F)$ is exact whenever some $U_\alpha = X$.
\end{rmk}

\begin{defn}
Use $\Cech^\bullet(\U, \F)$ to connected $\check{H}^\bullet(\U, \F)$ to $H^\bullet(X, \F)$. Choose an injective resolution $\I^\bullet$ of $\F$ in abelian sheaves to get a diagram,
\begin{center}
\begin{tikzcd}
0 \arrow[r] & \F \arrow[r] \arrow[d, equals] & \Cech^\bullet(\U, \F) \arrow[d, dashed]
\\
0 \arrow[r] & \F \arrow[r] & \I^\bullet
\end{tikzcd}
\end{center}
Therefore there is a morphism of complexes $\varphi : \Cech^\bullet(\U, \F) \to \I^\bullet$ unique up to homotopy. This gives a map $\varphi : \check{C}^\bullet(\U, \F) \to \Gamma(X, \I^\bullet)$ and thus a \textit{canonical} map on cohomology,
\[ \theta^n_\U : \check{H}^n(\U, \F) \to H^n(X, \F) \]
For $n = 0$ this is the identity map on $\Gamma(X, \F)$. For $n > 0$ next time we will discuss when this is an isomorphism. 
\end{defn}

\newcommand{\V}{\mathfrak{V}}

\begin{rmk}
What happens when we refine the cover? Suppose that $\V$ refines $\U$. Then we check that $\tau^\bullet_W  : \check{C}^\bullet(W \cap \U, \F) \to \check{C}^n(W \cap \V, \F)$ respect open inclusions $W' \embed W$ so we get,
\[ \tau^\bullet : \Cech^\bullet(\U, \F) \to \Cech^\bullet(\U, \F) \]
respecting $\id_\F$ on degree $0$ so giving a map of resolutions. This depends on $\tau$ but not on homology. Given $\I^\bullet$, if we choose $\psi : \Cech^\bullet(\V, \F) \to \I^\bullet$ as resolutions then we can \textit{pick} $\varphi$ to be,
\[ \Cech^\bullet(\U, \F) \xrightarrow{\tau^\bullet} \Cech^\bullet(\V, \F) \xrightarrow{\psi} \I^\bullet \]
and therefore get a commutative diagram,
\begin{center}
\begin{tikzcd}
\check{H}^n(\U, \F) \arrow[rd, "\theta^n_\U"'] \arrow[rr, "\text{can}"] & & \check{H}^n(\V, \F) \arrow[ld, "\theta^n_\V"] 
\\
& H^n(X, \F)
\end{tikzcd}
\end{center}
These allow us to apply $\varinjlim_{\U}$ to give a map,
\[ \theta^n  \check{H}^n(X, \F) \to H^n(X, \F) \]
Next time we will discuss when this map is an isomorphism.
\end{rmk}

\begin{rmk}
What about the $\delta$-functor structure. 
\end{rmk}

\begin{example}
Suppose that $X$ is a paracompact Hausdorff space so locally finite $\U$ are cofinal. In [Warner, 5.33] it is proved that $\check{H}^\bullet(X, \F)$ is equipped with the structure of an erasable $\delta$-functor making $\theta$ into a morphism of $\delta$-functors and hence an isomorphism. 
\end{example}

\begin{example}
Let $X$ be a separated scheme and $\U = \{ U_i \}$ is an affine cover (such $\U$ are cofinal). Then by separatedness every $U_{\underline{i}}$ is affine. Given a short exact sequence,
\begin{center}
\begin{tikzcd}
0 \arrow[r] & \F' \arrow[r] & \F \arrow[r] & \F'' \arrow[r] & 0
\end{tikzcd}
\end{center}
Then since $U_{\underline{i}}$ are affine we get an exact sequence,
\begin{center}
\begin{tikzcd}
0 \arrow[r] & \Gamma(U_{\underline{i}}, \F') \arrow[r] & \Gamma(U_{\underline{i}}, \F) \arrow[r] & \Gamma(U_{\underline{i}}, \F'') \arrow[r] & 0
\end{tikzcd}
\end{center}
Therefore, taking the product we see that,
\begin{center}
\begin{tikzcd}
0 \arrow[r] & \Cech^\bullet(\U, \F') \arrow[r] & \Cech^\bullet(\U, \F) \arrow[r] & \Cech^\bullet(\U, \F'') \arrow[r] & 0 
\end{tikzcd}
\end{center}
is an exact sequence of resolutions. Using Cartan-Eilenberg resolutions, can show that a short exact sequence of resolutions maps to a short exact sequence of injective resolutions. Therefore, we get a diagram,
\begin{center}
\begin{tikzcd}
0 \arrow[r] & \Cech^\bullet(\U, \F') \arrow[d] \arrow[r] & \Cech^\bullet(\U, \F) \arrow[r] \arrow[d] & \Cech^\bullet(\U, \F'') \arrow[r] \arrow[d] & 0
\\
0 \arrow[r] & \I'^\bullet \arrow[r] & \I^\bullet \arrow[r] & \I''^\bullet \arrow[r] & 0
\end{tikzcd}
\end{center}
Applying $\Gamma(X,-)$ this stays exact because we showed the top sequence is exact and the second sequence is made of injectives. Taking the associated long exact sequence of these short exact sequences of complexes, we get a $\delta$-functor structure (for quasi-coherent and affine $\U$) compatible with $\theta^\bullet_{\U}$.
\end{example}

\section{Feb. 11}

Consider the functors,
\[ \mathrm{Ab}(X) \xrightarrow{F} \mathrm{Ch}_{\ge 0}(\mathrm{Ab}(X)) \xrightarrow{H^0} \mathrm{Ab} \]
where $F : \F \mapsto \check{C}^\bullet(\U, \F)$. 
\bigskip\\
We have a $\delta$-functor $\{ H^n \}_{n \ge 0}$ on $\mathrm{Ch}_{\ge 0}(\mathrm{Ab}) \to \mathrm{Ab}$. from the snake lemma and this is erased by injectives so it is the derived functors of $H^0$. What about $R^q F$. Consider $T^q : \mathrm{Ab}(X) \to \mathrm{Ch}_{\ge 0}(\mathrm{Ab})$ sending,
\[ T^q(\F) = C^\bullet(\U, \underline{H}^q(\F)) \]
where $H^q(\F) : V \mapsto H^q(V, \F)$. To make $\{ T^q \}_{q \ge 0}$ into a $\delta$-functor, given a short exact sequence,
\begin{center}
\begin{tikzcd}
0 \arrow[r] & \F' \arrow[r] & \F \arrow[r] & \F'' \arrow[r] & 0
\end{tikzcd}
\end{center}
Then we get long exact sequences of cohomology on each $U_{\underline{i}}$ and taking products, this gives a long exact sequence of the $C^n(\U, \underline{H}^\bullet(\F))$. This is compatible with $\check{\d}^n$ so we get a long exact sequence in $\mathrm{Ch}_{\ge 0}(\mathrm{Ab})$. 

\begin{lemma}
We $\{ T^\bullet \} \cong \{ R^\bullet F \}$ uniquely as $\delta$-functors extending $T^0 = F$. 
\end{lemma}

\begin{proof}
We just need to show that $\{ T^\bullet \}$ is an erasable $\delta$-functor. In particular, it suffices to show that $T^q(\I) = 0$ for all $q \ge 1$ and $\I \in \mathrm{Ab}(X)$ an injective. However, this is clear because $H^q(U, \I) = 0$ for all $q \ge 1$ and any open $U$ (because $\I|_U$ is injective as well) and therefore 
\[ C^\bullet(\U, \underline{H}^q(\I)) = 0 \]
\end{proof}

\begin{lemma}
$F(\I)$ is $H^0$-acyclic for $\I \in \mathrm{Ab}(X)$ is injective. In particular, $\check{C}^\bullet(\U, \I)$ is exact in degrees $n \ge 1$. This holds if $\I$ is just flasque.
\end{lemma}

\begin{proof}
I claim that $\Cech^n(\U, \I)$ are all flasque and thus it computes $H^\bullet(X, \I)$ meaning,
\[ H^n(\check{C}^\bullet(\U, \I)) = H^n(\Gamma(X, \Cech^n(\U, \I))) = H^n(X, \I) \]
and $H^n(X, \I) = 0$ for $n \ge 0$. Indeed,
\[ \Cech^n(\U,\I) = \prod_{\underline{i} \in I^{n+1}} (j_{\underline{i}})_* (\I|_{U_{\underline{i}}}) \]
is flasque because restrictions of flasque are flat, pushforward of flasque are flasque, and products of flasque sheaves are flasque. 
\end{proof}

\section{Feb. 14}

\subsection{Affine Vanishing and Its Applications}

\begin{thm}
For $X = \Spec{A}$ and $\F$ quasi-coherent on $X$, then $H^n(X, \F) = 0$ for all $n > 0$. 
\end{thm}

\begin{proof}
First, I claim that $\check{H}^n(\U, \F) = 0$ for $n > 0$ and $\U$ a finite affine cover. We know that,
\begin{center}
\begin{tikzcd}
0 \arrow[r] & \F \arrow[r] & \Cech^\bullet(\U, \F) 
\end{tikzcd}
\end{center}
is a resolution so $\Cech^\bullet(\U, \F)$ is exact for $n > 0$. However, 
\[ \Cech^n(\U, \F) = \prod_{\underline{i} \in I^{n+1}} (j_{\underline{i}})_* (\F |_U) \]
is quasi-coherent because the cover is finite. Therefore,
\[ \check{C}^\bullet(\U, \F) = \Gamma(X, \Cech^\bullet(\U, \F)) \]
is also exact for $n > 0$ since $X$ is affine proving the vanishing. 
\bigskip\\
For $n = 1$ we know that,
\[ H^1(X, \F) = \check{H}^1(X, \F) = \varinjlim_{\U} \check{H}^1(\U, \F) = 0 \]
because finite affine covers are cofinal under refinement since $X$ is affine (and hence quasi-compact). Now we proceed by induction (over all pairs of an affine scheme and a quasi-coherent sheaf) in $n$. Assume $n \ge 2$ and the result is established for all $1 \le q \le n-1$. In particular, 
\[ \underline{H}^q(\F)(U_{\underline{i}}) = H^q(U_{\underline{i}}, \F) = 0 \]
for $1 \le q \le n - 1$. Now we apply the Cech-to-derived spectral sequence for $\U$,
\[ E^{p,q}_2(\U) = \check{H}^p(\U, \underline{H}^q(\F)) \implies H^{p+q}(X, \F) \]
We have $E^{p,q}_2(\U) = 0$ for $1 \le q \le n-1$ by the previous calculation. Furthermore because $\check{H}^p(\U, \F) = 0$ for all $p > 0$ we see that $E^{p,0}_2 = 0$ for all $q > 0$. Therefore we see that $H^n(X, \F) = E^{0,n}_{\infty}$ because all others with $p+q = 0$ are zero. We don't know,
\[ E^{0,n}_2(\U) = \check{H}^0(\U, \underline{H}^n(\F)) \]
However, if we pass to the limit $\varinjlim_{\U}$ to get,
\[ E^{p,q}_2 = \check{H}^p(X, \underline{H}^q(\F)) \implies H^{p+q}(X, \F) \]
and thus $E^{0,n}_2 = 0$ proving our claim. 
\end{proof}

\section{Serre's Converse}

\begin{thm}
Let $X$ be quasi-compact. If $H^1(X, \I) = 0$ for all quasi-coherent $\I \subset \struct{X}$ then $X$ is affine.
\end{thm}

\begin{proof}
WLOG $X \neq \empty$. Let's show that $X$ has a closed point. Apply Zorn's lemma to the poset of nonempty closed subsets $Z \subset X$ ordered by reverse inclusion to get a \textit{minimal} such $Z$ (using the finite intersection property for qc to conclude that the intersection of a chain is nonempty). Such $Z$ is \textit{indiscreet} (otherwise it would have a proper closed subset). I claim $Z$ is a point. Using the reduced structure $Z$ is a scheme. Choose a nonempty affine open $U \subset Z$ then $U = Z$ then a maximal ideal of $\struct{}(U)$ is a closed point so $Z$ is a single point.
\bigskip\\
We will now use the affineness criterion [H, ChII, Ex. 2.17(b)]. Seek to cover $X$ by finitely many open $X_{f_j} = \{ x \in X \mid f_j(x) \neq 0 \}$ for $f_1, \dots, f_j \in \Gamma(X, \struct{X})$ such that $X_{f_j}$ are affine. By quasi-compactness, we just need to show that for each closed point $P \in X$ there is exists an affine open $X_f$ containing $P$. Indeed, let
\[ \Omega = \bigcup_{P \in X^{\text{closed}}} X_{f_P} \]
This is open and it contains every closed point so $X \setminus \Omega$ is a closed subset (and hence quasi-compact) with no closed points (closed point of closed subset is closed) so $X \setminus \Omega$ is empty. Then by quasi-compactness $\Omega$ is a finite union of such $X_{f_P}$ proving the claim.
\bigskip\\
Let $P \in X$ be closed and $U$ an affine open neighborhood and let $Y = (X \setminus U)_{\red}$. It suffices to find $f \in \I_Y(X)$ with $f(P) \neq 0$ since then $X_f \subset U$ and thus $X_f = D(f)$ in the affine scheme $U$ and hence is affine and also $P \in X_f$. Consider the sequence,
\begin{center}
\begin{tikzcd}
0 \arrow[r] & \I_{Y \cup \{ P \}} \arrow[r] & \I_Y \arrow[r] & \kappa(P) \arrow[r] & 0
\end{tikzcd}
\end{center}
which is exact by checking on $X \setminus \{ P \}$ and on $U$. Applying global sections,
\begin{center}
\begin{tikzcd}
0 \arrow[r] & \I_{Y \cup \{ P \}}(X) \arrow[r] & \I_Y(X) \arrow[r] & \kappa(P) \arrow[r] & H^1(X, \I_{Y \cup \{ P \}}) 
\end{tikzcd}
\end{center}
but $H^1(X, \I_{Y \cup \{ P \}}) = 0$ and thus $1 \in \kappa(P)$ is in the image of $\I_Y(X) \to \kappa(P)$ so such a function $f \in \I_Y(X)$ exists proving our claim. 
\end{proof}

\begin{lemma}
If $X$ is separated and qc with a cover by $r+1$ affine opens, then $H^i(X, \F) = 0$ for $i > r$ and all quasi-coherent $\F$. 
\end{lemma}

\begin{proof}
We know $H^i(X, \F) = \check{H}^i(\U, \F)$ for the cover $\U = \{ U_0,  \dots, U_r \}$. Using the ordered Cech complex, $C^i(\U, -) = 0$ for $i > r$ and therefore $H^i(X, \F) = 0$.
\end{proof}

\begin{prop}
Let $X$ be projective over a noetherian ring $A$. Let $\F \in \Coh{X}$. Then,
\begin{enumerate}
\item all $H^i(X, \F)$ are $A$-finite
\item Fix $j : X \embed \P^r_A$ and $\struct{X}(1) = j^* (\struct{\P^r_A}(1))$ then $H^i(X, \F(n)) = 0$ for all $n > n_\F$ and $i > 0$. 
\end{enumerate}
\end{prop}

\begin{proof}
Use descending induction on $i$. Fix $j : X \embed \P^r_A$. Reduce to $X = \P^r_A$ because $H^i(X, \F(n)) = H^i(\P^r_A, j_* \F(n))$ and $j_* \F(n) = (j_* \F)(n)$ and is coherent. 
\bigskip\\
For $i > r$ everything is zero so we are done. Now suppose that $(a)$ holds for all coherent sheaves in $i + 1$. By Serre's theorem,
\begin{center}
\begin{tikzcd}
0 \arrow[r] & \G \arrow[r] & \bigoplus_{k = 1}^N \struct{}(-m_k) \arrow[r] & \F \arrow[r] & 0
\end{tikzcd}
\end{center}
Note $\G$ is coherent! Taking the long exact sequence,
\begin{center}
\begin{tikzcd}
\bigoplus\limits_{k = 1}^N H^i(\P^r_A, \struct{}(-m_k)) \arrow[r] & H^i(\P^r_A, \F) \arrow[r] & H^{i+1}(\P^r_A, \G) 
\end{tikzcd}
\end{center}
The first term is $A$-finite by an explicit calculation and the last term by the induction hypothesis. Therefore $H^i(\P^r_A, \F)$ is $A$-finite. 
\bigskip\\
For (b) suppose we know for degree $i + 1$ for all coherent sheaves. Only $H^j$ for $j \le r$ mattes so we can use descending induction. We apply the previous sequence twisted by $ - \ot \struct{}(n)$ and taking the long exact sequence gives,
\begin{center}
\begin{tikzcd}
\bigoplus\limits_{k = 1}^N H^i(\P^r_A, \struct{}(n-m_k)) \arrow[r] & H^i(\P^r_A, \F(n)) \arrow[r] & H^{i+1}(\P^r_A, \G(n)) 
\end{tikzcd}
\end{center}
We can choose $n$ such that $\G(n)$ has vanishing higher cohomology and for $n > m_k$ for each $k$ then both terms are zero so $H^i(\P^r_A, \F(n)) = 0$ for $n \gg 0$ depending only on $\F$.
\end{proof}

\section{Ampleness and Higher direct Images}

\begin{thm}[Serre]
Let $X$ be proper over a Noetherian ring $A$, $\L$ is invertible on $X$. Then $\L$ is ample if and only if $\F \in \Coh{X}$ then $H^i(X, \F \ot \L^{\ot n}) = 0$ for all $i > 0$ and $n \gg 0$.
\end{thm}

\begin{proof}
We proved $\implies$ last time. Now we prove the other direction (which will not actually involve properness). Build $n_1 \ge 1$ so that $\L^{\ot n_1}$ is generated by global sections and thus the same holds for $\L^{\ot n_1 q}$ for $q \ge 1$. For a closed point $x \in X$. Consider the exact sequence,
\begin{center}
\begin{tikzcd}
0 \arrow[r] & \I_x \L^{\ot n} \arrow[r] & \L^{\ot n} \arrow[r] & \L^{\ot n}(x) \arrow[r] & 0
\end{tikzcd}
\end{center}
where the last term is a skyscraper. Since $\L$ is flat we have $\I_x \L^{\ot n} \cong \I_x \ot \L^{\ot n}$. The cohomology sequence gives,
\begin{center}
\begin{tikzcd}
H^0(X, \L^{\ot n}) \arrow[r] & \L^{\ot n}(x) \arrow[r] & H^1(X, \I_x \ot \L^{\ot n}) 
\end{tikzcd}
\end{center}
By the vanishing, there is $N_x \ge 1$ so that if $n \ge N_x$ then $H^0(X, \L^{\ot n}) \onto \L^{\ot n}(x)$ and hence by Nakayama, $\L_x^{\ot n}$ is generated by global sections. By coherence, there is an open $W_x$ neighborhood of $x$ such that $\L^{\ot N_x} |_{W_x}$ is generated by global section $\Gamma(X, \L^{\ot N_x})$. Also the same happens on $W_x$ for all $\L^{\ot N_x q}$ for $q \ge 1$. Then, 
\[ \bigcup_{x \in X} W_x \subset X \] 
is open and contains all closed points so it is all of $X$ because $X$ is qc and thus there is a finite collection $W_1, \dots, W_m$ that cover $m$. Then,
\[ n_1 = \prod_{j = 1}^m N_{x_j} \]
gives $\L^{\ot n_1}$ generated by global sections. 
\bigskip\\
Step 2: choose $\F \in \Coh{X}$ and $x \in X$ a closed point. Seek an open $x \in U_x$ and $n_x \ge 1$ such that $\Gamma(X, \F \ot \L^{\ot (n_x + r)})$ generates $\F \ot \L^{\ot (n_x + r)}|_{U_x}$ for all $0 \le r \le n_1 - 1$. We know that for all $n \ge n_{x,\F}$ depending on $x$ and $\F$ we know that,
\[ H^1(X, \I_x \F \ot \L^{\ot n}) = 0 \]
and therefore we get a short exact sequence,
\begin{center}
\begin{tikzcd}
0 \arrow[r] & \I_x \F \ot \L^{\ot n} \arrow[r] & \F \ot \L^{\ot n} \arrow[r] & (\F \ot \L^{\ot n})(x) \arrow[r] & 0
\end{tikzcd}
\end{center}
to get long exact sequence of cohomology giving,
\begin{center}
\begin{tikzcd}
\Gamma(X, \F \ot \L^{\ot n}) \arrow[r] & (\F \ot \L^{\ot n})(x) \arrow[r] & H^1(X, \I_x \F \ot \L^{\ot n}) 
\end{tikzcd}
\end{center}
so by the vanishing an Nakayama we see that $\Gamma(X, \F \ot \L^{\ot n})$ generates $(\F \ot \L^{\ot n})_x$ over $\stalk{X}{x}$. The spreading out to the finite number of coherent sheaves $\F \ot \L^{\ot n}$ for $n = n_{x, \F} + r$ for $0 \le r \le n_1 - 1$ then we can spread out to some $U_{x, \F}$ which satisfies the hypotheses. 
\bigskip\\
Step 3: finale. Pick a coherent sheaf $\F \in \Coh{X}$ we want $\F \ot \L^{\ot n}$ to be generated by global sections for all $n \gg 0$. For closed point $x \in X$ and $n \ge n_{x, \F}$ we write,
\[ n = n_x + (n_1 q + r) \]
for $q \ge 0$ and $0 \le r \le n_1 - 1$ so we have,
\[ \F \ot \L^{\ot n} = (\F \ot \L^{\ot (n_x + r)}) \ot (\L^{\ot n_1})^{\ot q} \]
The second term is generated by global sections by step 1 and the second is generated by global sections over $U_{x, \F}$. Thus, if $n \ge n_x$ then $(\F \ot \L^{\ot n})|_{U_x}$ is generated by $\Gamma(X, \F \ot \L^{\ot n})$. Because the $U_x$ cover $X$ and $X$ is qc finitely many of then $U_{x_1}, \dots, U_{x_b}$ covers $X$. Therefore, we can take $N = \max{n_{x_j}}$ and we win.  
\end{proof}

\begin{defn}
Let $f : (X, \struct{X}) \to (Y, \struct{Y})$ map of ringed spaces, the \textit{higher direct images} $R^i f_* : \Mod{X} \to \Mod{Y}$ are the derived functors of $f_*$. 
\end{defn}

\begin{rmk}
Let $Y$ be a point, then $R^i f_* = H^i(X,-)$.
\end{rmk}

\begin{lemma}
Let $f : (X, \struct{X}) \to (Y, \struct{Y})$ map of ringed spaces,
\begin{enumerate}
\item $R^i f_*$ is unaffected by passing to $\Ab(X) \to \Ab(Y)$
\item $R^i f_* \F$ is the sheafification of $V \mapsto H^i(f^{-1}(V), \F)$ for opens $V \subset Y$.
\end{enumerate}
\end{lemma}

\begin{proof}
Notice that (b) $\implies$ (a) because cohomology works in the category of abelian sheaves. For (b) notice that this sheafification gives a $\delta$-functor and agreement for $i = 0$ is clear. This is erasable because $H^i(f^{-1}(V), \I) = 0$ for any injective sheaf $\I$. 
\end{proof}

\begin{lemma}
For $X \to \Spec{A}$ quasi-compact separated, $\F \in \QCoh{X}$ then for $a \in A$ the map $H^i(X, \F) \to H^i(X_a, \F_a)$ is linear for $A \to A_a$ and thus gives,
\[ H^i(X, \F)_a \to H^i(X_a, \F_a) \]
this is an isomorphism. Therefore, $R^i f_* \F = \wt{H^i(X, \F)}$. 
\end{lemma}

\begin{rmk}
Therefore $R^i g_*$ preserves quasi-coherence for any qc separated $g : X \to Y$ because $R^i g_* (\F)|_V \cong R^i(g_V)_* (\F|_{g^{-1}(V)})$ for open sets $V \subset Y$ and shrink to affine.
\end{rmk}

\begin{proof}
Choose a finite affine open cover $\U = \{ U_0, \dots, U_r \}$ of $X$ so all $U_{\underline{i}}$ are also affine. Then we have $C^\bullet(\U, \F) \to C^\bullet(\U_a, \F_a)$ by restriction inducing the map $\check{H}^i(\U, \F) \to \check{H}^i(\U_a, \F_a)$ which agrees with $\alpha : H^i(X, \F) \to H^i(X_a, \F_a)$. However, $C^\bullet(\U, \F)_a \iso C^\bullet(\U_a, \F_a)$ is an isomorphism because these are finite products and $\F$ is quasi-coherent so $\Gamma(U, \F)_a \iso \Gamma(U_a, \F_a)$ for any affine $a$. Finally, localization is exact so we get an isomorphism on cohomology. 
\end{proof}

\begin{cor}
Let $f : X \to Y$ be projective with $Y$ locally noetherian then $R^i f_*$ preserves coherence. 
\end{cor}

\begin{proof}
$R^i f_*$ preserves quasi-coherence. This is a local question so we may assume that $Y = \Spec{A}$ then $(R^i f_*) \F = \wt{H^i(X, \F)}$ but $H^i(X, \F)$ are $A$-finite.  
\end{proof}

\begin{thm}[Coherence of Higher Direct Images]
For a proper map $f : X \to Y$ with $Y$ locally noetherian, $R^i f_*$ preserves coherence. 
\end{thm}

\begin{proof}
Because this is local on $Y$, this is equivalent to when $Y = \Spec{A}$ for a noetherian ring $A$ we have that $H^i(X, \F)$ is $A$-finite since $R^i f_* \F = \wt{H^i(X, \F)}$ for any $\F \in \Coh{X}$. 
\end{proof}

\section{Feb. 23}

\subsection{Duality Formalism}

\subsubsection{Coherence of Higher Direct Images}

\begin{thm}
For $X$ proper over noetherian $A$, then all $H^i(X, \F)$ are $A$-finite for $\F \in \Coh{X}$. 
\end{thm}

\begin{proof}
We may assume $X \neq \empty$. We proceed by Noetherian induction. Suppose that there is a counterexample $\G$ then $\Supp{}{\G} \subset X$ is some closed subset so by Noetherian induction, we can pick a minimal such example and pass to the closed subscheme $X'$ defined by $\shAnn{\struct{X}}{\G}$. Then $\G = j_* \G'$ for $\G' \in \Coh{X'}$ and $H^\bullet(X, \G) = H^\bullet(X', \G')$ where $j : X' \embed X$ is the closed immersion. Now $X' \to \Spec{A}$ is proper because $X' \embed X$ is closed. Therefore, we may assume that for all proper closed subsets the theorem holds. Replace $X$ by $X'$.
\bigskip\\
Now we apply Chow's lemma to get $\pi : X' \to X$ projective and birational and $f : X' \to \Spec{A}$ is proper. Using properness of $\pi$ there is an open $U \subset X$ such that $\pi^{-1}(U) \iso U$ is an isomorphism. Choose $\F \in \Coh{X}$ consider,
\[ \alpha : \F \to \pi_* \pi^* \F \]
and set $\F' = \pi^* \F \in \Coh{X'}$. First, $\alpha|_U$ is an isomorphism so $\K = \ker{\alpha}$ and $\sC = \coker{\alpha}$ are supported on $Z = X \setminus U$. Since $\pi$ is projective, $\pi_* \F'$ is coherent so $\K$ and $\sC$ are also coherent. Therefore we get short exact sequences of coherent sheaves,
\begin{center}
\begin{tikzcd}
0 \arrow[r] & \K \arrow[r] & \F \arrow[r] & \sQ \arrow[r] & 0
\\
0 \arrow[r] & \sQ \arrow[r] & \pi_* \F' \arrow[r] & \sC \arrow[r] & 0
\end{tikzcd}
\end{center}
For $\G \in \Coh{X}$ let $\cP(\G)$ be the property that all $H^i(X, \G)$ are $A$-finite. Notice that $\cP$ satisfies the two-out-of-three property for short exact sequences using the long exact sequence of cohomology. By the induction, hypothesis, $\K$ and $\sC$ satisfy $\cP$. Thus it suffices to show that $\pi_* \F'$ satisfies $\cP$. However, by the Leray spectral sequence,
\[ E_2^{p,q} = H^p(X, R^q \pi_* \F') \implies H^{p+q}(X', \F') \]
By the projectivity of $X'$ we know that $H^{p+q}(X', \F')$ is $A$-finite. Furthermore, the $R^q \pi_* \F'$ are coherent sheaves (by projectivity of $\pi$) and ford $q > 0$ they are supported on $X \setminus U$ and hence $E_2^{p,q}$ are $A$-finite for $q > 0$ by the induction assumption. Therefore $E_{\infty}^{p,q}$ is finite for $q > 0$ automatically since $A$ is noetherian. Furthermore, $E^{p, 0}_{\infty}$ can only be modified by taking quotients from $A$-submodules a finite number of times before the differential leaves the positive quadrant and hence since $E_{\infty}^{p,0}$ is $A$-finite we see that $E_{2}^{p, 0}$ is $A$-finite proving the claim. 
\end{proof}

\subsection{Coherent Duality}

Now we want to motivate formulations of Serre duality for the cohomology of coherent sheaves on smooth projective schemes $X$ of pure $\dim = d$ over a field $k$ (we allow for nonconnected schemes to allow for non-geometrically connected varieties). 

\begin{defn}
A scheme $X$ over a field $k$ is \textit{smooth} if $X$ is locally finite type over $k$ and $X_{\bar{k}}$ is regular. 
\end{defn}

\begin{rmk}
In HW9, develop the notion of smooth morphism $f : X \to S$ to any locally noetherian $S$ in analogy with submersions in differential geometry. 
\end{rmk}

\begin{rmk}
For $M$ a compact $\CC$-manifold of pure dimension $d$ there is a trace map,
\[ t : H^d(M, \Omega^d_M) \iso H^{2d}(M, \CC) \iso H_{\dR}^{2d}(M, \CC) \to \CC \]
where the last map,
\[ \alpha \mapsto \frac{1}{(2 \pi i)^d} \int_M \alpha \]
And for a holomorphic vector bundle $\E$ on $M$ there is a pairing,
\[ H^j(X, \E) \times H^{d-j}(M, \E^\vee \ot \Omega^d_M) \xrightarrow{\smile} H^d(M, \E \ot E^\vee \ot \Omega^d_M) \xrightarrow{t} H^d(M, \Omega^d_M) \]
\end{rmk}

\begin{rmk}
Serre duality will be a version of this in algebraic geometry over any field $k$ with two caveats,
\begin{enumerate}
\item Allow any coherent $\F$, not just locally free, so we need to rewrite,
\[ H^i(X, \E^\vee \ot \Omega_X^d) = \Ext{i}{X}{\E}{\Omega_X^d} \]
\item Give up pinning down explicit trace map,
\[ t : H^d(X, \Omega^d_X) \to k \]
\end{enumerate}
\end{rmk}

\begin{defn}
Let $(X, \struct{X})$ be a ringed space over a ring $A$. Then define, 
\[ \Ext{\bullet}{X}{\F}{-} = R^\bullet \Hom{X}{\F}{-}  \]
as functors $\Mod{\struct{X}} \to \Mod{A}$. Therefore, for an injective resolutions $\G \to \I^\bullet$ then,
\[ \Ext{n}{X}{\F}{\G} = H^n(\Hom{\struct{X}}{\F}{\I^\bullet}) \]
\end{defn}

\begin{rmk}
We don't have enough projectives but for $\varphi : \F \to \F'$, get $\Hom{X}{\F'}{-} \to \Hom{X}{\F}{-}$ and therefore,
\[ \Ext{\bullet}{X}{\F'}{-} \to \Ext{\bullet}{X}{\F}{-} \]
We can upgrade this to a morphism of $\delta$-functors.
\end{rmk}

\begin{rmk}
Some facts,
\begin{enumerate}
\item For $X = \Spec{A}$ noetherian,
\[ \Ext{i}{X}{\wt{M}}{\wt{N}} = \Ext{i}{A}{M}{N} \]
when $M$ is $A$-finite. 

\item $\Ext{\bullet}{X}{\struct{X}}{-} = H^\bullet(X, -)$ because it is the derived functor of $\Hom{\struct{X}}{\struct{X}}{-} = \Gamma(X, -)$

\item $\Ext{1}{X}{\F''}{\F'}$ is in bijection with isomorphism classes of extensions,
\begin{center}
\begin{tikzcd}
0 \arrow[r] & \F' \arrow[r] & \F \arrow[r] & \F'' \arrow[r] & 0
\end{tikzcd}
\end{center}
\end{enumerate}
\end{rmk}

\section{Feb. 25}

Let $X$ be proper of pure dim $d$ over $k$. Say a coherent sheaf on $X$ is \textit{dualizing} if it represents the contravariant functor $\F \mapsto H^d(X, \F)^\vee$ on $\Coh{X}$ and denote it as $\omega_{X/k}$ and all have ``universal'' $t : H^d(X, \F)^\vee \iso k$ called the trace map. 

\begin{prop}
Let $P = \P^d_k$ and $\omega_P = \Omega^d_{P/k} \cong \struct{P}(-d-1)$. Then there exists $t : H^d(P, \omega_P) \iso k$. Then,
\begin{enumerate}
\item $(\omega_P, t)$ is dualizing meaning $\Hom{P}{\F}{\omega_P} \to H^d(P, \F)^\vee$ is an isomorphism.

\item there is a unique $\delta$-functor isomorphism $\Ext{j}{P}{\F}{\omega_P} \iso H^{d-j}(P, \F)^\vee$ for $\F \in \Coh{P}$ recovering the above for $j = 0$.
\end{enumerate}
\end{prop}

\begin{proof}
For (i), both sides turn right exact sequences in $\F$ into left-exact sequences of $k$-vectorspaces (because $H^{d+1}(X,-) = 0$). By Serre's theorem, we have,
\[ \E = \struct{}(-q)^{\oplus n} \onto \F \]
Therefore, applying this to the kernel we get an exact sequence,
\begin{center}
\begin{tikzcd}
\E' \arrow[r] & \E \arrow[r] & \F \arrow[r] & 0
\end{tikzcd}
\end{center}
where the first two are vector bundles. Therefore, there is a diagram of exact sequences,
\begin{center}
\begin{tikzcd}
0 \arrow[r] & \Hom{P}{\F}{\omega_P} \arrow[d] \arrow[r] & \Hom{P}{\E}{\omega_P} \arrow[r] \arrow[d] & \Hom{P}{\E'}{\omega_P} \arrow[d]
\\
0 \arrow[r] & H^d(P, \F)^\vee \arrow[r] & H^d(P, \E)^\vee \arrow[r] & H^d(P, \E')^\vee
\end{tikzcd}
\end{center}
therefore, it suffices to show that the maps are isomorphisms for $\F = \struct{P}(-q)$ for all $q \gg 0$ since it commutes with direct sums. In this case,
\begin{center}
\begin{tikzcd}
\Hom{P}{\struct{}(-q)}{\struct{}(-d-1)} \arrow[d, equals] \arrow[r] & H^d(P, \struct{}(-q))^\vee 
\\
H^0(P, \struct{}(q - d - 1))
\end{tikzcd}
\end{center}
and this is the map arising from the perfect pairing,
\[ H^0(P, \struct{}(q - d - 1)) \times H^d(P, \struct{}(-q)) \to H^d(P, \struct{}(-d-1)) \iso k \]
To get a unique $\delta$-functor map,
\[ \Ext{j}{\F}{\omega_P} \to H^{d-j}(P, \F)^\vee \]
agreeing with $j = 0$ and that it is an isomorphism it suffices to show that both sides are coerasable on $\Coh{P}$ meaning each $\F$ admits $\E \onto \F$ for $\E \in \Coh{P}$ which kills the functors for $j > 0$. We take $\E = \struct{}(-q)^{\oplus n}$ which surject. We need to show these kills for $q > 0$ and $n \ge 1$. However, by compatibility with direct sums it suffices to show that,
\[ \Ext{j}{P}{\struct{}(-q)}{\omega_P} = H^j(P, \struct{}(q - d - 1)) = 0 \]
for $q > 0$ and all $j > 0$ (even $j = d$) and likewise,
\[ H^{d-j}(P, \struct{}(-q))^\vee = 0 \]
for $j > 0$ because for $0 < j < d$ these always vanish and for $j = d$ then this is zero because $q > 0$.  
\end{proof}

\subsection{Ext Sheaves}

\begin{defn}
For a ringed space $X$, and $\F \in \Mod{X}$ define $\shExt{\bullet}{X}{\F}{-} : \Mod{X} \to \Mod{X}$ to be the derived functor of $\shHom{X}{\F}{-} : \Mod{X} \to \Mod{X}$. For $\varphi : \F \to \F'$ we get $\shHom{X}{\F'}{-} \to \shHom{X}{\F}{-}$ via $\alpha \mapsto \alpha \circ \varphi$ so we get $\shExt{\bullet}{X}{\F'}{-} \to \shExt{\bullet}{X}{\F}{-}$. We will also upgrade this to a $\delta$-functor.
\end{defn}

\begin{prop}
\begin{enumerate}
\item for open $U \subset X$ we have,
\[ \shExt{\bullet}{X}{\F}{\G}|_U \cong \shExt{\bullet}{U}{\F|_U}{\G|_U} \]
bi-$\delta$-functorially
\item for $\E$ locally free of finite rank $\shExt{\bullet}{X}{\F}{\E \ot \G} = \E \ot \shExt{\bullet}{X}{\F}{\G} = \shExt{\bullet}{X}{\E^\vee \ot \F}{\G}$
\item there is a spectral sequence,
\[ E^{p,q}_2 = H^p(X, \shExt{q}{X}{\F}{\G}) \implies \Ext{p+q}{X}{\F}{\G} \]
\item if $X$ is a locally noetherian scheme and $\F$ is coherent and $\G$ is quasi-coherent then all $\shExt{q}{X}{\F}{\G}$ are quasi-coherent and,
\[ \shExt{q}{X}{\F}{\G}_x = \Ext{q}{\stalk{X}{x}}{\F_x}{\G_x} \]
and if $X$ is additionally affine $X = \Spec{A}$ and $\F = \wt{M}$ and $\G = \wt{N}$ then, 
\[ \shExt{q}{X}{\F}{\G} = \wt{\Ext{q}{A}{M}{N}} \]
\end{enumerate}
\end{prop}

\begin{prop}
Let $j : X \embed P = \P^N_k$ be a closed immersion with pure dimension $d$ so pure codimension $c = N - d$. Then,
\begin{enumerate}
\item $\shExt{i}{P}{j_* \struct{X}}{\omega_P} = 0$ for $i < c$ 
\item $\omega_{X/k} = \shExt{c}{P}{j_* \struct{X}}{\omega_P}$ is dualizing on $X$.
\end{enumerate}
\end{prop}

\section{Theorem on Formal Functions}

\begin{rmk}
In topology, a continuous map $f : X \to Y$ is universally closed (called ``proper'' in Bourbaki) if for all continuous map $Z \to Y$ the map $X \times_Y Z \to Z$ is closed. By [Lang, Ch.II, Ex. 22-26] univ. closed iff closed + qu-compact fibers iff closed + preimage of quasi-compact is quasi-compact.
\end{rmk}

\begin{rmk}
For locally compact Hausdorff $X,Y$, closedness is automatic so proper if and only if the preimage of quasi-compact is quasi-compact.
\end{rmk}

\begin{defn}
Say a continuous map $f : X \to Y$ is \textit{(topologically) proper} if it is universally closed and separated. 
\end{defn}

\begin{rmk}
By [SGA1, Exp XII, Prop. 3.2(v)] for $f : X \to Y$ over $\CC$ between finite type $\CC$-schemes  then $f$ is proper if and only if $X(\C) \to Y(\C)$ is proper with the analytic topology. 
\end{rmk}

\begin{rmk}
For continuous map $f : X \to Y$ of topological spaces, $\F \in \Ab{X}$ if $y \in Y$ then get,
\[ R^i f_* (\F)_y = \varinjlim_{U \ni y} H^i(f^{-1}(U), \F) \to H^i(X_y, \F_y) \]
where the first equality holds because sheafification preserves stalks. Note that $\F_y = \F|_{X_y}$. 
\end{rmk}

\begin{thm}[Topological Proper Base Change]
If $f$ is (topologically) proper then all,
\[ \theta^i_y : R^i f_* (\F)_y \to H^i(X_y, \F|_{X_y}) \]
are isomorphisms.
\end{thm}

\begin{proof}
[SGA4I, ExpVbis, Cor. 4.12]
\end{proof}

\begin{rmk}
We want a variant of this theorem for schemes. First we give a variant construction of the base change map for locally ringed spaces.
\end{rmk}

\begin{defn}
For $f : X \to Y$ a map of locally ringed spaces, make $X_y$ a locally ringed space using $\struct{X_y} = \struct{X}|_{X_y} \ot_{\stalk{Y}{y}} \kappa(y)$ (probably this is the fiber product in LRS). For $\F \in \Mod{X}$ define $\F_y \in \Mod{X_y}$ by 
\[ \F_y = (\F|_{X_y}) \ot_{\stalk{Y}{y}} \kappa(y) = j_y^*(\F) \]
Then we define,
\[  R^i f_* (\F)_y \to H^i(X_y, \F|_{X_y}) \to H^i(X_y, \F_y) \]
which is linear over $\stalk{Y}{y} \to \kappa(y)$ giving an induced map,
\[ \theta^i_y : R^i f_* (\F)_y \ot_{\stalk{Y}{y}} \kappa(y) \to H^i(X_y, \F_y) \]
\end{defn}

\begin{example}
If $f : X \to Y$ is quasi-compact separated morphism of schemes and $\F \in \QCoh{X}$. Assume $Y = \Spec{A}$ (the construction is local over the base anyway) then we get,
\[ H^i(X, \F) \ot_A \kappa(y) \to H^i(X_y, \F_y) \]
\end{example}

\begin{rmk}
One more variant (the epsilon variant): infinitesimal fibers! For $n \ge 0$.,
\[ X_{y,n} = (X_y, \struct{X} \ot_{\stalk{Y}{y}} \stalk{Y}{y} / \m_y^{n+1}) \quad \text{ and } \quad \F_{y, n} = \F|_{X_y} \ot_{\stalk{Y}{y}} (\stalk{Y}{y}/\m_y^{n+1}) \]
We likewise get a base change map,
\[ \theta^i_{y,n} : R^i f_* (\F)_y / \m_y^{n+1} \to H^i(X_{y,n}, \F_{y,n}) \]
\end{rmk}

\begin{rmk}
Focus on a proper map of local. noetherian schemes $f : X \to Y$ and $\F \in \Coh{X}$ so $R^i f_*(\F) \in \Coh{Y}$ and thus $R^i f_* (\F)_y$ is $\stalk{Y}{y}$-finite. 
\end{rmk}

\subsection{Completions}

\begin{rmk}
We are about to take completions so lets review some properties.
\end{rmk}

\begin{prop}
For $M$ a finite $A$-module with $A$ noetherian and an ideal $I \subset A$, then the completion,
\[ \hat{M} = \varprojlim M / I^{n+1} M \]
is a module over $\hat{A}$ which is noetherian and,
\begin{enumerate}
\item $A \to \hat{A}$ is flat (faithfully flat if $A$ is local with maximal ideal $I$)
\item $M \ot_A \hat{A} \to \hat{M}$ is an isomorphism.
\end{enumerate}
\end{prop}

\begin{proof}
[Mat, Thms 8.7, 8.8]
\end{proof}

\begin{rmk}
The map $\Z \to \Z_p$ is not faithfully flat so we need $A$ to be local. 
\end{rmk}

\begin{defn}
Pass to the inverse limit of $\theta^i_{y,n}$ then,
\[ \hat{\theta}^i_y R^i f_*(\F)_y \ot_{\stalk{Y}{y}} \widehat{\stalk{Y}{y}} \cong \widehat{R^i f_*(\F)_y} \to \varprojlim_n H^i(X_{y,n}, \F_{y,n}) \]
\end{defn}

\begin{rmk}
In an ideal world, all $\theta^i_{y,n}$ would be isomorphisms. Grothendieck gave two ways to get satisfactory results:
\begin{enumerate}
\item base change theorems for cohomology with flatness results
\item Direct isomorphism for $\hat{\theta}^i_y$ with no further assumptions (even get more than with $\stalk{Y}{y}$!)
\end{enumerate}
\end{rmk}

\begin{thm}[Formal Functions]
For $f : X \to \Spec{A}$ proper with $A$ noetherian and $\F \in \Coh{X}$ and $I \subset A$ na ideal let,
\[ X_n = X \ot_A A /I^{n+1} \quad \F_n = \F \ot_A A / I^{n+1} \in \Coh{X_n} \]
Then the maps,
\[ \hat{\theta}^i_y : H^i(X, \F) \ot_A \hat{A} \to \varprojlim_n H^i(X_n, \F_n) \]
is an isomorphism.
\end{thm}

\begin{rmk}
\begin{enumerate}
\item For $A = \stalk{Y}{y}$ and $I = \m_y$ and $X \times_Y \Spec{\stalk{Y}{y}}$ then get $\hat{\theta}^i_y$ isomorphism in preceding discussion since coherent cohomology commutes with flat base change $\Spec{A_y} \to \Spec{A}$)
\item Since $A \to \hat{A}$ is flat, $H^i(X, \F) \ot_A \hat{A} \cong H^i(X_{\hat{A}}, \F_{\hat{A}})$ and therefore this is a statement this is a statement over a complete base. This is basically formal GAGA. 
\item $A = \Z[[q]]$ and $I = (q)$ is widely used in the arithmetic of modular forms... 
\end{enumerate}
\end{rmk}

\begin{thm}
If $f : X \to Y$ with $Y$ local noetherian is proper + quasi-finite then $f$ is finite.
\end{thm}

\begin{proof}
By properness, $f_* \struct{X}$ is $\struct{Y}$-finite (even coherent). If $f$ were affine, then $X = \rSpec{Y}{f_* \struct{X}}$ is finite over $Y$ so it suffices to show that $f$ is affine. We work locally on the base so assume that $Y = \Spec{A}$. Thus it suffices to show that $X$ is affine. Since $f$ is quasi-compact and separated, $X$ is quasi-compact and separated. We will check the cohomology criterion $H^i(X, \F) = 0$ for all $i > 0$ and $\F \in \Coh{X}$. Enough to show that $H^i(X, \F)_\m = 0$ for all $\m \in \mSpec{A}$. By finiteness (proper + coherent) and NAK it suffices to show that $H^i(X, \F) \ot_A A / \m = 0$ and since,
\[ \widehat{H^i(X, \F)} \ot_{\hat{A}} \hat{A} / \m = H^i(X, \F) \ot_A A / \m \]
it suffices to show that the completion is zero. By the theorem on formal functions,
\[ \widehat{H^i(X, \F)} = \varprojlim H^i(X_n, \F_n) \]
but the fiber spaces $X_n$ are finite discrete spaces and thus have vanishing higher cohomology completing the theorem.
\end{proof}

\section{Mar. 4}

\begin{thm}[Zariski's Main Theorem]
For noetherian $Y$ and $f : X \to Y$ quasi-finite and separated then there exists a diagram,
\begin{center}
\begin{tikzcd}
X \arrow[rd, "f"] \arrow[rr, hook] & & Y \arrow[ld, "\text{finite}"]
\\
& Y
\end{tikzcd}
\end{center}
where $Y \embed Y$ is an open immersion.
\end{thm}

\begin{proof}
[EGA III, Thm 4.4.3]
\end{proof}

\begin{rmk}
This is useful. Used to show that \etale maps $f : X \to Y$ Zariski locally looks like $\Spec{B} \to \Spec{A}$ where $B = (A[T]/(f))_{f'h}$ for monic $f \in A[T]$.
\end{rmk}

\begin{rmk}
Let $X$ be the line with two origins then the map $f : X \to \A^1$ is quasi-finite but not separated and indeed cannot be an open in a finite cover of $\A^1$ because such covers are affine over $\A^1$ and hence affine and hence separated but of course $X$ is not separated. 
\end{rmk}

\begin{thm}[Stein]
Let $f : X \to Y$ be proper then let $Y' = \rSpec{Y}{f_* \struct{X}}$. Because $f$ is proper $f_* \struct{X}$ is coherent and thus $\pi : Y' \to Y$ is finite. Furthermore, $f' : X \to Y'$ has $f'_* \struct{X} = \struct{Y}$ by construction. Then $f'$ has connected fibers.
\end{thm}

\begin{proof}
The only thing to show is the following.
\end{proof}

\begin{thm}[Zariski Connectedness]
If $f : X \to Y$ is proper and $f_* \struct{X} = \struct{Y}$ then $f$ has connected fibers.
\end{thm}

\begin{proof}
Assume $f : X \to Y$ is proper with $f_* \struct{X} = \struct{Y}$. Choose $y \in Y$. Want $X_y$ is connected. Pass to $X \times_Y \Spec{\stalk{Y}{y}} \to \Spec{\stalk{Y}{y}}$ this is a flat base change along $\Spec{\stalk{Y}{y}} \to Y$ and this commutes with cohomology so we can reduce to the case $Y = \Spec{R}$ with $f_* \struct{X} = \struct{Y}$ and $R$ is local noetherian where $y = \m$ is the closed point. We want to deduce that $X_0$ is connected. Consider the completion $R \to \hat{R}$ and apply flat base change and the special fiber does not change but preserves the cohomology condition so we can assume that $R$ is complete. The theorem of formal functions says that,
\[ R = H^0(X, \struct{X}) = \varprojlim H^0(X_n, \struct{X_n}) \] 
Assume that $X_0$ is disconnected write $X_0 = U_0 \sqcup V_0$ for nonempty open disjoint $U_0$ and $V_0$. Then we get nontrivial idempotents $e_0 |_{U_0} = 1$ and $e_0|_{V_0} = 0$. All the $X_n$ have compatible separations so get compatible sequence of $e_n$ and get $e = \varprojlim e_n \in R$ is an idempotent but it is not $0$ or $1$ at any finite level and therefore cannot be $0$ or $1$ but $R$ is local and thus $\Spec{R}$ is connected giving a contradiction. 
\end{proof}

\begin{rmk}
Consider,
\begin{enumerate}
\item In HW10, show the fibers are actually geometrically connected.
\item Consider $f : X \to Y$ surjective proper with $X$ integral and $Y$ normal noetherian and connected. Assume $X_\eta$ satisfies $H^0(X_\eta, \struct{X_\eta}) = \kappa(\eta) = K(Y)$ (e.g. if $X_\eta$ is geometrically connected and geometrically reduced over $\eta$) then $f_* \struct{X} = \struct{Y}$ because $f_* \struct{X}$ is finite over $\struct{Y}$ inside the function field and hence integral but $Y$ is normal so we get an equality. 
\end{enumerate}
\end{rmk}

\begin{example}
$Y = \A^1_k = \Spec{k[t]}$ and $X = \Proj{k[t][x,y,z]/(xy - t^2 z^2)}$ so the fibers over $t \neq 0$ are irreducible hyperbolas and for $t = 0$ the fiber is a reducible but connected union of two lines. Indeed $f_* \struct{X} = \struct{Y}$ so this connectedness follows from Zariski's theorem.  
\end{example}

\begin{rmk}
Now we turn to the proof of the theorem on formal functions.
\end{rmk}

\begin{theorem}
Let $f : X \to \Spec{A}$ be proper with $A$ noetherian. Let $I \subset A$ be an ideal and $\hat{A}$ the $I$-adic completion of $A$. For $\F \in \Coh{X}$ we have,
\[ H^i(X, \F) \ot_A \hat{A} = \widehat{H^i(X, \F)} \iso \varprojlim_n H^i(X_n, \F_n) \]
is a topological isomorphism for all $i \ge 0$. 
\end{theorem}

\begin{proof}
First notice that,
\[ H^i(X_n, \F_n) = H^i(X, \F / I^{n+1} \F) \]
Consider the long exact sequence associated to,
\begin{center}
\begin{tikzcd}
0 \arrow[r] & I^n \F \arrow[r] & \F \arrow[r] & \F_{n-1} \arrow[r] & 0
\end{tikzcd}
\end{center}
which gives,
\begin{center}
\begin{tikzcd}[column sep = small]
\cdots \arrow[r] & H^i(X, I^n \F) \arrow[r, "u_n"] & H^i(X, \F) \arrow[r] & H^i(X, \F_{n-1}) \arrow[r, "\delta"] & H^{i+1}(X, I^n \F) \arrow[r, "v_n"] & H^{i+1}(X, \F) \arrow[r] & \cdots
\end{tikzcd}
\end{center}
this produces compatible short exact sequences,
\begin{center}
\begin{tikzcd}
(*)_n & 0 \arrow[r] & H^i(X, \F) / \im{u_n} \arrow[r] & H^i(X, \F_{n-1}) \arrow[r, "\delta"] & \ker{v_n} \arrow[r] & 0
\end{tikzcd}
\end{center}
We have $I^n H^i(X, \F) \subset \im{u_n}$ because for $a \in I^n$ we have,
\begin{center}
\begin{tikzcd}
\F \arrow[rd] \arrow[rr, "a"] & & \F 
\\
& I^n \F \arrow[ru, hook]
\end{tikzcd}
\end{center}
The transition maps,
\[ H^i(X, \F) / \im{u_{n+1}} \to H^i(X, \F) / \im{u_n} \]
are surjective since $\im{u_{n+1}} \subset \im{u_n}$ (so it satisfies M-L condition) and therefore applying $\varprojlim_n$ is exact giving a short exact sequence,
\begin{center}
\begin{tikzcd}
0 \arrow[r] & \varprojlim H^i(X, \F) / \im{u_n} \arrow[r] & \varprojlim_n H^i(X, \F_{n-1}) \arrow[r] & \varprojlim_n \ker{v_n} \arrow[r] & 0
\end{tikzcd}
\end{center} 
We want $\{ \im{u_n} \}$ to define the $I$-adic topology on $H^i(X, \F)$ and for big $d > 0$ we want
\[ \ker{v_{n+d}} \to \ker{v_h} \] vanish for $n \ge d$ and thus $\varprojlim_n \ker{v_n} = 0$ (however the first statement is stronger). 
\bigskip\\
We introduce a graded module:
\[ S = \bigoplus_{n \ge 0} I^n \]
which is noetherian $\N$-graded and generated by $S_1$ over $S_0$. Let,
\[ M = \bigoplus_{n \ge 0} H^i(X, I^n \F) \]
is an $\N$-graded $S$-module. Consider the map,
\[ S_m \times H^i(X, I^n \F) \xrightarrow{*} H^i(X, I^{n+m} \F) \quad \text{via} \quad (a, \xi) \mapsto H^i(\mu_a)(\xi) \]
where $\mu_a : I^n \F \to I^{n+m} \F$ given by multiplication by $a \in S_n =  I^n$. Then,
\[ M = H^i(X, \bigoplus i^n \F) = H^i(X, g_* \F_S) \]
where
\begin{center}
\begin{tikzcd}
X_S \arrow[d] \arrow[r, "g"] & X \arrow[d]
\\
\Spec{S} \arrow[r] & \Spec{A}
\end{tikzcd}
\end{center}
because $g$ is an affine map so we get,
\[ H^i(X, g_* \F_S) = H^i(X_S, \F_S) \]
which is $S$-finite because $X_S \to \Spec{S}$ is proper and $\F_S \in \Coh{X_S}$. But $M \onto \bigoplus \im{u_n} = N$ is $S$-linear quotient so $N$ is $S$-finite graded. For big $n$ we have $N_{n+1}  = S_1 \cdot N_n$ so we get $\im{u_{n+1}} = I \im{u_n}$ inside $H^i(X, \F)$ which means that this sequence is $I$-adic. 
\end{proof}

\section{Mar 7}

\subsection{Base Change for Cohomology}

Zariski connectedness theorem also allows us to conclude (even in characteristic $p$) that the moduli space of curves $\M_g$ is connected. Deligne and Mumford compactified the moduli space and used Zariski connectedness to show that the new moduli space is irreducible. 
\bigskip\\
Last time we saw that in the formal limit cohomology and base change commute for proper maps. This time we talk about base change theorems without the formal limit but with flatness assumptions.
\bigskip\\
Consider a proper map $f : X \to S$ with $S$ locally noetherian, $\F \in \Coh{X}$ which we will usually take to be flat over $S$.

\begin{defn}
For a map of schemes $f : X \to S$ we say that a $\struct{X}$-module $\F$ is flat over $S$ if $\F_x$ is flat over $\stalk{X}{f(x)}$ for each $x \in X$. If $\F$ is quasi-coherent this is equivalent to $\F(U)$ being a flat $\struct{Y}(V)$-module for affines $U \to V$). 
\end{defn}

\begin{example}
If $f : X \to S$ is flat and $\F$ is locally free then $f$ is flat over $S$.
\end{example}

Consider the morphisms,
\[ \theta^i_s  : (R^i f_* \F)(s) \to H^i(X_s, \F_s) \]
We want to know when these are isomorphisms and relate it to local freeness of the higher pushforwards. Hopefully we will also so some natural examples where $\theta^i$ fails to be an isomorphism. 

\begin{rmk}
Often these questions arise in many algebraic situations for example:
\begin{enumerate}
\item $S$ is a ``moduli scheme'' and $X \to S$ is a ``universal family'' (e.g. of curves, abelian varieties, etc) and $\F = \struct{X}$ (or for $f$ smooth also  $(\Omega_{X/S}^1)^{\ot q}$ called the Hodge bundle). Then $(R^i f_*)(\F)$ arise for intersection theory on $S$ and, in the case of elliptic curves, for the arithmetic of modular forms.

\item Given a variety (lets say a curve for concreteness) defined over $\Q$ then we want to spread out to an ``arithmetic family'' over $S = \Spec{R}$ for $R$ a DVR with fraction field $K$ (in our case some number field). Then for $X_K \embed \P^n_K$ we can take the schematic closure of $X_K \embed \P^n_K \embed \P^n_R$ giving closed $X \embed \P^n_R$. Now the map $\struct{X} \to \struct{X_K}$ is injective so $\struct{X}$ is $R$-torsion free and hence $X \to \Spec{R}$ is proper and flat. If the generic fiber is geometrically connected and reduced, then by the Zariski connectedness theorem we know the special fiber is also geometrically connected which is our first step towards making it ``nice'' by blowing up. Supposing,
\[ H^1(X, \struct{X}) \ot_R k \to H^1(X_0, \struct{X_0}) \]
is an isomorphism then we can get a handle on ``improving'' the special fiber.
\end{enumerate}
\end{rmk}

Concrete question:

\begin{example}
For a smooth proper map $f : X \to S$ with all $X_{\bar{s}}$ connected curves, is the function,
\[ s \mapsto h^0(X_s, \Omega^1_{X_s/\kappa(s}) = h^1(X_s, \struct{X_s}) \]
sending $s$ to the genus of the fiber locally constant on $S$?  
\end{example}

\begin{example}
A variant of this: for a closed subscheme $X \embed \P^N_S$ such that $X \to S$ is flat, is the Hilbert polynomial,
\[ s \mapsto \chi_s(n) = \sum_{j} (-1)^j h^j(X_s, \struct{X_s}(n)) \in \Q[n] \]
locally constant in $s$ as a rational polynomial. 
\end{example}

The answer to both questions is yes (and the second will prove the first). However, $h^j(X_s, \F_S)$ can jump even though the Euler characteristics do not. 

\begin{example}
Let $C$ be a smooth proper geometrically connected curve, with genus $g > 0$, over a field $k$. Let $p \in C(k)$ be a rational point. Consider $X = C \times C$ and $S = C$ and let $f : X \to S$ be the second projection. This map is smooth because it is the base change of $C \to \Spec{k}$ along $C \to \Spec{k}$. On $X$ there is are divisors: the diagonal $\Delta \subset C \times C$ and the constant section $\{ p \} \times C \subset X$. Consider the invertible sheaf \[ \L = \struct{X}(\Delta - \{ p \} \times C) \cong \struct{X}(\Delta) \ot \struct{X}(\{ p \} \times C)^{\ot -1} = \I_\Delta^{\ot -1} \ot \I_{\sigma} \]
where $\sigma : S \to X$ is the section sending $x \mapsto (p, x)$. Now $X_s = C \ot_k \kappa(s)$ and we would like that $\L_s \cong \struct{C}(s - p_{\kappa(s)})$. This works because the formation of $\I_\Delta$ and $\I_\sigma$ are ideal sheaves of sections $S \embed X$ and therefore are flat over the base so the formation of their ideal sheaves commutes with base change. Indeed, given the situation,
\begin{center}
\begin{tikzcd}
Z \arrow[rd, "\text{flat}"] \arrow[r, "j"] & Y \arrow[d]
\\
& T 
\end{tikzcd}
\end{center}
then the sequence,
\begin{center}
\begin{tikzcd}
0 \arrow[r] & \I_Z \arrow[r] & \struct{Y} \arrow[r] & j_* \struct{Z} \arrow[r] & 0
\end{tikzcd}
\end{center}
is exact after any base change $T' \to T$ because $j_* \struct{Z}$ is flat over $T$ and so the sequence remains exact after any tensor product because Tor with coefficients in $j_* \struct{Z}$ vanishes.
\bigskip\\
Returning to our situation, what can we say about $h^i(\L_s)$ for $i = 0,1$.
\begin{enumerate}
\item[i = 0] \[ H^0(X_s, \L_s) = \Gamma(C_s, \struct{}(s - p)) = 
\begin{cases}
0 & s \neq p \text{ and } g > 0
\\
k & s = p
\end{cases} \]

\item[i = 1]
\[ h^1(X_s, \L_s) = - \chi(X_s, \L_s) + h^0(X_s, \L_s) = \deg_{\kappa(s)} (s - p) + 1 - g + h^0(X_s, \L_s) = h^0(X_s, \L_s) + 1 - g \]
and therefore is just $h^0(X_s, \L_s)$ but with a constant shift and therefore has the same jumping behavior. This is a constant because,
\[ g_s = \text{genus}(C_s) = \dim_{\kappa(s)} h^0(X_s, \Omega^1_{C_s /\kappa(s)}) = \dim_k h^0(C, \Omega^1_{C/k}) = g \]
\end{enumerate} 
Now let's look at the higher direct images $f_* \L$ and $R^1 f_*(\F)$ on $S = C$. Since $X$ is integral and $X \to S$ is dominant so $f_* \F$ is $\struct{C}$-torsion-free because its sections are sections on $X$. Then $f_* \L = 0$ because of the torsion-freeness it suffices to check that the generic fiber is zero but by flat base change,
\[ (f_* \L)_\eta = H^0(C_\eta, \L_\eta) = \Gamma(C_\eta, \struct{}(\eta - p_\eta)) = 0 \]
While we cannot use cohomology and base change to a random point, we can apply it to the generic point because $\eta \to C$ is flat. Using an upcoming result (next time), vanishing of $h^1(X_s, \L_s)$ for $s \in C \setminus \{ p \}$ does imply that $R^1 f_* (\F) |_{C \setminus \{ p \}} = 0$ so we show that $R^1 f_* (\F) \in \Coh{C}$ is some skyscraper sheave supported on $\{ p \}$. I claim however that $R^1 f_* (\L) \neq 0$. We will show that,
\[ \theta^1_p : (R^1 f_* \L)(p) \to H^1(C, \L_p) \cong H^1(C, \struct{}) \neq 0 \]
is surjective. Localize to $\Spec{\stalk{C}{p}} \to C$ so consider $f : Y \to \Spec{R}$ for a DVR $R$ whose fibers are $1$-dimensional and $f$ is proper and flat. Then there is an exact sequence,
\begin{center}
\begin{tikzcd}
0 \arrow[r] & \m \arrow[r] & R \arrow[r] & k \arrow[r] & 0
\end{tikzcd}
\end{center}
which by flatness of $\L$ over $R$ gives an exact sequence of sheaves on $Y$,
\begin{center}
\begin{tikzcd}
0 \arrow[r] & \m \ot \L \arrow[r] & \L \arrow[r] & \L_0 \arrow[r] & 0
\end{tikzcd}
\end{center}
and then the long exact sequence of cohomology gives,
\begin{center}
\begin{tikzcd}
H^1(X, \L) \arrow[r, "\theta"] & H^1(Y_0, \L_0) \arrow[r, "\delta"] & H^2(Y, \m \ot_R \L) 
\end{tikzcd}
\end{center}
By the theorem on formal functions,
\[ H^2(Y, \F) \ot_R \hat{R} = \varprojlim H^2(Y_n, \F_n) = 0 \]
because $\dim{Y_n} = 1$ for all $n$ but $R \to \hat{R}$ is faithfully flat and therefore $H^2(Y, \F) = 0$. 
\end{example}

\subsection{General Base Change Map}

Consider the diagram of ringed spaces,
\begin{center}
\begin{tikzcd}
X' \arrow[d, "f'"] \arrow[r, "g'"] & X \arrow[d, "f"]
\\
S' \arrow[r, "g"] & S
\end{tikzcd}
\end{center}
And let $\F$ be a $\struct{X}$-module and $\F' = g'^* \F$. Then there is a map,
\[ \theta^i : g^* R^i f_* (\F) \to R^i f'_* \F' \]
This arises from the presheaves,
\[ \underline{H}^i(\F) : U \to H^i(X_U, \F) \]
\[ \underline{H}^i(\F') : U' \to H^i(X'_{U'}, \F') \]
and there is a natural map,
\[ \underline{H}^i(\F)(U) \to \underline{H}^i(\F')(g^{-1}(U)) = g_* (\underline{H}^i(\F'))(U) \]
so we get $\underline{H}^i(\F) \to g_* (\underline{H}^i(\F'))$ and therefore applying sheafification,
\begin{center}
\begin{tikzcd}
\underline{H}^i(\F) \arrow[d] \arrow[r] & g_* (\underline{H}^i(\F')) \arrow[r] & g_* (R^i f_*'(\F')) 
\\
R^i f_*(\F) \arrow[rru] 
\end{tikzcd}
\end{center}
yields $\theta^i$. In scheme setting for quasi-coherent $\F$, over affine opens $\Spec{A} \subset S$ and $\Spec{A'} \subset S'$ mapping to $\Spec{A}$ we get,
\[ H^i(X_A, \F) \to H^i(X'_{A'}, \F') \]
which is linear over $A \to A^1$, so this gives,
\[ A' \ot_A H^i(X_A, \F) \to H^i(X'_{A'}, \F') \]

\begin{rmk}
Such $\theta^i$ being an isomorphism for all $S' \to S$ is closely linked to local freeness of $R^i f_* (\F)$ and the surjectivity of $\theta^i_s$ for all $s$. Next time we will see this and how the jumping phenomena are generic along closed subschemes. 
\end{rmk}

\section{Cohomology and Base Change II}

The theorem on formal functions only assumed properness not flatness but only gives an isomorphism on \textit{formal} neighborhoods not Zariski neighborhoods. Today we prove cohomology and base change for flat sheaves.
\bigskip\\
Consider a proper map $f : X \to S$ for locally noetherian $S$, and $\F \in \Coh{X}$ that is flat over $S$. We have the base change maps,
\[ \theta^i_s : (R^i f_* \F)(s) \to H^i(X_s, \F_s) \]
for $s \in S$. The main goal is the following theorem:

\begin{theorem}[Base Change]
Suppose that $\theta^i_{s_0}$ is surjective for some $s_0 \in S$ (with fixed $i$). Then,
\begin{enumerate}
\item The map $\theta^i_{s_0}$ is an isomorphism and there is an open $U \subset S$ neighborhood of $s_0$ such that $\theta^i_s$ is an isomorphism for all $s \in U$.
\item $\theta^{i-1}_{s_0}$ is surjective if and only if $R^i f_* (\F)_{s_0}$ is a free $\mathcal{O}_{s_0}$-free module (iff $R^i f_* (\F)$ is locally free at $s_0$ by coherence).
\end{enumerate}
When both $\theta^i_s$ and $\theta^{i-1}_s$ are surjective for all $s \in U$ then $R^i f_*(\F)|_U$ commutes with arbitrary base change (for any open $\Spec{A} \subset U$ we have $H^i(X_A, \F_A) \ot_A A' \iso H^i(X_{A'}, \F_{A'})$ for any $A$-algebra $A'$.
\end{theorem} 

\begin{corollary}
Extremely useful example [EGAIII2, 7] Assume $H^1(X_{s_0}, \F_{s_0}) = 0$. By (1) then $R^1 f_* (\F)_{s_0} = 0$ by Nakayama so $R^1 f_*(\F)|_{U} = 0$ on some open neighborhood $U$ of $s_0$. Thus this is locally free so by (2) we know that $\theta^0_s$ is surjective for all $s \in U$ and hence an isomorphism. By by definition, $\theta^{-1}_s$ is always surjective and thus we conclude that $f_* (\F)|_U$ is locally free and commutes with all base change.  
\end{corollary}

\subsection{Setup for the Proof of Cohomology and Base Change}

For the proof of the base change theorem, we may replace $S = \Spec{A}$ for a noetherian ring $A$ since everything is local on the base. Now we have $R^i f_* (\F) = \wt{H^i(X, \F)}$ which is coherent. To understand these groups we choose a finite affine open cover $\U = \{ U_i \}$ of $X$ such that,
\[ H^i(X, \F) = H^i(C^\bullet) \quad \text{ where} \quad C^\bullet = C^\bullet(\U, \F) \]
for the ordered Cech cover. Notice that $C^\bullet$ is a bounded complex and each term is comprised of a finite product of groups of the form $\Gamma(U_{\underline{i}}, \F))$ which are flat $A$-modules by the assumption that $\F$ is flat. Hence each $C^i$ is $A$-flat because it is a finite product of flat $A$-modules. Furthermore, by properness, all $H^i(C^\bullet)$ are finite $A$-modules (although the terms $C^\bullet$ are very much not finite). 
\bigskip\\
Our goal is to understand when $H^i(C^\bullet)$ commutes with $- \ot_A A'$ (e.g. $A' = k(s) = A / \m$). Explicitly, what we mean is, when is the natural map,
\[ A' \ot_A H^i(C^\bullet) \to H^i(A' \ot_A C^\bullet) \]
an isomorphism. We will turn this into a concrete problem of free $A$-modules and matrices using the following lemma.

\begin{lemma}
Consider a bounded complex $C^\bullet$ of flat $A$-modules (say with amplitude $[0,N]$) with all $H^p(C^\bullet)$ finite. Then there exists a complex $K^\bullet$ of $A$-modules with the same amplitude as $C^\bullet$ along with a map $g : K^\bullet \to C^\bullet$ such that,
\begin{enumerate}
\item all $K^p$ are finite free for $1 \le p \le N$ and $K^0$ is finite flat over $A$ (or equivalently finite locally free as an $A$-module)
\item for any $A$-module $M$ the natural map $H^p(K^\bullet \ot_A M) \iso H^p(C^\bullet \ot M)$ is an isomorphism for all $p$. 
\end{enumerate}
\end{lemma}

\begin{proof}
Idea: descending inductive construction. Choose generators of the cohomology and map a free module onto lifts of these generators in $C^\bullet$. [H, Ch. III, Lemma 12.9] except $K^\bullet$ is not bounded below. For the truncation in degree zero, see [Mum, Ch II, Section 5, $1^{\text{st}}$ Lemma 1]. 
\end{proof}

\begin{corollary}
The map $s \mapsto \chi(X_s, \F_s)$ is locally constant.
\end{corollary}

\begin{proof}
WLOG take $S = \Spec{A}$ and get a complex $K^\bullet$ as above. Shrinking the base we may assume that all $K^\bullet$ are free. Then,
\[ H^p(X_s, \F_s) = H^p(K^\bullet(s)) \]
for $M(s) = M \ot_A k(s)$. However, the complex,
\begin{center}
\begin{tikzcd}
0 \arrow[r] & K^0(s) \arrow[r] & \cdots \arrow[r] & K^N(s) \arrow[r] & 0
\end{tikzcd}
\end{center}
is a bounded complex of finite $k(s)$-vector spaces and therefore,
\[ \sum_p (-1)^p h^p(K^\bullet(s)) = \sum_p (-1)^p \dim{K^p(S)} = \sum_p (-1) \rank_A (K^p) \]
is constant in $s$. 
\end{proof}

\begin{definition}
Say a function $\varphi : |S| \to \Z$ is \textit{upper semicontinuous} if $\{s \in S \mid \varphi(s) \ge n \}$ is closed. 
\end{definition}

\begin{remark}
Informally, this means that $\varphi$ ``jumps up along closed sets''. 
\end{remark}

\begin{corollary}[Semicontinuity]
The map $s \mapsto h^p(X_s, \F_s)$ is upper semicontinuous. 
\end{corollary}

\begin{proof}
Passing to a small enough affine $S = \Spec{A}$ we can assume that $K^\bullet$ is a complex of free $A$-modules. We have a complex,
\begin{center}
\begin{tikzcd}
0 \arrow[r] & K^0 \arrow[r, "\delta^0"] & K^1 \arrow[r, "\delta^1"] & \cdots \arrow[r] & K^N \arrow[r] & 0
\end{tikzcd}
\end{center}
where each $\delta^p$ is presented as a matrix. By design,
\[ h^p(X_s, \F_s) = \dim_{k(s)} H^p(K^\bullet(s)) = \dim{\ker{\delta^p(s)}} - \dim{\im{\delta^{p-1}(s)}} \]
By the rank-nullity theorem for $\delta^p(s)$ this is,
\[ \dim{K^p(s)} - \dim{\im{\delta^p(s)}} - \dim{\im{\delta^{p-1}(s)}} = \rank_A{K^p} - (\rank{\delta^p(s)} + \rank{\delta^{p-1}(s)}) \]
Sufficient to show that $s \mapsto \rank{\delta^i(s)}$ is lower semicontinuous (meaning $\{s \in S \mid \rank{\delta^i(s)} \le m \}$ is closed. This says that all $(m+1) \times (m+1)$ minors vanish which is a closed condition. 
\end{proof}

\begin{remark}
The preceding proof shows that if $s \mapsto h^p(X_s, \F_s)$ is \textit{constant} then the ranks $\rank{\delta^p(s)}$ and $\rank{\delta^{p-1}(s)}$ never jump (for connected $S$) because \textit{a priori} they are upper semicontinuous but their sum is constant and therefore these are constant. 
\end{remark}

\begin{lemma}[Rank Lemma]
Let $Y$ be a reduced noetherian scheme and $\varphi : \E' \to \E$ be a map of vector bundles on $Y$ (e.g. $\wt{\delta}^p ; \wt{K}^p \to \wt{K}^{p+1}$). Then the following are equivalent,
\begin{enumerate}
\item $\rank{\varphi(y)} : \xi'(y) \to \xi(y)$ is constant $r$ for $y \in Y$
\item $\varphi$ factors as $\xi' \onto \nu \embed \xi$ onto some subbundle $\nu$ (meaning $\nu \embed \xi$ is a local direct summand) with $\rank{\nu} = r$.  
\end{enumerate}
\end{lemma}

\begin{proof}
(2) $\implies$ (1) is easy. And (1) $\implies$ (2) by [Mum, Ch II, Section 5, $2^{\text{nd}}$ Lemma 2]. 
\end{proof}

\begin{remark}
Reducedness is essential. Consider $A = k[\epsilon]$ and,
\[ \varphi = 
\begin{pmatrix}
0 & 0 
\\
0 & \epsilon 
\end{pmatrix} \]
Then $\varphi(y) = 0$ with rank $0$ but $\varphi : A^{\oplus 2} \to A^{\oplus 2}$ does not have locally free image. 
\end{remark}

\begin{remark}
$\varphi$ strongly rank $r$ implies that $\ker{\varphi} \subset \xi'$ is also locally free and a subbundle and all $\ker{\varphi}$ $\im{\varphi} = \nu$ and $\coker{\varphi}$ all commute with arbitrary base change.  
\end{remark}

\begin{theorem}[Grauert]
If $s \mapsto h^p(X_s, \F_s)$ is constant and $S$ is reduced then $R^p f_*(\F)$ is locally free and commutes with every base change. 
\end{theorem}

\begin{proof}
WLOG we assume $S = \Spec{A}$ and get $K^\bullet$ with all $K^p$ finite free. In the proof of semicontinuity, we saw that $\delta^p$ and $\delta^{p-1}$ have constant fiber rank. Since $S$ is reduced, by the rank lemma, we get that $\delta^p$ and $\delta^{p-1}$ are strongly of constant rank. Therefore,
\[ \im{\delta^{p-1}} \subset \ker{\delta^p} \subset K^p \]
are each subbundles of $K^-$ so $\im{\delta^{p-1}} \embed \ker{\delta^p}$ is a subbundle inclusion so $H^p(K^\bullet)$ is locally free and commutes with any base change.  
\end{proof}

\section{Proof and Applications of Base Change Theorem}

\begin{proof}
For proper $f : X \to S$ with $S$ locally noetherian and $\F \in \Coh{X}$ flat over $S$ assume that $\theta^i_{s_0} : (R^i f_* (\F))(s_0) \to H^i(X_{s_0}, \F_{s_0})$ is surjecitve. We want to show that,
\begin{enumerate}
\item $\theta^i_{s_0}$ is an isomorphism
\item $\theta^i_s$ is surjective and hence an isomorphism for all $s$ near $s_0$.
\end{enumerate}
By shrinking our base we can assume $S = \Spec{A}$ and have $K^\bullet$ complex with all terms free universally computing cohomology. Therefore, the map in question is,
\[ \theta^i_{s_0} : H^i(K^\bullet)(s_0) \to H^i(K^\bullet(s_0)) \]
the natrual map for tensor product and cohomology. Observe that localization is exact so $H^i(K^\bullet)(s_0) = H^i(K^\bullet_{s_0}) / \m_{s_0}$. Thus $\theta^i_{s_0}$ arises from the surjective map of complexes,
\[ K^\bullet_{s_0} \onto K^\bullet(s_0) \]
The map on the $H^i$ fits into the sequence,
\begin{center}
\begin{tikzcd}
0 \arrow[r] & \im{\delta^{i+1}_{s_0}} \arrow[d, two heads, "\gamma"] \arrow[r] & \ker{\delta^i_{s_0}} \arrow[d,"\alpha"] \arrow[r] & H^i(K^\bullet_{s_0}) \arrow[d, "\beta"] \arrow[r] & 0
\\
0 \arrow[r] & \im{\delta^{i-1}(s_0)} \arrow[r] & \ker{\delta^i(s_0)} \arrow[r] & H^i(K^\bullet(s_0)) \arrow[r] & 0
\end{tikzcd}
\end{center}
then $\gamma$ is surjective because the map of complexes is surjective. Therefore, by the snake lemma $\beta$ is surjective if and only if $\alpha$ is surjective. However, $\beta$ is surjective if and only if $\theta^i_{s_0}$ is surjective because it is a factorization. Nor a miracle: surjectivity of $\alpha$ if and only if $\wt{\delta}^i : \wt{K}^i \to \wt{K}^{i+1}$ being strongly of constant rank at $s_0$. 
\end{proof}

\begin{lemma}[Strong Rank Lemma]
For $Y$ a locally ringed space, $\varphi : \E' \to \E$ a map of vector bundles on $Y$, then for $y_0 \in Y$ the following are equivalent,
\begin{enumerate}
\item $\varphi$ is strongly of constant rank near $y_0$
\item $\ker{\varphi_{y_0}} \onto \ker{\varphi(y_0)}$
\item $\coker{\varphi}$ is local free near $y_0$.
\end{enumerate}
\end{lemma}

\begin{proof}
The equivalence of (a) and (c) is an easy exercise. The rest is in the handout.
\end{proof}

\begin{proof}[Completion of the proof]
Therefore $\theta^i_{s_0}$ is surjective if and only if $\delta^i$ being strongly of constant rank over some open neighborhood $s_0 \in U$. Therefore, in this case $\theta^i_s$ is surjective for all $s \in U$ and $\delta^i|_U : K^i_U \to K^{i+1}_U$ has $\ker{\delta^i|_U} \subset K^i_U$ is locally free and a subbundle. Formation comuting with any base change on $U$. However,
\[ H^i(K^\bullet_U) = \coker{(K^{i-1}_U \to \ker{\delta^i|_U})} \]
is $H^i(K^\bullet_U)$ commmutes with any base change on $U$, so $\theta^i_s$ is an isomorphism for all $s \in U$. Now we apply the strong rank lemma to $\delta^{i-1}|_U : K^{i-1}_U \to \ker{\delta^i|_U}$ which is a map of vector bundles. The cokernel $H^i(K^\bullet|_U)$ is locally free near $s_0$ if and only if $\delta^{i-1}$ is strongly of constant rank at $s_0$ which by the previous argument is equivalent to $\theta^{i-1}_{s_0}$ being surjective.
\end{proof}

\section{Applications}

[H] gives many applications of RR to intrinsic and extrinsic geometry of smooth proper geometrically connected curves $X$ over $k$ but only when $k = \bar{k}$. Using fpqc descent, we can often reduce suitable statments over $k$ to the setting $\bar{k}$. 

\begin{theorem}
For general $k$, Weil divisor $D$ on $X$,
\[ \struct{X}(D) \text{ is ample} \iff \deg_k D > 0 \]
Furthermore, if $\deg_k D \ge 2 g + 1$ then $\struct{X}(D)$ is very ample.
\end{theorem}

\begin{proof}
In handout on RR calculations, reduce to $k = \bar{k}$ so that closed points are rational. The handout describes $g = 0$ and $g = 1$ (with a $k$-point) over any $k$. 
\end{proof}

\begin{defn}
For a locally noetherian $S$, an \textit{elliptic curve} over $S$ is a pointed relative genus $1$ curve. 
\end{defn}

\begin{prop}
Then $s \mapsto \text{genus}(X_s)$ is locally constant on $S$ using cohomology and base change for $\struct{X}$. 
\end{prop}

\begin{rmk}
It is true that $f_* (\Omega^1_{X/S})$ is a rank $r$ vector bundle commuting with base change but the proof is actually hard when $S$ is nonreduced (can't use Grauert). 
\end{rmk}

\begin{defn}
An elliptic curve over locally noetherian $S$ is a genus-$1$ curve $E \to S$ equipped with $e \in E(S)$.
\end{defn}

We want to build a commutative $S$-group law on $(E, e)$-relativizing the moedthod via divisors over fields. We'll use the invertible sheaves on $E_T$ for $T \to S$ and base change theorems. By Yoneda, to make $m : E \times E \to E$ an $S$-group with identity $e$ is suffices to build a commutative group law on $(E(T), e_T)$ for locally noetherian $T \to S$ that is functionial in $T$. 

\begin{rmk}
By Exercise B(iii) HW10, uniqueness and functoriallity of the group law on $(E, e)$ reduecs to the geometric fibers. 
\end{rmk}

For $P \in E(T)$ notice that $E(T) = E_T(T)$ so we consider $P : T \to E_T$ as a section of $E_T \to T$ and this is proper so $P$ is a closed immersion and thus $P$ has an ideal sheaf $\I_P$ that is invertible and commutes with base change (HW10, Exercise B(i)). For $\L = \I_P^{\ot - 1}$ on $E_T$ then $\L_t = \struct{E_t}(P(t))$ has degree $1$. 

\begin{thm}
Any invertible $\L$ on $E$ with $\deg_{\kappa(s)}(\L_s) = 1$ for all $s \in S$ is of the form,
\[ \I_P^{\ot - 1} \ot f^* \M \]
for a unique section $P \in E(S)$ and line bundle on the base $\M \in \Pic{S}$. 
\end{thm}

\begin{rmk}
Over a field, the above is proved using RR: $\deg_k D = 1$ and $g = 1$ implies that $D \sim P$ for a unqiue $P \in E(k)$. 
\end{rmk}

\begin{cor}
Let $\L_1 \approx \L_2$ if $\L_1 \cong \_2 \ot f^* \M$ for some $\M \in \Pic{S}$. Cosndier $P, Q \in E(T) = E_T(T)$ then get $\I_P$ and $\I_Q$ on $E_T$ therefore,
\[ \L = \I_P^{\ot - 1} \ot \I_Q^{\ot -1} \ot \I_{e_T} \]
has fiberwise degree $1$ so $\L \approx \I^{-1}_{P \oplus Q}$ for a unqiue $P \oplus Q \in E(T)$. Clearly this is commutative and $P \oplus e_T = P$. Furthermore we get,
\[ E(T) \embed \Pic{E_T} / \Pic{T} \]
sending $P \mapsto \I_P^{\ot -1} \ot \I_{e_T}$ which sends the $\oplus$ operation on $E(T)$ to the $\ot$ operation on the target. Therefore $\oplus$ is associative. Finally, we need to check that inverses exist. Consider $\L \cong \I_P \ot \I_{e_T}^{\ot -2}$ which has fiber degree $1$ and thus $\L \approx \I_{P'}^{\ot -1}$ for a unique $P' \in E(R)$. Therefore,
\[ \I_{P} \ot \I_{e_T}^{\ot -1} \approx \I_{P'}^{\ot -1} \ot \I_{e_T} \implies \I_P^{\ot -1} \ot \I_{P'}^{\ot} \ot \I_{e_T} \approx \I_{e_T}^{\ot -1} \]
and therefore $P \oplus P' = e_T$.  
\end{cor}
\end{document} 


\documentclass[12pt]{article}
\usepackage{import}
\import{../}{AlgGeoCommands}


\begin{document}

\section{Jan. 10}

\begin{defn}
A subfunctor $F \embed G$ from schemes to sets is \textit{open} if for every $\alpha \in G(X)$ there is an open $j : U \embed X$ such that for any $f : Y \to B$ such that $f^* \alpha \in F(Y)$ then $f$ factors through $U \to B$.
\end{defn}

\begin{defn}
A morphism $F_1 \to F_2$ of functors is representable if for every diagram,
\begin{center}
\begin{tikzcd}
F_1 \times_{F_2} X \pullback \arrow[d] X \arrow[r] & X \arrow[d]
\\
F_1 \arrow[r] & F_2
\end{tikzcd}
\end{center}
with $X$ a scheme then $F_1 \times_{F_2} X$ is a scheme.
\end{defn}

\begin{defn}
Let $\cP$ be a local (on the target) property of maps of schemes. We say that a representable map $F_1 \to F_2$ of functors has property $\cP$ if for any diagram,
\begin{center}
\begin{tikzcd}
F_1 \times_{F_2} X \pullback \arrow[d] X \arrow[r] & X \arrow[d]
\\
F_1 \arrow[r] & F_2
\end{tikzcd}
\end{center}
with $X$ a scheme has $F_1 \times_{F_2} X \to X$ a morphism of schemes with property $\cP$.
\end{defn}

\begin{prop}
A subfunctor being open is equivalent to the map $F \embed G$ being representable and an open immersion.
\end{prop}

\begin{proof}
A morphism $X \to G$ is equivalent to a choice of $\alpha \in G(X)$. Then the property is exactly the universal property of $U$ being the fiber product $F \times_G X$. 
\end{proof}

\begin{prop}
Let $G$ be representable. Then every open subfunctor is representable by an open subscheme.
\end{prop}

\begin{proof}
Let $F \embed G$ be an open subfunctor. By the previous lemma, it is representable and therefore since $G$ is representable $F$ is representable by taking the pullback of $G$ along $F \to G$. 
\end{proof}

\begin{prop}
The intersection of open subfunctors in an open subfunctor. 
\end{prop}

\begin{proof}
This follows from the fact that $H_1 \cap H_2 \embed F$ is the fiber product of $H_1 \to F$ and $H_2 \to F$.
\end{proof}

\begin{prop}
Assume $\F$ is a sheaf (for the Zariski topology) and $\F', \F'' \subset \F$ are representable open subfunctors. Then the sheaf union $\F' \cup \F''$ is a representable open subfunctor. 
\end{prop}

\begin{proof}
$\F' \cap \F''$ is representable because it is an open subfunctor of $\F'$ which is representable. Then the diagram,
\begin{center}
\begin{tikzcd}
F' \cap F'' \arrow[d, hook] \arrow[r, hook] & F''
\\
F'
\end{tikzcd}
\end{center}
gives a diagram of the representing objects,
\begin{center}
\begin{tikzcd}
X' \cap X'' \arrow[d, hook] \arrow[r, hook] & X''
\\
X'
\end{tikzcd}
\end{center}
where the maps are open immersions. Then we can form the gluing to get a scheme $X$. Now I claim that $X$ represents $F' \cup F''$. This is because $\Hom{}{-}{X}$ and $F' \cup F''$ are both sheaves and the natural map $(F' \cup F'') \to \Hom{}{-}{X}$ is locally an isomorphism. 
\end{proof}

\begin{defn}
We say that a functor $\F$ is a sheaf in the Zariski topology if for all schemes $X$, the presheaf on $X$ of sets $U \mapsto \F(U)$ forms a sheaf. 
\end{defn}

\begin{prop}
Representable functors are sheaves. 
\end{prop}

\begin{proof}
This is just because maps to schemes glue in the Zariski topology.
\end{proof}

\begin{thm}
Let $F$ be a sheaf that has an open cover by representable subfunctors then $F$ is a representable functor.
\end{thm}

\section{Jan. 12}

\newcommand{\Set}{\mathrm{Set}}

\begin{prop}[Yoneda]
If $\C$ is a category and $X \in \C$ and $F : \C \to \Set$ is a functor then there is a natural bijection,
\[ \Hom{}{h^X}{F} \iso F(X) \]
sending $\alpha : h^X \to F$ to $\alpha_X(\id) \in F(X)$. 
\end{prop}

\begin{prop}
An open subfunctor of a sheaf if also a sheaf.
\end{prop}

\subsection{Grassmannian Bundle}

Let $X$ be a scheme and $\E$ a vector bundle on $\E$.
Consider the functor,
\[ B \mapsto \{ \rho : B \to X \text{ and } \varphi : \rho^* \E \onto \mathcal{V} \text{ where } \mathcal{V} \text{ is a rank } k \text{ vector bundle } \} \]
This is representable because it is locally represented by $U_i \times G(k, \E|_{U_i})$ where $\{ U_i \}$ is a trivializing cover.

\subsection{Flag Variety}



\subsection{The Hilbert Scheme}

\newcommand{\sHilb}
{\underline{\Hilb}}

Consider the Hilbert Functor,
\[ \sHilb_X : B \mapsto \{ X \embed X \times B \mid X \to B \text{ is flat and finitely presented} \} \]
For $X = \P^n$ the Hilbert polynomial is locally constant function $B \to \Q[t]$ and therefore,
\[ \sHilb_{\P^n} = \bigsqcup_{P \in \Q[t]} \sHilb_{P, \P^n} \]

\section{Jan. 19}

\subsection{Hilbert Functor} 

We consider diagram,
\begin{center}
\begin{tikzcd}
X \arrow[rd] \arrow[rr, hook] & & \P^n_B \arrow[ld]
\\
& B
\end{tikzcd}
\end{center}
for $X \to B$ flat and finitely presented.

\subsection{Flatening Stratification}

Let $\F$ be a coherent sheaf on $X$ a noetherian scheme. Then $\F$ is locally-free on some dense open. Then $\rank{\F} : X \to \Z$ is a upper semi-continuous function where,
\[ \rank_{x}{(\F)} = \dim_{\kappa(x)} \F_x \ot_{\stalk{X}{x}} \kappa(x) \]

\section{Jan. 21}

\subsection{Some Motivation}

\begin{rmk}
A finitely presented sheaf is presented and hence quasi-coherent.
\end{rmk}

Suppose that $\F$ is a coherent sheaf on $X$. If $X$ is reduced then on the open $U$ where $\F$ has minimal rank (rank is upper semi-continuous so the level sets are locally closed with the minimal one open) it is a vector bundle. Thus we can get a stratification of locally closed sets on which $\F$ is a vector bundle. 

\begin{thm}
Let $M$ be a finitely presented module over a local ring $A$ then the following are equivalent:
\begin{enumerate}
\item $M$ is flat
\item $M$ is free
\item $M$ is projective.
\end{enumerate}
\end{thm}

\begin{cor}
Let $\F$ is a finitely presented quasi-coherent sheaf on a scheme $X$. Then $\F$ is flat over $X$ if and only if $\F$ is locally free.
\end{cor}

\subsection{Flatening Stratification}

\begin{thm}
Let $X$ be a scheme, and $\F$ is a finitely presented sheaf on $X$. There exist uniquely determined locally closed subschemes $U_0, U_1, U_2, \dots, \embed X$ such that for all $\pi : Y \to X$ then $\pi^* \F$ is a rank $n$ locally free sheaf on $U$ if and only if $\pi$ factors through $U_n \embed X$.
\end{thm}

\begin{cor}
Gien $\pi : Y \to X$ then $\pi^* \F$ is a vector bundle if and only if $\pi$ factor through $U_0 \sqcup U_1 \sqcup U_2 \sqcup \cdots \to X$. 
\end{cor}

\begin{proof}
We use the locally closed rank stratification. Now we need a scheme structure. Let $p \in X$ and $\stalk{X}{p}$ be the local ring $\kappa(p) \stalk{X}{p} / \m_p$. Then let $n = \rank_p(\F)$ and fix an isomorphism,
\[ \kappa(p)^{\oplus n} \iso \F_p / \m_p \F_p \]
Then we get a surjection,
\begin{center}
\begin{tikzcd}
\stalk{X}{p}^{\oplus n} \onto \F_p \arrow[r] & 0
\end{tikzcd}
\end{center}
by Nakayama's lemma. Then we lift these germs to some small affine neighbrohood $p \in U = \Spec{A} \subset X$ so that $s_1, \dots, s_n \in \F(U)$ lift to $\struct{X}(U)^{\oplus n} \to \F(U)$. Therefore, let $\F|_{\Spec{A}} = \wt{M}$ so we get an sequence,
\begin{center}
\begin{tikzcd}
0 \arrow[r] & K \arrow[r] & A^{\oplus n} \arrow[r] & M \arrow[r] & C \arrow[r] & 0 
\end{tikzcd}
\end{center}
Then $L$ is finitely generated because $A$ is finitely generated and $K$ is finitely generated because $M$ is finitely presented. Therefore, localizing at $f \in A \setminus \p$ for $\p$ the point we can kill $C$. Therefore, we get,
\begin{center}
\begin{tikzcd}
A_f^{\oplus m} \arrow[r, "\varphi"] & A^{\oplus n}_f \arrow[r] & M \arrow[r] & 0
\end{tikzcd}
\end{center}
on $\Spec{A}$. For which $A/I$ can $M$ be locally free of rank $n$. Applying $- \ot_A A / I$,
\begin{center}
\begin{tikzcd}
(A/I)^{\oplus m} \arrow[r] & (A/I)^{\oplus n} \arrow[r] & M / I M \arrow[r] & 0
\end{tikzcd}
\end{center}
so $M / IM$ is locally free of rank $n$ exactly if the matrix of the first map is zero meaning it has entires in $I$. Then let $J$ be the ideal generated by the entires of this matrix then $M/IM$ is locally free of rank $n$ if and only if $J \subset I$. For which $A \to B$ can $M \ot_A B$ be a free $B$-module of rank $n$? This is exactly if $\varphi \ot \id = 0$ which is equivalent to $s \ot 1 = 0$ in $B$ for each entry in the matrix if and only if $J B = 0$ if and only if $A \to B$ factors as $A \to A / J \to B$ proving the universal property.
\end{proof}

\section{Jan. 24}

\subsection{Hilbert Polynomials are Constant in Flat Families}

Suppose $\F$ is coherent on $X \subset \P^n_S$. Then,
\[ s \mapsto P_s \in \Q[t] \quad \text{ where } \quad (P_s(m) = \chi(X_s, \F_s(m)))\]
is locally constant on $S$ if $\F$ is flat over $S$. 

\subsection{Cohomology and Base Change}

Consider the diagram,
\begin{center}
\begin{tikzcd}
X_s \arrow[r] 
\arrow[d] \pullback & X_B \arrow[r, hook] & X \arrow[d, "\pi"] 
\\
\Spec{\kappa(s)} \arrow[r, hook] & \Spec{B} \arrow[r, hook] & S
\end{tikzcd}
\end{center}
where $\Spec{B} \embed S$ is an affine open. Then $R^i \pi_* \F |_{\Spec{B}} = \wt{H^i(X_B, \F_B)}$. We want to know how this compares with the stalk compare with $H^i(X_s, \F_s)$.
\bigskip\\
Consider,
\begin{center}
\begin{tikzcd}
X_A \pullback \arrow[d] \arrow[r] & X
\\
\Spec{A} \arrow[r] & \Spec{B}
\end{tikzcd}
\end{center}
for any map $\Spec{A} \to \Spec{B}$. Consider the \v{C}ech complex $\check{C}$ of $\F$. Then get a new complex $C \ot_B A$. The cohomology of this complex is $H^i(X_A, \F_A)$ by the FHHF theorem. 
\bigskip\\
Then there is a map,
\[ F H \to H F \]
If $F$ is an exact functor i.e. $A$ is flat over $B$ then this map is flat.
\bigskip\\
For example $\Spec{\stalk{S}{s}} \to S$ is flat and therefore,
\[ H^i(X_{\Spec{\stalk{S}{s}}}, \F_s) = (R^i \pi_* \F)_{\Spec{\stalk{S}{s}}} \] 


\subsubsection{A Corollary}

\begin{prop}
Let $\pi : X \to S$ be proper with $S$ noetherian. If $\F$ is coherent $\struct{X}$-module flat over $S$ and $R^i \pi_* \F = 0$ for $i > 0$ then $\pi_* \F$ is a vector bundle. 
\end{prop}

\begin{proof}
This is a local question so we can reduce to the case that $S = \Spec{A}$ is affine. Then we take the Cech complex then we get an exact sequence,
\begin{center}
\begin{tikzcd}
0 \arrow[r] & 0 H^0(X, \F) \arrow[r] & \prod \F(U_i) \arrow[r] & \cdots \arrow[r] & \prod \F(U_1 \cdots \cap \cdots U_n) \arrow[r] & 0
\end{tikzcd}
\end{center}
by the vanishing of higher cohomology. But the Cech terms are flat over $A$ and therefore $H^0(X, \F)$ is a finite flat $A$-module and hence locally free. Therefore $\pi_* \F = \wt{H^0(X, \F)}$ is locally free. 
\end{proof}

\begin{cor}
In the previous situation, given a diagram,
\begin{center}
\begin{tikzcd}
X_T \arrow[d, "\rho"] \arrow[r] & X \arrow[d] \arrow[d,"\pi"]
\\
T \arrow[r, "\alpha"] & S
\end{tikzcd}
\end{center}
Then $R^i \rho_* \F_T = 0$ for $i > 0$ and $\rho_* \F_T = \alpha^* \pi_* \F$ is finite locally free.
\end{cor}

\begin{proof}
We may take $S$ and $T$ be affine, say $\Spec{A} \to \Spec{B}$. Then the extnded \v{C}ech complex is a bounded exact sequence of flat $A$-modules and hence remains exact under any base change. Therefore, 
\begin{center}
\begin{tikzcd}
0 \arrow[r] & H^0(X, \F) \ot_A B \arrow[r] & \cdots \arrow[r] & 0 
\end{tikzcd}
\end{center}
Is exact giving the theorem because this is the Cech complex of $\F_T$.
\end{proof}

\begin{cor}
In the previous situation. If $R^i \pi_* \F = 0$ for $i > 0$ then $h^0(X_s, \F_s)$ is a locally constant function of $S$.
\end{cor}

\begin{proof}
It is the rank of $\pi_* \F$ which is a vector bundle and thus locally constant. This is because cohomology commutes with base change $\Spec{\kappa(s)} \to S$ by the previous proposition.
\end{proof}

\subsection{The Proof of Constancy of Hilbert Polynomials}

\begin{prop}
Let $\F$ be a coherent sheaf on $\P^n_k$. Then $P_\F$ is a polynomial.
\end{prop}

\begin{proof}
There is an exact sequence,
\begin{center}
\begin{tikzcd}
0 \arrow[r] & A \arrow[r] & \F \arrow[r] & \F(1) \arrow[r] & B \arrow[r] & 0
\end{tikzcd}
\end{center}
for some choice of a hyperplane $H$ where $A$ and $B$ have support on $H$. Taking euler characteristics we see that,
\[ P_\F(m+1) - P_\F(m) = P_A(m) + P_B(m) \]
By induction this is a polynomial.
\end{proof}

\begin{thm}
Hilbert polynomials are locally constant in flat families.
\end{thm}

\begin{proof}
Consider $\F$ coherent on $\P^n_A \to \Spec{A}$ with $A$ Noetherian. 
\end{proof}

\begin{proof}
By Serre vanishing there is some $N$ so that $H^i(\P^n_A, \F(m)) = 0$ for all $m > N$. Then $R^i \pi_* \F(m) = 0$ for $i > 0$ and thus $\pi_* \F(m)$ is a vector bundle. However, $P_{\F,s}(m) = (\pi_* \F(m))_s$ for $m \gg 0$ and is therefor locally constant. 
\end{proof}

\begin{thm}
Let $\P^n_S \to S$ and $\F$ coherent on $\P^n_S$ with $S$ locally noetherian. Then $\F$ is flat over $S$ if and only if near every point $s \in S$ then there is $N > 0$ such that $\pi_* \F(n)$ is a vector bundle on that neighborhood. 
\end{thm}

\begin{proof}
We just did the ``only if`` condition. Now we do the ``if''.
This is local on the base. Therefore we may assume that $S = \Spec{A}$ for a noetherian ring $A$. 
\bigskip\\
Consider,
\[ M = \bigoplus_{n \ge N} H^0(\P^n_S, \F(n)) \]
is a graded module over $A[x_0, \dots, x_n]$. Then $\wt{M} \cong \F$ is flat over $A$ because each $H^0(\P^n_S \F(n))$ is a flat $A$-module since,
\[ H^0(\P^n_S, \F(n)) = H^0(S, \pi_* \F(n)) \]
and $\pi_* \F(n)$ is locally free. 
\end{proof}

\begin{proof}
W
\end{proof}

\section{Jan. 28}

\section{Jan. 31}

\begin{defn}
A coherent sheaf on $\P^n_A$ is $m$-regular if,
\[ H^i(\P^n, \F(m-i)) = 0 \]
for all $i > 0$.
\end{defn}

\begin{prop}
If a hyperplane $H$ misses all the associated point of $\F$ then there is an exact sequence,
\begin{center}
\begin{tikzcd}
0 \arrow[r] & \F(-1) \arrow[r] & \F \arrow[r] & \F_H \arrow[r] & 0
\end{tikzcd}
\end{center}
\end{prop}

\begin{rmk}
If $k$ is infinite then we can always find such a hyperplane for a coherent sheaf $\F$ (there are finitely many associated points). If $k$ is finite we can always pass to $\bar{k}$ by comology and flat base change. 
\end{rmk}

\begin{prop}
Consider an exact sequence of coherent sheaves on $\P^n$,
\begin{center}
\begin{tikzcd}
0 \arrow[r] & \F_1 \arrow[r] & \F_2 \arrow[r] & \F_3 \arrow[r] & 0
\end{tikzcd}
\end{center}
Then,
\begin{enumerate}
\item if $\F_1$ and $\F_3$ are $m$-regular then $\F_2$ is $m$-regular
\item if $\F_1$ is $(m-1)$-regular and $\F_2$ is $m$-regular then $\F_3$ is $m$-regular
\item if $\F_2$ is $m$-regular and $\F_3$ is $(m+1)$-regular then $\F_3$ is $m$-regular. 
\end{enumerate}
\end{prop}

\begin{proof}
Use the LES.
\end{proof}

\begin{prop}
If $\F$ is $m$-regular then $\F$ is $(m+1)$-regular.
\end{prop}

\begin{proof}
We prodceed by induction on $n = \dim{\P^n}$. If $n = 0$ then everything is $m$-regular for all $m$. Now we do the inductive step. Suppose that $\F$ is $m$-regular on $\P^{n+1}$. Then choose a hyperplane such that there is an exact sequence,
\begin{center}
\begin{tikzcd}
0 \arrow[r] & \F(-1) \arrow[r] & \F \arrow[r] & \F|_H \arrow[r] & 0
\end{tikzcd}
\end{center}
We see that $\F$ is $m$-regular and $\F(-1)$ is $(m+1)$-regular and therefore $\F|_H$ is $m$-regular. By the induction step, $\F_H$ is $(m+1)$-regular and hence $\F$ is $(m+1)$-regular concluding the proof. 
\end{proof}

\begin{prop}
If $\F$ is $m$-regular then,
\[ H^0(\P^n, \struct{}(1)) \ot H^0(\P^n, \F(r)) \xrightarrow{\mu} H^0(\F(r+1)) \]
is surjective for all $r \ge m$.
\end{prop}

\begin{proof}
We proceed by induction on $n$. For $n = 0$ this is obvious since $\struct{}(1) = \struct{}$. Then we proceed by induction. Consider a hyperplane $H \subset \P^n$ such that,
\begin{center}
\begin{tikzcd}
0 \arrow[r] & \F(-1) \arrow[r] & \F \arrow[r] & \F|_H \arrow[r] & 0
\end{tikzcd}
\end{center}
is exact. Then we get a diagram,
\begin{center}
\begin{tikzcd}
H^0,(\P^n, \struct{}(1)) \ot H^0(\P^n, \F(r)) \arrow[r, "\mu"] \arrow[d] & H^0(\P^n, \F(r+1)) \arrow[d]
\\
H^0,(H, \struct{H}(1)) \ot H^0(H, \F(r)) \arrow[r, "\mu"] & H^0(H, \F(r+1)) \arrow[d]
\\
& H^1(\P^n, \F(r))
\end{tikzcd}
\end{center}
But $\F$ is $m$-regular hence $r$-regular since $r \ge m$ and therefore $H^1(\P^n, \F(r)) = H^1(\P^n, \F(r-1)) = 0$. Hence, the downward maps are surjective. Furthermore, by the induction hypothesis the bottom map is surjective. Then the top map is surjective as well proving the claim.
\end{proof}

\begin{prop}
Let $\F$ be $m$-regular on $\P^n$. Then $\F(m)$ is generated by global sections.
\end{prop}

\begin{proof}
We use the following lemma to show that,
\[ H^0(\P^n, \F(r)) \ot \struct{\P^n} \to \F(r) \]
is surjective. By taking the kernel we take a left exact sequence,
\begin{center}
\begin{tikzcd}
0 \arrow[r] & \G \arrow[r] H^0(\P^n, \F(r)) \ot \struct{\P^n} \arrow[r] & \F(r) 
\end{tikzcd}
\end{center}
to show this is short exact it suffices to show that,
\begin{center}
\begin{tikzcd}
H^0(\P^n, \F(r)) \ot H^0(\P^n, \struct{\P^n}(s)) \arrow[r] & H^0(\P^n, \F(r+s)) 
\end{tikzcd}
\end{center}
is surjective for $s \gg 0$ which we proved in the previous lemma for \textit{all} $s \ge 0$. 
\end{proof}

\begin{lemma}
A sequence of coherent sheaves on $\P^n$,
\begin{center}
\begin{tikzcd}
0 \arrow[r] & \F_1 \arrow[r] & \F_2 \arrow[r] & \F_3 \arrow[r] & 0
\end{tikzcd}
\end{center}
is an exact sequence of sheaves if and only if,
\begin{center}
\begin{tikzcd}
0 \arrow[r] & H^0(\P^n, \F_1) \arrow[r] & H^0(\P^n, \F_2) \arrow[r] & H^0(\P^n, \F_3) \arrow[r] & 0
\end{tikzcd}
\end{center}
is an exact sequence for all sufficiently large $r \gg 0$.
\end{lemma}

\begin{proof}
One direction is Serre vanishing. The other direction follows from the fact that if $H^0(\P^n, \F(s)) = 0$ for all sufficiently large $s$ then $\F = 0$ because $\F$ is generated by global sections after some twist. 
\end{proof}

\subsection{Big Theorem}

\begin{thm}
There is some $m = m(n, r, p)$ such that for any coherent sheaf $\F \subset \struct{\P^n}^{\oplus r}$ with Hilbert polynomial $p \in \Q[t]$ such that $\F$ is $m$-regular.
\end{thm}

\begin{proof}
We proceed by induction on $n$. For $n = 0$ the case $m = 0$ works. Now we assume the induction hypothesis. Consider,
\begin{center}
\begin{tikzcd}
0 \arrow[r] & \F \arrow[r] & \struct{}^{\oplus r} \arrow[r] & \G \arrow[r] & 0
\end{tikzcd}
\end{center}
Then choose a hyperplane $H$ (after replacing $k$ by $\bar{k}$) missing the associated points of $\G$. Then consider,
\begin{center}
\begin{tikzcd}
0 \arrow[r] & \F(t-1) \arrow[r] & \F(t) \arrow[r] & \F|_H(t) \arrow[r] & 0 
\end{tikzcd}
\end{center}
Towards induction, consider,
\begin{center}
\begin{tikzcd}
\shTor{}{1}{\struct{H}}{\G} \arrow[r] & \F|_H \arrow[r] & \struct{H}^{\oplus r} \arrow[r] & \G|_H \arrow[r] & 0
\end{tikzcd}
\end{center}
but $H$ misses all the associated points of $\G$ and therefore $\shTor{}{1}{\struct{H}}{\G} = 0$ so we can apply the induction hypothesis because $\F_H \embed \struct{H}^{\oplus r}$.
The Hilbert polynomial of $\F|_H$ is,
\[ \chi(\F|_H(t)) = \chi(\F(t)) - \chi(\F(t-1)) = p(t) - p(t-1) \]
is a polynomial of degree one less. Thus $\F|_H$ is $m'$-regular for some $m' = m'(n, r, p)$ by the induction step. Then for $m \ge m' - 1$ we get an exact sequence,
\begin{center}
\begin{tikzcd}
0 \arrow[r] & \F(m-1) \arrow[r] & \F(m) \arrow[r] & \F|_H(m) \arrow[r] & 0
\end{tikzcd}
\end{center}
The long exact sequence gives,
\begin{center}
\begin{tikzcd}
H^{s-1}(H, \F|_H(m)) \arrow[r] & H^s(\P^n, \F(m-1)) \arrow[r] & H^s(\P^n, \F(m)) \arrow[r] & H^s(\F|_H(m)) 
\end{tikzcd}
\end{center}
but $H^{s-1}(H, \F|_H(m)) = H^s(\F|_H(m)) = 0$ for $s \ge 2$. The middle is an isomorphism,
\[  H^s(\P^n, \F(m-1)) \iso H^s(\P^n, \F(m)) \] and $H^s(\P^n, \F(m)) = 0$ for $m \gg 0$ so the shifting gives $H^s(\P^n, \F(m)) = 0$ for all $m \ge m' - 1$ as long as $s \ge 2$. Therefore we just need to consider the case $s = 1$. Consider, the long exact sequence,
\begin{center}
\begin{tikzcd}
H^0(\P^n, \F(m)) \arrow[r] & H^0(\F|_H(m)) \arrow[r] & H^1(\P^n, \F(m-1)) \arrow[r] & H^1(\F(m)) \arrow[r] & 0
\end{tikzcd}
\end{center}
Then we get,
\begin{center}
\begin{tikzcd}
H^0(\P^n, \F(m-1)) \arrow[d] \arrow[r, "\alpha_{m-1}"] & H^0(H, \F|_H(m-1)) \arrow[d, two heads] \arrow[r] & H^1(\P^n, \F(m-1)) 
\\
H^0(\P^n, \F(m)) \arrow[r, "\alpha_m"] & H^0(H, \F|_H(m)) \arrow[r] & H^1(\P^n, \F(m-1)) 
\end{tikzcd}
\end{center}
Therefore, if $\alpha_{m-1}$ is surjective then $\alpha_m$ is also surjective. If $H^1(\P^n, \F(m-1)) \onto H^1(\P^n, \F(m))$ is an isomorphism then $\alpha_{m-1}$ is an isomorphism and hence they are always isomorphic so by Serre vanishing $H^0(\P^n, \F(m-1)) = 0$. Hence, if $\alpha_{m-1}$ then $H^0(\P^n, \F(m-1)$ to $H^0(\P^n, \F(m))$ must strictly decrease in dimension. However,
\[ p(m') = h^0(\F(m')) - h^1(\F(m')) \]
because the higher cohomology vanishes already. Therefore,
\[ h^1(\F(m')) = h^0(\F(m')) - p(m') \le h^0(\struct{}^{\oplus r}) - p(m') = r - p(m') \]
Therefore $\alpha_m$ can only fail to be surjective for $r - p(m')$ steps so we win. 
\end{proof}

\section{Feb. 2}

\begin{thm}[Grauert]
Let $f : X \to Y$ be proper with $Y$ noetherian and $\F$ coherent and flat over $Y$. If $q \mapsto h^p(X_q, \F|_{X_q})$ is a locally constant function, and $Y$ is \textit{reduced} then $R^p \pi_* \F$ is a vector bundle and cohomology commutes with any base change of $Y$.
\end{thm}

\begin{thm}[Cohomology and Base Change Theorem]
Let $\pi : X \to Y$ be proper and $\F$ coherent and flat over $Y$.
Consider the map $\phi^p_q : (R^p \pi_* \F)|_q \to H^p(X_q, \F |_{X_q})$. Then if $\phi_q^p$ is surjective, then there is some neighborhood $U \subset U$ of $q$ such that,
\begin{enumerate}
\item if $\phi_q^p$ is surjective, then there is some neighborhood $U \subset U$ of $q$ such that, cohomology commutes with any base change to $U$
\item $R^p \pi_* \F$ is locally-free at $q$ if and only if $\phi^{p-1}_q$ is surjective.
\end{enumerate}
\end{thm}

\section{Feb. 4}

\subsection{The Mumford Complex}

Situation $\pi : X \to Y = \Spec{V}$ is proper $Y$ noetherian and $\F$ is coherent on $X$ and flat over $Y$. Then there is a bounded above complex of finitely-generated free $B$-modules that universally gives the cohomology of $\F$. You want to understand $R^p \pi_* \F$ and how it behaves under base change? This will be related to being strongly of constant rank. 

\subsection{Grauert's Theorem}

If $q \mapsto h^p(X_q, \F|_{X_q})$ is a locally constant function and $Y$ is reduced, then $R^p \pi_* \F$ is locally free and all cohomology and base change maps are isomorphisms.

\subsection{The Main Cohomology and Base Change Theorem}

If $\phi^p_q : (R^p \pi_* \F)|_q \to H^p(X_q, \F|_{X_q}$ is surjectivem then
\begin{enumerate}
\item there is some open $U \subset Y$ neighbrohood of $q$ so that cohomology commutes with any base change to $U$ (in particular $\phi^p_q$ is an isomorphism)
\item Furthermore, $\phi^{p-q}_q$ is urjective if and only if $R^p \pi_* \F$ is locally free at $q$. 
\end{enumerate}

\subsection{Constant Rank Maps}



\begin{defn}
Let $f : \E \to \F$ be a morphism of locally free sheaves on $X$. We say,
\begin{enumerate}
\item $f$ is \textit{weakly of constant rank} $a$ if for every $x \in X$ the map $f|_x : \E|_x \to \F|_x$ has rank $a$
\item strongly of constant rank $a$ if for every $x \in X$ there is an open $U$ neighborhood $x$ with a diagram,
\begin{center}
\begin{tikzcd}
\E|_U \arrow[d, "\sim"] \arrow[rr] & & \F|_U \arrow[d, "\sim"]
\\ 
\struct{U}^{\oplus (a + b)} \arrow[rr] & & \struct{U}^{\oplus (a + c)}
\end{tikzcd}
\end{center}
where the bottom map is the standard projection of rank $a$.
\end{enumerate}
\end{defn}

\begin{rmk}
Because these sheaves are finitely-presented, it suffices to check this condition at a stalk.
\end{rmk}

\begin{lemma}
If $f : \E \to \F$ is strongly of constant rank at $x$ then
\begin{enumerate}
\item $f$ is weakly of constant rank on a neighborhood of $x$
\item $\ker{f}$ and $\coker{f}$ are locally free at $x$
\item the sequence
\begin{center}
\begin{tikzcd}
0 \arrow[r] & (\ker{f})|_x \arrow[r] & \E|_x \arrow[r] & \F|_x \arrow[r] & (\coker{f})|_x \arrow[r] & 0
\end{tikzcd}
\end{center}
is exact.
\end{enumerate} 
\end{lemma}

\begin{proof}
This follows from computing the kernel and cokernel,
\begin{center}
\begin{tikzcd}
0 \arrow[r] & \struct{U}^{\oplus b} \arrow[r] & \struct{U}^{\oplus (a+b)} \arrow[r] & \struct{U}^{\oplus (a + c)} \arrow[r] & \struct{U}^{\oplus c} \arrow[r] & 0
\end{tikzcd}
\end{center}
of the standard projection of free modules. Therefore the kernel and cokernel are locally free on this neighborhood of $x$. Furthermore, an exact sequence of free modules stays exact after applying tensor product proving the last two properties.
\end{proof}

\begin{lemma}
The following are equivalent,
\begin{enumerate}
\item $f : \E \to \F$ is strongly of constant rank at $x$
\item $\coker{f}$ is locally free at $x$
\item $(\ker{f})|_x \to \ker{f|_x}$ is surjective
\end{enumerate}
Furthermore, if $X$ is reduced then the following is also equivalent,
\begin{enumerate}
\item[(d)] $f$ is weakly of constant rank in an open neighborhood of $x$.
\end{enumerate}
\end{lemma}

\begin{proof}
If $f : \E \to \F$ is strongly of constant rank at $x$ we showed that (b) and (c) hold above. Therefore, it suffices to prove that (b) implies (a) and (c) implies (b). 
\bigskip\\
Suppose that $\coker{f}$ is locally-free at $x$. Then the sequence,
\begin{center}
\begin{tikzcd}
0 \arrow[r] & \im{f} \arrow[r] & \F \arrow[r] & \coker{f} \arrow[r] & 0
\end{tikzcd}
\end{center}
splits at $x$ proving that $f$ is strongly of constant rank at $x$.
\bigskip\\
Now assume that $(\ker{f})|_x \to \ker{f|_x}$ is surjective. Localizing $\F_x$ and $\G_x$ are free $\stalk{X}{x}$-modules. We split the exact sequence into,
\begin{center}
\begin{tikzcd}
0 \arrow[r] & K \arrow[r] & \F_x \arrow[r] & I \arrow[r] & 0
\\
0 \arrow[r] & I \arrow[r] & \G_x \arrow[r] & C \arrow[r] & 0
\end{tikzcd}
\end{center}
it suffices to prove that $C$ is flat since it is finitely presented and hence it would be locally free. Applying $- \ot \kappa(x)$ we find,
\begin{center}
\begin{tikzcd}
K \ot \kappa(x) \arrow[r] & \F|_x \arrow[r] & I \ot \kappa(x) \arrow[d, equals] \arrow[r] & 0
\\
& & I \ot \kappa(x) \arrow[r] & \G|_x \arrow[r] & C \ot \kappa(x) \arrow[r] & 0
\end{tikzcd}
\end{center}
However, $K \ot \kappa(x) \onto \ker{(\F|_x \to \G|_x)}$ and therefore if $s \in I \ot \kappa(x)$ maps to zero then any lift $\bar{s} \in \F|_x$ is in $\ker{(\F|_x \to \G|_x)}$ and hence comes from $K \ot \kappa(x) \to \F|_x$ so $s = 0$. Therefore $I \ot \kappa(x) \to \G|_x$ is injective. Since $\G|_x$ is flat we see that $\Tor{}{1}{C}{\kappa(x)} = 0$. Thus by the local criterion for flatness, $C$ is flat so we conclude.
\bigskip\\
Finally, it is clear that (a) implies (d). Consider the exact sequence,
\begin{center}
\begin{tikzcd}
\E|_x \arrow[r] & \F|_x \arrow[r] & (\coker{f})|_x \arrow[r] & 0
\end{tikzcd}
\end{center}
and $f$ has constant rank on a neighborhood of $x$ so $\coker{f}$ has constant rank on a neighborhood of $x$ but $X$ is reduced and $\coker{f}$ is finitely presented so $\coker{f}$ is locally free at $x$.
\end{proof}

\subsection{Proof of Grauert's Theorem}

Suppose that $q \mapsto h^p(X_q, \F|_{X_q})$ is a locally constant function and $Y$ is reduced. 
\bigskip\\
Consider the Mumford complex at $p$, 
\begin{center}
\begin{tikzcd}
K^{p-1} \arrow[r, "\alpha"] & K^p \arrow[r, "\beta"] & K^{p+1}
\end{tikzcd}
\end{center}
Then,
\[ h^p(X_q, \F|_{X_q}) = \rank_q{K^p} - \rank_q{\alpha} - \rank_q{\beta} \]
Since the ranks are lower semi-continuous and $h^p(X_q, \F_{X_q})$ is constant we see that $\rank_q{\alpha}$ and $\rank_q{\beta}$ are constant and therefore $\alpha$ and $\beta$ are strongly of constant rank because $Y$ is reduced. Therefore, in the sequences,
\begin{center}
\begin{tikzcd}
0 \arrow[r] & \ker{\alpha} \arrow[r] & K^{p-1} \arrow[r, "\alpha"] & K^p \arrow[r] & \coker{\alpha} \arrow[r] & 0
\\
0 \arrow[r] & \ker{\beta} \arrow[r] & K^{p} \arrow[r, "\beta"] & K^{p+1} \arrow[r] & \coker{\beta} \arrow[r] & 0
\\
0 \arrow[r] & H^p \arrow[r] & \coker{\alpha} \arrow[r] & \ker{\beta} \arrow[r] & 0
\end{tikzcd}
\end{center}
every term is flat and finitely presented (because they are direct summands) and hence finitely locally free. Furthermore, because each term is flat, these exact sequences are preserved under base change. 

\subsection{Proof of Cohomology and Base Change} 

\begin{lemma}
Consider a morphism of complexes,
\begin{center}
\begin{tikzcd}
K^{p-1} \arrow[d, two heads] \arrow[r, "\delta_K^{p-1}"] & K^p \arrow[d] \arrow[r, "\delta_K^{p}"] & K^{p+1} \arrow[d] 
\\
J^{p-1} \arrow[r, "\delta_J^{p-1}"] & J^p \arrow[r, "\delta_J^{p}"] & J^{p+1} 
\end{tikzcd}
\end{center}
surjective at $p-1$ term. Then the map on cohomology $H^p(K^\bullet) \to H^p(J^\bullet)$ is surjective if and only if $\ker{\delta_K^p} \to \ker{\delta_J^p}$ is surjective.
\end{lemma}

\begin{proof}
Consider the morphism of short exact sequences,
\begin{center}
\begin{tikzcd}
0 \arrow[r] & \im{\delta_K^{p-1}} \arrow[d] \arrow[r] & \ker{\delta_K^p} \arrow[d] \arrow[r] & H^p(K^\bullet) \arrow[d] \arrow[r] & 0 
\\
0 \arrow[r] & \im{\delta_J^{p-1}} \arrow[r] & \ker{\delta_J^p} \arrow[r] & H^p(J^\bullet) \arrow[r] & 0 
\end{tikzcd}
\end{center}
Then by the snake lemma we get an exact sequence,
\begin{center}
\begin{tikzcd}
\coker{(\im{\delta_K^{p-1}} \to \im{\delta_J^{p-1}})} \arrow[r] & \coker{(\ker{\delta_K^p} \to \ker{\delta_J^p})} \arrow[r] & \coker{(H^p(K^\bullet) \to H^p(J^\bullet))} \arrow[r] & 0
\end{tikzcd}
\end{center}
I claim that $\im{\delta_K^{p-1}} \to \im{\delta_J^{p-1}}$ is surjective. Indeed, if $y \in \im{\delta_J^{p-1}}$ then choose $x \in J^{p-1}$ mapping to $y$.  By surjectivity there is some $x' \in K^{p-1}$ mapping to $x$. Then $x' \mapsto y$ along $K^{p-1} \to K^p \to J^p$ proving the surjectivity. Therefore,
\[ \coker{(\ker{\delta_K^p} \to \ker{\delta_J^p})} \cong \coker{(H^p(K^\bullet) \to H^p(J^\bullet))} \]
proving the claim.
\end{proof}

\begin{thm}[Cohomology and Base Change]
Let $\pi : X \to Y$ be a proper morphism with $Y$ noetherian. Let $\F$ be a coherent $\struct{X}$-module flat over $Y$. Suppose that,
\[ \phi^p_s : (R^p \pi_* \F)|_s \to H^p(X_s, \F_{X_s}) \]
is surjective then:
\begin{enumerate}
\item There is some neighborhood $U$ of $s \in Y$ such that for all $T \to U$ the morphism $\phi^p_T$ is an isomorphism on $\F$ (in particular $\phi^p_s$ is an isomorphism.
\item Furthermore, $\phi^{p-1}_s$ is surjective if and only if $R^p \pi_* \F$ is locally free at $s$.
\end{enumerate}
\end{thm}

\begin{proof}
This is local on $Y$ so we consider $Y = \Spec{A}$. Consider the Mumford complex and the surjective morphism of complexes,
\begin{center}
\begin{tikzcd}
K^{p-1} \arrow[r] \arrow[d, two heads] & K^p \arrow[r, "\delta^p"] \arrow[d, two heads] & K^{p+1} \arrow[d, two heads]
\\
K^{p-1} \ot_A \kappa(s) \arrow[r] & K^p \ot_A \kappa(s) \arrow[r, "\delta^p \ot \id_\kappa"] & K^p \ot_A \kappa(s)
\end{tikzcd}
\end{center}
The lemma shows that surjectivity of the cohomology\footnote{The left is the cohomology of the top complex after applying $(-)_s \ot \kappa(s)$ which does not change the surjectivity of the map.},
\[ \phi^p_s : (R^p \pi_* \F)|_s \to H^p(X_s, \F_{X_s}) \]
is equivalent to surjecitivty of
\[ (\ker{\delta^p}) \ot \kappa \to \ker{(\delta^p \ot \kappa)} \]
which is equivalent to $\delta^p$ being strongly of constant rank at $s$ menaing on some open neiborhood $U$. Thus, under this assumption, $\ker{\delta^p}$ commutes with base change to $U$ and thus from,
\begin{center}
\begin{tikzcd}
K^{p-1} \arrow[r] & \ker{\delta^p} \arrow[r] & H^p(K^\bullet) \arrow[r] & 0
\end{tikzcd}
\end{center}
we see that cohomology commutes with base change to $U$.
\bigskip\\
Now we assume that $\phi^p_s$ is surjective so $\delta^p$ is strongly of constant rank at $s$. Hence $\ker{\delta^p}$ is locally free at $s$. Therefore, 
\[ R^p \pi_* \F = \wt{H^{p}(K^\bullet)} = \coker{(K^{p-1} \to \ker{\delta^p})} \]
is locally-free if and only if $K^{p-1} \to \ker{\delta^p}$ is strongly of constant rank if and only if\footnote{Because we already know that $\ker{\delta^p} \to K^p$ is strongly of constant rank at $s$} $\delta^{p-1} : K^{p-1} \to K^p$ is strongly of constant rank at $s$ if and only if $\phi^{p-1}_s$ is surjective.
\end{proof}

\section{Feb. 9}

\subsection{The Quot Scheme}

\newcommand{\uQuot}{\underline{\Quot}}

\begin{defn}
The Quot functor $\Quot_{\F/X/S}$ for $\pi : X \to S$ and a coherent sheaf $\F$ on $X$ is,
\[ T \mapsto \{ \F \onto \sQ \mid \sQ \text{ is flat over } S \} \]
\end{defn}

\begin{rmk}
For $\F = \struct{X}$ we see that $\Quot_{\struct{X}/X/S} = \Hilb_{X/S}$. For $X = S$ and $\F$ locally we see that $\Quot_{\F/X/X}$ is the union of Grasmannians.
\end{rmk}

\begin{rmk}
There is a decompositon,
\[ \Quot_{\F/X/S} = \coprod_{\Phi \in \Q[\lambda]} \Quot_{\F/X/S}^{\Phi, \L} \]
for a relatively ample line bundle $\L$ on $X$.
\end{rmk}

\begin{rmk}
It suffices to prove representability for $X = \P^n_S$ and $S = \Spec{A}$ and $\F = \struct{X}^{\oplus r}$ so we specialize to this case now. 
\end{rmk}

\begin{thm}
The functor $\Quot_{n.r}^{p}$ for $\F = \struct{X}^{\oplus r}$ and $X = \P^n_A$ fixing $n$ and $r$ and $p \in \Q[\lambda]$ is representable by a projective scheme.
\end{thm}

\begin{proof}
Over a field $k$ suppose we have an object,
\begin{center}
\begin{tikzcd}
0 \arrow[r] & \E \arrow[r] & \struct{X}^{\oplus r} \arrow[r] & \F \arrow[r] & 0
\end{tikzcd}
\end{center}
where $\E$ and $\F$ have known Hilbert polynomial then we know that there is a uniform bound $M$ on $m$ such that $\F$ and $\E$ are $m$-regular. Therefore for $m \ge M$ we should consider sequences,
\begin{center}
\begin{tikzcd}
0 \arrow[r] & \E(m) \arrow[r] & \struct{}(m)^{\oplus r} \arrow[r] & \F(m) \arrow[r] & 0
\end{tikzcd}
\end{center}
For each $q \in B$ we there is no higher cohomology. Therefore,  
\[ h^0(s, \F(m)) = p_\F(m) \]
where $p_\F \in \Q[\lambda]$ is the fixed Hilbert polynomial. Consider the higher pushforwards,
\begin{center}
\begin{tikzcd}
0 \arrow[r] & \pi_* \E(m) \arrow[r] & \pi_* \struct{}(m)^{\oplus r} \arrow[r] & \pi_* \F(m) \arrow[r] & 0
\end{tikzcd}
\end{center}
We need to justify why $R^1 \pi_* \E(m)$. By cohomology and base change this is zero because $\E$ is flat (since the second two terms are flat) and each $H^0(X_s, \E_s(m)) = 0$. Furthermore, the zeroth pushforwards are locally free of rank equal to their Hilbert polynomials evaluated at $m$. Therefore, we get an element of the Grasmanian $\Spec{B} \to G$ where $G$ is the Grassmanian parametrizing quotients of the free vector bundle $\pi_* \struct{}(m)^{\oplus r}$ of rank $p_\F(m)$. Therefore, we get a morphism,
\[ \uQuot \to G \]
We need to show this is a closed embedding. First we want to recover the sequence,
\begin{center}
\begin{tikzcd}
0 \arrow[r] & \E \arrow[r] & \struct{}(m)^{\oplus r} \arrow[r] & \F \arrow[r] & 0
\end{tikzcd}
\end{center} 
from the Grassmanian information. It suffices to recover $\E \embed \struct{}(m)^{\oplus r}$ which is the information of a graded submodule $N \embed B[x_0, \dots, x_n]^{\oplus r}$. We define $N_m \embed (B[x_0, \dots, x_n]^{\oplus r})_m$,
\[ N_m = H^0(\pi_* \E(m)) \]
Then we consider the submodule generated by $N_m$. By MC-regularity this generates the entire submodule in sufficiently large degrees [Prop 7.0.6]. Therefore $\uQuot \to G$ is injective as a map of functors. 
\bigskip\\
Now we want to show that $\uQuot \to G$ is a locally closed embedding. Given,
\begin{center}
\begin{tikzcd}
0 \arrow[r] & V \arrow[r] & \struct{}^{\oplus N} \arrow[r] & W \arrow[r] & 0 
\end{tikzcd}
\end{center} 
how can we tell if it is in the image? We consider the universal such sequence over $G$. Consider the construction of the module $N$ which works in general, does the submodule $\E(m) \embed \struct{}(m)^{\oplus r}$ generated have flat quotient? We produce this quotient universally on $\P^n_G$. Therefore, we need the locus where the quotient $\F$ is flat over $G$ with Hilbert polynomial $p_\F$. This is the flattening stratification. This is locally closed subscheme so we get that $\uQuot \to G$ is a locally closed embedding and hence $\uQuot$ is representable by a quasi-projective scheme. Now we need to show projectivity. 
\end{proof}

\section{Feb. 16}

\begin{thm}
Given $\pi : X \to Y$ of projective flat locally noetherian $S$-schemes there is an open subscheme $U \subset S$ such that $T \to S$ factors through $U$ if and only if $X_T \to Y_T$ is an isomorphism. 
\end{thm}

\subsection{Flatness Facts}

\begin{prop}
If $\pi : X \to Y$ is a flat finite type map of locally noetherian schemes then $\pi$ is open.
\end{prop}

\begin{prop}
If $\pi : X \to Y$ is a finite type morphism fo locally noetherian schemes then the flat locus on $X$ is open. 
\end{prop}

\begin{prop}
The morphism $\pi : X \to Y$ is finite type and locally noetherian with $X$ and $Y$ flat over $S$ then $\pi$ is flat at $x$ if and only if $\pi_{s} : X_s \to Y_s$ if flat at $x$ where $x \mapsto s$.
\end{prop}

\subsection{Proof of the Theorem}

First we construct a $U$ such that $T \to S$ factors through $U$ if and only if $X_T \to Y_T$ is flat. Let $U = S \setminus \alpha(X \setminus V)$ where $V$ is the flat locus of $\pi$ and $\alpha : X \to S$ the structure map. Then $T \to S$ factors through $U$ if and only if every fiber of $X_U \to Y_U$ is flat if and only if $X_U \to Y_U$ is flat. This is because $X_t \to Y_t$ is just $X_s \to Y_s$ base changed by $\kappa(t) / \kappa(s)$ which is faithfully flat and thus flatness descends. Therefore $T \to S$ maps into the locus where the fibers are flat namely $U$.


\section{Feb. 21}

\begin{prop}
Let $\pi : X \to Y$ be proper and $\struct{}$-connected. If $\M$ is a line bundle on $Y$ then $\M \to \pi_* \pi^* \M$ is an isomorphism.
\end{prop}

\begin{proof}
This is local on $Y$ so shrink $Y$ until $\M = \struct{Y}$ and then $\struct{}$-connectedness gives by definition this isomorphism. 
\end{proof}

\begin{prop}
Let $\pi : X \to Y$ be flat, proper, geom. connected and reduced fibers (and thus universally $\struct{}$-connected). If $Y$ is reduced and noetherian and for all $q \in Y$ the line bundle $\L_q$ is trivial on $X_q$. Then $\M = \pi_* \L$ is a line bundle on $Y$ and $\pi^* \M \to \L$ is an isomorphism.
\end{prop}

\begin{proof}
Grauert's theorem (using that $H^0(X_q, \L_q) = k$ by geometric properties of the fibers and triviality of $\L_q$). Then we want to show that $\pi^* \pi_* \L \to \L$ is an isomorphism. Grauert tells us that all cohomology and base change maps commute. Since it suffices to show that $\pi^* \pi_* \L \to \L$ is an isomorphism on fibers we may assume that the base is $\Spec{k}$ and this is just the statement that $\struct{} \ot H^0(X, \L) \to \L$ is an isomorphism if $\L$ is trivial.  
\end{proof}

\begin{prop}
Let $\pi : X \to Y$ be flat, proper, fibers are geom. connected and reduced. If $\M$ is a line bundle on $Y$ and $\L = \pi^* \M$ then $\M \to \pi_* \L$ is an isomorphism.
\end{prop}

\begin{prop}
Lt $\pi : X \to Y$ be flat, proper, geom. connected and reduced fibers and $Y$ is locally noetherian. There is a best closed subscheme $Y_1 \embed Y$ and a line bundle $\M$ on $Y_1$ and an isomorphism $\pi^* \M_1 \iso \L$ on $X_1 = X \times_Y Y_1$ such that for all $f : Z \to Y$ such that if   
\end{prop}

\section{Feb.}

\section{Feb. 28}

\begin{thm}
Let $\pi : X \to Y$ be proper, flat, with geom. reduced and conncected fibers. Then the functor sending $Z \to Y$ to the isomorphism classes of line bundles $\M$ on $Z$ equipped with an isomorphism $\varphi : \pi_Z^* \M \to \L_Z$ is a locally closed subfunctor. 
\end{thm}

\begin{proof}
We so far got the set but now we need the scheme structure. Consider the locus $Y_{1/2}$ where $h^0(X_q, \L_q) = 1$. This is locally closed. Put the induced reduced structue on $Y_{1/2}$. By Grauert's theorem $X_{1/2} \to Y_{1/2}$ gives $\pi_* \L = \struct{}$. Choose some section $s$ of $\L$. Then the vanishing locus (WHAT)
\bigskip\\
We are going to locally describe the scheme structure of $Y_1$ on points of $Y_1$. We use the Mumford complex. Shink our affine open $\Spec{A}$ such that $Y_1$ is closed and shink enough that the Mumford complex of vector bundles is free so there is a complex,
\begin{center}
\begin{tikzcd}
0 \arrow[r] & A^{r_0} \arrow[r, "\phi_0"] & A^{r_1} \arrow[r] & 0 \cdots \arrow[r] & A^{r_n} \arrow[r] & 0
\end{tikzcd}
\end{center}
which computes $H^q(X, \L)$ universally. The trick is to consider the cokernel,
\begin{center}
\begin{tikzcd}
A^{r_1} \arrow[r, "\phi_0^\vee"] & A^{r_0} \arrow[r] & M \arrow[r] & 0
\end{tikzcd}
\end{center}
which applying the functor $\Hom{B}{- \ot_A B}{B}$ which takes right-exact sequences to left exact sequences we get,
\begin{center}
\begin{tikzcd}
0 \arrow[r] & \Hom{B}{M \ot_A B}{B} \arrow[r] & B^{r_0} \arrow[r] & B^{r_1}
\end{tikzcd}
\end{center}
Therefore,
\[ H^0(X_B, \L_B) = \Hom{B}{M \ot_A B}{B} = \Hom{A}{M}{B} \]
Fpr $B = A / \m$ corresponding to the point $q \in \Spec{A}$ around which this is a neighborhood we see that,
\[ H^0(X_B, \L_B) = \Hom{A}{M}{B} \]
is $1$-dimensional and therefore $M \ot_A A /\m$ is $1$-dimensional meaning we can shrink $A$ such that $M$ is generated by $1$ element so $M = A / I$. Take $B = A / J$ and consider,
\[ \Hom{A}{A/I}{A/J} = 0 \]
which happens iff $J \not\supset I$. There is a universal section for $A/I$ so we consider the closed subset where this section vanishes which does not hit the fiber over $q$ so if we remove the image of $V(s)$ then 
\end{proof}

\end{document}
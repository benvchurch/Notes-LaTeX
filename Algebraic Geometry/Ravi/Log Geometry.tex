\documentclass[12pt]{article}
\usepackage{import}
\import{../}{AlgGeoCommands}


\begin{document}

\newcommand{\cM}{\mathcal{M}}
\newcommand{\cN}{\mathcal{N}}

\section{Definitions}

\begin{defn}
A \textit{pre-log structure} on $X$ is a sheaf of commutative monoids $\cM$ on $X_{\et}$ along with a morphism of sheaves of monoids,
\[ \exp : \cM \to (\struct{X}, \times) \]
A morphism of pre-log structues $\alpha : (\cM_1, \exp_1) \to (\cM_2, \exp_2)$ is a morphism of sheaves of monoids $\alpha : \cM_1 \to \cM_2$ such that the diagram,
\begin{center}
\begin{tikzcd}
\cM_1 \arrow[rd, "\exp_1"'] \arrow[rr, "\alpha"] & & \cM_2 \arrow[dl, "\exp_2"]
\\
& \struct{X}
\end{tikzcd}
\end{center}
commutes.
\end{defn}

\begin{rmk}
From now on, when we write $\struct{X}$ in the category of monoids we mean $(\struct{X}, \times)$. 
\end{rmk}

\begin{example}
Some examples,
\begin{enumerate}
\item $\struct{X}^\times \embed \struct{X}$
\item $\struct{X} \to \struct{X}$
\end{enumerate}
\end{example}

\begin{example}
Suppose that $P$ is a commutative monoid and $\exp : P \to \Gamma(W, \struct{W})$ is a map then we get a pre-log structure $\exp : \underline{P} \to \struct{W}$ by adjunction.
\bigskip\\
For a monoid $P$ let $X_P = \Spec{k[P]}$ then we see,
\[ \Hom{}{W}{X_P} = \{\text{pre-log structures } \underline{P} \to \struct{W} \} \]
\end{example}

\begin{defn}
A pre-log structure $\exp : \cM \to \struct{X}$ is called a \textit{log structure} if $\exp^{-1}(\struct{X}^\times) \to \struct{X}^\times$ is an isomorphism. 
\end{defn}

\begin{rmk}
In particular, there is a unique morphism $\alpha : \struct{X}^\times \to \cM$ making the diagram,
\begin{center}
\begin{tikzcd}
\struct{X}^\times \arrow[rd, hook] \arrow[r, "\alpha"] & \cM \arrow[d, "\exp"] 
\\
& \struct{X}
\end{tikzcd}
\end{center}
commute so all log-structures ``lie between'' $\struct{X}^\times$ and $\struct{X}$ in the sense that $\struct{X}^\times$ is the initial object and $\struct{X}$ the terminal object of the categories of log structures.
\end{rmk}

\begin{defn}
A \textit{log scheme} $(X, \cM)$ is a scheme $X$ equipped with a log structure $\cM$.
\end{defn}

\begin{prop}
Let $Z \subset X$ be a closed subset. Let $\cM_Z \subset \struct{X}$ be the subsheaf of functions invertible on $U = X \setminus Z$. Then $\cM_Z$ is a log structure.
\end{prop}

\begin{proof}
This just says that if $f \in \struct{X}(V)$ then it is invertible on $U$ which is obvious by restriction. 
\end{proof}

\subsection{Logification}

\begin{prop}
There is a left-adjoint $\cM \mapsto \cM^{\log}$ to the forgetful functor $\{ \text{log}_X \} \to \{ \text{pre-log}_X \}$ given by,
\begin{center}
\begin{tikzcd}
\exp^{-1}{(\struct{X}^\times)} \arrow[d] \arrow[r, hook] & \cM \arrow[rdd, bend left, "\exp"] \arrow[d]
\\
\struct{X}^\times \arrow[rrd, bend right] \arrow[r] & \pushout \cM^{\log} \arrow[rd, dashed, "\exp'"] 
\\
& & \struct{X}
\end{tikzcd}
\end{center}
\end{prop}

\begin{proof}
(DO THIS!!!)
\end{proof}

\begin{defn}
Given a morphism $\pi : X \to Y$ and a pre-log structure $\exp : \cM \to \struct{Y}$ on $Y$. Then the pullback is $\pi^{-1} \cM \to \pi^{-1} \struct{Y} \to \struct{X}$. If $\cM$ is a log structure on $Y$ then the pullback is,
\[ \pi^\flat \cM = (\pi^{-1} \cM)^{\log} \]
By the universal property there is a map,
\begin{center}
\begin{tikzcd}
\pi^{-1} \cM \arrow[rr] \arrow[rd] & & \struct{X}
\\
& \pi^\flat \cM \arrow[ru]
\end{tikzcd}
\end{center}
\end{defn}

\begin{rmk}
There is a tautological log structure $\underline{P}^{\log} \to \cM_{X_P}$. In the toric case $P = \sigma^\vee \cap M$ we can define $Z = X_\sigma \setminus T$ where $T = \Spec{k[M]}$ is the torus then I claim that $\underline{P}^{\log} = \cM_Z$. (PROVE THIS!!)
\end{rmk}

\begin{rmk}
Consider a morphism $W \to X_P$ then the log structure $\underline{P}^{\log} \to \struct{W}$ is $\pi^\flat \cM_{P}$. (PROVE TIS!!)
\end{rmk}

\begin{prop}
Let $\pi : X \to Y$ be a morphism. Then the following diagram commutes,
\begin{center}
\begin{tikzcd}
\{ \text{pre-log}_Y \} \arrow[d, "\log"] \arrow[r, "\pi^{-1}"] & \{ \text{pre-log}_X \} \arrow[d, "\log"]
\\
\{ \text{log}_Y \} \arrow[r, "\pi^\flat"] & \{ \text{log}_X \}
\end{tikzcd}
\end{center}
\end{prop}

\begin{proof}
IS THIS ACTUALLY TRUE??
\end{proof}

\begin{defn}
A morphism $f : (X, \cM) \to (Y, \cN)$ of \textit{log schemes} is a morphism of schemes $f : X \to Y$ and a morphism of sheaves of monoids $\alpha$ such that,
\begin{center}
\begin{tikzcd}
\cN \arrow[r, "\alpha"] \arrow[d] & \pi_* \cM \arrow[d]
\\
\struct{Y} \arrow[r] & \pi_* \struct{X}
\end{tikzcd}
\end{center}
commutes. Equivalently this is a morphism of log structures $\alpha : \pi^\flat \cN \to \cM$ because then automatically the diagram,
\begin{center}
\begin{tikzcd}
\pi^{-1} \cN \arrow[d] \arrow[r] & \pi^\flat \cN \arrow[r] & \cM \arrow[d]
\\
\pi^{-1} \struct{Y} \arrow[rr] & & \struct{X}
\end{tikzcd}
\end{center}
commutes.
\end{defn}

\end{document}
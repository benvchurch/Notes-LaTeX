\documentclass[12pt]{article}
\usepackage{import}
\import{./}{AlgGeoCommands}

\begin{document}

\section{Introduction}

\begin{defn}
A \textit{valuation ring} is a domain $A$ contained in a field $K$ such that for all nonzero $x \in K$ either $x \in A$ or $x^{-1} \in A$.
\end{defn}

\begin{rmk}
Clearly in this case $K = \Frac{A}$.
\end{rmk}

\begin{prop}
Valuation rings are local.
\end{prop}

\begin{proof}
Let $A \subset K = \Frac{A}$ be a valuation ring and define,
\[ \m = \{ x \in A \mid x^{-1} \notin A \text{ or } x = 0 \} \]
I claim that $\m \subset A$ is an ideal. If $x \in \m$ and $y \in A$ then $y x \in \m$ since  if $(y x)^{-1} \in A$ then $(y x)^{-1} y = x^{-1} \in A$ which is not the case. Furthermore, if $x, y \in \m$ then $x + y \in A$ and suppose $(x + y)^{-1} \in A$ (if $x + y \neq 0$ in which case we are done). Now either $x/y \in A$ or $y/x \in A$, without loss of generality, take $x/y \in A$. Then $x + y = y(1 + x/y)$ and thus,
\[ y^{-1} = \frac{1 + x / y}{x + y} \in A \]
contradicting the fact that $y \in \m$. Therefore, $x + y \in \m$ so $\m$ is an ideal. Furthermore, it is clear that if $x \in A \setminus \m$ then $x \in A^\times$ so $A$ is local.
\end{proof}

\begin{defn}
Let $A, B \subset K$ be two local domains contained in a field $K$. Then we say $B$ \textit{dominates} $A$ if $A \subset B$ and $\m_B \cap A = \m_A$ i.e. $A \embed B$ is a \textit{local} extension of domains.
\end{defn}

\begin{prop}
Let $A$ be a local domain. Then $A$ is a valuation rings iff $A$ is maximal with respect to domination of local subrings of $K = \Frac{A}$.
\end{prop}

\begin{proof}
Let $B \subset K$ be a local domain dominating $A$ i.e. $A \subset B$ and $\m_B \cap A = \m_A$. Then take $x \in B \setminus A$. Since $x \in B \subset K$ we must have $x^{-1} A$ so $x^{-1} \in \m_A$ however, $\m_A = \m_B \cap A$ so $x^{-1} \in \m_B$ but $x \in B$ contradicting the fact that $\m_B$ is a maximal ideal contradicting the existence of $x \in B \setminus A$ so $B = A$. 
\bigskip\\
conversely, assume that $A \subset K = \Frac{A}$ is a local domain maximal with respect to domination. Take $x \in K$ and assume $x \notin A$ then take $A' = A[x]$ so $A' \supsetneq A$. Furthermore, for any prime $\p \subset A'$ we can take $A \subset A' \subset A'_\p \subset K$ and if $\p \cap A = \m$ this contradicts the maximality of $A$ so there are no primes of $A'$ lying above $\m$. Thus $V(\m A') = \varnothing$ so $\m A' = A'$. Therefore,
\[ 1 = \sum_{i = 0}^d t_i x^i \]
for $t_i \in \m$ and thus,
\[ (1 - t_0) (x^{-1})^d - \sum_{i = 1}^d t_i (x^{-1})^{d - i} = 0 \]
so $x^{-1}$ is integral over $A$. Thus $A'' = A[x^{-1}] \subset K$ is finite over $A$ and thus $\Spec{A'} \to \Spec{A}$ is surjective so there is an ideal $\p \subset A''$ lying over $\m$. And thus $A \subset A''_\p$ is a local extension but $A$ is a valuation ring so $A = A''_\p$ and thus $x^{-1} \in A$.
\end{proof}

\begin{prop}
Let $A \subset K$ be a local domain inside a field $K$. Then there exists a valuation ring $B$ dominating $A$ with fraction field $K$.
\end{prop}

\begin{proof}
Let $I$ be a totally ordered set and $A_i$ a totally ordered, via domination, chain of local domains indexed by $I$. Then consider,
\[ B = \bigcup_i A_i \]
I claim that $B$ is a local ring. If $x, y \in B$ then $x, y \in A_i$ for some $i \in I$ and thus $x + y, xy \in A_i \subset B$. Furthermore, let,
\[ \m = \bigcup_i \m_i \]
then for the same reason $\m$ is an ideal. Furthermore, if $x \in B \setminus \m$ then $x \in A_i \setminus \m_i$ for some $i \in I$ so $x$ is a unit in $A_i$ and thus $x$ has an inverse in $B$ so $B$ is a local ring with maximal ideal $\m$. 
By Zorn's lemma, there exists a maximal local subring of $K$ with respect ot domination containing $A$. Thus, it suffices to show that if $K \supsetneq \Frac{A}$ then $A$ is not maximal since then the maximal element will satisy $K = \Frac{A}$ and thus will be a valuation ring.
\bigskip\\
Let $A \subset K$ be a local domain with $\Frac{A} \subsetneq K$. Choose $t \in K \setminus \Frac{A}$. If $t$ is transcendental over $\Frac{A}$ then $A[t]_{(t, \m)}$ dominates $A$ otherwise $A[t]$ is finite over $A$ and thus $\Spec{A[t]} \to \Spec{A}$ is surjective so there is a prime $\p \subset A[t]$ above $\m$ and thus $A[t]_\p$ dominates $A$.
\end{proof}

\section{Valuative Criteria}

\begin{theorem}
A morphism of schemes $f : X \to S$ 
\end{theorem}

\section{Zariski-Riemann Spaces}

\begin{defn}
Let $k$ be a field and $K$ a finitely generated extension of $k$ and $A \subset K$ a sub $k$-algebra. Then consider the category $\C(K, A)$ of proper models of $K$ over $\Spec{A}$ with birational morphisms. Explicitly, $\C(K, A)$ is the category whose objects are proper $k$-varieties over $\Spec{A}$ together with a map $\Spec{K} \to X$ at the generic point inducing an isomorphism $K(X) \iso K$. Furthermore, $\Spec{K} \to X \to \Spec{A}$ corresponds to the fixed inclusion $A \embed K$. Morphisms $f : X \to Y$ are diagrams,
\begin{center}
\begin{tikzcd}
& \Spec{K} \arrow[ld] \arrow[rd]
\\
X \arrow[rr, "f"] \arrow[rd] & & Y \arrow[ld]
\\
& \Spec{A}
\end{tikzcd}
\end{center}
meaning that $f(\xi_X) = \xi_Y$ and $f$ induces an $A$-morphism $K(Y) \to K(X)$ compatible with the isomorphism $K(Y) \iso K$ and $K(X) \iso K$,
\begin{center}
\begin{tikzcd}
& K 
\\
K(Y) \arrow[ru, "\sim"] \arrow[rr, "f^*"] & & K(X) \arrow[ul, "\sim"']
\\
& A \arrow[ru] \arrow[lu]
\end{tikzcd}
\end{center}
and thus $f^* : K(Y) \to K(X)$ is an isomorphism showing that $f$ is a birational morphism.    
\bigskip\\
Now we define the Zariski Riemann space,
\[ \mathrm{ZR}(K, A) = \varprojlim_{X \in \C(K, A)} X \]
where the limit is taken in the category of locally ringed spaces.
\end{defn}

\begin{example}
Let $\trdeg{k}{K} = 1$ then $\mathrm{ZR}(K, k) = C$ where $C$ is the unique complete regular $k$-curve with function field $K$ since $\C(K, k)$ contains the single object $XC$ and no nontrivial automorphisms since any map $f : \C \to \C$ in $\C(K, k)$ must fix the function field and thus is the identity. 
\end{example}

\begin{theorem}
There is a natural identification of the points of $\mathrm{ZR}(K, A)$ with the valuation rings of $K$ containing $A$.
\end{theorem}

\begin{proof}
Consider a point $x \in \mathrm{ZR}(K, A)$ which corresponds to a point $x_i \in X_i$ in each $X_i \in \C(K, A)$ compatible with the maps. Since $X_i$ is integral, $\stalk{X_i}{x_i}$ is a local ring contained in $K$ under the identification $K(X_i) \iso K$. For a morphism $f : X_i \to X_j$ we get a diagram,
\begin{center}
\begin{tikzcd}
& A \arrow[ld, hook] \arrow[rd, hook]
\\
\stalk{X_j}{x_j} \arrow[d, hook] \arrow[rr] & & \stalk{X_i}{x_i} \arrow[d, hook]
\\ 
K(X_j) \arrow[rd, "\sim"'] \arrow[rr, hook] & & K(X_i) \arrow[ld, "\sim"]
\\
& K
\end{tikzcd}
\end{center}
therefore $f^* : \stalk{X_j}{x_j} \to \stalk{X_i}{x_i}$ is a local extension of domains containg $A$ and contained in $K$ i.e. $\stalk{X_i}{x_i}$ dominates $\stalk{X_j}{x_j}$. Then we may define,
\[ B = \varinjlim_{i \in I} \stalk{X_i}{x_i} \] 
\end{proof}



\end{document}
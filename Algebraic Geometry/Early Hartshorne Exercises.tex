\documentclass[12pt]{article}
\usepackage{hyperref}
\hypersetup{
    colorlinks=true,
    linkcolor=blue,
    filecolor=magenta,      
    urlcolor=cyan,
}
 
\urlstyle{same}
\usepackage{import}
\import{./}{AlgGeoCommands}


\AtBeginDocument{\renewcommand{\d}{\mathrm{d}}}
 

\begin{document}

\tableofcontents

\newpage

\section{I Varieties}

\subsection{Section 1}

\subsubsection{1.1 DO THIS}

\begin{enumerate}
\item Let $Y$ be the plane curve $y = x^2$. Let $A(Y)$ be the affine coodinate ring 
\[ A(Y) = k[x,y]/(y - x^2) \cong k[x] \]
 via the map $y \mapsto x^2$.

\item Let $Z$ be the plane curve $xy = 1$. Considre the affine coodinate ring $A(Y) = k[x,y]/(xy - 1)$. Consider a map $k[x,y]/(xy - 1) \to k[t]$ then $x, y$ map to units but $(k[t])^\times = k^\times$ and thus the map is not surjective. Therefore there cannot be such an isomorphism.

\item Let $f$ be any irreducible quadratic polynomial $f \in k[x,y]$ and let $W$ be the conic defined by $f$. Then write,
\[ f(x,y) = a_0 + a_{1,0} x + a_{0,1} y + a_{1,1} xy + a_{2,0} x^2 + a_{0,2} y^2 \]
where not all $a_{1,1}, a_{2,0}, a_{0,2}$ are zero. Let's do the characteristic not equal to two case first. When $a_{2,0} \neq 0$ we can write,
\[ f(x,y) = a_{2,0} (x - ay - b)^2 + a_{0,2} (y - a' x - b')^2 + a_0' \] 

\end{enumerate}

\subsubsection{1.2}

Let $Y \subset \A^3$ be the set $Y = \{(t, t^2, t^3) \mid t \in k \}$. Clearly, $Y \subset Z = Z(f_1, f_2, f_3)$ where $f_1 = x^2 - y$ and $f_2 = y^3 - z^2$ and $f_3 = z - x^3$. Furthermore, for any $p \in Z$ we know that $y = x^2$ and $z = x^3$ so $p = (x, x^2, x^3) \in Y$ and thus $Y = Z$. Clearly, $\dim{Y} = 1$ because it is is infinite and the image of $\A^1 \to \A^3$. Then,
\[ I(Y) = (y - x^2, z - x^3, y^3 - z^2) \]
Now consider,
\[ A(Y) = k[x,y,z]/I(Y) = k[x] \]
because $y \mapsto x^2$ and $z \mapsto x^3$. 

\subsubsection{1.3}

Let $Y$ be the algebraic set in $\A^3$ defined by the two polynomials $f_1 = x^2  - yz$ and $f_2 = xz - x$. Then $Y = Z(I)$ where $I = (x^2 - yz, xz - x)$. We need to find the minimal primes over $I$. Clearly $(x,y) \supset I$ and $(x,z) \supset I$ and $(z - 1, y - x^2) \supset I$. These are prime ideals and they are minimal because $I$ has height two. Furthermore,
\[ (x, y) \cap (x,z) \cap (z-1, y - x^2) = I \]
so $I$ has three irreducible components.

\subsubsection{1.5}

Let $B$ be a $k$-algebra. It is clear that if $B = A(Y)$ for some affine algebraic set then $B = A(Y) = k[x_1, \dots, x_n]/I(Y)$ is finitely generated and moreover $I$ is radical so $B$ is reduced.
\bigskip\\
Now suppose that $B$ is a reduced finite type $k$-algebra. Then there is a surjection $k[x_1, \dots, x_n] \onto B$ whose kernel is some ideal $I$. Therefore, $B \cong k[x_1, \dots, x_n]/I$. Since $B$ is reduced we see that $I$ is radical and thus $I = I(Z(I))$ and therefore $B = A(Z(I))$.

\subsubsection{1.7 (IN MY NOTES SOMEWHERE PRETTY OBVIOUS)}

\subsubsection{1.10}

\begin{enumerate}
\item Let $Y \subset X$ then choose a maximal chain of closed irreducibles,
\[ Z_0 \subsetneq Z_1 \subsetneq \cdots \subsetneq Z_n \]
inside $Y$ where $n = \dim{Y}$. Then taking closures in $X$ we see that,
\[ \overline{Z}_0 \subsetneq \overline{Z}_1 \subsetneq \cdots \subsetneq \overline{Z}_n \]
is also a chain of closed irreducibles. Furthermore, the inclusions are strict because $\overline{Z}_i \cap Y = Z_i$ and therefore if $\overline{Z}_i = \overline{Z}_{i+1}$ then $Z_i = Z_{i+1}$ which is false. Thus, $\dim{X} \ge n$.

\item Let $X$ be a topological space covered by a family of open subsets $\{ U_i \}$. By the previous part,
\[ \sup \dim{U_i} \le \dim{X} \]
Now choose a maximal chain of closed irreducibles,
\[ Z_0 \subsetneq Z_1 \subsetneq \cdots \subsetneq Z_n \]
in $X$. There is some $U_i$ such that $Z_0 \cap U_s$ is nonempty. Then I claim that $Z_i \cap U_s$ gives such a chain. It is clear that $Z_i \cap U_s$ is closed and irreducible now if $Z_i \cap U_s = Z_{i+1} \cap U_s$ then $U_s^C$ and $Z_i$ cover $Z_{i+1}$ but $Z_{i+1}$ is irreducible so $U_s^C \cap Z_{i+1} = \empty$ which is impossible because $Z_0 \subset Z_{i+1}$ so this must be a chain. Thus, $\dim{U_s} \ge \dim{X}$ proving the proposition.

\item Let $X = \Spec{\Z_p}$ then the point $(p) \in \Spec{\Z_p}$ is closed so $(0) \in \Spec{\Z_p}$ is open and also dense since this is an integral scheme (so all opens are dense). However, $U = \{ (0) \}$ clearly has dimension zero but $\dim{X} = 1$ since we have a chain $(0) \subsetneq (p)$. 

\item Let $Y$ be a closed subset of an irreducible finite-dimensional topological space $X$. Suppose that $\dim{Y} = \dim{X}$. If $Y \subsetneq X$ then any maximal chain in $Y$ can be augmented to give a longer chain by adding on $X$ (since closed sets in $Y$ are closed in $X$ since $Y \subset X$ is closed and irreducibility is not relative). Thus $\dim{Y} < \dim{X}$.

\item (EXAMPLE HERE!)
\end{enumerate}

\subsubsection{1.11}

Let $Y \subset \A^3$ be the curve given by $(t^3, t^4, t^5)$. Consider the ideal,
\[ I = (x^4 - y^3, x^5 - z^3, y^5 - z^4, xz - y^2, yz - x^3, x^2 y - z^2) = (xz - y^2, yz - x^3, x^2 y - z^2) \]
It is clear that $Y \subset Z(I)$. For any $p \in Z(I)$ we choose $t \in k$ such that $t^3 = x$ (we can do this since $k$ is algebraically closed). Then $y^3 = x^4 = t^{12}$ so we can change $t$ by a third root of unity such that $y = t^4$. Then $z^4 = y^5 = t^{20}$ so we can choose $z = t^5$ (WHY) and thus $Z(I) \subset Y$. Therefore $Y = Z(I)$. For dimension reasons ($\dim{Y} = 1$) we see that $\height{I} = 2$. We need to show that $I$ cannot have two generators. Then $I/I^2$ would have two generators as a $A/I$-module where $A = k[x,y,z]$. Then consider $\m = (x,y,z) \subset A$ then $I/I^2 \otimes_A A / \m$ would have two generators as a $A / \m$-module which is a field. However,
\[ M = I/I^2 \otimes_A A / \m = I/\m I \]
Suppose that $x^4 - y^3, x^5 - z^3, y^5 - z^4$ are dependent in $M$ then,
\[ \alpha (xz - y^2) + \beta (yz - x^3) + \gamma (x^2 y - z^3) \in \m I \]
However, every term in $\m  I$ has degree at least $3$ and thus $\alpha = \beta = 0$ because they cannot cancel eachother. Furthermore, there is no $z^3$ in any term of an element of $\m I$ and thus $\gamma = 0$. Thus $\dim{M} = 3$ contradicting the fact that it has two generators.

\subsubsection{1.12}

Consider $f = x^2(x-1)^2 + y^2 \in \RR[x,y]$ then $f$ is irreducible in $\RR[x,y]$ because of unique factorization in $\CC[x,y]$ we have,
\[ f = (x(x-1) + i y)(x(x-1) - iy) \]
but neither factor is in $\RR[x,y]$ and thus $f$ cannot factor. Furthermore, $Z(f)$ is the union of two points $(0,0)$ and $(1,0)$ and thus cannot be irreducible (it's not even connected!).  

\subsection{3}

\subsubsection{3.2}

\begin{enumerate}
\item Consider the map $\varphi : \A^1 \to \A^2$ be defined by $t \mapsto (t^2, t^3)$. Then $\varphi$ defines a map $\A^1 \to V(y^2 - x^3)$. Because $k$ is algebraically closed if $y^2 = x^3$ then letting $x = t^2$ we conclude that $y^2 = t^6$ and therefore $y = \pm t^3$ so we can choose $t$ or $-t$ (while fixing $x$) such that $(x,y) = (t^2, t^3)$. This defines a continuous inverse so $\varphi$ is a homeomorphism. However, $\varphi$ cannot be an isomorphism because $k[x,y]/(x^3 - y^2)$ is not regular at the origin while $\A^1$ is regular.
\item Let $k$ have characteristic $p > 0$ and $\varphi : \A^1_k \to \A^1_k$ be the map $t \mapsto t^p$. Notice that $k[t^p] \to k[t]$ is not surjective and thus $\varphi$ is not an isomorphism. However, $\varphi$ is topologically the identity map because $x \mapsto x^p$ fixes $k$ which are the poitns of $\A^1_k$. Therefore $\varphi$ is a homeomorphism. 
\end{enumerate}

\subsubsection{3.3 DO THIS!!}

\begin{enumerate}
\item Let $\varphi : X \to Y$ be a morphism. for each $P \in X$, 
\end{enumerate}

\subsection{5}

\section{II Schemes}

\subsection{1}

\subsubsection{1.1}

Let $A$ be an abelian group and $X$ a topological space. Let $\F_A$ be the presheaf $\F_A : U \to A$ for all $U \neq \empty$. There is a natural map of presheaves $\F_A \to \underline{A}$ sending $a \in A$ to the constant map $a$ on $U$. Then I claim this is an isomorphism on stalks. Given this fact, because $\underline{A}$ is a sheaf, by the universal property the map factors as $A \to A^+ \to \underline{A}$ and $A \to A^+$ is also an isomorphism on stalks so $A^+ \to \underline{A}$ is an isomorphism of stalks and since both are sheaves it is an isomorphism. 
\bigskip\\
Now for each $x \in X$ consider the stalk,
\[ \underline{A}_x = \varinjlim_{U \ni x} \underline{A}(U) = \varinjlim_{U \ni x} \mathrm{Cont}(U, A) \]
There is a natural map $A \to \underline{A}_x$ sending $a \mapsto f_a$ where $f_a$ is the germ of the constant map. It is clear that this is injective because if $f_a = f_b$ then they agree on some open and thus must agree at $x$ so $a = f_a(x) = f_b(x) = b$. Finally, we need to show that it is surjective. Let $[f] \in \underline{A}_x$ with a representative $f : U \to X$ on some $U$. Let $a = f(x)$. Because $A$ has the discrete topology, $a \in A$ is open so $V = f^{-1}(a)$ is open. Therefore, $f|_V = f_a|_V$ and thus $[f] = [f_a]$ in $\underline{A}_x$ proving surjectivity.

\subsubsection{1.2}

\begin{enumerate}
\item Let $\varphi : \F \to \G$ be a morphism of sheaves and $x \in X$ a point. Because sheafification preserves stalks, it suffices to show that filtered direct limits commute with kernels and images which follows from the fact that they are exact.

\item We know 
\[ \ker{\varphi} = 0 \iff \forall x \in X : (\ker{\varphi})_x = 0 \iff \forall x \in X : \ker{\varphi_x} = 0 \]
so $\varphi$ is injective if and only if $\varphi_x$ is injective for each $x \in X$. Likewise,
\[ \coker{\varphi} = 0 \iff \forall x \in X : (\coker{\varphi})_x = 0 \iff \forall x \in X : \coker{\varphi_x} = 0 \]
so $\varphi$ is surjective if and only if $\varphi_x$ is surjective for all $x \in X$.
\bigskip\\
We can also give a direct argument for the former. It is clear that if $\varphi$ is injective then $\varphi_x$ is injective since if $\varphi_x([f]) = 0$ then $\varphi(f)$ is zero on some open $U$ and thus $f|_U = 0$ so $[f] = 0$. Conversely, suppose that $\forall x \in X : \varphi_x$ is injective. Then consider $f \in \F(U)$ and suppose that $\varphi(f) = 0$. Then $\varphi_x(f_x) = 0$ by compatibility so $f_x = 0$ for each $x \in U$. This means there is some open $x \in V_x$ such that $f|_{V_x} = 0$ and since $\{ V_x \}$ cover $U$ we see that $f = 0$ by the sheaf condition so $\varphi$ is injective.

\item The sequence,
\begin{center}
\begin{tikzcd}
\cdots \arrow[r] & \F^{i} \arrow[r, "\varphi^{i}"] & \F^{i+1} \arrow[r, "\varphi^i"] & \cdots 
\end{tikzcd}
\end{center}
is exact at $\F^{i+1}$ iff $\im{\varphi^i} = \ker{\varphi^{i+1}}$. We can check if two subsheaves are equal by checking on stalks inside $\F^{i+1}_x$ and $(\im{\varphi^i})_x = \im{\varphi^i_x}$ and $(\ker{\varphi^{i+1}})_x = \ker{\varphi^{i+1}_x}$ so it suffices to check that,
\[ \im{\varphi^{i}_x} = \ker{\varphi^{i+1}_x} \]
and thus exactness of the sequence of sheaves is equivalent to exactness of the stalk sequences.
\end{enumerate}

\subsubsection{1.3}

\begin{enumerate}
\item Let $\varphi : \F \to \G$ be a morphism of sheaves on $X$. Note that $\varphi$ is surjective if and only if $\varphi_x$ is urjective for each $x \in X$. Suppose that for every open $U \subset x$ and $s \in \G(U)$ there is a covering $\{ U_i \}$ of $U$ and sections $t_i \in \F(U_i)$ such that $\varphi(t_i) = s|_{U_i}$. Then for each $x \in X$ consider $[s] \in \G_x$ and choose some representative $s \in \G(U)$ so there exists $\{ U_i \}$ covering $U$ and $t_i \in \F(U_i)$ such that $\varphi(t_i) = s|_{U_i}$. Because it is a cover, $x \in U_j$ for some $j$ and thus $\varphi(t_j) = s|_{U_j}$ means that $\varphi_x([t_j]) = [s]$ because $s$ and $s|_{U_j}$ agree on $U_j$. Therefore $\varphi_x$ is surjective for each $x \in X$.
\bigskip\\
Conversely, suppose that $\varphi_x$ is surjective for each $x \in X$. Let $U \subset X$ be open and $s \in \G(U)$. Then by surjectivity, for all $x \in U$ there is $t_x \in \F_x$ such that $\varphi_x(t_x) = s_x$. Therefore, chose a representative $\tilde{t}_x \in \F(V_x)$ for some open $V_x \ni x$ and $\varphi(t_x) = x_s$ meaning that there is some open $U_x \subset V_x$ containing $x$ such that $\varphi(\tilde{t}_x|_{U_x}) = s|_{U_x}$ since $x \in U_x$ then $\{ U_x \}$ form an open cover of $U$ proving the claim.

\item Let $X = \C^\times$ and let $\exp : \struct{X} \to \struct{X}^\times$ be the exponential map. This is surjective on local sections and thus surjective as a map of sheaves but not surjective because there is no gloval logarithm.
\end{enumerate}

\subsubsection{1.4}

\begin{enumerate}
\item Let $\varphi : \F \to \G$ be a morphism of presheaves such that $\varphi_U : \F(U) \to \G(U)$ is injective for all $U$. This implies that $\varphi_x$ is injective for all $x$. Therefore, consider the diagram,
\begin{center}
\begin{tikzcd}
\F \arrow[d, "\theta_\F"] \arrow[r, "\varphi"] & \G
\arrow[d, "\theta_\G"]
\\
\F^+ \arrow[r, "\varphi^+"] & \G^+
\end{tikzcd}
\end{center}
because $\theta_\F$ and $\theta_\G$ are isomorphism on stalks we see that $\varphi^+_x$ is also injective meaning that $\varphi^+$ is injective because it is a morphism of sheaves.

\item Let $\varphi : \F \to \G$ be a morphism of sheaves. Then the presheaf $\mathrm{im}^p(\varphi) \subset \G$ is naturally a subpresheaf of $\G$ so by the previous part $\im{\varphi} \embed \G$ since $\im{\varphi} = (\mathrm{im}^p(\varphi))^+$ and $\G^+ = \G$ since $\G$ is a sheaf.
\end{enumerate}

\subsubsection{1.5}

Let $\varphi : \F \to \G$ be a morphism of sheaves. If $\varphi$ is an isomorphism then it is clear that $\ker{\varphi} = 0$ and $\im{\varphi} = \G$ because it is injective and surjective (globally and locally) on sections. Conversely, suppose that $\varphi$ is injective and surjective then it suffices to show that $\varphi_U : \F(U) \to \G(U)$ is an isomorphism for each $U$ since then $\varphi_U^{-1}$ gives an inverse sheaf morphism.
\bigskip\\
Indeed, $\varphi$ injective and surjective is equivalent to $\varphi_x$ being an isomorphism for all $x \in X$. We have shown this implies that $\varphi_U$ is injective so we just need to prove that $\varphi_U$ is surjective. Since $\F \to \G$ is surjective, by the previous problem we know that for each $s \in \G(U)$ there is a cover $\{ U_i \}$ of $U$ and $t_i \in \F(U_i)$ such that $\varphi(t_i) = s|_{U_i}$. Then consider $\varphi(t_i)|_{U_i \cap U_j} = s|_{U_i \cap U_j}$ which means that $\varphi(t_i)|_{U_i \cap U_j} = \varphi(t_j)|_{U_i \cap U_j}$ by compatibility of restriction. Therefore, by injectivity $t_i |_{U_i \cap U_j} = t_j |_{U_i \cap U_j}$. Then by the sheaf condition, the sections $t_i \in \F(U_i)$ glue to a section $t \in \F(U)$ such that $t|_{U_i} = t_i$. Thus by compatibility of $\varphi$ with restriction, $\varphi(t)|_{U_i} = \varphi(t|_{U_i}) = \varphi(t_i) = s|_{U_i}$ so by the uniqueness part of the sheaf property $\varphi(t) = s$ since $\{ U_i \}$ is an open cover of $U$. Therefore $\varphi_U$ is surjective proving that $\varphi$ is an isomorphism. 

\subsubsection{1.6}

\begin{enumerate}
\item Let $\F'$ be a subsheaf of a sheaf $\F$. The map of presheaves $\F \to (\F / \F')^p$ is surjective and therefore surjective on stalks so $\F \to \F / \F'$ remains surjective on stalks and thus is surjective. A similar argument works for the kernel. Alternatively, the sequence,
\begin{center}
\begin{tikzcd}
0 \arrow[r] & \F' \arrow[r] & \F \arrow[r] & \F / \F' \arrow[r] & 0
\end{tikzcd}
\end{center}
is exact because it is exact on stalks using the fact that $(\F / \F')_x = (\F / \F')_x^p = \F_x / \F'_x$ where the last equality follows from the exactness of filtered colimits.

\item Conversely, suppose that,
\begin{center}
\begin{tikzcd}
0 \arrow[r] & \F' \arrow[r] & \F \arrow[r] & \F'' \arrow[r] & 0
\end{tikzcd}
\end{center}
is an exact sequence. Because $\F' \to \F$ is injective it is isomorphic to its image by the previous problem. Furthermore, since $\im{(\F' \to \F)} = \ker{(\F \to \to \F'')}$ we see that $\F \to \F''$ factors through $(\F / \F')^p \to \F''$ and therefore by the universal property of sheafification it factors through $\F / \F' \to \F''$ which is injective and surjective (it is clear from exactness or check on stalks) and thus an isomorphism.
\end{enumerate}

\subsubsection{1.7}

Let $\varepsilon : \F \to \G$ be a morphism of sheaves.
\begin{enumerate}
\item There is an exact sequence,
\begin{center}
\begin{tikzcd}
0 \arrow[r] & \ker{\varphi} \to \F \arrow[r] & \im{\varphi} \arrow[r] & 0
\end{tikzcd}
\end{center}
which can be checked on stalks (coming from the equivalent sequence for a map of modules $\varphi_x : \F_x \to \G_x$)
and therefore the previous problem shows that $\F / \ker{\varphi} \iso \im{\varphi}$

\item Likewise, there is an exact sequence,
\begin{center}
\begin{tikzcd}
0 \arrow[r] & \im{\varphi} \to \F \arrow[r] & \coker{\varphi} \arrow[r] & 0
\end{tikzcd}
\end{center}
which can be checked on stalks (coming from the equivalent sequence for a map of modules $\varphi_x : \F_x \to \G_x$)
and therefore the previous problem shows that $\F / \im{\varphi} \iso \coker{\varphi}$.
\end{enumerate}

\subsubsection{1.8}

Given a continuous map $f : X \to Y$ the functor $f^{-1} : \Sh(Y) \to \Sh(X)$ is a left-adjoint to the functor $f_* : \Sh(X) \to \Sh(Y)$. Therefore $f^{-1}$ is cocontinuous and right-exact and $f_*$ is continuous and left-exact. In fact, $f^{-1}$ is exact. 

\begin{lemma}
The functor $f^{-1}$ preserves stalks.
\end{lemma}

\begin{proof}
Let $\F$ be a sheaf on $Y$ and $f : X \to Y$ a continuous map. Then $f^{-1} \F$ is the sheafification of the presheaf,
\[ U \mapsto \varinjlim_{V \supset f(U)} \F(V) \]
The stalks of this presheaf are,
\[ S_x = \varinjlim_{x \in U} \varinjlim_{V \supset f(U)} \F(V) = \varinjlim_{f(x) \in V} \F(V) = \F_x \]
Since sheafification preserves stalks we have shown that $(f^{-1} \F)_x = \F_x$.
\end{proof}

\begin{proposition}
The functor $f^{-1}$ is exact. 
\end{proposition}

\begin{proof}
The functor $f^{-1}$ comutes with talking stalks. Therefore, applying $f^{-1}$ to an exact sequence preserves exactness on the stalks and thus exactness of the sequence.
\end{proof}

Consider two special cases. First, consider the constant map $C : X \to *$ sending all of $X$ to a point. Then $C_* \F = \Gamma(X, \F)$ is the sheaf which sends the only nonempty open set of $*$ to $\F(C^{-1}(*)) = \F(X)$. Furthermore, any abelian group $A$ is a sheaf on $*$ so $C^*(A)$ is the sheafification of $U \mapsto A$ and thus the constant sheaf $\underline{A}$ on $X$. Thus $\Gamma(X, -)$ is left-exact and $A \mapsto \underline{A}$ is exact.
\bigskip\\
Second, consider the inclusion $\iota_x : * \to X$ sendng $*$ to $x \in X$. Then given a sheaf $\F$ on $X$ we have,
\[ \iota_x^{-1} \F = \varinjlim_{x \in U} \F(U) = \F_x \]
and for an abelian group $A$ (as a sheaf on $*$) we have $(\iota_x)_* A$ is the skysraper sheaf at $x$ with stalk $A$. Thus, taking skyscrapers is left-exact and takiking stalks is exact.
\bigskip\\
Finally, this is easily proven directly. Given an exact sequence of sheaves,
\begin{center}
\begin{tikzcd}
0 \arrow[r] & \F \arrow[r, "f"] & \G \arrow[r, "g"] & \H 
\end{tikzcd}
\end{center}
then we must show that the sequence,
\begin{center}
\begin{tikzcd}
0 \arrow[r] & \Gamma(U, \F) \arrow[r, "f"] & \Gamma(U, \G) \arrow[r, "g"] & \Gamma(U, \H)
\end{tikzcd}
\end{center}
is exact for any open set $U \subset X$. We have already showed that if $\F \to \G$ is injective then $\F(U) \to \G(U)$ is injective for each open. Furthermore, since $f$ is injective it is an isomorphism onto $\im{f} = \ker{g}$ by previous problems. Therefore, $f_U : \F(U) \to \G(U)$ has image exactly $(\ker{g})(U) = \ker{(\G(U) \to \H(U))}$ proving exactness of the sequence. Alternatively, it suffices to show that,
\[ \Gamma(U, \ker{(\G \to \H)}) = \ker{(\Gamma(U, \G) \to \Gamma
(U, \H))} \]
which follows directly from the construction of kernels as $(\ker{\varphi})(U) = \ker{(\F(U) \to \G(U))}$.

\subsubsection{1.9}

Let $\F$ and $\G$ be sheaves on $X$. Consider the presheaf $U \mapsto \F(U) \oplus \G(U)$. We want to show this is a sheaf. Suppose that $(s,t) \in \F(U) \oplus \G(U)$ and $\{ U_i \}$ is an open cover of $U$ such that $(s,t)|_{U_i} = 0$ this by definition means that $s|_{U_i} = 0$ and $t|_{U_i} = 0$. Then by the sheaf property $s = 0$ and $t = 0$ so $(s,t) = 0$. Furthermore, given $(s_i, t_i) \in \F(U_i) \oplus \G(U_i)$ such that $(s_i, t_i)|_{U_i \cap U_j} = (s_j, t_j) |_{U_i \cap U_j}$ we see that $s_i |_{U_i \cap U_j} = s_j |_{U_i \cap U_j}$ and $t_i |_{U_i \cap U_j} = t_j |_{U_i \cap U_j}$. Therefore, we can glue to $(s, t) \in \F(U) \oplus \G(U)$ such that $(s,t)|_{U_i} = (s|_{U_i}, t|_{U_i}) = (s_i, t_i)$. 
\bigskip\\
Alternatively, this is just because finite direct sums are limits and limits commute with equalizers so the limit of sheaves sectionwise is still a sheaf. Furthermore, this is a limit in the category of sheaves because $\Gamma(U,-)$ is a right adjoint is commutes with limits and thus sends the limit of sheaves to the limit of the sections. Since sheaves are determined on their sections this proves that $(\lim \F_\alpha)(U) = \lim \F_{\alpha}(U)$.


\subsubsection{1.10}

Let $\{ \F_i \}$ be a firected system of sheaves and morphisms on $X$. We define,
\[ (\varinjlim \F_i)^p(U) = \varinjlim \F_i(U) \]
and $\varinjlim \F_i$ is the sheaf associated to this presheaf. Let $J : I \to \mathbf{Sh}(X)$ denote the directed diagram. Then we see that,
\[ \Hom{\mathbf{Sh}^I}{J}{\Delta \G} = \Hom{\mathbf{PSh}}{(\varinjlim \F_i)^p}{\G} = \Hom{\mathbf{Sh}}{\varinjlim \F_i}{\G} \]
where the second equality follows from the universal property of sheafification. The first equality follows because for each open $U \subset X$ we have $(\varinjlim \F_i)^p(U) = \varinjlim \F_i(U)$ and thus given maps $\F_i(U) \to \G(U)$ compatible with the diagram we get a unique such map from the limit and these are clearly compatible with restriction. Finally, given $(\varinjlim \F_i)^p \to \G$ compatible with the natrual maps $\F_i \to (\varinjlim \F_i)^p$ we get a unique morphism $\varinjlim \F_i \to \G$ compatible with the natural maps $\F_i \to \varinjlim \F_i$.

\subsubsection{1.11}

Let $X$ be noetherian. Then we need to show that $(\varinjlim_{\F_i})^p$ is already a sheaf. We need to show that it satisfies the equalizer diagram for any open cover $\{ U_\alpha \}$ of $U$. However, since $X$ is noetherian, $U$ is also noetherian and thus compact. Therefore the cover $\{ U_\alpha \}$ admits a finite subcover $\{ U_{\alpha_j} \}$. Sinite filtered colimits commute with finite limits (also here finite products are coproduts so we really only need that filtered colimits are exact) we see that,
\begin{center}
\begin{tikzcd}
0 \arrow[r] & \varinjlim \F_i(U) \arrow[r] & \prod_{j} \varinjlim \F_i(U_{\alpha_j}) \arrow[r] & \prod_{j, k} \varinjlim \F_i(U_{\alpha_j} \cap U_{\alpha_k}) 
\end{tikzcd}
\end{center}
is exact. Therefore, if $s|_{U_{\alpha}} = t|_{U_{\alpha}}$ for all $\alpha$ then in particular they agree on the subcover and therefore by the sequence $s = t$ so we get uniqueness for the entire cover. Now for gluing, consider $\{ s_{\alpha} \}$ such that $s_{\alpha}|_{U_\alpha \cap U_\beta} = s_{\beta} |_{U_\alpha \cap U_\beta}$ then from the sequence we can glue the $\{ s_{U_{\alpha_j}} \}$. Then 
\[ s|_{{U_\beta} \cap U_{\alpha_i}} = (s|_{U_{\alpha_i}})|_{U_{\beta} \cap U_{\alpha_i}} = s_{\alpha_i} |_{U_{\beta} \cap U_{\alpha_i}} = s_{\beta} |_{U_{\beta} \cap U_{\alpha_i}} \]
Since $\{ U_{\alpha_i} \}$ forms a cover of $U$ we see that $\{ U_\beta \cap U_{\alpha_i} \}$ form a cover of $U_\beta$ so applying the uniqueness part to $U_\beta$ we see that $s_\beta = s|_{U_\beta}$ and thus $s$ glues the sections $\{ s_\alpha \}$.

\subsubsection{1.12}

Let $\{ \F_i \}$ be an inverse system of sheaves on $X$. I claim that the presheaf $U \to \varinjlim \F_i(U)$ is a sheaf and is the categorical limit of this diagram in the category of sheaves.
\bigskip\\
This is general, we already saw that limits commute with limits and therefore this presheaf satisfies the sheaf condition. It is clear that $\varinjlim \F_i$ is the inverse limit in the category of presheaves and thus it is also the inverse limit in the category of sheaves because the category of sheaves is a full subcategory. 

\subsubsection{1.13}

\newcommand{\Spe}[1]{\text{Sp\'{e}}\left( #1 \right)}

Let $\F$ be a presheaf on $X$. We define a topological space $\Spe{\F}$ called the \textit{espace \'{e}tal\'{e}} whose underlying set is,
\[ \Spe{\F} = \bigcup_{P \in X} \F_P \]
with a projection map $\pi : \Spe{\F} \to X$ sending $s \in \F_P$ to $P$. For each open $U \subset X$ and each section $s \in \F(U)$ we obtain a map $\bar{s} : U \to \Spe{\F}$ by sending $P \mapsto s_P$ its germ at $P$ so that $\pi \circ \bar{s} = \id_U$. Therefore, it is a ``section'' of $\pi$ over $U$. We give $\Spe{\F}$ the strongest toplogy such that all the maps $\bar{s} : U \to \Spe{\F}$ for all $U$ and all $s \in \F(U)$ are continuous. 
\bigskip\\
Let $\wt{F}$ be the sheaf of \textit{continuous} sections of $\Spe{\F}$ over $U$ meaning it is the presheaf $U \to \wt{\F}(U)$ where $\wt{F}(U)$ is the set of continuous sections of $\Spe{\F}$. This is a sheaf because continuous maps between topological spaces are determined locally and glue.
\bigskip\\
I claim that $\wt{\F}$ equipped with the natural map $\F \to \wt{\F}$ sending $s \mapsto \bar{s}$ is the sheafification. Because $\wt{\F}$ is a sheaf, it suffices to prove that $\F \to \wt{\F}$ is an isomorphism on stalks because then $\F \to \wt{\F}$ factors as $\F \to \F^+ \to \wt{\F}$ and since $\theta : \F \to \F^+$ is an isomorphism on stalks this would show that $\F^+ \to \wt{\F}$ is an isomorphism on stalks and thus an isomorphism because $\F^+$ and $\wt{\F}$ are sheaves. 
\bigskip\\
Consider the map $\F \to \wt{\F}$. On stalks this gives,
\[ \F_P \to (\wt{\F})_P \]
sending $[s] \in \F_P$ to any lift $s \in \F(U)$ mapping to $\bar{s} \in \wt{\F}(U)$ then defining a class $[\bar{s}] \in (\wt{\F})_P$. We can describe an explicit inverse, $[\psi] \mapsto \psi(P) \in \F_P$ for the class of any section $[\psi] \in (\wt{\F})_P$. Clearly, because the equivalence relation is agreeing on open sets of $X$ containing $P$ the value $\psi(P)$ is a constant on the equivalence class so this is well-defined. We need to show these maps are inverses. First, given $[s] \in \F_P$ it is clear that $\bar{s}(P) = s_P = [s]$ by definition. Then given $[\psi] \in (\wt{\F})_P$ let $[s] = \psi(P) \in \F_P$ and $s \in \F(U)$and $\psi : U \to \Spe{\F}$ be some representatives on an open $U \subset X$. Consider $V = \{ x \in U \mid \psi(x) = s_{x} \}$. I claim that the set $W = \{ (x, s_x) \in \Spe{\F} \mid x \in U \}$ is open because for each $t \in \F(U')$ then $\bar{t}^{-1}(W) = \{ x \in U \cap U' \mid t_x = s_x \}$ which is open in $X$ because if $t_x = s_x$ then they agree on some open neighborhood $U''$ of $x$ by definition and thus $t_y = t_y$ for all $y \in U''$. Therefore, by the definition of the strong topology on $\Spe{\F}$ we see that $W \subset \Spe{\F}$ is open so $V = \psi^{-1}(W)$ is open by continuity. Since, we defined $s$ such that $\psi(P) = s_P$ we have $P \in V$ so there is some open neighborhood $P \in U' \subset V$ meaning $\psi(x) = s_x = \bar{s}(x)$ for all $x \in U'$ and thus $\psi|_{U'} = \bar{s}|_{U'}$ so $[\psi] = [\bar{s}]$ proving that $[\psi] \mapsto [\psi(P)] \mapsto [\overline{\psi(P)}]$ is the identity and thus $\F \to \wt{\F}$ is an isomorphism.

\subsubsection{1.14}

Let $\F$ be a sheaf on $X$ and $s \in \F(U)$ a section on some open set $U$. Then consider the set,
\[ \Supp{\F}{s} = \{ x \in U \mid s_x \neq 0 \} \]
Suppose $x \in U \setminus \Supp{\F}{s}$ then $s_x = 0$. Thus, there exists some open neighborhood $x \in V \subset U$ such that $s|_V = 0$. Then for each $y \in V$ we have $_y = (s|_V)_y = 0$ so $y \in U \setminus \Supp{\F}{s}$ and thus $V \subset U \setminus \Supp{\F}{s}$. Therefore, $U \setminus \Supp{\F}{s}$ is open so $\Supp{\F}{s}$ is closed.
\bigskip\\
We furthermore define $\Supp{}{\F} = \{ x \in X \mid \F_x \neq 0 \}$ which is not necessarily closed without further assumptions on $\F$. If $\F$ is a coherent $\struct{X}$-module then this holds because on affine opens $\Supp{}{\F} \cap U = \Supp{}{M} = V(\Ann{A}{M})$ which is closed in $U$ where $U = \Spec{A}$ and $\F |_U = \widetilde{M}$ a finitely-generated $A$-module. 

\subsubsection{1.15 (CHECK)}

Let $\F$ and $\G$ be sheaves of abelian groups on $X$ (in fact, $\F$ need only be a presheaf). Consider the presheaf $\shHom{}{\F}{\G}$ given by sending $U \mapsto \Hom{}{\F|_U}{\G|_U}$. I claim that this presheaf is actually a sheaf. First, let $f : \F|_U \to \G|_U$ be a morphism of sheaves and $\{ V_i \}$ and open cover of $U$ such that $f|_{V_i} = 0$ on each $V_i$. Let $\tilde{U} \subset U$ be any open subset and consider $f_{\tilde{U}} : \F(\tilde{U}) \to \G(\tilde{U})$. There is an open cover $\tilde{V}_i = \tilde{U} \cap V_i$ of $\tilde{U}$ and since $\tilde{V}_i \subset V_i$ we have $f|_{\tilde{V}_i} = 0$. Then for $s \in \F(V)$ we have \[ \res_{\tilde{V}_i, \tilde{U}} \circ f_{\tilde{U}}(s) = f_{\tilde{V}_i} \circ \res_{\tilde{V}_i, \tilde{U}}(s) = 0 \]
Therefore, $f_{\tilde{U}}(s)$ restricted to the cover $\tilde{V}_i$ is zero so by the sheaf property of $\G$ we have $f_{\tilde{U}}(s) = 0$. Thus, $f = 0$ proving the locality property of $\shHom{}{\F}{\G}$. 
\bigskip\\
Now, suppose that $V_i$ is an open cover of the open subset $U \subset X$ as before and we have $f_i \in \shHom{}{\F}{\G}(V_i) = \Hom{}{\F|_{V_i}}{\G|_{V_i}}$ which agree on the overlaps. Take any open $\tilde{U} \subset U$ and cover it viw $\tilde{V}_i = \tilde{U} \cap V_i$. Now define a morphism $f : \F|_U \to \G|_U$ such that,
\begin{center}
\begin{tikzcd}[column sep = large, row sep = large]
\F(\tilde{U}) \arrow[r, "f_{\tilde{U}}"] \arrow[d, "\res_{\tilde{V}_i, \tilde{U}}"']  & \G(\tilde{U}) \arrow[d, "\res_{\tilde{V}_i, \tilde{U}}"] 
\\
\F(\tilde{V}_i) \arrow[r, "(f_i)_{\tilde{V}_i}"] & \G(\tilde{V}_i)
\end{tikzcd}
\end{center}
as follows. Given $s \in \F(\tilde{U})$ let $s_i = s |_{V_i}$. Then the sections $(f_i)_{\tilde{V}_i}(s_i)$ agree on overlaps because,
\[ \res_{\tilde{V}_i \cap \tilde{V}_j, \tilde{V}_i} \circ (f_i)_{\tilde{V}_i}(s_i) = (f_i)_{\tilde{V}_i \cap \tilde{V}_j} \circ \res_{\tilde{V}_i \cap \tilde{V}_j, \tilde{V}_i}(s_i) = (f_i)_{\tilde{V}_i \cap \tilde{V}_j}(s |_{\tilde{V}_i \cap \tilde{V}_j}) \]
However, by assumption, $(f_i)_{\tilde{V}_i \cap \tilde{V}_j} = (f_j)_{\tilde{V}_i \cap \tilde{V}_j}$ and thus,
\[ \res_{\tilde{V}_i \cap \tilde{V}_j, \tilde{V}_j} \circ (f_j)_{\tilde{V}_j}(s_j) = (f_j)_{\tilde{V}_i \cap \tilde{V}_j}(s |_{\tilde{V}_i \cap \tilde{V}_j}) = \res_{\tilde{V}_i \cap \tilde{V}_j, \tilde{V}_i} \circ (f_i)_{\tilde{V}_i}(s_i) \]
Therefore, by the sheaf property of $\G$ these sections glue to form a unique section $f_{\tilde{U}}(s) \in \G(\tilde{U})$. We must check that the constructed $f$ is a homomorphism and satisfies the naturality conditions. Take $s,t \in \tilde{U}$ then,
\[ (f_i)_{\tilde{V}_i}((s + t)|_{\tilde{V}_i}) = (f_i)_{\tilde{V}_i}(s_i + t_i) = (f_i)_{\tilde{V}_i}(s_i) + (f_i)_{\tilde{V}_i}(t_i) \]
We know that these sections lift to $f_{\tilde{U}}(s)$ and $f_{\tilde{U}}(s)$ respectively showing that the sum lifts to $f_{\tilde{U}}(s) + f_{\tilde{U}}(t)$ because restriction is linear. Therefore, be definition the lift of these sections gives,
\[ f_{\tilde{U}}(s + t) = f_{\tilde{U}}(s) + f_{\tilde{U}}(t) \]
so $f$ is a collection of homomorphisms. Furthermore, take any open $W \subset \tilde{U}$. Then, consider the diagram,
\begin{center}
\begin{tikzcd}[column sep = large, row sep = large]
\F(\tilde{U}) \arrow[d, "\res_{W, \tilde{U}}"'] \arrow[r, "f_{\tilde{U}}"] & \G(\tilde{U}) \arrow[d, "\res_{W, \tilde{U}}"] 
\\
\F(W) \arrow[r, "f_{W}"'] & \G(W)
\end{tikzcd}
\end{center}  
Given a cover $V_i$ of $U$ we get covers $\tilde{V}_i = \tilde{U} \cap V_i$ of $\tilde{U}$ and $W_i = W \cap V_i = W \cap \tilde{V}_i$ of $W$. For any section $s \in \F(\tilde{U})$ consider $f_{W}(s|_W)$ which is the lift of $(f_i)_{W_i}(\res_{W_i, W} (s|_W))$ to $\G(W)$. However,
\[ \res_{W_i, W}(s|_W) = \res_{W_i, W} \circ \res_{W, \tilde{U}}(s) = \res_{W_i, \tilde{U}}(s) = \res_{W_i, \tilde{V}_i} \circ \res_{\tilde{V}_i, \tilde{U}}(s) = \res_{W_i, \tilde{V}_i}(s_i) \]
Therefore, using the naturality of $f_i$ on subsets of $V_i$,
\[ (f_i)_{W_i}(\res_{W_i, W} (s|_W)) = (f_i)_{W_i}(\res_{W_i, \tilde{V}_i}(s_i)) = \res_{W_i, \tilde{V}_i} \circ (f_i)_{\tilde{V}_i}(s_i) \]
Furthermore, we know that the sections $(f_i)_{\tilde{V}_i}(s_i)$ lift to $f_{\tilde{U}}(s)$. Thus,
\[ (f_i)_{W_i}(\res_{W_i, W} (s|_W)) = \res_{W_i, \tilde{V}_i} \circ \res_{\tilde{V}_i, \tilde{U}} \circ f_{\tilde{U}}(s) = \res_{W_i, W} \circ (\res_{W, \tilde{U}} \circ f_{\tilde{U}}(s)) \]
Therefore, the sections which lift to $f_W(s|_W)$ (i.e. the restrictions of $f_W(s|_W)$ to $W_i$) are exactly the restrictions of $\res_{W, \tilde{U}} \circ f_{\tilde{U}}(s)$. By the sheaf property of $\G$, gluing is unique so $f_W(s|_W) = \res_{W, \tilde{U}} \circ f_{\tilde{U}}(s)$. Thus locality gives,
\[ f_W \circ \res_{W, \tilde{U}} = \res_{W, \tilde{U}} \circ f_{\tilde{U}} \]
Therefore the morphisms $f_i$ glue to a unique $f \in \shHom{}{\F}{\G}(U) = \Hom{}{\F|_U}{\G|_U}$ so $\shHom{}{\F}{\G}$ is a sheaf. 


\subsubsection{1.16 (CHECK)}

\begin{enumerate}
\item[(a)] Let $X$ be an irreducible space and $\underline{A}$ a constant sheaf on $X$. Take any open sets $V \subset U \subset X$. By Lemmas \ref{open_of_irreducible} and \ref{irreducible_implies_connected} the sets $V$ and $U$ are connected. Therefore, any continuous map $f : V \to A$ (with $A$ given the discrete topology) is constant (since the only connected sets in the discrete topology are points) so $f : V \to A$ is the restriction of the corresponding constant map $\tilde{f} : U \to A$. Therefore, the restriction map $\res_{V,U} : \underline{A}(U) \to \underline{A}(V)$ is surjective. Thus, the constant sheaf $\underline{A}$ is flasque.  

\item[(b)] Consdier the exact sequence of sheaves over $X$,
\begin{center}
\begin{tikzcd}
0 \arrow[r] & \F \arrow[r, "f"] & \G \arrow[r, "g"] & \H \arrow[r] & 0
\end{tikzcd}
\end{center}
where $\F$ is flasque. For an open set $U \subset X$, applying the left-exact functor $\Gamma(U, -)$ we get an exact sequence,
\begin{center}
\begin{tikzcd}
0 \arrow[r] & \F(U) \arrow[r, "f_U"] & \G(U) \arrow[r, "g_U"] & \H(U)
\end{tikzcd}
\end{center}
It suffices to show that the map $\G(U) \to \H(U)$ is surjective. For each $x \in U$, consider the induced maps on stalks, 
\begin{center}
\begin{tikzcd}
0 \arrow[r] & \F(U) \arrow[d] \arrow[r, "f_U"] & \G(U) \arrow[r, "g_U"] \arrow[d] & \H(U) \arrow[d]
\\
0 \arrow[r] & \F_x \arrow[r, "f_x"] & \G_x \arrow[r, "g_x"] & \H_x \arrow[r] & 0 
\end{tikzcd}
\end{center}
For any section $s \in \H(U)$ its inclusion in the stalk $\H_x$ lifts to $t_x \in \G_x$. Therefore, there exists some open $W$ nbd. of $x$ such that $t_x \in \G(W)$ maps to $s|_{W} \in \H(W)$.  
\bigskip\\
Consder the poset $\mathcal{T}$ of pairs $(V, t)$ where $V \subset U$ is open, $t \in \G(V)$, and $g_V(t) = s|_V$. The ordering is $(V, t) \le (V', t')$ if and only if $V \subset V'$ and $t'|_V = t$. To apply Zorn's lemma, consider a totally ordered subset $(V_\alpha, t_\alpha) \subset \mathcal{T}$ with totally ordered index set $\alpha \in I$. Then take,
\[ V = \bigcup_{\alpha \in I} V_\alpha \]
and the unique $t$ which glues all $t_{\alpha}$ by the sheaf condition of $\G$. Such a gluing exists because for $\alpha < \alpha'$ we have $V_\alpha \subset V_{\alpha'}$ and $t_{\alpha'}|_{V_{\alpha}} = t$ where $V_{\alpha} \cap V_{\alpha'} = V_{\alpha}$ so these sections agree on the overlap. 
\bigskip\\
Now, by Zorn's lemma, there exists a maximal element $(V, t)$ in $\mathcal{T}$. It suffices to show that $V = U$ since then $g_U(t) = s$. For each $x \in U$ we have $(W, t_x) \in \mathcal{T}$ from before. Then,
\begin{align*}
g_{W \cap V}(t_x|_{W \cap V} - t|_{W \cap V}) & =  \res^\H_{W \cap V, W} \circ g_{W}(t_x) - \res^\H_{W \cap V, V} \circ g_V(t)
\\
& = \res^\H_{W \cap V, W}(s |_{W}) - \res^\H_{W \cap V, V}(s|_V) =  s|_{W\cap V} - s|_{W \cap V} = 0
\end{align*}
Therefore, the section $d = t_x|_{W \cap V} - t|_{W \cap V}$ lies in the image of $f_{W \cap V}$ and thus lifts to $q \in \F(W \cap V)$. 
\begin{center}
\begin{tikzcd}
0 \arrow[r] & \F(W) \arrow[d, two heads] \arrow[r, "f_W"] & \G(W) \arrow[r, "g_W"] \arrow[d] & \H(W) \arrow[d]
\\
0 \arrow[r] & \F(W \cap V) \arrow[r, "f_{W \cap V}"] & \G(W \cap V) \arrow[r, "g_{W \cap V}"] & \H(W \cap V)
\end{tikzcd}
\end{center}
Because $\F$ is flasque, the section $q$ lifts to $q' \in \F(W)$. Now, \[ \res^\F_{W \cap V, W} \circ f_W(q') = f_{W \cap V} \circ \res^\G_{W \cap V, W}(q') = f_{W \cap V}(q) = d \]
Therefore, 
\[ \res^\G_{W\cap V, W} (t_x - f_W(q')) = t_x |_{W \cap V} - d = t_{W \cap V} \]
Thus $t_x - f_W(q') \in \G(W)$ and $t \in \G(V)$ agree on the overlap and thus glue to a section $t' \in \G(W \cup V)$ by the sheaf property of $\G$. Furthermore, let $s' = g_{W \cup V}(t') \in \F(W \cup V)$. Then by exactness,
\[ s'|_W = \res^\H_{W, W \cup V} \circ g_{W \cup V}(t') = g_W(t'|_W) = g_W(t_x - f_W(q')) = g_W(t_x) = s|_W \]
and likewise,
\[ s'|_V = \res^\H_{V, W \cap V} \circ g_{W \cup V}(t') = g_V(t'|_V) = g_V(t) = s|_V \]
Then $g_{W \cup V}(t') = s' = s|_{W \cup V}$ since they restrict to the same sections on the open cover $W, V$ of $W \cup V$ so $(W \cup V, t') \in \mathcal{T}$. However, $W \cup V \supset W$ and, by construction, $t'|_V = t$  contradicting the maximality of $(V, t)$ unless $V = W \cup V$ i.e. $W \subset V$. Since $W$ was, by construction, a neighborhood of $x$, then for each $x \in U$ we have $x \in V \subset U$ so $V = U$ proving the claim.


\item[(c)] Suppose that,
\begin{center}
\begin{tikzcd}
0 \arrow[r] & \F \arrow[r, "f"] & \G \arrow[r, "g"] & \H \arrow[r] & 0
\end{tikzcd}
\end{center}
is an exact sequence of sheaves over $X$ with $\F$ and $\G$ flasque. Now for any open sets $V \subset U \subset X$, consider the commuative diagram,
\begin{center}
\begin{tikzcd}
0 \arrow[r] & \F(U) \arrow[d, two heads, "\res^\F_{V,U}"'] \arrow[r, "f_U"] & \G(U) \arrow[r, "g_U"] \arrow[d, two heads, "\res^\G_{V,U}"'] & \H(U) \arrow[d, "\res^\H_{V,U}"'] \arrow[r] & 0
\\
0 \arrow[r] & \F(V) \arrow[r, "f_{V}"] & \G(V) \arrow[r, "g_{V}"] & \H(V) \arrow[r] & 0
\end{tikzcd}
\end{center}
where the rows are exact by part (b) since $\F$ is flasque and the first two downward maps are surjective because $\F$ and $\G$ are flasque. Given a section $s \in \H(V)$ we can lift $s$ under $g_V$ (which is a surjection since $\F$ is flasque) and under $\res^\G_{V,W}$ (which is a surjection since $\G$ is flasque) to get a section $s' \in \G(U)$. By the commutativity of the diagram,
\[ \res^\H_{V, U} \circ g_U(s') = g_V \circ \res^\G_{V,U}(s') = s \]
Therefore the restriction map $\res^\H_{V, U} : \H(U) \to \H(V)$ is surjective so $\H$ is flasque. 

\item[(d)] Let $f : X \to Y$ be a continuous map and $\F$ a flasque sheaf on $X$. Then consider the sheaf $f_* \F$ on $Y$. For open sets $V \subset U \subset Y$, we have restriction maps,
\[ \res^{f_* \F}_{V,U} : f_* \F(U) \to f_* \F(V) \quad \text{given by} \quad \res^{\F}_{f^{-1}(V), f^{-1}(U)} : \F(f^{-1}(U)) \to \F(f^{-1}(V)) \]
which is surjective since $\F$ is flasque. Therefore, $f_* \F$ is flasque. 

\item[(e)] Let $\F$ be a sheaf on $X$. Consider the sheaf $\G$ constructed by sending open sets $U \subset X$ to the maps,
\[ s : U \to \coprod_{x \in U} \F_x \quad \text{such that} \quad \forall x \in U : s(x) \in \F_x \]
or equivalently,
\[ U \mapsto \prod_{x \in U} \F_x \]
This sheaf is globally,
\[ \G = \prod_{x \in X} (\iota_x)_*(\F_x) \]
where $\iota_x : \{x\} \to X$ is the inclusion of the point and $\F_x$ is viewed as a constant sheaf over $\{ x \}$. For open sets $V \subset U \subset X$, consider the restriction maps,
\[ \res^\G_{U, V} : \G(U) \to \G(V) \quad \text{given by} \quad \prod_{x \in V} \pi_x : \prod_{x \in U} \F_x \to \prod_{x \in V} \F_x \]
Clearly, this map is surjective so $\G$ is flasque. Furthermore, consider the canonical morphism $\F \to \G$ which is locally
\[ \F(U) \to \prod_{x \in U} \F_x \]
defined by mapping $s \in \F(U)$ to its image in the stalk at each $x \in U$. Suppose that $s \in \F(U)$ maps to zero under this canonical map i.e. that the image of $s$ in $\F_x$ is zero at each $x \in U$. Then there exists an open neighborhood of each $x \in U$ on which $s$ restricts to zero. Thus by locality of the sheaf $\F$ we have $s = 0$ since their restrictions are equal on an open cover of $U$. 
\end{enumerate}

\begin{proposition}
Flasque abelian sheaves on a space $X$ are $\Gamma(X, -)$-acyclic.
\end{proposition}

\begin{proof}
Let $\F$ be a flasque abelian sheaf on $X$. Since the category of abelian sheaves on $X$ has enough injectives we may form an exact sequence of sheaves on $X$,
\begin{center}
\begin{tikzcd}
0 \arrow[r] & \F \arrow[r] & \I \arrow[r] & \G \arrow[r] & 0
\end{tikzcd}
\end{center} 
where $\I$ is injective. Now both $\F$ and $\I$ are flasque so $\G$ is also flasque. Since $\F$ is flasque, applying the functor $\Gamma(X, -)$ we get an exact sequence,
\begin{center}
\begin{tikzcd}
0 \arrow[r] & \Gamma(X, \F) \arrow[r] & \Gamma(X, \I) \arrow[r] & \Gamma(X, \G) \arrow[r] & 0
\end{tikzcd}
\end{center} 
Furthermore, applying the long exact cohomology sequence we get,
\begin{center}
\begin{tikzcd}[column sep = small]
0 \arrow[r] & \Gamma(X, \F) \arrow[r] & \Gamma(X, \I) \arrow[r] & \Gamma(X, \G) \arrow[r] & H^1(X, \F) \arrow[r] & H^1(X, \I) \arrow[draw=none]{d}[name=Z, shape=coordinate]{} \arrow[r] & H^1(X, \G)
\arrow[dlllll,
rounded corners, crossing over,
to path={ -- ([xshift=2ex]\tikztostart.east)
|- (Z) [near end]\tikztonodes
-| ([xshift=-2ex]\tikztotarget.west)
-- (\tikztotarget)}]
\\ 
& H^2(X, \F) \arrow[r] & H^2(X, \I)  \arrow[r] & H^2(X, \G) \arrow[r] & H^3(X, \F) \arrow[r] & H^3(X, \I) \arrow[r] & H^3(X, \G) \arrow[r] & \cdots
\end{tikzcd}
\end{center}
Since $\I$ is an injective sheaf, $H^r(X, \I) = 0$ for $r > 0$ which gives an exact sequence,
\begin{center}
\begin{tikzcd}
0 \arrow[r] & \Gamma(X, \F) \arrow[r] & \Gamma(X, \I) \arrow[r] & \Gamma(X, \G) \arrow[r] & H^1(X, \F) \arrow[r] & 0 
\end{tikzcd}
\end{center}
and isomorphisms $H^r(X, \G) \cong H^{r+1}(X, \F)$ for $r > 0$. Combining this exact sequence with the earlier one derived from the flasqueness of $\F$ shows that the cokernel of $\Gamma(X, \I) \to \Gamma(X, \G)$ is zero and thus $H^1(X, \F) = 0$. Since $\G$ is also a flasque sheaf on $X$ we can use the isomorphisms $H^{r+1}(X, \F) \cong H^r(X, \G)$ for $r > 0$ to show that $H^r(X, \F) = 0$ for all $r > 0$ by induction.  
\end{proof}

\begin{proposition}
Let $(X, \struct{X})$ be a ringed space. The derived functors of $\Gamma(X, -)$ computed over the category $\Ab(X)$ of sheaves of abelian groups on $X$ and those computed over the category $\Mod{\struct{X}}$ of $\struct{X}$-modules agree. 
\end{proposition}

\begin{proof}
There are enough injectives in the category of $\struct{X}$-modules. Taking an injective resolution of $\struct{X}$-modules is a resolution  of flasque sheaves of abelian groups which we have shown computes the derived functors of $\Gamma(X, -)$ in the full category $\Ab(X)$ since flasque sheaves are acyclic. 
\end{proof}

\subsubsection{1.17}

Let $x \in X$ be some point and $\iota_x : \{ x \} \to X$ the inclusion. Then consider the sheaf $\iota_x(A) = (\iota_x)_*(\underline{A})$ where $\underline{A}$ is the constant sheaf on $\{ x \}$. Now for any open $U \subset X$, we have,
\[ \iota_x(A)(U) = \underline{A}(\iota_x^{-1}(U)) = \begin{cases}
A & x \in U
\\
0 & x \notin U
\end{cases} \] 
Now consider the stalks,
\[ \iota_x(A)_y = \lim_{y \in U} \iota_x(A)(Y) \]
If there exists some open $U$ containing $y$ but not $x$ then $\iota_x(A)_y = 0$. Otherwise, for any open with $y \in U$ then $x \in U$ so $\iota_x(A)(U) = A$ and thus $\iota_x(A)_y = A$. Furthermore, there exists such an open exactly when $y$ is not a limit point of $x$ and not equal to $x$ i.e. $y \notin \overline{\{ x \}}$. Therefore, 
\[ \iota_x(A)_y = \begin{cases}
A & y \in \overline{\{ x \}} 
\\
0 & y \notin \overline{\{ x \}}
\end{cases} \]

\subsubsection{1.18}

Let $f : X \to Y$ be a continuous map, $\F$ a sheaf on $X$ and $\G$ a sheaf on $Y$. The restriction maps define a map,
\[ \varinjlim_{V \supset f(U)} \F(f^{-1}(V)) \to \F(U) \]
since $f^{-1}(V) \supset U$ gives restriction maps $\F(f^{-1}(V)) \to \F(U)$ compatilbe with restricton. Sheafifying gives a natural map $f^{-1} f_* \F \to \F$. Furthermore, we can define a map $\F \to f_* f^{-1} \F$ as follows. Consider the sheafification map $(f^{-1} \F)^P \to f^{-1} \F$ giving $\F \to f_* (f^{-1} \F)^P \to f_* f^{-1} \F$. The first map is defined by,
\[ \F(U) \to \varinjlim_{V \supset f(f^{-1}(U))} \F(V) \]
given since $U \supset f(f^{-1}(U))$ and then take the inclusion map of the colimit. These maps are natural. We have produced two natural transformations, a unit $\eta : \id \to f_* f^{-1}$ and a counit $\epsilon : f^{-1} f_* \to \id$. Now we use the following proposition.

\begin{prop}
Let $F : \C \to \D$ and $G : \D \to \C$ be functors with unit and counit natural transformations $\eta : 1_{\C} \to G \circ F$ and $\epsilon : F \circ G \to \id_{\D}$ satisfying the coherence relations that,
\begin{center}
\begin{tikzcd}
F \arrow[r, "F \eta"] & F \circ G \circ F \arrow[r, "\epsilon F"] & F
\\
G \arrow[r, "\eta G"] & G \circ F \circ G \arrow[r, "G \epsilon"] & G
\end{tikzcd}
\end{center}
are the identity $\id_F$ and $\id_G$ respectively. 
Then $F$ is left-adjoint to $G$.
\end{prop}

\begin{proof}
We construct a natural transformation $\phi_{X,Y} : \Hom{\D}{F(X)}{Y} \to \Hom{\C}{X}{G(Y)}$ via sending,
\[ (f : F(X) \to Y) \mapsto (G(f) \circ \eta_X : X \to G\circ F(X) \to G(Y))\]
and a natural transformation $\psi_{X,Y} : \Hom{\C}{X}{G(Y)} \to \Hom{\D}{F(X)}{Y}$ given by sending,
\[ (g : X \to G(Y)) \mapsto (\epsilon_Y \circ F(g) : F(X) \to F \circ G(Y) \to Y) \]
I claim these are inverse to each other:
\[ f \mapsto G(f) \circ \eta_X \mapsto \epsilon_Y \circ FG(f) \circ F(\eta_X) \]  
However, $\epsilon : F \circ G \to \id_{\D}$ is a natural transformation so,
\begin{center}
\begin{tikzcd}
FG(F(X)) \arrow[d, "\epsilon_{F(X)}"] \arrow[r, "FG(f)"] & FG(Y) \arrow[d, "\epsilon_Y"]
\\
F(X) \arrow[r, "f"] & Y
\end{tikzcd}
\end{center}
commutes and therefore,
\[ \eta_Y \circ FG(f) \circ F(\eta_Y) = f \circ \eta_{F(X)} \circ F(\eta_X) = f \]
using the first coherence relation showing that $\psi _{X,Y} \circ \phi_{X,Y} = \id$. Furthermore, consider,
\[ g \mapsto \epsilon_Y \circ F(g) \mapsto G(\epsilon_Y) \circ GF(g) \circ \eta_X  \]
Hpwever, $\eta : \id_{\C} \to G \circ F$ is a natural transformation so,
\begin{center}
\begin{tikzcd}
X \arrow[r, "g"] \arrow[d, "\eta_X"] & G(Y) \arrow[d, "\eta_{G(Y)}"]
\\
GF(X) \arrow[r, "GF(f)"] & GF(G(Y))
\end{tikzcd}
\end{center}
commutes and therefore,
\[ G(\eta_Y) \circ GF(g) \circ \eta_X = G(\eta_Y) \circ \eta_{G(Y)} \circ g = g \]
by the second coherence relation showing that $\phi_{X,Y} \circ \psi_{X,Y} = \id$ and providing a natural isomorphism,
\begin{center}
\begin{tikzcd}
\Hom{\D}{F(X)}{Y} \arrow[r, "\phi_{X,Y}", shift left] & \Hom{\C}{X}{G(Y)} \arrow[l, "\psi_{X,Y}", shift left] 
\end{tikzcd}
\end{center}
\end{proof}
\noindent
Therefore it suffices to prove the coherence relations in our case. (DO THIS!)

\subsubsection{1.19}

Let $X$ be a topological space, $Z \subset X$ a closed subspace and $U = X \setminus Z$ open. Furthermore denote the inclusions $\iota : Z \to X$ and $j : U \to X$. 

\begin{enumerate}
\item Let $\F$ be a sheaf on $Z$. Then consider the sheaf $\iota_*(\F)$ on $X$. For $x \in Z$ we have, 
\[ (\iota_* \F)_x = \varinjlim_{x \in V} (\iota_* \F)(V) = \varinjlim_{x \in V} \F(V \cap Z) = \varinjlim_{x \in V \cap Z} \F(V \cap Z) = \F_x \]
where the equality holds because every open set of the subspace $Z$ is of the form $V \cap Z$  for some open $V \subset X$ and $x \in V \iff x \in V \cap Z$ since $x \in Z$.
For $x \notin Z$ then for any $x \in V \subset U$ we have $\iota_*(\F)(V) = \F(\varnothing) = 0$ so $(\iota_* \F)_x = 0$. 

\item Let $\F$ be a sheaf on $U$. Now consider the sheaf $j_! \F$ as the sheafification of the presheaf defined by,
\[ (j_! \F)^P(V) = \begin{cases}
\F(V) & V \subset U 
\\
0 & V \not\subset U
\end{cases} \]
The stalks of the sheaf $j_! \F$ are the same as those of the presheaf and thus may be computed as follows.
For $x \in U$ we have,
\[ (j_! \F)^P_x = \varinjlim_{x \in V} (j_! \F)^P(V) = \varinjlim_{x \in V \subset U} \F(V) = \F_x \]
because both direct limits satisfy the same universal properties. For $x \notin U$ then any open $V$ containing $x$ cannot be contained in $U$ so,
\[ (j_! \F)^P_x = \varinjlim_{x \in V} (j_! \F)^P(V) = 0 \]
Now suppose that $\G$ is some sheaf on $X$ such that $\G|_U = \F$ and for which $\G_x = \F_x$ for all $x \in U$ and $\G_x = 0$ otherwise. To prove that $\G = j_! \F$, it suffices to show that $\G = j_! (\G|_U)$ since $\G|_U = \F$ by assumption. Consider the inclusion map $(j_! \G|_U)^P \to \G$. Since $\G$ is a sheaf this inclusion factors uniquely through the sheafification as $(j_! \G |_U)^P \to j_! (\G|_U) \to \G$. By assumption, the inclusion $(j_! \G|_U)^P \to \G$ is an isomorphism on stalks since $\G_x = 0$ for $x \notin U$. Thus $j_! (\G|_U) \to \G$ is an isomorphism but $\G|_U = \F$ so we get an isomorphism $j_! \F \to \G$. 


\item Let $\F$ be a sheaf on $X$. By adjunction, there is a morphism $\F \to \iota_* \iota^* \F$.  By definition, $\iota^* \F = \F|_Z$ so we have a map $\F \to \iota_* (\F |_\Z)$ and the sheaf $\iota_* \iota^* \F$ has stalks,
\begin{align*}
(\iota_* \iota^* \F)_x & = 
\begin{cases}
(\iota^* \F)_x & x \in Z
\\
0 & x \notin Z
\end{cases}
\\
& = 
\begin{cases}
\F_x & x \in Z
\\
0 & x \notin Z
\end{cases}
\end{align*}
On stalks at $x \notin Z$ this gives $\F_x \to 0$ and on stalks at $x \in Z$ it gives the identity $\F_x \to \F_x$. Furthermore, we have shown there exists a map $j_! (\F |_U) \to \F$ above which is an isomorphism on stalks at $x \in U$ and is the map $0 \to \F_x$ on stalks at $x \notin U$. Thus consider the sequence,
\begin{center}
\begin{tikzcd}
0 \arrow[r] & j_! (\F |_U) \arrow[r] & \F \arrow[r] & \iota_* (\F|_Z) \arrow[r] & 0
\end{tikzcd}
\end{center}
On stalks at $x \in Z$ this sequence is,
\begin{center}
\begin{tikzcd}
0 \arrow[r] & 0 \arrow[r] & \F_x \arrow[r, "\id"] & \F_x \arrow[r] & 0
\end{tikzcd}
\end{center}
and on stalks at $x \notin Z$ i.e. $x \in U$ this sequence is,
\begin{center}
\begin{tikzcd}
0 \arrow[r] & \F_x \arrow[r, "\id"] & \F_x \arrow[r] & 0 \arrow[r] & 0
\end{tikzcd}
\end{center}
both of which are exact so the sequence of sheaves is exact.
\end{enumerate}

\subsubsection{1.20}

Let $Z \subset X$ be closed and $\F$ a sheaf on $X$. We say a section $s \in \F(X)$ has support in $Z$ if $\Supp{\F}{s} \subset Z$. In that case $s |_{X \setminus Z} = 0$ since for each $x \in X \setminus Z$ we have $x \notin \Supp{\F}{s}$ so $s_x = 0$ so $s |_{X \setminus Z} = 0$ by seperatedness. Conversely, if $s|_{X \setminus Z} = 0$ then for any $x \in X \setminus Z$ we have $s_x = 0$ so $x \notin \Supp{\F}{s}$ and thus $\Supp{\F}{s} \subset Z$. We denote the subgroup of $\Gamma(X, \F)$ of sections with support in $Z$ by $\Gamma_Z(X, \F)$.

\begin{enumerate}
\item Consider the presheaf $V \mapsto \Gamma_{Z \cap V}(V, \F|_V)$. Let $U \subset X$ be an open set and $\{ V_i \}$ be an open conver of $U$. Suppose that $s \in \Gamma_{Z \cap U}(U, \F|_U)$ is a section on $U$ with support in $Z \cap U$ such that $s|_{V_i} = 0$. Then since $\F$ is a sheaf $s = 0$. Furthermore, given sections $s_i \in \Gamma_{Z \cap V_i}(V_i, \F|_{V_i})$ with supports in $Z \cap V_i$ which agree on the overlaps, then since $\F$ is a sheaf, these sections glue to give $s \in \Gamma(X, \F)$. It suffices to prove that $s$ has support in $Z \cap U$. We know that $s |_{V_i} = s_i$ and thus for $x \in V_i$ we have $s_x = (s_i)_x$. Thus, 
\[ \Supp{\F}{s} = \bigcup_{i \in I} \Supp{\F}{s_i} \subset \bigcup_{i \in I} Z \cap V_i \subset Z \cap U \]
so this is a sheaf which we denote $\H^0_Z(\F)$. 
\item Let $U = X \setminus Z$ and $j : U \to X$ be the inclusion. Consider the map $\F \to j_* (\F |_U)$ given by adjunction. For a section $s \in \F(V)$ on some open set $V \subset X$ to be in the kernel we must have $s_x \mapsto 0$ at each stalk. Consider,
\begin{align*}
(j_* (\F |_U))_x = \varinjlim_{x \in V} \F|_U(U \cap V) = \varinjlim_{x \in V} \F(U \cap V) 
\end{align*}
Thus, if $x \in Z$ then $(j_* (\F |_U))_x = 0$. Otherwise, if $x \in U$, suppose that the map $\F_x \to (j_* (\F |_U))_x$ take $s_x \mapsto 0$. Then $s|_{U \cap V} = 0$ on some $V$ meaning that $s_x = 0$ since $x \in U \cap V$. Therefore, the map $\F_x \to (j_* (\F |_U))_x$ is injective for $x \in U$. Thus $s$ is in the kernel exactly when $s_x = 0$ for each $x \in U$ i.e. $\Supp{\F}{s} \subset Z$ so $\H^0_Z(\F)$ is the kernel of the map $\F \to j_* (\F|_U)$ making the following sequence exact,
\begin{center}
\begin{tikzcd}
0 \arrow[r] & \H^0_Z(\F) \arrow[r] & \F \arrow[r] & j_* (\F|_U) 
\end{tikzcd}
\end{center}
Furthermore, if $\F$ is flasque then the restriction map $\F(V) \to \F(U \cap V)$ is surjective meaning that the stalk maps $\F_x \to (j_* (\F |_U))_x$ are surjective which implies that the morphism of sheaves $\F \to j_* (\F |_U)$ is surjective.
\end{enumerate}

\subsubsection{1.21 (DO THIS ONE!!)}

Let $X$ be a variety over an algebraically closed field (as in Ch I) and let $\struct{X}$ be the sheaf of regular functions on $X$.

\begin{enumerate}
\item Let $Y \subset X$ be closed. For each open $U \subset X$ let $\J_Y(U) \subset \struct{X}(U)$ be the ideal of regular functions vanishing on $Y \cap U$. This is a sheaf because uniqueness is automatic for subpresheaves of a sheaf and if functions vanihsing on $Y$ glue then their values are preserved so they still vanish on $Y$.

\item Let $Y$ be a subvariety. 
\end{enumerate}

\subsubsection{1.22 (DONE IN BRIAN'S NOTES)}

\renewcommand{\U}{\mathfrak{U}}

Let $X$ be a topolgical space and $\U = \{ U_i \}$ an open cover. Suppose for each $U_i$ we are given a sheaf $\F_i$ and for each $(i,j)$ an isomorphism $\varphi_{ij} : \F_i |_{U_i \cap U_j} \iso \F_j|_{U_i \cap U_j}$ such that $\varphi_{ii} = \id$ and $\varphi_{ik} = \varphi_{jk} \circ \varphi_{ij}$ on $U_i \cap U_j \cap U_k$. 



\subsection{2}

\subsubsection{2.2 (DO THIS ONE!!)}

\begin{proposition}
Let $A$ be a ring. Then $A_\red = A / \nilrad{A}$ is reduced.
\end{proposition}

\begin{proof}
Take $f \in A$ then if $f^n \in \nilrad{A}$ then $f \in \sqrt{\nilrad{A}} = \nilrad{A}$ since $\nilrad{A} = \sqrt{(0)}$ is a radical idel. Thus, if $f^n = 0$ in $A_\red$ then $f = 0$ in $A_\red$. 
\end{proof}

\subsubsection{2.3 (BRIAN ADDITIONS)}

\begin{enumerate}
\item 
Let $X$ be a reduced schemes. Take $x \in X$ and consider the stalk,
\[ \stalk{X}{x} = \varinjlim_{x \in U} \struct{X}(U) \]
Each $\struct{X}(U)$ is a reduced ring so if $f \in \stalk{X}{x}$ satisfies $f^n = 0$ then on some open neighborhood $x \in U$ we have $f^n = 0$ and thus $f = 0$ on $U$ which shows that $f = 0$ in $\stalk{X}{x}$. Conversely, if all stalks are reduced then for any open set $U \subset X$ conisder an element $f \in \struct{X}(U)$. If $f^n = 0$ then $f^n = 0$ in each stalk $\stalk{X}{x}$ at $x \in U$ which implies $f = 0$ since $\stalk{X}{x}$ is reduced. Thus $f = 0$ in $\struct{X}(U)$ so $X$ is reduced. Thus,
\[ X \text{ is reduced} \iff \forall x \in X : \stalk{X}{x} \text{ is reduced} \]

\item Let $(X, \struct{X})$ be a scheme. Let $(\struct{X})_\red$ be the sheaf associated to the presheaf $U \mapsto \struct{X}(U)_\red$. Consider the ringed space $X_\red = (X, (\struct{X})_\red)$ which is locally ringed because the stalks of $(\struct{X})_\red$ are $(\stalk{X}{x})_\red$ which are reduced rings. Furthermore, there is a morphism of locally ringed spaced $(\id_X, f^\#) : X_\red \to X$ where $f^\#$ is the natural morphism of sheaves induced by the sheafification of the morphism of presheaves $\struct{X}(U) \to \struct{X}(U)_\red$. This is indeed a morphism of locally ringed spaces because the induced map $\stalk{X}{x} \to (\stalk{X}{x})_\red$ is local. It suffices to show that $X_\red = (X, (\struct{X})_\red)$ is indeed a scheme. Let $U_i = \Spec{A_i}$ be an affine cover of $X$ then I claim that $\tilde{U}_i = \Spec{(A_i)_\red}$ is an affine cover of $X_{\red}$. Firstly, $A$ and $A_\red$ have the same prime ideals because all primes lie above $\nilrad{A}$ so $U_i = \tilde{U}_i$ as topological spaces. Furthermore, the structure sheaf $\struct{\Spec{(A_i)_\red}}$ has exactly the correct structure to be the unique sheaf $(\struct{\Spec{A_i}})_\red = (\struct{X})_\red|_{U_i}$. Therefore, this cover is affine. 
\bigskip\\
To be clever, define the sheaf of ideals $\mathcal{N}_X$ to be the kernel of the sheaf map $\struct{X} \to (\struct{X})_\red$ or alternatively the sheaf associated to the presheaf 
\[ \sN_X(U) = \nilrad{\struct{X}(U)} \] 
Then there is an exact sequence of sheaves,
\begin{center}
\begin{tikzcd}
0 \arrow[r] & \sN_X \arrow[r] & \struct{X} \arrow[r] & (\struct{X})_\red \arrow[r] & 0
\end{tikzcd}
\end{center}
Then $X_\red$ is a closed subscheme of $X$. 

\item Let $f : X \to Y$ be a morphism of schemes and assue that $X$ is reduced. Consider the cokernel diagram,
\begin{center}
\begin{tikzcd}[column sep = large, row sep = large]
\sN_Y \arrow[r] & \struct{Y} \arrow[d, "\iota^\#"'] \arrow[r, "f^\#"] & f_* \struct{X}
\\
& (\struct{Y})_\red \arrow[ru, dashed, "g^\#"']
\end{tikzcd}
\end{center}
The top row composes to zero because on the stalks $(\sN_Y)_y \to (f_* \struct{X})_y$ the ring $(\sN_Y)_y = \nilrad{\stalk{Y}{y}}$ which only contains nilpotent elements. Furthermore, $X$ is reduced so $(f_* \struct{X})_y$ is a limit of reduced rings and thus reduced. Thus the image of $f$ is nilpotent in $(f_* \struct{X})_y$ and therefore zero. Thus the map of sheaves $\sN_Y \to f_* \struct{X}$ is zero so it factors through the cokernel $(\struct{Y})_\red$ uniquely as $g^\# : (\struct{Y})_\red \to f_* \struct{X}$. Therefore, the morphism of schemes $f : X \to Y$ factors via a unique morphism $g : X \to Y_\red$ with $g = (f, g^\#)$ through the closed immersion $Y_\red \to Y$.
\end{enumerate}

\subsubsection{2.4 (DONE IN CLASS AND ALSO WORKS MORE GENERALLY)}

\subsubsection{2.5}

$\Spec{\Z}$ has a point for each prime number $p \in \Z$ and a unique generic point corresponding to the ideal $(0) \subset \Z$.
\bigskip\\
Because $\Z$ is the initial object in the category of rings and,
\[ \Hom{\mathrm{LRS}}{X}{\Spec{\Z}} = \Hom{\mathrm{Ring}}{\Z}{\Gamma(X, \struct{X})} \]
there is a unique morphism $X \to \Spec{\Z}$ for any locally ringed space $X$ so $\Spec{\Z}$ is the final object in the category of locally ringed spaces and likewise in the category of schemes (which is a full subcategory).

\subsubsection{2.6}

Let $0$ denote the zero ring. Then $\Spec{0} = \empty$ with the unique sheaf $\empty \to 0$. This is initial for the category of schemes because there is a unique toplogical map $\empty \to X$ (the empty map) and a unique ring map $\struct{X}(U) \to 0$. Thus, $\Spec{0}$ is the initial object in the category of schemes (actually also locally ringed spaces).

\subsubsection{2.7}

Let $X$ be a scheme. For any $x \in X$, let $\stalk{X}{x}$ be the local ring at $x$ and $\m_x \subset \stalk{X}{x}$ its maximal ideal and $\kappa(x) = \stalk{X}{x} / \m_x$ the resuidue field. A morphism $\Spec{K} \to X$ gives a point $x \in X$ (the image of the unique point of $\Spec{K}$) and a local map $f : \stalk{X}{x} \to K$ meaning $\m_x = f^{-1}((0)) = \ker{f}$ so this factors uniquely through $\kappa(x) \embed K$. Conversely, given the data $(x \in X, \kappa(x) \embed K)$ we construct the map $\Spec{K} \to X$ affine locally via given a affine open $U = \Spec{A}$ containing $x$ then the map is given by $A \mapsto A_\p = \stalk{X}{x} \to \kappa(x) \to K$ and taking spectra gives $\Spec{K} \to \Spec{A} \embed X$. These are inverse to eachother because the factorization $\stalk{X}{x} \to \kappa(x) \to K$ is unique and a map $\Spec{K} \to X$ is determined by where it sends the unique point and the map on rings $A \to K$ which must factor through $\stalk{X}{x} \to K$ by compatiblity with restriction. 

\subsubsection{2.8 (THIS IS DONE BETTER IN MY NOTES)}

Let $X$ be a scheme. Let $T_x = \Hom{\kappa(x)}{\m_x / \m_x^2}{\kappa(x)}$. Now let $X$ be a $k$-scheme and let $D = k[\epsilon]/(\epsilon^2)$ be the ring of dual numbers over $k$.
\bigskip\\
Because $D$ is local, the data of a map $\Spec{D} \to X$ is the same as the data of a point $x \in X$ and a local map $\stalk{X}{x} \to D$ and thus $\kappa(x) \embed k$ as a $k$-algebra so $\kappa(x) = x$. Furthermore, $\stalk{X}{x} \to D$ must send $\m_x \mapsto (\epsilon)$ because this is local and because $\epsilon^2 = 0$ we get a map $\m_x / \m_x^2 \to k \epsilon$ and thus an element of $T_x$. Conversely, we construct a map $\stalk{X}{x} \to \stalk{X}{x} / \m_x^2 \to D$ using $\varphi : \m_x / \m_x^2 \to \epsilon$. This works because,
\begin{center}
\begin{tikzcd}
0 \arrow[r] & \m_x / \m_x^2 \arrow[r] & \stalk{X}{x} / \m_x^2 \arrow[r] & \kappa(x) \arrow[r] & 0
\end{tikzcd}
\end{center} 
splits as $k$-algebras because $\stalk{X}{x}/\m_x^2$ is equipped with a map $k \to \stalk{X}{x}/ \m_x^2$. 

\subsubsection{2.9 (THIS IS DONE IN MY NOTES)}


\subsubsection{2.10}

Consider $\Spec{\RR[x]}$. The closed points have residue field a finite extension of $\RR$ and thus are either $\RR$ or $\CC$. For $\RR[x] \onto \RR$ this evaluates at $x$ and thus corresponds to the ideal $(x - r)$ for $r \in \RR$. For $\RR[x] \onto \CC$ the minimal polynomial of $x$ is a quadratic so we have $\RR[x]/(m_\alpha(x))$ for $\alpha \in \CC \setminus \RR$. Finally, there is a unique generic point $(0) \subset \RR[x]$. Notice that $m_\alpha(x) = (x - \alpha)(x - \overline{\alpha})$ for $\alpha \in \CC \setminus \RR$ therefore we get a point for every pair of Galois conjugate elements in $\CC$. Thus $\Spec{\RR[x]}$ is $\CC / \Gal{\CC/\RR}$ plus a unique generic point with the Zariski topology.

\subsubsection{2.11}

Let $k = \FF_q$ be the finite field with $q$ elements. Then $\Spec{k[x]}$ has points corresponding to the prime ideals of $k[x]$. Because $k[x]$ is a PID we have a unique generic point $(0)$ and the other ideals are closed generated by an irreducible polynomial $f \in k[x]$. Thus the closed points correspond to ideals of the form $(f) \subset k[x]$ with residue field $k[x]/(f)$ which run over the finite extensions of $k$. For each $n$ there is a unique field $\FF_{q^n}$. We need to count the number of irreducible polynomials $f \in k[x]$ of degree $n$. This corresponds to the number of galois orbits of size $n$ in $\FF_{q^n}$. Let $a_n$ be the number of elements of $\FF_{q^n}$ that do not lie in any proper subfield containing $\FF_q$ (i.e. $\alpha \in \FF_{q^n}$ such that $\FF_{q^n} = \FF_q(\alpha)$). Then because there is a unique field $\FF_{q^d} \subset \FF_{q^n}$ for each $d \divides n$,
\[ q^n = \# \FF_{q^n} = \sum_{d \divides n} a_d \]
where by convention we set $a_1 = q$. Therefore, by M\"{o}bius inversion,
\[ a_d = \sum_{d \divides n} \mu(d) q^{\frac{n}{d}} \]
Then the number of monic irreducible polynomials of degree $n$ is the number of Galois orbits in $a_n$. Since every Galois orbit in $a_n$ has size $n$ we see that this number is,
\[ \# \{ \text{monic irreducible polynomials } f \in k[x] \} = \frac{a_n}{n} = \frac{1}{n} \sum_{d \divides n} \mu(d) q^{\frac{n}{d}} \]

\subsubsection{2.12 (ISNT THIS IN BRIAN'S NOTES)}

\subsubsection{2.13}


\begin{defn}
A topological space $X$ is Noetherian if every descending chain of closed sets stabilizes.
\end{defn}

\begin{lemma}
Subspaces of Noetherian subspaces are Noetherian.
\end{lemma}

\begin{proof}
Let $S \subset X$ with $X$ noetherian. Then the closed sets of $S$ are exactly $S \cap Z$ for $Z \subset X$ closed. Thus descending chains of closed sets in $S$ stabilize.
\end{proof}

\begin{defn}
A space is quasi-compact if every open cover has a finite subcover.
\end{defn}

\begin{lemma}
Noetherian spaces are quasi-compact. 
\end{lemma}

\begin{proof}
Let $U_{\alpha}$ be an open cover of $X$ which is Noetherian. Then consider the poset $A$ under inclusion of finite unions of the $U_\alpha$ all of which are open sets of $X$. Since $X$ is Noetherian any ascending chain of opens must stabilize so any chain in $A$ has a maximum. Then by Zorn's lemma $A$ has a maximal element which must be $X$ since the $U_\alpha$ form a cover. Therefore there exists a finite subcover.
\end{proof}

\begin{cor}
Every subset of a noetherian topological space is quasi-compact.
\end{cor}


\begin{enumerate}
\item From the above, if $X$ is noetherian then every subset is quasi-compact. Conversely, suppose that every open $U \subset X$ is quasi-compact. Suppose we have a chain,
\[ U_1 \subset U_2 \subset U_3 \subset \cdots \]
Then consider their union,
\[ U = \bigcup_{i = 1}^\infty U_i \]
This is open and $\{ U_i \}$ forms an open affine cover by definition. Because $U$ is quasi-compact this implies that there is a finite subcover which means exactly that the chain stabilizes at a finite stage.

\item Consider $X = \Spec{A}$. Let $\{ U_i \}$ be an open cover. Then we can refine this to an open cover $\{ D(f_i) \}$ by standard opens. Then, 
\[ X = \bigcup_{i \in I} D(f_i) \]
However, if $f_i \notin \p$ then $\sum_{i \in I} (f_i) \not\subset \p$. Conversely, if $\sum_{i \in I} (f_i) \not\subset \p$ then there is some element which must be a finite sum $g = a_1 f_1 + \cdots + a_n f_n \notin \p$ and thus at least one $f_i \notin \p$. Therefore,
\[ X = \bigcup_{i \in I} D(f_i) = D \left( \sum_{i \in I} (f_i) \right) \]
However, if an ideal $J$ is not contained in any prime then $I = A$ else it would be contained in a maximal ideal. Thus $1 \in \sum_{i \in I} (f_i)$ and therefore $1 = a_1 f_1 + \cdots + a_n f_n$ which implies that,
\[ X = D((f_1) + \cdots + (f_n)) = \bigcup_{i = 1}^n D(f_i) \]
and thus we find a finite subcover of the refinement and thus finitely many of the $U_i$ be involved since each $D(f_i)$ is contained fully inside some $U_j$. Thus $X$ is quasi-compact. 
\bigskip\\
However, consider $A = k[x_1, x_2, x_3, \dots]$ and then
\[ V(x_1) \supsetneq V(x_1, x_2) \supsetneq V(x_1, x_2, x_3) \supsetneq \cdots \]
is a chain that does not stabilize and therefore $\Spec{A}$ is not noetherian.

\item Let $A$ be noetherian. Then the closed sets of $\Spec{A}$ are exactly $V(I)$ for ideals $I \subset A$. Then if we have a chain,
\[ Z_1 \supset Z_2 \supset Z_3 \supset \cdots \]
with $Z_i = V(I_i)$ then $\sqrt{I_i} \subset \sqrt{I_{i+1}}$ and thus because $A$ is noetherian the chain of radical ideals stabilizes and therefore $Z_i = V(I_i) = V(\sqrt{I_i})$ stabilizes.

\item Let $A = k[x_1, x_2, \dots]/(x_1^2, x_2^2, \cdots)$ then $A$ is not noetherian but $A_{\red}$ is noetherian and $\Spec{A_{\red}}$ is homeomorphic to $\Spec{A}$ so $\Spec{A}$ is a noetherian topological space.
\end{enumerate}

\subsubsection{2.14}

\begin{enumerate}
\item Let $S$ be a graded ring. Suppose that $S_+$ consists of nilpotents then every prime ideal $\p \subset S$ contains $S_+$ and thus $\Proj{S} = \empty$. Now suppose that there is some $f \in S_+$ that is not nilpotent. Writing in terms of homogeneous parts $f = f_1 + \cdots + f_n$ we see that not all $f_i$ are nilpotent else $f$ would be nilpotent. Thus there is some homogeneous $f_i \in S_+$ that is not nilpotent. Then $D_+(f_i) \cong \Spec{S_{(f_i)}}$ and $S_{f_i}$ is nonzero because $f_i$ is not nilpotent so $S_{(f_i)}$ is nonzero (it contains $0$ and $1$ and $0 \neq 1$ because they are not equal in $S_f$) and thus $\Spec{S_{(f_i)}}$ is nonempty so $\Proj{S}$ is nonempty.

\item Let $\varphi : S \to T$ be a graded homomorphism of graded rings. Let $U = \{ \p \in \Proj{T} \mid \p \not\supset \varphi(S_+) \}$. Since $\p \supset \varphi(S_+)$ if and only if $\p$ contains the ideal $I$ generated by $\varphi(S_+)$ we see that $U = V_+(I)^C$ is open. More explicitly, for each $f \in S_+$ then $D_+(\varphi(f)) \subset U$ form an open cover because if $\p \not\supset \varphi(S_+)$ then there is some $f$ such that $\varphi(f) \notin \p$. 
\bigskip\\
We define the map as expected $\p \mapsto \varphi^{-1}(\p)$ which is clearly graded the only issue is if $\varphi^{-1}(\p) \supset S_+$. Notice that $\varphi^{-1}(\p) \supset S_+$ iff every $s \in S_+$ has $\varphi(s) \in S_+$ or equivalently $\p \supset \varphi(S_+)$. Therefore $U$ is exactly the locus on which this is defined. Notice that the ring maps $S_{(f)} \to T_{(\varphi(f))}$ for $f \in S_+$ give maps $\Spec{T_{(\varphi(f))}} \to \Spec{S_{(f)}} \embed \Proj{S}$ and these glue together because they are all localizations of $\varphi : S \to T$ giving a morphism $U \to \Proj{S}$.

\item Suppose that $\varphi_d : S_d \to T_d$ is an isomorphism for $d \ge d_0$. Now suppose that $\p \supset \varphi(S_+)$ for some homogeneous prime ideal $\p \subset T$. For any $q \in T_+$ we can assume that $q$ is homogeneous of degree $r \ge 1$ then $q^{d_0}$ has degree $rd_0$ so $q^{d_0} \in \varphi(S_+)$ because $\varphi_d$ is an isomorphism for all $d \ge d_0$ and thus $q^{d_0} \in \p$ so $q \in \p$ since $\p$ is prime and thus $\p \supset T_+$. Therefore, $U = \Proj{T}$ so we get a well-defined morphism $\varphi^* : \Proj{T} \to \Proj{S}$.
\bigskip\\
To check that $\varphi^*$ is an isomorphism, it suffices to check locally. Notice that for $f \in S_+$ we have $(\varphi^*)^{-1}(D_+(f)) = D_+(\varphi(f))$ because $f \notin \varphi^{-1}(\p) \iff \varphi(f) \in \p$. Then $\varphi^* |_{D_+(\varphi(f))}$ is given by $\Spec{T_{(\varphi(f))}} \to \Spec{S_{(f)}}$ from the ring map $\varphi : S_{(f)} \to T_{(\varphi(f))}$ sending $\frac{s}{f^n} \mapsto \frac{\varphi(s)}{\varphi(f)^n}$. I claim this is an isomorphism. Indeed, 
\[ \frac{s}{f^n} = \frac{s f^{d_0}}{f^{n + d_0}} \]
has degree $\ge d_0$ in the numerator and therefore because $\varphi_d : S_d \to T_d$ is an isomorphism we see that $S_{(f)} \to T_{(\varphi(f))}$ is an isomorphism.

\item Let $V$ be a projective variety (in the sense of Ch. I) with homogeneous coordinate ring $S$. Then $V \subset \P^n$ giving a surjection $k[x_0, \dots, x_n] \onto S$. Then each $S_{(x_i)}$ is the ring of regular functions on $V \cap \A^n$ and thus $t(V \cap \A^n) \cong \Spec{S_{(x_i)}}$ since we showed this for affine varieties (closed subspaces of $\A^n$). Since these form a cover, the morphism $t(V) \to \Proj{S}$ arising from the natural map $V \to \Proj{S}$ sending a point to the corresponding maximal homogeneous prime must be an isomorphism.
\end{enumerate}

\subsubsection{2.15 (DO THIS)}

\begin{enumerate}
\item Let $V$ be a variety over the algebraically closed field $k$. Then $t(V)$ is a scheme of finite type over $k$ so its closed points are exactly those with residue field $k$ (THIS IS SHOWN IN MY NOTES AND I THINK ITS A HARTSHORNE PROBLEM).

\item If $f : X \to Y$ is a morphism of schemes over $k$ and $P \in X$ is a point with $\kappa(P) = k$. Then the map $\stalk{Y}{f(P)} \to \stalk{X}{P}$ gives $\kappa(f(P)) \to \kappa(P)$. Since these are fields the map must be injective. Furthermore these are $k$-algebra maps and $k \to \kappa(f(P)) \to \kappa(P)$ is an isomorphism so $\kappa(f(P)) \to \kappa(P)$ is surjective and thus an isomorphism. 

\item Let $V, W$ are varieties over $k$ then the natural map,
\[ \alpha : \Hom{\mathbf{Var}}{V}{W} \to \Hom{\mathbf{Sch}_k}{t(V)}{t(W)} \]
If $\alpha(f) = \alpha(g)$ then clearly $f = g$ because $f$ is the restriction of $\alpha(f)$ on closed points. Now let $f : t(V) \to t(W)$ be a morphism of $k$-schemes. By the previous parts, $f$ restricts to $f_c : V \to W$ because it sends closed points to closed points. Now I claim that $\alpha(f_c) = f$. We use the fact that for a finite type $k$-scheme the closed points are dense (it is Jacobson) and by the reduced to separated theorem (USE SOMETHING MORE ELEMENTARY HERE) because $f$ and $\alpha(f_c)$ agree on the dense set of $k$-points (having $f(x) = f_c(x)$ suffices because $f^\# = f_c^\# : \kappa(x) \to \kappa(f(x))$ is automatic for closed points because $\kappa(x) = \kappa(f(x)) = k$ and these are morphisms of $k$-schemes so both must be the identity) and therefore $f = \alpha(f_c)$.
\end{enumerate}

\subsubsection{2.16}

Let $X$ be a locally ringed space and $f \in \Gamma(X, \struct{X})$. Define,
\[ X_f = \{ x \in X \mid f_x \notin \m_x \subset \stalk{X}{x} \} \]
\begin{enumerate}
\item For any $x \in X_f$ then $f_x \notin \m_x$ so $f_x \in \stalk{X}{x}^\times$ since the stalk is a local ring. Therefore, there exists some open $U  \subset X$ with $x \in U$ such that $f|_U$ is invertible $g \cdot f|_U = 1$. Under the restriction to $\stalk{X}{y}$ for any point $y \in U$ we have $g_y \cdot f_y = 1$ so $f_y \in \stalk{X}{y}$ is invertible and thus $f_y \notin \m_y$. Therefore, $x \in U \subset X_f$ so $X_f$ is open. Furthermore, since inverses are unique, the inverses of $f|_{U_x}$ for each $x \in X_f$ agree on overlaps and thus glue to an inverse of $f|_{X_f}$.
\bigskip\\
Furthermore, let $X$ be a scheme and $U = \Spec{B}$ be an affine open subscheme $U \subset X$ with $f|_U = \bar{f} \in \struct{X}(U) = B$. 
Consider,
\[ U \cap X_f = \{ \p \subset B \mid \bar{f} \notin \p B_\p \} \]
However, if $\bar{f} \in \p$ then $\bar{f} \in \p B_\p$ if $\bar{f} \notin \p$ then $\bar{f} \in B_\p^\times$ so $\bar{f} \notin \p B_\p$. Thus,
\[ U \cap X_f = \{ \p \subset B \mid \bar{f} \notin \p B_\p \} = \{ \p \in \Spec{B} \mid \bar{f} \notin \p \} = D(\bar{f}) \]
which is open in $U = \Spec{B}$. Thus we see again that $X_f$ is open.

\item Let $X$ be a quasi-compact scheme and $A = \Gamma(X, \struct{X})$. Take $a \in A$ such that $a|_{X_f} = 0$. Now take an affine open $U \subset X$ with $U = \Spec{B}$ and consider $a |_{U \cap X_f} = 0$ i.e. $\bar{a} |_{D(\bar{f})} = 0$. Therefore, $\bar{a} \in \struct{X}(U \cap X_f) = \struct{\Spec{B}}(D(\bar{f})) = B_{\bar{f}}$ is zero so $\bar{f}^n \bar{a} = 0$ for some $n$. Thus, on each affine open $U$ there is some $n$ such that $(f^n a)|_U = 0$. Now since $X$ is quasi-compact we may take a finite affine cover $\{ U_i \}$ of $X$ such that $(f^{n_i} a)|_{U_i} = 0$. Let $N = \max_i n_i$, which exists by the finiteness of the cover, such that $(f^N a)|_{U_i} = 0$ for each open $U_i$ implying that $f^N a = 0$.  

\item Suppose that $X$ has a finite affine open cover $\{ U_i \}$ with $U_i = \Spec{B_i}$ such that $U_i \cap U_j$ is quasi-compact. Let $b \in \Gamma(X_f, \struct{X_f})$. Now $b|_{U_i \cap X_f} \in (B_i)_{\bar{f}}$ and thus there exists $n_i$ such that $\bar{f}^{n_i}(b|_{U_i \cap X_f})$ is in the image of $B_i$. By finiteness of the cover, $n = \max_i n_i$ exists such that we may take $(f^{n} b)|_{U_i \cap X_f} = b_i |_{U_i \cap X_f}$ for some $b_i \in B_i$ i.e. some section $b_i \in \struct{X}(U_i)$. Now consider, $s_{ij} = (b_i - b_j)|_{U_i \cap U_j}$ which satisfies,
\[ s_{ij} |_{U_i \cap U_j \cap X_f} = b_i|_{U_i \cap U_j \cap X_f} - b_j|_{U_i \cap U_j \cap X_f} = (f^{n} b)|_{U_i \cap U_J \cap X_f} - (f^{n} b)|_{U_i \cap U_j \cap X_f} = 0 \]
By the quasi-compactness of $U_i \cap U_j$ we may apply the previous part to get some $n_{ij}$ such that $f^{n_{ij}} |_{U_i \cap U_j} s_{ij} = 0$. Using the finiteness of the cover again, we may take $m = \max_{ij} n_{ij}$ to find that $f^m |_{U_i \cap U_j} s_{ij} = 0$ and thus,
\[ (f^m b_i - f^m b_j) |_{U_i \cap U_j} = 0 \]
Therefore, the sections $f^m b_i \in B_i$ agree on overlaps and thus glue to a global section $a \in A = \Gamma(X, \struct{X})$. Furthermore, 
\[ \res_{U_i \cap X_f, X_f}(a|_{X_f}) = (f^m b_i)|_{U_i \cap X_f} = (f^{n+m} b)|_{U_i \cap X_f} \]
since $U_i \cap X_f$ is an open cover of $X_f$ we find that $a|_{X_f} = f^{n + m} b$ so $f^{n + m} b$ has a lift to a global section. 

\item With the above hypothesis, consider the restricton map 
\[ \res_{X_f, X} : \Gamma(X, \struct{X}) \to \Gamma(X_f, \struct{X_f}) \] 
under which $f$ is mapped to a unit. 
Therefore, this map factors uniquely through the localization, 
\[ r : A_f \to \Gamma(X_f, \struct{X_f}) \]
However, if $\res_{X_f,X}(a) = 0$ then $f^n a = 0$ for some $n$ i.e. $a = 0$ in $A_f$ so $\ker{r} = 0$. Furthermore, for any $b \in \Gamma(X_f, \struct{X_f})$ there is some $n$ such that $f^n b = \res_{X_f, X}(a)$ for $a \in A$. Thus, 
\[ r(a/f^n) = \res_{X_f, X}(a) / f^n = f^n b / f^n  = b \]
so $r$ is surjective making  $r$ an isomorphism giving $\Gamma(X_f, \struct{X_f}) \cong A_f$. Therefore,
\[ \Gamma(X_f, \struct{X_f}) \cong \Gamma(X, \struct{X})_f \]
\end{enumerate}

\subsubsection{2.17}

\begin{enumerate}
\item Suppose that $f : X \to Y$ is a morphism of schemes such that $Y$ can be covered by open subsets $U_i$ such that for each $i$, the induced map $f_i : f^{-1}(U_i) \to U_i$ is an isomorphism. Let $g_i : U_i \to f^{-1}(U_i)$ be its inverse. Note that on the overlaps $f_i$ and $f_j$ agree,
\[ f_i |_{f^{-1}(U_i \cap U_j)} = f_j |_{f^{-1}(U_i \cap U_j)} = f |_{f^{-1}(U_i \cap U_j)}  \]
therefore, by uniqueness inverses, we know that the maps $g_i$ also agree on overlaps,
\[ g_i |_{U_i \cap U_j} = g_j |_{U_i \cap U_j} \]
Therefore, these functons glue to give a map $g : Y \to X$ such that $g_i = g |_{U_i}$. Now consider,
\[ (g \circ f)|_{f^{-1}(U_i)} = g |_{u_i} \circ f |_{f^{-1}(U_i)} = g_i \circ f_i = \id_{f^{-1}(U_i)} \]
and likewise,
\[ (f \circ g)|_{U_i} = f|_{f^{-1}(U_i)} \circ g|_{U_i} = f_i \circ g_i = \id_{U_i} \]
Therefore $g \circ f = \id_X$ and $g \circ f = \id_Y$ since these functions are locally the identity.

\item Let $X$ be a scheme and $A = \Gamma(X, \struct{X})$. Suppose that $f_1, \dots, f_n \in A$ generate the unit ideal and further suppose that the open subsets $X_{f_i}$ are affine. First, the open sets $X_{f_i}$ cover $X$ since if $x \notin X_{f_i}$ then $f_i \in \m_x$ however $f_i$ generate the unit ideal so we cannot have $f_i \in \m_x$ for all $i$ so $x \in X_{f_i}$. There is a natural map $a : X \to \Spec{A}$ via adjunction of the identity on global sections. Consider the open cover $U_i = D(f_i)$ of $\Spec{A}$. For each open $U = U_i$ and $f = f_i$ consider the restriction of the map, $a : a^{-1}(U) \to U = D(f) = \Spec{A_{f}}$. Recall that $a(x) = \id^{-1} \circ \res^{-1}_x(\m_x) \in \Spec{A}$ so,
\[ f \in a(x) \iff f \in \res^{-1}_x(\m_x) \iff f_x \in \m_x \]
and therefore,
\[ x \in a^{-1}(U) \iff a(x) \in D(f) \iff f \notin a(x) \iff f_x \notin \m_x \iff x \in X_f \]
Thus, $a^{-1}(D(f)) = X_f$. However, by assumption, $X_{f}$ is an affine scheme so the map $a : X_{f} \to \Spec{A_{f}}$ is determined uniquely by the ring map on global sections $r : A_f \to \Gamma(X_f, \struct{X_f})$ which we have shown is an isomorphism. Thus $a : X_{f} \to \Spec{A_{f}}$ is an isomorphism of affine schemes for each $f$. Applying part (a) we find that $a : X \to \Spec{A}$ is an isomorphism so $X$ is affine.
\bigskip\\
Conversely, if $X$ is an affine scheme $X = \Spec{A}$ then take $f = 1 \in A$ which generates the unit ideal and $X_f = D(f) = \Spec{A}$ satisfying the criterion.
\end{enumerate}

\subsubsection{2.18}

\begin{enumerate}
\item Let $A$ be a ring and $X = \Spec{A}$ and $f \in A$. Then,
\[ f \in \nilrad{A} \iff \forall \p \in \Spec{A} : f \in \p \iff D(f) = 0 \]

\renewcommand{\P}{\mathfrak{P}}

\item Let $\varphi : A \to B$ be a homomorphism of rings, $X = \Spec{A}$ and $Y = \Spec{B}$, and let $f : Y \to X$ be the induced morphism of affine schemes. Suppose that $\varphi : A \to B$ is injective. Then the sheaf map $f^\# : \struct{X} \to f_* \struct{Y}$ on the standard open $D(g)$ is the map $A_g \to B_{f(g)}$ which is injective since $a / g^n \mapsto f(a) / f(g)^n$ is zero exactly when $f(g)^k f(a) = 0$ for some $k$ but $f(g)^k f(a) = f(g^k a) = 0$ thus $g^k a = 0$ by injectivity meaning that $a / g^n = 0$ in $A_g$. Therefore, the morphism of sheaves $f^\# : \struct{X} \to f_* \struct{Y}$ is injective. We may also check this on the stalks. For $\p \in \Spec{A}$ consider the stalk map $f^\#_\p : \stalk{X}{\p} \to (f_* \struct{Y})_\p$. Now,
\[ (f_* \struct{Y})_\p = \varinjlim_{g \notin \p} \struct{Y}(f^{-1}(D(g)) = \varinjlim_{g \notin \p} \struct{Y}(D(f(g)) = \varinjlim_{g \notin \p} B_{f(g)} = B_\p \]
where $B_\p = S_\p^{-1} B = B \otimes_A A_\p$. Since localization of $A$-modules is exact, the map $\varphi_\p : A_\p \to B_\p$ remains an injection. Thus the stalk maps are injections so $f^\# : \struct{X} \to f_* \struct{Y}$ is an injective morphism of sheaves.
\bigskip\\
Conversely, if $f^\# : \struct{X} \to f_* \struct{Y}$ is injective then it is injective on sections so in particular $f^\# : \struct{X}(X) \to \struct{Y}(Y)$ which is the map $\varphi : A \to B$ is injective.
\bigskip\\
\begin{lemma}
$\varphi^{-1}(\sqrt{I}) = \sqrt{\varphi^{-1}(I)}$
\end{lemma}
\begin{proof}
\begin{align*}
x \in \varphi^{-1}(\sqrt{I}) & \iff \varphi(x) \in \sqrt{I} \iff \varphi(x)^n = \varphi(x^n) \in I
\\
&  \iff x^n \in \varphi^{-1}(I) \iff x \in \sqrt{\varphi^{-1}(I)} 
\end{align*}
\end{proof}
\begin{lemma}
$\overline{f(V(I))} = V(\varphi^{-1}(I))$.
\end{lemma}
\begin{proof}
Consider $f(V(I)) \subset V(J)$ then $J \subset \varphi^{-1}(\P)$ for each prime ideal $\P \subset B$ above $I$ so $\varphi(J) \subset \sqrt{I}$. By the above lemma, $J \subset \varphi^{-1}(\sqrt{I}) = \sqrt{\varphi^{-1}(I)}$ and thus,
\[ V(J) \supset V(\sqrt{\varphi^{-1}(I)}) = V(\varphi^{-1}(I)) \] Furthermore, if $\p \supset f(V(I))$ then $\p = \varphi^{-1}(\P)$ with $\P \supset I$ so then $\p \supset \varphi^{-1}(I)$ and thus $\p \in V(\varphi^{-1}(I))$. Thus $f(V(I)) \subset V(\varphi^{-1}(I))$ which proves that $\overline{f(V(I))} = V(\varphi^{-1}(I))$.
\end{proof}

\begin{corollary}
$\overline{f(Y)} = V(\ker{\varphi})$ so $f$ is dominant iff $\ker{\varphi} \subset \nilrad{A}$.
\end{corollary} 
Therefore, in this case, $\ker{\varphi} = 0$ so $f$ is dominant. 

\item If $\varphi : A \to B$ is surjective then the stalk map $\varphi : A_{\varphi^{-1}(\P)} \to B_\P$ is clearly surjective because any $s'$ mapping to $s \in B \setminus \P$ lies in $\varphi^{-1}(B \setminus \P) = A \setminus \varphi^{-1}(\P)$. Thus, the sheaf map $f^\# : \struct{Y} \to f_* \struct{X}$ is surjective. Furthermore, let $I = \ker{\varphi}$ then $f : Y \to X$ is a homeomorphism of $Y$ to the closed subspace $V(I) \subset X$ by the lattice isomorphism theorem. 

\begin{proposition}
Let $\varphi : A \to B$ be a surjective map of rings with $I = \ker{\varphi}$. Then the induced map $f : \Spec{B} \to \Spec{A}$ is a homeomorphism onto its image, the closed subspace $V(I) \subset \Spec{A}$.
\end{proposition}

\begin{proof}
Define the map $g : V(I) \to \Spec{B}$ via $\p \mapsto \varphi(\p)$. We must show that this map is well-defined and continuous. However, first note that because $\varphi$ is surjective that 
$g \circ f(\P) = \varphi(\varphi^{-1}(\P)) = \P$
and $f \circ g(\p) = \varphi^{-1}(\varphi(\p))$ but,
\[ x \in \varphi^{-1}(\varphi(\p)) \iff \varphi(x) \in \varphi(\p) \iff \exists y \in \p : \varphi(x) = \varphi(y) \iff x \in \p + I \]
so if $\p \supset I$ then $f \circ g(\p) = \varphi^{-1}(\varphi(\p)) = \p$. Thus, these maps are inverses as maps of subsets.
\bigskip\\
Let $\p \supset I$ is prime, then $\varphi(\p)$ is an ideal because $\varphi$ is surjective. Furthermore, if $f(x) \cdot f(y) \in \varphi(\p)$ then $f(xy) \in \varphi(\p)$ so $xy \in \varphi^{-1}(\varphi(\p)) = \p$ implying that $x \in \p$ or $y \in \p$ and thus $f(x) \in \varphi(\p)$ or $f(y) \in \varphi(\p)$. Therefore, $\varphi(\p) \subset B$ is a prime ideal so $g$ is well-defined. 
\bigskip\\
Take an ideal $J \subset B$ corresponding to the closed subset $V(J) \subset \Spec{B}$. Consider,
\[ \p \in g^{-1}(V(J)) \iff \varphi(\p) \in V(J) \iff \varphi(\p) \supset J \iff \p \supset \varphi^{-1}(J) \iff \p \in V(\varphi^{-1}(J)) \]
where I have used the fact that $f$ and $g$ are inclusion preserving inverses and $\p \in V(I)$. Thus, $g^{-1}(V(J)) = V(\varphi^{-1}(J))$ which is closed in $V(I)$ because $\varphi^{-1}(J)$ is an ideal of $A$ containing $I$ so $V(I) \cap V(\varphi^{-1}(J)) = V(\varphi^{-1}(J))$. Therefore, $g : V(I) \to \Spec{B}$ is a continuous inverse of $f : \Spec{B} \to V(I)$. 
\end{proof}

\item Let $f : Y \to X$ be a morphism of affine schemes such that $f^\# : \struct{X} \to f_* \struct{Y}$ surjective. Consider the ring maps,
\begin{center}
\begin{tikzcd}
A \arrow[rr, "\varphi"] \arrow[rd, "\pi"'] & & B 
\\
& A / \ker{\varphi} \arrow[ru, "\tilde{\varphi}"']
\end{tikzcd}
\end{center}
Then consider the scheme $X' = \Spec{A / \ker{\varphi}}$ and the induced morphism of affine schemes,
\begin{center}
\begin{tikzcd}
X   & & Y \arrow[ll, "f"'] \arrow[dl, "\tilde{f}"] 
\\
& X' \arrow[ul, "p"]
\end{tikzcd}
\end{center}
These morphisms of schemes give a morphism of sheaves on $X$,
\begin{center}
\begin{tikzcd}
\struct{X} \arrow[rd, "p^\#"'] \arrow[rr, two heads, "f^\#"] & & f_* \struct{Y}
\\
& p_* \struct{X'} \arrow[ru, "p_* \tilde{f}^\#"', hook, two heads]
\end{tikzcd}
\end{center}
By assumption $f^\# : \struct{X} \to f_* \struct{Y}$ is surjective so $p_* \tilde{f}^\# : p_* \struct{X'} \to f_* \struct{Y}$ is surjective as well. Furthermore, the ring map $\tilde{\varphi} : A / \ker{\varphi} \to B$ is injective meaning that $\tilde{f}^\#$ is an injective morphism of sheaves and, since $p_*$ is a right-adjoint functor, $p_* \tilde{f}^\#$ is also injective. Therefore, $p_* \tilde{f}^\# : p_* \struct{X'} \to f_* \struct{Y}$ is a bijection of sheaves over $X$ and, in particular, surjective on sections i.e. in the sense of pre-sheaves. Furthermore, $\pi : A \to A / \ker{\varphi}$ is a surjection and thus $p^\#$ is surjective on global sections. Thus, the composition $f^\# = p_* \tilde{f}^\# \circ p^\#$ is surjective on global sections i.e. $f^\# : \struct{X}(X) \to (f_* \struct{Y})(X) = \struct{Y}(Y)$ which is the map $\varphi : A \to B$ is surjective. 
\end{enumerate}

\subsubsection{2.19}

Let $A$ be a ring. Suppose that $\Spec{A}$ is disconnected so there exist disjoint nonempty closed sets $V(I_1), V(I_2) \supset \Spec{A}$. Therefore, 
\[ V(I_1) \cap V(I_2) = V(I_1 + I_2) = \varnothing \implies I_1 + I_2 = A \]
and likewise,
\[ V(I_1) \cup V(I_2) = V(I_1 I_2) = \Spec{A} \implies I_1 I_2 \subset \nilrad{A} \]
Therefore, there must exist elements $e_1 \in I_1$ and $e_2 \in I_2$ such that $e_1 + e_2 = 1$ and furthermore $e_1 e_2 \in \nilrad{A}$. Note that,
\[ (e_1 + e_2)^n = e_1^2 + n e_1 e_2^{n-1} + \cdots + n e_1^{n-1} e_2 + e_n^n = 1 \]
Therefore, $1 - (e_1^n + e_2^n) \in \nilrad{A}$ so $e_1^n  + e_2^n \in A^\times$ and let $u \in A^\times$ be its inverse. Since $e_1 e_2$ is nilpotent there exists some $n \ge 0$ such that $(e_1 e_2)^n = 0$. Now set $\tilde{e}_1 = u e_1^n$ and $\tilde{e}_2 = u e_2^n$. Thus $\tilde{e}_1 + \tilde{e}_2 = u(e_1^n + e_2^n) = 1$ and $\tilde{e}_1 \tilde{e}_2 = u^2 e_1^n e_2^n = u^2 (e_1 e_2)^n = 0$. Finally, consider,
\begin{align*}
\tilde{e}_1 & = 1 \cdot \tilde{e}_1 = (\tilde{e}_1 + \tilde{e}_2) \tilde{e}_1 = \tilde{e}_1^2 + \tilde{e}_1 \tilde{e}_2 = \tilde{e}_1^2
\\
\tilde{e}_2 & = 1 \cdot \tilde{e}_2 = (\tilde{e}_1 + \tilde{e}_2) \tilde{e}_2 = \tilde{e}_1 \tilde{e}_2 + \tilde{e}_2^2 = \tilde{e}_2^2
\end{align*}
so $e_1$ and $e_2$ are perpendicular idempotent generators proving (i) $\implies$ (ii). 
\bigskip\\
First, note that because $e_i$ is idempotent the ideal $(e_i)$ is actually a ring with identity element $e_i$ since $e_i \cdot (a e_i) = a e_i^2 = a e_i$. Now, consider the ring map $\Phi : A \to (e_1) \times (e_2)$ via $a \mapsto (a e_1, a e_2)$ which indeed maps $1 \mapsto (e_1, e_2)$ the identity. Now suppose that $\Phi(a) = 0$ then $a e_1 = a e_2 = 0$ so $a = 1 \cdot a = (e_1 + e_2) \cdot a = 0$. Thus $\Phi$ is injective. Furthermore, for any $(a e_1, b e_2) \in (e_1) \times (e_2)$ consider the element $a e_1 + b e_2 \in A$. Then, 
\[ \Phi(a e_1 + b e_2) = (a e_1^2 + b e_2 e_1, a e_1 e_2 + b e_2^2) = (a e_1, b e_2) \]
so $\Phi$ is surjective. Thus $\Phi : A \xrightarrow{\sim} (e_1) \times (e_2)$ is an isomorphism.
\bigskip\\
Finally, suppose that $A = A_1 \times A_2$. Then $A_1, A_2 \subset A$ are ideals such that $A_1 A_2 = 0$ and $A_1 + A_2 = A$. Therefore in $\Spec{A}$ we have closed subsets $V(A_1)$ and $V(A_2)$ such that $V(A_1) \cup V(A_2) = V(A_1 A_2) = \Spec{A}$ and $V(A_1) \cap V(A_2) = V(A_1 + A_2) = V(A) = \varnothing$. Therefore, $\Spec{A}$ is disconnected. 

\renewcommand{\P}{\mathbb{P}}

\subsection{3}


\begin{rmk}
Notice that the following is \textit{false}. If $f : \Spec{A} \to \Spec{B}$ is an open immersion of affine schemes and $D(g) \subset \Spec{A}$ is a prinicpal affine open then $f(D(g))$ is a prinicpal affine open. For example, let $E$ be an elliptic curve minus the origin. Let $p \in E$ have infinite order. Then $E \setminus \{ p \}$ is affine open in $E$ but if $\{ p \} = V(f)$ (set theoretically) for some function $f \in \Gamma(E, \struct{E})$ then $p$ must be a torsion-point because some power of it is linearly equivalent to some power of the origin. On the other hand, we will see that a principal affine of the target which is contained in the image of the open immersion is a principal affine of the open.
\end{rmk}

\begin{lemma}
If $f : \Spec{A} \to \Spec{B}$ is an open immersion of affine schemes and $D(g) \subset f(\Spec{A})$ is a prinicpal affine open of $\Spec{A}$ then $f : D(f^\#(g)) \to D(g)$ is an isomorphism.
\end{lemma}

\begin{proof}
Because $f$ is an open immersion, since $D(g) \subset \im{f}$ then $f : f^{-1}(D(g)) \to D(g)$ is a homeomorphism but $f|_{f^{-1}(D(g))}$ is an open immersion so $f|_{f^{-1}(D(g))}$ is an isomorphism. Furthermore, $f^{-1}(D(g)) = D(f^\#(g))$ naturally.
\end{proof}

\begin{cor}[Nike]
Let $X$ be a scheme and $U, V \subset X$ affine opens. For each $x \in U \cap V$ there exists an affine open $x \in W \subset U \cap V$ such that $W$ is a prinicpal open of both $U$ and $V$.
\end{cor}

\begin{proof}
Since $U \cap V \subset V$ is open inside the affine $V$ then there is some $x \in D_V(g) \subset U \cap V \subset V$ with $D(g)$ a prinicpal affine open $V$. Then $D_V(g_V) \to U$ is an open immersion of affine schemes and $D_V(g_V) \subset U$ is open so we can take $x \in D_U(g_U) \subset D_V(g_V) \subset U$ which is a principal affine open of $U$. Then by the lemma since $D_U(g_U)$ is in the image of $D_V(g_V) \to U$ we see that $D_U(g_U)$ is a principal open of $D_V(g_V)$ and thus of $V$.
\end{proof}

\begin{rmk}
Indeed, $D_U(g_U) = D_V(g_{V}')$ where $g_V'$ is the image of $g_U$ under $\struct{X}(U) \to \struct{X}(D_V(g_V)) = (\struct{X}(V))_{g_V}$ after multiplying by $g_V$ sufficiently. Explicitly, if $U = \Spec{A}$ and $V = \Spec{B}$ and $f \in A$ such that $\Spec{A_f} \subset U \cap V$ then choose $g \in B$ such that $\Spec{B_g} \subset \Spec{A_f}$ which proves that $\Spec{B_g}$ is the prinicpal affine $\Spec{(A_f)_{g'}}$ of $A_f$ with $g'$ the image of $g$ in $B \to A_f$. Furthermore, if $g' = g'' / f^n$ then,
\[ (A_f)_{g'} = A_{fg''} \]
and thus $\Spec{B_g} = \Spec{A_{fg''}}$ is an affine open of $U$ and $V$.
\end{rmk}

\subsubsection{3.1}

Let $f : X \to Y$ be a morphism such that there exist affine open coverings $U_i = \Spec{A_i}$ and $V_i = \Spec{B_i}$ such that $f : U_i \to V_i$ makes $\varphi : B_i \to A_i$ finite type. Let $\Spec{B} = V \subset Y$ be any affine open and consider $V \cap V_i$ which is open and thus covered by affines $W_{ij} \subset V_i$ which are simultaneously principal affine opens of $V$ and $V_i$. Then, inside the affine $U_i$ we have $f^{-1}(V) \cap U_i$ covered by $f|_{U_i}^{-1}(W_{ij})$ which are principal affines because $f : U_i \to V_i$ is a map of affine schemes. Therefore, $f^{-1}(V)$ is covered by the affine opens $f^{-1}(W_{ij}) \cap U_i = \Spec{(A_i)_{f_{ij}}}$ for $f_{ij} \in B_i$. Furthermore, consider the maps $B \to (A_i)_{f_{ij}}$. Since $B_i \to A_i$ is finite type then $(B_i)_{f_{ij}} \to (A_i)_{f_{ij}}$ are finite type. Furthermore, $W_{ij} = \Spec{B_{g_{ij}}}$ because they are also principal affines of $\Spec{B}$ so $f^{-1}(W_{ij}) \cap U_i \to V$ is given by $B \to (B_{g_{ij}}) = (B_i)_{f_{ij}} \to (A_i)_{f_{ij}}$ which is finite type because $B \to B_{g_{ij}}$ is finite type. 

\subsubsection{3.2}

Suppose that $f : X \to Y$ is quasi-compact according to Hartshorne i.e. there exists an open affine cover $V_i = \Spec{B_i}$ such that $f^{-1}(V_i)$ is quasi-compact. Let $V \subset Y$ be affine then $V \cap V_i$ is covered by affine opens $D(f_{ij}) \subset V \cap V_i$ and by quasi-compactness there are finitely many $D(f_{ij})$ covering $V$. For each $V_i$ by quasi-compactness we can write $f^{-1}(V_i)$ as a finite union of affine opens $U_{ik} = \Spec{A_{ik}}$. Now, $f^{-1}(D(f_{ij}) ) \cap U_{ik} \subset U_{ik}$ is the affine open $(f|_{U_{ik}})^{-1}(D(f_{ij})) = D(\varphi(f_{ij}))$ for the map $\varphi : B_i \to A_{ik}$. Therefore, $f^{-1}(V)$ is the union of the finitely many affine opens $f^{-1}(D(f_{ij})) \cap U_{ik}$. Thus, $f^{-1}(V)$ is quasi-compact as the finite union of quasi-compacts.

\begin{definition}
A continuous map $f : X \to Y$ is quasi-compact if for any quasi-compact open $K \subset Y$ we have $f^{-1}(K)$ is quasi-compact. A morphism of schemes $f : X \to Y$ is quasi-compact if the underlying map of topologcial spaces is quasi-compact
\end{definition}  

\begin{theorem}
A morphism $f : X \to Y$ is quasi-compact iff the equivalent conditions above.
\end{theorem}

\begin{proof}
If $f : X \to Y$ is quasi-compact then for any affine open $V \subset Y$ we have $V$ is quasi-compact (since any open cover can be refined to $D(f_i)$ and $\bigcup D(f_i) = D(\sum (f_i) ) = D(1)$ so $\sum f_i$ generates the unit ideal so there must be some finite sum $f_{i_1} + \cdots + f_{i_n} = 1$ so $D(f_{i_1}) \cup \cdots \cup D(f_{i_n})$ is a finite subcover). Thus $f^{-1}(V)$ is quasi-comact. Coversely, suppose that any affine open $V \subset Y$ satisfies $f^{-1}(V)$ is quasi-compact. Then take any quasi-compact open $K \subset Y$ and consider $f^{-1}(K)$. By quasi-compactness we can write $K$ as a finite union of affine opens and then each has quasi-compact preimage. Thus $f^{-1}(K)$ is a finite union of quasi-compacts and thus is quasi-compact.
\end{proof}


\subsubsection{3.3}

\begin{definition}
Let $f : X \to Y$ be a morphism of schemes.
\begin{enumerate}
\item $f$ is finite type at $x \in X$ if there exist affine opens $\Spec{A} = U \subset X$ and $\Spec{B} = V \subset Y$ with $f(U) \subset V$ and $x \in U$ such that $B \to A$ is finite type
\item $f$ is locally finite type if it is finite type at each $x \in X$
\item $f$ is finite type if it is locally finite type and quasi-compact.
\end{enumerate}
\end{definition}

\begin{lemma}
A morphism $f : X \to Y$ is quasi-compact iff there exists an affine open cover $V_i \subset Y$  such that $f^{-1}(V_i)$ is the finite union of affine opens. 
\end{lemma}

\begin{proof}
If $f : X \to Y$ is quasi-compact then $f^{-1}(V)$ is quasi-compact for any affine open $V \subset Y$. Since affine opens form a base of the topology on $X$ the open $f^{-1}(V)$ is a union of affine opens which can by made finite by quasi-compactness. 
\bigskip\\
Conversely, if each $f^{-1}(V_i)$ is a finite union of affine opens then $f^{-1}(V_i)$ is quasi-compact so $f$ is quasi-compact by the above problem.
\end{proof}

\begin{enumerate}
\item If $f : X \to Y$ is finite type (see above) then it is quasi-compact by definition and for some affine open cover $V_i \subset Y$ we know $f^{-1}(V_i)$ can be covered by affine open $U_{ij}$ such that $U_{ij} \to V_i$ is finite type on rings. By quasi-compactness we can take the covering $U_{ij}$ of $f^{-1}(V_i)$ to be finite. Conversely, suppose $f : X \to Y$ is finite type according to Hartshorne then it is trivially locally finite type and $f^{-1}(V_i)$ is covered by finitely many affine opens and thus, by the lemma, is quasi-compact. 

\item If $f : X \to Y$ is finite type according to Hartshorne then we know it is locally finite type and by problem 3.2 we know for any affine open $V \subset Y$ we have $f^{-1}(V)$ is covered by affine opens $U_{i}$ such that $U_{i} \to V$ is finite type on rings. Furthermore, we have shown that $f$ is quasi-compact so $f^{-1}(V)$ is quasi-compact so we may take a finite subcover $U_i$ so $f^{-1}(V)$ is a finite union of affine opens with the finite type property. 

\item Let $f : X \to Y$ be locally of finite type according to Hartshorne and let $\Spec{B} = V \subset Y$ and $\Spec{A} = U \subset f^{-1}(V)$ be affine opens. Then consider $\varphi : B \to A$. We know that $f^{-1}(V)$ has a cover of affine opens $U_i = \Spec{A_i}$ with $B \to A_i$ finite type. Then consider $U \cap U_i$ which is open in $U_i$ and thus covered by simulteneous principal opens
\[ W_{ij} = \Spec{(A_i)_{f_{ij}}} = \Spec{A_{g_{ij}}} \]
Now, $B \to A_i \to (A_i)_{f_{ij}} = A_i[f_{ij}^{-1}]$ is finite type. Now the maps $A \to A_{g_{ij}} = (A_i)_{f_{ij}}$ are localization maps so they are finite type. Then consider,
\begin{center}
\begin{tikzcd}
B \arrow[r, "\varphi"] \arrow[rd, "\varphi"'] & A \arrow[d]
\\
& A_{g_{ij}}
\end{tikzcd}
\end{center} 
Since $W_{ij}$ cover $U$ and $B \to A_{g_{ij}}$ are finite type, I claim that $B \to A$ is finite type. 

\begin{lemma}
Let $\varphi : B \to A$ be a ring map such that there exist $f_1, \dots, f_n \in A$ generating the unit ideal such that $B \to A_{f_i}$ is finite type then $\varphi$ is finite type.
\end{lemma}

\begin{proof}
Choose $x_{ij} \in A_{f_i}$ generating as a $B$-algebra. Let,
\[ \sum_{i} \alpha_i f_i = 1 \]
for $\alpha_i \in A$. Then I claim that $B[x_{ij}, f_i, \alpha_i] \subset A$ is the entire ring. Indeed, for any $a \in A$ we know that for each $i$ there is some $N_i$ such that $f^{N_i} (a - p(x_{i1}, \dots, x_{ir})) = 0$ in $A$ for some polynomial $p \in B[X_1, \dots, X_r]$ and therefore taking $N = \max_i N_i$,
\[ f_i^{N} a \in B[x_{ij}] \]
and therefore,
\[ a = (\alpha_1 f_1 + \cdots + \alpha_n f_n)^{nN} a \in B[x_{ij}, f_i, \alpha_i] \]
and thus $A$ is finitely generated as a $B$-algebra.
\end{proof}
\end{enumerate}

\subsubsection{3.4}

Let $f : X \to Y$ be finite and $\Spec{B} = V \subset Y$ be an affine open. Take an affine open cover $\Spec{B_i} = V_i \subset Y$ such that $f^{-1}(V_i) = U_i = \Spec{A_i}$ is an affine open cover of $X$ and $B_i \to A_i$ is finite. Consider $f^{-1}(V) \cap U_i = f^{-1}(V \cap V_i)$. Since $V_i$ is affine open we have $V \cap V_i$ covered by prinipal opens $W_{ij}$ of both $V_i$ and $V$ and then $f^{-1}(W_{ij}) \cap U_i$ is a principal affine open of $U_i$ so by localization from $U_i \to V_i$, the map $f^{-1}(W_{ij}) \cap U_i \to W_{ij}$ is given by a finite ring map. Thus replacing $X$ by $f^{-1}(V)$ we reduce to the case $a : X \to \Spec{B}$ where $D(f_i) \subset \Spec{B}$ is an open affine cover (which we may take to be finite) and $a^{-1}(D(f_i)) = X_{a^\#(f_i)} = \Spec{A_i}$ is affine open with $B_{f_i} \to A_i$ is finite. Then $f_1, \dots, f_n$ generate the unit ideal of $B$ since they cover $\Spec{B}$. Thus, $a^\#(f_1), \dots, a^\#(f_n) \in \Gamma(X, \struct{X})$ generate the unit ideal. Therefore, by 2.17, $X = \Spec{A}$ is affine with $A_i = A_{a^\#(f_i)}$. Now, the map $a^\# : B \to A$ localizes to $B_{f_i} \to A_{a^\#(f_i)}$ which is finite. Then by lemma \ref{finiteness_local} $B \to A$ is finite. 

\begin{lemma} \label{finiteness_local}
Let $\varphi : A \to B$ be a ring map such that for $f_1, \dots, f_n$ generating the unit ideal of $A$ such that the localized maps $\varphi : A_{f_i} \to B_{\varphi(f_i)}$ are finite, then $\varphi$ is finite.
\end{lemma}

\begin{proof}
Let $x_{i1}, \dots, x_{in} \in B_{\varphi(f_i)}$ generate $B_{\varphi(f_i)}$ as an $A_f$-module. Multiplying by a suitable power of $f_i$ we may assume these elements lift to $B$. I claim that $\{ x_{ij} \}$ generate $B$ as an $A$-module. For any $b \in B$ we know that $f_i^{N_i} (a_1 \cdot x_{i1} + \cdots + a_n \cdot x_{in} - b) = 0$ in $B$ for some $n_i$.  Now, $f_1^N, \dots, f_n^N$ generate the unit ideal of $A$ where $N = \max_i N_i$ so for each $i$ we get,
\[ f_i^N b \in A x_{i1} + \cdots + A x_{in} \]
and thus,
\[ b \in  \sum A x_{ij} \]
since $f_1^N, \dots, f_n^N$ generate the unit ideal of $A$. Thus $A \to B$ makes $B$ a finite $A$-module.
\end{proof}

\subsubsection{3.5}

\begin{definition}
We say that $f : X \to Y$ is quasi-finite if for each $y \in Y$ the set $f^{-1}(y)$ is finite.
\end{definition}

\begin{enumerate}
\item Let $f : X \to Y$ be finite. For any $y \in Y$ there must exist affine open sets $\Spec{A} = U \subset X$ and $y \in \Spec{B} = V \subset Y$ such that $U = f^{-1}(V)$ and $B \to A$ is finite. Then $f^{-1}(y) \subset U$ so it suffices to show that the set of primes above $\p \in \Spec{B}$ is finite. The fibre is $X_y = \Spec{A \otimes_B \kappa(y)} \to \Spec{\kappa(y)}$ which is finite then we use the fact that a finite-dimensional $k$-algebra has finitely many prime ideals. 
\item Let $f : X \to Y$ be finite and $Z \subset X$ be closed. Because $f$ is finite we can find an affine open cover $V_i = \Spec{B_i}$ of $Y$ such that $U_i = f^{-1}(V)$ is affine, $U_i = \Spec{A_i}$ and $\varphi_i : B_i \to A_i$ is finite. Then $Z \cap U_i$ is closed in $\Spec{A_i}$ so there is an ideal $I_i \subset A_i$ such that $Z \cap U_i = V(I_i)$. Now, I claim that finite ring maps induce closed maps on spectra. 
\bigskip\\
Consider $V(I) \subset \Spec{A}$ and $\varphi : B \to A$. Then consider $\varphi^*(V(I))$
We can reduce to the case $I = 0$ since $B \to A \to A / I$ is finite and $\varphi^*(V(I))$ is the image of $\Spec{A / I} \to \Spec{B}$. We can also reduce to $B \to A$ injective since the image of $\Spec{A}$ is contained in $\Spec{B / \ker{(B \to A)}}$ which is closed in $\Spec{B}$. Thus, take $B \to A$ injective and consider $\Spec{A} \to \Spec{B}$. Since $B \to A$ is finite it is integral so the going up property holds. Thus it suffices to show that minimal primes of $B$ are in the image. If $\p \in \Spec{B}$ is minimal then $B_\p$ has a unique prime ideal then the localization $B_\p \to A_\p$ is injective so $\Spec{A_\p}$ is nonempty and hits the unique prime $\p \in \Spec{B_\p}$ so $\p$ is in the image of $\Spec{A} \to \Spec{B}$. Since $\Spec{A} \to \Spec{B}$ hits all minimal primes and has going up then it must be surjective. 
\bigskip\\
Therefore $f(Z \cap U_i)$ is closed in $V_i = \Spec{A_i}$. Now, $y \in f(Z) \cap V_i$ if $y \in V_i$ and $\exists x \in Z$ such that $f(x) = y$ so $x \in Z \cap f^{-1}(V_i) = Z \cap U_i$. Furthermore, $f(Z \cap U_i) \subset f(Z) \cap f(U_i) \subset f(Z) \cap V_i$ so $f(Z) \cap V_i = f(Z \cap U_i)$ so $f(Z) \cap V_i$ is closed. Then, I claim that $f(Z)$ is closed. 
\bigskip\\
If $x \in f(Z)^C$ then for some $V_i$ we have $x \in V_i \setminus f(Z)$ is open and $x \in V_i \setminus f(Z) \subset f(Z)^C$ so $f(Z)$ is closed. 

\item Consider the map $\Gm^k \coprod \A^1_k \to \A^1_k$ via $k[x] \to k[x,x^{-1}]$ and the identity. This is clearly surjective and finitely generated since on rings it is,
\[ k[x] \to k[x, x^{-1}] \times k[x] \]
Furthermore, this map is quasi-finite since the fibers have at most two points. To see this, consider, $y = (x - a) \in \Spec{k[x]}$ then $\kappa(y) = k[x]/(x - a)$ and the fibre is,
\begin{align*}
X_y & = \Spec{(k[x, x^{-1}] \times k[x]) \otimes_{k[x]} k[x]/(x  - a)} 
\\
& = \Spec{k[x, x^{-1}]/(x - a) \times k[x] / (x - a)} 
\\
& = \Spec{k[x, x^{-1}/(x - a)} \coprod \Spec{k[x]/(x - a)} 
\\
& = 
\begin{cases}
\Spec{k} & a = 0
\\
\Spec{k} \coprod \Spec{k} & a \neq 0
\end{cases}
\end{align*}
However, this map is not closed since $\Gm^k \subset \Gm^k \coprod \A^1_k$ is closed but its image is $\A^1_k \setminus \{ 0 \}$ which is not closed. Thus the map cannot be finite. In particular,
\[ k[x, x^{-1}] = \bigoplus_{n \ge 0} x^{-n} k[x] \]
so $k[x, x^{-1}]$ is not a finitely-generated $k[x]$-module.  
\end{enumerate}

\subsubsection{3.6}

Let $X$ be an integral scheme. Then $X$ is irreducible so it has a unique generic point $\xi \in X$. Since $\xi$ is generic, all points are its limit points i.e. it lies in every nonempty open $U \subset X$. In particular, if $U = \Spec{A}$ is an affine open then $\xi \in U$ corresponding to $\p_\xi \subset A$ such that $V(\p_\xi) = \Spec{A}$. Since $X$ is integral, $A$ is a domain then $\stalk{X}{\xi} = A_{\p_\xi} = \Frac{A}$ is a field.

\subsubsection{3.7}

\begin{definition}
A morphism $f : X \to Y$ with $Y$ irreduclbe is \textit{generically finite} if $X_\eta$ is finite at the generic point $\eta \in Y$. 
\end{definition}

\begin{definition}
A morphism $f : X \to Y$ is \textit{dominant} if $f(X) \subset Y$ is dense.
\end{definition}

\begin{lemma}
Let $f : X \to Y$ be a morphism of irreducible schemes and let $\eta_X \in X$ and $\eta_Y \in Y$ be their generic points. Then $f$ is dominant iff $f(\eta_X) = \eta_Y$.
\end{lemma}

\begin{proof}
If $f(\eta_X) = \eta_Y$ then $\overline{f(\eta_X)} = \overline{\eta_Y} = Y$. Conversely, if $\overline{f(X)} = Y$ then since $f$ is continuous $f(\overline{A}) \subset \overline{f(A)}$ for any set $A$. Thus,
\[ f(X) = f(\overline{\eta_X}) \subset \overline{f(\eta_X)} \]
Thus, $\overline{f(\eta_X)} = \overline{f(X)} = Y$ so $f(\eta_X) = \eta_Y$ since it is a point whose closure is $Y$.
\end{proof}

Let $f : X \to Y$ be a dominant, generically finite, finite-type morphism of integral schemes. Let $\eta_X \in X$ and $\eta_Y \in Y$ be their generic points. Then $f(\eta_X) = \eta_Y$ so we get a map $f^\# : \stalk{Y}{\eta_Y} \to \stalk{X}{\eta_X}$ whch is an extension of residue function, $K(Y) \embed K(X)$. 
\bigskip\\
First, take affine opens $\Spec{B} = U \subset X$ and $\Spec{A} = V \subset Y$ with $f : U \to V$ then $A$ and $B$ are domains and $\varphi : A \to B$ is finite type so there is a surjecton $A[x_1, \dots, x_n] \twoheadrightarrow B$. Since $f(\eta_X) = \eta_Y$ (because $f : U \to V$ is dominant), then $\ker{\varphi} = \varphi^{-1}(0) = (0)$ so $\varphi$ is injective so we get an extension of domains $A \subset B$. Furthermore, $K = \Frac{A} = K(Y)$ and $F = \Frac{B} = K(X)$. The morphism $f : U \to V$ must be generically finite which implies that the fibre,
\[ U_{\eta_Y} = \Spec{B \otimes_A K} = \Spec{B \otimes_A S^{-1}_A A} = \Spec{S_A^{-1} B} \]
is finite. However, $B_K = B \otimes_A K(Y)$ is a finitely generated $K$-algebra because the base change of the map $A[x_1, \dots, x_n] \twoheadrightarrow B$ gives $K[x_1, \dots, x_n] \twoheadrightarrow B_K$. Now we apply Noetherian normalization to the domain $B_K = S_A^{-1} B$ to get a finite (and hence integral) extension of domains $B_K \supset K[x_1, \dots, x_d]$ with $d = \dim{B_K}$. By Cohen, $\Spec{B_K} \to \Spec{K[x_1, \dots, x_d]}$ is surjective but for $d > 0$ the space $\Spec{K[x_1, \dots, x_d]}$ is infinite so $\dim{B_K} = 0$ and thus $B_K$ is a domain finite over $K$. Therefore, $B_K$ is a field but $B_K = S_A^{-1} B \subset F$ so $B_K = \Frac{B} = F$ meaining that $F / K$ is a finite extension of fields since $B_K / K$ is finite. 
\bigskip\\
Recall $x_1, \dots, x_n \in B$ generate as an $A$-algebra. Since $F / K$ is finite, each $x_1, \dots, x_n \in B \subset K$ must satisfy a monic $K$-equation. Let $g \in K$ be the product of the denominators of the coefficients then $A_g \subset B_g$ is a finite extension since $B_g$ is generated as an $A_g$-algebra by finitely many integral elements. Then $f : U_g \to V_g$ is finite with $U_g = \Spec{B_g}$ and $V_g = \Spec{A_g}$ and $V_g$ is dense since $Y$ is irreducible. Since $f^{-1}(V_g) \subset X$ is dense ($X$ is irreducible) replacing $X$ by $f^{-1}(V_g)$ and $Y$ by $V_g$ reduces to the case of $f : X \to \Spec{A}$ with an affine covering by $U_i = \Spec{B_i}$ such that $f : U_i \to \Spec{A}$ is finite and $A \embed B_i$ is a finite extension of domains. Then take,
\[ W = \bigcap U_i \]
which is nonempty since $X$ is irreducible. Now $U_i \setminus W$ is closed in $U_i$ so there is some ideal $\a_i \subset B_i$ strictly containing the nilradical (i.e. nonzero since these are domains) such that $U_i \setminus W = V(\a_i)$ (since $U_i \setminus W \subsetneq U_i$). If $\a_i \cap A = (0)$ then by Cohen $\a_i = (0)$ since there cannot be inclusions in the fibres of an integral extension. Therefore, $\a_i \cap A \supsetneq (0)$ so take some nonzero $f_i \in \a_i \cap A$ then in $U_i = \Spec{B_i}$ we have $D(f_i) \subset V(\a_i)^C = W$. Take $V = \Spec{A_{f_i}}$ which is open in $\Spec{A}$ and $f^{-1}(V) \cap U_i = D(f_i) \subset W \cap U_i$ meaning that $f^{-1}(V) \subset W$ since $U_i$ form a cover of $X$ so $f^{-1}(V) \subset U_i$ and so $f^{-1}(V) = \Spec{(B_i)_{f_i}}$ is affine. Finally, since $A \subset B_i$ is finite we know $A_{f_i} \to (B_i)_{f_i}$ is finite and thus $f : f^{-1}(V) \to V$ is finite and since $Y$ is irreducible and $V$ is nonempty open it is dense.

\subsubsection{3.8}

\begin{lemma}
Let $X$ be an integral scheme. Then the following are equivalent:
\begin{enumerate}
\item for each affine open $\Spec{A} \subset X$ the ring $A$ is an integrally closed domain.
\item for each $x \in X$ the local ring $\stalk{X}{x}$ is normal
\item for each closed $x \in X$ the local ring $\stalk{X}{x}$ is normal
\end{enumerate}
\end{lemma}

\begin{proof}
Because localization preserves integral closure, if $x \in \Spec{A}$ and $A$ is integrally closed then $\stalk{X}{x}$ is normal showing that (a) $\implies$ (b). It is clear that (a) $\implies$ (b). Finally, for each $\Spec{A} \subset X$ we have, 
\[ A = \bigcap_{\m \in \mSpec{A}} A_\m \]
inside the fraction field and each $A_\m = \stalk{X}{\m}$ is integrally closed by assumption so $A$ is integrally closed.
\end{proof}

Let $X$ be an integral scheme with generic point $\xi \in X$ and function field $K = \stalk{X}{\xi}$. For each affine open $\Spec{A} = U \subset X$ we define $\wt{U} = \Spec{\wt{A}}$ where $\wt{A} \subset K$ is the integral closure of $A$ in $K$. I claim that these glue properly. For each $x \in U \cap V$ we can find an open $x \in W \subset U \cap V$ which is standard open in both $U$ and $V$. Then I claim there are compatible open immersions $\wt{W} \embed \wt{U}$ and $\wt{W} \embed \wt{V}$ which provide the gluing data. Indeed, 
\[ \wt{(A_f)} = (\wt{A})_f \]
To see this, notice that $(\wt{A})_f$ is integrally closed and thus $\wt{(A_f)} \subset (\wt{A})_f$. Furthermore, $\wt{A} \subset \wt{(A_f)}$ and $f$ is a unit in $\wt{(A_f)}$ so $(\wt{A})_f \subset \wt{(A_f)}$. Thus $\wt{W} = \Spec{\wt{(A_f)}} = \Spec{(\wt{A})_f}$ is naturally a principal affine open of $\wt{U}$. Therefore, if $V = \Spec{B}$ then $W = \Spec{A_f} = \Spec{B_g}$ and therefore $\wt{W}$ is naturally isomorphic to a principal open of both $\wt{U}$ and $\wt{V}$. This provides gluing data to produce an integral scheme $\wt{X}$ (integral because each $\wt{A}$ is a domain). 
\bigskip\\
Furthermore, the maps $A \subset \wt{A}$ giving $\wt{U} \to U$ are compatible with these localizations and therefore glue to a morphism $\pi : \wt{X} \to X$. Now consider any dominant map $f : Z \to X$ from a normal integral scheme. Let $\eta \in Z$ be the generic point and $L = \stalk{Z}{\eta}$ its function field. Since $f$ is dominant, $\eta \mapsto \xi$ giving an injection $K \embed L$. Then for each pair of affine opens $\Spec{B} = V \subset Z$ and $\Spec{A} = U \subset X$ such that $f(V) \subset U$ we see that $A \to B \to L$ must be injective because $A \to K \to L$ is injective and $B \to L$ is injective. Since $B$ is integrally closed in $L$ we see that $B \cap K$ is integrally closed in $K$ and therefore $A \embed B$ factors as $A \subset \wt{A} \embed B$ giving a map $U \to \wt{V} \to V$. Because these maps are inclusions inside $L$, they are uniquely determined and therefore are compatible with principal opens and thus compatible on the overlaps. Thus these morphisms glue to give a unique morphism $\tilde{f} : Z \to \wt{X}$ such that $\pi \circ \tilde{f} = f$.
\bigskip\\
By definition, for any affine open $\Spec{A} = U \subset X$ we have $\pi^{-1}(U) = \wt{U} = \mathrm{Spec}(\wt{A}) \subset \wt{X}$ so $\pi$ is affine. Therefore, to show that $\pi$ is finite, it suffices to show that $A \to \wt{A}$ is a finite ring map. Now let $X$ be finite type over a field, then each $A$ is a finitely generated $k$-algebra domain. We can say that a field $k$ is Nagata and therefore universally Japanese so $A$ is N-2 (Japanese) and thus N-1 meaning that its integral closure $A \embed \wt{A}$ is finite by definition. 
\bigskip\\
We can alternatively argue without as much heavy machinery as follows. By noether normalization, $A$ is finite over $k[x_1, \dots, x_n]$. Furthermore, $k$ is obviously N-2 so by \chref{https://stacks.math.columbia.edu/tag/032O}{Tag 032O} we see that $k[x_1, \dots, x_n]$ is N-2 and thus by \chref{https://stacks.math.columbia.edu/tag/032I}{Tag 032I} the domain $A$ is N-2 and in particular N-1 so $A \embed \wt{A}$ is finite by definition.

\subsubsection{3.9}

\begin{enumerate}
\item Let $\A^1_k = \Spec{k[x]}$ where $k$ is algebraically closed. The points of $\A^1_k$ are ideals $(x - \mu)$ for $\mu \in k$ and $(0)$. However, the points of $\A^2_k = \A^1_k \times_k \A^1_k = \Spec{k[x,y]}$ are ideals $(x - \mu, y - \lambda)$ with $\mu, \lambda \in k$ plus $(f(x,y))$ for any irreducible $f(x,y) \in k[x,y]$ plus $(0)$. Therefore, $\A^2_k$ has all points in the product plus a bunch of generic points of closed subschemes.

\item Consider $k(s)$ and $k(t)$ with two independent inteterminants. These are fields so $\Spec{k(t)}$ and $\Spec{k(t)}$ are point point spaces. However, consider,
\[ X = \Spec{k(s)} \times_{\Spec{k}} \Spec{k(t)} = \Spec{k(s) \otimes_k k(t)} \]
This has at least as many closed points as $k^\times$ because the map $k(s) \otimes_k k(t) \to k(x)$ sending $s \otimes 1 \mapsto x$ and $1 \otimes t \mapsto r x$ for $r \in k^\times$ is surjective making its kernel $(rs \otimes 1 - 1 \otimes t)$  a maximal ideal. 
\end{enumerate}

\subsubsection{3.10 DO THIS!!}

\begin{enumerate}
\item Let $f : X \to Y$ be a morphism and $y \in Y$. Then consider the fibre $X_y$ defined as the pushout,
\begin{center}
\begin{tikzcd}[row sep = large]
X_y \arrow[r] \arrow[d] & X \arrow[d, "f"] 
\\
\Spec{\kappa(y)} \arrow[r] & Y 
\end{tikzcd}
\end{center}
First, note that $X_y \to X$ as a map of topological spaces has image inside the fibre $f^{-1}(y)$ since the diagram commues and the image of $\Spec{\kappa(y)} \to Y$ is the single point $y \in Y$. Thus it suffices to show that $X_y \to f^{-1}(y)$ is a homeomorphism.
\bigskip\\
For any point $x \in f^{-1}(y)$ there is a morphism $\Spec{\kappa(x)} \to X$ and $f$ gives a map $\kappa(y) \to \kappa(x)$ and thus a morphism $\Spec{\kappa(x)} \to \Spec{\kappa(y)}$ such that the diagram commutes,
\begin{center}
\begin{tikzcd}[row sep = large]
\Spec{\kappa(x)} \arrow[rrd, bend left] \arrow[rd, dashed] \arrow[ddr, bend right]
\\
& X_y \arrow[r] \arrow[d] & X \arrow[d, "f"] 
\\
& \Spec{\kappa(y)} \arrow[r] & Y
\end{tikzcd}
\end{center} 
Thus we get a point $\Spec{\kappa(x)} \to X_y$. Therefore, the map $X_y \to f^{-1}(y)$ is bijective. Therefore, it suffices to prove that $\iota : X_y \to f^{-1}(y)$ is closed. (DO THIS)
\item Let $k$ be an algebraically closed field of characteristic zero. Consider the scheme,
\[ X = \Spec{k[s,t]/(s - t^2)} \]
and $Y = \Spec{k[s]}$ and consider the morphism $f : X \to Y$ via $k[s] \to k[s,t]/(s - t^2)$. For the prime $y = (s - a) \in Y$ consider the residue field,
\[ \kappa(y) = k[s]_{(s - a)} / (s - a) = k[s]/(s - a) \]
then the fibre is,
\[ X_y = \Spec{k[s,t]/(s - t^2) \otimes_{k[s]} k[s]/(s - a)} \]
furthermore,
\[ k[s,t]/(s - t^2) \otimes_{k[s]} k[s]/(s - a) = k[t]/(a - t^2) \]
which implies that,
\[ X_y \cong V((a - t^2)) \subset \Spec{k[t]} \]
Thus, if $a \neq 0$ then $t^2 - a$ splits (since $k$ is algebraically closed and $\ch{k} = 0$) so $\Spec{k[t]/(a - t^2)}$ has two points and is reduced. For $a = 0$ we have $X_y = \Spec{k[t]/(t^2)}$ which is one point and not reduced since $\nilrad{k[t]/(t^2)} = (t)$.  
\bigskip\\
Finally, consider the fibre above the generic point $\eta = (0) \subset k[s]$ which has resuide field $\kappa(\eta) = k[s]_{(0)} = k(s)$. Therefore the fibre is,
\[ X_\eta = \Spec{k[s, t]/(s - t^2) \otimes_{k[s]} k(s)} = \Spec{k(s)[t]/(s - t^2)} \] 
The polynomial $t^2 - s \in k(s)[t]$ is irreducible then $(s - t^2)$ is maximal. Then $k(s)[t]/(s - t^2)$ is a field extension of $k(s)$ of degree 2 and thus it has one prime so $X_\eta$ is a one point space.
\end{enumerate}

\subsubsection{3.11 (DDDD CHECK THIS!!)}

\begin{enumerate}
\item Let $f : Z \to Y$ be a closed immersion and $X \to Y$ a morphism then consider $f' : Z \times_Y X \to X$. Being a closed immersion is a local property since surjectivity of sheaves is local on the source and target and being a homeomorphism onto a closed set is local since the image of closed sets being closed is local on the source and target. Thus it suffices to prove the case of affine schemes $X = \Spec{A}$ and $Y = \Spec{B}$ and $Z = \Spec{C}$. Then we get,
\begin{center}
\begin{tikzcd}
C \otimes_B A \arrow[from=r, two heads] & A
\\
C \arrow[u] & B \arrow[u] \arrow[l, two heads]
\end{tikzcd}
\end{center}
If $B \to C$ is surjective then by right-exactness $A \to C \otimes_B A$ is surjective. Furthermore, by surjectivity of $B \to C$ we get $C = B / I$ and thus this is the closed immersion 
$\Spec{B/I} \to \Spec{C}$. Then, $A \to (B / I) \otimes_B A = B / I B$ gives the closed immersion $\Spec{B / I B} \to \Spec{B}$.

\item Let $X = \Spec{A}$ be affine and $\iota : Y \embed X$ a closed subscheme. Let $\Spec{B_i} = U_i \subset Y$ be an affine open cover of $Y$ and consider $\iota|_{U_i} : U_i \to \iota(U_i)$. Since $\iota : Y \to \iota(Y)$ is a homeomorphism with $\iota(Y) \subset X$ closed so $\iota(U_i)$ is open in $\iota(Y)$ and thus of the form $U \cap \iota(Y)$ for some open $U \subset X$. Therefore, we can cover $\iota(U_i)$ by principal affines $D(f_{ij}) \cap \iota(Y)$,
\[ \iota(U_i) = \bigcup_{j = 1}^{n_i} D(f_{ij}) \cap \iota(Y) \]
Now, $\iota|_{U_i}^{-1}(D(f_{ij})) = U_i \cap \iota^{-1}(D(f_{ij})) = D(\varphi_i(f_{ij}))$ for $\varphi : A \to B_i$ inside $U_i = \Spec{B_i}$. However, $D(f_{ij}) \cap \iota(Y) \subset \iota(U_i)$ and $\iota$ is injective so,
\[ \iota^{-1}(D(f_{ij})) \subset U_i \]
and thus,
\[ \iota^{-1}(D(f_{ij})) = U_i \cap \iota^{-1}(D(f_{ij})) = \iota^{-1}(D(f_{ij})) = D(\varphi_i(f_{ij})) \]
Therefore, $Y$ is covered by affine opens $D(\varphi_i(f_{ij}))$.
Since $\iota(Y)$ is closed we can cover $X \setminus \iota(Y)$ by principal opens $D(f_k)$ to give a cover of $X$ by opens $D(f_i)$ such that $\iota^{-1}(D(f_i))$ is empty or affine and because $X = \Spec{A}$ is quasi-compact we can take this cover to be finite. Since the $D(f_i)$ cover $X$ we have $f_1, \dots, f_n \in A$ generate the unit ideal so $\iota^\#(f_1), \dots, \iota^\#(f_n) \in \Gamma(Y, \struct{Y})$ generate the unit ideal and $\iota^{-1}(D(f_i)) = Y_{\iota^\#(f_i)} = D(\varphi(f_i))$. Therefore, by [II, Ex. 2.17] we have that $Y = \Spec{B}$ is affine. Furthermore, by [II, Ex. 2.18(d)], since $Y \to X$ is a closed immersion we have $A \to B$ is surjective so $B \cong A / \a$ and our closed subscheme is equivalent to $\Spec{A / \a} \to \Spec{A}$.  

\item Let $Y \subset X$ be a closed subset and give $Y$ the reduced induced subscheme structure. Let $Y' \embed X$ be any other closed subscheme of $X$ whose underlying space is $Y$. This question is local so it suffices to show the case that $X = \Spec{A}$ is affine and thus $Y = \Spec{A / I}$ where,
\[ I = \bigcap_{\p \in V(I)} \p \]
i.e. $I = \sqrt{I}$ is radical. Then $Y' = \Spec{A / J}$ for any ideal such that $V(J) = V(I)$ i.e. $\sqrt{J} = \sqrt{I} = I$. Therefore, $J \subset I$ so  the map $\Spec{A / I} \to \Spec{A}$ factors through $\Spec{A/J} \to \Spec{A}$ since the ring map $A \to A / I$ factors through $A \to A / J$ because $J \subset I$.  

\item Let $f : Z \to X$ be a morphism. Consider the scheme theoretic image $Y$ of $f$ which is a closed subscheme of $Y$ such that $f$ factors $f : Z \to Y \to X$ and if $Y'$ is annother closed subscheme of $X$ such that $f$ factors as $f : Z \to Y' \to X$ then $Y \to X$ factors through $Y' \to X$,
\begin{center}
\begin{tikzcd}
Z \arrow[rr, "f"] \arrow[rd, "\tilde{f}"] \arrow[rdd, "\tilde{f}", bend right] &  & X
\\
& Y \arrow[ru, "\iota"] \arrow[d, dashed]
\\
& Y' \arrow[ruu, "\iota'", bend right]
\end{tikzcd}
\end{center}
Uniqueness is clear since if $Y$ and $Y'$ both satisfied this condition then we have morphisms $Y \to Y'$ and $Y' \to Y$ which compose to give an automorphism of $Y \to X$ which must be the identity since $Y \to X$ is a closed immersion.
\bigskip\\
We need to show that such a scheme exists. (SHOW THIS)
\end{enumerate}

\begin{lemma}
Given a morphism of schemes $f : X \to \Spec{A}$ and $g \in A$ we have, 
\[ f^{-1}(D(g)) = X_{f^\#(g)} = \{ x \in X \mid (f^\#(g))_x \notin \m_x \} \]
\end{lemma}

\begin{proof}
Recall that $f(x) = \p$ iff $\p = (f^\#)^{-1} \circ \res_x^{-1}(\m_x)$ because for $f(x) = \p$ the sheaf diagram,
\begin{center}
\begin{tikzcd}
A \arrow[d] \arrow[r, "f^\#"] & \struct{X}(X) \arrow[d, "\res_x"]
\\
A_\p \arrow[r, "f^\#_x"] & \stalk{X}{x}
\end{tikzcd}
\end{center} 
And furthermore, $f^\#_x : A_\p \to \stalk{X}{x}$ is local so $(f^\#_x)^{-1}(\m_x) = \p A_\p$ and thus, by commutativity, $\p = (f^\#)^{-1} \circ \res_x^{-1}(\m_x)$. 
Thus,
\begin{align*}
x \in f^{-1}(D(g)) & \iff f(x) \in D(g) \iff (f^\#)^{-1} \circ \res_x^{-1} (\m_x) \in D(g) 
\\
& \iff g \notin (f^\#)^{-1} \circ \res_{x}^{-1} (\m_x) \iff (f^\#(g))_x \notin \m_x 
\\
& \iff x \in X_{f^\#(g)}
\end{align*}
\end{proof}

\subsubsection{3.12 DO THIS!!}

\begin{enumerate}
\item Let $\varphi : S \to B$ be a surjective graded ring map. (DO 2.14)
\end{enumerate}

\subsubsection{3.13 DO THIS!!}

\subsubsection{3.14}

Let $X$ be a scheme locally of finite type over a field $k$. A point $x \in X$ is closed iff $\kappa(x)$ is a finite extension of $k$. Take any nonempty open $U \subset X$ which must contain an affine open $\Spec{A}$. Since $X$ is of finite type over $k$ the ring $A$ is a finite $k$-algebra which we may write as $k[x_1, \dots, x_n]/I$. Take a maximal ideal $\m$ containing $I$ such that $\m \in \Spec{A}$ and,
\[ \kappa(\m) = A_{\m} / \m A_{\m} = (A \setminus \m)^{-1} (A / \m) = A / \m \]
since $A / \m$ is a field. Furthermore, $A / \m$ is a finitely-generated $k$-algebra so, by the nullstellensatz, $A / \m$ is a finite extension of $k$ so $\m \in \Spec{A} \subset U$ is a closed point of $X$ (not just of $\Spec{A}$ which is obvious). Therefore, closed points are dense. 
\bigskip\\
Conversely, take any local ring $R$ which is not a field. Then $R$ has a unique maximal ideal which is a unique closed point so the closure of the closed points is a single point. However $\Spec{R}$ has more than one point.  

\subsubsection{3.15 (DO THIS)}

Let $X$ be a scheme of finite type over $k$. This problem is based on the following observation: every field extension $K / k$ is the limit of finitely generated field extensions and every finitely generated field extension $F / k$ can be factored as $F / L / E / k$ where $F/L$ is separable and $L/E$ is purely inseparable and $E / k$ is purely transcendental. Then we apply the primitive element theorem to see that $F = L(\alpha)$ and $L/k$ is a sequence of extensions of the form $k(\alpha)$ where $\alpha^p \in k$.

\begin{enumerate}
\item We say that $X$ is \textit{geomerically irreducible} if one of the three conditions hold,
\begin{enumerate}
\item[(i)] $X \times_k \Spec{\overline{k}}$ is irreducible
\item[(ii)] $X \times_k \Spec{k^{\text{sep}}}$ is irreducible
\item[(iii)] $X \times_k \Spec{K}$ is irreducible for every extension $K / k$
\end{enumerate}
We need to prove that these are equivalent. 
The map $k^{\text{sep}} \embed \overline{k}$ induces a surjective morphism,
\[ \Spec{k^{\text{sep}}} \to \Spec{\overline{k}}  \]
and therefore the base change $X \times_k \Spec{\overline{k}} \to X \times_k \Spec{k^{\text{sep}}}$ is surjective. Since the map $\Spec{k^{\text{sep}}} \to \Spec{\overline{k}}$ is surjective its base Now, the image of irreducible sets is irreducible so (i) $\implies$ (ii). Furthermore, (iii) $\implies$ (i) is trivial. Therefore, we just need to show that (ii) $\implies$ (iii).
\bigskip\\
By the same argument as (i) $\implies$ (ii) we see that (ii) implies that $X \times_k \Spec{k'}$ for every finite separable extension $k'/k$. Irreducibility is a local condition, so from the local nature of the fiber product it suffices to consider the case $X = \Spec{A}$ for $A$ finite type over $k$. We have $(A \ot_k k^\sep)_{\red}$ is a domain and thus $(A \ot_k k')_{\red} \subset (A \ot_k k^\sep)_{\red}$ is a domain for every finite separable extension. Because,
\[ A \ot_k K = A \ot_k \varinjlim_{K/F/k} k' = \varinjlim_{K/F/k} (A \ot_k F) \]
since tensor product commute with colimits we just need to show that $(A \ot_k F)_{\red}$ is a domain since if there are non-nilpotent zero divisors they must appear at some finite level. Therefore (chainging the base field each time), it suffices to show that $A \ot_k F$ is a domain for $F/k$ finitely separable, $F = k(\alpha)$ with $\alpha^p \in k$, or $F = k(x)$. We know the finite separable case by assumption. For $F = k(\alpha)$ we have,
\[ A \ot_k k(\alpha) = A \ot_k k[x]/(x^p - \alpha) = A[x]/(x^p - \alpha) \]
(SHOW THIS IS GOOD!!)
Finally, if $F = k(x)$ then, $A \ot_k k(x) = A \ot_k \Frac{k[x]}$ is a localization of $A[x]$ but $A[x]$ has no non-nilpotent zero divisors so neither does its localization. Therefore we have shown that (ii) $\implies$ (iii).

\newcommand{\perf}{\mathrm{perf}}

\item We say that $X$ is \textit{geomerically reduced} if one of the three conditions hold,
\begin{enumerate}
\item[(i)] $X \times_k \Spec{\overline{k}}$ is reduced
\item[(ii)] $X \times_k \Spec{k^{\perf}}$ is reduced
\item[(iii)] $X \times_k \Spec{K}$ is reduced for every extension $K / k$
\end{enumerate}
Again, being reduced is a local condition, so from the local nature of the fiber product it suffices to consider the case $X = \Spec{A}$ for $A$ finite type over $k$. If $A \ot_k \bar{k}$ is reduced then $A \ot_k k^{\perf} \subset A \ot \bar{k}$ is also reduced so (i) $\implies$ (ii). Furthermore, (iii) $\implies$ (i) trivially. Therefore, we just need to show (ii) $\implies$ (iii). By the same argument as above, if $A \ot_k k^\perf$ is reduced then $A \ot_k k'$ is reduced for each purely inseparable extension $k'/k$. Then,
\[ A \ot_k K = A \ot_k \varinjlim_{K/F/k} k' = \varinjlim_{K/F/k} (A \ot_k F) \]
so it suffices to show that $A \ot_k F$ is reduced for each finitely generated field extension. The hypothesis gives the purely inseparable case. For $F/k$ finite separable we have $F = k(\alpha) = k[x]/(f(x))$ for $f$ the minimal polynomial of $\alpha$ over $k$ which is separable. Therefore,
\[ A \ot_k F = A[x]/(f(x)) \]
(SHOW THIS IS GOOD!!)
Finally, if $F = k(x)$ then, $A \ot_k k(x) = A \ot_k \Frac{k[x]}$ is a localization of $A[x]$ but $A[x]$ is reduced so its localization $A \ot_k k(x)$ is also reduced. Therefore we have shown that (ii) $\implies$ (iii).

\item We say that $X$ is \textit{geometrically integral} if $X \times_k \overline{k}$ is integral. 
\bigskip\\
Let $k'/k$ be an inseparable but not purely inseparable field extension for example let $k = \FF_p(t)$ and $k' = k(t^{\frac{1}{2p}})$ for $p$ odd. Let $X = \Spec{k'}$. Then clearly $X$ is integral because $k'$ is a field but I claim that $X \times_k \Spec{k'}$ is neither reduced nor irreducible. Indeed, let $A = k' \ot_k k'$ and let $\alpha = t^{\frac{1}{2p}}$. Then,
\[ (\alpha^2 \ot 1 - 1 \ot \alpha^2)^p = (\alpha^{2p} \ot 1 - 1 \ot \alpha^{2p}) = (t \ot 1 - 1 \ot t) = 0 \]
giving a nonzero nilpotent. Furthermore,
\[ (\alpha^p \ot 1 + 1 \ot \alpha^p)(\alpha^p \ot 1 - 1 \ot \alpha^p) = (\alpha^{2p} \ot 1 - 1 \ot \alpha^{2p}) = 0  \]
giving a non-nilpotent zero divisor since if
\[ (\alpha^p \ot 1 + 1 \ot \alpha^p)^n = 0 \]
then this must be true for all sufficiently large $n$ so we can take $n = 2 p^k$ for a large enough power $k$ and thus we get,
\[ (\alpha^p \ot 1 + 1 \ot \alpha^p)^n = (t \ot 1 + 2 \alpha^p \ot \alpha^p + 1 \ot t)^{p^k} = 2(t^{p^k} \ot 1 + \alpha^{p^{k+1}} \ot \alpha^{p^{k+1}}) = 0 \] 
but for degree reasons this.
\end{enumerate}

\subsubsection{3.16}

Let $P$ be a property of closed subsets of a noetherian topological space $X$. Suppose that each closed subset $Y \subset X$ has the property that if $P$ is true for each proper closed subset of $Y$ then $P$ holds for $P$. Furthermore, suppose that $P$ holds for $\varnothing \subset X$.
\bigskip\\
Let $\Sigma = \{ Y \subset X \mid Y \text{ is closed and property P fails} \}$ which is a poset under inclusion. Assume that $\Sigma$ is nonempty. By the Noetherian propery all chains have a least element and thus, by Zorn's Lemma, $\Sigma$ has a least element $Y \in \Sigma$. Thus, any proper closed subset of $Y$ cannot lie in $\Sigma$ and thus has property $P$. By the induction assumption $Y$ has property $P$ contradicting $Y \in \Sigma$ so the assumption that $\Sigma \neq \varnothing$ must be false. Therefore, $X \notin \Sigma$ so $X$ has property $P$.

\subsubsection{3.17 (CHECK THIS!!)}

A topological space $X$ is a \textit{Zariski space} if it is noetherian and sober (every irreducible closed set has a unique generic point).

\begin{enumerate}
\item (DONE IN MY NOTES)

\item Let $X$ be a Zariski space. Let $Z \subset X$ be a nonempty closed subset. Suppose that $Z$ has more than one point. Then choose $x \in Z$ which is not the (unique) generic point. Then $\overline{ \{ x \} } \subsetneq Z$ must be closed but cannot contain the generic point $\xi \in Z$ else $\overline{\{ x \}} = Z$ which would imply that $x \in Z$ is generic. Therefore, $Z$ is not minimal. Thus if $Z$ is minimal then $Z = \{ \xi \}$.

\item Let $X$ be a Zariski space and let $x, y \in X$ be points. Let $Z_x = \overline{ \{ x \} }$ and $Z_y = \overline{ \{ y \} }$. If $x \notin Z_y$ then $x \in Z_y^C$ is an open containing $x$ and not $y$. If $y \notin Z_x$ then $y \in Z_x^C$ is an open containing $y$ and not $x$. Otherwise $x \in Z_y$ and $y \in Z_x$. Then because $Z_y$ is closed $Z_x \subset Z_y$ and because $Z_x$ is closed $Z_y \subset Z_x$ and thus $Z_x = Z_y = Z$ so $x, y \in Z$ are both generic points which implies that $x = y$ so $X$ is $T_0$.

\item If $X$ is an irreducible Zariski space and $\xi \in X$ be the unique generic point of the unique irreducible component. Let $U \subset X$ be an open. If $\xi \notin U$ then $\xi \in U^C$ which is closed so $U^C = X$ because $\overline{ \{ \xi \} } = X$ and thus $U$ is empty.

\item If $x_0, x_1 \in X$ are points and if $x_0 \in \overline{ \{ x_1 \} }$ we say that $x_1$ \textit{specializes} to $x_0$ written $x_1 \leadsto x_0$. We also say that $x_0$ is a \textit{specialization} of $x_1$ or that $x_1$ is a \textit{generalization} of $x_0$. Let $X$ be a Zariski space. 
\bigskip\\
Let $x$ be minimal for $x_1 \ge x_0$ if $x_1 \leadsto x_0$. Then if $y \in \overline{\{ x \}}$ then $x \leadsto y$ so $x \ge y$ and thus $x = y$ and thus $\overline{ \{ x \} } = \{ x \}$ so $x$ is a closed point. Furthermore, if $x$ is a closed point then $\overline{ \{ x \} } = \{ x \}$ so if $x \ge y$ then $y \in \overline{ \{ x \} }$ so $y = x$ and thus $x$ is minimal. 
\bigskip\\
Similarly, if $x$ is maximal, then let $Z = \overline{ \{ x \} }$. We can embedd $Z \subset Z'$ where $Z'$ is an irreducible component and $Z'$ has a generic point $\xi \in Z'$ then $\xi \leadsto x$ so by maximality $x = \xi$ and thus $x$ is the generic point of an irreducible component. Conversely, if $\xi \in Z$ is the generic point of an irreducible component then for $y \ge \xi$ then $\xi \in \overline{ \{ y \} }$ so $Z \subset \overline{ \{ y \} }$ but $Z$ is maximal (an irreducible component) so $Z = \overline{ \{ y \} }$ so $y = \xi$ by uniqueness of the generic point.
\bigskip\\
Let $Z$ be closed and $x_1 \in Z$ and $x_1 \leadsto x_0$. Then $x_0 \in \overline{ \{ x_1 \} } \subset Z$ because $Z$ is closed so $x_0 \in Z$ and thus $Z$ is stable under generalization. Let $U$ be open and $x_0 \in U$ and $x_1 \leadsto x_0$. Then $U^C$ is closed and $x_0 \notin U^C$ so because $U^C$ is stable under specialization if $x_1 \in U^C$ then $x_0 \in U^C$ so $x_1 \notin U^C$ and thus $x_1 \in U$ so $U$ is stable under generalization.

\item Let $t$ be the ``soberification'' functor on topological spaces. If $X$ is a noetherian topological space then $t(X)$ is still noetherian because $X \to t(X)$ induces a bijection on closed subsets. Furthermore, $t(X)$ is sober by construction because there is a unique point corresponding to each irreducible closed set. If $X \to t(X)$ is a homeomorphism then $X$ is a Zariski space because $t(X)$ is. Furthermore, if $t(X)$ is a Zariski space then $X \to t(X)$ is an isomorphism because it is a closed continuous bijection (CHECK THIS).
\end{enumerate}

\subsubsection{3.18}

\newcommand{\cF}{\mathcal{F}}

Let $X$ be a Zariski topological space. A \textit{constructible subset} of $X$ is a subset which belongs to the smallest family $\cF$ of subsets such that,
\begin{enumerate}
\item $\cF$ contains every open
\item $\cF$ is closed under finite intersections
\item $\cF$ is closed under complements.
\end{enumerate}

\begin{enumerate}
\item Since by complements and finite intersections $\cF$ is closed under finite unions. Therefore, it is clear that any finite union of locally closed subsets is in $\cF$. To prove the converse, it suffices to shows that the set $\cF'$ of finite disjoint unions of locally closed subsets satisfies the properties of $\cF$. Since an open is locally closed, $\cF'$ contains every open. Next, I will show that $\cF'$ is closed under finite intersections. Consider,
\[ C \cap C' = \left( \bigcup_{i = 1}^n U_i \cap Z_i \right) \cap \left( \bigcup_{j = 1}^m U_j' \cap Z_j' \right) \]
I prove this by induction on $n$. For the case $n = 1$,
\[ C \cap C' = \bigcup_{j = 1}^n (U_1 \cap Z_1) \cap (U_j' \cap Z_j') = \bigcup_{j = 1}^n (U_1 \cap U_j') \cap (Z_1 \cap Z_j') \]
is a union of disjoint (because the $U_i' \cap Z_i'$ are already disjoint) locally closed subsets. Now for the induction step we write,
\[ C = (\tilde{C} \cup (U_n \cap Z_n)) \]
and therefore,
\[ C \cap C' = \tilde{C} \cap C' \cup (U_n \cap Z_n) \cap C' \]
but $\tilde{C} \cap C' \in \cF'$ by the induction hypothesis and $(U_n \cap Z_n) \cap C' \in \cF'$ by the base case and furthermore $\tilde{C}$ and $U_n \cap Z_n$ are disjoint so we see that $C \cap C' \in \cF'$. Now we show that $\cF'$ is closed under complements. Indeed, for any $C \in \cF'$ write,
\[ C = \bigcup_{i = 1}^n (U_i \cap Z_i) \]
and therefore,
\[ C^C = \bigcap_{i = 1}^n (U_i^C \cup Z_i^C) \]
Thus by the finite intersection property it suffices to show that $U_i^C \cup Z_i^C \in \cF'$. Indeed,
\[ U_i^C \cup Z_i^C = U_i^C \cup (Z_i^C \cap U_i) \]
is a disjoint union of locally closed (in this case a closed and an open) subsets and thus $C^C \in \cF'$. Therefore, using complements and intersections we see that $\cF'$ is closed under finite unions and thus $\cF \subset \cF'$ but obviously $\cF' \subset \cF$ proving that $\cF = \cF'$.

\item Let $X$ be an irreducible Zariski space with generic point $\xi \in X$ and $C \subset X$ constructible. If $\xi \in C$ then $X = \overline{ \{ \xi \} } \subset \overline{C}$ so $\overline{C} = X$ meaning $C$ is dense. Conversely, suppose that $C$ is dense. Write,
\[ C = \bigcup_{i = 1}^n U_i \cap Z_i \subset \bigcup_{i = 1}^n Z_i \]
and the union of the $Z_i$ is finite so it is closed. Therefore, if $C$ is dense then 
\[ \bigcup_{i = 1}^n Z_i = X \]
but $X$ is irreducible so we must have $Z_i = X$ for some $i$. Therefore, $U_i \subset C$ so $C$ contains a nonempty open and we know every nonempty open contains $\xi$ so $\xi \in C$.

\item Let $S \subset X$ be closed then $S$ is constructible and stable under specialization. Conversely, let $S$ be constructible and stable under specialization. Write,
\[ S = \bigcap_{i = 1}^n U_i \cap Z_i \]
We can assume that each $U_i \cap Z_i \neq \empty$ and $Z_i$ is irreducible since in a noetherian space every closed is a finite union of its irreducible components. Then the generic points $\xi_i \in Z_i$ are in $U_i$ because otherwise $U_i \cap Z_i = \empty$ since $U_i \cap Z_i \subset Z_i$ is open. This $\xi_i \in S$. For any $x \in Z_i$ then $\xi_i \leadsto x$ so $x \in S$ because it is stable under specialization and thus $Z_i \subset S$ meaning that,
\[ \bigcup_{i = 1}^n Z_i \subset S \subset \bigcup_{i = 1}^n Z_i \]
and thus $S = \bigcup_{i = 1}^n Z_i$ so $S$ is closed since the union is finite. (DOES THIS NEED NOETHERIAN?)

\item If $f : X \to Y$ is a continuous map of Zariski spaces and $C \subset Y$ is constructible. Then $f^{-1}(C)$ is constructible because the preimage of open subsets are open and closed subsets are closed and $f^{-1}$ preserves unions and intersections. Then writing,
\[ C = \bigcup_{i = 1}^n U_i \cap Z_i \]
we see that,
\[ f^{-1}(C) = \bigcup_{i = 1}^n f^{-1}(U_i) \cap f^{-1}(Z_i) \]
is constructible. 
\end{enumerate}

\subsubsection{3.19 (TELL DANIEL HIS SOLUTION HAS A MISTAKE IN IT I THINK NEED NOETHERIAN)}

Let $f : X \to Y$ be a morphism of finite type of noetherian schemes. 

\begin{enumerate}
\item Because $f(C_1 \cup C_2) = f(C_1) \cup f(C_2)$ it suffices to show that $f(C)$ is constructible where $C \subset X$ is locally closed with irreducible closure since $X$ is noetherian so every closed set is a union of its irreducible components. Write $C = U \cap Z$ then reducing to $g : Z \to \overline{g(Z)}$ with the reduced subscheme structure we need to show that $g(U')$ is constructible where $U' = U \cap Z \subset Z$ is open. Because $Z$ is irreducible then $Y' = \overline{g(Z)}$ is also irreducible so these are integral schemes with their reduced structures. Since $Z$ and $Y'$ are noetherian and thus quasi-compact, we can cover $Y'$ by finitely many affine opens $V_i$ and cover $Z$ by finitely many affine opens $U_{ij}$ with $g(U_{ij}) \subset V_i$. Then $g(Z) = \bigcup_{ij} g(U_{ij})$ so it suffices to prove that $g(U_{ij}) \subset V_i$ is constructible. However, $U_{ij}$ and $V_i$ are noetherian integral affine schemes and $g : U_{ij} \to V_i$ is finite type and dominant because $g : Z \to Y'$ is dominant and these are irreducible so every open contains the generic point of $Z$ which maps to the generic point of $Y'$. 

\item We use the following commutative algebra fact: let $A \subset B$ be an finite type extension of noetherian domains, given a nonzero $b \in B$ there exists a nonzero $a \in A$ such that if $\varphi : A \to K$ is a homomoprhism with $K$ algebraically closed and $\varphi(a) \neq 0$ then we can extend,
\begin{center}
\begin{tikzcd}
K
\\
A \arrow[u, "\varphi"] \arrow[r, hook] & B \arrow[lu, "\varphi"', dashed] 
\end{tikzcd}
\end{center}
such that $\varphi'(b) \neq 0$.
\bigskip\\
First we apply this to prove that if $f : \Spec{B} \to \Spec{A}$ is a dominant finite type morphism of noetherian integral schemes then $f(\Spec{B})$ contains a nonempty open. We get a finite type map $\psi : A \embed B$ which is injective because $f$ is dominant. Taking $b = 1$ there exists $a \in A$ with the above property. For any prime $\p \subset A$ with $a \notin \p$ we get a map $\varphi : A \to \overline{\Frac{A/\p}}$ such that $\varphi(a) \neq 0$ because $a \notin \p$ and therefore, this extends to a nonzero map $\varphi' : B \to \overline{\Frac{A/\p}}$. Since $\psi : A \embed B$ is injective we see that $\p = \ker{\varphi} = \psi^{-1}(\ker{\varphi'})$ and therefore $\p \in f(\Spec{B})$ so $D(a) \subset f(\Spec{B})$ proving the claim.
\bigskip\\
Now we prove the algebra fact. Suppose there are elements $t_1, \dots, t_n \in B$ such that 
\[ A[t_1, \dots, t_n] = B \]
We proceed by induction on $n$. The case $n = 1$ is the base case and also proves the induction step because $B = (A[t_1, \dots, t_{n-1}])[t_n]$ so if we can extend to $B' = A[t_1, \dots, t_{n-1}]$ then we can extend to $B = B'[t_n]$.
\bigskip\\
Consider the case $n = 1$ meaning there is $t \in B$ such that $A[t] = B$. Thus the map $A[x] \onto B$ via $x \mapsto t$ gives $A[x]/\p \cong B$ for some prime $\p \subset A[x]$. Because $A \embed B$ we see that $\p \cap A = (0)$ hence there is some prime $\q \subset Q[x]$ for $Q = \Frac{A}$ with $\p = \q \cap A[x]$. Since $Q[x]$ is a PID $\q = (f)$ for some $f \in A[x]$ (clearing denominators if necessary). Let $I = f A[x]$. Since $f \in \q \cap Q[x]$ clearly $f \in \p$ so $I \subset \p$. Because $A$ is noetherian, so is $A[x]$ and thus $\p = (p_1, \dots, p_r)$. Since $\p = \q \cap A[x]$ we see that $\p \subset \q$ and therefore $f \divides p_i$ in $Q[x]$ so there are polynomials $q_1, \dots, q_r \in Q[x]$ such that $p_i = f q_i$. Let $c$ be the product of the denominators of the $q_i$ then $p_i = f q_i$ in $A_c[x]$ and therefore $\p_c \subset I_c$ as ideals of $A_c[x]$ so $\p_c = I_c$. Thus, there is an inclusion $A[x]/\p \embed A_c[x]/I_c = A_c[x] / f A_c$. Moreover, write,
\[ b = \sum_{i = 0}^r c_i t^i \]
with $c_i \in A$ and thus we can define $g \in A[x]$ by,
\[ g(x) = \sum_{i = 0}^r c_i x^i \]
so that $b = g(t)$. Since $b \neq 0$ we see that $f$ does not divide $g$ in $Q[x]$ else $g \in \q \cap A = \p$ which would imply that $b = g(t) = 0$. Now either $f = 0$ or $f$ is irreducible. If $f = 0$ then choose $a = c c_0$. Otherewise we can find $s_1, s_2 \in Q[x]$ such that $s_1 f + s_2 g = 1$ then clearing denominators we get $\tilde{s}_1, \tilde{s}_2 \in A[x]$ with $\tilde{s}_1 f + \tilde{s}_2 g = h$ then choose $a = ch$.
\bigskip\\
Given a homomorphism $\varphi : A \to K$ with $\varphi(a) \neq 0$ we can choose a root $\alpha \in K$ of the polynomial $\varphi(f) \in K[x]$ because $K$ is algebraically closed. Therefore, we get a map $A[x] \to K$ sending $x \mapsto \alpha$ and $\varphi(a) \neq 0$ so $c$ and $h$ are invertible so we get a map $A_c[x] \to K$ such that $f(x) \mapsto \varphi(f)(\alpha) = 0$ and therefore this descends to a map $A_c / (f) \to K$ so we get,
\[ \varphi' : B \cong A[x]/\p \embed A_c[x]/(f) \tolabel{\bar{\varphi}} K \]
such that $\varphi'(b) = \varphi(g)(b)$ but $\tilde{s}_1(t) f(t) + \tilde{s}_2(t) g(t) = h$ meaning,
\[ \varphi(h) = \varphi'(\tilde{s}_1(t)) \varphi(f)(\alpha) + \varphi'(\tilde{s}_2(t)) \varphi'(b) \]
but $\varphi(f)(\alpha) = 0$ and therefore,
\[ \varphi(h) = \varphi'(\tilde{s}_2(t)) \varphi'(b) \]
and $\varphi(h) \neq 0$ so $\varphi'(b) \neq 0$ proving the claim.

\item Let $f : X \to Y$ be a finite type dominant morphism of affine integral noetherian schemes. We need to show that $f(X)$ is constructible. We have shown that $f(X)$ contains a dense open. We prove that $f(X)$ is constructible by Noetherian induction on $Y$. We consider the poset of closed irreducible subsets $Z \subset Y$ such that $f(X) \cap Z$ is constructible. It is clear that if $Z = \empty$ then $f(X) \cap = \empty$ is constructible. Now we do the inductive step. Suppose that $Z \subset Y$ is closed such that for every proper irreducible subset $W \subset Z$ we have $W \cap f(X)$ is constructible. If $f(X) \cap Z$ is not dense in $Z$ then it is contained in a proper closed subset $W \subsetneq Z$ so $f(X) \cap Z = f(X) \cap W$ is constructible by the induction hypothesis. Otherwise $f : f^{-1}(Z) \to Z$ is a dominant finite type morphisms of noetherian affine schemes. Furthermore $Z$ with its reduced structure is integral so each irreducible component $Z' \subset f^{-1}(Z)$ with its reduced structure is integral and some $Z'$ must hit the generic point so there is a dominant $f : Z' \to Z$. Then applying the previous part $f(Z')$ contains a dense open $U$ and thus $U \subset f(X) \cap Z$. Therefore $f(X) \cap Z = U \cup (f(X) \cap U^C)$ is constructible because $f(X) \cap U^C$ is constructible by the induction hypothesis and $U$ is open. Therefore by Noetherian induction we see that $f(X) = f(X) \cap Y$ is constructible.

\item Consider $V(xy - 1) \subset \A^2_{\CC} \to \A^2_{\CC}$ where the map $\A^2_{\CC} \to \A^2_{\CC}$ is the map $(x,y) \mapsto (x, 0)$. If the image is open or closed then its intersection with the closed points is also open or closed. However, its image on closed points is $\{ (x,0) \mid x \neq 0 \}$ which is neither open nor closed.
\bigskip\\
A similar example is given by $\A^2 \to \A^2$ sending $(x,y) \mapsto (xy, x)$ whose image is the set $ \{ (x,y) \mid y \neq 0 \} \cup \{ (0, 0) \}$ which is neither open nor closed.
\end{enumerate}

\subsubsection{3.20 (CHECK THIS!!)}

\begin{lemma}
Let $Y \subset X$ be any subset then $\dim{Y} \le \dim{X}$. 
\end{lemma}

\begin{proof}
Let $Y = Z_0 \supsetneq Z_1 \supsetneq \cdots \supsetneq Z_d$ be a maximal chain of closed irreducible sets in $Y$. Then consider $W_i = \overline{Z_i}$. I claim that $W_i$ is irreducible and forms a proper chain. First, if $W_i \supset C_1 \cup C_2$ for closed $C_1, C_2$ then $Z_i \subset W_i \subset C_1 \cup C_2$ but $Z_i$ is irreducible so (WLOG) $Z_i \subset C_1$ and thus $W_i = \overline{Z_i} \subset C_1$ since $C_1$ is closed. Thus $W_i$ is irreducible. 
\bigskip\\
Now suppose that $\overline{Z_i} = \overline{Z_{i+1}}$ then $Z_i \subset \overline{Z_{i+1}}$ which implies that $Z_i \subset \overline{Z_{i + 1}} \cap Y = Z_{i + 1}$ (since $Z_{i + 1}$ is closed in $Y$) which contradicts the fact that $Z_i \supsetneq Z_{i+1}$. Thus the chain is proper so $\dim{X} \ge d = \dim{Y}$.  
\end{proof}

\begin{definition}
If $Z \subset X$ is irreducible closed then $\codim{Z,X}$ is the length of the longest chain,
\[ Z = Z_0 \subsetneq Z_1 \subsetneq \cdots \subsetneq Z_n \]
Furthermore, if $Y \subset X$ is closed then we define,
\[ \codim{Y, X} = \inf_{Z \subset Y} \codim{Z, X} \]
\end{definition}

\begin{lemma}
Let $X$ be a topological space and $Z \subset X$ be an irreducible closed subspace. Let $U \subset X$ be open such that $U \cap Z$ is nonempty then $\codim{Z, X} = \codim{Z \cap U, U}$.
\end{lemma}

\begin{proof}
Let $Z = Z_0 \subsetneq Z_1 \subsetneq \cdots \subsetneq Z_d$ be a maximal chain of closed irreducible subsets in $X$ so $d = \dim{X}$. Consider,
\[ U \cap Z_0 \subset U \cap Z_1 \subset \cdots \subset U \cap Z_d \]
Then $U \cap Z_i$ is nonempty (contains $U \cap Z$) and thus is irreducible and furthermore is closed in $U$. Consider $(U^C \cap Z_i) \cup \overline{U \cap Z_i} \supset Z_i$ and both are closed so $\overline{U \cap Z_i} \supset Z_i$ since $U \cap Z_i$ is non empty which implies that $U^C \cap Z_i$ does not contain $Z_i$. Furthermore, $Z_i \supset U \cap Z_i$ and $Z_i$ is closed so $Z_i \supset \overline{U \cap Z_i}$. Therefore, $\overline{U \cap Z_i} = Z_i$. Thus we cannot have $U \cap Z_{i} = U \cap Z_{i+1}$ since their closures are $Z_i \subsetneq Z_{i+1}$ which are distinct. Thus, $\codim{U \cap Z, U} \ge d = \codim{Z, X}$ and by the argument in the previous lemma, $\codim{U \cap Z, U} \le \codim{Z, X}$.
\end{proof}

\begin{corollary}
If $U \subset X$ is open and contains a point $x \in X$ such that $\dim_x(X) = \dim{X}$ then $\dim{U} = \dim{X}$.
\end{corollary}

\begin{proof}
We have, \[ \dim_x(X) = \codim{x, X} = \codim{x, U} = \dim_x(U) \le \dim{U} \]
Thus, $\dim{X} \le \dim{U}$ but we have shown that $\dim{U} \le \dim{X}$.
\end{proof}

\begin{lemma}
Let $X$ be sober and $Z \subset X$ a closed irreducible subspace and $\xi \in Z$ its generic point. Then $\codim{Z, X} = \dim{\stalk{X}{\xi}}$.
\end{lemma}

\begin{proof}
Choose any affine open $U \subset X$ containing the generic point $\xi \in Z$. Then,
\[ \codim{Z, X} = \codim{Z \cap U, U} \]
However, $U = \Spec{A}$ and $Z \cap U = V(\p)$ since it is closed irreducible in $U$ with $\p = \xi \in U$ its generic point. Then chains of irreducible subsets about $V(\p)$ correspond to chains of prime ideals below $\p$ so,
\[ \codim{Z \cap U, U} = \codim{V(\p), \Spec{A}} = \height{(\p)} = \dim{A_\p} = \dim{\stalk{X}{\xi}} \]
\end{proof}

Let $X$ be an integral scheme of finite type over a field $k$.
\begin{enumerate}
\item Let $x \in X$ be a closed point and $U \subset X$ any affine open containing $x$. Then $U = \Spec{A}$ with $A$ a finitely-generated $k$-algebra domain with $x$ corresponding to $\m \subset A$. In this case, the height satisfies,
\[ \dim{\stalk{X}{x}} = \dim{A_\m} = \height{\m} = \dim{A} - \dim{A / \m} \]
However, $A / \m$ is a field so $\dim{A / \m} = 0$. Thus,
\[ \dim{\stalk{X}{x}} =  \dim{A} \]
Now, let $X = Z_0 \supsetneq Z_1 \supsetneq \cdots \supsetneq Z_d = \{ p \}$ (with $p \in X$ closed) be a maximal chain of closed irreducible subsets so $d = \dim{X}$. Take an affine open $V = \Spec{B}$ with $p \in V$ then by the same argument $\dim{B} = \dim{\stalk{X}{p}}$. Furthermore, since $X$ is irreducible $U \cap V$ is nonempty open. Furthermore, since $X$ is finite type over $k$ closed points are dense so we can choose a closed point $q \in U \cap V$ and then we have,
\[ \dim{\stalk{X}{x}} = \dim{A} = \dim{\stalk{X}{q}} = \dim{B} = \dim{\stalk{X}{p}} \]
By the lemma, $\dim{V} = \dim{X}$ and thus,
\[ \dim{X} = \dim{V} = \dim{B} = \dim{\stalk{X}{p}} =  \dim{\stalk{X}{x}} \]
\item For any finitely-generated $k$-algebra domain $A$ we have $\dim{A} = \trdeg{k}{\Frac{A}}$ and thus, for any affine open $U \subset X$ with $U = \Spec{A}$ then we have shown,
\[ \dim{X} = \dim{A} = \trdeg{k}{\Frac{A}} \]
However, $\Frac{A} = K(X)$ is the function field $\stalk{X}{\xi}$ so we have,
\[ \dim{X} = \trdeg{k}{K(X)} \]
\item 
Now, in the case of integral schemes of finite type over $k$, consider a closed subset $Y \subset X$ then,
\[ \codim{Y, X} = \inf_{Z \subset Y} \codim{Z, X} \]
for irreducible closed subsets $Z \subset Y$. First, for an irreducible closed subset $Z \subset X$ 

We know that $\codim{Z, X} = \dim{\stalk{X}{\xi}}$ where $\xi \in Z$ is the generic point. Therefore, 
\[ \codim{Y, X} = \inf \{ \dim{\stalk{X}{\xi}} \mid \xi \in Y \} \]
since schemes are sober so points of $Y$ correspond exactly to closed irreducibe subspaces via closure. 

\item 

Let $Y \subset X$ be a closed subset. First, suppose that $Y$ is irreducible then for any affine open $U \subset X$ intersecting $Y$ we have,
\[ \codim{Y, X} = \codim{U \cap Y, U} \]
Furthermore, $U = \Spec{A}$ and $Y \cap U$ is irreducible closed in $U$ so $Y \cap U = V(\p)$. Thus,
\[ \codim{U \cap Y, U} = \codim{V(I), \Spec{A}} =\height{\p} \]
However, since $A$ is a finitely generated $k$-algebra domain,
\[ \dim{A} = \height{\p} + \dim{A / \p} \]
Furthermore, since $U \subset X$ and  $U \cap Y \subset U$ are nonempty open subspaces, we have shown that, $\dim{X} = \dim{U} = \dim{A}$ and,
\[ \dim{Y} = \dim{U \cap Y} = \dim{V(I)} = \dim{A / I} \]
Therefore, we have,
\[ \dim{X} = \codim{Y, X} + \dim{Y} \]
Now suppose that $Y$ is not irreducible we then define,
\[ \codim{Y, X} = \inf_{Z \subset Y} \codim{Z, X} \]
for all irreducible closed subsets of $Y$ which are thus irreducible closed subsets of $X$. Thus,
\[ \codim{Y, X} = \inf_{Z \subset Y} \left[ \dim{X} -  \dim{Z} \right] = \dim{X} - \sup_{Z \subset Y} \dim{Z} = \dim{X} - \dim{Y} \]
since by definition,
\[ \dim{Y} = \sup_{Z \subset Y} \dim{Z} \]
is the maximal length of irreducible closed chains in $Y$. 

\item Let $U \subset X$ be a nonempty open. Since $X$ is finite type over $k$ the closed points of $X$ are dense so there is a closed point $x \in U$ and then we have shown that,
\[ \dim{X} = \dim{\stalk{X}{x}} = \dim{U} \]
since $U$ is a scheme satisfying the conditions. Furthermore, for any affine open $\Spec{A} \subset U$ then we know,
\[ \dim{A} \le \dim{U} \le \dim{X} \]
and $\dim{A} = \dim{X}$ so $\dim{U} = \dim{X}$.
  
\item Assume that $k$ is a perfect field (HOW TO DO IT WHEN NOT PERFECT). Consider an extension $k \embed k'$ and base change to $X \times_k \Spec{k'}$. First, if $A$ is a finitely generated $k$-algebra domain then base chaning to $A' = A \otimes_k k'$ is a finitely generated $k'$-algebra. Furthermore, since $k' / k$ is seperable then $A' = A \otimes_k k'$ is reduced. Therefore, the irreducible components of $X \times_k k'$ are integral scheme of finite type over $k'$. Then, as we have shown, for any affine open $U = \Spec{A'}$ in a irreducible component $\dim{X'} = \dim{U} = \dim{A'}$. Furthermore, $\dim{X} = \dim{A}$ where $\Spec{A}$ is an affine open of $X$. Thus, it suffices to show that $\dim{A} = \dim{A'}$. However, these are finitely-generated $k$-algebra domains so we know that,
\[ \trdeg{k}{\Frac{A}} = \trdeg{k'}{\Frac{A'}} \]
and thus $\dim{A} = \dim{A'}$. (FIX THIS PROOF).
\end{enumerate}

\subsubsection{3.21}

Let $R$ be a discrete valuation ring containing its residue field $k$ for example $R = k[[X]]$ and let $K = \Frac{R} = R_\varpi$. Let $X = \Spec{R[t]}$. We have $\dim{X} = 2$. 
\bigskip\\
However, consider the ideal $\m = (\varpi t - 1) \subset R[t]$ and the quotient $R[t]/(\varpi t - 1) \to K$ via $t \mapsto \varpi^{-1}$ is an isomorphism so $\m = (\varpi t - 1)$ is maximal. However, $\height{\m} = 1$ because it is principal (see Lemma \ref{principal_ideals_height_one}). Thus,
\[ \dim{R[t]_{\m}} = \height{\m} = 1 \]
but $\dim{X} = 2$ so property (a) does not hold.
\bigskip\\
Now consider the closed set $Y = V(\m)$ where $\m = (\varpi t - 1)$ then $\dim{Y} = \dim{R[t] / \m} = \dim{K} = 0$ and $\codim{Y, X} = \height{\m} = 1$ however $\dim{X} = 2$ so,
\[ \dim{X} \neq \codim{Y, X} + \dim{Y} \]
and thus property (d) does not hold. 
\bigskip\\
Consider the open $D(\varpi) \subset X$ then $D(\varpi) \cong \Spec{R[t]_\varpi} = \Spec{K[t]}$. Therefore,
\[ \dim{D(\varpi)} = \dim{K[t]} = 1 \]
but $\dim{X} = 2$ so property (e) does not hold.

\begin{lemma} \label{principal_ideals_height_one}
Let $A$ be Noetherian. Any principal prime ideal $\p = (a)$ has $\height{\p} \le 1$.
\end{lemma}

\begin{proof}
If $\p = (a)$ is prime then consider the local ring$A_{(a)}$ which is Noetherian and has unique maximal ideal $a A_{(a)}$. Thus $A_{(a)}$ is Noetherian and has every maximal ideal prinipal so $A_{(a)}$ is a PID and thus $\height{(a)} = \dim{A_{(a)}} \le 1$.  
\end{proof}

\subsubsection{3.22 CHECK THIS}

(NOTE LOOK AT https://math.stackexchange.com/questions/2360128/hartshorne-exercise-ii-3-22-c-dimension-of-fibers)

Let $f : X \to Y$ be a dominant morphism of integral schemes of finite type over a field $k$

\begin{enumerate}
\item Let $Y' \subset Y$ be a closed irreducible subset of $Y$ whose generic point $\eta' \in Y'$ is contained in $f(X)$. Let $Z \subset f^{-1}(Y')$ be an irreducible component such that $\eta' \in f(Z)$ and $\xi' \in Z$ its generic point. By (Ex. 20) we know that $\codim{Z,X} = \dim \stalk{X}{\xi'}$ and $\codim{Y',Y} = \dim \stalk{Y}{\eta'}$. Because $Z \to Y'$ is dominant we have $\xi' \mapsto \eta'$, and the map $\stalk{Y}{\eta'} \to \stalk{X}{\xi'}$ is injective because $f : X \to Y$ is dominant. Suppose that $\zeta \in X$ is a point with $f(\zeta) = \eta'$. Then $\zeta \in f^{-1}(Y')$ so $\overline{ \{ \zeta \} } \subset f^{-1}(Y')$ because $f^{-1}(Y')$ is closed. Therefore, if $\zeta \leadsto \xi'$ then $\zeta = \xi'$ because $Z$ is an irreducible component of $f^{-1}(Y')$ so $\xi' \in \stalk{X}{\xi'}$ is the unique prime mapping to $\eta' \in \stalk{Y}{\eta'}$. 
\bigskip\\
Since $A = \stalk{Y}{\eta'}$ is a noetherian local ring of dimension $d = \dim \stalk{Y}{\eta'}$ there is an ideal of definition $I \subset A$ genetated by $d$ elements. Let $B = \stalk{X}{\xi'}$ and $\varphi : A \embed B$ the map induced by $f$. Let $\m_A = \eta' \subset A$ and $\m_B = \xi' \subset B$ then $\varphi^{-1}(\m_B) = \m_A$ and $\m_B$ is the unique prime with this property. I claim that $I B$ is an ideal of definition of $B$ and thus $\dim{B} \le d$. It suffices to show that $\sqrt{IB} = \m_B$. Suppose that $\q \subset B$ is a prime with $IB \subset \q$ then $\varphi^{-1}(\q) \supset \varphi^{-1}(IB) \supset I$ but $\sqrt{I} = \m_A$ so $\varphi^{-1}(\q) \supset \m_A$ and thus $\varphi^{-1}(\q) = \m_A$ because $\m_A$ is maximal. Thus by the uniqueness, $\q = \m_B$ so $IB$ is an ideal of definition. Every ideal of definition of $B$ is generated by at least $\dim{B}$ elements so $\dim{B} \le d$. Explicitly, by Krull's height theorem, since $\m_B$ is minimal over $IB$ we see that $\dim{B} = \height{\m_B} \le d$. Therefore, $\dim{B} \le \dim{A}$ proving that,
\[ \codim{Z,X} \le \codim{Y',Y'} \]


\item Let $e = \dim{X} - \dim{Y}$ be the \textit{relative dimension} of $X$ over $Y$. For any point $y \in f(X)$ we need to show that each irreducible component $Z \subset X_y$ we have $\dim{Z} \ge e$. Let $Y' = \overline{ \{ y \} }$. Then $f^{-1}(Y')$ contains the closure of $X_y$. Let $x \in Z$ be the generic point (in $X_y$). 
By [Ex. 3.20(b)] we have $\dim{Y'} = \trdeg{k}{\kappa(y)}$ and $\dim{Z} = \trdeg{\kappa(y)}{\kappa(x)}$ because $X_y$ is a finite type scheme over $\Spec{\kappa(y)}$. Therefore,
\[ \dim{Z} = \trdeg{\kappa(y)}{\kappa(x)} = \trdeg{k}{\kappa(x)} - \trdeg{k}{\kappa(y)} = \dim{\overline{Z}} - \dim{Y'} \]
Notice that any irreducible component $Z' \subset f^{-1}(Y')$ hitting the generic point $y \in Y'$ must have its generic point map to $y$ and thus $Z' \subset \overline{X_y}$. Therefore, $\overline{Z} \subset f^{-1}(Y')$ is an irreducible component. Indeed, if $\overline{Z} \subset Z'$ for $Z'$ irreducible component of $f^{-1}(Y')$ with generic point $\zeta$ then $Z'$ hits $y$ so $Z' \cap X_y$ is dense in $Z'$ ($\zeta \mapsto y$ so $\zeta \in X_y$) but $Z = \overline{Z} \cap X_y \subset Z' \cap X_y$ but $Z$ is an irreducible component of $X_y$ so $\zeta \in Z$ and thus $\overline{Z} = Z'$. Also, $y \in f(\overline{Z})$ and thus by the previous part,
\[ \codim{\overline{Z}, X} \le \codim{Y', Y} \]
and thus using [Ex. 3.20 (d)],
\[ \dim{X} - \dim{\overline{Z}} \le \dim{Y} - \dim{Y'} \]
Plugging in we see that,
\[ \dim{Z} = \dim{\overline{Z}} - \dim{Y'} \ge \dim{X} - \dim{Y} = e \]

\item For any open $U \subset X$ we see that,
\[ \dim{U_y} = \max_{x \in U_y} \trdeg{\kappa(y)}{\kappa(x)} \ge e \]
Since $X$ is integral, every nonempty open $U \subset X$ is already dense and thus we can reduce to $X = \Spec{B}$ and shrink further. Then by shrinking $\Spec{B}$ further we can assume that $f(\Spec{B}) \subset \Spec{A} \subset Y$ for some affine open $\Spec{A} \subset Y$ so we reduce to the affine case with $\varphi : A \to B$ injective since $f : X \to Y$ is dominant. Since $f$ is finite type, $B$ is a finitely generated $A$-algebra. Choose $t_1, \dots, t_e \in B$ (clearing denominators if necessary) which form a transcendence basis of $K(X)$ over $K(Y)$ notice that,
\[ \trdeg{K(Y)}{K(X)} = \trdeg{k}{K(X)} - \trdeg{k}{K(Y)} = \dim{X} - \dim{Y} = e \]
Let $X_1 = \Spec{A[t_1, \dots, t_e]}$ with $A[t_1, \dots, t_e] \subset B$ and thus $g : \Spec{B} \to X_1$ is dominant and $K(X_1) \embed K(X)$ is finite and thus generically finite. Therefore, by [Ex. 3.7] since $g$ is a generically finite and dominant finite type morphism of integral schemes there is a dense open $V \subset X_1$ such that $g^{-1}(V) \to V$ is finite. Taking $U = g^{-1}(V)$ proves the claim because the fibers of $\pi : X_1 \to \Spec{A}$ are $\A^e_{\kappa(y)}$ which have dimension $e$ manifestly. Therefore, for $x \in U$ we have $f(x) = \pi(g(x))$ and thus $U_{f(x)} \to V_{\pi(g(x))}$ is finite because $g : U \to V$ is a finite map over $\Spec{A}$ and $\dim{V_{\pi(g(x))}} = e$ since these are nonempty open subsets of $\A^e_{\kappa(\pi(g(x)))}$. Thus for every $y \in f(U)$ we have $\dim{U_y} = e$ because finite maps preserve dimension.

\item Return to $f : X \to Y$. Let $h$ be an integer and
\[ E_h = \{ x \in X \mid \dim_x{X_{f(x)}} \ge h \} \] 
We showed in (b) that $\dim{X_{f(x)}} \ge e$ for all $x \in X$ and thus $E_e = X$. If $h > e$ then we see that $E_h \subset U^C$ where $U$ is the dense open of (c) because if $x \in U$ then we see that,
\[ \dim_x{X_{f(x)}} \le \dim{U_{f(x)}} = e \]
and thus $x \notin E_h$. The first inequality follows from the definition,
\[ \dim_x{W} = \inf \{ \dim{U} \mid x \in U \text{ open} \} = \sup \{ \dim{Z} \mid x \in Z \text{ irreducible component} \} \]
However, $U$ is a dense open so $U^C$ is a proper closed subset and thus $E_h$ is not dense.
\bigskip\\
Now I claim that $E_h$ is closed. We proceed by induction on $\dim{X}$. The case $E_e = X$ is obviously closed. Then for $h > e$ we have $E_h \subset U^C$. Now I claim that,
\[ E_h = \bigcup_{Z \subset U^C} E_h(Z \to f(Z)) \]
with $Z$ runing over the irreducible components of $U^C$. This holds because for any $x \in E_h$ we have $\dim_x{X_{f(x)}} \ge h$ and therefore $\dim_{x'}{X_{f(x')}} = \dim_x{X_{f(x)}} \ge h$ for all $x'$ on the maximal dimensional irreducible components of $X_{f(x)}$ containing $x$. Therefore, the maximal dimensional irreducible components of $X_{f(x)}$ through $x$ are contained in $E_h$ and thus in some irreducible component $Z \subset U^C$ showing that $x \in E_h(Z \to f(Z))$. Since $\dim_x{(X_{f(x)} \cap Z)} \le \dim_x{(X_{f(x)})}$ the converse is clear as well. Thus by the induction hypothesis applied to the dominant map $f : Z \to f(Z)$ of integral (with the reduced structure) schemes of finite type over $k$ we conclude that $E_h(Z \to f(Z))$ is closed (since $\dim{Z} < \dim{X}$ because $U$ is dense) and since $X$ is noetherian $U^C$ has finitely many irreducible components so $E_h$ is closed. Thus we conclude by induction.

\item For any integer $h$ let,
\[ C_h = \{ y \in Y \mid \dim{X_y} = h \} \]
Notice that if $y \notin f(X)$ then $X_y = \empty$ so $\dim{X_y} = - \infty$ and thus $y \notin C_h$. Furthermore, 
\begin{align*}
\dim{X_y} & = \sup_{x \in X_y} \dim_x X_y = h \iff \exists x \in X_y : \dim_x{X_y} = h \text{ and } \forall x \in X_y : \dim_x{X_y} < h + 1 \
\\
& \iff \exists x \in X_y : \dim_x{X_y} \ge h \text{ and } \forall x \in X_y : \dim_x{X_y} < h + 1
\end{align*}
meaning that $C_h = f(E_h) \setminus f(E_{h+1})$. Since each $E_h$ is closed, by Chevalley's theorem $f(E_h)$ is constructible so $C_h = f(E_h) \setminus f(E_{h+1})$ is constructible.
\bigskip\\
Because $C_e$ is constructible, to show that $C_e$ contains a dense open it suffices to prove that $C_e$ contains the generic point. However, the generic fiber $X_{\eta}$ is an integral scheme (for affine opens $\Spec{B} \to \Spec{A}$ the generic fiber is covered by affine opens $\Spec{B \otimes_A K} \to \Spec{K}$ but $B \otimes_A K$ is a localization of $B$ and thus a domain) and thus 
\[ \dim{X_{\eta}} = \dim_{\xi}{X_{\eta}} = \dim_{\xi}{U_{\eta}} = e \]
because $\xi \in U$ since $U \subset X$ is a dense open therefore proving that $\eta \in C_e$.
\end{enumerate}

\subsubsection{3.23}

Let $V, W$ be varieties (in the Chapter I sense) over an algebraically closed field $k$ then we need to show that $t(V \times W) = t(V) \times_k t(W)$. Because the fiber product is constructed via affine patches, it suffices to show this when $V$ and $W$ are affine. By [Ex. I.3.15] we have $A(V \times W) = A(V) \otimes_k A(W)$ and therefore,
\[ t(V \times W) = \Spec{A(V \times W)} = \Spec{A(V) \otimes_k A(W)} = \Spec{A(V)} \times_k \Spec{A(W)} = t(V) \times_k t(W) \]
completing the proof.

\subsection{4}

\begin{definition}
A morphism $f : X \to Y$ is \textit{proper} if it is separated, of finite type, and universally closed. 
\end{definition}

\begin{lemma}
Finite morphisms are preserved under base change. 
\end{lemma}

\begin{proof}
This is local so we only need to check this for affine schemes. Then it follows from the fact that finite ring maps are preserved under tensor product because surjections are preserved by tensor products.
\end{proof}

\subsubsection{4.1}

Let $f : X \to S$ be a finite morphism. Finite morphisms are affine and thus separated and clearly finite morphisms are of finite type. Furthermore, finite morphisms are closed and finite morphisms are preserved under base change so they are universally closed.  

\subsubsection{4.2 DO!! (DONE WELL IN MY NOTES)}

Let $X$ be a reduced scheme over $S$ and $Y$ be a seperated scheme over $S$. Consider two $S$-morphisms $f,g : X \to Y$ and a dense open set $U \subset X$ such that $f|_U = g|_U$. Now, consider the map $F : X \to Y \times_S Y$ defined by $f$ and $g$. Now consider the diagonal map $\Delta : Y \to Y \times_S Y$ which is a closed immersion because $Y$ is seperated. Since $f|_U = g|_U$ we may factor $F |_U : X \to Y \times_S Y$ as $F|_U = \Delta \circ f|_U$ and thus $F(U) \subset \Delta(Y) \subset Y \times_S Y$ is closed. Therefore, $F(X) = F(\overline{U}) \subset \overline{F(U)} \subset \Delta(Y)$ since it is closed. Therefore, topologically, $f(x) = g(x)$ for $x \in X$. Thus, it remains to prove that the sheaf maps agree. By hypothesis, the map on stalks, $f^\#_x : \stalk{Y}{f(x)} \to \stalk{X}{x}$ and $g^\#_x : \stalk{Y}{g(x)} \to \stalk{Y}{x}$ agree for each $x \in U$. Consider a section $s \in \struct{Y}(V)$ for some open $V \subset Y$. Then consider the section $s' = f^\#(s) - g^\#(s) \in \struct{X}(f^{-1}(V))$. We know that for each $x \in f^{-1}(V) \cap U$ that $s'_x = f_x(s) - g_x(s) = 0$. Now the vanishing $V(s') = \{ x \in f^{-1}(V) \mid s'_x \in \m_x \}$ of the section $s'$ is closed in $f^{-1}(V)$. However, $\forall x \in U \cap f^{-1}(V)$ we know that $s'_x = 0$ so $x \in V(s')$. Thus, $V(s')$ is a closed set containing a dense set and thus $V(s') = f^{-1}(V)$. Therefore, $s'_x \in \m_x$ for each $x \in f^{-1}(V)$. Thus, on each affine open the restriction of $s'$ lies in every prime ideal and thus in the nilradical. Thus $s' \in \nilrad{\struct{X}(f^{-1}(V)}$. Since $X$ is reduced, $\nilrad{\struct{X}(f^{-1}(V)} = 0$ so $s' = 0$. Thus $f^\# = g^\#$. 

(FINISH)

\subsubsection{4.3}

Let $X$ be a separated scheme over an affine scheme $S = \Spec{A}$ and $U, V \subset X$ be affine open subsets. The diagonal morphism $\Delta : X \to X \times_S X$ is a closed immersion which is affine. Then conider $U \times_S V \subset X \times_S X$ which is affine and we have $U \cap V = \Delta^{-1}(U \times_S V)$ is affine since $\Delta$ is affine.
\bigskip\\
However, take $\A^2_k$ with two origins. Then each copy of $\A^2_k$ is clearly affine but their intersection is $\A^2_k \setminus \{ 0 \}$ which is not affine. 

\begin{lemma}
Closed immersions are affine.
\end{lemma}


\subsubsection{4.4 DO!! (DONE WELL IN MY NOTES)}

(LOOK UP IN NOTES)

\subsubsection{4.5 (CHECK CHOW VERSION!!!!!)}

Let $X$ be an integral scheme of finite type over a field $k$ with function field $K$. We say that a valuation of $K / k$ has \textit{center} $x$ on $X$ if the valuation ring $R \subset K$ dominates the local ring $\stalk{X}{x}$.

\begin{enumerate}
\item Let $X$ be separated and suppose that $R \subset K$ has centers $x,y \in X$. Then $R$ dominates $\stalk{X}{x}$ and $\stalk{Y}{y}$. However, by the valuative criterion of separatedness for the diagram,
\begin{center}
\begin{tikzcd}
\Spec{K} \arrow[d] \arrow[r] & X \arrow[d]
\\
\Spec{R} \arrow[ru, dashed] \arrow[r] & \Spec{k}
\end{tikzcd}
\end{center}
any lift $\Spec{R} \to X$ is unique. However, either local map $\stalk{X}{x} \embed R$ and $\stalk{X}{y} \embed R$ gives such a lift via $\Spec{R} \to \Spec{\stalk{X}{x}} \to X$ and $\Spec{R} \to \Spec{\stalk{Y}{y}} \to Y$ so $x = y$.

\item Let $X$ be proper over $k$ and let $R \subset K$ be a valuation ring containing $k$. Then by the valuative criterion of properness for the diagram,
\begin{center}
\begin{tikzcd}
\Spec{K} \arrow[d] \arrow[r] & X \arrow[d]
\\
\Spec{R} \arrow[ru, dashed] \arrow[r] & \Spec{k}
\end{tikzcd}
\end{center}
there exists a unique lift $\Spec{R} \to X$ giving a local map $\stalk{X}{x} \to R$ for $x \in X$ the image of the maximal ideal of $R$. Since $R \subset K$ this is an injective local map $\stalk{X}{x} \embed R$ so $R$ has a center at $x$. Since $X$ is separated, the previous part shows that the center is unqiue.

\item First we prove the separatedness part. Assume that a every valuation ring of $K/k$ has at most one center. To show that $X$ is separated, it suffices to show that $\Delta : X \to X \times_k X$ is closed. Let $\xi \in X$ be the generic point. Then let $Z = \overline{ \{ \Delta(\xi) \} } = \overline{\Delta(X)}$. We need to show that $\Delta(X) = Z$. First notice that $\Delta : X \to Z$ and $\pi_i : Z \to X$ are birational morphisms because $\pi_i \circ \Delta = \id$ so the map $\Delta^\# : K(Z) \to K(X)$ has a section $\pi_i^\# : K(X) \to K(Z)$ so that $\Delta^\# \circ \pi_i^\# = \id$ proving that $\Delta^\#$ is surjective but $K(Z) \to K(X)$ is automatically injective so it is an isomorphism and thus $\pi_i^\#$ is also an isomorphism. Thus for $z \in Z$ consider $g : \Spec{\stalk{Z}{z}} \to Z$. Choose a valuation ring $R$ of $K/k$ dominating $\stalk{Z}{z}$. Let $\pi_i(z) = x_i$ then we get local maps $\stalk{X}{x_i} \to \stalk{Z}{z}$ induced by $\pi_i$ which are thus inclusions inside $K$ so $R$ dominates $\stalk{X}{x_i}$ so by the uniqueness of centers $x_1 = x_2$ and there is a unique map $\stalk{X}{x_i} \to \stalk{Z}{z}$ as subrings of $K$ so $\pi_1 \circ g = \pi_2 \circ g$ because maps $\Spec{\stalk{Z}{z}} \to X$ are determined by a point $x \in X$ and a local map $\stalk{X}{x} \to \stalk{Z}{z}$. Therefore $g$ factors through $\Delta$ so in particular $z \in \Delta(X)$ proving that $\Delta(X) = Z$ so $\Delta$ is closed and $X$ is separated.
\bigskip\\
Now assume that every valuation ring of $K/k$ has a unique center. By the uniqueness we see that $X$ is separated. We apply the following version of Chow's lemma: let $X$ be a finite type separated $k$-scheme then there exists a proper surjective birational morphism $g : X' \to X$ such that $X'$ is integral and quasi-projective [EGA II, 5.6]. Choose an immersion $\iota : X' \embed \P^n$. It suffices to prove that $\iota$ is closed because in that case $\iota$ is a closed immersion so $X'$ is proper and thus its image under $X' \to X$ is proper but $g$ is surjective proving that $X$ is proper.
\bigskip\\
Let $\xi' \in X'$ be the generic point. Let $Z = \overline{ \{ \iota(\xi') \} } = \overline{\iota(X')}$ then $y \in Z$ so giving $Z$ the reduced structure it is an integral scheme and $\iota : X' \to Z$ is birational because (WHY!!). It suffices to show that $\iota(X) = Z$. For any $z \in Z$ let $R$ be a valuation ring of $K/k$ dominating $\stalk{Z}{z}$ (recall that $K = K(Z) = \Frac{\stalk{Z}{z}}$ via $\iota$). Then by the existence property there is a center $x \in X$ of $R$ so $R$ dominates $\stalk{X}{x}$ giving a morphism $\Spec{R} \to X$ fitting into the diagram,
\begin{center}
\begin{tikzcd}
\Spec{K} \arrow[d] \arrow[r] & X' \arrow[d, "g"]
\\
\Spec{R} \arrow[ru, dashed] \arrow[r] & X 
\end{tikzcd}
\end{center}
but $g : X' \to X$ is proper so there exists a lift $\Spec{R} \to X'$ call the image of the maximal ideal $x' \in g^{-1}(x)$. Thus $R$ dominates $\stalk{X'}{x'}$ in $K$. Now, let $\iota(x') = z'$ so under $\iota$ we get a local map $\stalk{Z}{z'} \to \stalk{X'}{x'}$ as subrings of $K$ meaning that $R$ dominates $\stalk{Z}{z'}$ but $Z \embed \P^n$ so $Z$ is separated and therefore centers in $K$ on $Z$ are unique meaning that $z = z'$ proving that $z \in \iota(X')$ and thus $\iota(X') = Z$ so $\iota$ is closed and thus $X'$ and thus also $X$ are proper.

\item Let $X$ be proper over $k$. Then $\Gamma(X, \struct{X})$ is a finite (finiteness of cohomology) $k$-algebra domain and thus a finite field extension of $k$ so if $k$ is algebraically closed $\Gamma(X, \struct{X}) = k$.
\bigskip\\
Alternatively, let $k$ be algebraically closed and let $a \in \Gamma(X, \struct{X})$ and $a \notin k$. Consider the subring $A = k[a^{-1}] \subset K$ and consider the ideal $(a^{-1}) \subset A$. If $(a^{-1}) = A$ then $a \in A$ meaning,
\[ a = c_n a^{-n} + \cdots + c_0 \]
and therefore dividing by $a$ we see that $a^{-1}$ is algebraic over $k$ so by algebraic closure $a^{-1} \in k$ contradicting our assumption on $a$ so $(a^{-1})$ is a proper ideal. Furthermore $A / (a^{-1}) = k$ so it is maximal. Then $A_{(a^{-1})} \subset K$ is a local subring so there is some valuation ring $R \subset K$ dominating $A_{(a^{-1})}$ (see \chref{https://stacks.math.columbia.edu/tag/00IA}{Tag 00IA}) and thus $(a^{-1}) = \m_R \cap A_{(a^{-1})}$ so $a^{-1} \in \m_R$. By properness, $R$ has center at some $x \in X$ so $\m_R \cap \stalk{X}{x} = \m_x$ but $a \in \stalk{X}{x}$ so if $a^{-1} \in \stalk{X}{x}$ then $a^{-1} \in \m_x$ which is impossible because $a^{-1} \in \stalk{X}{x}$ is a unit. Therefore, $a^{-1} \notin \stalk{X}{x}$. However, $a \in \Gamma(X, \struct{X})$ so $a \in \stalk{X}{x}$ but if $a \notin \m_x$ then $a$ would be a unit in $\stalk{X}{x}$ meaning $a^{-1} \in \stalk{X}{x}$ so we must have $a \in \m_x$ and thus $a \in \m_R$ contradicting $a^{-1} \in \m_R$. Therefore $\Gamma(X, \struct{X}) = k$.
\end{enumerate}

\subsubsection{4.6 (CHECK THIS)}

Let $f : X \to Y$ be a proper morphism of affine varieties $X = \Spec{B}$ and $Y = \Spec{A}$ over $k$. We can reduce to the case that $f : X \to Y$ is dominant i.e. $A \to B$ is an extension of $k$-algebra domains (HOW). By the valuative criterion for properness for any valuation ring $R$ and diagram,
\begin{center}
\begin{tikzcd}
\Spec{K} \arrow[d] \arrow[r] & X \arrow[d, "f"]
\\
\Spec{R} \arrow[ru, dashed] \arrow[r] & Y
\end{tikzcd}
\end{center}
there is a unique map $\Spec{R} \to X$ making the diagram commute. Let $K = \Frac{B}$ and $R \subset K$ be any valuating ring containing $A$ so there is a map $\Spec{R} \to \Spec{A}$ giving a diagram,
\begin{center}
\begin{tikzcd}
\Spec{K} \arrow[d] \arrow[r] & \Spec{B} \arrow[d, "f"]
\\
\Spec{R} \arrow[ru, dashed] \arrow[r] & \Spec{A}
\end{tikzcd}
\end{center}
corresponding to inclusions of rings,
\begin{center}
\begin{tikzcd}
B \arrow[r, hook] & K
\\
A \arrow[r, hook] \arrow[u, hook] & R \arrow[u, hook] \arrow[lu, hook, dashed]
\end{tikzcd}
\end{center}
Because the map $R \to K$ is injective we see that the dashed arrow $R \to B$ must also be injective so every valuation ring in $K$ containing $A$ also contains $B$. Thus $B$ is contained in the integral closure of $A$ in $K$ so the extension $A \subset B$ is integral. Furthermore, $A \to B$ is finite type since both are finite type $k$-algebras. Thus, by the lemma, $B$ is a finite $A$-module and thus $f : X \to Y$ is finite. It appears we only need that $X$ and $Y$ are integral affine schemes (i.e. $A, B$ are domains) and $f : X \to Y$ is proper since properness implies finite type.

\begin{lemma}
If $A \to B$ is finite type and integral then it is finite.
\end{lemma}

\begin{proof}
Tag 02JJ or Lang's Algebra, Proposition VII.1.2.
\end{proof}

\subsubsection{4.7}

\newcommand{\R}{\mathbb{R}}
\renewcommand{\C}{\mathbb{C}}

For any scheme $X_0$ over $\R$ let $X = X_0 \times_\R \Spec{\C}$ and $\alpha : \C \to \C$ be complex conjucation. Then $\sigma : X \to X$ is the automorphism $\id \times \alpha$ so $\sigma$ is semi-linear i.e. making the diagram,
\begin{center}
\begin{tikzcd}
X \arrow[d] \arrow[r, "\sigma"] & X \arrow[d]
\\
\Spec{\C} \arrow[r, "\alpha"] & \Spec{\C}
\end{tikzcd}
\end{center}
commute. Furthermore, $\sigma^2 = \id$ so $\sigma$ is an involution.

\begin{enumerate}
\item Let $X$ be a separated scheme of finite type over $\C$ and assume that $X$ is equiped with a semi-linear involution $\sigma : X \to X$. Furthermore, assume that for each pair of points $x_1, x_2 \in X$ there exists an affine open $U \subset X$ containing $x_1$ and $x_2$. Let $X_0 = \coeq{(\sigma, \id)}$. We need to show that $X_0$ exists and is the unique $\R$-scheme such that $X = X_0 \times_\R \Spec{\C}$ compatibly with $\sigma$.
\bigskip\\
Let $G = \Z/2\Z$ act on $X$ via $\sigma$. If $X = \Spec{A}$ is affine then take $X_0 = \Spec{A^G}$ which satisfies the universal property since $A^G = \ker{(\sigma - \id)}$. Otherwise, for any point $x \in X$ choose an affine open $U \subset X$ containing $x, \sigma(x)$. Since $\sigma^2 = \id$ we see that $\sigma(x), x \in \sigma(U)$. Therefore, $x, \sigma(x) \in V = \sigma(U) \cap U$ which is affine because $X$ is separated. Furthermore, $\sigma(V) = U \cap \sigma(U) = V$. Thus we can cover $X$ by $\sigma$-stable affine opens. The affine opens $V_i = \Spec{A_i}$ give gluing data for $X$ which descends to gluing data $V_i^G = \Spec{A^G_i}$ since $(-)^G$ is functorial which define $X_0$.
\bigskip\\
Define a map $X \to X_0 \times_\R \Spec{\C}$ via the canonical $\R$-maps $X \to X_0$ and $X \to \Spec{\C}$. This map is affine given locally on $X_0 \times_\R \Spec{\C}$ via $\varphi : A^G \otimes_\R \C \to A$. We can write any element as $a \otimes 1 + b \otimes i$ for $a,b \in A^G$. Suppose $\varphi(a \otimes 1 + b \otimes i) = a + i b = 0$ then $i b \in A^G$ so $i b = \sigma(ib) = -i \sigma(b) = -i b$ and thus $b = 0$ so $\ker{\varphi} = 0$. Furthermore, for any $a \in A$ we have $a + \sigma(a) \in A^G$ and $i (\sigma(a) - a)  \in A^G$ therefore,
\[ \tfrac{1}{2} (a + \sigma(a)) \otimes 1 + \tfrac{i}{2} (\sigma(a) - a) \otimes i \mapsto \tfrac{1}{2} (a + \sigma(a)) + \tfrac{1}{2}(a - \sigma(a)) = a \]
so $\varphi$ is surjective. Thus $X \to X_0 \times_\R \Spec{\C}$ is an isomorphism. Furthermore, $\alpha$ acts on $A^G \otimes_\R \C$ via $a \otimes 1 + b \otimes i \mapsto a \otimes 1 - b \otimes i$ which corresponds to $a + ib \mapsto a - bi$ and $\sigma(a + ib) = a - ib$ so $\id \times \alpha$ corresponds to $\sigma$. We can see this categorically,
\begin{center}
\begin{tikzcd}[column sep = small]
X \arrow[rd, "\sigma"] \arrow[dd] \arrow[rr] & & X_0 \times_\R \Spec{\C} \arrow[dd] \arrow[rd, "\id \times \alpha"] \arrow[rr] & & X_0 \arrow[dd] \arrow[rd, "\id"]
\\
& X \arrow[rr, crossing over] & & X_0 \times_\R \Spec{\C} \arrow[rr, crossing over] & & X_0 \arrow[dd] 
\\
\Spec{\C} \arrow[rd, "\alpha"] \arrow[rr, "\id"] & & \Spec{\C} \arrow[rd, "\alpha"] \arrow[rr] & & \Spec{\R} \arrow[rd, "\id"]
\\
& \Spec{\C} \arrow[from=uu, crossing over] \arrow[rr, "\id"] & & \Spec{\C} \arrow[from=uu, crossing over] \arrow[rr] & & \Spec{\R}
\end{tikzcd}
\end{center}
commutes because the map $X \to X_0$ is the coequalizer of $\id : X \to X$ and $\sigma : X \to X$ and thus the diagram,
\begin{center}
\begin{tikzcd}[row sep = large, column sep = large]
X \arrow[r, "\sigma"] \arrow[d] & X \arrow[d] 
\\
X_0 \arrow[r, "\id"] & X_0
\end{tikzcd}
\end{center}
commutes. Finally suppose that $X_0$ is any separated scheme of finite type over $\R$ equiped with an $\R$-morphism $\varphi : X \to X_0$ such that $\varphi \times \iota : X \to X_0 \times_\R \Spec{\C}$ is an isomorphim and the previous diagrams commute. However, via the earlier construction  we know that,
\[ X_0 = \coeq{(\id \times \alpha, \id : X_0 \times_\R : \Spec{\C} \to X_0 \times_\R \Spec{\C})} \]
Since the previous diagram commutes we see that $X_0 = \coeq{(\sigma, \id : X \to X)}$. Since this is a colimit it is unique up to unique isomorphism so there exists a unique such $X_0$.

\item By our construction, if $X = \Spec{A}$ is affine then $X_0 = \Spec{A^G}$ is affine as well and $X_0$ is unique so any such $X_0$ must be affine.

\item Let $X_0, Y_0$ be two such schemes over $\R$. Given a morphism $f_0 : X_0 \to Y_0$ over $\R$ we get a morphism $f : X \to Y$ with $f = f_0 \times \id$ satisfying,
\[ f \circ \sigma_X = (f_0 \times \id) \circ (\id \times \alpha) = f_0 \times \alpha = (\id \times \alpha) \times (f_0 \times \id) = \sigma_Y \circ f \]
Conversely, suppose that $f : X \to Y$ is a morphism satisfying $f \circ \sigma_X = \sigma_Y \circ f$. Projecting, we get a map $f' : X \to Y \to Y_0$ such that $f' \circ \sigma_X = f'$ and thus a map $f_0 : X_0 \to Y_0$ factoring $f'$ through $X \to X_0$ and thus we get a commutative diagram,
\begin{center}
\begin{tikzcd}[row sep = large, column sep = large]
X \arrow[d] \arrow[r, "f"] & Y \arrow[d] 
\\
X_0 \arrow[r, "f_0"] & Y_0 
\end{tikzcd}
\end{center}
Furthermore, since $f : X \to Y$ is a $\C$-morphism it commutes with $\id : \Spec{\C} \to \Spec{\C}$ so $f = f_0 \times \id$ by the universal property of the product.

\item Suppose that $X \cong \A^1_\C$. We need to classify semilinear involutions. If $\sigma : X \to X$ is a semilinear involution then $\bar{\sigma} = (\id \times \alpha) \circ \sigma$ is a $\C$-linear involution so we need to study automorphism of order $2$. Automorphisms of $\A^1_\C$ are given by $z \mapsto az + b$ for $a, b \in \C$. We need $z \mapsto az + b \mapsto a^2 z + ab + b = z$. Thus $b(a + 1) = 0$ and $a^2 = 1$ so either $b = 0$ and $a = \pm 1$ or $a = -1$ and $b = 1$. Thus either $\bar{\sigma} = \id$ or $\bar{\sigma}(z) = -z$ or $\bar{\sigma}(z) = 1 - z$.
\bigskip\\
However, I claim that each of these involutions is equivalent to $\id \times \alpha$. Indeed,

\begin{center}
\begin{tikzcd}[row sep = large, column sep = large]
\A^1_\C \arrow[d, "z \mapsto i z"'] \arrow[r, "z \mapsto -\bar{z}"] & \A^1_\C \arrow[d, "z \mapsto i z"]
\\
\A^1_\C \arrow[r, "z \mapsto \bar{z}"] & \A^1_\C
\end{tikzcd}
\qquad
\begin{tikzcd}[row sep = large, column sep = large]
\A^1_\C \arrow[d, "z \mapsto i(z - \frac{1}{2})"'] \arrow[r, "z \mapsto 1-\bar{z}"] & \A^1_\C \arrow[d, "z \mapsto i(z - \frac{1}{2})"]
\\
\A^1_\C \arrow[r, "z \mapsto \bar{z}"] & \A^1_\C
\end{tikzcd}
\end{center}
Therefore, up to $\C$-isomorphism there is only one semi-linear involution of $X$, namely $\sigma = \id \times \alpha$. Thus $X_0 \cong \A^1_\R$.

\item Suppose that $X \cong \P^1_\C$. We need to classify semilinear involutions. If $\sigma : X \to X$ is a semilinear involution then $\bar{\sigma} = (\id \times \alpha) \circ \sigma$ is a $\C$-linear involution so we need to study automorphism of order $2$. Automorphisms of $\P^1_\C$ are given by $\PGL{2}{\C} = \SL{2}{\C}/\{ \pm 1 \}$ matrices,
\[ A = \begin{pmatrix}
a & b 
\\
c & d 
\end{pmatrix} \]
with $ad - bc = 1$. Such a matrix has order $2$ if $a^2 + bc = bc + d^2$ and $ab + bd = 0$ and $c a + cd = 0$. So $a = \pm d$. If $a = d$ then $2 ba = 0$ and $2 ca = 0$ so either $a = d = 0$ or $b = c = 0$ giving,
\[ 
A_1 = \begin{pmatrix}
1 & 0
\\
0 & 1 
\end{pmatrix}
\quad \quad 
A_2 = \begin{pmatrix}
0 & -b
\\
b^{-1} & 0 
\end{pmatrix} \]
If $a = -d$ then we get,
\[ A_3 = \frac{1}{bc + 1}
\begin{pmatrix}
1 & b
\\
c & -1 
\end{pmatrix}
 \]
with $bc \neq -1$. These give,
\[ \bar{\sigma}(z) = z \quad \quad \bar{\sigma}(z) = - \frac{b^2}{z} \quad \quad \bar{\sigma}(z) = \frac{z + b}{cz - 1} \]
However, sending $z \mapsto b z$ transforms the second into $\bar{\sigma}(z) = - z^{-1}$ and the thrid can be diagonalized via the change of basis matrix,
\[ C =
\begin{pmatrix}
1 - \sqrt{1 + bc} & c
\\
1 + \sqrt{1 + bc} & c
\end{pmatrix}
\quad \quad C^{-1} A_3 C = 
\begin{pmatrix}
-1 & 0
\\
0 & 1
\end{pmatrix} \]
and thus $A_3$ is equivalent to the involution $\bar{\sigma}(z) = -z$. Therefore, we have classified the three $\C$-linear involutions of $X = \P^1_\C$,
\[ \bar{\sigma}_1(z) = z \quad \quad \bar{\sigma}_2(z) = - z^{-1} \quad \quad \bar{\sigma}_3(z) = - z\]
\bigskip\\
However, I claim that $\sigma_3$ is equivalent to $\sigma_1 = \id \times \alpha$. Indeed,

\begin{center}
\begin{tikzcd}[row sep = large, column sep = large]
\P^1_\C \arrow[d, "z \mapsto i z"'] \arrow[r, "z \mapsto -\bar{z}"] & \P^1_\C \arrow[d, "z \mapsto i z"]
\\
\P^1_\C \arrow[r, "z \mapsto \bar{z}"] & \P^1_\C
\end{tikzcd}
\end{center}
However, $\sigma_2$ and $\sigma_1$ are innequivalent up to $\C$-isomorphism and therefore there are two nonisomorphic $X_0$ such that $X \cong X_0 \times_\R \Spec{\C}$. Corresponding to $\sigma_1 = \id \times \alpha$ is $X_0 = \P^1_\R$. Corresponding to $\sigma_2$ is $X_0$ which we now investigate.
\bigskip\\
Consider $A = \C[z, z^{-1}]$ and $G = \Z / 2\Z$ acts on $A$ via $\sigma : A \to A$ defined by $z \mapsto - z^{-1}$ and complex conjuation on the coefficients. Then,
\[ A^G = \ker{(\sigma - \id)} = \R[\tfrac{1}{2}(z - z^{-1}), \tfrac{i}{2}(z + z^{-1})] \subset \C[z, z^{-1}] \]
Let $x = \tfrac{1}{2}(z - z^{-1})$ and $y = \tfrac{i}{2}(z + z^{-1})$ but these are not indpendent. In fact, the satisfy a single relation,
\[ x^2 + y^2 + 1 = \tfrac{1}{4} (z^2 - 2 + z^{-2}) - \tfrac{1}{4} (z^2 + 2 + z^{-2}) + 1 = 0 \]
Thus, $A^G \cong \R[x,y]/(x^2 + y^2 + 1)$. So an affine open of $X_0$ is isomorphic to $\Spec{\R[x,y]/(x^2 + y^2 + 1)}$. Continuing this argument we can see that,
\[ X_0 \cong \Proj{\R[x_0, x_1, x_2]/(x_0^2 + x_1^2 + x_2^2)} \]
Notice this has no $\R$-points but it does have $\C$-points and $X_0(\C) \cong \P^1_\C$ as analytic spaces and $X_0 \times_\R \Spec{\C} \cong \P^1_\C$ as $\C$-schemes.
\end{enumerate}

\renewcommand{\C}{\mathcal{C}}

\subsubsection{4.8 (DONE IN NOTES)}

(DONE, LINK TO NOTES)

\subsubsection{4.9 (DONE IN NOTES)}

(DONE, LINK TO NOTES)

\subsubsection{4.10 FINISH!!}

Let $X \to S$ be proper and $S$ noetherian. We want to show there is a projective $S$-scheme $X'$ and a morphism $g : X' \to X$ such that there is a dense open $U \subset X$ so that $g : g^{-1}(U) \to U$ is an isomorphism.

\begin{enumerate}
\item (DO!!)

\item Consider the proper map $f : X \to S$. Locally, let $V = \Spec{A} \subset S$ be an affine scheme and (DO!!)

\item Let $U = \bigcap U_i$ and consider the map,
\[ U \to X \times_S P_1 \times_S \cdots \times_S P_n \]
given by $U \to X$ and $U \to P_i$. Let $X'$ be the scheme theoretic image and $g : X' \to X$ the projection onto the first factor and $h : X' \to P$ projection onto the second factor where $P = P_1 \times_S \cdots \times_S P_n$. Because $X \to S$ is proper, the base change $X \times_S P \to P$ is also proper and therefore $h : X' \to P$ is closed. Furthermore, $U \to X \times_S P \to P$ is an immersion because each $U \to P_i$ is an open immersion then applying Lemma \ref{product_of_immersion}.  

\item Furthermore, there is a diagram,
\begin{center}
\begin{tikzcd}
U \arrow[r] \arrow[rrd] & X' \arrow[r, hook] & X \times_S P \arrow[d]
\\
& & X
\end{tikzcd}
\end{center}
Where $U \to X$ is an open immersion. Thus, if we restrict to $U \subset X$ the composition becomes an isomorphism and since scheme theoretic immage commutes with flat base change we see that $g^{-1}(U)$ is the scheme theoretic immage of $U \to U \times_S P$ which is isomorphic to $U$. 
\end{enumerate}

\subsubsection{4.11}

\renewcommand{\O}{\mathcal{O}}

\begin{enumerate}
\item Let $(\O, \m)$ be a noetherian local domain with quotient field $K$ and let $L / K$ be a finitely generated field extension. Choosing a transcendence basis $t_1, \dots, t_n \in L$ let $F = K(t_1, \dots, t_n)$ we get $L /F / K$ with $L / F$ finite. Then let $\O' = \O[t_1, \dots, t_n]_{(\m, t_1, \dots, t_n)}$ with $\m' = (\m, t_1, \dots, t_n)$ which is a noetherian (because it is a localization of a finite type algebra over the noetherian ring $\O$) local ring (this is local because $\O'/\m' = \O/\m$ which is a field so this is the unique maximal ideal). Furthermore, $\m' \cap \O = \m$ so $\O'$ dominates $\O$. Therefore, it suffices to find a DVR $R \subset L$ dominating $\O$ since then $R$ dominates $\O$. Since $F = \Frac{\O'}$ we reduce to the case where $L/K$ is finite. 
\bigskip\\
Now suppose that $K = \Frac{\O}$ and $L/K$ is finite. We can assume that $\m \neq 0$ since if $\O$ is a field then we can take $R = \O$. Let $\m = (x_1, \dots, x_n)$. I claim there exist some $j$ such that $x_j^{N-1} \notin \m^N$ for all $N > 0$. However, notice that if $x_j^{N-1} \in \m^N$ then $x_j^{N - 1 + k} \in \m^{N+k}$ for all $k \ge 0$ so if $j$ fails at stage $N$ then it fails every stage $N' \ge N$. Since there are finitely many $j$ it suffices to, for each $N > 0$, find a $j$ that works at stage $N$ because if every $j$ that works at stage $N$ later fails at some stage, then there can be no $j$ that work at sufficiently large $N$. Now suppose that at a fixed stage $N > 0$ for all $j$ we have $x_j^{N-1} \in \m^N$. Since every monomial in $\m^{nN-1}$ must contain some $x_j^{N-1}$ we see that $\m^{nN-1} = \m^N$ and thus by Nakayama's lemma $\m^{nN-1} = 0$ which implies that every elment in $\m$ is nilpotent so $\m = 0$ because $\O$ is a domain. Thus in the case that $\O$ is not a field (which we reduced to) we see that there must exist some $j$ such that $x_j^{N-1} \in \m^N$ for all $N > 0$. Set $\O' = \O[x_1/x_j, \dots, x_n/x_j]$ and $\a = (x_j) \subset \O'$. I claim that $\a$ is not the unit ideal. If $\a = \O'$ then $x_j^{-1} \in \O'$ which implies that,
\[ x_j^{-1} = \sum_{|I| \le N} a_I x^I/x_j^{|I|} \]
for some $N > 0$ and $a_I \in \O$. Therefore,
\[ x_j^{N-1} = \sum_{|I| \le N} a_I x^I x_j^{N - |I|} \in \m^N \]
contradicting the defining property of $x_j$. Thus we see that $\a$ is not the unit ideal so we may choose $\p$ a minimal prime of $\O'$ over $\a$. Then let $\O'_\p$ be the localization at $\p$. Since $\O'$ is noetherian it is clear that $\O'_\p \subset K$ is a noetherian domain.  Furthermore, because $\a$ is prinicpal in $\O'$ by Krull's height theorem the minimal primes over $\a$ are of height $1$ and therefore $\O'_\p$ has dimension $1$. Now let $\wt{\O'_\p}$ the integral closure of $\O'_\p$ inside $L$. By the theorem of Krull-Akizuki, $\wt{\O'_\p}$ is a noetherian dimension $1$ ring. Let $R$ be the localization of $\wt{\O'_\p}$ at a maximal prime which is then a noetherian dimension $1$ normal local domain and thus a DVR. Furthermore, because $x_i/x_j \in \O'$ we see that $x_1, \dots, x_n \in \a \subset \p$ and therefore $\m \subset \p \cap \O$ but $\m$ is maximal and $\p$ is prime so $\p \cap \O$ is prime and thus $\m = \p \cap \O$. Furthermore, therefore, $\O \to \O'_\p$ is local and for any maximal prime $\q \subset \wt{\O'_\p}$ we see that $\q \cap \O'_\p = \p$ because the extension $\O'_\p \to \wt{\O'_\p}$ is integral so it preserves maximal ideals and there is a unique maximal ideal of $\O'_\p$. Therefore, $\O'_\p \to R$ is local so we see that $R$ dominates $\O'_\p$ which dominates $(\O, \m)$ proving the claim. 

\item Let $f : X \to Y$ be a morphism of finite type of noetherian schemes. Clearly valuative criteria imply valuative criteria for discrete valuation rings. Conversely, suppose we know the valuative criteria for discrete valuation rings. Then the proof that $f$ is separated/proper only requires the existence of valuation rings dominating local rings of the form $\stalk{\overline{\{x\}}}{y}$ for $x \leadsto y$ on noetherian scheme so $\stalk{\overline{\{x\}}}{y}$ is noetherian inside the field $\kappa(z)$ for $z \mapsto x$ under a finite type morphism and thus $\kappa(z)$ is a finitely generated field extension of $\kappa(x) = \Frac{\stalk{\overline{\{x\}}}{y}}$. We can apply part (a) to produce \textit{discrete} valuation rings with the required property.
\end{enumerate}

\subsubsection{4.12 DO!!}

Let $k$ be algebraically closed.

\begin{lemma}
Valuation rings are integrally closed.
\end{lemma}

\begin{proof}
Let $R \subset K$ be a valuation ring and $a \in K$ be integral over $R$ write,
\[ a^n + c_{n-1} a^{n-1} + \cdots + c_0 = 0 \]
for $c_i \in R$. If $a \notin R$ then $a^{-1} \in R$ so we see that,
\[ a = -(c_{n-1} a^{-1} + \cdots + c_0 a^{-n+1}) \in R \]
so $a \in R$. Therefore $R$ is integrally closed in $K$.
\end{proof}

\begin{enumerate}
\item We first classify the valuating rings of $k[t]$ with $K = k(t)$. Let $R \subset K$ be a valuation ring. Because $k$ is algebraically closed every $f \in k[t]$ is a product of monomials and therefore each $q \in K(t)$ is a product of terms of the form $(t - a)^{\pm 1}$ for $a \in k$. Thus either $R = K$ or $R$ does not contain some $(t - a)^{-1}$ (or $(t - a)$ but the argument is identical in this case). Since $R$ is a valuation ring $(t - a) \in R$ and $k \subset R$ so we see that $t \in R$ and thus $k[t] \subset R$. Now $(R, \m)$ is local so $\m \cap k[t]$ is a prime ideal. However, $(t - a)^{-1} \notin R$ so $(t - a) \in \m$ and thus $(t - a) \in \m \cap k[t]$ is a prime ideal so $\p = \m \cap k[t] = (t - a)k[t]$. Therefore, for $f \in k[t] \setminus \p$ we see that $f \notin \m$ and thus $f^{-1} \in R$ so $R$ contains $k[t]_{\p}$ and furthermore dominates $k[t]_\p$ because $k[t] \cap R = \p$. However, $k[t]_\p$ is a DVR and in particular a valuation ring of $K$ which is maximal with respect to domination so $R = k[t]_\p$.  
\bigskip\\
Let $K$ be a function field of dimension $1$ over $k$. Then choose a transendental $t \in K$ then $K$ is finite over $F = k(t)$. Let $R \subset K$ be a valuation ring. Then $R \cap F \subset F$ is a valuation ring (for each $x \in F$ we have $x \in K$ so either $x \in R$ or $x^{-1} \in R$ and thus since $x^{-1} \in F$ either $x \in R \cap F$ or $x^{-1} \in R \cap F$). Therefore, $R \cap F$ is a DVR by the above calculation. Since $A = R \cap K$ is noetherian and integrally closed and $L/K$ is finite we know that the integral closure $\wt{A} \subset L$ of $A$ in $L$ is finite over $A$ but $R$ is integrally closed and contains $A$ so $\wt{A} \subset R$. Since $A \to \wt{A}$ is finite we see that $\wt{A}$ is noetherian of dimension $1$.  Let $\q = \wt{A} \cap \m$ be a prime. Then $\q \cap A = \m \cap A = \m_A$ which implies that $\q$ is maximal because $A \to \wt{A}$ is finite $\q \mapsto \m$ is maximal iff $\q$ is maximal. Thus $\wt{A}_\q$ is a noetherian integrally closed domain of dimension $1$ and thus a DVR. However, $\wt{A}_\q \subset R$ because $R$ is local and $\q = \m \cap R$ and thus $\wt{A}_\q \to R$ is a local map so $R$ dominates $\wt{A}_\q$ which is a valuation ring and thus $R = \wt{A}_\q$ menaing that $R$ is a DVR.

\item Let $K/k$ be a function field of dimension two. Suppose that $X$ is a complete nonsingular surface with function field $K$.
\begin{enumerate}
\item If $Y \subset X$ is an irreducible curve with generic point $x_1$ then $R = \stalk{X}{x_1} \subset K$ then $\dim{R} = \codim{Y,X} = 1$ and since $X$ is regular $\stalk{X}{x_1}$ is a regular local ring so $R$ is a DVR obviously with center $x_1$.

\item Let $f : X' \to X$ be a birational morphism and $Y' \subset X'$ an irreducible curve in $X'$ whose image in $X$ is a single closed point $x_0 \in X$. (DO I ASSUME THAT $X'$ IS REGULAR WHAT ABOUT BLOWING UP AT A REDUCED STRUCTURE ON A POINT?) Let $y_1 \in Y'$ be the generic pont. 

\item Let $x \in X$ be a closed point. Let $f : X_1 \to X$ be the blowing-up of $x_0$ and $E_1 = f^{-1}(x_0)$ be the exceptional curve. (WAIT WHUT??)
\end{enumerate}
\end{enumerate}


\subsection{5}

\subsubsection{5.1}

Let $(X, \struct{X})$ be a ringed space and $\E$ a locally free $\struct{X}$-module of finite rank. Define $\E^\vee = \shHom{\struct{X}}{\E}{\struct{X}}$. 

\begin{enumerate}
\item There is a natural map $\E \to (\E^\vee)^\vee$ defined by sending a section $s$ to $\varphi_s : \E^\vee \to \struct{X}$ defined via $\varphi_s(\psi) = \psi(s)$. Locally,  $\E|_U \cong \struct{U}^{\oplus n}$ so we need to check that the above map is an isomorphism for $\E = \struct{X}^{\oplus n}$. Indeed, $\shHom{\struct{X}}{\shHom{\struct{X}}{\struct{X}^{\oplus n}}{\struct{X}}}{\struct{X}} = \struct{X}$ because a map $\varphi : \struct{X}^{\oplus n} \to \struct{X}$ is determined by a section of $\struct{X}^{\oplus n}$ (where each basis is sent) via $\varphi \mapsto (\varphi(e_i))$. Under this identification the map in question becomes $(s_i e_i) \mapsto (\psi \mapsto \psi(s_i e_i)) \mapsto ((t_i e_i) \mapsto s_i t_i) = (s_i e_i)$ which is the identity. 

\item Let $\F$ be any $\struct{X}$-module. Consuder the map $\E^\vee \otimes_{\struct{X}} \F \to \shHom{\struct{X}}{\E}{\F}$ defined (on the tensor presheaf then sheafified) via $\varphi \otimes f \mapsto (s \mapsto \varphi(s) \cdot f)$. Locally $\E|_U \cong \struct{U}^{\oplus n}$ so we need to check this map is an isomorphism when $\E = \struct{X}^{\oplus n}$. However, $\shHom{\struct{X}}{\struct{X}^{\oplus n}}{\F} = \F^{\oplus n}$ via evaluating on the basis and $\E^\vee \otimes_{\struct{X}} \F = \F^{\oplus n}$ by multiplication. Under these identification the map becomes $(f_i) \mapsto e^i \otimes f_i \mapsto ((s_i e_i) \mapsto s_i f_i) \mapsto (f_i)$ which is the identity. 

\item Let $\F$ and $\G$ be $\struct{X}$-modules. Consider the map,
\[ \Hom{\struct{X}}{\F \otimes_{\struct{X}} \E}{\G} \to \Hom{\struct{X}}{\F}{\shHom{\struct{X}}{\E}{\F}} \]
via $F \mapsto (f \mapsto (s \mapsto F(f \otimes s)))$. Locally $\E|_U \cong \struct{U}^{\oplus n}$ so we need to check this for $\E = \struct{X}^{\oplus n}$. However,
\[ \Hom{\struct{X}}{\F \otimes_{\struct{X}} \struct{X}^{\oplus n}}{\G} = \Hom{\struct{X}}{\F^{\oplus n}}{\G} = \bigoplus_{i = 1}^n \Hom{\struct{X}}{\F}{\G} \]
Likewise,
\[ \Hom{\struct{X}}{\F}{\shHom{\struct{X}}{\struct{X}^{\oplus n}}{\G}} = \Hom{\struct{X}}{\F}{\G^{\oplus n}} = \bigoplus_{i = 1}^n \Hom{\struct{X}}{\F}{\G} \]

\item Consider a morphism $f : (X, \struct{X}) \to (Y, \struct{Y})$ of ringed spaces. Let $\E$ be a finite locally free $\struct{Y}$-module and $\F$ a $\struct{X}$-module. Consider the natural map $f^* f_* \F \to \F$ then tensoring with $f^* \E$ gives a natural map $f^* f_* \F \otimes_{\struct{X}} f^* \E = f^*(f_* \F \otimes_{\struct{Y}} \E) \to \F \otimes_{\struct{X}} f^* \E$. By adjunction, this is equivalent to giving a natural map $f_* \F \otimes_{\struct{Y}} \E \to f_*(\F \otimes_{\struct{X}} f^* \E)$. To show this is an isomorphism it suffices to check locally on $Y$ where $\E|_U \cong \struct{U}^{\oplus n}$ so we can reduce to the case that $\E = \struct{Y}^{\oplus n}$. Then,
\[ f_* \F \otimes_{\struct{Y}} \E = (f_* \F)^{\oplus n} = f_* (\F \otimes_{\struct{X}} f^* \E) \]
because $f^* \struct{Y} = \struct{X}$.
\end{enumerate}

\subsubsection{5.2}

Let $R$ be a DVR with $K = \Frac{R}$ and $X = \Spec{R}$. A $\struct{X}$-module $\F$ on $X$ has two pieces of data, an $R$-module $M = \F(X)$ and a $K$-module $L = \F(D(\varpi))$ where $\varpi$ is a uniformizer because these are the only two open sets. Furthermore, there is a restriction map $\rho : M \to L$ compatible with $R \to K$ which is equivalent to giving a map $\rho : M \otimes_R K \to L$ of $K$-modules.
\bigskip\\
Now $\F$ is quasi-coherent iff $\F = \wt{M}$ iff $\rho : M \to L$ is localization at $(0)$ i.e. $\rho : M_{(0)} = M \otimes_R K \to L$ is an isomorphism. 

\subsubsection{5.3}

Let $X = \Spec{A}$ be an affine scheme. Let $M$ be an $A$-module and $\F$ be a sheaf of $\struct{X}$-modules. Then there is a map,
\[ \Phi : \Hom{\struct{X}}{\wt{M}}{\F} \to \Hom{A}{M}{\Gamma(X, \F)} \]
defined by sending a sheaf map $f : \wt{M} \to \F$ to its value on global sections $f_X : M \to \Gamma(X, \F)$. Given a map $\varphi : M \to \Gamma(X, \F)$ we define a $\struct{X}$-module map $f_\varphi : \wt{M} \to \F$ on $D(s)$ via $f_{\varphi}(m/s) = s^{-1} \cdot \res(\varphi(s))$ since $\struct{X}(D(s)) = A_s$ acts on $\F(D(f))$. This construction $\Psi$ is clearly right inverse to $\Phi$. Furthermore, given a morphism of $\struct{X}$-modules $f : \wt{M} \to \F$ over $D(s)$ we see that $f_{D(s)}(m/s) = s^{-1} f_{D(s)}(m/1) = \res(f_X(m))$ and thus $\Psi$ is left inverse to $\Phi$ so $\Phi$ is an isomorphism. Clearly, the isomorphisms,
\[ \Hom{\struct{X}}{\wt{M}}{\F} = \Hom{A}{M}{\Gamma(X, \F)} \]
are natural so $\wt{(-)} : \Mod{A} \to \QCoh{X}$ is left adjoint to $\Gamma$ (in fact it is adjoint on the entire category of $\struct{X}$-modules).

\subsubsection{5.4}

Let $\F$ be a sheaf of $\struct{X}$-modules. If $\F$ is quasi-coherent then for each $x \in X$ there is an affine open $U = \Spec{A}$ containing $x$ such that $\F|_U \cong \wt{M}$ for some $A$-module $M$. Since $M$ is presented,
\[ \bigoplus_{j \in J} A \to \bigoplus_{i \in I} A \onto M \to 0 \]
Then applying the exact functor $\wt{-}$ we get a presenation,
\[ \bigoplus_{j \in J} \struct{U} \to \bigoplus_{i \in I} \struct{U} \onto \F|_U \to 0 \]
so $\F$ is locally presented. Conversely, if $\F$ is locally presented then for each $x \in X$ there is an open $U$, which by shrinking we may assume is affine $U = \Spec{A}$, such that,
\[ \bigoplus_{j \in J} \struct{U} \to \bigoplus_{i \in I} \struct{U} \onto \F|_U \to 0 \]
and therefore we can construct a module via,
\[ \bigoplus_{j \in J} A \to \bigoplus_{i \in I} A \onto M \to 0 \]
and therefore by the uniqueness of cokernels we see that $\F|_U \cong \wt{M}$ and thus $\F$ is quasi-coherent.
\bigskip\\
Furthermore, if $X$ is noetherian and $\F$ is coherent then we can assume that $M$ is finitely presented proving that $\F$ is locally finitely presented. Convereley, if $\F$ is locally finitely presented then the same argument gives $\F|_U \cong \wt{M}$ where $M$ is finitely presented.

\subsubsection{5.5}

\begin{enumerate}
\item Consider $f : \A^1_k \to \Spec{k}$ then $f_* \struct{\A^1_k} = k[x]$ as a $k$-module which is not finite and thus not coherent.
\item Let $\iota : Z \embed X$ be a closed immersion. Choose an affine open $U \subset X$ with $U = \Spec{A}$. Then $Z \cap U = V(I)$ where $\wt{I} = \ker{(\struct{X} \to \iota_* \struct{Z})}|_U$ which is a module because it is quasi-coherent and $U$ is affine. Thus $\iota$ is affine and locally given by $A \to A/I$ which is clearly finite so $\iota$ is finite.
\item Let $f : X \to Y$ be a finite morphism of noetherian schemes and $\F$ a coherent $\struct{X}$-module. Cover $Y$ be affines $V_i = \Spec{A_i}$ such that $f^{-1}(V_i) = U_i = \Spec{B_i}$. Then $\F|_{U_i} = \wt{M_i}$ and $(f_* \F)|_{V_i} = \wt{(M_i)_{A_i}}$ where $M_i$ is a finite $B_i$-module. However, the map $A_i \to B_i$ is finite so $(M_i)_{A_i}$ is a finite $A_i$-module and thus $f_* \F$ is coherent.
\end{enumerate}

\subsubsection{5.6}

\begin{enumerate}
\item Let $A$ be a ring and $M$ an $A$-module. Let $X = \Spec{A}$ and $\F = \wt{M}$. For any $m \in M = \Gamma(X, \F)$ consider,
\[ \Supp{}{m} = \{ \p \in X \mid \nexists s \in A \setminus \p : s m = 0 \} = \{ \p \in X \mid \p \supset \Ann{A}{m} \} = V(\Ann{A}{m}) \]
\item Suppose that $A$ is Noetherian and $M$ is finitely generated. Then,
\[ \Supp{\struct{X}}{\F} = \{ \p \in X \mid \F_\p \neq 0 \} = \Supp{A}{M} \]
If $s \in A \setminus \p \cap \Ann{A}{M}$ then $M_\p = 0$ since $s \cdot m = 0$ for all $m \in M$ so $\Supp{A}{\F} \subset V(\Ann{A}{M})$. Furthermore, if $M_\p = 0$ then for each generator $e_i$ there exits $s_i \in A \setminus \p$ such that $s_i \cdot e_i = 0$. Then $s = s_1 \cdots s_n \in \Ann{A}{M}$ so $\p \not\supset \Ann{A}{M}$ and thus,
\[ \Supp{A}{M} = V(\Ann{A}{M}) \]
\item Let $\F$ be a coherent sheaf on $X$ a Noetherian scheme. Then affine locally on $U = \Spec{A}$ we have $\F|_U = \wt{M}$ and $\Supp{\struct{X}}{\F} \cap U = V(\Ann{A}{M})$ is closed so $\Supp{\struct{X}}{\F}$ is closed.
\item Let $\a \subset A$ be an ideal and define the submodule of $M$,
\[ \Gamma_\a(M) = \{ m \in M \mid \a^n m = 0 \text{ for some } n > 0 \} \]
Assume that $A$ is noetherian and $M$ is $A$-module and let $Z = V(\a)$ and $\F = \wt{M}$. Since,
\begin{center}
\begin{tikzcd}
0 \arrow[r] & \H_Z^0(\F) \arrow[r] & \F \arrow[r] & j_* (\F|_U) 
\end{tikzcd}
\end{center}
and $\F$ is coherent we see that $\H^0_Z(\F)$ is quasi-coherent. Furthermore, 
\begin{align*}
\Gamma_Z(\F) & = \{ m \in M \mid m_\p \neq 0 \implies \p \in Z \} = \{ m \in M \mid \p \supset \Ann{A}{m} \implies \p \supset \a \} 
\\
& = \{ \m \in M \mid \sqrt{\Ann{A}{m}} \supset \a \} = \{ m \in M \mid \a^n m = 0 \text{ for some } n > 0 \} 
\\
& = \Gamma_\a(M)
\end{align*}
The last equality follows because $A$ is Noetherian: if $\sqrt{\Ann{A}{m}} \supset \a$ then for all $a \in \a$ we have $a^n \in \Ann{A}{m}$ for some $n > 0$ and thus $a^n \cdot m = 0$ meaning $\a^n \cdot m = 0$ because $\a$ is finitely generated. Likewise, if $\a^n \cdot m = 0$ for some $n > 0$ then $\a^n \subset \Ann{A}{m}$ so $\a \subset \sqrt{\Ann{A}{m}}$. Thus, since $\H^0_Z(\F)$ is quasi-coherent and has global sections $\Gamma_Z(\F)$ we see that $\H^0_Z(\F) = \wt{\Gamma_{\a}(M)}$.

\item Let $X$ be a noetherian scheme and $Z \subset X$ a closed subset. Let $\F$ be quasi-coherent (resp. coherent). From the exact sequence,
\begin{center}
\begin{tikzcd}
0 \arrow[r] & \H_Z^0(\F) \arrow[r] & \F \arrow[r] & j_* (\F|_U) 
\end{tikzcd}
\end{center}
it is immediate that $\H^0_Z(\F)$ is quasi-coherent as long as $j_* (\F|_U)$ is quasi-cohernet. However, since $X$ is Noetherian $j : U \to X$ is quasi-compact (i.e. $U$ is retrocompact) and separated so indeed $j_* (\F|_U)$ is quasi-coherent. If furthermore $\F$ is coherent then the quasi-coherent submodule $\H^0_Z(\F) \subset \F$ is coherent since $X$ is noetherian.

\end{enumerate}

\subsubsection{5.7}

Let $X$ be a Noetherian scheme and $\F$ a coherent sheaf on $X$.

\begin{enumerate}
\item Suppose that the stalk $\F_x$ is a free $\stalk{X}{x}$-module for some $x$. We may reduced to an affine open $X = \Spec{A}$ with $A$ noetherian (since $X$ is a Noetherian scheme) and $\F = \widetilde{M}$ for some finitely generated $A$-module $M$ since $\F$ is coherent. Suppose that $M_\p$ is a free $A_\p$-module for some prime $\p \subset A$. Let $e_1, \dots, e_r$ be an $A_\p$ basis of $M_\p$ which we may choose to be elements of $M$ since we may reintroduce denominators via multiplication by $A_\p$. Now consider the exact sequence,
\begin{center}
\begin{tikzcd}
0 \arrow[r] & \ker{E} \arrow[r] & A^{\oplus r} \arrow[r, "E"] & M \arrow[r] & \coker{E} \arrow[r] & 0 
\end{tikzcd}
\end{center}
where $E(a_1, \dots, a_r) = a_1 e_1 + \cdots + a_r e_r$. However, $\coker{E}$ is finitely genrerated because $M$ is and $\ker{E} \subset A^{\otimes r}$ is finitely generated because $A$ is Noetherian. Furthermore, we know that,
\begin{center}
\begin{tikzcd}
0 \arrow[r] & (\ker{E})_\p \arrow[r] & A_\p^{\oplus r} \arrow[r, "E"] & M_\p \arrow[r] & (\coker{E})_\p \arrow[r] & 0 
\end{tikzcd}
\end{center}
remains exact and $A_\p^{\oplus r} \xrightarrow{\sim} M_\p$ is an isomorphism so $(\ker{E})_\p = (\coker{E})_\p = 0$. Therefore, there exists some $f \notin \p$ such that $(\ker{E})_f = (\coker{E})_f = 0$ since they are finitely generated (take the products of elements in $A \setminus \p$ killing their generating sets). Now localizing the exact sequence, we get an exact sequence,
\begin{center}
\begin{tikzcd}
0 \arrow[r] & (\ker{E})_f \arrow[r] & A_f^{\oplus r} \arrow[r, "E"] & M_f \arrow[r] & (\coker{E})_f \arrow[r] & 0 
\end{tikzcd}
\end{center}
but $(\ker{E})_f = (\coker{E})_f = 0$ so $A_f^{\oplus r} \to M_f$ is an isomorphism. Therefore, 
\[ \F|_{D(f)} = \widetilde{M_f} = \widetilde{A_f^{\oplus n}} = \struct{X}|_{D(f)}^{\oplus n} \] 
is a free sheaf. 

\item Suppose that $\F_x$ is a free $\stalk{X}{x}$-module for each $x \in X$. Then, by above, there exsits an open cover of $X$ on which $\F$ is free so $\F$ is a locally-free sheaf. Conversely, if $\F$ is a locally-free sheaf. Then for each $x \in X$ there exists an open neighbrohood with $x \in U$ such that $\F|_U \cong \struct{X} |_U^{\oplus n}$. Then the induced map $\F_x \cong \stalk{X}{x}^{\oplus n}$ is an isomorphism so $\F_x$ is a free $\stalk{X}{x}$-module for each $x \in X$. 

\item A invertible sheaf is a locally free sheaf of rank $1$. First, suppose there exists a coherent sheaf $\G$ such that $\F \otimes_{\struct{X}} \G = \struct{X}$. Then for each $x \in X$ we have,
\[ \F_x \otimes_{\stalk{X}{x}} \G_x = \stalk{X}{x} \]
Since $\stalk{X}{x}$ is local, Lemma \ref{tensor_inverse} implies that $\F_x \cong \stalk{X}{x}$ and $\G_x \cong \stalk{X}{x}$ for each $x \in X$. Therefore, $\F$ is is locally free of rank $1$.
\bigskip\\
Conversely, suppose that $\F$ is an invertible $\struct{X}$-module. Consider the dual module,
\[ \F^* = \shHom{\struct{X}}{\F}{\struct{X}} \]
and then the evaluation map $\ev : \F \otimes_{\struct{X}} \F^* \to \struct{X}$ which is a morphism of $\struct{X}$-modules.
Consider the induced map on stalks $\F_x \otimes_{\stalk{X}{x}} \F^*_x \to \stalk{X}{x}$. Since $\F$ is invertible, $\F_x \cong \stalk{X}{x}$. By Hartshorne III 6.8 we know that,
\[ \F^*_x = \Homover{\stalk{X}{x}}{\F_x}{\stalk{X}{x}} \cong \Homover{\stalk{X}{x}}{\stalk{X}{x}}{\stalk{X}{x}} = \stalk{X}{x} \] and thus, $\ev_x(r \otimes (1 \mapsto r')) = rr'$ gives the natural map $\stalk{X}{x} \otimes_{\stalk{X}{x}} \stalk{X}{x} \to \stalk{X}{x}$ which is an isomorphism. Thus $\ev$ is an isomorphism since it is on the stalks. Therefore, 
\[ \F \otimes_{\struct{X}} \F^* = \struct{X} \]
\end{enumerate}


\subsubsection{5.8}

Let $X$ be a noetherian scheme and $\F$ a coherent sheaf. Recall that,
\[ \rank_{x}(\F) = \dim_{\kappa(x)} \F_x \otimes_{\stalk{X}{x}} \kappa(x) \]
where $\kappa(x) = \stalk{X}{x} / \m_x$. 

\begin{enumerate}
\item Choose an affine open $U = \Spec{A}$ such that $\F|_U = \wt{M}$ for some $A$-module $M$. For a point $x \in U$ suppose $\rank_{x}(\F) = n$ then via Nakayama we can lift a basis of $M_\p \otimes_{A_\p} A_\p / \p A_\p$ to a generating set of $M_\p$. Clearing denominators we get an exact sequence, 
\begin{center}
\begin{tikzcd}
0 \arrow[r] & \ker{f} \arrow[r] & A^n \arrow[r, "f"] & M \arrow[r] & \coker{f} \arrow[r] & 0
\end{tikzcd}
\end{center}
However, localizing at $\p$ we get a surjection $A_\p^n \onto M_\p$ and thus $(\coker{f})_\p = 0$. However, $\coker{f}$ is finitely generated so there exists some $g \in A$ such that $(\coker{f})_g = 0$ and thus we get an exact sequence,
\begin{center}
\begin{tikzcd}
0 \arrow[r] & (\ker{f})_g \arrow[r] & A_g^n \arrow[r] & M_g \arrow[r] & 0
\end{tikzcd}
\end{center}
Therefore $M_g$ is generated by $n$ elements so for each $\q \in D(f) \subset U$ there is a surjection $\kappa(\q)^n \onto M_\q \otimes_{A_\q} \kappa(\q)$ so $\rank_\q(\F) \le n$ on $D(f)$. Therefore, for each $x \in X$ there exists a neighborhood $x \in U \subset X$ such that $\forall y \in U : \rank_y(\F) \le \rank_x(\F)$ thus, in particular, $\{ x \in X \mid \rank_x(\F) < n \}$ is open for any $n \in \Z$ proving that rank is upper semi-continuous.

\item Suppose further that $\F$ is locally free. Then locally $\F|_U \cong \struct{U}^{\oplus n}$ showing that $\rank_x{\F} = n$ for all $x \in U$ i.e. $\rank{\F}$ is locally constant. Then for any $n \in \Z$ the set $S_n = \{ x \in X \mid \rank_x(\F) = n \}$ is clopen because if $x \in S_n$ then $x \in U \subset S_n$ and if $x \notin S_n$ then $x \in U \subset S_n^C$. Thus if $X$ is connected then $S_n = X$ (choose $x \in X$ and set $n = \rank_x(\F)$ so $x \in S_n$ is not empty) so $\rank{\F} = n$ is constant.

\item Now suppose that $X$ is reduced and $\rank(\F)$ is constant. Choose a point $x \in X$ such that $\rank_x(\F) = n$ and an affine open neighborhood $x \in U = \Spec{A}$. Then $\F|_U = \wt{M}$ for some finite $A$-module $M$. Let $\p \subset A$ correspond to $x \in U$. Proceeding as above we lift a basis of $M \otimes_A (A/\p)_\p$ to $A^n \onto M$ (localizing if necessary) to get an exact sequence,
\begin{center}
\begin{tikzcd}
0 \arrow[r] & \ker{f} \arrow[r] & A^n \arrow[r] & M \arrow[r] & 0
\end{tikzcd}
\end{center}
For any $\q \in U$, tensoring by $\kappa(\q)$ gives a surjection $\kappa(\q)^n \onto M \otimes_{A} \kappa(\q)$. However, by assumption, this vectorspace has dimension $n$ since $\F$ has constant rank so
\[ f \otimes \id : A^n \otimes_{A} \kappa(\q) \to M \otimes_{A} \kappa(\q) \]
is an isomorphims for each $\q \in U$. 
Now using the next following lemma, we conclude that $\ker{f} \subset \q \cdot A^n$ for each $\q \in \Spec{A}$ and thus $\ker{f} \subset \nilrad{A} \cdot A^n = 0$. Thus, $M = A^n$ so $\F$ is locally free.

\begin{lemma}
Let $f : M \to N$ be a map of $A$-modules with $N$ finite and $\p \subset A$ prime such that $f \otimes \id : M \otimes \kappa(\p) \to N \otimes \kappa(\p)$ is an isomorphism. Then $\ker{f} \subset \p \cdot M$ and $(\coker{f})_\p = 0$.
\end{lemma}

\begin{proof}
If $m \in \ker{f}$ then $\bar{m} \in M/\p M$ is sent to zero under $f \otimes \id$ and thus $\bar{m} = 0$ so $m \in \p M$. Furthermore, by left-exactness $(\coker{f})_\p \otimes_{A_\p} \kappa(\p) = 0$ but $A_\p$ is local and $\coker{f}$ is finitely generated so $(\coker{f})_\p = 0$. 
\end{proof}
\end{enumerate}

\subsubsection{5.9}

\newcommand{\sS}{\mathscr{S}}

Let $S$ be a graded ring, generated by $S_1$ as an $S_0$-algebra, let $M$ be a graded $S$-module, and let $X = \Proj{S}$. 

\begin{enumerate}
\item Consider the canonical graded map $\alpha : M \to \Gamma_*(\wt{M})$ sending $m \in M_d$ to the global section of $\wt{M(d)}$ defined by $m$ on $\wt{M(d)}(D_+(f))$ which makes sense because $m$ has degree zero in $M(d)$. This is clearly natural because $\varphi : M \to N$ acts sectionwise on $\wt{M} \to \wt{N}$.

\item Assume that $S_0 = A$ is a finitely generated $k$-algebra for some field $k$, that $S_1$ is finite $A$-module, and that $M$ is a finitely generated $S$-module. Because $M$ is noetherian, we can find a finite filtration,
\[ 0 = M^0 \subset M^1 \subset \cdots M^r = M \]
by graded submodules, where for each $i$, we have $M^i/M^{i-1} \cong (S/\p_i)(n_i)$ for some homogeneous prime ideal $\p_i \subset S$ and some integer $n_i$. By naturality of $\alpha$ there is a commutative diagram,
\begin{center}
\begin{tikzcd}
0 \arrow[r] & M^{i-1} \arrow[d] \arrow[r] & M^i \arrow[r] \arrow[d] & (S/\p_i)(n_i) \arrow[r] \arrow[d] & 0
\\
0 \arrow[r] & \Gamma_*(\wt{M^{i-1}}) \arrow[r] & \Gamma_*(\wt{M^i}) \arrow[r] & \Gamma_*(\wt{S/\p_i})(n_i)
\end{tikzcd}
\end{center}
By [III Ex. 5.10] (I think this is shown in a special case earlier), the bottom sequence is actually exact in sufficiently large degrees.
Therefore, if the outside downward arrows are isomorphisms in sufficiently large degrees then the middle arrow must also be an isomorphism in degree large enough so that the bottom sequence is exact and the outer maps are isomorphisms. To then conclude by induction, it suffices to show that $S/\p_i \to \Gamma_*(S/\p_i)$ is an isomorphism in sufficiently large degree (then the same will be true for the diagram with thus degree plus $n_i$). By replacing $S$ by $S / \p_i$ (and noticing that $\Gamma(X, \wt{S/\p_i}) = \Gamma(\Proj{S/\p_i}, \wt{S/\p_i})$ via pushforward) we reduce to showing that $S \to \Gamma_*(\wt{S})$ is an isomorphism in sufficiently large degrees where $S$ is a graded integral domain. This is actually showin in [Prop. 5.19]. An alternative argument goes as follows. Since $S$ is an integral domain, $S_d \mapsto (S(d))_{(f)}$ is an injection because localization is injective so $\alpha$ is injective. Because $\F = \wt{S}$ is quasi-coherent, $\beta : \wt{\Gamma_*(\F)} \to \F$ is an isomorphism by [Prop. 5.15] meaning that the exact sequence,
\begin{center}
\begin{tikzcd}
0 \arrow[r] & \wt{S} \arrow[r, equals] & \wt{S'} \arrow[r] & \wt{S'/S} \arrow[r] & 0
\end{tikzcd}
\end{center}
proves that $\wt{S'/S} = 0$. Then from the diagram,
\begin{center}
\begin{tikzcd}
0 \arrow[r] & S \arrow[d] \arrow[r] & S' \arrow[d, equals] \arrow[r] & S'/S \arrow[d] \arrow[r] & 0
\\
0 \arrow[r] & \Gamma_*(\wt{S}) \arrow[r, equals] & \Gamma_*(\wt{S'}) \arrow[r] & 0 \arrow[r] & 0
\end{tikzcd}
\end{center}
where $\alpha : S' \to \Gamma_*(\wt{S'})$ is an isomorphism because $\wt{S'} = \wt{S}$ and by definition $S' = \Gamma_*(\wt{S})$. Therefore, to show that $S \to \Gamma_*(\tilde{S})$ is an isomorphism in sufficiently large degree, it suffices to show that $S'/S \to 0$ is an isomorphism in sufficiently large degrees. By [Prop. 5.19] we know that $S'$ is a finitely generated $S$-module. Therefore we reduce to showing that if $M$ is a finitely generated graded $S$-module with $\wt{M} = 0$ then $M$ is nonzero in only finitely many degrees. Using finiteness as before, we choose a graded filtration $M^i$ with quotients $M^{i} / M^{i-1} \iso (S / \p_i)(n_i)$. Then the exact sequences,
\begin{center}
\begin{tikzcd}
0 \arrow[r] & \wt{M^{i-1}} \arrow[r] & \wt{M^i} \arrow[r] & \wt{(S/\p_i)}(n_i) \arrow[r] & 0
\end{tikzcd}
\end{center}
show inducively that $\wt{M^i} = 0$ and $\wt{S/\p_i} = 0$ meaning that $\p_i \supset S_+$ so $(S/\p_i)(n_i)$ only has elements in degree $n_i$. Therefore, using the exact sequences,
\begin{center}
\begin{tikzcd}
0 \arrow[r] & M^{i-1} \arrow[r] & M^i \arrow[r] & (S/\p_i)(n_i) \arrow[r] & 0
\end{tikzcd}
\end{center} 
we see that $M^{i}$ only has elements in finitely many degrees if $M^{i-1}$ only has elements in finitely many degrees.
Since the base case $M^0 = 0$ is trivial in all degrees we conclude that $M = M^r$ has elements in only finitely many degrees by induction.

\item Define an equivalence relation on graded $S$-modules by $M \approx M'$ if for sufficiently large degrees they are isomorphic. We say that $M$ is \textit{quasi-finitely generated} if it is equivalent to a finitely generated module. We have showed that for any coherent $\struct{X}$-module we have $\beta : \wt{\Gamma_*(\F)} \iso \F$ and for any quasi-finitely generated module (we can replace it by a finitely generated one up to equivalence) $M$ that $\alpha : M \to \Gamma_*(\wt{M})$ is an isomrophism in sufficiently large degree and thus $M \approx \Gamma_*(\wt{M})$. Therefore, it suffices to show that if $\F$ is coherent then $\Gamma_*(\F)$ is quasi-finitely generated and that if $M$ is quasi-finitely generated then $\wt{M}$ is coherent and $\wt{-}$ is well-defined. 
\bigskip\\
First, if $M \approx M'$ then for any $f \in S_1$ since $\frac{m}{f^n} = \frac{m f^d}{f^{n+d}}$ where $M_m \cong M'_m$ for $m \ge d$ then we see that $M_{(f)} \cong M'_{(f)}$. Therefore $\wt{-}$ is well-defined modulo the equivalence relation $\approx$. Thus, if $M$ is quasi-finitely generated, we can replace $M$ by a finitely generated module $M'$ and thus $\F = \wt{M} \cong \wt{M'}$ is coherent because $\F|_{D_+(f)} = M'_{(f)}$ for $f \in S_1$ is finitely generated by the generators of $M'$ divided by an appropriate power of $f$. Now, if $\F$ is coherent, then by [Thm. 5.17] we see that $\F(n)$ is generated by a finite number of global sections for sufficiently large $n$. Taking $M' \subset \Gamma_*(\F)$ generated by those sections (which appear in degree $n$) then $\wt{M'}(n) \onto \F(n)$ is a surjection but $\wt{M'} \subset \wt{\Gamma_*(\F)} = \F$ so $\wt{M'} \iso \F$ is an isomorphism. Therefore, $\alpha : M' \to  \Gamma_*(\wt{M'}) \to \Gamma_*(\F)$ is an isomorphism in sufficiently large degree and thus $\Gamma_*(\F) \approx M'$ and $M'$ is finitely generated by construction. 
\end{enumerate}

\subsubsection{5.10}

Let $A$ be a ring and $S = A[x_0, \dots, x_r]$ and $X = \Proj{S}$. 

\begin{enumerate}
\item For any homogeneous ideal $I \subset S$, we define the \textit{saturation},
\[ \overline{I} = \{ s \in S \mid \forall i : \exists n : x^n_i s \in I \} \]
and say that $I$ is \textit{saturated} if $\overline{I} = I$. Suppose that $s \in \overline{I}$. Write $s = s_0 + \cdots + s_d$ as a sum of homogeneous elements. Then $x_i^n(s_0 + \cdots + s_d) \in I$ but each term is still homogeneous and $I$ is a homogeneous ideal so $x_i^n s_k \in I$ and thus $s_k \in \overline{I}$ for each $k$. Thus $\overline{I}$ is homogeneous.

\item Notice that $\overline{I}_{(x_i)} = I_{(x_i)}$ because for any element $s/x_i^m \in \overline{I}_{(x_i)}$ we have $s/x_i^m = (x^n_i s)/x_i^{n+m}$ but $x^n_i s \in I$. Therefore, $\wt{\overline{I}} = \wt{I}$ because the natural map $\wt{I} \to \wt{\overline{I}}$ is an isomorphism on the open cover $D_{+}(x_i)$. In particular, $I$ and $\overline{I}$ define the same closed subscheme since this subscheme is determined by $\wt{I}$ so clearly if $I_1$ and $I_2$ have the same saturation they define the same closed subscheme. Conversely, suppose that $I_1, I_2 \subset S$ define the same closed subscheme $Y$. Then we know that $\I_Y \cong \wt{I_1} \cong \wt{I_2}$. In particular, we must have $(I_1)_{(x_i)} = (I_2)_{(x_i)}$ which implies that the saturations of $I_1$ and $I_2$ are equal.

\item Let $Y \subset X$ be a closed subscheme and define $I = \Gamma_*(\I_Y)$. Take $s \in S$ and suppose that $x_i^n s \in I$ for each $i$. Splitting up $s$ we may assume it is homogeneous of degree $d$ i.e. $s \in S_d$. Then $x_i^n s$ is a global section of $\I_Y \otimes_{\struct{X}} \struct{X}(d+n)$ 
However, $x_i^n$ globally generate $\struct{X}(n)$ so the induced map,
\[ \struct{X} \to \bigoplus_{i = 0}^r \struct{X}(n) \]
is injective. Therefore there is a morphism of exact sequences,
\begin{center}
\begin{tikzcd}
0 \arrow[r] & \I_Y(d) \arrow[d, "x_i^n", hook] \arrow[r] & \struct{X}(d) \arrow[r] \arrow[d, "x_i^n", hook] & \struct{Y}(d) \arrow[d, "x_i^n", hook] \arrow[r] & 0
\\
0 \arrow[r] & \bigoplus\limits_{i = 0}^r \I_Y(d+n) \arrow[r] & \bigoplus\limits_{i = 0}^r \struct{X}(d+n) \arrow[r] & \bigoplus\limits_{i = 0}^r \struct{Y}(d+n) \arrow[r] & 0
\end{tikzcd}
\end{center}
where the rightmost downward arrow is surjective because $x_i^n$ globally generate $\struct{Y}(n)$ as a $\struct{Y}$-module.  Thus if $x^n_i s$ is in the kernel for each $i$ then $s \in \Gamma(X, \I_Y(d))$ so $s \in I$ and thus $I$ is saturated.

\item There is a correspondence,
\[ \{ \text{closed subschemes } Y \subset X \} \leftrightarrow \{ \text{saturated ideals } I \subset S \} \]
given by $Y \mapsto \Gamma_*(\I_Y)$ which we showed is saturated and $I \mapsto \Proj{S/I}$. Furthermore, if $I_1$ and $I_2$ define the same subscheme $Y$ then their saturations are equal. In particular if $I_1$ and $I_2$ are saturated then $I_1 = I_2$ so this mapping is injective. Furthermore, we know that every closed subscheme arises from some ideal and thus from its saturation so $I \mapsto \Proj{S/I}$ is a bijection. Furthermore, $Y \mapsto \Gamma_*(\I_Y)$ is its inverse since $\Gamma_*(\wt{I})$ is the saturation of $I$ and $\Gamma_*(\I_Y)$ defines $Y$ as proven before (CHECK THIS).
\end{enumerate}


\subsubsection{5.11 DO!! ASK DANIEL ABOUT THIS}

Let $S$ and $T$ be graded rings with $S_0 = T_0 = A$ and define $S \times_A T$ as the graded ring,
\[ (S \times_A T)_n = S_n \otimes_A T_n \]
Let $X = \Proj{S}$ and $Y = \Proj{T}$. I need to show that $\Proj{S \times_A T} \cong X \times_A Y$ and that $\struct{}(1)$ on $\Proj{S \times_A T}$ is isomorphic to $p_1^*(\struct{X}(1)) \otimes p_2^*(\struct{Y}(1))$. 
\bigskip\\
(TELL DANIEL HIS SOLUTION IS WRONG)

(MAYBE USE UNIVERSAL NONSENSE!?)

\subsubsection{5.12}

\begin{enumerate}
\item Let $X$ be a scheme over $Y$ and let $\L$ and $\M$ be two very ample invertible sheaves on $X$. These determine immersions $\iota_1 : X \embed \P^n_Y$ and $\iota_2 : X \embed \P^m_Y$. Consider the diagram,
\begin{center}
\begin{tikzcd}
& \P^n_Y 
\\
X \arrow[r, dashed] \arrow[ru, hook] \arrow[rd, hook] & \P^n_Y \times_Y \P^m_Y \arrow[u, "\pi_1"] \arrow[d, "\pi_2"] \arrow[r, hook] & \P^{N}_Y
\\
& \P^m_Y
\end{tikzcd}
\end{center}
where $s : \P^n_Y \times_Y \P^m_Y \embed \P^{N}_Y$ is the Segre embedding which satisfies,
\[ s^* \struct{\P^{N}_Y}(1) = \pi_1^* \struct{\P^n_Y}(1) \otimes_{\struct{}} \pi_2^* \struct{\P^m_Y}(1) \]
Now, consider,
\begin{align*}
(\iota_1, \iota_2)^* s^* \struct{\P^N_Y}(1) & = (\iota_1, \iota_2)^* [\pi_1^* \struct{\P^n_A}(1) \otimes_{\struct{}} \pi_2^* \struct{\P^m_A}(1)] 
\\
& = [(\iota_1, \iota_2)^* p_1^* \struct{\P^n_A}(1)] \otimes_{\struct{}} [(\iota_1, \iota_2)^* p_2^* \struct{\P^m_A}(1)] 
\\
& = [\pi_1 \circ (\iota_1, \iota_2)]^* \struct{\P^n_A}(1) \otimes_{\struct{}} [\pi_2 \circ (\iota_1, \iota_2)]^* \struct{\P^m_A}(1)
\\
& = \iota_1^* \struct{\P^n_A}(1) \otimes_{\struct{X}} \iota_2^* \struct{\P^m_A}(1)
\\
& = \L \otimes_{\struct{}} \M 
\end{align*}
Now we need to show that $(\iota_1, \iota_2) \circ \Delta$ is an immersion which follows because we can factor it as $\Gamma_{\iota_2} : X \to X \times_Y \P^m_Y$ then $\iota_1 \times \id : X \times \P^m_Y = \P^n_Y \times_Y \P^n_Y$. However, $\iota_1 \times \id$ is an immersion by base change and $\Gamma_{\iota_2}$ is a closed embedding because $\P^m_k$ is separated.

\item Let $f : X \to Y$ and $g : Y \to Z$ be two morphisms of schemes. Let $\L$ be a very ample invertible sheaf on $X$ relative $Y$, and let $\M$ be a very ample invertible sheaf on $Y$ relative to $Z$. These define closed immersions $\iota_1 : X \embed \P^n_Y$ and $\iota_2 : Y \embed \P^m_Z$.
Consider the following diagram,
\begin{center}
\begin{tikzcd}
& \P^n_Y \arrow[r, "\id \times g"] & \P^n_Z
\\
X \arrow[ru, "\iota_1", hook] \arrow[rd, "f"'] \arrow[rr, dashed, "q"] & & \P^n_Z \times_Z \P^m_Z \arrow[u, "\pi_1"'] \arrow[d, "\pi_2"] \arrow[r, hook] & \P^N_Z
\\
& Y \arrow[r, "\iota_2", hook] & \P^m_Z 
\end{tikzcd}
\end{center}
where $s : \P^n_Z \times_Z \P^m_Z \embed \P^{N}_Z$ is the Segre embedding which satisfies,
\[ s^* \struct{\P^{N}_Z}(1) = \pi_1^* \struct{\P^n_Z}(1) \otimes_{\struct{}} \pi_2^* \struct{\P^m_Z}(1) \]
Therefore,
\begin{align*}
(s \circ q)^* \struct{\P^N_Z}(1) & = q^* s^* \struct{\P^N_Z}(1) = q^* [\pi_1^* \struct{\P^n_Z}(1) \otimes_{\struct{}} \pi_2^* \struct{\P^m_Z}(1)] 
\\
& = (\pi_1 \circ q)^* \struct{\P^n_Z}(1) \otimes_{\struct{}} (\pi_2 \circ q)^* \struct{\P^m_Z}(1)
\\
& = ((\id \times g) \circ \iota_1)^* \struct{\P^n_Z}(1) \otimes_{\struct{}} (\iota_2 \circ f)^* \struct{\P^m_Z}(1)
\\
& = \iota_1^* (\id \times g)^* \struct{\P^n_Z}(1) \otimes_{\struct{}} f^* \iota_2^* \struct{\P^m_Z}(1)
\\
& = \iota_1^* \struct{\P^n_Y}(1) \otimes_{\struct{}} f^* \M 
\\
& = \L \otimes_{\struct{}} f^* \M
\end{align*}
Thus it suffices to show that $q$ is an immersion. We can factor the morphism as,
\[ X \xrightarrow{\iota_1} \P^n_Y \xrightarrow{\iota_2'} \P_Z^n \times_Z \P^m_Z \] where the second map is the base change,
\begin{center}
\begin{tikzcd}
\P_Y^n \arrow[d] \arrow[r, "\iota_2'"] & \P_Z^n \times_Z \P^m_Z \arrow[d, "\pi_2"]
\\
Y \arrow[r, "\iota_2"] & \P_Z^m
\end{tikzcd}
\end{center}
then $\iota_2' = \id \times \iota_2$ and thus the diagram becomes,
\begin{center}
\begin{tikzcd}
& \P^n_Y \arrow[r, "g'"] \arrow[rd, "\iota_2'"', hook] & \P^n_Z
\\
X \arrow[ru, "\iota_1", hook] \arrow[rd, "f"'] \arrow[rr, dashed, "q"] & & \P^n_Z \times_Z \P^m_Z \arrow[u, "\pi_1"'] \arrow[d, "\pi_2"] \arrow[r, hook] & \P^N_Z
\\
& Y \arrow[r, "\iota_2", hook] & \P^m_Z 
\end{tikzcd}
\end{center}
where the top triangle is the base change via $\P^n_Z \to Z$ of
\begin{center}
\begin{tikzcd}
Y \arrow[rd, "g"'] \arrow[r, "\iota_2", hook] & \P^m_Z \arrow[d]
\\
& Z
\end{tikzcd}
\end{center}
Since $\iota_1$ is an immersion and $\iota_2'$ is the base change of a closed immersion and thus a closed immersion as well we see that $q = \iota_2' \circ \iota_1$ is an immersion.

\end{enumerate}


\subsubsection{5.13}

Let $S$ be a graded ring, generated by $S_1$ as an $S_0$-algebra. For any integer $d > 0$ define $S^{(d)}$ to be the graded ring,
\[ S^{(d)} = \bigoplus_{n \ge 0} S^{(d)}_{n} \]
where $S^{(d)}_n = S_{nd}$. 
\bigskip\\
Consider the graded ring map $\psi : S \to S^{(d)}$ via sending $f \mapsto f^d$ and the (multipying grading) ring map $\iota : S^{(d)} \embed S$. Notice that if $\p \subset S$ is a homogeneous prime not containing $S_+$ then $\iota^{-1}(\p) \subset S^{(d)}$ is a homogeneous prime and if $S^{(d)}_+ \subset \iota^{-1}(\p)$ then for each $f \in S_+$ we would have $f^d \in \p$ and thus $f \in \p$ proving that $\p \supset S_+$. Therefore, $\iota$ defines a map $\Proj{S} \to \Proj{S^{(d)}}$ which becomes a map of schemes by seeing that locally on $D_+(f)$ for $f \in S^{(d)}_1$ it comes from a map,
\begin{center}
\begin{tikzcd}
\Proj{S} \arrow[r, "\iota"] & \Proj{S^{(d)}}
\\
\Spec{S_{(f)}} \arrow[u, hook] \arrow[r] & \Spec{S^{(d)}_{(f)}} \arrow[u, hook]
\end{tikzcd}
\end{center}
with $S^{(d)}_{(f)} \to S_{(f)}$ the inclusion which is actually an isomorphism because $\deg{f} = d$ (in $S$) so by degree reasons any fraction in $S_{(f)}$ has numerator of degree $n d$. Furthermore, since $S^{(d)}$ is generated in degree $1$ (because $S$ is generated in degree $1$ we see that $S_{nd}$ is generated by $S_d$) we see that $D_+(f)$ cover $\Proj{S^{(d)}}$ since a prime with $f \in \p$ for all $f$ would contain the irrelevant ideal. Furthermore, the $D_+(f) \subset \Proj{S}$ also cover $\Proj{S}$ because if for any $g \in S_1$ we have $g^d \in S^{(d)}_1$ and if $g^d \in \p$ then $g \in d$ so the only primes of $S$ containing $S^{(d)}_1$ contain the irrelevant ideal. Therefore, this map is locally on the target an isomorphism and therefore an isomorphism. 
\bigskip\\
Furthermore, under this map $S^{(d)}(1)$ is pulled back to $S^{(d)}(1) \otimes_{S^{(d)}} S = S(d)$ showing that this isomorphism identifies $\struct{\Proj{S^{(d)}}}(1)$ with $\struct{\Proj{S}}(1)$. 


\subsubsection{5.14 CHECK!!}

Let $A$ be a ring and $X \subset \P^r_A$ a closed subscheme. Define the homogeneous coordinate ring,
\[ S(X) = A[x_0, \dots, x_r]/I(X) \]
where $I(X) = \Gamma_*(\I_X)$. We say that $X$ is projectively normal (for the given embedding), if $S(X)$ is an integrally clsoed domain.
\bigskip\\
Assume that $X$ is a connected normal closed subscheme of $\P^r_A$ where $A$ is a finitely generated $k$-algebra (DO WE NEED $A = k$ and $k = \bar{k}$)

\begin{enumerate}
\item Let $S = S(X)$ and $S' = \Gamma_*(X, \struct{X})$. I claim that if $X$ is normal and connected then it is integral. Indeed, the local rings of $X$ are domains and $X$ is connected so $X$ is integral. Furthermore, since $X$ is normal its sheaf of rings is a sheaf of integrally closed domains (see the lemmas at the end which we will use frequently). Now consider the sheaf,
\[ \sS = \bigoplus_{n \ge 0} \struct{X}(n) \]
Since $\struct{X}(n)$ is locally trivial, we see that locally $\sS|_U \cong \struct{U}[t]$ is the polynomial ring over $\struct{U}$ and therefore $\sS$ is also a sheaf of integrally closed domains. Because $X$ is quasi-compact (this is necessary for direct sums to commute with global sections otherwise sheafification is needed),
\[ \Gamma(X, \sS) = \bigoplus_{n \ge 0} \Gamma(X, \struct{X}(n)) = S' \]
and therefore $S'$ is an integrally closed domain. 
\bigskip\\
Now we need to determine the relationship between $S$ and $S'$. There is an exact sequence of sheaves on $\P^r_A$,
\begin{center}
\begin{tikzcd}
0 \arrow[r] & \bigoplus\limits_{n \ge 0} \I_X(n) \arrow[r] & \bigoplus\limits_{n \ge 0} \struct{\P}(n) \arrow[r] & \bigoplus\limits_{n \ge 0} \iota_* \struct{X}(n) \arrow[r] & 0
\end{tikzcd}
\end{center}
giving an exact sequence on global sections,
\begin{center}
\begin{tikzcd}
0 \arrow[r] & \Gamma_*(\P^r_A, \I_X) \arrow[r] & \Gamma_*(\P^r_A, \struct{\P}) \arrow[r] & S' 
\end{tikzcd}
\end{center}
giving an injective map $S \embed S'$ because $S = \Gamma_*(\P^n_A, \struct{\P}) / \Gamma_*(\P^n_A, \I_X)$ proving that $S$ is a domain. Furthermore, let $\xi \in X$ be the generic point. Then,
\[S' \embed \sS_\xi = \bigoplus_{n \ge 0} (\struct{X}(n))_\xi = S_{(0)} = \Frac{S} \]
because $(\struct{X}(n))_\xi$ is the homogeneous of degree $n$ part of the localization of $S_{(0)}$. 
The natural homomorphism $S \to S' = \Gamma_*(\wt{S})$ is an isomorphism in sufficiently large degree by [Ex. 5.9] since $S$ is generated by $S_1$ as an $S_0$-algebra. This means that $S'$ is integral over $S$.
Therefore, $S'$ is the integral closure of $S$ in $\Frac{S}$ because it is an integrally closed extension on $K$.

\item We did this already in (a).

\item By the previous part, for sufficiently large $d$, we see that $S^{(d)} = S'^{(d)}$ and therefore it suffices to show that if $S'$ is integrally closed then $S'^{(d)}$ is also integrally closed because in that case $S^{(d)}$ will be integrall closed proving that the $d$-uple embedding is projectively normal. Let $A$ be a graded domain generated by $A_1$ as an $A_0$-algebra which is integrally closed. Consider $A^{(d)}$. Then $\Frac{A^{(d)}} \subset \Frac{A}$ and I claim that $A \cap \Frac{A^{(d)}} = A^{(d)}$. If,
\[ \frac{a_0 + a_d + a_{2d} + \cdots + a_{rd}}{b_0 + b_d + b_{2d} + \cdots + b_{rd}} = c_0 + c_1 + \cdots + c_k  \iff a_0 + \cdots + a_{rd} = (b_0 + \cdots + b_{rd}) (c_0 + \cdots + c_k) \]
Choose some $b_{md} \neq 0$ then, the terms of degree $md + jd + i$ for $0 < i < d$ are exactly,
\[ b_{md} (c_{jd + 1}  + \cdots + c_{(j+1)d - 1}) = 0 \]
and thus because $A$ is a domain and the direct sum grading structure $c_{jd + i} = 0$ for all $j \ge 0$ and all $0 < i < d$ proving that $c \in A^{(d)}$. Thus, if $f \in \Frac{A^{(d)}}$ is integral over $A^{(d)}$ then it is integral over $A$ so $f \in A$ and thus $f \in \Frac{A^{(d)}} \cap A = A^{(d)}$ proving that $A^{(d)}$ is integrally closed.

\item Now consider a general closed subscheme $X \subset \P^r_A$. Suppose that $X$ is normal and for $n \ge 0$ the map $\Gamma(\P^r_A, \struct{\P^r_A}(n)) \onto \Gamma(X, \struct{X}(n))$ is surjective. The case $n = 0$ proves that $X$ is connected . Then in the above construction, $S = S'$ proving that $S$ is an integrally closed domain and thus $X$ is projectively normal. Conversely, suppose that $X$ is projectively normal. Then $X$ is normal because $\stalk{C(X)}{x} = \stalk{X}{x}[t]_{(t)}$ which is a normal domain and thus $\stalk{X}{x}$ must be a normal domain. Furthermore, if $X$ is not connected then the ideal $I(X)$ is not prime because $\I_X$ is not prime (explicity if $I(X)$ were prime then $\I_X \cong \wt{I(X)}$ would locally be given by prime ideals and therefore would cut out an integral subscheme) and thus $S(X)$ is not a domain contradicting our assumption that $S(X)$ is an integrally closed domain. Therefore, we can apply the construction of (a) to get $S \subset S'$ with $S'$ the integral closure. But since $S$ is integrally closed because we assumed that $X$ is projectively normal then $S = S'$. Since the map $S \embed S'$ is graded this proves that,
\[ \Gamma(\P^r_A, \struct{\P^r_A}(n)) \to \Gamma(X, \struct{X}(n)) \]
is surjective for each $n \ge 0$.
\end{enumerate}


\begin{lemma}
Let $X$ be a noetherian scheme. Then the following are equivalent,
\begin{enumerate}
\item $X$ is integral
\item $X$ is connected and $\stalk{X}{x}$ is a domain for each $x \in X$.
\end{enumerate}
\end{lemma}

\begin{proof}
If $X$ is integral then it is irreducible and every ring is a domain. Conversely, if $X$ is connected and $\stalk{X}{x}$ is a domain then it suffices to show that $X$ is irreducible since then it is also reduced and thus integral. Suppose let $Z_1$ and $Z_2$ be two irreducible components of $X$. If $x \in Z_1 \cap Z_2$ then they correspond to minimal primes in $\stalk{X}{x}$ which is a domain and thus $Z_1 = Z_2$. Therefore, the irreducible components of $X$ are disjoint. Since $X$ is noetherian, there are finitely many irreducible components and thus because they are disjoint they are open and thus clopen. Because $X$ is connected, this implies that there is a unique irreducible component and thus $X$ is irreducible.
\end{proof}

\begin{rmk}
Noetherianness is necessary (see \href{https://mathoverflow.net/questions/7477/non-integral-scheme-having-integral-local-rings}{this} answer). Intuitively, $\stalk{X}{x}$ being a domain implies that $X$ is reduced at $x$ and has only one irreducible component passing through $x$. However, if there are infinitely many irreducible components (and thus they need not be open) this does not imply that $X$ is either disconnected or only one irreducible component. For example, $\RR$ is connected but its irreducible components are points and thus each point lies on only one irreducible component.
\end{rmk}

\begin{rmk}
In the above proof, I heavily relied on the following lemma which I will repeat for completeness.
\end{rmk}

\begin{lemma}
Let $X$ be a scheme and $x \in X$. Then there is an inclusion-reversing bijection between closed irreducible subsets $Z \subset X$ containing $x$ and $\Spec{\stalk{X}{x}}$ where irreducible components containing $x$ correspond to minimal primes. Furthermore, the bijection is given by sending an irreducible closed subset $Z \subset X$ with reduced sheaf of ideals $\I$ to $\I_x \subset \stalk{X}{x}$ and sending a prime $\p \subset \stalk{X}{x}$ to $\overline{ \{ \p \} }$ under $\Spec{\stalk{X}{x}} \to X$.
\end{lemma}

\begin{rmk}
The exercise made significant use of the following lemma. Also see \href{https://stacks.math.columbia.edu/tag/0AVQ}{Tag 0AVQ} for properties of quasi-coherent torsion-free modules on integral schemes.
\end{rmk}

\begin{lemma}
Let $X$ be an integral scheme with generic point $\xi \in X$ and $\F$ a quasi-coherent torsion-free $\struct{X}$-module. Then $\res_{V,U} : \F(U) \to \F(V)$ is injective and therefore $\F(U) \subset \F_\xi$ and,
\[ \F(U) = \{ a \in \F_\xi \mid \forall x \in U : a \in \F_x \} = \bigcap_{x \in U} \F_x \]
with the intersection taken inside $\F_\xi$.
\end{lemma}

\begin{proof}
Since $\F(U) \to \F_\xi$ factors through $\F(U) \to \F(V) \to \F_\xi$ for each $V \subset U$ it suffices to show that $\F(U) \to \F_\xi$ is injective.
Choose an affine cover $U_i = \Spec{A_i}$ of $U$. Suppose that $f \in \F(U)$ restricts $f_\xi = 0$ in $\F_\xi$. But $A_i$ is a domain and $\F|_{U_i} \cong \wt{M_i}$ for some torsion-free $A_i$-module $M_i$ and thus $\F|_{U_i} \to \F_\xi$ is $M_i \mapsto (M_i)_{(0)} = M_i \ot_A \Frac{A}$ which is injective because $M_i$ is torsion-free. Therefore, we must have $f|_{U_i} = 0$ and thus $f = 0$ by the sheaf condition proving injectivity of $\F(U) \to \F_\xi$. Furthermore, for any $x \in X$, the map $\F_x \to \F_\xi$ is injective because any germ $f \in \F_x$ is represented by $f \in \F(U)$ for some open $U \ni x$ and if $f_\xi = 0$ then $f = 0$. Clearly, if $f \in \F(U) \subset \F_\xi$ then $f \in \F_x \subset \F_\xi$ for all $x \in X$. Conversely, if $f \in \F_x \subset \F_\xi$ for all $x \in U$ then there are $f_{U_x} \in \F(U_x)$ for open neighborhoods $x \in U_x \subset U$ such that $(f_{U_x})_\xi = f$. Therefore, by injectivity of $\F(U_x \cap U_y) \to \F_\xi$ these sections agree on the overlaps so they glue to a section $f \in \F(U)$. 
\end{proof}

\begin{lemma}
Let $X$ be an integral scheme and $\sS$ be a torsion-free quasi-coherent sheaf of $\struct{X}$-algebras such that $\sS_x$ is an integrally closed domain for each $x \in X$. Then $\sS$ is a sheaf of integrally closed domains.
\end{lemma}

\begin{proof}
Because $\sS(U) \to \sS_\xi$ is injective and $\sS_\xi$ is a domain we see that $\sS$ is a domain. Furthermore,
\[ \sS(U) = \bigcap_{x \in X} \sS_x \subset \sS_\xi \]
Now each of $\sS_x$ is an integrall closed domain. Let $K = \Frac{\sS(U)}$ and suppose that $f = \frac{a}{b} \in K$ is integral over $\sS$. Because $\sS(U) \to \sS_\xi$ is injective we get $\Frac{\sS(U)} \embed \Frac{\sS_\xi}$ and therefore $f \in \Frac{\sS_\xi}$ is integral over $\sS_x$ for each $x \in U$ because $\sS(U) \to \sS_x \to \sS_\xi$ so the coefficients lie in $\sS_x$ for each $x \in U$. Therefore $f \in \sS_x$ for each $x \in U$ because $\sS_x$ is integrally closed. Therefore, by the lemma $f \in \sS(U)$ proving that $\sS$ is integrally closed.
\end{proof}

\begin{rmk}
In particular, if $X$ is integral and normal then $\struct{X}(U)$ is an integrally closed domain for each open $U \subset X$.
\end{rmk}

\subsubsection{5.15 (IS THIS DONE?)!!}

\begin{enumerate}
\item Let $X = \Spec{A}$ be a Noetherian affine scheme. Suppose that $\F = \wt{M}$ is a quasi-coherent sheaf for $M$ some $A$-module. Now the coherent subsheaves correspond to finite $A$-submodules $N \subset M$. Clearly,
\[ M = \bigcup_{N \subset M \text{ finite}} N \]
and thus $\F$ is the union of its finite submodules since union commutes with localization.

\item Let $X = \Spec{A}$ be an affine noetherian scheme and $U \subset X$ an open subset and $\F$ a coherent sheaf on $U$. Let $\iota : U \to X$ be the inclusion which is quasi-compact since $X$ is noetherian (meaning $U$ is retrocompact) and separated so $\iota_* \F$ is quasi-coherent. By (a) the quasi-coherent sheaf $\iota_* \F$ is the union of its coherent subsheaves. For any coherent subsheaf $\F' \subset \iota_* \F$ we have $\F'|_U \subset \F$ is a coherent subsheaf of $\F$. However, since $U$ is noetherian and $\F$ is coherent its sheaves satisfy the ascending chain condition (Tag 01Y8) so the union must stabilize at a finite point giving a coherent sheaf $\F'$ on $X$ such that $\F'|_U = \F$.

\item Let $X = \Spec{A}$ be an affine noetherian scheme and $U \subset X$ an open subset. Let $\F$ be a coherent sheaf on $U$ and $\G$ a quasi-coherent sheaf on $X$ such that $\F \subset \G|_U$. Consider the natural map $\rho : \G \to \iota_* (\G|_U)$. Now $\iota_* \F \subset \iota_* (\G|_U)$ so consider the quasi-coherent submodule $\rho^{-1}(\iota_* \F) \subset \G$. However, on $U$ the map $\rho : \G \to \iota_* (\G|_U)$ is an isomorphism so $\rho^{-1}(\iota_* \F) |_U = \F$. Therefore, for any coherent subsheaf $\F' \subset \rho^{-1}(\iota_* \F)$ we have $\F'|_U \subset \F$ and thus the union of coherent subsheaves stabilizes giving $\F' \subset \rho^{-1}(\iota_* \F)  \subset \G$ such that $\F'|_U = \F$.

\item Now let $X$ be any noetherian scheme, $U \subset X$ an open subset. Then let $\F$ be a coherent sheaf on $U$ and $\G$ a quasi-coherent sheaf on $X$ such that $\F \subset \G|_U$. Since $X$ is Noetherian there is a finite affine cover $U_i = \Spec{A_i}$ with $A_i$ Noetherian.
First, since $\F|_{U_1 \cap U}$ is coherent we can extend it to a coherent sheaf $\F' \subset \G|_{U_1}$ on $U_1$ and then glue to a coherent sheaf $\F'_1 \subset \G|_{U \cup U_1}$ such that $\F'_1 |_U = \F$ since $\F'|_{U_1 \cap U} = \F|_{U_1 \cap U}$. Then we have a coherent sheaf $\F'_1 \subset \G|_{U_1 \cup U}$ on $U_1 \cup U$. Repeating this process (replace $U$ by $U_1 \cup U$) we can extend to a coherent sheaf $\F'$ on $X = (U_1 \cup U_2 \cup \cdots \cup U_n) \cup U$ such that $\F'|_U = \F$ and $\F \subset \G|_U$.

\item Let $X$ be a noetherian scheme and $\F$ a quasi-coherent sheaf. To show that $\F$ is the union of its coherent subsheaves, we need to show that for each section $s \in \F(U)$ there is a coherent subsheaf $\F' \subset \F$ such that $s \in \F'(U)$. Let $\F_s \subset \F|_U$ be the image sheaf of $\struct{U} \xrightarrow{s} \F_U$ which is coherent because it is a quotient of $\struct{U}$ and $X$ is noetherian. Then by (d) there is a coherent subsheaf $\F' \subset \F$ such that $\F'|_U = \F_s$ so $s \in \F'(U)$ proving that $\F$ is the union of its coherent subsheaves.
\end{enumerate}

\subsubsection{5.16 DO!!}

\begin{enumerate}
\item This is clear because $T^r$ and $S^r$ and $\bigwedge^r$ commute with localization so on affine opens these are just the above functors applied to the module so we reduce to the case of a free module and it is obvious.

\item Let $\F$ be locally free of rank $n$. 
\end{enumerate}

\subsubsection{5.17 DO!!}

A morphism $f : X \to Y$ of schemes is \textit{affine} if there is an open affine cover $\{ V_i \}$ of $Y$ such that $f^{-1}(V_i)$ is affine for each $i$.

\begin{enumerate}
\item THIS IS IN MY NOTES

\item Finite morphisms are affine by definition. Let $f : X \to Y$ be affine. Then the affine cover $\{ V_i \}$ has $f^{-1}(V_i)$ is affine and thus quasi-compact. Since for quasi-compactness it suffices to check on an affine open cover we see that $f$ is quasi-compact. Furthermore, separatedness is local on the target because the diagonal morphism commutes with pullbacks and $X \times_Y X$ is covered by the pullback of a cover from $Y$. Therefore, it suffices to show that $f^{-1}(V_i) \to V_i$ is separated but this is a morphism of affine schemes and thus separated.

\item Let $Y$ be a scheme and $\sA$ be a quasi-coherent sheaf of $\struct{Y}$-modules. For all affine opens $V \subset Y$ consider the set of morphisms $\Spec{\sA(V)} \to \Spec{\struct{Y}(V)} = V$. Because $\sA$ is quasi-coherent, if $V' \subset V$ is an open inclusion of affine opens then $\sA(V) \ot_{\struct{Y}(V)} \struct{Y}(V') \to \sA(V')$ induced by restriction is an isomorphism. Therefore we get Cartesian diagrams,
\begin{center}
\begin{tikzcd}
\Spec{\sA(V')} \arrow[d, "\res"] \arrow[r] & \Spec{\sA(V)} \arrow[d]
\\
V' \arrow[r] & V
\end{tikzcd}
\end{center}
with the top map induced by $\res : \sA(V) \to \sA(V')$ proving that $\Spec{\sA(V')} \to \Spec{\sA(V)}$ is an open immersion. Because these diagrams are Cartesian, there are unique isomorphisms of the pullbacks over the intersections which, by uniqueness, must therefore satisfy all cocycle conditions. (DO THIS BETTER)
\bigskip\\
Therefore, by the gluing construction, there exists a unique scheme $\rSpec{Y}{\sA}$ along with open immersions $\Spec{\sA(V)} \embed \rSpec{Y}{\sA}$ over $V \embed Y$. (DO THIS BETTTER)

\item If $\sA$ is a quasi-coherent $\struct{Y}$-algebra, then $f : X = \rSpec{Y}{\sA} \to Y$ is an affine morphism because $f^{-1}(V) \cong \Spec{\sA(V)}$ is affine from the open immersions for each affine open $V \subset Y$. 
\bigskip\\
Now suppose that $f : X \to Y$ is an affine morphism then $f$ is quasi-compact and separated and thus $\sA = f_* \struct{X}$ is a quasi-coherent sheaf of $\struct{Y}$-algebras. (FINISH THIS!!)

\item Let $f : X \to Y$ be an affine morphism and let $\sA  = f_* \struct{X}$. 
\end{enumerate}

\subsubsection{5.18 DO!!}

Let $Y$ be a scheme. A \textit{geometric vector bundle} of rank $n$ over $Y$ is a scheme $X$ and a morphism $f : X \to Y$, together with additional data consisting of an open covering $\{ U_i \}$ of $Y$. 

\end{document}
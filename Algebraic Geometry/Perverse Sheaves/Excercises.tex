\documentclass[12pt]{article}
\usepackage{import}
\import{../}{AlgGeoCommands}

\newcommand{\heart}{\ensuremath\heartsuit}

\begin{document}

\section{Week 1}

\subsection{Bhatt (Problems 1)}

\subsubsection*{2}

Let $\cA$ be an abelian category. Any category admits all colimits iff it admits coequalizers and all coproducts (easy exercise). Since $\cA$ is abelian it admits cokernels and therefore coequalizers and thus $\cA$ admits all colimits iff it admits all direct sums (coproducts).

\subsubsection*{3}

\begin{enumerate}
\item the category of finite $k$-vectorspaces has finite direct sums nut not countable direct sums. Likewise for countably generated vector spaces.

\item The opposite category of torsion abelian groups.

\item 
\end{enumerate}

\subsubsection*{4}

Let $\C$ be the category of torsion abelian groups. It is clear that $\C$ is abelian as kernels and cokernels of torsion groups are torsion since subgroups and quotients are torsion. Furthermore all direct sums exist in $\C$ because elements are zero all but finitely often and thus torsion since the nonzero entries are torsion.
\bigskip\\
Because $\C$ has cokernels and all coproducts it has all colimits. Furthermore, filtered colimits in $\Ab$ are exact so they are exact in $\C$ as well. For a generator, consider,
\[ X = \bigoplus_{n \in \Z^+} \Z / n \Z \]
Then for each element $a \in A$ for $A$ a torsion abelian group we get a map $X \to A$ whose image contains $a$ sending $\Z / n \Z \to 0$ unless $n$ is the order of $a$ in which case $1 \mapsto a$. Therefore we get a surjection,
\[ X^{\oplus A} \onto A \]

\subsubsection*{5}

Let $\cA$ be Grothendieck abelian and $I$ a category.  Let $\C = \mathrm{Fun}(I, \cA)$ be the functor category. Clearly, $\C$ is additive and admits kernels, cokernel, and infinite direct sums (constructed pointwise). 

\subsubsection*{7}

Let $\cA$ be an abelian category. Now $\Ch{\cA}$ is the subcategory of functors from $\Z$ as a poset to $\cA$ such that the composition of sucessive maps is zero. (CAN WE REDUCE THIS TO PREVIOUS EXERCISE?)

\subsubsection*{11}

Let $\cA$ be an abelian category and $\cA^{\N} = \Hom{}{\N^{\op}}{\cA}$ the category of projective systems. Assume that $\cA$ admits infinite direct sums and products.

\begin{enumerate}
\item 
Taking limits is right adjoint to the constant diagram functor $\Delta : \cA \to \cA^{\N}$ defined via $A \mapsto (n \mapsto A)$ with identity transition maps. Therefore $\lim : \cA^{\N} \to \cA$ preserves limits and thus is, in particular, left exact.

\item Note that given a projective system $\{ X_n \} \in \cA^\N$,
\[ \lim X_n = \ker{\left( \prod_{n \in \N} X_n \to \prod_{n \in \N} X_n \right)} \]
where on the $n^{\mathrm{th}}$ factor the map is the difference of projection $\prod X_{n'} \to X_n$ and $f_n \circ (\prod X_{n'} \to X_{n+1})$ where $f_n : X_{n+1} \to X_n$ is the transition map. (FINISH THIS)

\item 
\end{enumerate}

\subsubsection*{12}

Fairly obvious.

\subsubsection*{13}


Let $\cA$ be an abelian category and $f : K^\bullet \to L^\bullet$ be a map in $\Ch{\cA}$. Recall that,
\[ C(f) = K[1] \oplus L \]
where the differential is,
\[ \d_{C(f)} = \begin{pmatrix}
\d_{K[1]} & 0
\\
f[1] & \d_L
\end{pmatrix} \]
Specifically, $C(f)^i = K^{i+1} \oplus L^i$ and $\d(x,y) = (-\d{x}, \d{y} + f(x))$.

\begin{enumerate}
\item Let $A^\bullet \in \Ch{\cA}$ be a complex. Consider,
\[ g \in \Hom{\Ch{\cA}}{C(f)}{A^\bullet} \]
Then $g^i = (k^{i}, h^i)$ where $k^i : K^{i+1} \to A^i$ and $h^i : L^i \to A^i$ which satisfy,
\[ g^{i+1} \circ \d_C^i = \d_A^{i+1} \circ g^i \]
Explicitly,
\[ -k^{i+1} \circ \d_K^{i+1}(x) +  h^{i+1} \circ \d_L^i(y) + h^{i+1} \circ f^{i+1}(x) = \d_A^{i+1} \circ (k^i(x) + h^i(y)) \]
Setting $x = 0$ we find that,
\[ h^{i+1} \circ \d_L^i(y) = \d_A^{i+1} \circ h^i(y) \]
and therefore $h \in \Hom{\Ch{\cA}}{L^\bullet}{A^\bullet}$. 
Setting $y = 0$ we find that,
\[  h^{i+1} \circ f^{i+1}(x) = \d_A^{i+1} \circ k^i(x) + k^{i+1} \circ \d_K^{i+1}(x) \]
therefore $k$ is a nullhomotopy of $h \circ f$ so we see that,
\[ \Hom{\Ch{\cA}}{C(f)}{-} = \{ k : L^\bullet \to A^\bullet \text{ and } h : K^{\bullet + 1} \to A^\bullet \mid h \text{ is a nullhomotopy of } k \circ f \} \]

\item If $L$ is acyclic then from the long exact sequence for the exact triangle,
\[ K \xrightarrow{f} L \to C(f) \to K[1] \]
shows that $H^i(L) \to H^i(C(f))$ is an isomorphism.

\item 
\end{enumerate}

\subsection{Bhatt Lectures}

\subsubsection*{2.4}

Let $\C$ be a category such that $\C$ is enriched over $\Ab$ with finite coproducts. Given $f, g : A \to B$ there exists a map $f + g : A \to B$. To show that being abelian is a property, we must describe $f + g$ in terms of internal properties of the category. That is, there is a unique additive structure on any additive category.
\bigskip\\
Consider the map $A \to A \oplus A \to B$ defined by,
\[ (f, g) \circ (\iota_1 + \iota_2) = (f, g) \circ \iota_1 + (f, g) \circ \iota_2 = f + g \]
Therefore, it suffices to show that $h = \iota_1 + \iota_2$ is internal to the category. There are zero maps $A \to 0$ (where $0$ is the initial object) b/c $\Hom{\C}{A}{0}$ has an identity. Then $(\id, 0) \circ h = \id + 0 = \id$ and $(0, \id) \circ h = \id$. Call $\pi_1 = (\id, 0)$ and $\pi_2 = (0, \id)$ then these make $A \oplus A$ a product and $h$ the diagonal so $h$ is unique.
\bigskip\\
To prove this consider $a : C \to A$ and $b : C \to B$ then $q = \iota_1 \circ a + \iota_2 \circ b$ satisfies $\pi_1 \circ q = a$ and $\pi_2 \circ q = b$. Furthermore, let $q' : C \to A \oplus B$ be any map with this property. Then $q' = (\iota_1 \circ \pi_1 + \iota_2 \circ \pi_2) \circ q' = \iota_1 \circ a + \iota_2 \circ b$ becuase,
\[ (\iota_1 \circ \pi_1 + \iota_2 \circ \pi_2) \circ \iota_i = \iota_i + 0 = \iota_i \]
and thus $(\iota_1 \circ \pi_1 + \iota_2 \circ \pi_2) = \id$ because $A \oplus B$ is a coproduct.  
\bigskip\\
Notice that this construction only relied on the choice of a zero map $A \to 0$. However, the identity of $0 : \Hom{\C}{0}{0}$ must be $\id_0 : 0 \to 0$ because $0$ is initial so this set has a unique element. Therefore, for any $f : A \to 0$ we have $f = \id_0 \circ f = 0 \in \Hom{\C}{A}{0}$ because $\id_0$ is the identity of the group and $- \circ f : \Hom{\C}{0}{0} \to \Hom{\C}{A}{0}$ is a group map. Therefore, $\Hom{\C}{A}{0}$ has a single element so there is no choice of zero map $A \to 0$.
\bigskip\\
Since there is a unique map $A \to 0$ we see that $0$ is initial and final. 


\subsubsection*{2.11}

Solved in Bhatt problems 2,3.

\subsubsection*{2.20}

\begin{enumerate}
\item Bhatt problems 7

\item Let $\cA$ be a Grothendieck abelian category. We construct the injective resolution inductively. First, $X \to (X \embed I(X))$ is functorial. Assume there is a functorial assignment,
\[ X \mapsto (X \embed I^0(X) \to I^1(X) \to \cdots \to I^n(X)) \]
Then consider 
\[ \coker{(I^{n-1}(X) \to I^n(X))} \embed I(\coker{(I^{n-1}(X) \to I^n(X))}) = I^{n+1}(X) \]
which is funtorial in $X$ because cokernels and $C \mapsto I(C)$ is thus giving,
\[ X \mapsto (X \embed I^0(X) \to I^1(X) \to \cdots \to I^{n+1}(X)) \]
\end{enumerate}


\section{Week 2}

\subsection{Bhatt (Problems 1)}

\subsubsection*{15}

\begin{enumerate}
\item a
\end{enumerate}

\subsection{Bhatt (Problems 2)}


\subsubsection*{6}


\subsubsection*{7}


\subsubsection*{8}


\subsubsection*{9}

\subsection{Bhatt (Lectures)}

\subsubsection*{2.25}

\subsubsection*{6.12}

\subsubsection*{6.13}

\subsection{Tsai (Problems)}

\subsubsection*{1}

\subsubsection*{2}

\subsubsection*{3}

\section{Week 4}

\subsection{Bhatt (Problems 3)}

\subsubsection*{1}


Let $\D$ be a triangulated category equiped with a $t$-structure. Let $X, Y \in \D^{\heart}$. Recall that,
\[ \Ext{-n}{\D}{X}{Y} = \Hom{\D}{X}{Y[-n]} \]
Suppose that $n > 0$, since $Y \in \D^{\ge 0}$ we see that $Y[-n] \in \D^{\ge n} \subset \D^{\ge 1}$ and furthermore $X \in \D^{\le 0}$ and therefore,
\[ \Ext{-n}{\D}{X}{Y} = \Hom{\D}{X}{Y[-n]} = 0 \]
when $n > 0$.

\subsubsection*{2}

Let $X$ be a topological space and $K \in D(X)$,

\begin{enumerate}
\item Consider $X = \P^1$ and $K = \struct{X} \oplus \struct{X}(-2)[1]$ Then we consider,
\begin{align*}
\Hom{D(X)}{K|_U}{K|_U} & = \Hom{D(X)}{\struct{U}}{\struct{U}} \oplus \Hom{D(X)}{\struct{U}}{\struct{U}(-2)[1]}
\\
& \oplus \Hom{D(X)}{\struct{U}(-2)[1]}{\struct{U}} \oplus \Hom{D(X)}{\struct{U}(-2)[1]}{\struct{U}(-2)[1]}
\\
& = \Gamma(U, \struct{U}) \oplus H^1(U, \struct{U}(-2)) \oplus \Gamma(U, \struct{U})
\end{align*} 
which is not a sheaf because of the $H^1(U, \struct{U}(-2))$ term. We use,
\[ \Hom{D(X)}{\struct{U}}{\struct{U}(-2)[1]} = \Ext{1}{D(X)}{\struct{U}}{\struct{U}(-2)} = H^1(U, \struct{U}(-2)) \]
\item Suppose that $\Ext{i}{K|_U}{K|_U} = 0$ for all $i < 0$ and open $U \subset X$.

\begin{lemma}
If the cohomology sheaves $H^i(K) = 0$ for all $i < d$ then $U \mapsto \mathbb{H}^d(U, K)$ is a sheaf.
\end{lemma}
 
\begin{proof}
$K \cong \tau^{\ge d} K$ is an equivalent so we may assume $K$ is zero in deg $< d$. Then choose a quis $K \iso I$ for an injective resolution. Then,
\[ \mathbb{H}^d(U, K) = \ker{(I^d(U) \to I^{d+1}(U))} \]
and therefore $H^d(-, K) = \ker{(I^d \to I^{d+1})}$ is a sheaf.
\end{proof} 
\noindent
Let $L,K$ be complexes. Assume that $\Ext{i}{D(X)}{L|_U}{K|_U} = 0$ for $i < 0$ and $U \subset X$ open. Now $H^i(\mathrm{RHom}(L,K))$ is the sheafification of,
\[ U \mapsto \Ext{i}{D(U)}{L|_U}{K|_U} \]

\item 
\end{enumerate}

\subsubsection*{3}

Let $\cA$ be an abelian category with enough projectives. Assume that $\Ext{2}{\cA}{X}{Y} = 0$ for all $X, Y \in \cA$.

\begin{enumerate}
\item Let $K \in D^b(\cA)$. Choose a projective resolution $P \to K$ 

\item 
\end{enumerate}

\subsubsection*{4}

Let $D^b_f(k)$ be the derived category of bounded complexes of $k$-vectorspaces with finitely generated cohomology.

\begin{lemma}
A t-structure is determined by $\D^{\le 0}$.
\end{lemma}

\begin{proof}
We can recover 
\[ \D^{\ge 1} = \{K \in \D \mid \Hom{\D}{\D^{\le 0}}{X} \} \]
\end{proof}

\subsubsection*{5}

Let $\D$ be a triangulared category with a $t$-structure. Let $K \in \D$ be a direct summand of $L \in \D^{\le 0}$. Consider,
\[ L = K \oplus F \]
We know $\tau^{\ge 1} (K \oplus F) = 0$ but $\tau^{\ge 1}$ is a left adjoint and thus preserves colimits so $\tau^{\ge 1} (K) \oplus \tau^{\ge 1}(F) = 0$ therefore $\tau^{\ge 1}(K) = 0$ so $K \in \D^{\le 0}$. 

\subsubsection*{6}


\section{Week 4}

(DOOOOO THHHHIIIIIS!!!!!)

\section{Week 7}

\subsection{Tsai (Problems 1)}

\subsubsection*{4}

\subsubsection*{5}

\subsubsection*{6}

\subsection{Tsai (Problems 2)}

\subsubsection*{1}

Let $E \subset \P^2$ be an elliptic curve. Let $V \subset \A^3$ be the corresponding affine cover over $E$. Let $o$ be the origin and $U := V \setminus \{ o \}$ be the smooth locus of $V$. Wrtie $\iota : \{ o \} \embed V$ and $j : U \embed V$ the embeddings. We want to compute $\iota^* R j_* \underline{\Q}_U$.
\bigskip\\
(DO THIISSS!!)

\subsubsection*{2}

As in the last problem we want to show that $D_V \underline{\Q}_V[2] \not\cong \underline{\Q}_V[2]$.
\bigskip\\
(DO THIS!!!!)

\subsubsection*{3}

As in the last problem, we want to show that,
\[ R\Gamma(V, \tau_{\le -1} R j_* \underline{\Q}_U[2])) \]
is dual to
\[ R \Gamma_c(V, \tau_{\le -1}(Rj_* \underline{\Q}_U[2])) \]

\subsubsection*{4}

Let $X$ be a smooth complete variety over $\mathbb{\C}$ and let $x \in X$ be a fixed point. Let $\iota_x : \{ x \} \embed X$ be the inclusion. We want to compute $\F = \mathrm{RHom}{}{\iota_x \underline{\Q}}{\underline{\Q}_X}$.

\newcommand{\Perv}[1]{\mathrm{Perv}\left( #1 \right)}

\newcommand{\RHom}[3]{\mathrm{RHom}_{#1} \left( #2, #3 \right)}


\section{The Perverse t-structure}

Every scheme will be reduced finite type separated over a field $k$ either finite or algebraically closed. Let,
\[ D(X) := D_c^b(X, \overline{\Q}_\ell) \]
with $\ell$ not the characteristic. Let $D_X K$ be the Verdier dual of $K \in D(X)$.

\begin{prop}[Poincare Duality]
For a morphism $ : X \to Y$ then,
\[ Rf_* (D_X K) \cong D_Y (Rf_! K) \]
for any $K \in D_c^b(X, \overline{\Q}_\ell)$. 
\end{prop}

\begin{prop}[Biduality]
We have $D_X \circ D_X = \id$ via natural maps $K \to D_X D_X K$. Furthermore,
\begin{enumerate}
\item $D_Y \circ R f_! = Rf_* \circ D_X$
\item $D_Y \circ R _* = R f_! \circ D_X$
\item $D_X \circ f^* = f^! \circ D_Y$
\item $D_X \circ f^! = f^* D_Y$
\end{enumerate}
\end{prop}

\begin{defn}
We define the \textit{perverse t-structue} $(D_p^{\le 0}(X), D_p^{\ge 0}(X))$ on $D(X)$. As follows,
\begin{enumerate}
\item $B \in D_p^{\le 0}(X) \iff \dim \supp{}{H^{-i}(B)} \le i$ for all $i \in \Z$
\item $B \in D_p^{\ge 0}(X) \iff \dim \supp{}{H^{-i} (D_X B)} \le i$ for all $i \in \Z$
\end{enumerate}
The the abelian category of \textit{perverse sheaves} is $\Perv{X} = D_p^{\le 0} \cap D_p^{\ge 0}$ is the heart. 
\end{defn}

\begin{example}
If $X$ is \textit{smooth} of dimension $n$ and $K \in D(X)$ is \textit{lisse} (i.e. it has lcc cohomology sheaves) then,
\[ D_X K = \RHom{X}{K}{\overline{\Q}_\ell(n)}[2n] \]
and therefore,
\[ K \in D_p^{\le 0}(X) \iff H^i(K) = 0 \text{ for } i > - n \]
and likewise,
\[ K \in D_p^{\ge 0} \iff H^i(K) = 0 \text{ for } i < - n \]
\end{example}

\begin{proof}
To show that this is a t-structue we use that any $K \in D(X)$ admits a stratification (because of constructibility) $\Sigma = (S_i)_i$ of $X$ s.t. $K|_{S_i}$ is lisse for every $i$ and,
\[ D(X) = \bigcup_{\Sigma} D_{\Sigma}(X) \]
where $D_\Sigma(X)$ is the full subcategory of $D(X)$ consisting of complexes which are ``lisse along $\Sigma$''. To show that $D_\Sigma(X)$ has a t-structue, appeal to the gluing formalisim using the stratification (See KW, III.3).
\end{proof}

\begin{lemma}
Let $j : U \to X$ open, $\iota : Y \to X$ its complementary closed. Then for $K \in D(X)$,
\begin{enumerate}
\item $B \in D_p^{\le 0}(X) \iff j^* D^{\le 0}(U) \text{ and } \iota^* B \in D_p^{\le 0}(Y)$
\item $B \in D^{\ge 0}(X) \iff j^! B = j^* B \in D^{\ge 0}(U) \text{ and } R \iota^! B \in D_p^{\ge 0}(Y)$
\end{enumerate}
\end{lemma}

\begin{prop}
If $j : U \to X$ is an open embedding and $\iota : Y \to X$ is the complementary closed, then for the perverse t-structure,
\begin{enumerate}
\item $j^!$ and $\iota^*$ are t-right exact 
\item $\iota_*$ and $j^*$ are t-exact
\item $R j_*, R \iota^!$ are t-left exact
\end{enumerate}
\end{prop}

\begin{proof}
Use adjunction of the various functors and duality.
\end{proof}

\begin{rmk}
In what follows, t-exactness will always refer to the perverse t-structue.
\end{rmk}

\subsubsection{Intermediate Extensions}

Fix a dense open embedding $j : U \to X$ and a complementary closed $\iota : Z \to X$.

\begin{defn}
A perverse sheaf $\overline{B}$ on $X$ is a \textit{extension} of $B$ on $U$ if $j^* \overline{B} = B$. 
\end{defn}

\begin{rmk}
The intermediate extension is the ``best'' extension of $B$ characterized amoung extensions by the following equivalent properties,
\begin{enumerate}
\item $\overline{B}$ has no subobjects or quotients in $\iota_* \Perv{Y}$ which is a full subcategory of $\Perv{X}$ because $\iota_*$ is t-exact and fully faithful.
\item $H^0_p(\iota^* \overline{B}) = 0$ and $H_p^0(R \iota^! \overline{B}) = 0$
\item $\iota^* \overline{B} \in D_p^{\le -1}(Y)$ and $R \iota^! \overline{B} \in D_p^{\ge 1}(Y)$
\end{enumerate}
[More equivalences in KW, III.5.1]. We denote the intermediate extension by $j_{!*} B$ and moreover this is computed via,
\[ j_{!*} B = ``\im{(j_! B \to R j_* B)}`` = \im{(H_p^0(j_! B) \to H_p^0(R j_* B))} \]
where we take $H_p^0$ to make these actually perverse sheaves (recall that $j_!$ and $R j_*$ are not fully t-exact so they don't preserve perverse sheaves so to take images in an abelian category we need to project onto the heart).
\end{rmk}

\begin{proof}
First claim that,
\begin{enumerate}
\item $\iota_* H_p^0(R \iota^1 \overline{B}) \embed \overline{B}$ is the largest subobject from $\iota_* \Perv{Y}$
\item $\overline{B} \onto \iota_* H^0_p(\iota^* \overline{B})$ is the largest quotient from $\iota_* \Perv{Y}$. 
\end{enumerate}
The proof follows from explointing some adjunctions. We also use that $\iota^*$ is t-right exact and that $R \iota^!$ is t-left exact.
\end{proof}

\begin{example}
Suppose $X$ is smooth and $B = \F|_U[d]$ for $\F$ an lcc $\overline{\Q}_\ell$-sheaf on $X$. Then, $j_{!*} B = \F[d]$. Indeed,
\[ \iota^* \F[d] \in D_p^{\le - 1}(Y) \text{ and } R \iota^! \F[d] \in D_p^{\ge 1}(Y) \]
because,
\[ \dim{\supp{}{H^\nu(\iota^* \F[d])}} = 
\begin{cases}
-\infty & \nu \neq d
\\
\dim{Y} & \nu = d 
\end{cases} \]
and therefore it satisfies the perversity condition. The other condition follows from duality.
\end{example}

\subsection{Comparison}

In this section only let $k = \CC$. Let $X$ be finite type reduced separated $\CC$-scheme. 

\begin{thm}[BBD, 6.12]
For every prime $\ell$, there is an equivalence of categories,
\[ \{ \text{constructible } \Z_\ell\text{-sheaves on } X_{\et} \} \iso \{ \text{constructible sheafs of } \Z_\ell-\text{modules on } X(\CC) \} \]
extending to an equivalence,
\[ D_c^b(X_\et, \Z_\ell) \iso D^b_c(X(\CC), \Z_e\ll) \]
and cohomology agrees on both sides and the equivalence commutes with the six operations. 
\end{thm}

\begin{rmk}
Replacing $\Z_\ell$ by $\Q_\ell$ or $\overline{\Q}_\ell$ still gives a functor but it is only fully faithful (though every other claim is true). Say that $X$ is smooth. Then lcc $\Q_\ell$-sheaves correspond to continuous representations of $\pi_1^{\et}(X, \bar{x})$. Since $\pi_1^{\et}(X, \bar{x})$ is profinite, it is exact, and therefore factors through a lattice $\GL{n}{\Z_\ell} \to \GL{n}{\Q_\ell}$. 
\end{rmk}

\begin{example}
If $X = \Gm$ and $\F$ is the rank $1$ local system on $\Gm(\C) = \C^\times$ with monodromy given by multiplication by $\ell$. This does not preserve a lattice so it does not come from an \etale sheaf. 
\end{example}

\begin{rmk}
On both sides of the equivalence there are perverse t-sructures both defined by dimension bounds on the support of cohomology. Thus they agree and we get $\Perv{X_{\et}} \embed \Perv{X(\C)}$.
\end{rmk}

\begin{example}
if $\F$ is a rank $2$ lcc $\overline{\Q}_\ell$-shraf on $Y = \Gm \embed \A$ with monodromy,
\[
\begin{pmatrix}
1 & 1 
\\
0 & 1
\end{pmatrix}
\]
then $j_{!*} \F[1]$ is concentrated in deg $-1$ and $\iota^* j_{! *} \F[1] = \overline{Q}_\ell[1]$. 
\end{example}

\begin{example}
If $X$ is the affine nodal cubic $\{ y^2 = x^3 + x^2 \}$ and $p$ is the singular point them
\[ \iota^* j_{!*} (\overline{\Q}_\ell[1]) = \overline{\Q}_\ell^{\oplus 3}[1] \] 
\end{example}

\subsubsection{Finiteness properties of $\Perv{X}$}

A perverse sheaf $B$ on $X$ is simple iff $B = \iota_{0,*} j_{0!*} \G[d]$ where $\iota_0 : Y_0 \embed X$ is a closed embedding with $Y_0$ irred. of dim $d$ and $j_0 : V_0 \embed Y_0$ is an open desne embedding with $V_0$ smooth and $\G$ is a irreducible lcc $\overline{Q}_\ell$-sheaf on $V_0$. 
\bigskip\\
Take $j : U \embed X$ and $\iota : Y \embed X$ such that is dense and smooth and use the earlier claim,
\[ \iota_* H_p^0(\iota^! b) \embed B \]
the largest perverse subobject. Then,
\[ B \to \iota_* H_p^0(\iota^* B) \]
Shrink $U$ to assume $B|_U$ is lisse and use noetherian induction. For the other direction, show $\iota_{0*}, j_{0!*}$ preserve simplicity, and that $\G[d]$ is simple as a perverse sheaf. For the second part, say $B \embed \G[d]$ is a nontrivial sub. After shrinking $V_0$, can assume that $B$ is lisse, hence $\F[d]$ for $\F$ lcc.
\bigskip\\
$\Perv{X}$ is artinian and noetherian so there are Jordan-Holder decompositions. Furthermore $\Perv{U}$ is a full subcategory of $\Perv{X}$ via $B \mapsto j_{!*} B$ i.e.,
\[ \Hom{\Perv{X}}{j_{!*} B_1}{j_{!*} B_2} = \Hom{\Perv{U}}{B_1}{B_2} \]

\subsubsection{Exactness Theorems}

\begin{theorem}
We have the following exactness theorems,
\begin{enumerate}
\item if $f : X \to Y$ is affine then $R f_*$ is t-right exact and $R f_!$ is t-left exact
\item if $f : X \to Y$ is smooth of relative dimension $d$ then $f^*[d] = f^![-d](-d)$ is t-exact. 
\end{enumerate}
\end{theorem} 

\begin{proof}
(a) uses Artin's theorem on pushforwards under affine morphisms,
\[ \dim{\supp{}{R^q f_* \F}} \le \dim{\sup{}{\F}} - q \]
(SGA4, Exp. XIV, Thm 3.1). For $R f_!$ we apply duality. 
\bigskip\\
For (b) (using duality) we just need to show that $f^*[d]$ is t-right exact. This follows from,
\[ \dim{\supp{}{H^{-\nu}(f^* B)}} = \dim \supp{}{f^* H^{-\nu}(B)} \le \dim{\supp{}{H^{-\nu}(B)}} + d \]
\end{proof}

\subsubsection{The Fourier Transform}

Fix $\psi : \FF_q \to \overline{\Q}_\ell^\times$ a nontrivial additive character (throughout, $k = \FF_q$, subscript 0 denotes something defined over $k$). Then $\L_0(\psi)$ is the Artin-Schrier sheaf on $\A^1_0$ coming from,
\begin{center}
\begin{tikzcd}
\pi_1(\A_0^1, \bar{0}) \onto \FF_q \arrow[r, "\psi"] & \overline{\Q}_\ell^\times
\end{tikzcd}
\end{center}

\begin{rmk}
The map $\pi_1(\A^1_0, \bar{0}) \to \FF_q$ comes from the Artin-Schreier cover $\A^1_0 \to \A^1_0$ by $x \mapsto x^q - x$. 
\end{rmk}
\noindent
Let $n,r,s$ be natural numbers, $n = r + s$ and consider,
\begin{center}
\begin{tikzcd}
& \A^r_0 \times \A_0^r \A_0^s \arrow[rr, "m_{12}"] \arrow[dl, "p_{13}"] \arrow[dr, "p_{23}"]  & & \A_0^1
\\
\A_0^r \times \A_0^s & & \A_0^r \times \A_0^s
\end{tikzcd}
\end{center}

\begin{defn}
Let $T_\psi : D^b_c(\A^n_0, \overline{\Q}_\ell) \to D^b_c(\A^n_0, \overline{\Q}_\ell)$ via,
\[ T_\psi K = R(p_{13})_! (p_{23}^* K \otimes m_{12}^* \L_0(\psi))[r] \]
\end{defn}
More generally, if $Y_0$ is a $k$-scheme, we define,
\[ T_\psi : D(\A^n_0 \times Y_0) \to D(\A^n_0 \times Y_0) \]
which depends on $r$ as well as $\psi_n$.

\begin{thm}
For $B \in D(\A^n_0 \times Y_0)$ the following hold,
\begin{enumerate}
\item $T_{\psi^{-1}} T_{\psi} B = B(-r)$
\item If $B$ is mixed, $w(B) \le B \iff w(T_\psi B) \le w + r$ where,
\[ w(B) = \max_{\nu} w(H^\nu(B)) - \nu \]
\item $T_\psi : \Perv{\A_0^n \times Y_0} \to \Perv{\A^n_0 \times Y_0}$ is an equivalence of categories. 
\end{enumerate}
\end{thm}

\begin{proof}
(a) $r = n = 1$, and $Y_0 = \Spec{k}$ was covered in Koji's talk, KW say it is easy to reduce to this case by proper base change. 
\bigskip\\
(b) the forward direction follows from the Weil conjectures and the backwards direction follows from (a)
\bigskip\\
Note $p_{23}$ is smooth with equidimensional $r$ fibers. By a previous exactness theorem, $p_{23}^*[r]$ is t-exact. Also, $- \otimes  m_{12}^* \L_0(\psi)$ is t-exact and since $p_{13}$ is affine, Artin's theorem says that $R(p_{13})_!$ is t-left exact. So $T_\psi$ is t-left exact. So $T_\psi B \in D_p^{\ge 0}$. We have a dist. triangle,
\[ H_p^0(T_\psi B) \to T_\psi B \to \tau_{\ge 1}^p T_\psi B \]
which equals
\[ T_{\psi^{-1}} H_p^0(T_\psi B) \to B(-r) \to T_{\psi^{-1}} \tau_{\ge 1}^p T_\psi B \]
By left exactness, $T_{\psi^{-1}} \tau_{\ge 1}^p T_{\psi B} \in D_p^{\ge 1}$ so the rightmost map is $0$. By Fourier inversion, $T_\psi B \to \tau_{\ge 1}^p T_\psi B$ is 0. Thus,
\[ (H^0_p T_\psi B)[1] \cong \mathrm{cone} \left( T_\psi B \to \tau^p_{\ge 1} T_\psi B) \right) = T_\psi B[1] \oplus \tau_{\ge 1}^p T_{\psi} B \]
Then shift by $[-1]$ and apply $\tau_{\ge 1}^p$ to get $\tau^p_{\ge 1} T_\psi B = 0$ which suffices becasuse we already know that $\tau_{\ge 1}^p T_\psi B \in D^{\le 1}$ (OR SOMETHING LIKE THAT??)
\end{proof}

\begin{rmk}
If $B \in \Perv{\A^r_0 \times \A^r_0 \times \A^s_0}$ then $B \otimes m^* \L$ is also perverse. We need to check,
\[ \supp{}{H^{-\nu}(B \otimes m_{12}^* \L)} = \supp{}{H^{-\nu}(B)} \]
and therefore,
\[ B \otimes m_{12}^* \L \in D_p^{\le 0} \]
Then,
\[ D(B \otimes m_{12}^* \L) = D(B \otimes DD m_{12}^* \L) = D \RHom{}{D m_{12}^* \L}{B} \]
\end{rmk}

\section{Oct. 12}

\subsection{Decomposition Theorem (Gabber)}

Let $k_0$ be the base field, finite and $k$ its algebraic closure. Let $X_0$ be defined over $k_0$ and $X$ its base change to $k$. Fix an isomorphism $\tau : \CC_\ell \to \CC$.

\begin{theorem}[Usual Form]
Let $f_0 : X_0 \to Y_0$ be proper then,
\[ R f_* IC_X \]
can be decomposed into a direct sum of $B[i]$ where $B \in \Perv{Y}$ are simple perverse sheaves.
\end{theorem}

\newcommand{\Qbar}{\overline{\Q}}

\begin{thm}[Gabber]
Let $B_0 \in D^b_c(X_0, \Qbar_\ell)$ is $\tau$-pure of weight $w$ and $B \in D^b_c(X, \Qbar_\ell)$ its base change. Then,
\[ B \cong \bigoplus_{g} A[i] \]
with $A$ simple perverse sheaf on $X$.
\end{thm}

\begin{defn}
Let $B_0 \in D^b_c(X_0, \Qbar_\ell)$ is $\tau$-mixed if all its cohomology sheaves are $\tau$-mixed in the usual sense. Then define $D_{\text{mixed}}(X_0)$ is the triangulated subcategory of $D^b_c(X_0, \Qbar_\ell)$ spanned by all $\tau$-mixed complexes.
\end{defn}

\begin{rmk}
The six operations preserve $\tau$-mixed complexes.
\end{rmk}

\begin{defn}
Suppose that $B_0 \in D^b_c(X_0, \Qbar_\ell)$ is $\tau$-mixed then define,
\[ w(B_0) = \max_{\nu} \left( w(\mathcal{H}^\nu(B_0)) - \nu \right) \]
where as usual $w(\F)$ for a sheaf $\F$ is the maximum of all eigenvalues of $F_{\bar{x}}$ on $\F_{\bar{x}}$ for all $x$ of the number,
\[ \frac{2\log{|\tau(\alpha)|}}{\log{|\# \kappa(x)|}} \]
meaning it has norm the correct power of $q$ if $k_0 = \FF_q$.
\end{defn}

\begin{defn}
Now let,
\[ D^n_{\le w}(X_0) = \{ K_0 \in D_{\text{mixed}}(X_0) \mid w(K_0) \le w \} \]
and similarly,
\[ D^n_{\ge w}(X_0) = \{ K_0 \in D_{\text{mixed}}(X_0) \mid w(D(K_0)) \ge - w \} \]
Then we say that $K_0$ is $\tau$-pure of weight $w$ if,
\[ K_0 \in D^b_{\le w}(X_0) \cap D^b_{\ge w}(X_0) \]
\end{defn}

\begin{rmk}
If $X_0 = \Spec{k_0}$ and $K_0$ is a Weil sheaf given by a $\Z$-rep $V$ then $K_0$ is pure of wirhgt $w$ iff all eigenvalues $\alpha$ of $F_{k_0}$ satisfy,
\[ | \tau \alpha| = (\# k_0)^{\frac{w}{2}} \]
\end{rmk}

\begin{rmk}
Let $K_0 \in D_{\mathrm{mixed}}(X_0)$ is an actual sheaf, then being $\tau$-pure in the above sense is \textit{not} the same as being $\tau$-pure in the sense we had last lecture. The ideal is that, when $X0$ is not smooth or $K_0$ is not smooth then $D K_0$ can be not as expected. However, if $X_0$ is smooth and $K_0$ is lisse then these two notions agree.
\end{rmk}

\begin{rmk}
If $w(K_0) \ge w$ this does not imply that $K_0 \in D^b_{\le w}(X_0)$ goes wrong in $\Spec{k_0} = X_0$ case. However, $K_0 \in D^b_{\ge w}(X_0)$ implies $w(K_0) \ge w$.
\end{rmk}

\begin{prop}
If $K_0 \in D^b_{\text{mixed}}(X_0)$ then,
\begin{enumerate}
\item $w(K_0[n]) = w(K_0) + n$
\item $w(K_0(m)) = w(K_0) - 2m$
\item If $f : X_0 \to Y_0$ then,
\[ R f_!(D_{\le w}(X_0)) \subset D_{\le w}(Y_0) \]
and there is a spectral sequence,
\[ R f_!^p (\mathcal{H}^q(K_0)) \implies \mathcal{H}^{p+q}(R f_! K_0) \]
\item for $f : X_0 \to Y_0$ then,
\[ R f_* (D_{\ge w}(X_0)) \subset D_{\ge w}(Y_0) \]
by the dual of the previous.
\end{enumerate}
\end{prop}

\begin{lemma}[Semicontinuity of Weights]
if $j : U_0 \embed X_0$ is open dense, $\iota : Y_0 \embed X_0$ its complement and $\overline{B_0} \in \Perv{X_0}$ is $\tau$-mixed s.t. 
\[ j^*(\overline{B_0}) = B_0 \]
assume $H_p^0(\iota^* (\overline{B}_0)) = 0$. Then,
\[ w(\overline{B}_0) = w(B_0) \]
\end{lemma}

\begin{rmk}
For $B_0 \in \Perv{U}$ this lemma apply for $j_{!*} B$.
\end{rmk}

\begin{lemma}[Subquotient cannot increase weight]
If $B_0 \in \Perv{X_0}$ is $\tau$-mixed, then for any $A_0$ perverse subquotient of $B_0$,
\[ w(A_0) \le w(B_0) \]
\end{lemma}

\begin{lemma}[Weight Filtration]
For $E_0 \in \Perv{X_0}$ and $\tau$-mixed, then $E_0$ has a unique finite increasing $\tau$-weight filtration,
\[ 0 \subset E_0^{(w_1)} \subset \cdots E_0^{(w_r)} = E_0 \]
such that $E_0^{(w_i)}/E_0^{(w_{i-1})}$ is nonzero and $\tau$-pure (in the sense of complexes not sheaves) of weight $w_i$ such that $w_i < w_j$ for $i < j$. Furtehrmore, this filtration is functorial.
\end{lemma}

\begin{theorem}
Gabber's theorem implies ``usual form''.
\end{theorem}

\begin{proof}
Consider,
\[ IC_{X_0} = j_{!*} \left( \Qbar_\ell[d] (-d/2) \right) \]
Then,
\[ D(IC_{X_0}) = j_{!*}(D \Qbar_\ell) = j_{!*} \Qbar_\ell = IC_{X_0} \]
This implies that $IC_{X_0} \in D_{\ge 0}(X_0)$. Thus $IC_{X_0}$ is pure of weight $0$.  Then if $f : X_0 \to Y_0$ is proper then $R f_* = R f_!$. On smooth $U$ we can compare,
\[ K_U \cong \Qbar_\ell[2d](d) \]
and then,
\[ \RHom{}{\Qbar_\ell[d](d/2)}{\Qbar_\ell[2d](d)} \cong \Qbar_\ell[d](d/2) \]
Therefore,
\[ Rf_* IC_{X_0} \subset D_{\ge 0}(Y_0) \cap D_{\le 0}(Y_0) \]
pure of weight $0$ then we conclude by Gabber's theorem. 
\end{proof}

\begin{prop}
For $A_0 \in D^b_{\le w}(X_0)$ and $B_0 \in D^b_{\ge w'}(X_0)$ is $\tau$-mixed. Then,
\begin{enumerate}
\item if $w' > w + 1$ then $\Hom{D_c^b}{A_0}{B_0} = 0$
\item if $w' > x$ then $\Hom{D^b_c}{A_0}{B_0} \to \Hom{D^b_c}{A}{B}^F = 0$
\end{enumerate}
\end{prop}

\begin{proof}
Observe that,
\[ \Hom{D^b_c}{A_0}{B_0} \cong H^0(X_0, \RHom{}{A_0}{B_0}) \]
Call this RHom $\F_0^\bullet \in D^b_c(X_)$. Then for $\pi : X_0 \to \Spec{k_0}$ we use the Leray spectral sequence,
\[ R \Gamma_{X_0} = R \Gamma_{k_0} \circ R \pi_* \]
Therefore,
\[ E_2^{p,q} = H^p(\Gal{k/k_0}, (R^q \pi_* \F_0)_{\bar{x}}) \implies H^{p+q}(X_0, \F_0) \]
Furthermore, $(R^q \pi_* \F)_{\bar{x}} = H^q(X, \F)$ by flat base change or something. Also,
\[ \Gal{k/k_0} \cong \hat{\Z} \]
and so the only interesting Galois cohomology lives in degree $0$ and $1$. Therefore the spectral sequence has two columns so the spectral sequence has degenerated and therefore the result is filtered by these entries so we get an exact sequence,
\begin{center}
\begin{tikzcd}
0 \arrow[r] & H^{-1}(X, \F)_F \arrow[r] & \Hom{D^b_c}{A_0}{B_0} \arrow[r] & H^0(X, \F) \arrow[r] & 0
\end{tikzcd}
\end{center}
Now what about the weights of $\F_0$. I claim that $\F_0 \in D^b_{\ge w' - w}(X_0)$. By double duality,
\[ \RHom{A_0}{B_0} \cong \RHom{}{A_0}{\RHom{}{D B_0}{K_X}} \cong D(A_0 \otimes^{\mathbb{L}} D B_0) \]
and thus $D \F \cong^{\mathbb{L}} A_0 \otimes D B_0$ haing weights less that $w$ and $-w'$ repspectively so $w(D \F_0) \le w - w'$.  Then $R \pi_*$ preserves $D_{\ge w' - w}$ and therefore, $F \acts H^0(X, \F)$ has norm of eigenvalues at least $q^{\frac{1}{2}(w'-w)} > 1$. Therefore $F \acts H^{-1}(X, \F)$ has norm of its eigenvalues at least $q^{\frac{1}{2}(w' - w - 1)} > 1$ if $w' - w > 1$.
\end{proof}

\begin{cor}
If $K_0 \in D_{\text{mixed}}(X_)$ and $K_0 \not\cong 0$ then $w(D K_0) \ge - w(K_)$ if $K_0 \in D_{\ge w}(X_0)$ then $w(K_0) \ge w$.
\end{cor}

\begin{proof}
Consider $\id_{K_0} \mapsto \id_K$ from the map
\[ \Hom{D^b_c}{K_0}{K_0} \to \Hom{D^b_c}{K}{K} \]
Then,
\[ K_0 \in D_{\ge -w(DK_0)}(X_0) \]
and
\[ K_0 \in D_{\le w(K_0)}(X_0) \]
then if $w(D K_0) < - w(K_0)$ then
\[ - w(D K_0) > w(K_0) \]
by the previous proposition we would see that the right hand side of the map is zero (at least after taking Frob invariants) and thus $\id_K = 0$ which would mean that $K \cong 0$. (WHY IS THIS A CONTRADICTION?!)
\end{proof}

\section{Oct. 19}

Recall $B_0 \in D_{\text{mixed}}(X_0)$ then,
\[ \omega(B_0) := \max_{\nu} \left( \omega(\cH^\nu(B_0)) - \nu \right) \]
And $D_{\le w}(X_0)$ if 
\bigskip\\
Then $B_0$ is pure of weight $w$ if $B_0 \in D_{\le w}(X_0) \cap D_{\ge w}(X_0)$.

\begin{lemma}[Semi-continuity of weights]
Let $j : U_0 \embed X_0$ be an open dense, $\iota : Y_0 \embed X_0$ the closed complement. Let $\overline{B}_0 \in \Perv{X_0}$ be $\tau$-mixed such that $j^*(\overline{B}_0) = B_0$ and assume that $H_p^0(\iota^* (\overline{B}_0)) = 0$ then
\[ w(\overline{B}_0) = w(B_0) \]
\end{lemma}

\begin{proof}
The inequality $w(B_0) \le w(\overline{B}_0)$ is obvious so we just need to show the other direction. We apply the Fourier transform,
\begin{center}
\begin{tikzcd}
& \arrow[ld, "p_{13}"] \arrow[rd, "p_{23}"] \A_0^r \times \A_0^r \times \A_0^s \arrow[rr, "m_{12}"] & & \A_0^1
\\
\A_0^r \times \A_0^s & &  \A_0^r \times \A_0^s
\end{tikzcd}
\end{center}
For $B \in D_c^b(\A_0^r \times \A_0^s)$ then,
\[ T_\psi(B) = R (p_{13})_! (p_{23}^* B \otimes^{\mathbb{L}} m_{12}^* \L_0(\psi))[r] \]
Then the followin hold,
\begin{enumerate}
\item $T_{\psi^\top} T_{\psi} B = B(-r)$
\item If $B$ is $\tau$-mixed then,
\[ w(B) \le w \iff w(T_\psi B) \le w + r \]
\item $T_\psi : \Perv{\A_0^r \otimes \A_0^s} \to \Perv{\A_0^r \times \A_0^s}$ is an equivalence of categories.
\end{enumerate}
Now we do the proof.
\end{proof}

\section{Oct. 26}

Let $X_0$ be finite type scheme over $\FF_q$.

\begin{defn}
For $K_0 \in D_c^b(X_0, \Qbar_\ell)$ define the function $f^{K_0} : X_0(k) \to \CC$ via,
\[ f^{K_0}(x) = \tau \sum_{\nu} (-1)^\nu \tr{F_x | \mathcal{H}^\nu(K_0)} \] 
\end{defn}

\begin{thm}
The dictionary:
\begin{enumerate}
\item $K_0 \to L_0 \to M_0$ a distingulished triangle in $D^b_c(X_0, \Qbar_\ell)$ then,
\[ f^{L_0} = f^{K_0} + f^{M_0} \]
\item for $K_0, L_0 \in D^b_c(X_0, \Qbar_\ell)$ then,
\[ f^{K_0 \otimes^{\mathbb{L}} L_0} = f^{K_0} \cdot f^{L_0} \]
\item Let $K_0, L_0$ be semisimple perverse sheaves on $X_0$ the following are equivalent,
\begin{enumerate}
\item $f_m^{K_0}(x) = f^{L_0}_m(x)$ for all $m \in \Z^+$ for all $x \in X_0(\F_{q^m})$.
\item WWWWWW
\end{enumerate}
\item WWWWW

\item $f^{K_0[d](n)}(x) = (-1)^d g^{-n} f^{K_0}(x)$ 

\item Let $K_0 \in D_{\text{mixed}}(X_0, \Qbar)$ be $\tau$-pure of weight $w$ then $f^{DK_0}(x) = q^{-w} \overline{f^{K_0}(x)}$
\end{enumerate}
\end{thm}


\begin{proof}
The first follows from the long exact sequence. The tensor one follows from,
\[ (K \otimes^{\mathbb{L}} L)_{\bar{x}} = K_{\bar{x}} \otimes^{\mathbb{L}} L_{\bar{x}} \]
\bigskip\\
Constructible $\Qbar_\ell$-sheaf is smooth on a dense open. Assume $X_0$ is reduced, find $U_0$ smooth dense open in $X_0$ over which $\H^\nu(K_0)$ and $\H^\nu(L_0)$ are all smooth. Then $j : U_0 \embed X_0$ with complement $\iota : Z_0 \embed X_0$. Then $K_0 = j_{!*} F_0 \oplus \iota_* \F_0'$ and $L_0 = j_{!*} G_0 \oplus \iota_* G_0'$ so,
\[ f^{K_0}_m(x) = f_m^{j_{!*} F_0}(x) = f_m^{F_0}(j(x)) \]
for $x \in U_0(\FF_{q^m})$ because,
\[ j^* (j_{!*} F_0) = F_0 \]
Therefore,
\[ \forall x \in U_0(\FF_{q^m} : f_m^{F_0}(x) = f_m^{G_0}(x) \]
Then we apply Cebotarev densitiy for smooth variety over $\FF_q$ to conconlude that,
\[ \{ F_{\bar{x}} : x \in |X_0| \} \subset \pi_1(U, \bar{x_0}) \]
is dense. Therefore $F_0 \cong G_0$. 
\end{proof}


\end{document}
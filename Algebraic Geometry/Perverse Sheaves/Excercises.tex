\documentclass[12pt]{article}
\usepackage{import}
\import{../}{AlgGeoCommands}

\begin{document}

\section{Week 1}

\subsection{Bhatt (Problems 1)}

\subsubsection*{2}

Let $\cA$ be an abelian category. Any category admits all colimits iff it admits coequalizers and all coproducts (easy exercise). Since $\cA$ is abelian it admits cokernels and therefore coequalizers and thus $\cA$ admits all colimits iff it admits all direct sums (coproducts).

\subsubsection*{3}

\begin{enumerate}
\item the category of finite $k$-vectorspaces has finite direct sums nut not countable direct sums. Likewise for countably generated vector spaces.

\item The opposite category of torsion abelian groups.

\item 
\end{enumerate}

\subsubsection*{4}

Let $\C$ be the category of torsion abelian groups. It is clear that $\C$ is abelian as kernels and cokernels of torsion groups are torsion since subgroups and quotients are torsion. Furthermore all direct sums exist in $\C$ because elements are zero all but finitely often and thus torsion since the nonzero entries are torsion.
\bigskip\\
Because $\C$ has cokernels and all coproducts it has all colimits. Furthermore, filtered colimits in $\Ab$ are exact so they are exact in $\C$ as well. For a generator, consider,
\[ X = \bigoplus_{n \in \Z^+} \Z / n \Z \]
Then for each element $a \in A$ for $A$ a torsion abelian group we get a map $X \to A$ whose image contains $a$ sending $\Z / n \Z \to 0$ unless $n$ is the order of $a$ in which case $1 \mapsto a$. Therefore we get a surjection,
\[ X^{\oplus A} \onto A \]

\subsubsection*{5}

Let $\cA$ be Grothendieck abelian and $I$ a category.  Let $\C = \mathrm{Fun}(I, \cA)$ be the functor category. Clearly, $\C$ is additive and admits kernels, cokernel, and infinite direct sums (constructed pointwise). 

\subsubsection*{7}

Let $\cA$ be an abelian category. Now $\Ch{\cA}$ is the subcategory of functors from $\Z$ as a poset to $\cA$ such that the composition of sucessive maps is zero. (CAN WE REDUCE THIS TO PREVIOUS EXERCISE?)

\subsubsection*{11}

Let $\cA$ be an abelian category and $\cA^{\N} = \Hom{}{\N^{\op}}{\cA}$ the category of projective systems. Assume that $\cA$ admits infinite direct sums and products.

\begin{enumerate}
\item 
Taking limits is right adjoint to the constant diagram functor $\Delta : \cA \to \cA^{\N}$ defined via $A \mapsto (n \mapsto A)$ with identity transition maps. Therefore $\lim : \cA^{\N} \to \cA$ preserves limits and thus is, in particular, left exact.

\item Note that given a projective system $\{ X_n \} \in \cA^\N$,
\[ \lim X_n = \ker{\left( \prod_{n \in \N} X_n \to \prod_{n \in \N} X_n \right)} \]
where on the $n^{\mathrm{th}}$ factor the map is the difference of projection $\prod X_{n'} \to X_n$ and $f_n \circ (\prod X_{n'} \to X_{n+1})$ where $f_n : X_{n+1} \to X_n$ is the transition map. (FINISH THIS)

\item 
\end{enumerate}

\subsubsection*{12}

Fairly obvious.

\subsubsection*{13}


Let $\cA$ be an abelian category and $f : K^\bullet \to L^\bullet$ be a map in $\Ch{\cA}$. Recall that,
\[ C(f) = K[1] \oplus L \]
where the differential is,
\[ \d_{C(f)} = \begin{pmatrix}
\d_{K[1]} & 0
\\
f[1] & \d_L
\end{pmatrix} \]
Specifically, $C(f)^i = K^{i+1} \oplus L^i$ and $\d(x,y) = (-\d{x}, \d{y} + f(x))$.

\begin{enumerate}
\item Let $A^\bullet \in \Ch{\cA}$ be a complex. Consider,
\[ g \in \Hom{\Ch{\cA}}{C(f)}{A^\bullet} \]
Then $g^i = (k^{i}, h^i)$ where $k^i : K^{i+1} \to A^i$ and $h^i : L^i \to A^i$ which satisfy,
\[ g^{i+1} \circ \d_C^i = \d_A^{i+1} \circ g^i \]
Explicitly,
\[ -k^{i+1} \circ \d_K^{i+1}(x) +  h^{i+1} \circ \d_L^i(y) + h^{i+1} \circ f^{i+1}(x) = \d_A^{i+1} \circ (k^i(x) + h^i(y)) \]
Setting $x = 0$ we find that,
\[ h^{i+1} \circ \d_L^i(y) = \d_A^{i+1} \circ h^i(y) \]
and therefore $h \in \Hom{\Ch{\cA}}{L^\bullet}{A^\bullet}$. 
Setting $y = 0$ we find that,
\[  h^{i+1} \circ f^{i+1}(x) = \d_A^{i+1} \circ k^i(x) + k^{i+1} \circ \d_K^{i+1}(x) \]
therefore $k$ is a nullhomotopy of $h \circ f$ so we see that,
\[ \Hom{\Ch{\cA}}{C(f)}{-} = \{ k : L^\bullet \to A^\bullet \text{ and } h : K^{\bullet + 1} \to A^\bullet \mid h \text{ is a nullhomotopy of } k \circ f \} \]

\item If $L$ is acyclic then from the long exact sequence for the exact triangle,
\[ K \xrightarrow{f} L \to C(f) \to K[1] \]
shows that $H^i(L) \to H^i(C(f))$ is an isomorphism.

\item 
\end{enumerate}

\subsection{Bhatt Lectures}

\subsubsection*{2.4}

Let $\C$ be a category such that $\C$ is enriched over $\Ab$ with finite coproducts. Given $f, g : A \to B$ there exists a map $f + g : A \to B$. To show that being abelian is a property, we must describe $f + g$ in terms of internal properties of the category. That is, there is a unique additive structure on any additive category.
\bigskip\\
Consider the map $A \to A \oplus A \to B$ defined by,
\[ (f, g) \circ (\iota_1 + \iota_2) = (f, g) \circ \iota_1 + (f, g) \circ \iota_2 = f + g \]
Therefore, it suffices to show that $h = \iota_1 + \iota_2$ is internal to the category. There are zero maps $A \to 0$ (where $0$ is the initial object) b/c $\Hom{\C}{A}{0}$ has an identity. Then $(\id, 0) \circ h = \id + 0 = \id$ and $(0, \id) \circ h = \id$. Call $\pi_1 = (\id, 0)$ and $\pi_2 = (0, \id)$ then these make $A \oplus A$ a product and $h$ the diagonal so $h$ is unique.
\bigskip\\
To prove this consider $a : C \to A$ and $b : C \to B$ then $q = \iota_1 \circ a + \iota_2 \circ b$ satisfies $\pi_1 \circ q = a$ and $\pi_2 \circ q = b$. Furthermore, let $q' : C \to A \oplus B$ be any map with this property. Then $q' = (\iota_1 \circ \pi_1 + \iota_2 \circ \pi_2) \circ q' = \iota_1 \circ a + \iota_2 \circ b$ becuase,
\[ (\iota_1 \circ \pi_1 + \iota_2 \circ \pi_2) \circ \iota_i = \iota_i + 0 = \iota_i \]
and thus $(\iota_1 \circ \pi_1 + \iota_2 \circ \pi_2) = \id$ because $A \oplus B$ is a coproduct.  
\bigskip\\
Notice that this construction only relied on the choice of a zero map $A \to 0$. However, the identity of $0 : \Hom{\C}{0}{0}$ must be $\id_0 : 0 \to 0$ because $0$ is initial so this set has a unique element. Therefore, for any $f : A \to 0$ we have $f = \id_0 \circ f = 0 \in \Hom{\C}{A}{0}$ because $\id_0$ is the identity of the group and $- \circ f : \Hom{\C}{0}{0} \to \Hom{\C}{A}{0}$ is a group map. Therefore, $\Hom{\C}{A}{0}$ has a single element so there is no choice of zero map $A \to 0$.
\bigskip\\
Since there is a unique map $A \to 0$ we see that $0$ is initial and final. 


\subsubsection*{2.11}

Solved in Bhatt problems 2,3.

\subsubsection*{2.20}

\begin{enumerate}
\item Bhatt problems 7

\item Let $\cA$ be a Grothendieck abelian category. We construct the injective resolution inductively. First, $X \to (X \embed I(X))$ is functorial. Assume there is a functorial assignment,
\[ X \mapsto (X \embed I^0(X) \to I^1(X) \to \cdots \to I^n(X)) \]
Then consider 
\[ \coker{(I^{n-1}(X) \to I^n(X))} \embed I(\coker{(I^{n-1}(X) \to I^n(X))}) = I^{n+1}(X) \]
which is funtorial in $X$ because cokernels and $C \mapsto I(C)$ is thus giving,
\[ X \mapsto (X \embed I^0(X) \to I^1(X) \to \cdots \to I^{n+1}(X)) \]
\end{enumerate}


\section{Week 2}

\subsection{Bhatt (Problems 1)}

\subsubsection*{15}

\begin{enumerate}
\item a
\end{enumerate}

\subsection{Bhatt (Problems 2)}


\subsubsection*{6}


\subsubsection*{7}


\subsubsection*{8}


\subsubsection*{9}

\subsection{Bhatt (Lectures)}

\subsubsection*{2.25}

\subsubsection*{6.12}

\subsubsection*{6.13}

\subsection{Tsai (Problems)}

\subsubsection*{1}

\subsubsection*{2}

\subsubsection*{3}

\section{Week 4}

\subsection{Bhatt (Problems 3)}

\subsubsection*{1}

\newcommand{\heart}{\ensuremath\heartsuit}

Let $\D$ be a triangulated category equiped with a $t$-structure. Let $X, Y \in \D^{\heart}$. Recall that,
\[ \Ext{-n}{\D}{X}{Y} = \Hom{\D}{X}{Y[-n]} \]
Suppose that $n > 0$, since $Y \in \D^{\ge 0}$ we see that $Y[-n] \in \D^{\ge n} \subset \D^{\ge 1}$ and furthermore $X \in \D^{\le 0}$ and therefore,
\[ \Ext{-n}{\D}{X}{Y} = \Hom{\D}{X}{Y[-n]} = 0 \]
when $n > 0$.

\subsubsection*{2}

Let $X$ be a topological space and $K \in D(X)$,

\begin{enumerate}
\item Consider $X = \P^1$ and $K = \struct{X} \oplus \struct{X}(-2)[1]$ Then we consider,
\begin{align*}
\Hom{D(X)}{K|_U}{K|_U} & = \Hom{D(X)}{\struct{U}}{\struct{U}} \oplus \Hom{D(X)}{\struct{U}}{\struct{U}(-2)[1]}
\\
& \oplus \Hom{D(X)}{\struct{U}(-2)[1]}{\struct{U}} \oplus \Hom{D(X)}{\struct{U}(-2)[1]}{\struct{U}(-2)[1]}
\\
& = \Gamma(U, \struct{U}) \oplus H^1(U, \struct{U}(-2)) \oplus \Gamma(U, \struct{U})
\end{align*} 
which is not a sheaf because of the $H^1(U, \struct{U}(-2))$ term. We use,
\[ \Hom{D(X)}{\struct{U}}{\struct{U}(-2)[1]} = \Ext{1}{D(X)}{\struct{U}}{\struct{U}(-2)} = H^1(U, \struct{U}(-2)) \]
\item Suppose that $\Ext{i}{K|_U}{K|_U} = 0$ for all $i < 0$ and open $U \subset X$.

\begin{lemma}
If the cohomology sheaves $H^i(K) = 0$ for all $i < d$ then $U \mapsto \mathbb{H}^d(U, K)$ is a sheaf.
\end{lemma}
 
\begin{proof}
$K \cong \tau^{\ge d} K$ is an equivalent so we may assume $K$ is zero in deg $< d$. Then choose a quis $K \iso I$ for an injective resolution. Then,
\[ \mathbb{H}^d(U, K) = \ker{(I^d(U) \to I^{d+1}(U))} \]
and therefore $H^d(-, K) = \ker{(I^d \to I^{d+1})}$ is a sheaf.
\end{proof} 
\noindent
Let $L,K$ be complexes. Assume that $\Ext{i}{D(X)}{L|_U}{K|_U} = 0$ for $i < 0$ and $U \subset X$ open. Now $H^i(\mathrm{RHom}(L,K))$ is the sheafification of,
\[ U \mapsto \Ext{i}{D(U)}{L|_U}{K|_U} \]

\item 
\end{enumerate}

\subsubsection*{3}

Let $\cA$ be an abelian category with enough projectives. Assume that $\Ext{2}{\cA}{X}{Y} = 0$ for all $X, Y \in \cA$.

\begin{enumerate}
\item Let $K \in D^b(\cA)$. Choose a projective resolution $P \to K$ 

\item 
\end{enumerate}

\subsubsection*{4}

Let $D^b_f(k)$ be the derived category of bounded complexes of $k$-vectorspaces with finitely generated cohomology.

\begin{lemma}
A t-structure is determined by $\D^{\le 0}$.
\end{lemma}

\begin{proof}
We can recover 
\[ \D^{\ge 1} = \{K \in \D \mid \Hom{\D}{\D^{\le 0}}{X} \} \]
\end{proof}

\subsubsection*{5}

Let $\D$ be a triangulared category with a $t$-structure. Let $K \in \D$ be a direct summand of $L \in \D^{\le 0}$. Consider,
\[ L = K \oplus F \]
We know $\tau^{\ge 1} (K \oplus F) = 0$ but $\tau^{\ge 1}$ is a left adjoint and thus preserves colimits so $\tau^{\ge 1} (K) \oplus \tau^{\ge 1}(F) = 0$ therefore $\tau^{\ge 1}(K) = 0$ so $K \in \D^{\le 0}$. 

\subsubsection*{6}


\section{Week 4}

(DOOOOO THHHHIIIIIS!!!!!)

\section{Week 7}

\subsection{Tsai (Problems 1)}

\subsubsection*{4}

\subsubsection*{5}

\subsubsection*{6}

\subsection{Tsai (Problems 2)}

\subsubsection*{1}

Let $E \subset \P^2$ be an elliptic curve. Let $V \subset \A^3$ be the corresponding affine cover over $E$. Let $o$ be the origin and $U := V \setminus \{ o \}$ be the smooth locus of $V$. Wrtie $\iota : \{ o \} \embed V$ and $j : U \embed V$ the embeddings. We want to compute $\iota^* R j_* \underline{\Q}_U$.
\bigskip\\
(DO THIISSS!!)

\subsubsection*{2}

As in the last problem we want to show that $D_V \underline{\Q}_V[2] \not\cong \underline{\Q}_V[2]$.
\bigskip\\
(DO THIS!!!!)

\subsubsection*{3}

As in the last problem, we want to show that,
\[ R\Gamma(V, \tau_{\le -1} R j_* \underline{\Q}_U[2])) \]
is dual to
\[ R \Gamma_c(V, \tau_{\le -1}(Rj_* \underline{\Q}_U[2])) \]

\subsubsection*{4}

Let $X$ be a smooth complete variety over $\mathbb{\C}$ and let $x \in X$ be a fixed point. Let $\iota_x : \{ x \} \embed X$ be the inclusion. We want to compute $\F = \mathrm{RHom}{}{\iota_x \underline{\Q}}{\underline{\Q}_X}$.

\end{document}
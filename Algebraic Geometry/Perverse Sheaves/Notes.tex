\documentclass[12pt]{article}
\usepackage{import}
\import{../}{AlgGeoCommands}

\newcommand{\heart}{\ensuremath\varheartsuit}

\begin{document}

\section{Sep. 21}

Interested in studying perverse sheaves in the Euclidean and \etale topologies for studying Poincare duality for singular varieties. We studied:
\begin{enumerate}
\item tiangulated categories
\item t-structures
\item gluing t-structues $\D_Z \to \D \to \D_U$ with nice functors between these triangulated categories
\item perverse sheaves over $\C$
\end{enumerate}
\noindent
We are now going to develop the theory of perverse sheaves in the \etale setting for \etale sheaves and \etale cohomology. The main goal for the first two weeks is to define the derived categories for $\overline{\Q}_\ell$-sheaves and six functor formalism. Then we study the theory of weights due to Deligne (Weil II) for varieties over $\FF_q$.

\subsection{Etale Sheaves}

Assume all schemes are Noetherian. Then $X_{\et}$ is the small \etale site with objects: schemes \etale over $X$ and covers and jointly surjective families of \etale maps. 

\subsubsection{Constructibility}

Let $\Lambda$ be a finite ``coefficient'' ring. We say that an \etale sheaf of sets $\F$ on $X_{\et}$ is locally constant constructible (lcc) if it is represented by an object of $\mathrm{FEt}(X)$ i.e. there is a finite \etale map $X' \to X$ such that,
\[ \F(U) = \Hom{X}{U}{X'} \]
We say that a sheaf $\F$ is $\Lambda$-modules is \textit{constructible} if each $\F_{x}$ is a finite $\Lambda$-module and there exists a stratification,
\[ X = \bigcup X_i \]
such that $X_i \subset X$ is locally closed s.t. $\F|_{X_I}$ is lcc. Furthermore, $\F \in D^b(X, \Lambda)$ is constructible if each $H^i(\F)$ is constructible. This gives the triangulated subcategory $D_c^i(X, \Lambda)$.
\bigskip\\
Now given a map of Noetherian schemes $f : X \to Y$ we get a six functor formalism. We have $f_* : \mathrm{Ab}(X) \to \mathrm{Ab}(Y)$ and $R f_* : D^+(x, \Lambda) \to D^+(Y, \Lambda)$ and $f^* : \mathrm{Ab}(Y) \to \mathrm{Ab}(X)$ which is exact. Then there is an adjointness,
\[ \Hom{X}{f^* \F}{\G} = \Hom{Y}{\F}{f_* \G} \]
When $\iota : Z \embed X$ is a closed immersion and $j : U \embed X$ is the complementary  open immersion we can define the ``exceptional'' functors. We define,
\[ j_! : \mathrm{Ab}(Y) \to \mathrm{Ab}(X) \]
which is exact because it preserves stalks. Then $\iota_*$ is also exact because it preserves stalks (and extends by zero on $U$). Furthermore, 
\[ \iota^! : \mathrm{Ab}(X) \to \mathrm{Ab}(Z) \]
is defined by,
\[ \iota^! \F = \iota^* \ker{(\F \to j_* j^* \F)} \]
which is the subsheaf of sections supported on $Z$. 
\bigskip\\
However, $R f_!$ is not the naive derived functor of $f_!$ unfortunately. Assume that $f : X \to Y$ is separated of finite type. Then by Nagata, there is a compactification,
\begin{center}
\begin{tikzcd}
X \arrow[rd, "f"] \arrow[r, hook] & \overline{X} \arrow[d, "\bar{f}"]
\\
& Y
\end{tikzcd}
\end{center}
Then we define $R f_! = D^+(X, \Lambda) \to D^+(Y, \Lambda)$ by $R f_! = (R \bar{f}_*) \circ j_!$. 

\begin{rmk}
It does not work to define $R f_! = R(f_!)$ where $f_! = \bar{f}_* \circ j_!$. For example for a curve $X$ over a field $k$ and $f : X \to \Spec{k}$. Then,
\[ \Gamma_c(X, \Lambda) = \bigoplus_{x \in X} \Gamma_x(X, \F) \]
where $\Gamma_x$ is local cohomology. But this derived functor can be too large.
\end{rmk}

Then we can also define $R f^! : D^+(Y, \Lambda) \to D^+(X, \Lambda)$ to be the right adjoint of $R f_!$. For example if $f = \iota$ is a closed immersion then $R f^! = R(\iota^!)$ is actually a derived functor. If $f$ is a smooth map of relative dimension $d$ (and $n \Lambda = 0$ for $n \in \struct{Y}^\times)$ then $R f^! = f^*(\alpha) [2d]$ 

\newcommand{\RHom}[3]{\mathrm{RHom}_{#1}\left( #2, #3 \right)}
\newcommand{\dtimes}{\otimes^{\mathbb{L}}}

\begin{theorem}
Let $f : X \to Y$ be finite tpye over a field $k$. Then the six functors presserve $D_c^+(X, \Lambda)$ and we have biduality when $n \Lambda = 0$ for $n \in k^\times$ and $\Lambda$ is an injective $\Lambda$-module. Let $f : X \to \Spec{k}$ and set $K_X = R f^! \Lambda$ and $DL := \RHom{}{L}{K_X} \in D^b(X, \Lambda)$. Then $L \iso DDL$ in $D_c^b(X, \Lambda)$.
\end{theorem}
\noindent
How to define $D_c^b(X, \overline{\Q}_\ell)$? Let $k$ be a finite field or a separably closed field $\ell \in k^\times$. Let $X$ be seperated of finite type over $k$. Fact 1 for $E / \Q_\ell$ finite extension then $\struct{E}$ have triangulated cat $D^b_c(X, \struct{E})$ and standard nontrivial t-structure. For $r \ge 1$ let $\struct{r} = \struct{E} / \lambda^r$ where $\lambda$ is a uniformizer. Then $\struct{r}$ is constructible.
\bigskip\\
Let $D^b_{ctf}(X, \struct{r}) \subset D^b_c(X, \struct{r})$ be defined by objects isomorphic to bounded complexes of flat $\struct{r}$-module that give $D_c^b(X, \struct{E})$ is defined as $K = (K_r)_r$ where $K_r \in D^b_{ctf}(X, \struct{r})$ and $K_{r+1} \dtimes \struct{r} \cong K_r$. Moreover,
\[ \Hom{D^b_c(X, \struct{E})}{K}{L} = \varprojlim_V \Hom{D^b_c(X, \struct{r})}{K_r}{L_r} \]
Furthermore, $D^b_c(X, \struct{E})$ is a triangulated category. Furthermore, there is a standard t-structure on $D^b_c(X, \struct{E})$. The problem is that $\tau_{\le n}$ and $\tau_{\ge n}$ dont preserve $D^b_{cft}(X, \struct{r})$ but Deligne defined them in such a way to preserve everything you want.

\begin{rmk}
There are some serious problems with trying to first define an abelian category of $\Z_\ell$-sheaves and then take derived categories. For example, if $X = \Spec{k}$ then sheaves of $\Z / \ell^n$-modules is the same as $\Z/\ell^n[\Gal{\bar{k}/k}]$-modules but the limit of these categories gives the category of continuous $\Gal{\bar{k}/k}$-representations over $\Z_\ell$ which is not an abelian category. Look at Bhatt-Scholtze and condensed mathematics.
\end{rmk}

\subsection{Theory of Weights}

Let $X_0$ be a separated scheme of finite type over $\FF_q$ and let $X = X_0 \otimes \overline{\FF_q}$. Then $F : X \to X$ is the geometric Frobenius i.e. $F = \id \otimes \Spec{F}$ where $F : \FF_q \to \FF_q$ is the inverse of $x \mapsto x^p$. Let $|X_0|$ denote the closed points and $x \in | X_0 |$ we have $\# \kappa(x) = N(x) = q^{d(x)}$. For $X_0$ geometrically connected over $\FF_q$ we have an exact sequence,
\begin{center}
\begin{tikzcd}
1 \arrow[r] & \pi_1^\et(X, \bar{x}) \arrow[r] & \pi_1^\et(X_0, \bar{x}) \arrow[r] & \Gal{\overline{\FF_q}/\FF_q} \arrow[r] & 1
\end{tikzcd}
\end{center}
and $\Gal{\overline{\FF_q}/\FF_q} = \hat{\Z}$ generated by geometric Frobenius. Therefore, we have an exact squence,
\begin{center}
\begin{tikzcd}
1 \arrow[r] & \pi_1^\et(X, \bar{x}) \arrow[r] & \pi_1^\et(X_0, \bar{x}) \arrow[r] & \Gal{\overline{\FF_q}/\FF_q} \arrow[r] & 1
\\
1 \arrow[r] & \pi_1^\et(X, \bar{x}) \arrow[r] \arrow[u, equals] & W(X_0, \bar{x}) \arrow[u, hook] \arrow[r] & \Z \arrow[u, hook] \arrow[r] & 1
\end{tikzcd}
\end{center}

\begin{defn}
A Weil sheaf on $X_0$ is a $\overline{\Q}_\ell$-sheaf $\F$ on $X$ with an isomorphism $\varphi : F^* \F \to \F$. 
\end{defn}

\begin{rmk}
Let $g : X \to X_0$ be the canonical map. Then $g \circ F = g$ so there is a natural isomorphism $\eta : F^* \circ g^* \to g^*$ and thus for any $\overline{\Q}_\ell$-sheaf $\F_0$ on $X_0$ we have an natural isomorphism $\eta : F^* g^* \F_0 \iso g^* \F_0$ and thus $\F := g^* \F_0$ and $\eta : F^* \F \to \F$ gives a Weil sheaf.
\end{rmk}

\begin{exercise}
For $X_0 = \Spec{\FF_q}$ we have,
\[ \{ \text{Weil Sheaves} \} \iff \{ \text{continuous reps of } W(X_0) \cong \Z \text{ on finite dimensional } \overline{\Q}_\ell \text{ vector spaces} \} \]
furthermore, the subcategory of $\overline{\Q}_\ell$-sheaves correspond to those representations that extend to $\hat{Z}$.
\end{exercise}

\begin{exercise}
In the rank $1$ case let $X_0$ be a normal geometrically connected / $\FF_q$ then,
\[ \Im{\pi_1(X, \bar{x}) \to W(X_0, \bar{x})^\et} \]
is a finite group times a pro-$p$ group. 
\end{exercise}

\subsection{Weights}

Fix an isomorphism $\iota : \overline{\Q}_\ell \iso \C$. 

\begin{defn}
Let $\F_0$ be a Weil sheaf on $X_0$ and $\beta \in \RR$ then,
\begin{enumerate}
\item $\F_0$ is (pointwise) $\iota$-pure of weight $\beta$ if $\forall x \in |X_0|$ and $\alpha$ is an eigenvalue of $F_x \acts (\F_0)_{\bar{x}}$ then $|\iota \alpha| = q^{\frac{\beta d(x)}{2}}$. 
\item $\F_0$ is $\iota$-mixed if there is a filtration,
\[ 0 = \F_0^{(0)} \subset \cdots \subset \F_0^{(r)} = \F_0 \]
by Weil subsheaves s.t. $\F_0^{(i+1)} / \F_0^{(i)}$ is $\iota$-pure of some weight
\item $\F_0$ is pure of weight $\beta$ / mixed if it is $\iota$-pure of weight $\beta$ / $\iota$-mixed for any $\iota$.
\end{enumerate}
\end{defn}

\begin{example}
The sheaf $\F_0 = \Q(1)$ is pure of weight $-2$ since $F \zeta_{\ell^n} = \zeta_{\ell^n}^{q^{-1}}$. If $X_0$ is normal geometrically connected then any rank $1$ smooth Weil sheaf on $X_0$ is $\iota$-pure.
\end{example}

\begin{thm}[Deligne]
For $f_0 : X_0 \to Y_0$ and $\F_0$ on $X$ a $\iota$-mixed of largest weight $\beta$ then,
\[ R^k (f_0)_* \F_0 \]
is $\iota$-mixed with weights $\le \beta + k$. 
\end{thm}

\begin{example}
For $f_0 : X_0 \to \Spec{\FF_q}$ smooth and proper of dimension $d$ and $\F_0$ a smooth sheaf on $X_0$ that is $\iota$-pure of weight $\beta$ then by Poincare duality,
\[ F \acts H^k(X_{\et}, \F) \cong H^{2d - k}(X_{\et}, \F^\vee(d))^\vee \]
By Deligne's theorem, the left hand side has weights $\le \beta + k$. Furthermore, the sheaf $\F^\vee(d)$ is $\iota$-pure of weight $-\beta - 2d$ and thus by Deligne's theorem the cohomology group has weights $\le - \beta - k$ and thus dualizing the weights are $\ge \beta + k$. Therefore, the weights are all $\beta + k$ because both sides are isomorphic..
\end{example}

\begin{theorem}[semi-continuity]
Let $j_0 : U_0 \to X_0$ be a dense open immersion with $S_0 = X_0 \setminus U_0$ with the reduced scheme structure. Let $\F_0$ be a smooth Weil sheaf on $X_0$. Assume that there is $\beta \in \RR$ such that $\forall x \in |U_0|$ and any eigenvalue $\alpha$ of $f_x \acts \F_{\bar{x}}$ then $|\iota \alpha| \le q^{\frac{\beta d(x)}{2}}$ then $\forall s \in |S_0|$ and any eigenvalue $\alpha$ of $F_s \acts \F_{\bar{s}}$ then $|\iota \alpha| \le q^{\frac{\beta d(s)}{2}}$.  
\end{theorem}

\begin{proof}
We can always take a chain of curves connecting $x$ and $s$ so we reduce to the case $\dim{X_0} = 1$ with $X_0$ geometrically irreducible and affine. Recall,
\[ L(X_0, \F_0, t) = \prod_{x \in |X_0|} \det{(1 - F_x t^{d(x)} | \F_{\bar{x}})}^{-1} = \frac{\det{(1 - F t | H^1_c(X, \F))}}{\det{(1 - F t | H^0_c)} \det{1 - Ft | H^2_c)}} \]
Then, $H^0_c(X, \F) = 0$ because $X$ is affine and $\F_0$ is smooth so there are no global sections with compact support. Furthermore,
\[ H^2_c(X, \F) \cong H^0(X, \F^\vee(1))^\vee \] by Poincare duality. Fix $x \in |U_0|$ and because $\F_0$ is lisse then it corresponds to some representation $V$ of $W(X_0, \bar{x})$ on $\F_{\bar{x}}$. Then, $H^0(X, \F) = V^{\pi_1(X)}$ and likewise,
\[ H^2_c(X, \F) = H^0(X, \F^\vee)^\vee(-1) = V_{\pi_1(X)}(-1) \]
because the dual of invariants are coinvariants. Take $\alpha$ an eigenvalue of $F$ acting on $V_{\pi_1(X)}$ then $\alpha^{d(x)}$ is an eigenvalue of $F^{d(x)} = F_x$ acting on $V_{\pi_1(X)}$ which is a quotient of $\F_{\bar{x}}$ and thus $q | \iota \alpha| \le q^{\frac{\beta d(x)}{2} + 1}$ and thus $q \alpha$ is an eigenvalue of $F \acts H^2_c$ iff $(q \alpha)^{-1}$ is a zero of $\det{(1 - F t | H^2_c)}$. Therefore, the possible poles of $\iota L(X_0, \F_0, t)$ is $\iota(q \alpha)^{-1}$ for $\alpha$ as above. Therefore $\iota L(X_0, \F_0, f)$ has no poles for $|t| < q^{-\frac{\beta}{2} - 1}$. However, 
\[ L(X_0, \F_0, t) = L(U_0, \F_0 |_{U_0}, t) \cdot \prod_{s \in S_0} \det{(1 - F_s t | \F_{\bar{s}})}^{-1} \]
I claim that $\iota L(U_0, \F_0 |_{U_0}, t)$ converges and has no zeros for $|t| < q^{-\frac{\beta}{2} - 1}$. Then,
\[ \iota \left( \frac{L'(U_0)}{L(U_0)} \right) = \iota\left( \sum_{n = 1}^\infty \left( \sum_{\substack{x \in |U_0| \\ d(x) \divides n}} d(x) \iota \Tr{F_x^{\frac{n}{d(x)}}} \right) t^{n-1} \right) \]
Then, 
\[ | \iota \Tr{F_x^{\frac{n}{d(x)}}} | \le r \left( q^{\frac{\beta d(x)}{2}} \right)^{\frac{n}{d(x)}} = r q^{\frac{\beta n}{2}} \]
Therefore,
\[ \sum_{\substack{x \in |U_0| \\ d(x) \divides n}} d(x) = \# U_0(\FF_{q^n}) \le C q^n \]
and therefore the logarithmic derivative is dominated by,
\[ \sum_{n = 1}^\infty r C q^{n \left( \frac{\beta}{2} + 1 \right)} t^{n-1} \]
which converges absolutely for $|t| < q^{- \frac{\beta}{2} - 1}$ and thus $|\iota \alpha| \le q^{\beta + 1}$. 
\bigskip\\
Now we apply the same argument to $\G_0 = \F_0^{\otimes k}$ so $\alpha^k$ is an eigenvalue of $F_s \acts \G_{\bar{s}}$ and thus $| \iota \alpha^k | \le q^{\frac{k \beta}{2} + 1}$ which implies that,
\[ | \iota \alpha | \le q^{\frac{\beta}{2} + \frac{1}{k}} \]
and thus taking $k \to \infty$ this goes to $q^{\frac{\beta}{2}}$ so $| \iota \alpha| \le q^{\frac{\beta}{2}}$. 
\end{proof}

\section{Sep. 28}

\begin{thm}
Let $f_* : X_0 \to Y_0$ be separated of finite type over $\FF_q$ and $\F_0$ a Weil sheaf on $X_0$. If $\F_0$ is $\iota$-mixed of largest weight $\beta$ then for all $k$ the sheaf $R^k (f_0)_! \F_0$ is $\iota$-mixed of weight at most $\beta + k$.
\end{thm}

\begin{rmk}
Today we will prove this for $X_0 \to \Spec{\FF_q}$ where $X_0$ is a smooth curve, $\F_0$ is smooth and $\iota$-pure of weight $\beta$. Recall that purity means that for each $x \in X_0$ and $\alpha$ an eigenvalue of $F_x \acts \F_{\bar{x}}$ then $|\iota \alpha| = q^{\frac{d(x) \beta}{2}}$
\end{rmk}

\begin{thm}[Semicontinuity]
Let $j_0 : U_0 \embed X_0$ be a dense open immersion and let $S_0$ be the complement. Let $\F_0$ be a smooth Weil sheaf thus corresponding to some $\pi_1(X_0)$-representation $V$. For $\beta \in \RR$ consider the condition,
\[ (*)_x : \forall \alpha \text{ eigenvalue of } F_x \acts \F_{\bar{x}} : |\iota \alpha| \le q^{\frac{d(x) \beta}{2}} \]
If $(*)_x$ holds for $x \in |U_0|$ then $(*)_x$ holds on the boundary.
\end{thm}

\begin{cor}
With the same notation as above,
\begin{enumerate}
\item if $\F_0 |_{U_0}$ is $\iota$-pure of weight $\beta$ then $\F_0$ is $\iota$-pure of weight $\beta$
\item assume $X_0$ is normal and geometrically irreducible and $\F_0$ is irreducible (as a representation). If $\F_0 |_{U_0}$ is $\iota$-mixed then $\F_0$ is $\iota$-pure.
\end{enumerate}
\end{cor}

\begin{proof}
For (a) we apply semi-continuity to $\F_0$ and $\F_0^\vee$. Now for (b) set $\G_0 = \F_0 |_{U_0}$. Because $\G_0$ is $\iota$-mixed, there is a filtration, $\G_0^{(i)}$ such that the graded pieces are $\iota$-pure. Since each term in the filtration is constructible, by shrinking $U_0$ we can assume that each term is smooth. Since $X_0$ is normal $W(U_0) \onto W(X_0)$ so we see that $\G_0 = \F_0|_{U_0}$ is irreducible so there are no subrepresentations thus the filtration is trivial so $\G_0$ must be $\iota$-pure.
\end{proof}

\subsubsection{Two More Results that use $L$-function Techniques}

\begin{defn}
We say that $\F_0$ a Weil sheaf on $X_0$ is \textit{$\iota$-real} if for each $x \in X_0$ then,
\[ \iota \det{(1 - F_x t^{d(x)} | \F_{\bar{x}})} \in \RR[t] \]
\end{defn}

\begin{example}
If $\F_0$ is smooth and $\iota$-pure of weight $\beta$ then $\F_0 \oplus \overline{\F_0}$ is $\iota$-real where $\overline{\F_0} = \F_0^\vee \otimes \L_0(\beta)$ where $\L_0(\beta)$ corresponds to the representation $W(X_0) \to W(\FF_q) \to \overline{\Q}_\ell^\times$ via $F \mapsto q^\beta$. 
\end{example}

\begin{rmk}
Suppose that $\F_0$ is $\iota$-real then,
\[ \iota \det{(1 - F_x t^{d(x)} | (\F^{\otimes 2k})_{\bar{x}})} \in \RR_{\ge 0}[t] \]
\end{rmk}

\begin{defn}
Let $X_0$ be geometrically irreducible and normal. Recall that any rank $1$ smooth Weil sheaf is $\iota$-pure. For $\F_0$ smooth Weil sheaf on $X_0$ let $\F_0$ be an irreducible constituent. Then, the det $\iota$-weight of $\G_0$ is defined to be $\beta / \mathrm{rk}(\G_0)$ where $\beta$ is the $\iota$-weight of $\det{\G_0}$ the top exterior power.
\end{defn}

\begin{rmk}
If $\F_0$ has det $\iota$-weight with multiplicitly $m(\beta)$ then the det $\iota$-weights of,
\[ \bigwedge^k \F_0 \]
are,
\[ \sum n_\beta \beta \text{ where } \sum n_\beta = k \text{ and } 0 \le n_\beta \le m(\beta) \]
This takes some work. We need to study the $G_0 = \overline{\rho(\pi_1(X))} \subset \GL{}{\Q_\ell}$ where $\F_0^{SS}$ corresponds to a representation $\rho : W(X_0) \to \GL{}{\Q_\ell}$.
\end{rmk}

\begin{rmk}
If $X_0$ is a smooth curve, $H_c^2(X, \F) = H^0(X, \F^\vee)^\vee(-1) = (\F_{\bar{x}})_{\pi_1(X, \bar{x})}(-1)$. Therefore, $\F_0 \onto \G_0$ where this is the largest quotient constant on $X$ this corresponds to the representations,
\begin{center}
\begin{tikzcd}
W(X, \bar{x}) \acts \F_{\bar{x}} \arrow[d]
\\
W(\FF_q) \acts (\F_{\bar{x}})_{\pi_1(X)} 
\end{tikzcd}
\end{center}
Therefore, we conclude that a det $\iota$-weight of $\G_0$ is $\beta$ iff an eigvenvalue of $F \acts H^2_c(\F)$ has $\iota$-weight $\beta + 2$. 
\end{rmk}

\begin{thm}
$X_0$ geometrically irreducible smooth curve, $\F_0$ smooth on $X_0$ and $\F_0$ is $\iota$-real then each irreducible factor of $\F_0$ is $\iota$-pure.
\end{thm}

\begin{proof}
We may assume that $X_0$ is affine. Let $\beta$ be the largest det $\iota$-weight of $\F_0$. Then $2k \beta$ should be the largest det $\iota$-weight on $\F_0^{\otimes 2k}$. Now consider the $L$-function,
\[ \frac{\iota \det{(1 - Ft | H^1_c(\F^{\otimes 2k}))}}{\iota{(1 - F t | H^2_C(\F^{\otimes 2k}))}} = L(X_0, \F_0^{\otimes 2k}, t) = \prod_{x \in X} \frac{1}{\iota \det{(1 - F_x t^{d(x)} | \F^{\otimes 2k}_{\bar{x}})}} \]
By the second remark this converges for $|t| < q^{-\frac{2k \beta + 2}{2}}$ and thus each factor converges in this region and by remark 1 the local factor on the right are positive real polynomials. Thus if $\alpha$ is an eigenvalue of $F_x \acts \F_{\bar{x}}$ then $|\iota \alpha|^{-2k} \ge q^{-\frac{2 k \beta + 2}{2}}$ and thus taking $k \to \infty$ we see that,
\[ |\iota \alpha| \le q^{\frac{\beta d(x)}{2}} \]
\end{proof}

\begin{example}
Let $\F_0 = \G_0 \oplus \G_0'$ with ranks $3$ and $2$ and det $\iota$-weights $\beta > \beta'$ respectively. Let the eigenvalues of $F_{\bar{x}}$ be $\alpha_1, \alpha_2, \alpha_3$ and $\alpha_1', \alpha_2'$ respectively. The first inequality says that $|\iota \alpha_i| \le q^{\frac{\beta d(x)}{2}}$ but $|\iota \alpha_1 \alpha_2 \alpha_3| = q^{\frac{3 \beta d(x)}{2}}$ which implies that $|\iota \alpha_i| = q^{\frac{\beta d(x)}{2}}$.
\end{example}

\begin{cor}
Let $\F_0$ be $\iota$-real on $X_0$ which is finite type over $\FF_q$ then,
\begin{enumerate}
\item $\F_0$ is $\iota$-mixed
\item if $X_0$ is geometrically irreducicble and normal and $\F_0$ si smooth, then rach irreducible component of $\F_0$ is $\iota$-pure. 
\end{enumerate}
\end{cor}

\begin{defn}
Let $X_0$ be a smooth curve, $\F_0$ a Weil-sheaf on $X_0$. Define $f_n^{\F_0} : X_0(\FF_{q^n}) \to \C$ by,
\[ f_n^{\F_0}(x) = \iota \tr{F_x^{\frac{n}{d(x)}} | \F_{\bar{x}}} \]
There is also a norm,
\[ (f_n, g_n)_n = \sum_{x \in X_0(\FF_{q^n})} f_n(x) \overline{g_n(x)} \quad \text{ and } \quad || f_n ||_n^2 = (f_n, f_n)_n \]
\end{defn}

\begin{prop}
If $\F_0$ is $\iota$-mixed and $H_c^0(X, \F) = 0$ then,
\[ \sup \left\{ \rho \mid \limsup_n \frac{|| f_n^{\F_0} ||_n^2}{q^{n(1+\rho)}} > 0 \right\} = \sup_{x \in |X_0|} \sup_{\substack{\alpha \text{e.v. of} \\ F_x \acts \F_{\bar{x}}}} \frac{\log{|\iota \alpha|^2}}{\log{q^{d(x)}}} \]
\end{prop}

\begin{proof}
We will not prove this. However, if $\F_0$ is smooth and $\iota$-pure then we study $\log{\iota L(X_0, \F_0 \otimes \overline{\F_0}; t)}$.
\end{proof}

\subsection{Fourier Transform}

Fix $\psi : \FF_q \to \overline{\Q}_\ell^\times$ a nontrivial additive character. Thenm $\L_0(\psi)$ a smooth sheaf of $\A^1_0$ corresponding to $W(\A^1_0, \bar{0}) \to \FF_q \xrightarrow{\psi} \overline{\Q}_\ell^\times$ defined by the Artin-Schier extension. Then for $x \in \A^1_{0}$ then $\psi_x : k(x) \to \overline{\Q}_\ell^\times$ by $\psi_x(y) = \psi(\tr{xy})$.

\begin{defn}
Let $T \psi : D_c^b(\A^1_0, \overline{\Q}_\ell) \to D_c^b(\A^1_0, \overline{\Q}_\ell)$ be defined by,
\[ T_\psi(K_0) = R (\pi^1_0)_! (\pi^{2*}_0(k_0) \otimes m_0^*(\L_0(\psi))) [1] \]
where $m : \A^1 \times \A^1 \to \A^1$ is the multiplication map ($\A^1$ is a ring object). 
\end{defn}

\begin{rmk}
By proper base change, for all $x \in \A^1_0(\overline{\FF}_q)$ we have,
\[ T_\psi(K_x)_{\bar{x}} = R \Gamma_c(\A^1, K \otimes \L(\psi_x))[1] \]
\end{rmk}

\begin{example}
For $K-0 \in D^b_C(\A^1_0, \overline{\Q}_\ell)$ define $f_n^{K_0}(t):= \sum (-1)^i f_n^{H^i(K_0)}$ then,
\[ \tr{F_x | T_\psi(K_0)_{\bar{x}}} = - \tr{F^{d(x)} | R \Gamma_c(\A^1, K \otimes \L(\psi_x))} \]
By Grothendieck-Lefschetz trace formula,
\[ = -\sum_{y \in \A^1_0(\FF_q^{d(x)})} \Tr{f_y | K_y)} \psi_x(-y) \]
Therefore,
\[ f_n^{T_\psi(K_0)}(t) = - \sum_{y \in \FF_{q^n}} f_n^{K_0}(y) \psi_t(-y) \]
and thus,
\[ || f_n^{T_\psi(K_0)} ||_n^2 = \sum_{t \in \FF_{q^n}} \sum_{y,z \in \FF_{q^n}} f_n^{K_0}(y) \overline{f_n^{K_0}(Z)} \psi_t(-y) \overline{\psi_t(-z)} = q^n || f_n^{K_0} ||_n^2 \]
which uses a version of Placherel's formula. 
\end{example}

\begin{thm}[Fourier inversion]
$T_{\psi^{-1}} \circ T_\psi(K_0) = K_0(-1)$ for any $K_0$.
\end{thm}

\begin{proof}
First,
\[ T_{\psi^{-1}} \circ T_\psi(K_0) = R \pi^1_! \left( \pi^{2*} R \pi^1_! [\pi^{2*} K_0 \otimes m^* \L_0(\psi)] \otimes m^* \L_0(\psi^{-1}) \right)[2] \]
By the projection formula, 
\end{proof}

\begin{thm}
Let $X_0$ be geometrically irreducible smooth curve, $j_0 : X_0 \embed \overline{X_0}$ and $\F_0$ smooth $\iota$-pure of weight $\beta$. Then,
\begin{enumerate}
\item $\iota$-weights of $F \acts H^i(\overline{X}, j_* \F)$ are $\le \beta + i$
\item $\iota$-weights of $F \acts H^i_c(X, \F)$ are $\le \beta + i$.
\end{enumerate}
\end{thm}

\begin{rmk}
by Poincare duality there is a perfect pairing,
\[ H^i(\overline{X}, j_* \F) \times H^{2-i}(\overline{X}, j_* \F^\vee)(1) \]
are dual. Then $(1)$ implies that $H^i(\overline{X}, j_* \F)$ is $\iota$-pure of weight $\beta + i$. Furthermore $(1) \iff (2)$ because of the exact sequence,
\begin{center}
\begin{tikzcd}
0 \arrow[r] & j_! \F \arrow[r] & j_* \F \arrow[r] & \mathcal{Q} \arrow[r] & 0
\end{tikzcd}
\end{center} 
with $\Q$ supported on the boundary. Then $\mathcal{Q}_s$ has weights less than $\beta$ by semi-continuity.
\end{rmk}

\begin{proof}
Proof of (1) in the case $i = 1$. We may shrink $X_0$ + some minor arguments $\overline{X}_0 \to \P^1$ we reduce to the case the following claim over a point: consider $U_0 \subset \A^1_0 \subset \P^1_0$ and $\G_0$ geometrically irreducible smooth $\iota$-pure sheaf of weight $\beta$ on $U_0$ (unramified at $\infty$?) then we need $\iota$-weights of $F \acts H^1_c(U, \G)$ are at most $\beta + 1$. 
\end{proof}

\section{Oct. 5}


\subsection{Review of t-Structures}

\begin{defn}
A t-structure on a triangulated category $\D$ is a pair ($\D^{\le 0}, \D^{\ge 0}$) such that,
\begin{enumerate}
\item letting $\D^{\le n} := \D^{\le 0}[-n]$ and $\D^{\ge n} := \D^{\ge 0}[-n]$ then $\D^{\le n} \subset \D^{\le 1}$ and $\D^{\ge 1} \subset \D^{\ge 0}$.
\item $\Hom{\D}{\D^{\le 0}}{\D^{\ge 1}} = 0$
\item for every $X \in \D$ there is a distringulished triangle,
\begin{center}
\begin{tikzcd}
X^{\le 0} \arrow[r] & X \arrow[r] & X_{\ge 1} \arrow[r] & X^{\le 0}[1]
\end{tikzcd}
\end{center}
where $X_{\le 0} \in \D^{\le 0}$ and $X^{\ge 1} \in \D^{\ge 1}$.
\end{enumerate}
\end{defn}

\begin{prop}
Axioms (a) and (c) imply the existence of truncation functors $\tau_{\le n} : \D \to \D^{\le n}$ right adjoint to $\D^{\le n} \subset \D$ and $\tau_{\ge n} : \D \to \D^{\ge n}$ left adjoint to $\D^{\ge n} \subset \D$ and thus the exact triangles in (c) are actually functorial.
\end{prop}

\begin{defn}
The \textit{heart} of a t-structure on $\D$ is $\D^\heart = \D^{\le 0} \cap \D^{\ge 0}$.
\end{defn}

\begin{thm}
The category $\D^{\heart}$ is an abelian category and a sequence $A \to E \to B$ in $\D^{\heart}$ is exact iff there exists an exact triangle in $\D$ $A \to E \to B \to A[1]$ (where only the connecting map is extra data).
\end{thm}

\begin{thm}
For each $n$, there is an additive functor $H^n : \D \to \D^{\heart}$ given by $H^n(X) = \tau_{\ge n} \tau_{\le n} X = \tau_{\le n} \tau_{\ge n} X$. We rtie,
\[ H^n(X) \in \D^{\le n} \cap D^{\ge n} \cong \D^\heart \]
where the last isomorphism is given by shifting. Furthermore, $H^n$ is cohomological meaning it sends short exact sequences to long exact sequences.
\end{thm}

\begin{defn}
Given trianguled categories $\D_1, \D_2$ with t-structures we say that a triangulated functor $F : \D_1 \to \D_2$ is 
\begin{center}
\item \textit{t-left exact} if $F(\D_1^{\ge 0}) \subset \D_2^{\ge 0}$
\item \textit{t-right exact} if $F(\F_1^{\le 0}) \subset \D_2^{\le 0}$
\end{center}
\end{defn}

\section{Oct. 19}

Recall $B_0 \in D_{\text{mixed}}(X_0)$ then,
\[ \omega(B_0) := \max_{\nu} \left( \omega(\cH^\nu(B_0)) - \nu \right) \]
And $D_{\le w}(X_0)$ if 
\bigskip\\
Then $B_0$ is pure of weight $w$ if $B_0 \in D_{\le w}(X_0) \cap D_{\ge w}(X_0)$.

\begin{lemma}[Semi-continuity of weights]
Let $j : U_0 \embed X_0$ be an open dense, $\iota : Y_0 \embed X_0$ the closed complement. Let $\overline{B}_0 \in \Perv{X_0}$ be $\tau$-mixed such that $j^*(\overline{B}_0) = B_0$ and assume that $H_p^0(\iota^* (\overline{B}_0)) = 0$ then
\[ w(\overline{B}_0) = w(B_0) \]
\end{lemma}

\begin{proof}
The inequality $w(B_0) \le w(\overline{B}_0)$ is obvious so we just need to show the other direction. We apply the Fourier transform,
\begin{center}
\begin{tikzcd}
& \arrow[ld, "p_{13}"] \arrow[rd, "p_{23}"] \A_0^r \times \A_0^r \times \A_0^s \arrow[rr, "m_{12}"] & & \A_0^1
\\
\A_0^r \times \A_0^s & \A_0^r \times \A_0^s
\end{tikzcd}
\end{center}
For $B \in D_c^b(\A_0^r \times \A_0^s)$ then,
\[ T_\psi(B) = R (p_{13})_! (p_{23}^* B \otimes^{\mathbb{L}} m_{12}^* \L_0(\psi)) \]
\end{proof}

\end{document}



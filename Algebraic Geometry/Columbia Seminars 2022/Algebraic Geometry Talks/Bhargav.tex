\documentclass[12pt]{article}
\usepackage{hyperref}
\hypersetup{
    colorlinks=true,
    linkcolor=blue,
    filecolor=magenta,      
    urlcolor=cyan,
}
 
\urlstyle{same}
\usepackage{import}
\import{../}{AlgGeoCommands}

\newcommand{\Qbar}{\overline{\Q}}
\newcommand{\tors}{\mathrm{tors}}
\newcommand{\Frob}{\mathrm{Frob}}


\begin{document}
\section{Kodaira Vanishing Revisited}

\subsection{Classical Kodaira Vanishing}

\begin{theorem}
Let $X / \CC$ be smooth projective $d = \dim{X}$ and $\L \in \Pic{X}$ ample. Then,
\[ H^{< d}(X, \L^{\ot -1}) = 0 \]
or equivalently,
\[ H^{> 0}(X, \omega_X \ot \L) = 0 \] 
\end{theorem}

\begin{rmk}
There is a version for $\L$ big and nef (Kawamata-Viehweg). Useful because big + nef is stable under pullback along proper birational maps unlike ampleness (e.g. consider blowups). 
\end{rmk}

\subsection{Applications}

Say $H = V(s) \subset X$ with $\L = \struct{X}(H)$ is a hyperplane section. Then,
\[ H^0(X, \omega_X(2H)) \to H^0(H, \omega_H(H)) \]
is surjective. 

\begin{proof}
There is an exact sequence,
\begin{center}
\begin{tikzcd}
0 \arrow[r] & \omega_X \arrow[r] & \omega_X(H) \arrow[r] & \omega_H \arrow[r] & 0
\end{tikzcd}
\end{center}
and then tensor by $\struct{X}(H)$ and apply Kodaira vanishing to get $H^1(X, \omega_X(H)) = 0$.
\end{proof}

\subsection{Proof of Kodaira Vanishing}

Say $\L = \struct{X}(H)$ for $H \subset X$ meaning $\L$ is effective. We want to show that $H^{<d}(X, \L^{\ot -1}) = 0$. Hodge theory,
\[ H^{<d}(X, \L^{\ot -1}) \subset_{\text{summand}} H^{<d}_c(X \sm H, \CC) \cong H^{>d}(X \sm H, \CC)^\vee = 0 \]
using Poincare duality and Artin vanishing because $X \sm H$ is affine.  

\subsection{What Happens over $\FF_p$ or $\Z$}

\begin{enumerate}
\item False in general over $\FF_p$
\begin{enumerate}
\item Mumford (singular surface over $\FF_p$)
\item Raynaud (over $\FF_p$)
\item Totaro (over $\ZZ$)
\end{enumerate}
\item Deligne-Illusie: it is true in $\dim < p$ in the liftable case. 
\end{enumerate}

Salvage: work up to finite covers. (In characteristic zero, vanishing after a finite cover implies vanishing but not in positive characteristic). 

\begin{example}
$X = S^1$ and $H^1(S^1, \FF_p) = \FF_p$ but $[p] : S^1 \to S^1$ annhilates this class, 
\[ [p]^* : H^1(S^1, \FF_p) \to H^1(S^1, \FF_p) \]
is zero. 
\end{example}
 
\section{Kodaira Vanishing in Mixed Characteristic}

\begin{theorem}
Say $X / \Z_p$ is proper flat variety  of relative dimension $d$ and $\L \in \Pic{X}$ is semiample and big. Then there exist a finite surjective map $\pi : Y \to X$ such that,
\[ \pi^* : H^{\bullet}(X, \L^{\ot a})_{\tors} \to H^\bullet(Y, \pi^* \L^{\ot a})_{\tors} \]
is $0$ for $a \in \{-1, 0, 1\}$.
\end{theorem}

\begin{rmk}
Analog over $\FF_p$ is also true (Hochster-Huneke Smith 90s) for $H^{<d}(X, \L^{\ot -1})$ and $H^{>0}(X, \L^{\ot a})$ for $a \in \{0,1\}$.
\end{rmk}

\begin{example}
$X = E$ elliptic curve over $\FF_p$ and $H^1(X, \struct{X}) = \FF_p$ then $[p] : E \to E$ does the job. 
\end{example}

\begin{rmk}
In positive characteristic, the $\L$ in positive degrees case is easy because we can take, $H^i(X, (\Frob_p^n)^* \L) = H^i(X, \L^{\ot p^n}) = 0$ by Serre vanishing. 
\end{rmk}

\begin{rmk}
Ramification is necessary (ex consider K3 surface).
\end{rmk}

\begin{rmk}
There is a relative varient over complete noetherian local domains $(R, \m)$ with $p \in \m$.
\end{rmk}

\begin{rmk}
Try to prove the following: say $X / \ZZ_p$ is a relative curve. Then there exists $\pi : Y \to X$ finite cover such that $H^1(X, \struct{X}) \to H^1(X, \struct{Y})$ is divisible by $p$. 
\end{rmk}

\subsection{Application}

\begin{theorem}[BMPSTWW, TY]
One can run the MMP for arithmetic 3-folds (relative dimension $2$) over $\Z[30^{-2}]$.
\end{theorem}

\subsection{Local Kodaira Vanishing}

Let $X \subset \P^n$ be a projective variety then the affine cone $C(X)$ preserves nice properties of $X$.
\bigskip\\
Principle: projective geometry of $X \subset \P^n$ is equivalent to local geometry of Cone at $0$.

\begin{theorem}[Local Kodaira]
Let $(R, \m)$ be an excellent noetherian local domain. Let $R^+$ be the absolute integral closure (integral closure of $R$ in $\overline{\Frac{R}}$). Then $R^+ / p$ is a Cohen-Macalay module over $R/p$.
\end{theorem}

\begin{rmk}
Say we have $\Z_p[[x_1, \dots, x_n]] \embed R$ is finite. Then there exists a fintie extension $R \embed S$ with the following feature. Any relation $\sum a_i x_i = 0$ in $R/p$ is trivial in $S/p$.
\end{rmk}


\begin{rmk}
This thm implies ``homological conjectures'' in commutative algebra (e.g. direct summand conjecture). Easy to deduce the direct summand conjecture from CM theorem. 
\end{rmk}

\subsection{What Goes Into the Proof?}

$p$-adic Riemann-Hilbert correspondences (joint with Lurie)

\begin{theorem}
Say $C / \Q_p$ is complete and algebraically closed and $X / \struct{C}$ is proper flat scheme. There exists a natural exact functor,
\[ RH : D(X_C, \FF_p) \to D_{qc}(X_{p = 0}) \]
such that perverse $\FF_p$-sheaves are taken to almost CM complexes. 
\end{theorem}

Why is this useful. Say $X / \struct{C}$ as before and consdier the absolute integral closure $\pi : X^+ \to X$. Fact,
\[ RH(\pi_* \FF_p |_{X_C}) = \pi_* \struct{X^+} / p \]
Using this, can ``almost'' prove theorem via:

\begin{lemma}
$Y / \CC$ any variety of $\dim{Y} = d$ and $\pi : Y^+ \to Y$ absolute integral closure. Then $\pi_* \underline{\FF_p} [d]$ is perverse.
\end{lemma}

To go to honest statement (not almost) use prismatic cohomology. 

\subsection{Question}

Say $(R, \m)$ is a $p$-adically complete excellent domain.

\begin{defn}
The \textit{test ideal} is,
\[ \tau(\omega_R) = \bigcap_{R \embed S \embed R^+} \im{(\tr : \omega_S \to \omega_R)} \subset \omega_R \]
with $R \embed S$ finite.
\end{defn}

Question: Does $R \mapsto \tau(\omega_R)$ commute with localization?
\bigskip\\
Evidence:
\begin{enumerate}
\item True if $p = 0$ in $R$ (Smith)
\item True ``up to $p$-perturbation''.
\end{enumerate}

\begin{cor}
$\tau(\omega_R)[p^{-1}]$ is the Grauert-Riemenschneider sheaf in $\omega_{R[p^{-1}]}$.
\end{cor}


\begin{rmk}
$\Spec{\struct{C}}$ has two points with residue fields $C$ and $\overline{\FF_p}$ respectively. 
\end{rmk}

\end{document}

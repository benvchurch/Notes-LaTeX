\documentclass[12pt]{article}
\usepackage{hyperref}
\hypersetup{
    colorlinks=true,
    linkcolor=blue,
    filecolor=magenta,      
    urlcolor=cyan,
}
 
\urlstyle{same}
\usepackage{import}
\import{../}{AlgGeoCommands}


\begin{document}
\section{Motivation}

\begin{defn}
A \textit{quiver} is a finite oriented graph.
\end{defn}

\begin{defn}
A \textit{quiver representation} $M$ is a vectorspace $M_i$ for each $i \in Q_0$ and a map $M_a : M_{s(a)} \to M_{t(a)}$ for $a \in Q_1$. A morphism $\varphi : M \to N$ of quiver representations is a set of morphisms $\varphi_i : M_i \to N_i$ commuting with the $M_a$ and $N_a$.
\end{defn}

\begin{defn}
A quiver $Q$ is acyclic if there do not exist cycles.
\end{defn}

\begin{defn}
Let $\M_{d,S}^{\theta\text{-ss}}$ be the moduli stack of $\theta$-ss representations of $Q$ of dimension $d \in \N^{Q_0}$ (meaning $\dim{M_i} = d_i)$.
\end{defn}

In what follows, let $S$ be a noetherian scheme.

\begin{theorem}
If $Q$ is acyclic there exists a projective adequate moduli space $\M_{d,S}^{\theta\text{-ss}} \to M_{d,S}^{\theta\text{-ss}}$.
\end{theorem}

\begin{theorem}
Let $Q$ be any quiver and $\theta = - \left< -, \beta \right>$ then there exists a separated adequate moduli space and a morphism $M^{\theta\text{-ss}}_{d,S} \to M_{d,S}$ which is proper.
\end{theorem}

\begin{theorem}
Let $k$ be a field, $Q$ any quiver. For a natural construction $\L_{\theta} \in \Pic{\M_{d,S}^{\theta\text{-ss}}}$. There exists an explicit bound $m_0$ such that $\L_\theta^{\ot m}$ is globally generated for $m \ge m_0$.
\end{theorem}

\section{Moduli Stacks}

A Stack $\X$ is a pseudofunctor $\X : \Sch_S^{\et} \to \mathrm{Gpd}$ with a descent condition.

\begin{defn}
An stack is \textit{algebraic} if,
\begin{enumerate}
\item $\Delta_{\X} : \X \to \X \times_S \X$ is representable by algebraic spaces
\item there exists a smooth surjective morphism from a scheme $U \to \X$.
\end{enumerate}
\end{defn}

\begin{defn}
We define,
\[ \M_{d,S}(T) = \{ \F_i \text{ loc free on } T \text{ of rank } d_i \text{ with } \varphi_a : \F_{s(a)} \to \F_{t(a)} \} \]
\end{defn}

\begin{prop}
$\M_{d,S} \cong [R_d / G_d]$ where,
\[ R_d = \prod_{a \in Q_1} \A_S^{d_{t(a)} d_{s(a)}} \quad G_d = \prod_{i \in Q_0} \GL_{d_i} \]
Therefore we have a presentation $\A_S^N \onto \M_{d,S}$. 
\end{prop}

\begin{cor}
$\M_{d,S}$ is an algebraic stack smooth and finite type over $S'$.
\end{cor}

\begin{prop}
$\M_{d, S}$ has affine diagonal.
\end{prop}

\begin{defn}
A \textit{stability function} $\theta  : \Z^{Q_0} \to \Z$ is a group homomorphism.
\end{defn}

\begin{rmk}
Given a stability function we will write $\theta(M) := \theta(\dim{M})$ where $\dim{M} \in \Z^{Q_0}$ is the vector of dimensions.
\end{rmk}

\begin{defn}
$M \in \M_{d,S}$ is $\theta$-\textit{semistable} if $\theta(M) = 0$ and $\forall M' \subset M$ we have $\theta(M') \le 0$.
\end{defn}

\begin{prop}
$\M_{d,S}^{\theta\text{-ss}} \embed \M_{d,S}$ is an open subfunctor. 
\end{prop}

\begin{example}
If $\theta = 0$ then $\M_{d,S}^{\theta\text{-ss}} = \M_{d,S}$.
\end{example}

\section{Characterizing Semistability}

Over $S = \Spec{k}$ and $k = \bar{k}$. 

\begin{defn}
$\M_{d,k}$ has the universal representation $\E = (\E_i)$. The \textit{determinantal line bundle} is,
\[ \L_{\theta} = \bigotimes_{i \in Q_0} (\det{\E_i})^{\ot - \theta_i} \]
where $\theta_i = \theta(e_i)$. 
\end{defn}

Let $Q$ be a quiver and $M,N$ be two representations. Every $M$ has a 2-step projective resolution and thus,
\[ \left< M, N \right> := \chi(\RHom{Q}{M}{N}) \]
is the Euler pairing. This only depends on the dimension vectors $\dim{M}$ and $\dim{N}$.

Assume that $\theta = - \left< -, \beta \right>$ for $\beta \in \N^{Q_0}$ and $\theta(d) = 0$ so $\inner{d}{\beta} = 0$.


For $V \in \M_{\beta}$ then,
\[ \RHom{Q}{\E}{V \ot \struct{\M_d}} = [K^0 \xrightarrow{\d} K^1] \]


We can take $\det{d} : \det{K^0} \to \det{K^1}$ whcih corresponds to a section $\sigma_V \in H^0(\det{K^0}^\vee \ot \det{K^1}) = H^0(\det{\RHom{}{\E}{V \ot \struct{}}}) = \L_{\theta} \]

\begin{prop}
$M \in \M_d$ is $\theta$-ss for $\theta = -\inner{-}{\beta}$ iff there is $m > 0$ and $V \in \M_{m \beta}$ s,t, $\RHom{Q}{M}{V} = 0$.
\end{prop}

\begin{proof}
$\RHom{Q}{M}{V} = 0$ iff $\d$ is an isomorphism at the point $M$ iff $\det{\d}$ is nonzero. This this is equivalent to $\sigma_V$ being nonzero. Thus we see that $\L_theta^{\ot m}$ is globally generated on $\M_d^{\theta\text{-ss}}$. 
\end{proof}

\begin{rmk}
$\L_{m \theta} = \L_{\theta}^{\ot m}$.
\end{rmk}

\section{Good and adequate Moduli Space}

For $\theta = 0$ we have $\M_d^{\theta\text{-ss}} = \M_d$.

\begin{prop}
$Q$ is acyclic implies that adequate moduli of $\M_{d,S} \to S$ is isomorphic to $S$.
\end{prop}

\begin{defn}
Let $\X$ be quasi-separated. A \textit{good moduli space} of $\X$ is $f : \X \to X$ (qcqs) to be an algebraic space $X$ s.t.
\begin{enumerate}
\item $f_* \struct{\X} = \struct{X}$
\item $f$ is ``cohomologically affine'' meaning $f_*$ is exact on $\QCoh{\X}$.
\end{enumerate}
\end{defn}

\begin{example}
If $G \acts X$, say $X \to Y$ is a good quotient. Then $[X/G] \to Y$ is a good moduli space.
\end{example}

\begin{defn}
$f : \X \to X$ is an adequate moduli space if,
\begin{enumerate}
\item $f_* \struct{\X} = \struct{X}$
\item $f$ is ``adequately affine'' meaning for any $\cA \onto \cB$ of quasi-coherent algebras, \etale-locally $\sqrt{f_* \cA} \onto f_* \B$. (??)
\end{enumerate}
\end{defn}

\begin{theorem}[Alper]
Let $f : \X \to X$ be an adequate moduli space
\begin{enumerate}
\item $f$ is surjective, universall closed
\item over $k = \bar{k}$ the map $f$ induces a bijection on closed $k$-points
\item if $X$ is a scheme implies that it is initial in $\Sch_S$ under $\X$
\item base change of an adequate moduli space is homeom. to the adequate moduli space of the base change. 
\end{enumerate}
\end{theorem}

\begin{example}
If $Q$ is acyclic then,
\[ \M_{d, \bar{k}} = [ \bigoplus \mathrm{Hom} / G_d] \]
has only $1$ closed point and $\Spec{k}$ is the adequate moduli space. 
\end{example}

\begin{theorem}[Alper]
If $\X$ is finite type over $S$ and locally reductive and $\Theta$-reductive and $S$-complete then $\X$ has an adequate moduli space. If also $\X \to S$ satisfies uniqueness of the valuative criterion of properness then the adequate moduli space is proper. 
\end{theorem}

Apply this theorem.

The proof of Thm B

\end{document}

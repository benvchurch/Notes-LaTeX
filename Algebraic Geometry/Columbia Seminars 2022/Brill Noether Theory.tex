\documentclass[12pt]{article}
\usepackage{hyperref}
\hypersetup{
    colorlinks=true,
    linkcolor=blue,
    filecolor=magenta,      
    urlcolor=cyan,
}
 
\urlstyle{same}
\usepackage{import}
\import{../}{AlgGeoCommands}

\DeclareMathOperator{\reg}{\mathrm{reg}}

\begin{document}

\section{Introduction}

\subsection{References}

\begin{enumerate}
\item A sampling of vector bundle techniques, Lazarsfeld. 
\end{enumerate}

\subsection{Divisors}

\begin{rmk}
Let $X$ be a projective variety over $k = \bar{k}$.
\end{rmk}

A divisor is a formal sum,
\[ D = \sum a_i D_i \]
for $a_i \in \Z$ and $D_i$ is a codimension $1$ subvariety. We also will allow $a_i \in \Q$ or $\RR$. 

\begin{defn}
$N^1(X)_{\RR} = \{ \RR\text{-divisors} \} / \sim$ 
where,
\[ D_1 \sim D_2 \iff D_1 \cdot C = D_2 \cdot C \]
for all integral curves $C \subset X$. 
\end{defn}

\begin{defn}[Ample]
A Line bundle $\L$ is \textit{ample} if one of the following equivalent conditions hold,
\begin{enumerate}
\item $\L^{\ot m}$ (for some $m \ge 0$) is very ample meaning $\L$ defines an embedding $X \embed \P^N$
\item for any coherent sheaf $\F$ there exists $n(\F)$ s.t. $m \ge n(\F)$ implies $\F \ot \L^{\ot m}$ is globally generated
\item for any coherent sheaf $\F$ there exists $n(\F)$ s.t. $m \ge n(\F)$ implies $H^i(X, \F \ot \L^{\ot m}) = 0$ for all $i > 0$
\item (over $\CC$) positive in the sense of admiting a positive hermitian connection.
\end{enumerate}
\end{defn}

\begin{theorem}[Nakai-Moishezon]
On $X$ a line bundle $\L$ is ample if and only if
\[ (\L^{\dim{V}} \cdot V) > 0 \]
for all subvarieties $V \subset X$.
\end{theorem}

\begin{defn}
$\L$ is \text{nef (numerically effective)} if,
\[ (\L \cdot C) \ge 0 \]
for all curves $C \subset X$. 
\end{defn}

\begin{theorem}[Kleiman]
If $\L$ is net then for any subvariety $V \subset X$,
\[ \L^{\dim{V}} \cdot V \ge 0 \]
\end{theorem}

\begin{rmk}
However, $\L \cdot C > 0$ does not imply that $\L$ is ample meaning it does not imply that the intersection against all subvarities is positive. 
\end{rmk}

\begin{prop}
\begin{enumerate}
\item non-negative linear combinations of nef divisors are nef.
\item if $f : X \to Y$ is proper and $\L$ on $Y$ is nef then $f^* \L$ is nef.
\item if $f : X \to Y$ is surjective and proper and $f^* \L$ is nef then $\L$ is nef. 
\end{enumerate}
\end{prop}

\begin{cor}
\begin{enumerate}
\item Let $X$ be projective, $D$ is a nef $\RR$-divisor, and $H$ is any ample $\RR$-divisor. Then $D + \epsilon H$ is ample for all $\epsilon > 0$. 
\item fix $\RR$-divisors $D$ and $H$, if $(D + \epsilon H)$ is ample for all small $\epsilon > 0$ then $D$ is nef. 
\end{enumerate}
\end{cor}

\begin{proof}
For (2) we have,
\[ D \cdot C = \lim_{\epsilon \to 0} (D + \epsilon H) \cdot C \ge 0 \]
For (1) we need to show that,
\[ (D + \epsilon H)^{\dim{V}} \cdot V > 0 \]
for any subvariety $V \subset X$. Now,
\[ (D + \epsilon H)^{\dim{V}} = [D^{\dim{V}} + \cdots + (\epsilon H)^{\dim{V}} ] \cdot V \]
Since $D$ is nef, all the intersections are $\ge 0$ and $\epsilon^{\dim{V}} H^{\dim{V}} \cdot V > 0$ because $\epsilon > 0$ and $H$ is ample and thus we conclude.
\end{proof}

\begin{prop}
Let $f : X \to T$ be surjective, proper and $\L$ is a line bundle on $X$. Suppose for some $t_0 \in T$, that $L_{t_0}$ is nef on $X_{t_0}$. Then there exists a countable union of proper subvarities $B \subset T$ such that $L_t$ is nef on $X_t$ for all $t \notin B$. 
\end{prop}

\newcommand{\Amp}{\mathrm{Amp}}
\newcommand{\Nef}{\mathrm{Nef}}

\begin{defn}
The ample cone is,
\[ \Amp(X) = \{ D \in N^1(X)_{\RR} \mid D \text{ is ample} \} \subset N^1(X)_{\RR} \]
and the nef cone,
\[ \Nef(X) = \{ D \in N^1(X)_{\RR} \mid D \text{ is nef} \} \subset N^1(X)_{\RR} \]
The corollaries tell us that $\Amp(X)$ is an open convex cone and $\overline{\Amp(X)} = \Nef(X)$.
\end{defn}

\begin{example}
$X = \P^1 \times \P^1$. Then $N^1(X)_{\RR} = \RR \left< F_1, F_2 \right>$ with basis $F_i = [\pi_i^{-1}(\text{pt})]$. The $F_i$ are both nef but $F_i^2 = 0$ so they are not ample. The ample cone is the first quadrant and the nef cone is the first quadrant plus the positive axes. 
\end{example}

\begin{example}
Let $E$ be an elliptic curve general in $\M_1$. Let $X = E \times E$. Then,
\[ N^1(X)_{\RR} = \RR \left< F_1, F_2, \Delta \right> \]
Claim: any effective class on $X = E \times E$ is nef. Indeed this is because we can freely move classes by translation until they intersect properly. 
\end{example}

\begin{lemma}
Let $X = E \times E$ . A class $\alpha \in N^1(X)_{\RR}$ is nef iff $\alpha^2 \ge 0$ and $\alpha \cdot h \ge 0$ for some ample $h$.
\end{lemma}

\begin{prop}
Let $X$ be a surface and $D$ an integral divisor s.t. $D^2 > 0$ and $(D \cdot H) > 0$ for some ample $H$, then $m D$ is effective for some $m > 0$. 
\end{prop}

\begin{proof}
Consider,
\[ \chi(X, mD) = \tfrac{1}{2} (m D) \cdot (mD - K_X) + \chi(\struct{X}) \]
Now since $D^2 > 0$ we can make $\chi(X, mD) \to \infty$ as $m \to \infty$. Furthermore, $h^2(X, mD) = h^0(X, K_X - mD) = 0$ for large enough $m$ if $D \cdot H > 0$. Otherwise, there would be an effective $D' \sim K_X - mD$ and then $D' \cdot H > 0$ since $H$ is ample but $D' \cdot H = K_X \cdot H - m D \cdot H < 0$ for large enough $m$ since $D \cdot H > 0$. Therefore, we must have $h^0(X, mD) \to \infty$ as $m \to \infty$.
\end{proof}

\begin{rmk}
This proves the previous lemma using that the nef cone is closed and that eny effective class is nef. 
\end{rmk}

\begin{rmk}
Back to the example, let $\alpha = x F_1 + y F_2 + z \Delta$ and $h = F_1 + F_2 + \Delta$. Applying the lemma gives the inequalities of the nef cone,
\[  x + y + z \ge 0 \quad xy + xz + y z \ge 0 \]
This is a round cone. 
\end{rmk}

\subsection{Schedule}

\begin{enumerate}
\item Castelnuovo-Mumford regularity
\item Introduction to Brill-Noether Theory
\item Petri's condition and Brill-Noether Theory on K3 surfaces: 
\[ \mu_0 : H^0(C, A) \ot H^0(C, \omega_C \ot A^\vee) \to H^0(C, \omega_C) \]
 for a line bundle $A$ and a curve $C$ when is this injective?
\item Lazarsfeld-Mukai bundles on K3 surfaces.
\item Proof of the Brill-Noether-Petri.
\item $\dim = 2$ case of Fujita's conjecture.
\item Moduli of sheaves on K3s? Other topics?
\end{enumerate}

\section{Mumford-Castounovo Regularity}

\begin{theorem}[Serre Vanishing]
Let $X \to \Spec{A}$ be proper and $\L$ ample on $X$. Then for any $\F \in \Coh{X}$ there is some $n(\F)$ such that for all $n \ge n(\F)$ and $i > 0$,
\[ H^i(X, \F \ot \L^{\ot n}) = 0 \]
\end{theorem}

\begin{rmk}
Today we want to quantify how the minimal $n(\F)$ grows.
\end{rmk}

\begin{defn}
Let $X = \P^n_k$. Let $\F \in \Coh{X}$ and $m \in \Z$. Then $\F$ is $m$-\textit{regular} if,
\[ H^i(X, \F(m-i)) = 0 \]
for all $i > 0$. Then \textit{the regularity} of $\F$ is,
\[ \reg(\F) = \inf \{ m \in \Z \mid \F \text{ is } m\text{-regular} \} \]
\end{defn}

\begin{rmk}
If $\F$ is supported on a finite set then $\reg(\F) = - \infty$. Otherwise $\reg{\F}$ is a finite number.
\end{rmk}


\begin{example}
Consider $X = \P^1 \times \P^1$ and $\L = \struct{X}(-1,-3)$. Then,
\[ H^i(X, \L) = 0 \]
for all $i$ by Kunneth since $\struct{\P^1}(-1)$ has no cohomology. However,
\[ \dim H^i(X, \L(1)) = 
\begin{cases}
0 & i \neq 1
\\
1 & i = 1
\end{cases} \]
because,
\[ H^1(X, \L(1)) \cong H^0(\P^1, \struct{\P^1}) \ot H^1(\P^1, \struct{\P^1}(-2)) \]
The cohomology can all vanish but can jump up after a \textit{positive} twist. However, $\reg(\L) = 3$ so after twisting three times the higher cohomology stays zero. 
\end{example}

\begin{example}
$\F = \struct{\P^n}(a)$ is $(-a)$-regular. If $X \subset \P^n$ is a degree $d$-hypersurface then $\iota_* \struct{X}$ is $(d-1)$-regular. 
\end{example}

\begin{prop}
Let $\F \in \Coh{X}$ be $m$-regular. Then for $k \ge 0$,
\begin{enumerate}
\item $\F$ is $(m+k)$-regular
\item $\F(m + k)$ is generated by global sections
\item the natural map,
\[ H^0(X, \F(m)) \ot H^0(X, \struct{X}(k)) \onto H^0(X, \F(m+k)) \]
is surjective.
\end{enumerate}
\end{prop}

\begin{proof}
By flat base change, we can ssume that $k$ is algebraically closed. Then we do induction on $n = \dim{X}$. For $\F \in \Coh{X}$ the support $\Supp{}{\F}$ is a closed subscheme so it has finitely many components and hence there exists a hyperplane mising each generic point (using that $k$ is infinite). Therefore, we get an exact sequence,
\begin{center}
\begin{tikzcd}
0 \arrow[r] & \F(-1) \arrow[r] & \F \arrow[r] & \G \arrow[r] & 0
\end{tikzcd}
\end{center}
with $\G = \iota_* (\F|_{H})$ supported on $H \cong \P^{n-1}$. When $n = 0$ the statements are obvious. By the sequence, if $\F$ is $m$-regular then $\G$ is $m$-regular. By the induction hypothesis, $\G$ is $(m+k)$-regular. Thus for $i > 0$ and $k \ge 0$ we have $H^i(X, \G(m+k-i)) = 0$ so if $H^i(X, \F(m+k-1-i)) = 0$ then $H^i(X, \F(m+k-i)) = 0$ so if $\F$ is $(m+(k-1))$-reglar then $\F$ is $(m+k)$-regular so by induction $\F$ is $(m+k)$-regular for all $k \ge 0$ proving (a). Now, consider the diagram,
\begin{center}
\begin{tikzcd}
0 \arrow[r] & H^0(\F(m+k-1)) \ot \struct{X} \arrow[d] \arrow[r] & H^0(\F(m+k)) \ot \struct{X} \arrow[d] \arrow[r] & H^0(\G(m+k)) \ot \struct{X} \arrow[d] \arrow[r] & 0
\\
0 \arrow[r] & \F(m+k-1) \arrow[r] & \F(m+k) \arrow[r] & \G(m+k) \arrow[r] & 0
\end{tikzcd}
\end{center}
Since $\F$ is $(m+k)$-regular $H^1(X, \F(m+k-1)) = 0$ so the top sequence is short exact. By the induction hypothesis, for all $k \ge 0$ the map $H^0(\G(m+k)) \ot \struct{X} \onto \G(m+k)$ is surjective (on $H$ this is the induction hypothesis and outside $H$ this hold because $\G$ vanishes). By Serre, there is some $k \gg 0$ such that $\F(m+k)$ is globally generated and thus by downward induction we see that $\F(m+k)$ is globally generated for all $k \ge 0$ proving (b). Then consider the diagram,
\begin{center}
\begin{tikzcd}
& 0 \arrow[d]
\\
H^0(X, \F(m)) \ot H^0(X, \struct{X}(k-1)) \arrow[d] \arrow[r] & H^0(X, \F(m+k-1)) \arrow[d]
\\
H^0(X, \F(m)) \ot H^0(X, \struct{X}(k)) \arrow[d, two heads] \arrow[r] & H^0(X, \F(m+k)) \arrow[d, two heads]
\\
H^0(\G(m)) \ot H^0(H, \struct{H}(k)) \arrow[r, two heads] & H^0(H, \G(m+k)) \arrow[d]
\\
& 0
\end{tikzcd}
\end{center}
By induction on $n$ the bottom map is surjective. The bottom downward maps are surjective because $\F$ and $\G$ are $m$-regular so $H^1(X, \F(m-1)) = 0$ and likewise for $\G$. Now we use induction on $k$. The case $k = 0$ is clear so we can assume that the top map is surjective and thus the middle map is also surjective completing the induction step. Therefore, 
\[ H^0(X, \F(m)) \ot H^0(X, \struct{X}(k)) \onto H^0(X, \F(m+k)) \]
is surjective proving (c). 
\end{proof}

\begin{prop}
Given an exact sequence,
\begin{center}
\begin{tikzcd}
0 \arrow[r] & \F_1 \arrow[r] & \F_2 \arrow[r] & \F_3 \arrow[r] & 0
\end{tikzcd}
\end{center}
Then,
\begin{enumerate}
\item if $\F_1$ and $\F_3$ are $m$-regular then $\F_2$ is $m$-regular
\item if $\F_1$ is $(m+1)$-regular and $\F_2$ is $m$-regular then $\F_3$ is $m$-regular
\item if $\F_2$ is $m$-regular and $\F_3$ is $(m-1)$-regular then $\F_1$ is $m$-regular
\item $\reg(\F_1) \le \max \{ \reg(\F_2), \reg(\F_3) + 1 \}$
\item $\reg(\F_2) \le \max \{ \reg(\F_1), \reg(\F_2) \}$
\item $\reg(\F_3) \le \max \{ \reg(\F_1) - 1, \reg(\F_2) \}$
\end{enumerate}
\end{prop}

\begin{proof}
Consider the long exact sequence,
\begin{center}
\begin{tikzcd}
H^i(X, \F_1(m-i)) \arrow[r] & H^i(X, \F_2(m-i)) \arrow[r] & H^i(X, \F_3(m-i)) \arrow[r] & H^{i+1}(X, \F_1(m-i))
\end{tikzcd}
\end{center}
DO THIS
\end{proof}

\begin{prop}
Consider a coherent resolution,
\begin{center}
\begin{tikzcd}
0 \arrow[r] & \F_n \arrow[r, "\d_n"] & \F_{n-1} \arrow[r, "\d_{n-1}"] & \arrow[r] & \cdots \arrow[r] & \F_1 \arrow[r, "\d_1"] & \F_0 \arrow[r, "\d_0"] & \F \arrow[r] & 0
\end{tikzcd}
\end{center}
with each $\F_j$ is $(m+j)$-regular. Then $\F$ is $m$-regular and $H^0(X, \F_0(m)) \onto H^0(\F(m))$. 
\end{prop}

\begin{proof}
Given $H^i(X, \F_j(m + j - i)) = 0$ for all $i > 0$ and $j \ge 0$. We want to show that $H^i(X, \F(m-i)) = 0$. DO THIS
\end{proof}

\begin{prop}
A coherent sheaf $\F \in \Coh{X}$ is $m$-regular iff there exists a resolution,
\begin{center}
\begin{tikzcd}
0 \arrow[r] & \struct{X}(-m-(n+1))^{\oplus a_{n+1}} \arrow[r] & \cdots \arrow[r] & \struct{X}(-m-1)^{\oplus a_1} \arrow[r] & \struct{X}(-m)^{\oplus a_0} \arrow[r] & \F \arrow[r] & 0
\end{tikzcd}
\end{center}
\end{prop}

\begin{prop}
Let $\F \in \Coh{X}$ and $\E$ a vector bundle. If $\F$ is $m$-regular and $\E$ is $\ell$-regular then $\F \ot \E$ is $(m+\ell)$-regular.
\end{prop}

\begin{proof}
We apply the resultion property to $\F$ and then applying $- \ot \E$ gives a resultion of $\F \ot \E$ since $\E$ is flat. Then we apply the previous proposition. 
\end{proof}

\begin{cor}
If $\E$ is a an $m$-regular vector bundle then,
\begin{enumerate}
\item $\E^{\ot r}$
\item $\bigwedge^r \E$
\item $S^r \E$ (for characteristic zero).
\end{enumerate}
all are $(rm)$-regular. 
\end{cor}

\begin{proof}
Regularity of $\E^{\ot r}$ is immediate. Then consider the exact sequence,
\begin{center}
\begin{tikzcd}
0 \arrow[r] & I \arrow[r] & \E^{\ot r} \arrow[r] & \bigwedge^r \E \arrow[r] & 0
\end{tikzcd}
\end{center}
which has a section (CHECK)
\end{proof}

\begin{defn}
Let $X$ be a projective variety and $\L$ a globally generated line bundle. Then $\F$ is $m$-regular with respect to $\L$ if,
\[ H^i(X, \F \ot \L^{\ot (m-i)}) = 0 \]
for all $i > 0$.
\end{defn}


\begin{prop}[Green's Theorem]
Let $W \subset H^0(X, \struct{X}(d))$ be a codimension $n$ basepoint-free linear system. Then for $k \ge c$,
\[ \zeta_k : W \ot H^0(X, \struct{X}(k)) \onto H^0(X, \struct{X}(d+k)) \]
is surjective. 
\end{prop}

\begin{proof}
Consider $W \ot \struct{X}$ then there is a map,
\[ W \ot \struct{X} \onto \struct{X}(d) \]
which is surjective as a map of sheaves since $W$ is base-point free. Let $\M_W$ be its kernel. Then surjectivity is equivalent to $H^1(X, \M_W(k)) = 0$. Similarly, define,
\begin{center}
\begin{tikzcd}
0 \arrow[r] & \M_d \arrow[r] & H^0(X, \struct{X}(d)) \ot \struct{X} \arrow[r] & \struct{X}(d) \arrow[r] & 0
\end{tikzcd}
\end{center}
Wind that $\M_d$ is $1$-regular and $\bigwedge^k \M_d$ is $k$-regular. Since $\codim{W} = c$ we have,
\begin{center}
\begin{tikzcd}
0 \arrow[r] & \M_W \arrow[r] & \M_d \arrow[r] & \struct{X}^{\ot c} \arrow[r] & 0
\end{tikzcd}
\end{center}
Using the Egan-Northcott complex we have $\M_W$ is $(c+1)$-regular. If $k \ge c$ then $\M_W$ is $(k+1)$-regular and thus,
\[ H^1(X, \M_W(k)) = H^1(X, \M_W(k+1-1)) = 0 \]
\end{proof}

\begin{conj}[Fujita]
Let $X$ be a smooth projective varitety $\dim{X} = n$. Let $D$ be an ample divisor. Then,
\begin{enumerate}
\item $k \ge n+1$ implies that $K_X + k D$ is basepoint free
\item $k \ge n+2$ implies that $K_X + k D$ is very ample.
\end{enumerate}
\end{conj}

\begin{rmk}
This is true for curves, surfaces, and projective spaces. 
\end{rmk}

\begin{rmk}
$h^0$ can be hard to compute but $\chi$ is easier. If $H^i = 0$ for $i > 0$ then $\chi = h^0$.
\end{rmk}

\begin{example}
Let $X \subset \P^r$. What is the dimension of quadtric supersurfaces containing $X$. Consider $h^0(\I_X(2))$. We have,
\begin{center}
\begin{tikzcd}
0 \arrow[r] & \I_X(2) \arrow[r] & \struct{\P^r}(2) \arrow[r] & \struct{X}(2) \arrow[r] & 0
\end{tikzcd}
\end{center}
Therefore, we need vanishing $H^1(\I_X(2)) = 0$ to compute $h^0(\I_X(2))$. 
\end{example}

\begin{example}
Let $\ell \subset \P^3$ be a line. What is the dimension of degree $d$ surfaces containing $\ell$. Since $\ell$ is a complete intersection $\ell = H_1 \cap H_2$ for two hyperplanes. We have a Kozul resolution,
\begin{center}
\begin{tikzcd}
0 \arrow[r] & \struct{}(-2) \arrow[r] & \struct{}(-1) \oplus \struct{}(-1) \arrow[r] & \struct{} \arrow[r] & 0
\end{tikzcd}
\end{center}
By the previous result, $H^0(\struct{}) \onto H^0(\struct{\ell})$ is surjective and thus $H^1(\I_\ell(d)) = 0$. Alternatively, the resultion gives that $\I_{\ell}$ is $0$-regular.
\end{example}

\begin{example}
Noether-Lefschetz: $\Pic{S_d} = \Z$ for very general hypersurface $S_d \subset \P^3$ of degree $d$. However, if $S_d \supset \ell$ then it is not very general. 
\end{example}

\begin{example}
Let $H_1, \dots, H_e \subset \P$ be hypersurfaces with $\deg{H_i} = d_i$ and a complete intersection,
\[ X = H_1 \cap \cdots \cap H_e \]
Then,
\[ \reg(\I_X) = d_1 + \cdots + d_e - e + 1 \]
\end{example}

\section{Andres: Moduli of Vector Bundles on Curves}

Let $C$ be a curve a smooth projective curve over $k$. Vector bundles on $C$ vary in continuous families.

\begin{example}
If $C$ is an elliptic curve then there is a bijection between,
\[ C(k) \iso \{ \text{rank 1 vector bundles of degree 1} \} \]
via the map,
\[ p \mapsto \struct{C}(p) \cong \I_p^\vee \]
\end{example}

Let $\Sigma_{n,d}$ be the set of all vector bundle s on $C$ of fixed rank $n$ and degree $d$. Assume that $(n,d) = 1$. We want that $\Sigma_{n,d}$ is the $k$-points of some projective variety.
\bigskip\\
Let $T$ be a varierty or a scheme, we can consider a vector bundle $\E$ on $T \times C$ this gives a map $T(k) \to \Sigma_{n,d}$ via $t \mapsto \E_{C_t} \in \Sigma_{n, d}$. Therefore, we use this as the functor of points of the desired variety.
\bigskip\\
Consider the finest topology on $\Sigma_{n,d}$ such that for all $T$ and all $\E$ on $C \times T$ the induced map $T(k) \to \Sigma_{n,d}$ is continuous. 

\section{Brill-Noether Theory}

Let $C$ be a smooth projective curve of genus $g$. Then we want to consider the space of line bundles $\L$ on $C$ with $V \subset H^0(C, \L)$ of dimension $r+1$ giving a map $C \rat \P^r$ of degree $d$. We get a moduli space $G^r_d(C)$. We ask the following questions:

\begin{enumerate}
\item when is $G^r_d(C)$ nonempty
\item what is the dimension of $G^r_d(C)$
\item how many components does $G^r_d(C)$ have and are they equidimensional?
\end{enumerate}

\begin{defn}
The Brill-Noether number,
\[ \rho = g - (r+1) (g - d + r) \]
is the ``expected dimension'' of $G^r_d(C)$ for a general curve $C$.
\end{defn}

\begin{defn}
There is a universal fibration $\G^r_d \to \M_g$ of the Brill-Noether moduli spaces. 
\end{defn}

\begin{theorem}[Brill-Noether]
There is an open locus of $\M_g$ such that if,
\begin{enumerate}
\item $\rho < 0$ then $\G^r_d|_U$ is empty
\item $\rho \ge 0$ then $\G^r_d |_U$ has constant fiber dimension $\rho$ and is smooth
\item $\rho > 0$ then $\G^r_d$ has connected fibers (over all of $\M_g$).
\end{enumerate}
\end{theorem}

\begin{rmk}
If $\rho \ge 0$ then $G^r_d(C)$ is nonempty for all $C$ but need not be smooth or of the correct dimension. 
\end{rmk}

\newcommand{\g}{\mathfrak{g}}
\newcommand{\Fitt}{\mathrm{Fitt}}

\begin{example}
Hyperelliptic curves have nontrivial $\g^1_2$ but
\[ \rho = g - 2(g - 1) = 2 - g \]
is negative for large $g$ . 
\end{example}

\begin{defn}
Consider the space
\[ W^r_d(C) = \{ \L \mid \L \text{ line bundle wth } \deg{\L} = d \text{ and } \dim{H^0(C, \L)} \ge r + 1 \} \]
Then clearly there is a map $\beta : G^r_d(C) \to W^r_d(C)$.
\end{defn}

\subsection{Definition of Moduli Spaces}

\newcommand{\rk}{\mathrm{rk}}
\renewcommand{\Pic}{\mathrm{Pic}}

\begin{defn}
Let $F_1, F_2$ be free modules of finite rank over $R$ and consider,
\begin{center}
\begin{tikzcd}
F_1 \arrow[r, "\varphi"] & F_2 \arrow[r] & M \arrow[r] & 0
\end{tikzcd}
\end{center}
Then the $a^{\text{th}}$ fitting ideal $\Fitt_a(M)$ is the ideal generated by the $(\rk F_2 - a) \times (\rk F_@ - a)$ minors of the matrix representing $\varphi$. This is independent of the presentation.
\end{defn}

\begin{defn}
Using the universal line bundle $\L$ on $C \times \Pic_C^d$ we define,
\[ W^r_d(C) = \Fitt_{g - d + r  -1} (R^1 \nu_* \L) \]
where $\nu : C \times \Pic_C^d \to \Pic_C^d$. 
\end{defn}

\begin{rmk}
Notice that $R^1 \nu_* \L$ has fibers $H^1(C, L)$ over the point $[L]$ for $L$ of degree $d$. Choose high enough degree divisor $\Gamma$ on $C$ we get,
\begin{center}
\begin{tikzcd}
0 \arrow[r] & L \arrow[r] & L(\Gamma) \arrow[r] & L(\Gamma) / L \arrow[r] & 0
\end{tikzcd}
\end{center}
Then the long exact sequence gives,
\begin{center}
\begin{tikzcd}
0 \arrow[r] & H^0(C, L) \arrow[r] & H^0(C, L(\Gamma)) \arrow[r, "\gamma"] & H^0(C, L(\Gamma)/L) \arrow[r] & H^1(C, L) \arrow[r] & 0
\end{tikzcd}
\end{center}
Then by Riemann-Roch $h^0(C, L(\Gamma)) = d - g + 1 + m$ and $h^0(C, L(\Gamma)/L) = m$ where $\deg{\Gamma} = m$.
Then we have,
\[ |W^r_d(C)| = \{ L \in \Pic^d \mid \rank \gamma \le m - (g - d + r - 1) - 1 = m - g + d - r \} \]
which is exactly the conditions of the fitting ideal. 
\end{rmk}

\begin{rmk}
Naive dimension count for $W^r_d(C)$ is,
\[ \dim{\Pic_C^d} - \# \{ \text{minors} \} = g - (m - (m - g + d - r))(d - g + 1 + m - (m - g + d - r)) = g - (r+1)(g-d+r) = \rho \] 
\end{rmk}

\subsection{Petri's Condition}

Let $C$ be a smooth projective curve. We say that $C$ satisfies (P) if for all $\L \in \Pic{(C)}$,
\[ \mu_{\L} : H^0(C, \L) \ot H^0(C, \omega_C \ot \L^{\ot -1}) \to H^0(C, \omega_C) \]
is injective. 

\begin{theorem}[Gieseker]
Petri's condition holds for a general $C$. 
\end{theorem}

\begin{cor}
\begin{enumerate}
\item If $\rho < 0$, for a general $C$, then $G^r_d$ and $W^r_d$ are empty
\item if $\rho \ge 0$, for a general $C$, then $G^r_d$ is smooth of diemension $\rho$ and $W^r_d$ is smooth away from $W^{r+1}_d$ and has dimension $\rho$
\item if $\rho \ge 1$, for a general $C$, then $G^r_d$ and $W^r_d$ are irreducible. 
\end{enumerate}
\end{cor}

\begin{proof}
Consider infinitessimal deformation theory, given $(L, V) \in G^r_d(\CC)$ we consider,
\[ T_{(L, V)} G^r_d = \{ (L', V') \mid L' \text{ extending } L \text{ and } V' \subset H^0(L') \text{ free restricting to } V \} \]
The tangent space fits into a sequence,
\begin{center}
\begin{tikzcd}
0 \arrow[r] & T_{(L, V)} \beta^{-1}(L) \to T_{(L,V)} G^r_d \arrow[r, "\beta"] & T_{L} \Pic^d 
\end{tikzcd}
\end{center}
and recall that $T_{L} \Pic^d \iso H^1(C, \struct{C})$.
\bigskip\\
When does $\phi \in T_L \Pic^d$ lie in the image of $T \beta$? We can represent $\phi$ by a Cech 1-cocycle $\phi_{\alpha \beta} \in \struct{C}(U_{\alpha \beta})$. For a given $[L] \in H^1(C, \struct{C}^\times)$ represented by a cocycle $\{ g_{\alpha \beta} \}$ then we can represent the lift with a given class by the cocycle $\{ g_{\alpha \beta}' = g_{\alpha \beta} (1 + \epsilon \phi_{\alpha \beta}) \}$. There needs to exist an extension $(L, s)$ to $(L', s')$ for $s \in W \subset H^0(L)$. For $s'$ to be an extension of $s$ we should have,
\[ s'_\alpha = s_\alpha + \epsilon t_\alpha \]
for $t_\alpha \in \struct{C}(U_\alpha)$ and we want $s'_\beta = g'_{\alpha \beta} s_\alpha'$. This gives,
\[ - \phi_{\alpha \beta} s_\alpha = t_\alpha - g_{\beta \alpha} t_\beta \]
Therefore we require that $- \phi \cdot s$ is zero in $H^1(L)$. 
\bigskip\\
Thus $\phi \in \im{T \beta}$ is zero prcisely when $\phi \cdot W \subset H^1(L)$ is zero. Therefore,
\begin{align*}
\im{T \beta} & = \{ \phi \in H^1(\struct{C}) \mid \phi \cdot W = 0 \} = \{ \phi \in H^1(\struct{C}) \mid \forall s : \left< \phi W, s \right> = 0 \}
\\
& = \{ \phi \in H^1(\struct{C}) \mid \forall s : \left< \phi, W \cdot s \right> = 0 \} 
\\
& = \{ \phi \in H^1(\struct{C}) \mid \left< \phi, t \right> = 0 \} 
\end{align*}
over all $t \in \im{(H^0(L) \ot H^0(\omega_C \ot L^{-1}) \to H^0(\omega_C))}$. Therefore,
\[ \dim{T_{(L, W)} G^r_d} = \dim{\im{T \beta}} + (r+1) (h^0(L) - (r+1)) \]
using that $\beta^{-1}(L)$ is a Grasmannian and thus,
\[ T_{(L, V)} \beta^{-1}(L) = \Hom{}{W}{H^0(L)/W} \]
Therefore,
\begin{align*}
\dim{T_{(L, W)} G^r_d} & = g - \dim{\im{\mu_L}} + (r + 1) (h^0(L) - (r+1))
\\
& = g - ((r+1) h^0(\omega_C \ot L^{-1}) - \ker{\mu_L}) + (r+1) (h^0(L) - (r+1))
\\
& = g + (r+1) (h^0(L) - h^0(\omega_C \ot L^{-1}) - (r+1)) + \ker{\mu_L}
\\
& = g + (r+1) (g - g - r) + \ker{\mu_L}
\\
& = \rho + \ker{\mu_L} 
\end{align*}
Therefore, $G^r_d$ has tangent space of the expected imension iff $\mu_L$ is injective. We already know $\dim{G^r_d} \ge \rho$ from the naive dimension count. Then $G^r_d$ is smooth at $(L, W)$ of dimension $\rho$ iff $\mu_L |_W$ is injective. 
\bigskip\\
Then $\beta : G^r_d \to W^r_d$ is an siomrophism away from $W^{r+1}_d$ and $W^r_d \sm W^{r+1}_d$ is dense in $W^r_d$. Furthermore, $\mu_L$ is injective implies that $W^r_d$ is smooth of $\dim = \rho$ away from $W_d^{r+1}$. 
\end{proof}

\subsection{Riemann-Roch in Geometric Terms}

Let $D$ be an effective divisor. Then,
\[ r(D) = h^0(D) - 1 \]
is the number of independent relations between the canonical image $\phi(D)$ meaning under the canonical embedding $\phi : C \to \P^{g-1}$.

\begin{example}
If $C$ is hyperelliptic and $D$ is degree $d$ effective divisor with $r(D) = r$. Then,
\[ D \sim r \g^1_2 + p_1 + \cdots + p_{d-2r} \]
\end{example}

\begin{example}
If $g = 4$ and $d = 3$ and $r = 1$ then $\rho = 0$. If $C$ is hyperelliptic then,
\[ D = \g^1_2 + p \]
and therefore $W^1_3 \cong C$ is $1$-dimensional. If $C$ is not hyperelliptic then under the canonical embedding $C \embed \P^3$ we have $C = Q \cap S$ for a quadric $Q$ and a cubic $S$ surface. Then if $D$ is degree $3$ and $r(D) = 1$ then $\phi(D)$ should be colinear and hence the line is on $Q$. Therefore, $W^1_3$ is the set of linear equivalence classes of rullings on $Q$ so $\# W^1_3 = 1$ if $Q$ is a cone and $\# W^1_3 = 2$ if $Q$ is smooth. 
\end{example}

\section{Oct 11. Brill Noether Theory on K3 Surfaces, Lazarsfeld-Mukai bundles}

\begin{defn}
$X / \CC$ is a $K3$-surface if it is a smooth, projective variety of $\dim{X} = 2$ such that $K_X = \Omega^2_{X/K} \cong \struct{X}$ and $H^1(X, \struct{X}) = 0$. 
\end{defn}

\begin{example}
Let $X \subset \P^3$ be a smooth quartic then $\omega_X = \omega_{\P^3} \ot \struct{X}(4) \cong \struct{X}$.
\end{example}

\begin{lemma}
Let $X$ be a K3 surface then $\chi(X, \struct{X}) = 2$.
\end{lemma}

\begin{proof}
$\chi = h^0 - h^1 + h^2 = 2 h^0 = 2$. 
\end{proof}

\begin{prop}
Let $C \subset X$ be a smooth irreducible curve of genus $\ge 1$ then $|C|$ has no base points and defines a morphism $\phi : X \to \P^g$ such that $\phi |_C : X \to \P^{g-1}$ is the canonical one.
\end{prop}

\begin{proof}
The sequence,
\begin{center}
\begin{tikzcd}
0 \arrow[r] & \struct{X} \arrow[r] & \struct{X}(C) \arrow[r] & \struct{C}(C) \arrow[r] & 0
\end{tikzcd}
\end{center}
and use that $H^1(X, \struct{X}) = 0$ and thus,
\[ H^0(X, \struct{X}(C)) \onto H^0(C, \struct{C}(C)) = H^0(C, \omega_C) \]
\end{proof}

\begin{lemma}
Let $C \subset X$ be a smooth irreducible curve with $g \ge 1$ and $\L = \struct{X}(C)$. Then $c_1(\L)^2 = 2 g - 2$ and $h^0(X, \L) = g+1$. Also, if $\ell \ge 1$ then $h^0(X, \L^{\ell}) = (\ell^2/2) c_1(\L)^2 + 2 = (g - 1) \ell^2 + 2$. 
\end{lemma}

\begin{proof}
Riemann-Roch gives,
\[ 2 g - 2 = C \cdot (C + K_X) = \L^2 \]
Then Riemann-Roch for surfaces gives,
\[ \chi(X, \L) = \tfrac{1}{2} c_1(\L)^2 + 2 = g  +1 \]
also $h^2(X, \L) = h^0(X, \L^\vee) = 0$. Therefore,
\[ h^0(X, \L) \ge g + 1 \]
Furthermore, $h^1(X, \L) = 0$ by Kodaira vanishing or something else. 
\end{proof}

\begin{theorem}
Let $C \subset X$ be a smooth irreducible curve of genus $g \ge 2$. Suppose every divisor in $|C|$ is reduced and irreducible then,
\begin{enumerate}
\item for all $\L \in \Pic{(C)}$ the number $\rho(\L) = g(C) - h^0(\L) h^1(\L) \ge 0$ 
\item Petri's condition holds for a general member $C' \in |C|$.
\end{enumerate}
\end{theorem}

\begin{rmk}
The assumption on the linear series is essential. For a counterexample, let $|C| = |n D|$ with $D \subset X$ a curve of genus $g \ge 2$ and $n \ge 2$. Let $\L = \struct{X}(D)|_D$. Claim that $\rho(\L) < 0$. Consider the exact sequence,
\begin{center}
\begin{tikzcd}
0 \arrow[r] & \struct{X}(D-C) \arrow[r] & \struct{X}(D) \arrow[r] & \struct{C}(D) \arrow[r] & 0
\end{tikzcd}
\end{center}
\end{rmk}

\subsection{Lazarsfeld-Mukai Bundle}

From now on, $X$ is a K3 surface and $C \subset X$ is a smooth irreducible curve. Recall that $V^r_d(C) \subset \Pic^d(C)$ is the open subset of $W^r_d(C)$ consisting of line bundles $\L$ such that,
\begin{enumerate}
\item $h^0(\L) = r + 1$ and $\deg{\L} = d$
\item $\L$ and $\omega_C \ot \L^\vee$ are globally generated. 
\end{enumerate}

\begin{defn}
Fix $\L \in V^r_d(C)$. Let $\iota : C \embed X$ be the inclusion. For each pair $(C, \L)$ define, $\F_{C,\L}$ as the kernel of,
\[ \ev : H^0(\L) \ot_{\CC} \struct{X} \onto \iota_* \L \]
\end{defn}

\begin{lemma}
Let $\E$ be a vector bundle on $X$ with a surjection $\varphi : \E |_C \onto \L$. Then consider the exact sequence,
\begin{center}
\begin{tikzcd}
0 \arrow[r] & \F \arrow[r] & \E \arrow[r] & \iota_* \L \arrow[r] & 0
\end{tikzcd}
\end{center}
Then $\F$ is locally free.
\end{lemma}

\begin{proof}
Work locally, assume $\L = \struct{X}$. Then there is a locally free resolution,
\begin{center}
\begin{tikzcd}
0 \arrow[r] & \struct{X}(-C) \arrow[r] & \struct{X} \arrow[r] & \struct{C} \arrow[r] & 0
\end{tikzcd}
\end{center}
Therefore the homological dimension of $\L$ is $\le 1$ and therefore the homological dimension of $\F$ is $0$ and thus $\F$ is locally free.
\end{proof}

\begin{cor}
The Lazarsfeld-Mukai bundle $\F_{C,\L}$ is a vector bundle.
\end{cor}

\begin{proof}
Consider the sequence,
\begin{center}
\begin{tikzcd}
0 \arrow[r] & \F \arrow[r] & H^0(\L) \ot_{\CC} \struct{X} \arrow[r] & \iota_* \L \arrow[r] & 0
\end{tikzcd}
\end{center}
and apply the previous lemma.
\end{proof}

\begin{lemma}
Let $\F = \F_{C, \L}$. Then,
\begin{enumerate}
\item $\F^\vee$ is globally generated
\item $c_1(\F) = -[C]$ and $c_2(\F) = \deg{\F} = d$
\item $H^0(\F) = H^2(\F^\vee) = 0$ and $H^1(\F) = H^2(\F^\vee) = 0$ and,
\[ h^0(\F^\vee) = h^0(\L) + h^1(\L) \]
\end{enumerate}
\end{lemma}

\begin{proof}
Consider the sequence,
\begin{center}
\begin{tikzcd}
0 \arrow[r] & H^0(\L)^\vee \ot \struct{X} \arrow[r] & \F^\vee \arrow[r] & \iota_*(\omega_C \ot \L^\vee) \arrow[r] & 0
\end{tikzcd}
\end{center}
By assumption the third term is globally generated and $H^0(\F^\vee) \onto H^0(\m
_C \ot \L^\vee)$ because $H^1(X, \struct{X}) = 0$. Therefore, $\F^\vee$ is globally generated.
\bigskip\\
In general we have a formula,
\[ c_1(\iota_* \L) = [C] \quad c_2(\iota_* \L) = [C]^2 - \iota_* c_1(\L) = [C]^2 - (\deg{\L}) [pt] \]
Then from the exact sequence,
\begin{center}
\begin{tikzcd}
0 \arrow[r] & \F \arrow[r] & H^0(\L) \ot \struct{X} \arrow[r] & \iota_* \L \arrow[r] & 0
\end{tikzcd}
\end{center}
We get that,
\[ c_1(\F) = - c_1(\L) = -[C] \quad c_2(\F) = -c_1(\iota_* \L) c_1(\F) - c_2(\iota_* \L) = [C]^2  - [C]^2 + (\deg{\L}) [pt] = (\deg{\L}) [pt] \]
\end{proof}

\begin{lemma}
Let $\F = \F_{C, \L}$ then,
\[ \chi(\F \ot \F^\vee) = 8 - \Delta(\F) = 8 + C^2 - 4 \deg{\L} \]
\end{lemma}

\section{Oct 18}

\subsection{Proof of (a)}

\begin{theorem}[Main]
Let $C \subset X$ be a smooth irreducible curve of genus $g \ge 2$ on the K3 surface $X$. Assume that every divisor in the linear series $|C|$ is reduced and irreducible. Then,
\begin{enumerate}
\item for each $\L \in \Pic{(C)}$ we have $\rho(\L) \ge 0$
\item Petri's condition holds for a general element $C' \in |C|$.
\end{enumerate}
\end{theorem}

\begin{lemma}
Let $\L \in \Pic{(C)}$ for a smooth proper curve $C$ with $\deg{\L} \in (0, 2g - 2)$. There is a line bundle $\L = \L'(D)$ such that $\L'$ and $\omega_C \ot \L'^\vee$ are globally generated and $\rho(\L') \le \rho(\L)$.
\end{lemma}

\begin{proof}
Let $D_1$ be the divisor of base points of $\L$. Then $\L(-D_1)$ is globally generated because $|\L| = |\L(-D_1)| + D_1$. Let $D_2$ be the divisor of basepoints of $K_C - c_1(\L) + D_1$. Then $K_C - c_1(\L) + D_1 - D_2$ is base-point free. I claim that $\L(D_2 - D_1)$ is also globally generated. If $\L(D_2 - D_1 - P)$ does not drop dimension then by Riemann Roch $K_C - c_1(\L) + D_1 + P - D_2$ must increase dimension 
\end{proof}

\begin{proof}[Proof of (a)]
Suppose that $\rho(\L) < 0$ (REPLACE WITH BPF)

Let $\E = \F_{C, \L}^\vee$ which is a vector bundle since $\L \in V^r_d(C)$. We showed that,
\[ 2 h^0(X, \F \ot \F^\vee) \ge \chi(\F, \F) = 2 - 2 \rho(\L) \ge 4 \]
thus $\E$ has a nontrivial endomorphism $\varphi : \F \to \F$ meaning $\varphi \neq \lambda \id$. Choose a point $x \in X$ and let $\lambda$ be an eigenvalue of $\varphi(x)$. Then $\psi = \varphi - \lambda \id$ is nonzero but is not of full rank at $x$. Thus $\det{\psi} \in \Hom{X}{\det{\E}}{\det{\E}} = H^0(X, \struct{X})$ has a zero and hence is zero. Let $\E_1 = \im{\psi}$ and $\E_2 = \coker{\psi}$ so there is a sequence,
\begin{center}
\begin{tikzcd}
0 \arrow[r] & \E_1 \arrow[r] & \E \arrow[r] & \E_2 \arrow[r] & 0
\end{tikzcd}
\end{center}
so we have $c_1(\E) = c_1(\E_1) + c_1(\E_2)$ and $c_1(\E) = [C]$. Then if $c_1(\E_1)$ and $c_1(\E_2)$ are represented by nonzero effective divisors. We showed last time that $\E$ is globally generated and $H^0(X, \E^\vee) = 0$.  Thus since $\E \onto \E_i$ we see that $\E_i$ are globally generated so $c_1(\E_i) = [C_i]$ for some effective class $C_i$. (SHOW BOTH CLASSES ARE NONTRIVIAL).
Hence $C \sim C_1 + C_2$ contradicting the assumption on the linear system. 
\end{proof}

\subsection{Mukai's Theorem}

\begin{defn}
Let $X$ be a proper $k$-scheme. A vector bundle $\E$ on $X$ is \textit{simple} if,
\[ \Hom{X}{\E}{\E} = k \]
\end{defn}

\begin{rmk}
Simple vector bundles are indecomposable. If $X$ is geometrically irreducible then all line bundles are simple.
\end{rmk}


In this section, let $X$ be a (smooth projective) K3 surface over $\CC$. Therefore, all line bundles are simple.

\begin{rmk}
If $\E$ is simple, then by Serre duality using that $\omega_X \cong \struct{X}$,
\[ \Ext{2}{X}{\E}{\E} \cong \Ext{2}{X}{\E \ot \E^\vee}{\omega_X} = H^0(X, \E \ot \E^\vee)^\vee = \Hom{X}{\E}{\E}^\vee = \CC \]
\end{rmk}

\begin{defn}
Let $\M(X, r, c_1, c_2)$ be the moduli space of simple vector bundles on $X$ of rank $r$ and with Chern classes $c_1$ and $c_2$. 
\end{defn}

\begin{rmk}
Because the objects of $\M$ are simple, the stabilzers groups are $\Gm$ and hence $\M \to M$ is a $\Gm$-torsor over a coarse space $M$. 
\end{rmk}

\begin{rmk}
The moduli problem has tangent-obstruction theory at a point $\E \in \M$,
\[ T^i = \Ext{i}{X}{\E}{\E} \]
Therefore, since the fiber direction $B \Gm$ have trivial tangent direction we see that,
\[ T_{[\E]} M = \Ext{1}{X}{\E}{\E} \]
\end{rmk}

\begin{rmk}
The cup product gives a nondegenerate holomorphic $2$-form on $M$ defined by,
\[ \Ext{1}{X}{\E}{\E} \times \Ext{1}{X}{\E}{\E} \to \Ext{2}{X}{\E}{\E} = \CC \]
Therefore, $M$ gives an example of a holomorphic symplectic variety. When $\dim{M} = 2$ it turns out that $M$ is also a K3 surface.
\end{rmk}

\begin{theorem}[Mukai]
The moduli space $M(X, r, c_1, c_2)$ is smooth.
\end{theorem}

\newcommand{\ob}{\mathrm{ob}}
\newcommand{\Def}{\mathrm{Def}}

\begin{proof}
By descent along the flat map $\M \to M$ it suffices to show that $\M$ is smooth. Alternatively we can develop directly tangent-obstruction theory for $M$. Either way, it suffices to show that obstruction classes $\ob(E) \in \Ext{2}{X}{\E}{\E}$ vanish. Let $\E \in \M$ be a closed point (corresponding to a simple vector bundle $\E$ on $X$) and a small extension of Artin local $k$-algebras $A \subset B$,
\begin{center}
\begin{tikzcd}
\Def_{\M}(B) \arrow[d, "\det"] \arrow[r] & \Def_{\M}(A) \arrow[r, "\ob"] \arrow[d, "\det"] & \Ext{2}{X}{\E}{\E} \arrow[d, "\tr"]
\\
\Def_{\Pic}(B) \arrow[r] & \Def_{\Pic}(A) \arrow[r, "\ob"] & \Ext{2}{X}{\struct{X}}{\struct{X}}
\end{tikzcd}
\end{center}
but $\Pic_X$ is smooth so we see that $\tr \circ \ob = 0$. However, using Serre duality,
\begin{center}
\begin{tikzcd}
\Ext{2}{X}{\E}{\E} \arrow[r, "\sim"] \arrow[d, "\tr"] & H^0(X, \E \ot \E^\vee) \arrow[d, "\tr"] \arrow[r, "\sim"] & \Hom{X}{\E}{\E} \arrow[d, "\tr"]
\\
\Ext{2}{X}{\struct{X}}{\struct{X}} \arrow[r, "\sim"] & H^0(X, \struct{X}) \arrow[r, equals] & H^0(X, \struct{X})
\end{tikzcd}
\end{center}
but since $\E$ is simple the map $\tr : \Hom{X}{\E}{\E} \to H^0(X, \struct{X})$ is an isomorphism. Thus $\tr \circ \ob = 0$ implies that $\ob = 0$.
\end{proof}

\subsection{Proof of (b)}

\newcommand{\cV}{\mathcal{V}}

\begin{proof}[Sketch of Proof of (b)]
Recall that for $\L \in V^r_d(C')$ we know that the tangent space of $V^r_d(C')$ and $G^r_d(C')$ are isomorphic and hence injectivity of $\mu_{\L}$ is equivalent to $V^r_d(C')$ being smooth of the expected dimension.
Consider the variety,
\[ \cV^r_d = \{ (C', \L) \mid C' \in |C| \text{ smooth curve and } \L \in V^r_d(C') \} \]
and denote,
\[ \pi^r_d : \cV^r_d \to |C| \]
the natural map. By generic smoothness, to show that $V^r_d(C')$ is smooth (and hence $\mu_{\L}$ is injectve) for a generic $C'$ is suffices to show that $\cV^r_d$ is smooth.
\bigskip\\
Consider the fibration,
\[ \pi : \G \to M = M(X, r+1, [C], d) \]
where $\G$ is the space of pairs $(\E, V)$ for a simple vector bundle $\E$ of rank $r+1$ of $X$ with $c_1(\E) = [C]$ and $c_2(\E) = d$ and $V \subset H^0(X, \E)$ of dimension $r+1$. By Mukai's theorem $M$ is smooth and hence $\G$ is smooth since it is a Grassmannian bundle so we can compute the tangent space at the point $\E = \F_{C, \L}^\vee$ to get,
\begin{align*}
\dim{\G} & = \dim{M} + (r+1)(\dim{H^0(X, \E)} - r - 1) 
\\
& = \Ext{1}{X}{\E}{\E} + (r+1)(\dim{H^0(X, \E)} - r - 1) 
\\
& = 2 \rho(r,d,g) + (r+1) (g - d + r) = g + \rho(r, d, g) 
\end{align*}
using a lemma we proved last time. Thus it suffices to show that $\cV^r_d$ has an open embedding in $\G$.
\bigskip\\
Let $U \subset \G$ denote the open set consisting of pairs $(E, V)$ such that,
\begin{enumerate}
\item $E$ is globally generated and $H^1(X, E) = H^2(X, E) = 0$
\item the natural map $\ev : V \ot_{\CC} \struct{X} \to \E$ drops rank on a smooth curve $C_V$ and $\coker{\ev}$ is a line bundle on $C_V$.
\end{enumerate}
Then we have exact sequences,
\begin{center}
\begin{tikzcd}
0 \arrow[r] & \E^\vee \arrow[r] & V^\vee \ot \struct{X} \arrow[r] & \L_V \arrow[r] & 0
\\
0 \arrow[r] & V \ot \struct{X} \arrow[r] & \E \arrow[r] & \omega_C \ot \L_V^\vee \arrow[r] & 0
\end{tikzcd}
\end{center}
(WHY) SHOW THE EQUAIVALENCE
\end{proof}

\section{Oct 25}

\subsection{Setup}

$X$ is a smooth projective surface over $\CC$ and $L$ a line bundle on $X$. Then we have the following two facts,
\begin{enumerate}
\item[adjunction] if $C \sub X$ is an effective curve then,
\[ p_a(C) - 1 = \tfrac{1}{2} C \cdot (C + K_X) \]
\item[Hodge index]
if $D, H$ are divisors on $X$ with $H^2 \ge 0$ and $D \cdot H = 0$ then $D^2 \le 0$ and $D^2 = 0$ iff $D \sim 0$.
\end{enumerate}

\begin{rmk}
If $C$ is integral then $p_a(C) \ge 0$.
\end{rmk}

\subsection{Linear Systems}

\newcommand{\Bs}{\mathrm{Bs}}

Let $V \subset H^0(X, L)$ be a linear system. Then the base locus is,
\[ \Bs(V) = \{ p \in X \mid \forall s \in V : s(p) = 0 \in L(p) \} \]
Note we use the notation $L(p) = L_p / \m_p L_p$. Consider the map,
\[ \Phi_V : X \sm \Bs(V) \to \P(V) \]
Note that $p \in \Bs(V)$ iff $H^0(X, L) \to H^0(Z, \L|_{Z})$ is zero. 

\begin{prop}
$\Bs(V) = \empty$ then $\Phi_V : X \to \P(V)$ is a closed embedding iff,
\begin{enumerate}
\item $V$ separates points meaning $\forall p,q \in X$ with $p \neq q$ there is $s \in V$ with $s(p) = 0$ and $s(q) \neq 0$ or vice versa
\item $V$ separates tangent directions,
\[ \{ s \in V \mid s_p \in \m_p L_p \} \]
generates $\m_p L_p / \m_p^2 L_p$ as a vector space.
\end{enumerate}
\end{prop}

\begin{rmk}
We can reformulate the conditions as follows,
\begin{enumerate}
\item $p,q \in X$ with $p \neq q$ let $Z = \{ p,q \}$ redced thne,
\[ H^0(X, L) \to H^0(Z, L|_Z) \]
is surjective
\item $p \in X$ and $t \in \m_p / \m_p^2$ and $Z$ is cut out by $\m_p^2 + (t)$ locally then,
\[ H^0(X, L) \to H^0(Z, L|_Z) \]
is surjective. 
\end{enumerate}
\end{rmk}

\begin{theorem}[Reider]
Let $L$ be a nef line bundle,
\begin{enumerate}
\item let $(L \cdot L) \ge 5$. Let $p$ be a base point of $|K_X + L|$. Then there is n effective divisor $D \subset X$ wuth $p \in D$ such that either,
\begin{enumerate}
\item $(L \cdot D) = 0$ and $D^2 = - 1$
\item $(L \cdot D) = 1$ and $D^2 = 0$ 
\end{enumerate}
\item $(L \cdot L) \ge 10$. Let $p \in X$ and $q \in X$ with $p \neq q$ which are not separated by $|K_X + L|$ or $q \in \m_p / \m_p^2$ and $p,q$ not separated by $|K_X + L|$. Then there is an effective divisor $D \subset X$ with $Z_{p,q} \subset D$ such that one of the three conditions holds,
\begin{enumerate}
\item $(L \cdot D) = 0$ and $(D \cdot D) \in \{ -1, - 2 \}$ 
\item $(L \cdot D) = 1$ and $(D \cdot D) \in \{ 0, -1 \}$
\item $(L \cdot D) = 2$ and $(D \cdot D) = 0$.
\end{enumerate}
\end{enumerate}
\end{theorem}

\begin{example}
Let $X = \P^2$ and $L = \struct{X}(2)$ then $(L \cdot L) = 4$ and $K_X = \struct{X}(-3)$ then $K_X + L = \struct{X}(-1)$ which has every point as a base point. Let $D \subset X$ and $D \in | k H|$ then $D^2 = k^2$ but $L \cdot D = 2k$ so these cannot satisfy the conclusion of the theorem. This shows that $(L \cdot L) \ge 5$ is strict in the theorem.
\end{example}

\subsection{Fujita's Conjecture}


\begin{conj}[Fujita 1985]
Let $X$ be a compact complex manifold of dimension $n$ and $L$ an ample line bundle.
\begin{enumerate}
\item $m \ge n + 1 \implies K_X \ot L^{\ot m}$ is base point free
\item $m \ge n + 2 \implies K_X \ot L^{\ot m}$ is very ample.
\end{enumerate}
\end{conj}

\begin{proof}[Proof in the $n = 2$ case]
\begin{enumerate}
\item We know $X$ is projective use Nkai-Moishecon. Let $(L \cdot L) \ge 1$ then $m L$ is nef if $m \ge 3$ then $(m L \cdot m L) \ge 3^2 \ge 5$. Then is $p$ is a base point of $|K_X + K|$ then there is an effective divisor $D \subset X$ with $p \in D$ such that $(m L \cdot D) = 0$ and $D^2 = 1$ or $(m L \cdot D) = 1$ which is not possible since $m > 1$ and hence we have $(L \cdot D) = 0$ and $D^2 = -1$. We write,
\[ D = D_1 + \cdots + D_r \]
But $L$ is ample so $(D_i \cdot L) > 0$ and $D$ must have some component since $p \in D$ and thus $(D \cdot L) > 0$ giving a contradiction. 

\item Use the same sort of argument with the second part of Reider's theorem. 
\end{enumerate}
\end{proof}

\subsection{Pluricanonical Mappings}

Let $X$ be a surface of general type. Consider the pluricanonical maps,
\[ \Phi_m : X \rat \P(H^0(m K_X)) \]
defined by the complete linear system $|m K_X|$.

\begin{prop}
If $X$ is minimal then $K_X$ is nef and $K_X^2 \ge 1$.
\end{prop}

\begin{defn}
A $(-2)$-curve on $X$ is a smooth rational curve $C \subset X$ with $C^2 = -2$.
\end{defn}

\begin{prop}
If $X$ is minimal then $X$ has finitely many $-2$-curves. In fact, it is at most $\rho(X) - 1$. 
\end{prop}

\begin{theorem}[Bombieri]
Let $X$ be a minimal surface of general type. Let,
\[ F = \bigcup C \subset X \]
be the union of the $-2$-curves.

\begin{enumerate}
\item if $m \ge 4$ or $m \ge 3$ and $K_X^2 \ge 2$ then $\Phi_m$ is a morphism

\item if $m \ge 5$ or $m \ge 4$ and $K_X^2 \ge 2$ or $m \ge 3$ and $K_X^2 \ge 3$ then $\Phi_m$ is an embedding on $X \sm F$. 
\end{enumerate}
\end{theorem}

\begin{proof}
Let $L = (m-1) K_X$ is nef then $L \cdot L \ge 5$. Apply Reider's theorem. Let $p$ be a base point of $|K_X + L| = |m K|$. Then there is an effective divisor $D \subset X$ with $p \in D$ such that $(L \cdot D) = 0$ and $D^2 = 1$ since $L \cdot D = 1$ is impossible. Then,
\[ -1 = D^2 = D \cdot (D + K_X) = 2 p_a(D) - 2 \]
which is a contradiction. 
\end{proof}

\subsection{Bogomolov's Theorem}

\begin{theorem}[Bogomolov]
Let $E$ be a vector bundle of rank $e$ on a surface $X$. If $c_1(E)^2 > \frac{2 e}{e-1} c_2(E)$ then $E$ is $H$-unstable with respect to every ample class $H$.
\end{theorem}

\begin{rmk}
$c_2(E) \in H^4(X, \ZZ)$ so we view $c_2(E)$ as an integer under the canonical isomorphism $H^4(X, \ZZ) \iso \ZZ$ using that $X$ is oriented (as a complex manifold).
\end{rmk}

\subsubsection{Stability for Curves}

Let $C$ be a smooth projective irreducible curve. Let $E$ be a vector bundle on $C$. 

\begin{defn}
The slope,
\[ \mu(E) = \frac{\deg{E}}{\rank{E}} \]
where $\deg{E} = \deg{\det{E}}$.
\end{defn}

\begin{example}
Let $C = \P^1$ then $\mu(\struct{C}) = 0$ and $\mu(\struct{C}(1)) = 1$ and,
\[ \mu(\struct{C}(a) \oplus \struct{C}(b)) = \frac{a + b}{2} \]
\end{example}

\begin{defn}
Let $F \subset E$ be a coherent subsheaf. Then $F$ is locally free of constant rank almost everywhere. Then $c_1(F) := \det{F}^{\vee \vee}$ is a line bundle 
\end{defn}

\begin{defn}
The slope of a torsion-free sheaf $F$ is,
\[ \mu(F) = \frac{\deg{F}}{\rank{F}} \]
\end{defn}

\begin{defn}
$E$ is \textit{stable} if, for every $F \subset E$ with,
\[ 0 < \rank{F} < \rank{E} \]
we have $\mu(F) < F(E)$ and \textit{semistable} if $\mu(F) \le \mu(E)$.
\end{defn}

\begin{rmk}
It is trivial that line bundles are stable. 
\end{rmk}

\begin{example}
Let $C = \P^1$ then $\struct{} \oplus \struct{}(1)$ is unstable because $\struct{}(1) \subset \struct{} \oplus \struct{}(1)$,
\[ \mu(\struct{}(1)) > \mu(\struct{} \oplus \struct{}(1)) = \tfrac{1}{2} \]
\end{example}

\begin{theorem}
Let $E$ be a vector bundle on $C$ and $L$ a line bundle on $C$,
\begin{enumerate}
\item $E$ is (semi)-stable iff $E \ot L$ is (semi)-stable

\item $E$ is semistable, $\deg{E} < 0$ implies $H^0(C, E) = 0$ 

\item if $E$ is semi-stable then $\Sym{n}{E}$ is semi-stable for $n \ge 1$. 
\end{enumerate}
\end{theorem}

\begin{rmk}
The last statement is not true for stable instead of semi-stable or in positive characteristic. 
\end{rmk}

\subsubsection{Stability For Surfaces}

Let $H$ be an ample divisor on a surface $X$.

\begin{defn}
The $H$-slope is defined,
\[ \mu_H(E) := \frac{c_1(E) \cdot H}{\rank{E}} \]
\end{defn}

\begin{defn}
For $F \subset E$ we have $c_1(F) = \det{F}^{\vee \vee}$ is a reflexive sheaf of rank $1$ and hence is a line bundle on a surface. Then we can set,
\[ \mu_H(F) = \frac{c_1(F) \cdot H}{\rank{F}} \]
\end{defn}

\begin{defn}
We say $E$ is $H$-stable if for every $F \subset E$ with,
\[ 0 < \rank{F} < \rank{E} \]
if $\mu_H(F) < \mu_H(E)$ and semi-stable if $\mu_H(F) \le \mu_H(E)$. 
\end{defn}

\section{Nov. 1 Bogomolov's Theorem}

Let $(X, \L)$ be a polarized surface over $\CC$.

\begin{defn}
A sheaf $\F$ on $X$ is called torsion-free if for all $U \subset X$ open, the group $\F(U)$ is torsion-free module over $\struct{X}(U)$.
\end{defn}

\begin{rmk}
Recall that $\F^\vee = \shHom{\struct{X}}{\F}{\struct{X}}$. There is a natural morphism $\F \to \F^{\vee \vee}$.
\end{rmk}

\begin{prop}
$\F$ is torsion-free iff $\F \to \F^{\vee \vee}$ is injective.
\end{prop}

\begin{defn}
We say that $\F$ is reflexive if $\F \to \F^{\vee \vee}$ is an isomorpism.
\end{defn}

\begin{prop}
Any reflexive sheaf on a regular $\dim{X} \le 2$ scheme is locally free.
\end{prop}

\begin{example}
Let $p \in |X|$ be a closed point and $\I_p \embed \struct{X}$ the sheaf of ideals. Then $\I_p$ is torsion-free but not locally-free. 
\end{example}

\begin{defn}
The rank of a torsion-free sheaf $\F$ is defined to be,
\[ \rank{\F} = \ell(\F_{\et}) \]
where $\eta \in X$ is the generic point.
\end{defn}

\begin{defn}
A sheaf $\F$ is called $\mu$-\textit{semistable} (with respect to $\L$) if $\F$ is torsion-free for all nontrivial proper subsheaves $\E \subset \F$ we have,
\[ \frac{c_1(\E) \cdot c_1(\L)}{\rank{\E}} \le \frac{c_1(\L) \cdot c_1(\F)}{\rank{\F}} \]
\end{defn}

\begin{rmk}
For the definition of $\mu$-stable you need nontrivial proper subsheaves with strictly smaller rank. To see why this is necessary, consider,
\begin{center}
\begin{tikzcd}
0 \arrow[r] & \I_p \arrow[r] & \struct{X} \arrow[r] & k_p \arrow[r] & 0
\end{tikzcd}
\end{center}
Then we get $c_1(\I_p) = 0$ and hence we don't get a strict inequality,
\[ \frac{c_1(\I_p) \cdot c_1(\L)}{\rank{\I_p}} \le \frac{c_1(\struct{X}) \cdot c_1(\L)}{\rank{\struct{X}}} \]
\end{rmk}

\begin{example}
\begin{enumerate}
\item if $\F$ has rank $1$ then $\F$ is $\mu$-semistable for all polarizatios
\item if $\F$ is $\mu$-semistable and $\H \in \Pic{X}$ then $F \ot \H$ is $\mu$-semistable
\item if $\F$ is $\mu$-semistable then $\F^{\vee \vee}$ is $\mu$-semistable. 
\end{enumerate}
\end{example}

\begin{defn}
Let $\F$ be torsion-free of rank $r$, then $\Delta(\F) = 2r c_2 - (r-1) c_1^2$.
\end{defn}

\begin{theorem}[Bogomolov]
If $\F$ is $\mu$-semistable on $(X, \L)$ then $\Delta(\F) \ge 0$.
\end{theorem}

\begin{rmk}
Since $\Delta(\F)$ is independent of the polarization, Bogomolov's theorem gives an obstruction to be $\mu$-semistable with respect to \textit{any} polarization.
\end{rmk}

\begin{prop}
Recall,
\[ \ch{\F} = \rank{\F} + c_1(\F) + \tfrac{1}{2}(c_1^2 - 2 c_2) \]
Then we can compute with $r = \rank{\F}$,
\[ \log{ \left( \frac{\ch{\F}}{r} \right)} = \log{(1 + \square)} = \left[ \frac{c_1}{r} + \frac{c_1^2 - 2 c_2}{2r} \right] - \frac{c_1^2}{2 r} = \frac{c_1}{r} + \frac{1}{2 r^2} \left( (r-1) c_1^2 - 2 r c_2 \right) \]
Therefore,
\[ \log{ \left( \frac{\ch{\F}}{r} \right)} = \frac{c_1}{r} + \frac{1}{2 r^2} \Delta(\F) \]
Because $\log$ sends multiplication to addition, we have,
\[ \frac{\Delta(\F \ot \G)}{(\rank{\F})^2 (\rank{\G})^2} = \frac{\Delta(\F)}{(\rank{\F})^2} + \frac{\Delta(\G)}{(\rank{\G})^2} \]
\end{prop}

\begin{prop}
\begin{enumerate}
\item if $\F$ is a line bundle then $\Delta(\F) = 0$
\item if $\F$ is locally free then $\Delta(\F) = \Delta(\F^\vee)$ 
\item for $\H \in \Pic{X}$ we have,
\[ \frac{\Delta(\F \ot \H)}{(\rank{\F})^2 1^2} = \frac{\Delta(\F)}{(\rank{\F})^2} + 0 \implies \Delta(\F \ot \H) = \Delta(\F) \]
\item if $\F$ is locally free, then $\Delta(\End{\F}) = 2 (\rank{\F})^2 \Delta(\F)$
\item if $\F$ is torsion free, then,
\begin{center}
\begin{tikzcd}
0 \arrow[r] & \F \arrow[r] & \F^{\vee \vee} \arrow[r] & \mathcal{Q} \arrow[r] & 0
\end{tikzcd}
\end{center}
and then,
\[ \Delta(\F^{\vee \vee}) = 2 r c_2(\F^{\vee \vee}) - (c-1) c_1(\F^{\vee \vee}) = 2 r (c_2(\F) + \ell(\mathcal{Q}))  - (r - 1) c_1(\F)^2 = \Delta(\F) + 2 r \ell(\mathcal{Q}) \]
\end{enumerate}
\end{prop}

\begin{rmk}
By the last property, since $\ell(\mathcal{Q}) \ge 0$ we see that if $\Delta(\F^{\vee \vee}) \le 0$ then $\Delta(\F) \le 0$. Therefore, it suffices to prove the theorem for reflexive and hence locally free $\F$,
\end{rmk}

\begin{proof}
Proof reductions,
\begin{enumerate}
\item can assume $\F \cong \F^{\vee \vee}$ by above remark
\item $\Delta(\End{\F}) = 2 r^2 \Delta(\F)$ so can assume that $\det{\F} = \struct{X}$.
\end{enumerate}
We need to show that $c_2(\F) \le 0$ for $\F$ such that,
\begin{enumerate}
\item $\F$ is a vector bundle
\item $\det{\F} \cong \struct{X}$
\item $\F$ is $\mu$-semistable for some $\L$.
\end{enumerate}
Consider,
\[ \F_n = \Sym{nr}{\F} \]
We use the following lemmas.
\end{proof}

\begin{lemma}
\begin{enumerate}
\item $\det{\F_n} \cong \struct{X}$
\item there is a formula,
\[ \chi(X, \F_n) = - \frac{\Delta(\F) n^{r+1} r^{r}}{2 (r+1)!} + O(n^r) \]
\end{enumerate}
\end{lemma}

\begin{proof}
Represent $\F$ as $[\xi] \in H^1(X, \SL_n)$ because $\det{\F} \cong \struct{X}$. Then $\SL_{r} \acts \Sym{nr}{\CC^r}$ gives an action $\SL_r \acts \det{\Sym{nr}{\CC^r}}$ which is a character of $\SL_r$ and hence is trival. Therefore, the map $H^1(X, \SL_r) \to H^1(X, \Gm)$ given by taking determinants is trivial. 
\bigskip\\
Consider, $\pi : \P_X(\F) \to X$. Look at $\struct{\P(\F)}(1)$. Then,
\[ \chi(\P(\F), \struct{\P(\F)}(nr)) = \chi(X, R \pi_* \struct{\P(\F)}(nr)) = \chi(X, \Sym{nr}{\F}) = \chi(X, \F_n) \]
Now, by Riemann-Roch we have,
\[ \chi(\P(\F), \struct{\P(\F)}(nr)) = \frac{(nr)^{r+1} c_1(\struct{\P(\F)}(1))^{r+1}}{(r+1)!} + O(n^r) \]
But by the projective bundle formula (or the Grothendieck definition of Chern classes), 
\[ c_1(\struct{\P(\F)}(1))^{r} - \pi^* c_1(\F) c_1(\struct{\P(\F)}(1))^{r-1} + \pi^* c_2(\F) c_1(\struct{\P(\F)}(1))^{r-2} = 0 \]
We assumed that $c_1(\F) = 0$. Therefore, 
\[ c_1(\struct{\P(\F)}(1))^{r+1} = - \pi^* c_2(\F) c_1(\struct{\P(\F)}(1))^{r-1} \]
Now we have,
\[ \deg{(c_1(\struct{\P(\F)}(1))^{r+1})} = -\deg{\pi_*( \pi^* c_2(\F) c_1(\struct{\P(\F)}(1))^{r-1})} = - \deg{(c_2(\F) \cdot \pi_* [c_1(\struct{\P(\F)}(1)^{r-1})])} \]
Now $\pi_* [c_1(\struct{\P(\F)}(1))^{r-1}] = [X]$ since on each fiber since this is $H^{r-1}$ on $\P^{r-1}$ where $H$ is the hyperplane class. Since $c_1(\F) = 0$ we have $\Delta(\F) = 2 r c_2(\F)$ and thus,
\[ \chi(X, \F_n) = - \frac{\Delta(\F) n^{r+1} r^r}{2 (r+1) !} + O(n^r) \]
\end{proof}

\newcommand{\td}{\mathrm{td}}

\begin{rmk}
We explicitly complete the GRR calculaiton. Let $\wt{X} = \P_X(\F)$. By GRR,
\[ \chi(\wt{X}, \G) = \deg{(\ch(\G) \cdot \td_{\wt{X}})} \]
and because $R \pi_* \struct{\wt{X}}(nr) = \Sym{nr}{\F}[0]$ we have that,
\[ \chi(X, \Sym{nr}{\F}) = \chi(X, R \pi_* \struct{\wt{X}}(nr)) = \chi(\wt{X}, \struct{\wt{X}}(nr)) = \deg{(\ch(\struct{\wt{X}}(nr)) \cdot \td_{\wt{X}})} \]
Now let $\xi = c_1(\struct{\wt{X}}(1))$ then we see,
\[ \chi(X, \Sym{nr}{\F}) = \deg{(e^{nr \xi} \cdot \td_{\wt{X}})} \]
Let $d = \dim{\wt{X}} = r - 1 + \dim{X} = r + 1$. Then the leading term as a polynomial in $n$ gives,
\[ \chi(X, \Sym{nr}{\F}) = \frac{(nr)^d \xi^d}{d!} \cdot 1 + O(n^{d-1}) \]
because the first term of $\td_{\wt{X}}$ is $1$. This gives, 
\end{rmk}

\begin{proof}
Now we complete the proof. From the lemma, to show that $\Delta(\F) \le 0$ it suffices to show that $\chi(X, \F_n) \le C n^r$ as $n \to \infty$. This will follow if we show that $H^0(X, \F_n) \le C_1 n^r$ as $n \to \infty$ and $H^2(X, \F_n) \le C_2 n^r$ as $n \to \infty$. 
\end{proof}

\section{Nov 15}

\newcommand{\cN}{\mathcal{N}}

From $\varphi : C \to \P^{g-1}$ the canonical morphism we get,
\begin{center}
\begin{tikzcd}
0 \arrow[r] & T_C \arrow[r] & \varphi^* T_{\P^{g-1}} \arrow[r] & \cN_C \arrow[r] & 0
\end{tikzcd}
\end{center}

\begin{theorem}
Let $k = \bar{k}$ and $g \neq 2,4,6$ and $C$ general canonical curve of genus $g$ then $\cN_C$ is semi-stable.
\end{theorem}

\begin{rmk}
$\rank{\cN_C} = g - 2$ then $\deg{\varphi^* T_{\P^{g-1}}} = 2 g (g - 1)$ and thus $\deg{\cN_C} = (g+1) 2(g-1) - 2 (g^1 - 1)$. Therefore,
\[ \mu(\cN_C) = \frac{2 (g^2 - 4 + 3)}{g-2} = 2 (g+2) + \frac{6}{g-2} \]
\end{rmk}

\begin{example}
$g = 3$ then $C \subset \P^2$ is a plane quartic and $\cN_C \cong \struct{C}(4)$ is semistable.
\end{example}

\begin{example}
$g = 5$ then $C \subset \P^4$ is $C = Q_1 \cap Q_2 \cap Q_3$ is a complete intersection of three quadrics. Then $\cN_C = \struct{C}(2)^{\oplus 3}$ is semi-stable. 
\end{example}

\begin{example}
Let $g = 4$ then $C = Q \cap X_3 \subset \P^3$ so $\cN_C \cong \struct{C}(2) \oplus \struct{C}(3)$ is destabilized. 
\end{example}

\begin{example}
Let $g = 6$ then $C \subset X \subset \P^5$ where $X$ is a del-Pezzo surface, the blowup of $\P^2$ at three points anticanonically embedded in $\P^5$ and $C$ is a quartic section. Then $\cN_{C/X} \subset \cN_{C}$ will destabilize it. 
\end{example}

\begin{rmk}
The $g = 7$ case is Aprodu-Farkas-Ortega. The $g = 8$ case by Bruns. 
\end{rmk}

If $(6, g-2) = 1$ then any sub $\F \subset \cN_C$ has $\rank{\F} \le g - 3$ and hence $\cN_C$ is stable because equality is impossible since the fraction $\mu(\cN_C)$ is irreducible. 

\begin{cor}
$g \equiv 1,3 \mod 6$ then $\cN_C$ is stable.
\end{cor}

Let $C$ be a connected nodal curve. Let $V$ be a vector bundle on $C$. Then consider the normalization $\nu : \wt{C} \to C$. Let $\wt{p}_1$ and $\wt{p}_2$ be the two preimages of the node. There is a canonical isomorphism,
\[ \nu^* V |_{\wt{p}_1} \iso \nu^* V |_{\wt{p}_2} \]
and $\F \subset \nu^* V$ a subbundle then it makes sense to compare $\F|_{\wt{p}_1}$ and $\F|_{\wt{p}_2}$. 

\newcommand{\adj}{\mathrm{adj}}

\begin{defn}
The \textit{adjusted slope} of $\F \subset \nu^* V$ is,
\[ \mu^{\adj}_C(\F) = \mu(\F) - \frac{1}{\rank{\F}} \sum_{p \in C^{\text{sing}}} \codim{\F|_{\wt{p}_1} \cap \F_{\wt{p}_2}}{\F|_{\wt{p}_1}} \]
Say $V$ is \textit{semi-stable} on $C$ if,
\[ \mu^{\adj}_C(\F) \le \mu^{\adj}(nu^* V) = \mu(V) \]
for any subbunle (of constant rank) $\F \subset \nu^* V$ and \textit{stable} if there is a strict inequality for nontrivial subbundles.
\end{defn}

\begin{prop}[CIV, 2022]
Let $\C \to \Delta = \Spec{R}$ be a family of connected nodal curves over a DVR. Let $V$ be a vector bundle on $\C$ and $V_0$ is semistable on $\C_0$ then $V_\eta$ is semistable on $\C_\eta$.
\end{prop}

\begin{lemma}
Let $C = X \cup Y$ be nodal. Let $V$ be a vector bundle with $V|_X$ and $V|_Y$ semistable. Then $V$ is semistable. Furthermore, if one of $V|_X$ and $V|_Y$ is stable, then $V$ is stable.
\end{lemma}

\section{Nov 29 Reider's Theorem}

\begin{theorem}
Let $X$ be a surface and $L$ a nef line bundle.
\begin{enumerate}
\item if $L^2 > 4$ and $p$ is a base point of $K_X \ot L$ then there is some $p \in D \subset X$ effective such that one of
\begin{enumerate}
\item $L \cdot D = 0$ and $D^2 = -1$
\item $L \cdot D = 1$ and $D^2 = 0$
\end{enumerate}
is true

\item if $L^2 > 9$ and $p,q$ are not separated by $K_X \ot L$ then there is $p,q \in D \subset X$ such that one of,
\begin{enumerate}
\item $L \cdot D = 0$ and $D^2 = -1,-2$
\item $L \cdot D = 1$ and $D^2 = 0,-1$
\item $L \cdot D = 2$ and $D^2 = 0$ 
\end{enumerate}
is true.
\end{enumerate}
\end{theorem}

\begin{proof}
Step 1: build an extension,
\begin{center}
\begin{tikzcd}
0 \arrow[r] & \struct{X} \arrow[r] & E \arrow[r] & I_Z \ot L \arrow[r] & 0
\end{tikzcd}
\end{center}
with $Z = \{ p \}$ or $\{p, q \}$ and $E$ locally free.
\bigskip\\
Step 2: use Bogomolov inequality, to show that $E$ is unstable and we can find a sequence that destabilizes $E$,
\begin{center}
\begin{tikzcd}
0 \arrow[r] & A \arrow[r] & E \arrow[r] & I_W \ot V \arrow[r] & 0
\end{tikzcd}
\end{center}
for every ample class. The fact that it destabilizes for every ample is important because then we can derive results for nef classes by taking limits.
\bigskip\\
Step 3: inequalities. 
\end{proof}


\subsection{Special Cycles}

\begin{defn}
Let $X$ be a variety and $\struct{X}(D)$ a line bundle on $X$. An zero-dimesional subscheme $Z \subset X$ is said to be \textit{in special position} with respect to $\struct{X}(D)$ if 
\begin{enumerate}
\item $H^0(X, \struct{X}(D)) \to H^0(Z, \struct{X}(D)|_Z)$ is not surjective
\item for all $Z' \subset Z$ with $\length{\struct{Z} / \struct{Z'}} = 1$ the map
\[ H^0(X, \I_{Z}(D)) \to H^0(X, \I_{Z'}(D)) \]
is an isomorphism
\end{enumerate}
\end{defn}

\begin{lemma}
Suppose that $\struct{X}(D)$ does not separate some collection of $m$ distinct points on $U$. Then there exists a reduced zero-dimensional subscheme $Z \subset U$ with $\length{Z} \le m$ in special position with respect to $\struct{X}(D)$.
\end{lemma}

\begin{proof}
Let $Z_0 = \{ p_1, \dots, p_m \}$ be some collection of $m$ distinct points in $U$ such that
\[ H^0(X, \struct{X}(D)) \to H^0(Z_0, \struct{X}(D)|_{Z_0}) \]
is not surjective. For each $p_i \in Z_0$ let $Z_0' = Z_0 \sm \{ p_i \}$ and consider the map
\[ H^0(X, \I_{Z_0'}(D)) \to H^0(X, \I_{Z_0}(D)) \]
if this is an isomorphism for all $i$ then $Z_0$ is in special position. Otherwise, there is $i$ for which it is not an isomorphims. This means there is a section $s$ vanishing on $Z_0'$ nonvanishing at $p_i$. If $\struct{X}(D)$ separates $Z_0'$ then it also separates $Z_0$ because we can use $s$ to make the value at $p_i$ anything without changing the values on $Z_0'$. Thus $Z_1 := Z_0'$ is a collection of points not separated by $\struct{X}(D)$. Repeating, we get $Z_k$ with length $m - k$ in special position. The only case where this might fail is if we get down to $k = m - 1$. Then $\struct{X}(D)$ is not globally generated on $U$. Since every section vanishing at some $p$ then $Z = \{ p \}$ is in special position so we automatically win.  
\end{proof}

{\color{red} IS THIS THE SAME AS $Z$ SCHEME?}

\begin{theorem}
Let $S$ be a smooth surface, $L$ a line bundle and $Z$ an effective zero-cycle on $S$. The following are equivalent:
\begin{enumerate}
\item $Z$ is in special position with respect to $|K_S + L|$, where $K_S$ is the canonical divisor of $S$,
\item there exist a pair $(\E, e)$ where $\E$ is a rank $2$ bundle on $S$ and $e$ a section such that $\E = \wedge^2 \E = L$ and $V(e) = Z$.
\end{enumerate}
\end{theorem}

\begin{proof}
Let $Z$ be in special position. Then consider the extension
\[ 0 \to \struct{S} \to \E \to \I_Z \ot L \to 0 \]
whose extension class in
\[ \Ext{1}{S}{\I_Z \ot L}{\struct{S}} = H^1(X, \I_Z \ot L \ot \omega_S)^\vee \]
is the element defined by 
\[ H^0(Z, L \ot \omega_S|_Z) \to H^1(X, \I_Z \ot L \ot \omega_S) \]
{\color{red} DO THIS}
\end{proof}

For step 2 we have,
\begin{center}
\begin{tikzcd}
0 \arrow[r] & \struct{X} \arrow[r] & E \arrow[r] & I_Z \ot L \arrow[r] & 0
\end{tikzcd}
\end{center}
and $\rank{E} = 2$ and $\det{E} = L$ and $c_1(E) = c_1(L)$ and $c_2(E) = \deg{Z} = 1$ (doing part 1). Notice that $c_1(E)^2 > 4 c_2(E)$. Then by Bogomolov, there is a sequence
\begin{center}
\begin{tikzcd}
0 \arrow[r] & A \arrow[r] & E \arrow[r] & I_W \ot B \arrow[r] & 0
\end{tikzcd}
\end{center}
$W$ is a $0$-dimensional subscheme such that
\begin{enumerate}
\item $A \ot B = L$
\item $(c_1(A) - c_1(B))^2 > 0$
\item for any ample $H$ we have $c_1(A) \cdot H > c_1(B) \cdot H$
\item $c_2(\E) = \length{Z} = A \cdot B + \length{W}$. 
\end{enumerate}
Then consider the diagram,
\begin{center}
\begin{tikzcd}
& & 0 \arrow[d]
\\
& & A \arrow[d] \arrow[ld, dashed] \arrow[rd, "t"]
\\
0 \arrow[r] & \struct{X} \arrow[r] & E \arrow[d] \arrow[r] & I_Z \ot L \arrow[r] & 0
\\
& & B \ot I_W \arrow[d]
\\
& & 0
\end{tikzcd}
\end{center}
Claim $t \neq 0$ otherwise, there a nonzero map $A \to \struct{X}$ and thus $H^0(X, A^\vee) \neq 0$ but $0 < (A-B) \cdot H = (2 A - L) \cdot H$ and hence $(-A) \cdot H < - \frac{1}{2} L \cdot H \le 0$ is a contradiciton. Therefore $t \neq 0$. 
\bigskip\\
Let $D$ be the effective divisor defined by $t$. Since $Z \subset D$ and $L = A \ot \struct{X}(D)$ so consider the sequence,
\begin{center}
\begin{tikzcd}
0 \arrow[r] & I_Z \ot L \ot A^\vee \arrow[r] & L \ot A^\vee \arrow[r] & (L \ot A^\vee)|_Z \arrow[r] & 0
\end{tikzcd}
\end{center}

Now we enter step 3. Since $A + B = L$ we can replace $B$ by $D$ i.e.\ ensure that $B$ is effectove. We have constructed divisors $L, A, D$ with $L - A = D$ thus $A - D = L - 2 D$

\begin{lemma}
We have $L \cdot D \ge 0$ and $L \cdot D - 1 \le D^2 \le \frac{1}{2} L \cdot D$ 
\end{lemma}

\begin{proof}
Since $D$ is effective and $L$ is nef $L \cdot D \ge 0$. Since $L$ is a limit of ample divisors we see that $L \cdot (A - D) \ge 0$ but $ L = A + D$ so $A^2 \ge D^2$ {\color{red} WHY DO THEY SAY NOT EQUALITY ALLOWED}. Also 
\[ (A - D)^2 > 0 \implies A^2 - 2 A \cdot D + D^2 \ge \]

Then $A \cdot D + \length{W} = \length{Z} = 1$ implies $A \cdot D \le 1$. 
Hodge index theorem gives
\[ A^2 D^2 \le (A \cdot D)^2 \le 1 \]
and thus $D^2 \le 0$ since $A^2 \ge D^2$ {\color{red} WHY?}

Now $0 \le L \cdot D = A \cdot D + D^2$
\end{proof}

\section{A Langer: On boundedness of semistable sheaves}

Let $X$ be a projective varierty over $k = \bar{k}$. Fix some ample class $H$ on $X$ then we get slop semistability for torsion-free sheaves $H$-semistability.

\begin{theorem}[Boundedness]
Let $P \subset \Q[n]$ be an integer valued polynomial. Then the set,
\[ S = \{ H\text{-semistable torsor-free sheaves with Hilbert polynomial } P \} \]
is bounded i.e. there is a scheme $Y$ of finite type over $k$ and a sheaf $\F$ on $X \times_k Y$ such that for all $\G \in S$ there is a $k$-point $y \in Y(k)$ such that $\F_y \cong \G$. 
\end{theorem}

Bogomolov's inequality for $\P^d_K \implies$ some restriction theorem for sheaves on $\P^d_k \implies$ boundedness of $H$-semistable sheaves. 
\bigskip\\
The main result of Langer is to prove Bogomolov's inequality for $\P^d_K$. Strategy: induction of the dimension $d$ by using pencils. 

\begin{theorem}[Bogomolov Inequality]
If $\F$ is $H$-semistable, then,
\[ \Delta(\F) \cdot H^{d-2} = \left(2 \rank{(\F)} \cdot c_2(\F) - (\rank{\F} - 1) c_1(\F)^2 \right) \cdot H^{n-2} \ge 0 \] 
\end{theorem}

\begin{proof}
Start with $\F$ on $\P^d_K$. Choose a general pencil $\P^1 \cong \Lambda \subset |\struct{\P^d_K}(1)|$ which has base locus $B \subset \P^d_K$. Blow it up,
\begin{center}
\begin{tikzcd}
\mathrm{Bl}_B(\P^d_K) \arrow[d] \arrow[r] & \P^1
\\
\P^d_K \arrow[ru, dashed]
\end{tikzcd}
\end{center}
Now you show that Bogomolov for the blowup withrespt to the (now not ample) pullback of $H$. The fibers are projective spaces so we can reduce to smaller dimension. 
\end{proof}

\end{document}
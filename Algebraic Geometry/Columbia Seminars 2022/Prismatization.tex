\documentclass[12pt]{article}
\usepackage{hyperref}
\hypersetup{
    colorlinks=true,
    linkcolor=blue,
    filecolor=magenta,      
    urlcolor=cyan,
}
 
\urlstyle{same}
\usepackage{import}
\import{../}{AlgGeoCommands}


\begin{document}

\section{Week 1}

\section{Week 2}

\begin{enumerate}
\item $\delta$-rings
\item prisms
\item prismatic site
\item Hodge-Tate comparison
\item $\etale$ comparison
\item Nygaard filtration.
\end{enumerate}

\subsection{$\delta$-rings}

\begin{defn}
A $\delta$-ring is a pair $(A, \delta : A \to A)$ where $\delta$ is not a ring homomorphism but satisfies the property that,
\[ \phi : A \to A \quad \phi(x) = x^p + p \delta(x) \]
is a ring homomorphism. 
\end{defn}

\begin{rmk}
We will usually assume that $A$ is a $\Z_{(p)}$-algebra but this is not necessary. On $\Z_{(p)}$ we have $\phi(x) = x$ is the only possible lift so,
\[ \delta(x) = \frac{x - x^p}{p} \]
\end{rmk}

\begin{prop}
$(\Z_{(p)}, \delta)$ is the initial $\delta$-ring.
\end{prop}

\begin{rmk}
Suppose that $p^n = 0$ in $A$ then $\delta'(p^n) = 0$
\end{rmk}

\begin{prop}
The forgetful functor $\delta$-Rings to Rings has a left and right inverse. The right adjoint is given by Witt vectors the left adjoint is given by free $\delta$-rings
\end{prop}

\begin{rmk}
The free $\delta$-ring structure $A \{ x \} = A[x_1, x_2, \dots ]$ with $\delta(x_i) = x_{i+1}$ 
\end{rmk}

\begin{defn}
$\alpha \in A$ is distinguished if $\delta(\alpha)$ is a unit.
\end{defn}

\begin{example}
The following are distinguished,
\begin{enumerate}
\item $p \in \Z_p$ 
\item $x - p \in \Z_p[[x]]$
\item for $A_{\inf} = W(R^{\flat})$ each $\xi \in \ker{(A_{\inf} \to R)}$ is distinguished.  
\end{enumerate}
\end{example}

\begin{defn}
A $\delta$-ring is \textit{perfect} if $\phi$ is an isomorphism.
\end{defn}

\begin{prop}
The following categories are equivalent:
\begin{enumerate}
\item perfect $p$-complete $\delta$-rings
\item $p$-complete $p$-torsion-free rings $A$ with $A / p$ perfect
\item perfect $\FF_p$-algebras.
\end{enumerate}
\end{prop}

\begin{proof}

\end{proof}

\subsection{Prisms}

\begin{defn}
A \textit{prism} is a pair $(A, I)$ with $I \subset A$ Cartier divisor such that $A$ is $(p, I)$-adically complete with $p \in I + \phi(I) A$. The category of prisms is a full subcategory of the category of pairs $(A, I)$ with maps $f : (A, I) \to (B, J)$ meaning $f(I) \subset J$.
\end{defn}

\begin{rmk}
$p \in I + \phi(I) A \iff \text{after ind-Zariski localization} I = (\zeta) \text{ where } \zeta \text{ is distinguished}$.
\end{rmk}

\begin{defn}
A prism is,
\begin{enumerate}
\item \textit{perfect} if $A$ is perfect
\item \textit{bounded} if $A[p^\infty] = A[p^n]$.
\item \textit{orientable} if $I$ is principal
\item \textit{oriented} if $I$ is given a canonical generator
\item \textit{crystaline} if $I = (p)$ which implies oriented and bounded.
\end{enumerate}
\end{defn}

\begin{prop}
If $f : (A, I) \to (B, J)$ is a morphism of prisms then $f(I) B = J$.
\end{prop}

\begin{theorem}
There is an equivalence of categories between,
\begin{enumerate}
\item perfect prisms
\item perfectoid rings.
\end{enumerate}
\end{theorem}

\begin{rmk}
Thus perfect prisms are orientable. 
\end{rmk}

\begin{prop}
If $(A, I)$ is perfect then,
\[ \Hom{}{A/I}{B/J} = \Hom{}{(A,I)}{(B,J)} \]
\end{prop}

\subsection{Tilting Equivalence}

For $R \to S$ a map of perfectoid rings, let $R = A/I$ and $S = B/J$ for some perfect prisms. Then $R^\flat = A / p$ and $S^\flat = B / p$. By the lifting we get a unique map $(A, I) \to (B, J)$ and hence a map of $\delta$-rings $A \to B$ which gives a map $A/p \to B/p$ giving the map of tilts $R^{\flat} \to S^{\flat}$. 

\subsection{Prismatic Site}

\newcommand{\Spf}[1]{\mathrm{Spf}\left( #1 \right)}
\newcommand{\pris}{\mathbb{\Delta}}
\newcommand{\barpris}{\overline{\pris}}
\newcommand{\derot}{\ot^{\mathbb{L}}}

Let $X$ be a $p$-adic formal scheme. Let $(A, I)$ be a bounded prism with a map $X \to \Spf{(A/I)}$. The site $(X/A)_{\pris}$ whose objects are pairs of a prism $(B, J)$ over $(A,I)$ and a map $\Spf{B/J} \to X$ over $\Spf{A/I}$.
\bigskip\\
The absolute prismatic site $(X)_{\pris}$ are just prisms $(B, J)$ equipped with a map $\Spf{B/J} \to X$.
\bigskip\\
If $X = \Spf{A/I}$ then $(X/A)_{\pris}$ are just prisms over $(A,I)$.
\bigskip\\
There are important sheaves,
\[ \struct{\pris} : (B, J) \mapsto B \quad \text{and} \quad \overline{\struct{\pris}} : (B, J) \mapsto B/J \]
There is a pushforward map $\nu_* : \Sh{(X/A)_{\pris}} \to \Sh{(X_{\et})}$. Then the prismatic cohomology is,
\[ \pris_{X/A} = R \nu_* \struct{\pris} \in D(X_{\et}, A) \quad \text{ and } \overline{\pris}_{X/A} = R \nu_* \overline{\struct{\pris}} \in D(X_{\et}, \struct{X}) \]
Then as $A/I$-modules,
\[ \overline{\pris}_{X/A} = \pris_{X/A} \ot^{\mathbb{L}}_A A / I \]

\subsection{Hodge-Tate Comparison}

For any $A / I$-module $M$ we define,
\[ M \{ i \} := M \ot_{A/I} I^i / I^{i+1} \]
Then then there is a distinguished triangle,
\[ \overline{\pris}_{X/A} \{ i + 1 \} \to \pris_{X/A} \derot_A I^i / I^{i+2} \to \overline{\pris}_{X/A} \{ i \} \]
which gives rise to a differential,
\[ \beta_I : H^i(\barpris_{X/A} \{ i \}) \to H^{i+1}(\barpris_{X/A} \{ i+1 \}) \]
and therefore,
\[ H^\bullet(\barpris_{X/A} \{ \bullet \} \] is a differential-graded ring. Therefore,
\[ \eta^0_X : \struct{X} \to H^0(\barpris_{X/A} \{ 0 \}) \]
extends to a map,
\[ \eta_X : \Omega^\bullet_{X/(A/I)} \to H^\bullet(\barpris_{X/A} \{ \bullet \}) \]

\begin{theorem}[Comparison]
If $X$ is smooth the map $\eta_X$ is an isomorphism (in the derived category). 
\end{theorem}

\subsection{Fibers}

\newcommand{\Spa}[1]{\mathrm{Spa}\left( #1 \right)}

For $X \to \Spf{R}$ we define $X_\eta := X \times_{\Spf{R}} \Spa{R[\frac{1}{p}], R}$. Then 


\section{FFFF}

\section{Sept 29}

\newcommand{\LL}{\mathbf{LL}}

\begin{defn}
Let $k$ be a (noncommutative) ring and $A$ a (noncommutative) $k$-algebra. If $A$ is a flat $k$-algebra then the \textit{Hochschild Homology} $HH_\bullet(A/k)$ is defined as the complex associated (via Dold-Kan) to the simplicial ring $SHH_\bullet(A/k)$ with,
\[ SHH_n(A/k) = A^{\ot (n+1)} \]
where $d_i(a_0 \ot \cdots \ot a_n) = a_0 \ot \cdots \ot a_i a_{i+1} \ot \cdots \ot a_n$ for $i < n$ and $d_n(a_0 \ot \cdots \ot a_n) = a_n a_0 \ot \cdots \ot a_{n-1}$ which is distinct because $A$ is noncommutative. 
\end{defn}

\begin{rmk}
Then $HH_0(A/k) = A / [A,A]$ and in the commutative case $HH_1(A/k) = \Omega^1_{A/k}$.
\end{rmk}

\begin{prop}
There are natural isomorphisms,
\[ HH_n(A/k) \iso \Tor{A \ot_k A^{\op}}{n}{A}{A} \]
\end{prop}

\begin{defn}
We say that an $A'$-algebra $A$ is \etale if $A'$ is projective as an $A$-module and $A'$ is a projective $A' \ot_A A'$-module.
\end{defn}

\begin{rmk}
When everything is commutative,
\begin{enumerate}
\item if $A'$ is \etale over $A$ then,
\[ HH_\bullet(A / k) \ot_A^{\LL} A' \iso HH_{\bullet}(A'/k) \]
\end{enumerate}
\end{rmk}

\begin{theorem}
Let $A$ be $k$-smooth (and everything is commutative). Then,
\[ \Omega^\bullet_{A/k} \to \widehat{HH_\bullet}(A/k) \]
is a map of graded rings. 
\end{theorem}

\subsection{Extending HH to all $k$-algebras}

Consider the diagram of functors,
\begin{center}
\begin{tikzcd}
& D
\\
P \arrow[rr] \arrow[ru, "HH"] & & k\text{-alg} \arrow[ul, dashed]
\end{tikzcd}
\end{center}
Where $D$ is the derived category of $k$-modules. Then $\wt{HH}_\bullet(A/k)$ is the left Kan extension. In general, in the commutative case,
\[ HH_\bullet(A/k) \iso A \ot^{\LL}_{A \ot^{\LL}_{k} A} A \]
in the derived category.

\subsection{Cyclic Homology}

Work with $A$ flat over $k$. Then $HH_n(A/k)$ has a circle action of $\Z / (n+1) \Z$ on,
\[ HH_n(A/k) = A^{\ot (n+1)} \]
given by the brading of tensor product (exchanging terms),
\[ t_n : a_0 \ot \cdots \ot a_n = a_n \ot a_0 \ot \cdots \ot a_{n-1} \]
Then we define,
\[ N = \sum_{i = 0}^n (-1)^i t_n^i : A^{\ot (n+1)} \to A^{\ot (n+1)} \]
and likewise,
\[ B = (1 - (-1)^n t_n) s_0 N : A^{\ot n} \to A^{\ot n} \to A^{\ot (n+1)} \to A^{\ot (n+1)} \]
where $s_0$ is a degeneracy map. Write $\pm t$ for $(-1)^n t_n$.

\begin{lemma}
\begin{enumerate}
\item $(1 - \pm t) b' = n(1 - \pm t)$
\item $b' N = N b$
\item $s_0 b' + b' s = 1$
\item $B^2 = 0$
\item $B b = - bB$.
\end{enumerate}
Therefore, $B$ gives a map on the homology of $HH_\bullet(A/k)$ such that,
\begin{center}
\begin{tikzcd}
HH_n(A/k) \arrow[r, "B"] & HH_{n+1}(A/k)
\\
\Omega^n(A/k) \arrow[u] \arrow[r, "\d"] & \Omega^{n+1}(A/k) \arrow[u]
\end{tikzcd}
\end{center}
commutes.
\end{lemma}

\section{Intro to $\Sigma$}

\begin{enumerate}
\item $W_S$-modules and definition

\item points and divisors 

\item contracting prop. of F

\item global sections

\item line bundles.
\end{enumerate}

Prismatisation functor $X \mapsto X^\pris$. Then,
\[ D_{\pris}^\bullet(X) \iso D^\bullet(X^{\pris}) \]
This is like how sheaves on the de Rham site correspond to sheaves on the de Rham stack. 
\bigskip\\
Let $X = \Spf{\Z_p}$ then,
\[ \pris_{X} = \Z_p[[x]][0] \]

Maps $X \to \Spf{\Z_p}$ should correspond to maps $X^\pris \to \Sigma$.

\subsection{Witt Vectors}

\newcommand{\prim}{\mathrm{prim}}

Let $W = \Spec{\Z[x_0, x_1, \dots]} \to \Spf{\Z_p}$
Then,
\[ Z_p[x_0,x_1, \dots] = \Spec{\Z_p \{ x \}} \]
is the free $\delta$-ring. Let $Z \subset W$ be cut out by $p = x_0 = 0$ and $x_1 \neq 0$. Then $W_{\text{prim}} = \widehat{W_Z}$ is the completion. Then,
\[ W_{\prim} = \Spf{\Z_p[x_0,x_1, \dots][x_1^{-1}]^{\vee}, (p, x_0)} \]
Then $W_{\prim}$ is Frobenius stable so we get a diagram,
\begin{center}
\begin{tikzcd}
W_{\prim} \arrow[r, "F"] \arrow[d] & W_{\prim} \arrow[d]
\\
W \arrow[r, "F"] & W
\end{tikzcd}
\end{center}
Also $W$ is a ring scheme. Consider,
\[ W^\times \times W \to W \quad \text{ via } (\lambda, x) \mapsto \lambda^{-1} x \]
Then get $W^\times \to W_{\prim} \to W_{\prim}$. Therefore we can define the following stack.

\begin{defn}
Let $\Sigma = [ W_{\prim} / W^\times]$ meaning it is the stackification of the functor,
\[ R \mapsto W_{\prim}(R) / W^\times(R) \]
\end{defn}


\begin{defn}
Let $W_S = W \times_{\Spf{\Z_p}} S$. Then a $W_S$-module is a commutative affine group scheme with an action of $W_S$. 
\end{defn}

\begin{defn}
Then $M$ is invertible if fpqc locally isomorphic to $W_S$. This is equivalent to a $W_S^\times$-torsor.
\end{defn}

\begin{rmk}
An $S$-point of $[W_{\prim} / W^\times]$ is by definition an $W^\times$-torsor over $S$ equipped with a map to $(W_{\prim})_S$. This is the same data as a $W_S$-module $M$ with a map $\xi : M \to W_S$ factoring through $(W_{\prim})_S$ which is the same as having fibers in kernel of $\xi_1$ but in kernel of $\xi_2$.
\end{rmk}

\begin{defn}
A better definition of $\Sigma$ is therefore,
\[ \Sigma : S \mapsto \{ (M, \xi) \mid M \text{ invertible } W_S\text{-module and } \xi : M \to W_S \text{ is primitive} \} \]
which is a category fibered in groupoids.
\end{defn}

\section{Oct 13}

\newcommand{\Cart}{\mathrm{Cart}}

\begin{defn}
A \textit{generalized Cartier divisor} $(\I, \alpha)$ is a pair of an invertible $\struct{X}$-module $\I$ equipped with a map (not necessarily injective),
\[ \alpha : \I \to \struct{X} \]
Let $\Cart(X)$ be the groupoid of generalized Cartier divisors. 
\end{defn}

\begin{rmk}
A generalized Cartier divisor $(\I, \alpha)$ is a Cartier divisor exactly when $\alpha$ is injective.
\end{rmk}

\begin{prop}
Given $f : X \to Y$ there is a map $f^* : \Cart(Y) \to \Cart(X)$ taking $\alpha : \I \to \struct{Y}$ to $f^* \alpha : f^* \I \to f^* \struct{Y} = \struct{X}$ making $\Cart$ a pre-stack.
\end{prop}

\begin{rmk}
If $f$ is non-flat it need not pullback Cartier divisors to Cartier divisors. This is why we introduce generalized Cartier divisors.
\end{rmk}

\begin{prop}
$\Cart$ is an algebraic stack in the fpqc topology. Indeed $\Cart = [\A^1 / \Gm]$. 
\end{prop}

\begin{rmk}
We will write $\Cart(R) := \Cart(\Spec{R})$.
\end{rmk}

\newcommand{\WCart}{\mathrm{WCart}}

Now let $R$ be $p$-nilpotent and consider $W(R) \to R$ giving a map $\Cart(W(R)) \to \Cart(R)$ which is functorial in $R$. Then $\wt{WCart}$ is the stack sending $R \mapsto \Cart(W(R))$. Thus there is a map $\wt{\WCart} \to \Cart$. 

\section{Oct 27}

\begin{defn}
A morphism $f : X \to S$ is \textit{syntomic} if,
\begin{enumerate}
\item $f$ is locally of finite presentation
\item $f$ is flat
\item $f$ is a locally a complete intersection morphism meaning for each $x \in X$ there is an affine open $x \in U \to V$ with ring map $A \to B$ is a local complete intersection.
\end{enumerate}
\end{defn}

\begin{prop}
Let $f : X \to S$ be flat and finitely presented. Then the following are equivalent,
\begin{enumerate}
\item $f$ is syntomic
\item $H_i(\mathbb{L}_{X/S}) = 0$ for $i < 0$ and $H_0(\mathbb{L}_{X/S})$ is locally free.
\end{enumerate}
\end{prop}

\begin{rmk}
Note that if we requred $\mathbb{L}_{X/S} \iso \Omega_{X/S}[0]$ in the derived category for a locally-free $\Omega_{X/S}$ then $f$ would be smooth. We are requring less. 
\end{rmk}

\begin{lemma}
\begin{enumerate}
\item smooth maps are syntomic
\item syntomic morphisms are stable under composition and base change
\item Consider a diagram,
\begin{center}
\begin{tikzcd}
X \arrow[rr, "f"] \arrow[rd]  & & Y \arrow[ld]
\\
& S
\end{tikzcd}
\end{center}
with $f$ flat and $X,Y$ smooth over $S$ then $f$ is syntomic

\item if $A$ is noetherian, and $X \to \Spec{A}$ is syntomic then $X$ locally is of the form,
\[ \Spec{A[T_1, \dots, T_n]/(f_1, \dots, f_r)} \]
with $f_1, \dots, f_r$ Koszul-regular. 
\end{enumerate}
\end{lemma}

\begin{cor}
If $p$ is zero on $A$ then the map,
\[ \phi A[T_1, \dots, T_d] \to A[T_1, \dots, T_d] \]
given by $T_i \mapsto T_{i}^p$ is syntomic and faithfully flat.
\end{cor}

\begin{proof}
Flat because $\phi$ is finite locally free (in fact globally free) and then faithfully flat by surjectivity. Then we conclude by (c) of the previous lemma.
\end{proof}

\begin{cor}
The sequence,
\begin{center}
\begin{tikzcd}
0 \arrow[r] & \mu_{p^n} \arrow[r] & \Gm \arrow[r] & \Gm \arrow[r] & 0
\end{tikzcd}
\end{center}
is exact in the syntomic topology. 
\end{cor}

\begin{proof}
The only nontrivially part is the surjectivity of $p^n : \Gm \to \Gm$ as sheaves on the syntomic topology. This holds because $p^n : \Gm \to \Gm$ is syntomic and faithfully flat. Thus for any map $X \to \Gm$ we pullback to get a surjective syntomic cover of $X$ lifting the map. 
\end{proof}

\end{document}
\documentclass[12pt]{article}
\usepackage{import}
\import{./}{AlgGeoCommands}

\DeclareMathOperator{\Exc}{\mathrm{Exc}}
\newcommand{\cN}{\mathcal{N}}

\begin{document}

section{Clemens Proof}


\begin{defn}
An immersed curve $f : C \to X$ is a map from a smooth curve which is everywhere maximal rank menaing the map is unramified. Associated to such a mapping is the conormal bundle $\C_f := \ker{(f^* \Omega_X \to \Omega_C)}$ and the normal bundle $\cN_f = f^* \T_X / \T_C$ which is also a bundle by the assumptions.
\end{defn}

\begin{theorem}
Let $X$ be a generic hypersurface of degree $m$ in $\P^n$. Then $X$ does not admit an irreducible family of immersed curves of genus $g$ and degree $d$ which cover a variety of codimension $< D$ where,
\[ D = \frac{2 - 2 g}{d} + m - (n+1) \]
\end{theorem}

\begin{cor}
Therefore, if $(n-1) - D \le 0$ i.e.
\[ m - 2 n \ge \frac{2g - 2}{d} \]
then there do not exist any curves of genus $g$ and degree $d$ on $X$.
\end{cor}


\subsection{Semipositivity of the Normal Bundle}

\newcommand{\Gr}{\mathbf{Gr}}

\begin{defn}
Let $C$ be a complete nonsingular curve and $\E$ a vector bundle. We say that $\E$ is \textit{semipositive} if every quotient bundle of $\E$ has non-negative degree.
\end{defn}

\begin{lemma}
Suppose that $\E$ is a flat family of vector bundles on $C$ (meaning a vector bundle on $C \times T$ flat over $T$). Let $x, y \in T$ with $x \spto y$ then
\[ \E_y \text{ is semipositive} \implies \E_x \text{ is semipositive} \]
\end{lemma}

\begin{proof}
Suppose not, then there exists $\E_x \onto \G$ of rank $s$ and $\deg{\G} < 0$. Chosoe a dvr $\Spec{R} \to T$ hitting $x \spto y$ so we reduce to $C_R$. The quotient defines $C_K \to \Gr_{C_R}(\E, s)$ which we want to extend to $C_R$. Indeed, this extends over codimension $1$ so it intersects $C_\kappa$ and then this extends to a morphism $C_\kappa \to \Gr_{C_R}(\E, s)$ hence we get a quotient $\E_y \onto \G'$ and $\deg{\G'} \le \deg{\G} < 0$ giving a contradiction. Indeed, note that $\deg{\G}$ is the degree of the map $C_K \to \Gr_{C_K}(\E, s)$ wrt the universal determinant bundle so we apply the following lemma. In order to make the universal determinant ample, we twist by a constant ample on $C$ which just shifts all degrees to be positive.
\end{proof}

\begin{lemma}
Let $C_R$ be a smooth curve over a dvr $R$. Let $X \to \Spec{R}$ be a proper flat scheme with an ample line bundle $\L$ on $X$. Suppose that $\varphi_K : C_K \to X_K$ is a map over $R$ then this extends to a map $\varphi_U  : U \to X$ on some open $U \subset C_R$ of codimension at least $2$ hence we get a map $\varphi_\kappa : C_\kappa \to X_\kappa$ called the specialization. Then $\deg{\varphi_K^* \L_K} \ge \deg{\varphi_\kappa^* \L_\kappa}$.
\end{lemma}

\begin{proof}
Since $\deg{\L^{\ot n}} = n \deg{\L}$ we can replace $\L$ with a power such that it is very ample and hence we can replace $X$ by $\P^n_R$ and set $\L = \struct{\P^n}(1)$. Blowing up $C_R$ we can resolve the morphism to $\wt{\varphi} : \wt{C}_R \to \P_R^n$. Now the special fiber consits of a copy of $C_\kappa$ along with exceptional fibers $E_i$. Now,
\[ \deg{\wt{\varphi}_\kappa^* \L_\kappa} = \chi(\wt{C}_\kappa, \wt{\varphi}_\kappa^* \L_\kappa) - \chi(\wt{C}_\kappa, \struct{\wt{C}_\kappa}) = \chi(\wt{C}_K, \wt{\varphi}_\kappa^* \L_K) - \chi(\wt{C}_K, \struct{\wt{C}_L}) = \deg{\wt{\varphi}_K \L_K} \]
by flatness of $\wt{C}_R \to \Spec{R}$. Furthermore, I claim that,
\[ \deg{\wt{\varphi}_\kappa \L_\kappa} = (\wt{\varphi}_\kappa)_* [\wt{C}_\kappa] \cdot H_\kappa \]
Given this it is easy to conclude because,
\[ (\wt{\varphi}_\kappa)_* [\wt{C}_\kappa] = (\varphi_\kappa)_* [C_\kappa] + \sum_i (\wt{\varphi}_\kappa)_* [E_i] \]
and because $H$ is ample $(\wt{\varphi}_\kappa)_* [E_i] \cdot H \ge 0$ so we conclude by observing that 
\[ \deg{\varphi_\kappa^* \L_\kappa} = (\varphi_\kappa)_* [C_\kappa] \cdot H \] 
\end{proof}

\begin{lemma} \label{lemma:sections_imply_semipositive}
If the global sections of $\E$ span the fiber $\E_p$ for some $p \in C$ then $\E$ is semi-positive.
\end{lemma}

\begin{proof}
The condition implies that $\bigwedge^r \E$ has a nonzero section for any $r \le \rank{\E}$. Therefore, if $\E \onto \G$ then $\wedge^{\rank{\G}} \E \to \det{\G}$ is surjective so since there is a nonzero section of $\wedge^{\rank{\G}}{\E}$ there is also a nonzero section of $\det{\G}$ hence $\deg{\G} \ge 0$.
\end{proof}

\begin{lemma}
Let $C$ be a smooth curve over $k$ and $\E$ a vector bundle and let $K = \stalk{C}{\xi}$ where $\xi \in C$ is the generic point. For any $K$-subspace $V \subset \E_\xi$ there exists a unique subbundle $\E_V \embed \E$ such that $(\E_V)_\xi = V$.
\end{lemma}

\begin{proof}
Via the quotient, $V$ defines a rational map $C \rat \Gr_{C}(\E, s)$ where $s = \rank{\E} - \dim{V}$. Since $\Gr_{C}(\E, s)$ is proper, this extends to a unique map $C \to \Gr_{C}(\E, s)$ hence defining the required subbundle.
\end{proof}

\begin{rmk}
We will use this lemma in a few ways. For example, if $\varphi : \E \to \E'$ is a map of vector bundles then there exists a subbundle $K \subset \E$ such that $K \subset \ker{\varphi}$ and $\E / K \to \E'$ is injective and hence injective at almost all points. 
\end{rmk}

\begin{lemma}
Consider an exact sequence,
\begin{center}
\begin{tikzcd}
0 \arrow[r] & \E_1 \arrow[r] & \E_2 \arrow[r] & \E_3 \arrow[r] & 0
\end{tikzcd}
\end{center}
if $\E_1$ and $\E_3$ are semi-positive then $\E_2$ is semi-positive.
\end{lemma}

\begin{proof}
Suppose not and take $T = \ker{(\E_2 \to \G)}$ which falsifies the semi-positivity. Let $S$ be the minimal subbundle of $\E_2$ containing $T$ and $\E_1$. Consider the map,
\[ \eta : T \oplus \E_1 \to S \]
Then the lemma produces a subbundle $K$ of $T \oplus \E_1$ such that $(T \oplus \E_1)/K \to S$ is an isomorphism at the generic point and hence there is a nonzero map between the determinants showing that $\deg{((T \oplus \E_1)/K)} \ge \deg{S}$.
Since $K$ is a subbundle of $\E_1$ and $\E_1$ is semipositive we have $\deg{K} \le \deg{\E_1}$ and hence, 
\[ \deg{((T \oplus \E_1)/K)} \ge \deg{T} \]
Therefore $\deg{S} \ge \deg{T}$. Thus $\deg{\E_2/S} < 0$ because by assumption $\deg{\E_2} - \deg{T} = \deg{\E_2/T} < 0$ so 
\[ \deg{\E_2 / S} = \deg{\E_2} - \deg{S} \le \deg{\E_2} - \deg{T} < 0 \]
However, $S$ contains $\E_1$ so $\E_2 / S$ is a quotient of $\E_3$ contradicting the semipositivity of $\E_3$. 
\end{proof}

\subsection{Semipositivity of Normal Bundles}

Let $X$ be a smooth hypersurface of degree $m$ in $\P^n$ and let $f : C \to X$ be an immersion of degree $d$. Let $W$ be a generically chosen hypersurface of degree $m$ in $\P^{n+m}$ such that $\P^n \cap W = X$. 

\begin{lemma}
The normal bundle $N_{f}$ to the mapping,
\[ f : C \to X \subset W \]
is semi-positive.
\end{lemma}

\begin{proof}
Since we assume throughout that $m \ge 2$, we can specialize $W$ to a hypersurface $W'$ of degree $m$ in $\P^{n+m}$ which contains $\P^n$ and is non-singular at point of $f(C)$. By the specialization lemma, it suffices to prove the assertion of the lemma for $f : C \to W$ where $W$ is generic such that it contains the $\P^n$. There is a sequence of normal bundles,
\[ 0 \to N_{f, \P^n} \to N_{f, W} \to f^* N_{\P^n, W} \to 0 \]
and the fact that $N_{f, \P^n}$ is semi-positive since it is globally generated (since $\T_{\P^n}$ is globally generated) thus we just need to find some $W$ such that $f^* N_{\P^n, W}$ is semi-positive. Consider the sequence,
\[ 0 \to f^* N_{\P^n, W} \to f^* N_{\P^n, \P^{n+m}} \to f^* M_{W, \P^{n+m}} \to 0 \]
If we can find some special $W$ for which,
\[ f^* N_{\P^n, W} \cong \struct{C}^{\oplus (m-1)} \]
then the proof will be complete since any quotient of this is, by definition, globally generated hence will be semi-positive.
\bigskip\\
We do this by direct computation. Suppose $f(C)$ does not intersect the linear space of codimension $2$ given by,
\[ x_0 = x_1 = 0 \]
Then let $W$ be the hypersurface given by,
\[ x_{n+1} x_0^{m-1} + x_{n+2} x_0^{m-2} x_1 + \cdots + x_{n+m} x_1^{m-1} = 0 \]
In this case, we rewrite the map $\lambda$ in the sequence,
\begin{center}
\begin{tikzcd}
0 \arrow[r] & f^* \cN_{\P^n | W} \arrow[r] & f^* \cN_{\P^n | \P^{n+m}} \arrow[r, "\lambda"] & f^* \cN_{W|\P^{n+m}} \arrow[r] & 0
\end{tikzcd}
\end{center}
as the map,
\begin{align*}
f^* \struct{\P^n}(1)^{\oplus m} & \to f^* \struct{\P^n}(m) 
\\
(\alpha_j) & \mapsto \sum_{j = 1}^{m-1} \alpha_j x_0^{m-1-j} x_1^j 
\end{align*}
The kernel of this map is generated by,
\begin{align*}
(x_1, -x_0, & 0, \dots, 0) 
\\
(0, x_1, -x_0, & 0, \dots, 0) 
\\
(0, 0, x_1, - x_0, & 0, \dots, 0)
\\
& \vdots
\\
(0, \dots, 0, & x_1, -x_0)
\end{align*}
Since $x_0$ and $x_1$ do not vanish simulateneously on $f(C)$ (by assumption) we see,
\[ f^* \cN_{\P^n | W} \cong \struct{C}^{\oplus (m-1)} \]
\end{proof}

\subsection{Setup for Proof of the Main Theorem}

Let $X$ be a generic hypersurface of degree $m$ in $\P^n$ and we suppose that there is an irreducible algebraic family,
\begin{center}
\begin{tikzcd}
\C \arrow[r] \arrow[d] & X
\\
W 
\end{tikzcd}
\end{center}
whose image is a quasi-projective variety of codimension $D$ in $X$. Let $Y \subset \P^{n+s}$ be a smooth hypersurface with $Y \cap P^n = X$ and for generic $w \in W$ consider $f_w : \C_w \to X \subset Y$ let,
\[ R \subset H^0(\cN_{f, Y}) \]
be a subspace. We denote, for $p \in C = \C_f$, by $R_p$ the image of $R$ under $H^0(\cN_{f,Y}) \to (\cN_{f,Y})_p$. Then there is a unique sub-bundle,
\[ S \subset \cN_{f,Y} \]
such that $R \subset H^0(S)$ and, for almost all $p \in C$, the fiber of $S$ is exactly $R_p$. Next, consider the diagram,
\begin{center}
\begin{tikzcd}
R \arrow[r, hook] & H^0(\cN_{f,Y}) \arrow[d, "\nu"] 
\\
H^0(\cN_{X,Y}) \arrow[r, "\mu"] & H^0(f^* \cN_{X,Y})
\end{tikzcd}
\end{center}
Now we make an assumption,
\begin{equation}
\nu(R) = \mu(H^0(\cN_{X,Y})) \tag{$*$}
\end{equation}
Then the sections of $R$ must generate the fibers of $f^* \cN_{X,Y}$ at each point because $\cN_{X,Y}$ is globally generated (indeed $X$ is the complete intersection of sections in $\struct{Y}(1)$ so $\cN_{X,Y} = \struct{X}(1)^{\oplus s}$.  Consider the sequence,
\[ 0 \to \cN_{f,X} \to \cN_{f,Y} \to \cN_{X,Y} \to 0 \]
We know that $S \subset \cN_{f,Y}$ is a sub-bundle. The claim is that,
\[ T = S \cap \cN_{f,X} \]
is also a sub-bundle. 

\begin{lemma}
Suppose that,
\begin{center}
\begin{tikzcd}
0 \arrow[r] & \E_1 \arrow[r] & \E_1 \arrow[r] & \E_3 \arrow[r] & 0
\end{tikzcd}
\end{center}
is an exact sequence of vector bundles and $\F \embed \E_2$ is a subbundle such that $H^0(\F)$ generates $\E_3$ (meaing that $H^0(\F) \ot \struct{X} \onto \E_3$ is surjective). Then $\F \cap \E_1$ is a sub-bundle of $\E_1$.
\end{lemma}

\begin{proof}
Indeed, $\F \cap \E_1 = \ker{(\F \to \E_3)}$ which is a vector bundle as along as $\F \to \E_3$ is surjective. In our case this is obvious because the global sections of $\F$ gnerate $\E_3$. 
\end{proof}

Returning to the proof, we have a subbundle $T \subset \cN_{f,X}$ and hence an exact seqeunce,
\begin{center}
\begin{tikzcd}
0 \arrow[r] & \cN_{f,X} / T \arrow[r] & \cN_{f,Y} / T \arrow[r] & f^* \cN_{X,Y} \arrow[r] & 0
\end{tikzcd}
\end{center}
of vector bundles. Since $S \onto f^* \cN_{X,Y}$ is surjective, we have a splitting defined by,
\[ f^* \cN_{X,Y} \iso S / T \to \cN_{f,X} / T \]

\subsection{Semipositivity of the Bundle $T$}

\begin{theorem}
Supposew we are given an irreducible family of curves $W$ on $Y$ and $f$ is a generic member. If $R \subset H^0(\cN_{f,Y})$ is a subspace satisfying $(\ast)$ then $L \ot T$ is semipositive where $L = f^* \struct{\P^{n+s}}(1)$.
\end{theorem}

We want to show that $L \ot T$ is semipositive where $L = f^* \struct{\P^n}(1)$. To see this, let $p \in C$ be a point such that the sections in $R$ generate the fiber of $S$ at $p$. By Lemma~\ref{lemma:sections_imply_semipositive} we need to show that $(L \to T)_p$ is generated by global sections. Doing this amounts to, for each $t_p \in T_p$ finding a meromorphic section $\tau$ of $T$ such that,
\begin{enumerate}
\item $\tau(p) = t_p$
\item the poles of $\tau$ are contained in a a hyperplane section of $f(C)$
\end{enumerate}
Choose a section $\rho \in R$ such that $\rho_p = t_p$ (recall that we assume that $R$ generates $S_p$ and hence $T_p$. If $\rho \in H^0(\cN_{f, X})$ we can take $\tau = \rho$. Othewise, $\rho$ determines a nontrivial section of $f^* \cN_{X,Y}$ by the assumption $\nu(R) = \mu(H^0(\cN_{X,Y}))$ this is the restriction of a section $\bar{\rho}$ on $\cN_{X,Y}$. Choose a projection,
\[ \cN_{X,Y} \to \struct{X}(1) \]
such that $\bar{\rho}$ maps to a nonzero section. Choose a base-point free pencil consisting of a two dimension subspace of $f^* \struct{X}(1)$ which arises from a two-dimenisonal subspace,
\[ R_0 \subset R \]
such that $\rho \in R_0$. {\color{red} Let $R_1$ be an affine line in $R_0$ passing through $\rho$ but not the origin of $R_0$. Define $\tau_q = \rho'_q$ where $\rho'$ is unique section in $R_1$ whose image in $H^0(f^* \struct{X}(1))$ vanishes at $q$.}

\subsection{Completing the Proof}

Suppose $X$ is generic and $W$ is an irreducible variety parametrizing a family of genus $g$ curves on $X$ covering a subvariety of codimension $D$. Choose $Y$ as previous, a generic hypersurface $Y \subset \P^{n+m}$ containing $\P^n$ such that $X = Y \cap \P^n$. We will be able to extend $W \subset W'$ to an irreducible family of genus $g$ curves on $Y$ such that,
\begin{enumerate}
\item if $w \in W'$ then $f_w : C_w \to Y$ spans a linear space of dimension $\le n$
\item for generically chosen $w \in W'$ the moduli map $T_w W' \to H^0(\cN_{f, Y})$ maps isomorphically onto a subspace $R \subset H^0(\cN_{f, Y})$ satisfying ($\ast$)
\end{enumerate}

Indeed, such a family $W'$ exists because if $W$ exists on the generic hypersurface then such a family exists on the universal hypersurface so we can take $W'$ to be the subvariety of the universal $\wt{W}$ mapping into $Y \cap V$ for various linear spaces $V \subset \P^{n+m}$ of dimension $n$. Hence (a) is automatically satisfied. For (b), this holds because $\cN_{X,Y}$ is globally generated and for each deformation (change in the linear space $V$) there is a corresponding deformation in $W'$ {\color{red} PROVE THIS}

This means we can apply the results of the previous section to get sub-bundles,
\[ S \subset \cN_{f,Y} \]
and 
\[ T = S \cap \cN_{f,X} \]
and a split sequence,
\begin{center}
\begin{tikzcd}
0 \arrow[r] & \cN_{f,X} / T \arrow[r] & \cN_{f,Y}/T \arrow[r] & L^{\oplus m} \arrow[r] & 0
\end{tikzcd}
\end{center}
Also $L \ot T$ is semi-positive. Furthermore, we proved that $\cN_{f,Y}$ is semi-positive hence so is $\cN_{f,X}/T$ because it is a quotient of $\cN_{f,Y}$. In particular, $\deg{\cN_{f,X}/T} \ge 0$. Furthermore, there is a unique sub-bundle 
\[ T_V \subset T \]
such that the sections of the tangent space to $\G$ at $f$ considered as a subspace of $H^0(\cN_{f,X})$ lie in $T_V$ and generate almost all fibers of $T_V$. Now,
\[ \rank{T_V} = (n-2) - D \]
{\color{red} WHY??}
so that,
\[ \rank{T / T_V} \le D \]
By adjunction sequence,
\[ \deg{\cN_{f,X}} = \deg{\T_X|_C} - \deg{\T_C} = (n+1 - m) \deg{L} - (2 - 2 g) \]
but also,
\[ \deg{\cN_{f,X}} = \deg{T/T_V} + \deg{T_V} + \deg{\cN_{f,X}/T} \ge \deg{T/T_V} \]
because $\cN_{f,X}/T$ is semi-positive and $T_V$ is generated generically by global sections hence also semi-positive. Since $L \ot T$ is semi-positive and hence also $L \ot (T/T_V)$,
\[ \deg{L \ot (T/T_V)} \ge 0 \]
so 
\[ \deg{T/T_V} \ge - (\rank{T/T_V}) (\deg{L}) \]
Putting everything together,
\[ (n+1 - m) \deg{L} - (2 - 2 g) = \deg{\cN_{f,X}} \ge \deg{T/T_V} \ge - (\rank{T/T_V}) (\deg{L}) \ge - D \deg{L} \]
Therefore,
\[ D \ge m - (n+1) + \frac{2 - 2g}{\deg{L}} \]

\section{Ming Hao Reading Group}

\newcommand{\barkap}{\bar{\kappa}}
\newcommand{\kapbar}{\bar{\kappa}}
\newcommand{\cO}{\mathcal{O}}

Main conjectures:

\begin{conj}[Superadditivity]
Let $f : X \to Y$ be an algebraic fiber space between smooth projective varieties, and let $V \subset Y$ be the open subset over shcuh $f$ is smooth. Then,
\[ \kappa(F) + \bar{\kappa}(V) \ge \kappa(X) \]
\end{conj}

\begin{theorem}[PS22]
Superadditivity holds assumping MMP
\end{theorem}


Most general conjecture:

\begin{conj}[open smooth additivity]
If $f : U \to V$ s a smooth projective algebraic fiber space between smooth \textit{quasi}-projective varities with general fiber $F$ then,
\[ \bar{\kappa}(U) = \kappa(F) + \bar{\kappa}(V) \]
\end{conj}

\begin{theorem}
This conjecture holds for:
\begin{enumerate}
\item $V$ is log general type 
\item the base is open in an ableian variety [MP]
\item $U$ is log general type [Park22]
\item $V$ is a curve [Park22]
\item $F$ is canonically polarized and $\bar{\kappa}(V) \ge 0$ [Park22]
\item $F$ has semiample canonical bundle [Cam22]
\end{enumerate}
\end{theorem}

\begin{rmk}
Does [Cam22] imply that conjecture holds assuming MMP? 
\end{rmk}

\begin{conj}[open $C^+_{n,m}$]
Let $f : U \to V$ be a projective algebraic fiber space, with $U, V$ smooth quasi-projective varities and $\barkap(V) \ge 0$. If $F$ is the generic fiber of $F$ then
\[ \barkap(U) \ge \kappa(F) + \max \{ \barkap(V), \Var(f) \} \]
\end{conj}

\begin{conj}[log version]
Let $f : X \to Y$ be an algebraic giber space between smooth projective varities, and let $E$ be an SNC divisor on $X$ and $D$ an SNC divisor on $Y$ such that $\Supp{}{f^* D} \subset E$. Assume that $f$ is log-smooth over $V = Y\sm D$, and let $F$ be a general fiber over a point of $V$. Then,
\[ \kappa(X, K_X + E) = \kappa(Y, K_Y + D) + \kappa(F, K_F + E_F) \]
\end{conj}

\begin{theorem}[Kawamata and Maehara]
Above conjecture holds if $\kappa(Y, K_Y + D) = \dim{Y}$ ie $(Y, D)$ is log general type.
\end{theorem}

\begin{conj}
If $f : U \to V$ is smooth projective morphism of smooth quasi-projective varities, then
\[ \dim{U} - \barkap(U) \ge \dim(V) - \barkap(V) \]
\end{conj}

\begin{rmk}
This follows from the open-smooth conjecture. It holds for $V$ open of an abelian variety by [PS14].
\end{rmk}

The papers:

\subsubsection{Meng and Popa}

Main result: additivity for smooth locus in map to abelian variety.

\begin{theorem}
Let $f : X \to A$ be an algebraic fiber space, with $X$ a smooth projective variety and $A$ an abelian variety. If $f$ is smooth over an open $V \subset A$ let $U = f^{-1}(V)$  and $F$ be the general fiber then,
\begin{enumerate}
\item $\bar{\kappa}(V) = \bar{\kappa}(U) \ge \kappa(X)$
\item if $V$ is big then $\barkap(U) = \barkap(X) = \barkap(F)$. 
\end{enumerate}
\end{theorem}

\subsection{Proof of Lemma 2.6}

\begin{lemma}
Let $f : X \to Y$ be a proper birational morphism between normal varties. Let $E$ be the locus of $X$ on which $f$ is not an isomorphism. Then $f_* \struct{X}(n E) = \struct{Y}$ for all $n \ge 0$. 
\end{lemma}

\begin{proof}
First, $E \to f(E) = Z$ has connected fibers. Indeed, $f_* \struct{X} = \struct{Y}$ because $Y$ is normal. Then $E$ is a divisor, otherwise the map would be \etale in codimension $1$ and hence \etale and hence an isomorphism. Therefore, $f : X \setminus E \to Y \setminus Z$ is an isomorphism so by Harthogs we see that,
\[ H^0(X, \struct{X}) \to H^0(X \setminus Z, \struct{X}) \]
is an isomorphism thus proving the claim.
\end{proof}

\begin{lemma}
Let $D, D'$ be effective $\Q$-divisors with $\Supp{}{D} = \Supp{}{D'}$ then $\kappa(X, D) = \kappa(X, D')$. 
\end{lemma}

\begin{proof}
Indeed, there are rational numbers $a,b \in \QQ_{>0}$ such that $a D \le D' \le b D$ and therefore
\[ \kappa(X, a D) \le \kappa(X, D') \le \kappa(X, b D) \]
but $\kappa(X, r D) = \kappa(X, D)$ for all $r \in \QQ_{>0}$ and therefore these are equalities. 
\end{proof}

\begin{lemma}
Let $X$ be a smooth projective variety with $\kappa(X) \ge 0$. Let $Z \subset X$ be a closed reduced subscheme and $V = X \setminus Z$. Assume that $Z = W \cup D$ where $\codim{X, W} \ge 2$ and $D$ is a divisor. Then,
\[ \barkap(V) = \kappa(X, K_X + D) \]
\end{lemma}

\begin{proof}
Choose a resolution $\mu : Y \to X$ of $(X, Z)$ such that $\mu$ is an isomorphism over $X \setminus Z$ and $\mu^{-1}(Z)$ is an snc divisor with support $G$. Hence by definition,
\[ \kapbar(V) = \kappa(Y, K_Y + G) \]
Since $\kappa(X) \ge 0$ there is an effective $\Q$-divisor $E$ such that $K_X \sim_{\Q} E$. Therefore,
\[ K_Y \sim \mu^* K_X + F \sim_{\Q} \mu^* E + F \]
where $F$ is an effective $\mu$-exceptional dvisor on $Y$ such that $\Supp{}{F} = \Exc(\mu)$ because $Y$ is smooth. Then
\[ \kappa(X, K_X + D) = \kappa(Y, \mu^* (K_X + D)) = \kappa(Y, \mu^* E + \mu^* D) = \kappa(Y, \mu^* E + \mu^* D + F) \]
because $\mu_* \struct{Y}(n F) = \struct{X}$ for all $n \ge 0$ since $F$ is exceptional. On the other hand,
\[ \kappa(Y, K_Y + D) = \kappa(Y, \mu^* E + F + G) \]
However, these divisors are effective with the same support\footnote{We needed to add $F$ to get $\mu^* (K_X + D) + F$ otherwie this migth not have the same support as $K_Y + G$ if $W$ has components disjoint from $D$ that we had to blow up.} so by the previous lemma,
\[ \kappa(Y, \mu^* E + \mu^*D + F) = \kappa(Y, \mu^* E + F + G) \]
\end{proof}

\begin{rmk}
This is false if $\kappa(X) = -\infty$. Consider $X = \P^2$ and $Z$ a highly singular plane curve of degree at least $4$. Then $\kappa(X, K_X + D) = 2$ but if the singularities are bad enough than in the log resolution,
\[ K_Y + E = \mu^* (K_X + D) + \sum a_i E_i \]
where $a_i$ are the log discrepancies we may have the $a_i$ negative enough so that $\kappa(Y, K_Y + E) < 2$. Indeed, if $k$ is large enough so that $k a_i$ are integers then,
\[ H^0(Y, \struct{Y}(k(K_Y + E))) = H^0(X, \struct{X}(k (d-3) H) \ot \I) \]
where $\I$ is the ideal at the non-lc centers with vanishing order $k a_i$ at the negative ones. The $k (d - 3) H$ grows as a polynomial of degree $2$ but the number of conditions to vanish to order $k a_i$ also grows as a polynomial of degree $2$ so it is a careful balancing act. 
\bigskip\\
The problem is that while $K_Y + G$ may be effective, it might not be effective with support along each exceptional so we cannot say that $K_Y + G$ and $\mu^* (K_X + D) + F$  contain effective divisors with the same support in their linear series.
\end{rmk}

\subsection{Questions and Takeaways}

\begin{enumerate}
\item cool trick to reduce to the case where the map $f : X \to A$ is smooth away from codim $2$ on the base
\item Is the only point of doing the isogeny trick in Thm B to get ampleness of the part pulled back from $A \to A_k$? 
\item It looks like the following is true: if $f : X \to A$ is a surjection map smooth away from codimension $2$ on $A$ then the Stein factorization $f : X \to B \to A$ is given by an isogeny $B \to A$ of abelian varities and $X \to B$ is also smooth away from codim $2$.
\end{enumerate}

\begin{lemma}
Let $f : X \to Y$ be a proper morphism of smooth $\CC$-varieties and let $X \to Y' \to Y$ be the Stein factorization. Then if $f$ is smooth over $y \in Y$ then $X \to Y'$ is smooth over each point in the fiber of $y$ and $Y \to Y'$ is \etale at $y$. 
\end{lemma}

\begin{proof}
We can shrink $Y$ to assume that $f$ is smooth. By $E_1$-degeneration of the Hodge-to-de Rham spectral sequence an Ehresmann's theorem, $R^i f_* \struct{X}$ are vector bundles compatible with base change. Hence $Y' = \rSpec{Y}{f_* \struct{X}}$ is flat over $Y$. IT is faithfully flat also so we conclude that $f' : X \to Y'$ is flat and hence smooth since its fibers are smooth since they are reduced components of the fibers of $f$. Then we use the sequence, 
\begin{center}
\begin{tikzcd}
0 \arrow[r] & f'^* \Omega_{Y'/Y} \arrow[r] & \Omega_{X/Y} \arrow[r] & \Omega_{X/Y'} \arrow[r] & 0
\end{tikzcd}
\end{center}
But $\Omega_{X/Y}$ is a vector bundle and $\Omega_{Y'/Y}$ is torsion since $g$ is generically \etale so we conclude that $\Omega_{Y'/Y} = 0$ because $f'$ is faithfully flat. Thus $g$ is \etale. 
\end{proof}


Is this true in general? We can shrink $Y$ to assume that $f$ is smooth. Then there is an exact sequence,
\begin{center}
\begin{tikzcd}
0 \arrow[r] & f'^* \Omega_{Y'/Y} \arrow[r] & \Omega_{X/Y} \arrow[r] & \Omega_{X/Y'} \arrow[r] & 0
\end{tikzcd}
\end{center}

\subsubsection{Campana}

\begin{theorem}
Let $f : X \to Y$ be a submersive projective holomorphic map between connected complex quasi-projective manifolds. Assume that its fibers have semiample canonical bundles then,
\[ \barkap(X) = \kappa(X_y) + \barkap(Y) \]
\end{theorem}


\subsubsection{VIEHWEG’S HYPERBOLICITY CONJECTURE FOR FAMILIES
WITH MAXIMAL VARIATION}

Shows that Viehweg's conjecture is true for maximal variation and fiber with good minimal model (eg general type). 


\subsubsection{ALGEBRAIC FIBER SPACES OVER ABELIAN VARIETIES}

Main theorem:

\begin{theorem}
Let $f : X \to Y$ be an algebraic fiber space with general fiber $F$. Assume that $Y$ has maximal Albanese dimension, then $\kappa(X) \ge \kappa(F) + \kappa(Y)$. 
\end{theorem}

\section{Nathan References}

\begin{enumerate}
\item \chref{https://arxiv.org/abs/1612.05921}{Junyan Cao}

\item \chref{https://arxiv.org/pdf/1912.12742.pdf}{BB in char p}

\item \chref{https://arxiv.org/pdf/1206.0438.pdf}{Moduli of products}
\end{enumerate}

\section{Rational Singularitie}

\begin{theorem}[Formal Functions, EGAIII Theorem 4.1.5]
Let $f : X \to Y$ be a proper morphism of noetherian scheme and $\F$ a coherent sheaf on $X$. Let $Z \subset Y$ be a closed subsheme. Then,
\[ (R^i f_* \F)^{\wedge}_{Z} = R^i f_* (\F|_{\hat{X}}) := \ilim R^i f_* (\F \ot_ {\struct{X}} \struct{Y}/\I_Z^{n+1}) \] 
\end{theorem}

{\color{red} CHECK WHAT THIS MEANS}

\begin{lemma}
Let $Y$ be a smooth variety and $Z \subset Y$ a smooth subvariety of codimension $c$. Then let $X = \Bl_{Z}(Y)$ and $\pi : X \to Y$ be the natural map with $E = \pi^{-1}(Z)$. Then
\[ R^i \pi_* \struct{X}(n E) =
\begin{cases}
\struct{Y} & i = 0
\\
0 & i \neq 0, c-1
\\
(\struct{Y} / \I_Y^{k - (c-1)})^\vee \ot \det{\cN_{Z|Y}} & i = c - 1
\end{cases} \]
\end{lemma}

\begin{proof}
We use the theorem on formal functions and induction. Over $Y \setminus Z$ this is obvious so we just need to check on stalks for $z \in Z$. Let $E_n$ be the formal completion of the exceptional whose structure sheaf is $\struct{X} \ot_{\struct{X}} \struct{Y} / \I_Z^{n+1}$. Therefore, there is an exact sequence,
\begin{center}
\begin{tikzcd}
0 \arrow[r] & \struct{X}(-nE)|_E \arrow[r] & \struct{E_n} \arrow[r] & \struct{E_{n-1}} \arrow[r] & 0
\end{tikzcd}
\end{center}
Furthermore,
\[ (R^i \pi_* \struct{X}(k E))^{\wedge}_Z = \ilim_n R^i \pi_* \struct{E_n}(k E) \]
so it suffices to show this is zero for $i > 0$ and the natural map from $\struct{Y}$ is an isomorphism for $i = 0$. Note that $E = \P_Z(\C_{E/Z})$ is a projective bundle and $\struct{X}(E)|_E = \struct{P}(-1)$. Therefore, the sequence is,
\begin{center}
\begin{tikzcd}
0 \arrow[r] & \struct{P}(n-k) \arrow[r] & \struct{E_n}(k E) \arrow[r] & \struct{E_{n-1}}(k E) \arrow[r] & 0
\end{tikzcd}
\end{center}
Since $R^{i} \pi_* \struct{P}(n-k) = 0$ for $0 < i < r$ and similarly for $\struct{E}(kE) = \struct{P}(-k)$ we win except for $i = 0$ and $i = r$. For top cohomology, the maps are always surjective 
\end{proof}

\begin{prop}
Let $f : X \to Y$ be a proper birational morphism of smooth varities. Then $\RR f_* \struct{X} = \struct{Y}$.
\end{prop}

\begin{proof}
There is a morphism $\pi : X' \to X$ given as an iterated blowup at smooth centers such that $X' \to X \to Y$ is also an iterated blowup at smooth centers. Therefore, $\RR \pi_* \struct{X'} = \struct{X}$ so $\RR f_* \struct{X} = \RR (f \circ \pi)_* \struct{X'} = \struct{Y}$. 
\end{proof}

\begin{cor}
Let $f : X \to Y$ be a proper birational morphism of smooth varities. Then $\RR f_* \omega_X = \omega_Y$.
\end{cor}

\begin{proof}
Via Grothendieck duality,
\begin{align*}
\RR f_* \omega_X & = \RR f_* \RR \shHom{\struct{X}}{\struct{X}}{\omega_X} = \RR f_* \RR \shHom{X}{\struct{X}}{f^* \omega_Y \ot \omega_X \ot f^* \omega_Y^{-1}}
 = \RR f_* \RR \shHom{X}{\struct{X}}{f^! \omega_Y} 
\\
&= \RR \shHom{Y}{\RR f_* \struct{X}}{\omega_Y} = \RR \shHom{Y}{\struct{Y}}{\omega_Y} = \omega_Y
\end{align*}
\end{proof}

\section{Koll\'{a}r Vanishing}

\begin{lemma} 
Let $X$ be a variety over an algebraically closed field $k$, and let $\L$ be a line bundle and $s \in H^0(X, \L^{\ot m}$ for some $m \ge 1$ with $D = Z(s)$. Then there exsits a finite flat morphism $f : Y \to X$ of degree $m$ where $Y$ is a $k$-scheme such that setting $\L' = f^* \L$ there is a section
\[ s' \in H^0(Y, \L') \quad (s')^{\ot m} = f^* s \]
moreover,
\begin{enumerate}
\item if $X$ and $D$ are smooth then so are $Y$ and $D' = Z(s')$
\item the map $f : D' \to D$ is an isomorphism
\item there is a canonical isomorphism,
\[ f_* \struct{Y} \cong \struct{X} \oplus \L^{-1} \oplus \cdots \L^{-(m-1)} \]
of $\struct{X}$-algebras where the RHS is given the multiplication structure via $s^{\ot m} : \L^{-m} \to \struct{X}$
\item for every $p \ge 1$ one has
\[ f_* \Omega^p_Y \cong \Omega^p_X \oplus \bigoplus_{i = 1}^{m-1} \Omega_X^p(\log D) \ot \L^{-i} \]
\end{enumerate}
\end{lemma}

\begin{theorem}
Let $X$ be a smooth projective variety, $\L$ a line bundle on $X$, and $s \in H^0(X, \L^{\ot m})$ a nonzero section such that $D = Z(S)$ is a smooth divisor. Then the multiplication by $s$ map,
\[ H^j(X, \omega_X \ot \L) \xrightarrow{s} H^j(X, \omega_X \ot \L^{m+1}) \]
is injective. 
\end{theorem}

\begin{proof}
Consider the cyclic covering construction $f : Y \to X$ defined by $s$. Since $f$ is finite,
\[ H^j(Y, \struct{Y}) = H^j(Y, f_* \struct{Y}) \cong H^j(X, \struct{X}) \oplus \bigoplus_{i = 1}^{m-1} H^j(X, \L^{-j}) \]
and
\[ H^j(Y, \Omega_Y^p) = H^j(Y, f_* \Omega_Y^p) \cong H^j(X, \Omega_X^p) \oplus \bigoplus_{i = 1}^{m-1} H^j(X, \Omega_X^p(\log D) \ot \L^{-i}) \]
The exterior derivative $\d : \struct{Y} \to \Omega_Y^1$ induces
\[ \d : H^j(Y, \struct{Y}) \to H^j(Y, \Omega_Y^1) \]
which is identically zero by the $E_1$-degeneration of the Hodge-to-de Rham spectral sequence. Note that $\d$ is compatible with the above decompositions and hence induces map
\[ \d : H^j(X, \L^{-1}) \to H^j(X, \Omega_X^1(\log D) \ot \L^{-1}) \]
which are also identically zero. Recall there is a residue exact sequence,
\begin{center}
\begin{tikzcd}
0 \arrow[r] & \Omega_X^1 \arrow[r] & \Omega_X^1(\log D) \arrow[r] & \cO_D \arrow[r] & 0
\end{tikzcd}
\end{center}
Tensoring by $\L^{-1}$ and passing to cohomology, we find that the induced map
\[ H^j(X, \L^{-1}) \to H^j(D, \L|_D^{-1}) \]
is zero. However, by a check in local coordinates, this is the map (up to multiplication by $m$) induced by restriction. Therefore, by the exact sequence,
\begin{center}
\begin{tikzcd}
0 \arrow[r] & \L^{-1} \ot \cO(-D) \arrow[r] & \L^{-1} \arrow[r] & \L^{-1}|_D \arrow[r] & 0
\end{tikzcd}
\end{center}
there are surjections
\[ H^j(X, \L^{-1} \ot \cO(-D)) \onto H^j(X, \L^{-1}) \]
Recall that $\L^{\ot m} = \struct{X}(D)$ then Serre duality implies the conclusion.
\end{proof}

\begin{theorem}[Kodara vanishing]
Let $\L$ be an ample line bundle and $X$ a smooth projective variety. Then $H^j(X, \L \ot \omega_X) = 0$ for all $j > 0$.
\end{theorem}

\begin{proof}
For $m \gg 0$ Serre vanishing implies $H^j(X, \omega_X \ot \L^{m+1}) = 0$ for $j > 0$ and $\L^{m}$ is globally generated and hence $| m \L |$ contains a smooth divisor. Then we apply the injectivity theorem to conclude that,
\[ H^j(X, \omega_X \ot \L) \to H^j(X, \omega_X \ot \L^{m + 1}) = 0 \]
is injective so we conclude.
\end{proof}

\begin{theorem}[Koll\'{a}r Vanishing]
Let $f : X \to Y$ be a morphism from a smooth projective variety $X$ to a projective variety $Y$, and let $L$ be an ample line bundle on $Y$. Then,
\[ H^j(Y, R^i f_* \omega_X \ot \L) = 0 \]
for all $i$ and all $j > 0$.
\end{theorem}

\begin{proof}
Let $m \gg 0$ be large enough that $\L^{\ot m}$ is very ample. Let $B \in | m \L |$ be a general element and $D = f^* B$ then Bertini's theorem implies that $D$ is a smooth hypersurface of $X$ since the map is basepoint free. We apply the injectivity result to $f^* \L$ and the divisor $D$ to see that the maps,
\[ H^j(X, \omega_X \ot f^* \L) \xrightarrow{D} H^j(X, \omega_X \ot f^* \L^{\ot m + 1}) \] 
are injective for all $j$. Denote $f_D : D \to B$ the restriction of $f$ to $D$. By induction on dimension, we may assume that,
\[ H^j(Y, R^i f_{D*} \omega_D \ot L|_B) = 0 \]
for all $i$ and all $j > 0$. The adjunction formula $\omega_D = \omega_X|_D \ot \struct{D}(D)$ implies that,
\[ \omega_D \cong \omega_X|_D \ot f_D^* \L|_B^{\ot m} \]
Therefore there is a short exact sequence,
\begin{center}
\begin{tikzcd}
0 \arrow[r] & \omega_X \ot f^* \L \arrow[r] & \omega_X \ot f^* \L^{\ot m + 1} \arrow[r] & \omega_D \ot f_D^* \L|_B \arrow[r] & 0 
\end{tikzcd}
\end{center}
which induces a long exact sequence
\begin{center}
\begin{tikzcd}
\cdots \arrow[r] & R^i f_* \omega_X \ot \L \arrow[r, "B"] & R^i f_* \omega_X \ot \L^{\ot m+1} \arrow[r] & R^i f_{D*} \omega_D \ot \L|_B \arrow[r] & \cdots 
\end{tikzcd}
\end{center}
We can however choose $B$ sufficiently general to miss all the associated points of all the $R^i f_* \omega_X$ and hence
\[ R^i f_* \omega_X \ot \L \xrightarrow{B} R^i f_* \omega_X \ot \L^{\ot m+1} \]
are injective. Therefore, the long exact sequence splits into short exact sequence,
\begin{center}
\begin{tikzcd}
0 \arrow[r] & R^i f_* \omega_X \ot \L \arrow[r, "B"] & R^i f_* \omega_X \ot \L^{\ot m+1} \arrow[r] & R^i f_{D*} \omega_D \ot \L|_D \arrow[r] & 0
\end{tikzcd}
\end{center}
We can also choose $m \gg 0$ suhc that the higher cohomology of all $R^i f_* \omega_X \ot \L^{\ot m+1}$ vanishes. Compinded with the inductive assume we conclude that,
\[ H^j(Y, R^i f_* \omega_X \ot \L) = 0 \]
for all $j \ge 2$ and all $i$. However, for $j = 1$ we cannot control the term $H^0(R^i f_{D*} \omega_D \ot \L|_B)$ mapping to it. Instead, we use the Leray spectral sequence
\[ E^{p,q}_2 = H^p(Y, R^q f_* \omega_X \ot \L) \implies H^{p+q}(X, \omega_X \ot f^* \L) \]
Since $E^{p,q}_2 = 0$ for $p \ge 2$ and all $q$ the spectral sequence is concentrated in two adjacted columns and hence degenerates at $E_2$. This means there are injections
\[ E_2^{1,i} = H^1(Y, R^i f_* \omega_X \ot \L) \embed H^{i+1}(X, \omega_X \ot f^* \L) \]
but the injectivity theorem shows that,
\[ H^{i+1}(X, \omega_X \ot f^* \L) \embed H^{i+1}(X, \omega_X \ot f^* \L^{\ot m + 1}) \]
Applying the spectral sequence also for $\L$ replaced by $\L^{\ot m+1}$ we get a commuting square of injections
\begin{center}
\begin{tikzcd}
H^1(Y, R^i f_* \omega_X \ot \L) \arrow[r] \arrow[d] & H^{i+1}(X, \omega_X \ot f^* \L) \arrow[d]
\\
H^1(Y, R^i f_* \omega_X \ot \L^{\ot m+1}) \arrow[r] & H^{i+1}(X, \omega_X \ot f^* \L^{\ot m+1}) 
\end{tikzcd}
\end{center}
but the bottom left term vanishes by Serre vanishing so we conclude that $H^1(Y, R^i f_* \omega_X \ot \L) = 0$ completing the proof by induction.
\end{proof}


\section{Prismatic Cohomology}

\newcommand{\BDel}{\Delta}
\newcommand{\LL}{\mathbb{L}}
\newcommand{\ad}{\mathrm{ad}}

Our goal will be the following theorem about the topology of algebraic varities. 

\begin{theorem}
et $X$ be a smooth, proper, $\CC$-variety with unramified good reduction at $p$. Let $i < p - 2$ and $W \subset X$ and Zariki open. Then the image of the restriction map,
\[ H^i(X, \FF_p) \to H^i(W, \FF_p) \]
has dimension at least $h_X^{0,i} := \dim H^0(X, \Omega_X^i)$.
\end{theorem}

This statement amounts to showing that certain cohomology classes are not $p$-divisible.
\bigskip\\
There is a version with $\Q$-coefficients that follows from Hodge theory.

\begin{theorem}
Let $X$ be a smooth, proper, complex variety and $W \subset X$ any Zariki open. Then the image of the restiction map,
\[ H^i(X, \QQ) \to H^i(W, \QQ) \]
has dimension at least $h_X^{0,i} := \dim H^0(X, \Omega_X^i)$.
\end{theorem}

\begin{proof}
The map $H^i(X, \QQ) \to H^i(W, \QQ)$ is a morphism of mixed hodge structures. Posibly passing to a log resolution $\pi : \wt{X} \to X$ of $Z = X \sm W$ we may assume that $\pi^{-1}(Z) = D$ is an snc divisor (note the birational modification does not change $h^{0,i}_X$ and the map $H^i(\wt{X}, \QQ) \to H^i(W, \QQ)$ factors through $H^i(X, \QQ)$ so its image is the same). Then there is a commutative diagram,
\begin{center}
\begin{tikzcd}
H^0(\wt{X}, \Omega^i_{\wt{X}}) \arrow[d] \arrow[r] & H^0(\wt{X}, \Omega^i_{\wt{X}}(\log{D})) \arrow[d]
\\
H^i(\wt{X}, \QQ) \ot_{\QQ} \CC \arrow[r] & \mathrm{Gr}^W_i H^i(W, \QQ) \ot_{\QQ} \CC
\end{tikzcd}
\end{center}
where the top map is injective and the downward maps are injective. This immediately implies the claim. 
\end{proof}

The real power of our main result is that it works integrally. This has applications to essential dimension to be discussed later.

\subsection{Mod-p Cohomology}

We need the following about Delinge-Illusie's treatment of de Rham cohomology and basics of prismatic cohomology.


\subsubsection{Log de Rham cohomology}

Let $k$ be a perfect field of characteritic $p$, and let $X$ be a smooth $k$-scheme. Suppose that $X$ is equipped with a normal crossings divisor $D \subset X$. Let $\Omega^\bullet_{X/k}(\log{D})$ denote the de Rham complex with log poles in $D$. 
\par 
Let $(X^1, D^1)$ be the base change by Frobenius $F_k : \Spec{k} \to \Spec{k}$ and $F_{X/k} : X \to X^1$ denote the relative Frobenius. It is a finite flat map (since $X$ is smooth) of $k$-schemes such that $F_{X/k} : D \to D^1$.

\begin{lemma}
Suppose that $(X, D)$ admits a lift to $W_2(k)$ called $(\wt{X}, \wt{D})$ with $\wt{D}$ a snc divisor flat over $W_2(k)$. Then for $j < p$,
\[ H^0(X^1, \Omega^j_{X^1/k}(\log{D^1})) \embed H^j(X, \Omega^\bullet_{X/k}(\log{D})) \]
is canonically a direct summand. 
\end{lemma}

\begin{proof}
This follows from the existence of the Cartier operator in the same way as in Deligne-Illusie.
\end{proof}

\subsubsection{Prisms}

Let $K$ be a field of characteristic $0$. By a \textit{p-adic valuation} on $K$ we mean a rank one valuation $\nu$ on $K$, with $\nu(p) > 0$. We suppose that $K$ is complete with respect to $\nu$ with ring of integers $\cO_K$ and perfect residue field $k$. We will only recall exactly as much about prismatic cohomology as necessary.

\begin{defn}
A $\delta$\textit{-ring} is a pair $(R, \delta)$ where $R$ is a commutative ring and $\delta : R \to R$ is a set map such that,
\begin{enumerate}
\item $\delta(0) = \delta(1) = 0$
\item $\delta(xy) = x^p \delta(y) + y^p \delta(x) + p \delta(x) \delta(y)$
\item $\delta(x + y) = \delta(x) + \delta(y) + \frac{x^p + y^p - (x+y)^p}{p}$
\end{enumerate}
Note that the last term exists as some universal polynomial with integer coefficents. 
\end{defn}

We think of this as a sort of ``derivation along the $p$-direction''. It is also related to lifting Frobenius on $R / p$. Indeed, if $\phi(x) = x^p + p \delta(x)$ then $\phi : R \to R$ is a ring map by property (c) and obviously it lifts $x \mapsto x^p$ on $R / p$. In fact, if $R$ is $p$-torsionfree then lifts of Frobenius are exactly the same as $\delta$-ring structures.

\begin{defn}
Let $(A, I)$ be a pair where $A$ is a $\delta$-ring and $I \subset A$ is an ideal. The pair is a \textit{prism} if
\begin{enumerate}
\item  $I \subset A$ is invertible (defines a Cartier divisor on $\Spec{A}$) 
\item $A$ is derived $(p,I)$-complete
\item $p \in I + \phi(I) A$
\end{enumerate}
\end{defn}

\begin{example}
Let $A$ be a $p$-torsionfree and $p$-complete $\delta$-ring then $(A, (p))$ is a prism.
\end{example}

\begin{example}
The \textit{Breuil-Kisin} prism. Assume that $\nu$ on $K$ is discrete. Set $A = W(k)[[u]]$ equipped with Frobenius $\varphi$ extending Frobenius on $W(k)$ by $u \mapsto u^p$. Equip $A$ with the map $A \to \cO_K$ sending $u \mapsto \pi$ some uniformizer. It kernel is generated by an Eisenstein polynomial $E(u) \in W(k)[u]$ for $\pi$. In fact, in applications we will assume $\cO_K = W(k)$ and $\pi = p$. Then $(A, E(u) A)$ is the Breuil-Kisin prism.
\end{example}

\begin{example}
Suppose that $K$ is algebraically closed. Let $R = \ilim \cO_K / p$ taking the limit over Frobenius. We take $A = W(R)$. Any element $(x_0, x_1, \dots) \in R$ lifts uniquely to a sequence $(\hat{x}_0, \hat{x}_1, \dots,) \in \cO_K$ with $\hat{x}_i^p = \hat{x}_{i-1}$. Then there is a natural surjective map of rings $\theta : A \to \cO_K$ sending a Teichmuller element $x$ as above to $\hat{x}_0$. The kernel of $\theta$ is principal, generated by $\xi = p - [\ul{p}]$ where $\ul{p} = (p, p^{1/p}, \dots)$ then $(A, \xi A)$ is an example of a perfect prism. 
\end{example}

\subsubsection{Logarithmic Cohomology}

We will use logarithmic formal schemes over $\cO_K$. We will consider logarithmic \etale cohomology meaning the natural cohomology on the site of log \etale covers of logarithmic schemes. The main fact we will use is the following comparison result:

\begin{theorem}
Let $k$ be an algebraically closed field and $X$ a smooth $k$-scheme. Let $D \subset X$ be an snc divisor and $X_D^{\log}$ the log structure induced by $D$. Then there is a canonical isomorphism,
\[ H^i_{\et}(X_D^{\log}, \mu) \iso H^i(X \sm D, \mu) \]
{\color{red} COEFFICIENTS}
\end{theorem}

\begin{proof}
Idea: show that any finite \etale map $Y \to X \sm D$ extends canonically to a finite log-\etale map $\ol{Y} \to X_D$ which proves the statment for $i = 1$ then use dimension shifting and some spectral sequence. To show the claim, take the normalization of $Y$ in $X$ which gives a finite map $Y \to X$ ramified only over $D$ by Zariski nagata purity. Then a local check shows that this map is log-\etale {\color{red} WHY?} 
\end{proof}

\subsubsection{Prismatic Cohomology}

Let $K$ be either discretely valued or algebraically closed. Let $X$ be a formal smooth $\cO_K$-scheme equipped with a relative normal crossings divisor $D$. Write $X_D$ for log structure induced by $D$. We will denote by $X_{D,K}$ the associated log adic space giving by analytification. 
\par 
The \textit{prismatic cohomology} of $X_D$ is the complex of $A$-modules $R \Gamma_{\BDel}(X_D/A)$ equipped with a $\varphi$-semi-linear map $\varphi$. The mod $p$ cohomology is given by setting,
\[ \ol{R \Gamma_{\BDel}(X_D/A)} = R \Gamma_{\BDel}(X_D/A) \ot_A^{\LL} A / p A \]
and we will denote by $\ol{H^i_{\BDel}(X_D/A)}$ the cohomology of $\ol{R \Gamma_{\BDel}(X_D/A)}$. Then we have the following properties:
\begin{enumerate}
\item There is a canonical isomorphism of commutative algebras in $D(A)$
\[ R \Gamma(\Omega^\bullet_{X_k/k}(\log{D_k})) \cong \ol{R \Gamma_{\BDel}(X_D/A)} \ot^{\LL}_{A/pA, \varphi} l \]
\item If $K$ is algebraically closed then there is an isomorphism of commutative algebras in $D(A)$
\[ R \Gamma_{\et}(X_{D,K}, \FF_p) \cong \ol{R \Gamma_{\BDel}(X_D/A)}[1/\xi]^{\varphi=1} \]
\item the linear map,
\[ \varphi^* \ol{R \Gamma_{\BDel}(X_D/A)} \to \ol{R \Gamma_{\BDel}(X_D/A)} \]
becomes an isomorphism in $D(A)$ after inverting $u$ (resp $\xi$) if $K$ i discrete (resp. algebraically closed). For each $i \ge 0$, there is a canonical map,
\[ V_i : \ol{H^i_{\BDel}(X_D/A)} \to H^i(\varphi^* \ol{R \Gamma_{\BDel}(X_D/A)} \]
\item Let $K'$ be a field complete with respect to a $p$-adic valuation, and which is either discrete or algebraically closed. Let $A' \to \cO_{K'}$ be the corresponding prism, as defined above. Suppose $K \to K'$ is a map of valued field and $A \to A'$ is compatbile with the projection to $\cO_K \to \cO_{K'}$ and Frobenius. Then there is a canonical isomorphism
\[ \ol{R \Gamma_{\BDel}(X_D/A)} \ot_A^{\LL} A' \cong \ol{R \Gamma_{\BDel}(X_{D, \cO_{K'}}/A')} \]
\item When $X$ is proper over $\cO_K$ then $\ol{R \Gamma_{\BDel}(X_D/A)}$ is a perfect complex of $A/p$-modules.
\item Suppose that $K$ is algebraically closed, and that $X$ is proper over $\cO_K$ then for each $i \ge 0$ there are natural isomorphisms
\[ H^i_{\et}(X_{D,K}, \FF_p) \ot_{\FF_p} A / p A[1/\xi] \cong \ol{H^i_{\BDel}(X_D/A)}[1/\xi] \]
\end{enumerate}

\subsection{Main Result}

\begin{prop} \label{main_prop}
Let $X$ be a proper smooth scheme over $\cO_K$ equipped with a relative normal crossings divisor $D \subset X$. Set $U = X \sm D$ and $W \subset U_C$ be a dense open subscheme. If $0 \le i < p -2$ then,
\[ \dim_{\FF_p} \im{(H^i_{\et}(U_C, \FF_p) \to H^i_{\et}(W, \FF_p))} \ge h^{0,i}_{(X_C, D_C)} \]
\end{prop}
\bigskip
Let's see how this implies the theorem. Let $Y$ be a proper smooth scheme over $\CC$ and $D \subset Y$ a normal crossings divisor. We say that $(Y, D)$ has \textit{good reduction at} $p$ if there exists an algebraically closed field $C \embed \CC$ over which $(Y, D)$ is defined and a $p$-adic valuation on $C$ with ring of integers $\cO_C$ and an extension to a smooth proper $\cO_C$-scheme $Y^\circ$ with a relative normal crossings divisor $D^\circ \subset Y^\circ$ over $\cO_C$ extending $D$. We say that $(Y, D)$ has \textit{unramified good reduction} at $p$ if in addition $(Y^\circ, D^\circ)$ can be chosen so that it descends to an absolutely unramified\footnote{meaning unramified over $\Z_{(p)}$} dvr $\cO \subset \cO_C$. 

\begin{rmk}
This condition is actually easily checkable. Indeed if $Y$ is a smooth proper finite type $\CC$-scheme then it spreads out to a smooth proper scheme $\mathcal{Y} \to \Spec{A}$ over some finite type $\ZZ$-algebra $A \subset \CC$. Now suppose there exists $\p \subset A$ such that $\Spec{A} \to \Spec{\ZZ}$ is smooth at $\p$ and $\p \mapsto (p)$. This is nothing more than saying that $p$ is not contained in the Jacobian ideal. Then choose a minimal prime $\xi$ over $p A$ since $\xi \spto \p$ we see that $\Spec{A} \to \Spec{\Z}$ is smooth at $\xi$ and hence $A_{\xi} \subset \CC$ is a $p$-adic dvr unramified over $\ZZ_{(p)}$ by smoothness. Then we extend this $p$-adic valuation to $\CC$ and $\cO = A_{\xi}$ is our requisite unramified dvr. 
\end{rmk}

\begin{cor}
Let $Y$ be a proper smooth connected $\CC$-scheme and $D \subset Y$ a normal crossing divisor and $W \subset U := Y \sm D$ a dense ope nsubscheme. Suppose that $(Y, D)$ has unramified good reduction at $p$. If $0 \le i < p - 2$ then,
\[ \dim_{\FF_p} \im{(H^i_{\et}(U, \FF_p) \to H^i_{\et}(W, \FF_p))} \ge h^{0,i}_{(X, D)} \]
\end{cor}

This proves the main theorem if we take $D = \emptyset$.

\begin{proof}
Since the \etale cohomology groups do not change upon base change to algebraically closed fields. By assumption, we may assume that $(Y, D)$ is defined over $\cO$ unramified. Then taking the $p$-adic completion $C \subset C'$ we get $\cO \subset \cO'$ which is unramified and $p$-adically complete so we reduce to the previous case.  
\end{proof}

\begin{proof}[Proof of Proposition~\ref{main_prop}]
Let $k_C$ be the residue field of $C$. We may replace $X$ by it base change to $W(k_C)$ and assume that $C$ and $K$ have the ame residue field. Denote by $\wh{X}$ and $\wh{D}$ the formal completions of $X$ and $D$. Let $\wh{W} \subset \wh{X}$ be the formal open subschem, which is the complement of $Z_k$. Note that we have $\wh{W}_C \subset W^\ad$ so there is a commutative diagram,
\begin{center}
\begin{tikzcd}
H^i_{\et}(X_{D,C}, \FF_p) \arrow[dd, "\alpha"] \arrow[r] & H^i_{\et}(W, \FF_p) \arrow[d] 
\\
& H^i_{\et}(W^\ad, \FF_p) \arrow[d]
\\
H^i(\wh{X}_{D,C}, \FF_p) \arrow[r, "\beta"] & H^i_{\et}(\wt{X}_{C}, \FF_p) 
\end{tikzcd}
\end{center}
We need to show the following facts,
\begin{enumerate}
\item $\alpha$ is an isomorphism
\item $\dim_{\FF_p} \im{\beta} \ge h^{0,i}_{(X,D)}$
\item $H^i_{\et}(X_{D,C}, \FF_p) \cong H^i_{\et}(U_C, \FF_p)$
\end{enumerate}
\end{proof}

{\color{red} WHY IS THE FIRST LEMMA 2.2.10}

Now we will prove these three facts. 

The only hard one:

Let $X$ be a proper, smooth formal scheem over $\cO_K$ equipped with a relative normal crossing divisor $D \subset X$. Let,
\[ h^{0,i}_{(X,D)} := \dim_K H^0(X_K, \Omega^i_{X_K/K}(\log{D})) \]

\begin{prop}
Let $W \subset X \sm D$ be a dense open formal subscheme. Then for $0 \le i < p - 2$
\[ \dim_{\FF_p} \im{(H^i_{\et}(X_{D,C}, \FF_p) \to H^i_{\et}(W_C, \FF_p)} \ge h^{0,i}_{(X,D)} \]
\end{prop}

\begin{proof}
Take the prism $A$ to be $W(k)[[u]]$ with $E(u) = u - p$. We obtain a prism $A_C \to \cO_C$. There is a Frobenius compatible map $A \to A_C$ sending $u \mapsto [\ul{p}]$. Set,
\[ M_{\BDel} = \im{(\ol{H^i_{\BDel}(X_D/A)} \to \ol{H^i_{\BDel}(W/A)})} \]
which is a finitely generated $A / pA = k[[u]]$-module. There is an isomorphism,
\[ \ol{H^i_{\BDel}(X_D/A)} \ot_A^{\LL} A_C \iso \ol{H^i_{\BDel}(X_{D, \cO_C}/A_C)} \]
and similarly for the open $W$. Therefore, by {\color{red} PROPERTY} there is an isomorphism
\[ \ol{H^i_{\BDel}(X_D/A)} \ot_A^{\LL} A_C \iso \ol{H^i_{\BDel}(X_{D, \cO_C}/A_C)} \]
Then there are maps,
\begin{align*}
\ol{H^i_{\BDel}(X_D/A)} & \ot_A^{\LL} A_C [1/\xi] \cong H^i_{\et}(X_{D,C}, \FF_p) \ot_{\FF_p} A_C / p A_C [1/\xi]
\\
& \to H^i_{\et}(W_C, \FF_p) \ot_{\FF_p} A_C / p A_C[1/\xi] \to \ol{H^i_{\BDel}(W/A)} \ot_A A_C[1/\xi] 
\end{align*}
the composite is the natural map. Hence,
\[ \dim_{\FF_p} \im{(H^i_{\et}(X_{D,C}, \FF_p) \to H^i_{\et}(W_C, \FF_p))} \ge \dim_{k((u))} M_{\BDel}[1/u] \]
By {\color{red} LEMMA} $M_{\BDel}$ is a finitely generated free $k[[u]]$-module. Hence it suffices to show $\dim_k M_{\BDel} / u M_{\BDel} \ge h^{0,i}_{(X,D)}$. 


Hence using Lemma 2.2.1 again, we see that $\ol{H^j_{\BDel}(X_D/A)}$ is $u$-torsion free for $0 \le j \le i + 1$. Hence there are maps,
\begin{align*}
H^i(X_k, \Omega^\bullet_{X_k/k}(\log{D_k})) & \cong \ol{H^i_{\BDel}(X_D/A)} \ot_{A, \varphi} k \to M_{\BDel} \ot_{A, \varphi} k 
\\
& \to \ol{H^i_{\BDel}(W/A)} \ot_{A, \varphi} k \to H^i(W_k, \Omega^\bullet_{W_k/K}(\log{D})) 
\end{align*}
where the composition is the natural map. This shows that the image has dimension $\le \dim_k M_{\BDel} / u M_{\BDel}$ and it suffices to how that this dimension is $\ge h^{0,i}_{(X,D)}$. Since $W \subset X$ is dense, the map,
\[ H^0(X_k, \Omega^i_{X_k/k}(\log{D})) \to H^0(W_k, \Omega^i_{X_k/k}) \]
is injective. Hence the image has dimension at leat $\dim_k H^0(X_k, \Omega^i_{X_k/k}(\log{D_k})) \ge h^{0,i}_{(X,D)}$ {\color{red} I THINK THIS WAS AN ERROR IN THE PAPER NEED LOG D} where the last inequality follows from the upper semi-continuity of $h^0$. 
\end{proof}


\section{Running Notes}

[MP] Theorem A(2) plus $C^+_{nm}$ conjecture implies that if $f : X \to Y$ is a dominant map from a smooth projective variety to an abelian variety smooth in codimension $1$ on $A$ then $f$ is birationally isotrivial. Indeed, by the Stein factorization arugment $f$ is a fiber space composed with an isogeny then apply A(2) to get the Kodaira dimension equality and then apply $C^+_{nm}$ to get variation zero. 
\bigskip\\
In particular, this should apply to smooth maps whose fibers have good minimal models. What about the case that the fibers are rationally connected? Well Thm A(2) also shows that any rationally connected fibration over $A$ must be singular in codimension $1$ actually I think by the trick this means it must be ramified along an ample divisor (I'm not sure this follows form Park22 since its unclear if the ramification upstairs should be log general type). 
\bigskip\\
Note that [B. Taji, Birational geometry of smooth families of varieties admitting good minimal models,] proves the $C^+_{n,m}$ conjecture for smooth families whose fibers admit good minimal models. Therefore, the above argument is unconditional at least in this case.

\subsection{Numerical Kodaira Dimension}

Here are some facts: if $\kappa_\sigma(K_X) = 0$ then \chref{https://www.ms.u-tokyo.ac.jp/~gongyo/papers/nu=08.pdf}{this paper} proves that $X$ has a good minimal model. I want to apply this to the Iitaka fibration which would prove MMP in general. Clearly this can't work.
\bigskip\\
If $D$ is a pseduo-effective $\Q$-divisor then $\kappa_{\sigma}(D) \ge \kappa_{\sigma}^{-}(D) \ge \kappa(D)$ by [Prop 2.7(2) of ``Numerical Kodiara Dimension''] but if $F$ is the general fiber of the Iitaka fibration we can assume that $K_F$ is pseduo effective since otherwise $F$ would be uniruled and hence $X$ would be uniruled. So if we are in the non-uniruled case then we win by the MRC fibration (maybe this will not give all of MMP since its a conjecture that the base of MRC is not uniruled maybe, or is it just a conjecture that the base has $\kappa(X) \ge 0$ hmm?). Therefore it suffices to show that $\kappa_{\sigma}(K_F) \le 0$. Okay maybe this is the entire difficulty. Can we assume that $K_F$ is effective since it is sort of a blowup of the putative minimal Iitaka fibration whose $K_X$ is pulled back from the base? 
\bigskip\\
Okay it looks like this reduction is probably already well known. 

\end{document}
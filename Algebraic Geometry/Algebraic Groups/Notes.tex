\documentclass[12pt]{article}
\usepackage{import}
\import{../}{AlgGeoCommands}


\begin{document}

\section{Sep. 20}

\subsection{History}

\begin{enumerate}
\item 19th century
\begin{enumerate}
\item $Z(f_1, \dots, f_n) \subset \CC^n$ using analytic tools
\item Riemann's idea of moduli of algebraic curves (1857)
\end{enumerate}
\item 20th century
\begin{enumerate}
\item $Z(f_1, \dots, f_n) \subset \CC^n$ using algebraic tools (commutative algebra)
\item replace $\CC$ with algebraically closed field $k$
\item number theory: want $k = \F_p$ or $\Q$ not algebraically closed. Examples: Fermat's Last Theorem:
\[ u^n + v^n = 1 \]
want geometry for,
\[ \Q[u,v]/(u^2 + v^2 - 1) \]
but this only has finitely many points so how is there a ``geometry''. 
\end{enumerate}
\end{enumerate}

\subsubsection{Question: for any field $k$ is there a ``geometry'' for $k[X_1, \dots, X_n]/I$?}

First attempt (Weil and Zariski 1930s - 1940s) use Galois theory with algebraic sets in $\overline{k}^n$ for ideals $I \subset k[X_1, \dots, X_n]$. This only works for perfect fields (not $\F_p(t)$ which we want to consider generic families of equations over $\Z$). Weil's Foundations of Algebraic Geometry.

\subsubsection{The Weil Conjectures (1948)}

For $f_1, \dots, f_r \in \FF_p[x_1, \dots, x_n]$ then define,
\[ N_m = \{ x \in \FF_{p^m}^r \mid f_1(x) = \cdots = f_r(x) = 0 \} \]
Then we define a Zeta function,
\[ \zeta(s) = \exp{ \left( \sum_{m = 1}^\infty \frac{N_m}{m} p^{-sm} \right) } \]
which sould controll the behavior of $N_m$ as $m \to \infty$. Furthermore,
\[ Z_\CC(F_1, \dots, F_r) \subset \CC^n \quad \text{ and } \quad Z_p = \Z(F_1, \dots, F_r) \subset \overline{\FF_p}^n \]
are closely related where algebraic topology invariants of $Z_\CC$ gives formulas for counts of $Z_p$. 

\subsubsection{1950's: Chaos (K\"{a}hler, Shimura, Nagata, etc.)}

Proposal for algebraic geometry over Dedekind domains but chaotic and confusing. Then Schemes resolve all of these problems to give ``geometry over any commutative ring''. 

\subsection{Affine Algebraic Sets}

Let $k$ be an algebraically closed field. Let $\A^n = k^n$ and define a subset $Z \subset \A^n$ to be \textit{algebraic} if $Z = Z(\Sigma)$ where $\Sigma \subset k[X_1, \dots, X_n]$ is a set of polynomials. Then $Z(\Sigma) = Z(I)$ where $I$ is the ideal generated by $\Sigma$.

\begin{thm}
The algebraic sets form (the complement of) a topology on $\A^n$.
\end{thm}

\begin{rmk}
We call this the Zariski topology.
\end{rmk}
\noindent
There is a base of open sets given by,
\[ U_f = \{ x \in \A^n \mid f(x) \neq 0 \} \]

\subsubsection{Examples}

The Zariski topology on $\A^1$ has the cofinite topology. However, $\A^2 \neq \A^1 \times \A^1$ as a topological space. 

\begin{rmk}
Some weird properties of the Zariski topology,
\begin{enumerate}
\item In $\CC^n$ any nonzero open ball is Zariski dense.
\item $Z(f) = Z(f^n)$ and $Z(I) = Z(\sqrt{I})$.
\end{enumerate}
\end{rmk}

\begin{defn}
For any $Y \subset \A^n$ define,
\[ I(Y) = \{ f \in k[X_1, \dots, X_n] \mid \forall y \in Y : f(y) = 0 \} \]
which is a radical ideal.
\end{defn}

\begin{prop}[Nullstellensatz]
For a field $k$ and $\m \subset K[X_1, \dots, X_n]$ a maximal ideal. Then $K[X_1, \dots, X_n]/\m$ is a finite dimensional $K$-vector space.
\end{prop}

\begin{proof}
210B [Mat, Thm 5.3]
\end{proof}

\begin{cor}
If $K$ is algebraically closed then $K \to K[X_1, \dots, X_n]/\m$ is an isomorphism so we have $a_i \mapsto X_i$ and thus $X_i - a_i = 0$ in the quotient so,
\[ \m = (X_1 - a_1, \dots, X_n - a_n) \]
is the kernel of the map $K[X_1, \dots, X_n] \to K[X_1, \dots, X_n]/\m$. Therefore, points in $Z(J)$ correspond to $\m \supset J$. Therefore,
\[ I(Z(J)) = \bigcap_{\m \supset J} \m \]
\end{cor}

\begin{thm}
The following hold,
\begin{enumerate}
\item $I_1 \subset I_2 \implies Z(I_1) \supset Z(I_2)$
\item $Y_1 \subset Y_2 \implies I(Y_1) \supset I(Y_2)$
\item $I(Y_1 \cup Y_2) = I(Y_1) \cap I(Y_2)$
\item $Z(I_1 \cap I_2) = Z(Y_1) \cup Z(Y_2)$
\item $Z(I(Y)) = \overline{Y}$
\item $I(Z(J)) = \sqrt{J}$ (Hilbert's Nullstellensatz). 
\end{enumerate}
\end{thm}

\begin{proof}
The first three follow directly from definitions. Suppose that $x \notin Z(I_1)$ and $x \notin Z(I_2)$ then there is some $f_i \in I_i$ such that $f_i(x) \neq 0$ so $f_1(x) f_2(x) \neq 0$ but $f_1 f_2 \in I_1 \cap I_2$ so $x \notin Z(I_1 \cap I_2)$. 
\bigskip\\
Now $Y \subset Z(I(Y))$ so $\overline{Y} \subset Z(I(Y))$. Pick $x \notin \overline{Y}$ so there is some $x \in U_f$ such that $U_f \cap \overline{Y} = \empty$. Therefore, $U_f \cap Y = \empty$ so $f|_Y = 0$ since if $x \in Y$ then $x \notin U_f$. Thus $f \in I(Y)$ so $x \notin Z(I(Y))$ proving that $Z(I(Y)) \subset \overline{Y}$.
\bigskip\\
Since $J \subset I(Z(J))$ is radical we see that $\sqrt{J} \subset I(Z(J))$. The key is to apply the Nullstellensatz.
\end{proof}



\end{document}
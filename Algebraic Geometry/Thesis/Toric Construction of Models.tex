\documentclass[12pt]{article}
\usepackage{import}
\import{../}{ThesisCommands}

\begin{document}

\section{Toric Construction of Models}

In this section we discuss the results of [Tim, Models] (2018) which gives a method of explicitly constructing a regular normal crossings model of a curve and explicilty describing its special fiber using the preceeding methods characterizing curves on toric surfaces. 
\begin{rmk}
In [Tim, Models], Dokchitser uses ``curve'' to refer to an integral separated \textit{geometrically connected} scheme of finite type over a field. To mitigate any confusion, we render any results quoted from his work with ``curve'' replaced by ``geometrically connected curve'' when necessary.
(OKAY THIS MAY BE WRONG!!)
\end{rmk}

\subsection{Notations and Definitions}

We work in the case of a discretely definition valued field $K$ with valuation $\nu : K^\times \onto \Z$, valuation ring $R$, uniformizer $\varpi$, and residue field $\kappa$. Given a smooth projective geometrically connected curve over $K$ our goal will be to construct a regular normal crossings model over $R$. First we need to fix some notation.

\begin{defn}
Given a Laruent polynomial $f \in K[x^{\pm 1}, y^{\pm 1}]$ recall the Newton polygon is,
\[ \Delta(f) = \Conv{ \{ (i,j) \in \Z^2 \mid a_{ij} \neq 0 \} } \subset \R^2 \]
We will assume througout that $\vol{\Delta} > 0$. 
Now we refine the Newton polygon with respect to the valuation $\nu : K^\times \onto \Z$,
\[ \Delta_\nu(f) = \mathrm{LowerConvHull} \left( \{ (i, j, v(a_{ij})) \mid (i, j) \in \Delta(f) \cap \Z^2 \} \right) \subset \R^2 \times \R \]
The projection $\pi : \Delta_\nu \to \Delta$ is a homeomorphism. Thus,
for each point $P \in \Delta$ there is a unique point $(P, \nu(P)) \in \Delta_\nu$ which defines a piecewise affine function $v : \Delta \to \R$ extending the valuation. 
\bigskip\\
The bijection $\pi : \Delta_\nu \to \Delta$ pushes the polyhedral structure on $\Delta_\nu$ onto $\Delta$. Because $\Delta_\nu$ is the lower convex hull of finitely many points in $\R^2 \times \R$ it decompses into faces of dimension $0, 1, 2$. Under the projection $\pi : \Delta_\nu \to \Delta$ the $v$-\textit{vertices} $P$ of $\Delta$ are the images of the $0$-faces, the $v$-\textit{edges} $L$ are the images of the $1$-faces, the $v$-\textit{faces} $F$ are the images of the $2$-faces. These define a polygonal partition of $\Delta$. 
\end{defn}

\begin{defn}
For each edge $L$ and face $F$ there is an associated integer $\delta_\lambda$ (with $\lambda = L$ or $F$), the \textit{denominator}, defined as smallest positive $m$ such that $\nu(P) \in \frac{1}{m} \Z$ for each $P \in \lambda \cap \Z^2$. 
\end{defn}

\begin{rmk}
We now consider how to restrict a polynomial with respect to the $\nu$-partition to form a Laurent polynomial supported on the faces and vertices. First, following the Notation of [Tim, Models] we define how to restrict the polynomial to some subset of a lattice.
\end{rmk}

\begin{defn}[Restriction]
Let $S \subset \Z^n$ be a nonempty subset of a lattice and take $\Lambda$ to be the smallest affine lattice $S \subset \Lambda \subset \Z^n$ containing $S$. Let $\Lambda$ have rank $r$ and choose an isomorphism $\phi : \Z^r \to \Lambda$. Then for a Laurent polynomial $g \in K[\bf{x}^{\pm 1}]$ we define the restriction $g|_S \in K[\bf{y}^{\pm 1}]$,
\[ g|_S = \sum_{\bf{i} \in \phi^{-1}(S)} c_{\phi(\bf{i})} \bf{y}^{\bf{i}} \in K[\bf{y}^{\pm 1}] \quad \text{for} \quad g = \sum_{\bf{i} \in \Z^n} c_{\bf{i}} \bf{x}^{\bf{i}} \]
Note that different choices of an isomorphism $\Lambda \to \Z^r$ are related by an automorphism in $\GL{r}{\Z}$ acting on the variables $\bf{y}$. 
\end{defn}

\begin{rmk}
The notational complexity of the above definition derives from making the polynomial $g|_S$ an element of a standard Laurent polynomial ring $k[\bf{y}^{\pm 1}]$. We can simplify the above notation using our previous abstract notation used for the toric constructions. Given a lattice $M$ and a Laurent polynomial $g \in K[M]$ and a subset $S \subset M$ we define the restriction,
\[ g|_S = \sum_{m \in S} c_m \chi^m \in K[\left< S \right>] \quad \text{where} \quad g = \sum_{m \in M} c_m \chi^m \]
where $\left< S \right>$ is the sublattice of $M$ generated by $S$. The above definition is recovered choosing some isomorphism $\phi : \left< S \right> \to \Z^r$ giving an isomorphism $K[\left< S \right>] \cong K[\bf{y}]$. 
\end{rmk}

\begin{defn}[Reduction]
For a Laurent polynomial $h \in K[x^{\pm 1}, y^{\pm 1}]$, there exist integers, $c, m, n \in \Z$ such that $\tilde{h}(x, y) = \varpi^c h(\varpi^m x, \varpi^n y)$ has coefficients in $R$ and $\tilde{h} \mod \varpi \in \kappa[x, y]$ has the same Newton polygon as $h$. Then we say that $\overline{h} = \tilde{h} \mod \varpi$ is \textit{reduction} of $h$. 
\end{defn}

\begin{example}
(DO THIS!!!!!!!)
\end{example}

\begin{defn}
In paticular for $\lambda$ an edge $L$ or face $F$ we define the restriction $f |_\lambda = f|_S$ for the set, $S = \{ P \in \lambda \cap \Z^2 \mid \nu(P) \in \Z \}$. Note, $S$ contains the vertices of $L$ or $F$.  
\end{defn}

\begin{rmk}
Reduction gives, for each edge $L$ and face $F$, polynomials $\overline{f|_L} \in \kappa[t]$ and $\overline{f|_F} \in \kappa[x, y]$. This gives affine curves over $\kappa$ on each edge and face which we complete in a toric compactification as follows.
\end{rmk}

\begin{defn}[Components]
We define the following schemes over $\kappa$:
\begin{enumerate}
\item $X_L = V(\overline{f_L}) \subset \Gm{\kappa}$ 
\item $X_F = V(\overline{f_F}) \subset \Gm{\kappa}^2$
\item $\overline{X}_F = X_F^\Delta$ is the completion of $X_F$ with respect to its Newton polygon $F$ i.e. the closure of the immersion $X_F \embed \Gm{\kappa}^2 \embed \Toric_F$. By Theorem \ref{baker_refinement}, $\overline{X}_F$ is connected and, in fact, the Theorem applies for any finite extension $\kappa' \supset \kappa$ showing that $\overline{X}_F$ is geometrically connected.
\end{enumerate}
\end{defn}

\begin{example}
(DO EXAMPLE!!!!!!)
\end{example}

\begin{defn}
We say that $f \in k[x^{\pm 1}, y^{\pm 1}]$ is \textit{strictly} $\Delta_\nu$-\textit{regular} if all $X_F$ and $X_L$ are smooth over $\kappa$. 
\end{defn}

\begin{rmk}
The condition that all $X_L$ are smooth implies that $f$ is nondegenerate with respect to its Newton polgon since it implies that $f$ restricted to each edge is smooth. 
\end{rmk}

\begin{defn}
A Laurent polynomial $f \in K[x^{\pm 1}, y^{\pm 1}]$ is $\Delta_\nu$-\textit{regular} if each $X_F$ is smooth and for the interior edges $L$ and edges $L \subset \partial F$ with $\delta_L \neq \delta_F$ we require $X_L$ is smooth and otherwise we require $\overline{X}_F$ is \textit{outer-regular} i.e. smooth at the points corresponding to $L$ via,
\[ \overline{X}_F(\bar{\kappa}) \setminus X_K(\bar{\kappa}) \longleftrightarrow \coprod_{L \supset \partial F} X_L(\bar{\kappa}) \]
\end{defn}

\begin{rmk}
As with toric regularity, we have defined the notion of $\Delta_\nu$-regular with respect to a given Lauret polynomial i.e. to a given affine model $C_0$ of a curve. As before, we extend this definition to an arbitrary curve in the obvious way.
\end{rmk}

\begin{defn}
A curve $C$ over $K$ is (strictly) $\Delta_\nu$-regular if $C$ is birationally equivalent to some affine $U_f \subset \Gm{K}^2$ for some (strictly) $\Delta_\nu$-regular Laruent polynomial $f \in K[x^{\pm 1}, y^{\pm 1}]$. 
\end{defn}

\begin{rmk}
We need one more notion in order to describe the model of $C$ which is a combinatorial connectivity between two adjacent faces $F_1, F_2$ sharing an edge $L$. 
\end{rmk}

\begin{defn}[Slopes]
Edges are either \textit{inner/interior} meaning they form the boundary between two $\nu$-faces $F_1$ and $F_2$ or \textit{outer/exterior} on the boundary of $\Delta$. For an edge $L$ there is a unique affine function $L^*_{(F_1)} : \Z^2 \onto \Z$ with $L^*_{(F_1)} |_L = 0$ and $L^*_{(F_1)}|_{F_1} \ge 0$. Then the edge has two corresponding integers called the \textit{slopes} Defined as follows. Choose $P_0, P_1 \in \Z^2$ with $L^*_{(F_1)}(P_0) = 0$ and $L_{(F_1)}^*(P_1) =1$. Then,
\[ s_1^L = \delta_L (\nu_1(P_1) - \nu_1(P_0)) \quad \quad s_2^L = 
\begin{cases}
\delta_L (\nu_2(P_1) - \nu_2(P_0)) & L \text{ inner}
\\
\lfloor s_1^L - 1 \rfloor & L \text{ outer}
\end{cases} \]
where $\nu_i : \Z^2 \onto \Z$ is the unique affine function which agrees with $\nu$ on $F_i$ (recall that the faces are defined such that $\nu$ is affine when restricted to each face. Given the slopes, we may consider a sequence of rational numbers $\frac{m_i}{d_i} \in \Q$ such that,
\[ s_1^L = \frac{m_0}{d_0} > \frac{m_1}{d_1} > \frac{m_2}{d_2} > \cdots > \frac{m_r}{d_r} > \frac{m_{r+1}}{d_{r+1}} \quad \text{and} \quad 
\begin{vmatrix}
m_i & m_{i+1}
\\
d_i & d_{i+1}
\end{vmatrix} = m_i d_{i+1} - m_{i+1} d_i = 1 \]
Then $r(L)$, the minimal length of this sequence, and the denominators $d_i$ of this minimal sequence, are important combinatorial parameter of the edge $L$. It turns out such a minimal sequence is unique.
\end{defn}

\begin{rmk}
The existence of such a sequence needs some consideration. Take all rational numbers in $[s_1^L, s_2^L] \cap \Q$ with denominators bounded by the largest denominator of $s_1^L$ and $s_2^L$. This is a shifted Farey series. We define the Farey series $F^n$ to be the ordered sequence of rational numbers with denominator less than or equal to $n$ put in lowest terms. Then, if $\frac{a}{b} < \frac{c}{d}$ are consecutive terms in the Farey series then $\frac{c}{d} - \frac{a}{b} = \frac{1}{bd}$. Therefore,
\[ \begin{vmatrix}
a & c 
\\
b & d 
\end{vmatrix}
= ad - bc = 1 \] 
[Tim, Remark 3.15] [HW, Ch. III, Thm. 28, Thm. 29]
Furthermore, if the sequence in $[s_1^L, s_2^L] \cap \Q$ with bounded denominators contains consecutive terms,
\[ \frac{a}{b} > \frac{a + c}{b + d} > \frac{b}{d} \]
then we must have,
\[ a(b + d) - b(a + c) = ab + ad - ab - bc = ad - bc = 1 \]
meaning that $\frac{a}{b} > \frac{b}{d}$ have the required adjacency property and thus $\frac{a + c}{b + d}$ may be removed from the sequence. We will reinterpret this as a blowdown of regular normal crossings models after we state the main theorems describing the structure of such models from the above combinatorial data.
\end{rmk}

\subsection{Main Theorems}

We describe the properties of the model $\C_\Delta$ over $R$ associated to some polygonal parition of $\Delta$ defined by a Laurent polynomial $f \in K[x^{\pm 1}, y^{\pm 1}]$. 

\begin{theorem}[DK, Thm. 3.13]
Let $C$ be a smooth projective $\Delta_\nu$-regular curve birational to $U_f$ for a $\Delta_\nu$-regular Laurent polynomial $f \in K[x^{\pm 1}, y^{\pm 1}]$. Then $\C_\Delta / R$ is a regular normal crossing model of $C$ and the special fiber $\C_\kappa$ geometrically decomposes into components as follows:
\begin{enumerate}
\item each $\nu$-face $F$ of $\Delta$ gives a smooth complete curve $\overline{X}_F \times_\kappa \kappa^{\sep}$ (CORRECT?) over $\kappa^\sep$ with multiplicity $\delta_F$ and genus $g_F = |\{ P \in F^\circ \cap \Z^2 \mid \nu(P) \in \Z \} |$
\item each $\nu$-edge $L$ with sequence $\frac{m_i}{d_i} \in \Q$ ($0 \le i \le r + 1$) gives $|X_L(\kappa^\sep)|$ chains of length $r$ of closed subschemes intersecting transversally each isomorphic to $\P^1_{\kappa^{\sep}}$ with multiplicities in $\C_\kappa$ given by $\delta_L d_1, \dots, \delta_L d_r$.
\end{enumerate}
Furthermore, the $\Gal{\kappa^\sep / \kappa}$-action on $\C_{\kappa} \times_\kappa \kappa^{\sep}$ is given by acting on each component $X_F \times_\kappa \kappa^\sep$ and permuting the $\P^1_{\kappa^\sep}$ chains via the natural action of $\Gal{\kappa^\sep / \kappa}$ on $|X_L(\kappa^\sep)|$. 
\end{theorem}

\begin{question}
In the paper, the decomposition is certianally geometric but he says the scheme is $\overline{X}_F$ rather than base changing over the algebraic closure. How should I write it? Also, is it the case that $\overline{X}_F \times_\kappa \kappa^\sep$ is smooth over $\kappa^\sep$? This is just the invariance of smoothness under base change right?
\end{question}

\begin{rmk}
The genus of $\overline{X}_F$ is exactly the number of lattice points interior to the Newton polygon defining $X_F \subset \Gm{\kappa}^2$ by Baker's theorem. Recall this Newton polygon is the the restriction of $f$ to the set $S = \{ P \in F \cap \Z^2 \mid \nu(P) \in \Z \}$ so the lattice generated by $S$ only has lattice points where at points of $\Z^2$ where $\nu(P) \in \Z$ explaining the genus formula above. 
\end{rmk}

\begin{rmk}
(ASK JOHAN ABOUT THIS???)
What about $f = \varpi x^2 + y^2  + 1$ which has $\nu(i,j) = \tfrac{1}{2} i$ and thus the single interior point $P$ has $\nu(P) = \tfrac{1}{2}$. However, $g = \varpi^2 x^2 + y^2 + 1$ has $\nu(P) = 1$.
\end{rmk}

(GIVE THIS AS EXAMPLE)

\begin{theorem}[DK, Thm. 3.14]
Let $f \in K[x^{\pm 1}, y^{\pm 1}]$ be any Laurent polynomial defining a 1-dimensional scheme $C_0 = U_f \subset \Gm{K}^2$. Then $\C_\Delta / R$ is a proper flat model of the toric completion $C = C_0^\Delta$ with respect to the Newton polygon $\Delta = \Delta(f)$. The special fiber $\C_\kappa$ is a union of closed subschemes $\overline{X}_F$ indexed by $\nu$-faces $F$ and chains $X_L \times_\kappa \Gamma_L$ where $\Gamma_L$ is a union of $\P^1_{\kappa}$ intersecting transversally as follows:
\begin{enumerate}
\item for each $\nu$-edge $F$: the scheme $\overline{X}_F$ has multiplicity $\delta_F$ and, via Theorem \ref{baker_refinement} are geometrically connected and have arithmetic genus $ g_F = |\{ P \in F^\circ \cap \Z^2 \mid \nu(P) \in \Z \} | $.
\item for each $\nu$-edge $L$ choose a sequence $\frac{m_i}{d_i} \in \Q$ ($0 \le i \le r+1$) then let $\Gamma_L = \Gamma^1_L \cup \cdots \cup \Gamma^r_L$ with each $\Gamma^i_L$ isomorphic to $\P^1_k$ embedded with multiplicity $\delta_L d_i$ and meeting transversally where we identify $0 \in \Gamma^i_L$ with $\infty \in \Gamma^{i+1}_L$. If $r = 0$ then let $\Gamma_L = \Spec{k}$.
\item The subscheme $X_L \times \{ 0 \} \subset X_L \times \Gamma^1_L$ is identified with $\overline{X}_{F_1} \setminus X_{F_1}$ for the $\nu$-face $F_1$ boardering $L$ and, when $L$ is inner, likewise $X_L \times \{ \infty \} \subset X_L \times \Gamma^r_L$ is identified with $\overline{X}_{F_2} \setminus X_{F_2}$ for the other $\nu$-face $F_2$ boardering $L$. These intersections are transversal and, in fact, as a scheme the intersection is $V(\overline{f_L}^{\delta_L}) \subset X_L \subset \overline{X}_F$. 
\end{enumerate}
Furthermore, the model $\C_\Delta$ is geometricall regular ar,
\begin{enumerate}
\item the smooth locus of $X_F$, for each $\nu$-face $L$
\item the smooth locus of $X_L \times \Gamma_L$, for each $\nu$-edge $L$
\item the smooth points of $\overline{X}_F \setminus X_F$ corresponding to $L$ when $L \subset \partial F$ is an outer edge with $\delta_L = \delta_F$ and $r = 0$. 
\end{enumerate}
Furthermore, if $C_0$ is $\Delta_\nu$-regular then $C = C_0^\Delta$ is smooth and thus the unique smooth proper curve birational to $C_0$ and $\C_\Delta / R$ is a regular normal corossings model of $C$. 
\end{theorem}

\section{Relationships Between Toric Notions of Regularity}

We have discussed a number of regularity conditions on curves originating from their compatiblity in some sense with a certain set of toric embeddings. The utility of these conditions is the ability to verify them from the equation defining some affine model of the curve in $\Gm{k}^2$. Although these notions are clearly related, we here show that they are, indeed, inequivalent. In this situation, we have a discrete valued field $K$ with valuation ring $R$ and residue field $\kappa$. On the special fiber, we will distinguish between the arithmetic ($\kappa$ non-algebraically closed) and geometric ($\kappa$ arithmetically closed) situations. The main result of this section is as follows.

\begin{prop}
Let $C$ be a smooth curve over $K$. Then we have the following implications,
\begin{center}
\begin{tikzcd}
C \text{ is strict } \Delta_\nu \text{-regular} \arrow[Rightarrow, r] \arrow[Rightarrow, d] & C \text{ is nondegenerate } \arrow[Rightarrow, d] 
\\
C \text{ is } \Delta_\nu \text{-regular} \arrow[Rightarrow, r] & C \text{ is weakly nondegenerate}
\end{tikzcd}
\end{center}
Furthermore, no implication is reversible. 
\end{prop}
\noindent\\
The fact that $\Delta$-nondegeneracy implies weak $\Delta$-nondegeneracy is simply an application of Baker's theorem (recall that this notion was created, by design, as a weaker form of $\Delta$-nondegeneracy, hence the name). Likewise, strict $\Delta_\nu$-regularity implying $\Delta_\nu$-regularity is also a consequence of Baker's theorem since the outer-regular condition introduced in the definition of $\Delta_\nu$-regularity is satisfied when each $X_F$ is smooth and $X_L$ is smooth since these imply that $\overline{X}_F$ is smooth as well via Baker's theorem. 
\par
That $\Delta_\nu$-regularity implies weak nondegeneracy follows from the main theorem

\section{Stuff}

It does not suffice to take \textit{any} affine open as the following example shows, we must indeed take a sufficiently small open so the notion of birationality here is actually necessary.

\begin{prop}
There exists a smooth affine curve $C$ over $k$ with no closed immersion $C \embed \Gm{k}^2$. 
\end{prop}

\begin{proof}
We use the obstruction that any curve embedded in $\Gm{k}^2$ must have a trivial canonical bundle (Lemma \ref{plane_curve_trivial_canonical}). Therefore, it sufficees to produce a smooth affine curve with a nontrivial canonical bundle. The affiness is easy to arrange since for any smooth complete curve $\overline{C}$ then removing a single point leaves an affine curve (Lemma \ref{affine_remove_point}). Setting $C = \overline{C} \setminus \{ P \}$, the inclusion $j : C \to \overline{C}$ induces an exact sequence,
\begin{center}
\begin{tikzcd}
\Z \arrow[r] & \Cl{\overline{C}} \arrow[r, "j^*"] & \Cl{C} \arrow[r] & 0
\end{tikzcd}
\end{center}
where the first map is $1 \mapsto [P]$. Therefore, a divisor class $D$ is sent to zero under $f^* D \sim 0$ iff $D \sim \deg{D} \cdot [P]$. We need to show that the canonical divisor does not vanish $K_C \not \sim 0$ and thus that $K_{\overline{C}} \not\sim (2g - 2) \cdot [P]$ since $j : C \embed \overline{C}$ is \etale so $\Omega_{C/k} = f^* \Omega_{\overline{C}/k}$. Therefore, it suffices to produce a curve $\overline{C}$ with a point $P \in \overline{C}$ such that $K_{\overline{C}} \not\sim (2g - 2) \cdot [P]$. Note that because the $(2g - 2)$-torsion in the Picard group for $g \ge 2$ is a finite group, all but finitely many choices for $P$ on any smooth complete curve of genus $g \ge 2$ will work.
\bigskip\\
Specificially, take $\overline{C} = \Proj{k[X, Y, Z]/(X^4 - X^2 Z^2 + (Y - Z)^4 - Z^4)}$ which is easily verfified to be smooth in characteristic $p \neq 2,5$ and has genus $g = 3$ since it is a plane curve of degree $4$ in $X = \P^2_k$. Then choose $P = [0 : 0 : 1]$. Notice that under $\iota : \overline{C} \embed \P^2_k$ we have $\Omega_{\overline{C}/k} \cong \iota^* \struct{X}(1)$ by the adjunction formula. We need to check that $(2g - 2) \cdot [P]$ is not one of the effective divisors linearly equivalent to $K_{\overline{C}}$. Such divisors are parametrized by sections $H^0(\overline{C}, \Omega_{\overline{C}/k}) = H^0(X, \iota_* \iota^* \struct{X}(1))$. By the projection formula $\iota_* \iota^* \struct{X}(1) = \iota_* \struct{\overline{C}} \otimes_{\struct{X}} \struct{X}(1)$. To compute the sections of this coherent $\struct{X}$-module we apply the exact sequence of the Cartier divisor $\overline{C}$ twisted by $\struct{X}(1)$,
\begin{center}
\begin{tikzcd}
0 \arrow[r] & \struct{X}(-3) \arrow[r] & \struct{X}(1) \arrow[r] & \iota_* \struct{\overline{C}} \otimes_{\struct{X}} \struct{X}(1) \arrow[r] & 0
\end{tikzcd}
\end{center}
and applying cohomology,
\begin{center}
\begin{tikzcd}
H^0(X, \struct{X}(-3)) \arrow[r] & H^0(X, \struct{X}(1)) \arrow[r] & H^0(\overline{C}, \iota^* \struct{X}(1)) \arrow[r] &  H^1(X, \struct{X}(-3)) 
\end{tikzcd}
\end{center}
but $H^q(X, \struct{X}(-3)) = 0$ for $q \le 1$ so the map $H^0(X, \struct{X}(1)) \onto H^0(\overline{C}, \iota^* \struct{X}(1))$ given by pulling back sections, $s \mapsto \iota^* s$, is bijective. In particular, since any section $s \in H^0(\overline{C}, \Omega_{\overline{C}/k})$ is the pullback of some hyperplane equation $h \in H^0(X, \struct{X}(1))$, the divisor of zeros associated to $s$ is the hyperplane section $\iota^{-1}(H) = \overline{C} \cap H$ with the hyperplane $H = V(h)$. However, I claim that any line passing through $P$ intersects $\overline{C}$ in more than one point. To see this, consider the tangent line $L$ to $\overline{C}$ at $P$ is $\P^1_k \to X$ given by $[T_0 : T_1] \mapsto [T_0 : 0 : T_1]$ but $L \cap \overline{C} = \{[0 : 0 : 1], [1 : 0 : 1], [-1 : 0 : 1] \}$. Therefore, there cannot be any line passing through only $P$ meaning that $\{ P \}$ cannot be the support of any effective divisor in canonical linear system $H^0(\overline{C}, \Omega_{\overline{C}/k})$. Thus, $K_{\overline{C}} \not\sim (2g - 2) \cdot [P]$ so $C = \overline{C} \setminus \{ P \}$ has nontrivial canonical bundle yet is affine providing the required example.
\end{proof}

\begin{rmk}
Notice that although $C$ is not an affine plane curve (in the sense of having a closed immersion $C \embed D(q) \subset \A^2_k$ to some principal affine open) there is an immersion $C \embed \P^2_k$ since $\overline{C} \embed \P^2_k$ is a complete plane curve.
\end{rmk}
\noindent
We conclude by providing proofs of the required lemmas.

\begin{lemma} \label{plane_curve_trivial_canonical}
Let $C \embed D(q) \subset \A^2_k$ be a smooth curve embedded in some standard open $D(q)$ in the affine plane. Then the canoncial bundle $\Omega^1_{C/k} \cong \struct{C}$ is trivial.
\end{lemma}

\begin{proof}
Let $A = k[x, y, q^{-1}]$ so $D(q) = \Spec{A}$. Note that $C = \Spec{R}$ with $R = A/I$ where $I = \ker{(A \to \Gamma(C, \struct{C}))}$. Furthermore, $I = (f)$ since $\dim{C} = 1$ thus $\height{I} = 1$ but $C$ is irreducible and thus $I$ is prime and since $A$ is a UFD, $I = (f)$ since each height one prime is principal. Furthermore, $C$ is smooth so $(f, f_x, f_y) = A$ where $f_x, f_y$ are the partial derivatives of $f$ with respect to $x$ and $y$. Therefore, we can choose $g,h \in A$ such that $g f_x + h f_y = 1$ in $R$. Now, consider the $R$-module of differentials, $\Omega_{R/k} = (R \d{x} \oplus R \d{y}) /(f_x \d{x} + f_y \d{y})$. 
\bigskip\\
Consider the map $\phi : R \to \Omega_{R/l}$ sending $1 \mapsto h \d{x} - g \d{y}$. Note that,
\[ \d{x} = g f_x \d{x} + h f_y \d{x} = h f_y \d{x} - g f_y \d{y} \implies f_y \mapsto \d{x} \] 
\[ d{y} = g f_x \d{y} + h f_y \d{y} = g f_x \d{y} - h f_x \d{x} \implies -f_x \mapsto \d{y} \]
so $\phi$ is surjective. Furthermore, suppose that $\phi(a) = 0$ then $\phi(f_x a) = \phi(f_y a) = 0$ so in $R \d{x} \oplus R \d{y}$ we have $a \d{x}, a \d{y} \in (f_x \d{x} + f_y \d{y})$  meaning $a \d{x} = c_1(f_x \d{x} + f_y \d{y})$ and $a \d{y} = c_2(f_x \d{x} + f_y \d{y})$ giving $c_1 f_y = 0$ and $c_2 f_x = 0$ and $c_1 f_x = c_2 f_y = a$ since $R \d{x} \oplus R \d{y}$ is free. But then \[ a = g f_x a + h f_y a = g f_x c_2 f_y + h f_y c_1 f_x = 0 \]
since $c_2 f_x = c_1 f_y = 0$ so $\phi$ is injective. Thus $\Omega_{R/k} \cong R$ and sheafifying gives, $\Omega_{C/k} \cong \struct{C}$. 
\end{proof}

\begin{lemma} \label{affine_remove_point}
Let $C$ be a smooth proper curve and $P \in C$ a point. Then $C \setminus \{ P \}$ is affine. 
\end{lemma}

\begin{proof}
The divisor $D = \nu [P]$ is very ample for sufficiently large $\nu$ (in fact for $\nu \ge 2 g + 1$) [Har, IV.3.2]. Therefore, the linear system $|\nu [P]|$ defines a closed ($C$ is proper) immersion $j : C \embed \P^{\nu - 1}_k$ such that $\struct{C}(D) = j^* \struct{\P^{\nu-1}_k}(1)$ with the hyperplane sections pulling back to a basis of $H^0(C, \struct{C}(D))$. Since $D$ is effective, there is some section $s \in H^0(C, \struct{C}(D))$ with $V(s) = \{ P \}$ and thus some hyperplane section $h \in H^0(\P^{\nu - 1}_k, \struct{\P^{\nu-1}_k}(1))$ with $s = j^* h$ and thus $\{ P \} = j^{-1}(H \cap j(C))$ where $H = V(h) \subset \P^{\nu - 1}_k$. Finally, $C  \setminus \{ P \} = j^{-1}(\P^{\nu - 1}_k \setminus H)$ which is affine since $j$ is a closed immersion and thus affine and the complement of a hyperplane in projective space is a standard open.
\end{proof}

\section{Example}

In order to extend our argument to heigher genus curves, we need to apply Riemann-Roch. However, since this example only works in the arithmetic setting we require a slight modification to the standard statement of Riemann-Roch found in Hartshorne. To ensure there is no confusion, we will provide a proof here.

\begin{theorem}[Riemann-Roch]
Let $X$ be a smooth proper curve over $k$ with $g = \dim_k H^0(X, \omega_X)$ its genus. Then for any line bundle $\L$,
\[ \chi(\L) = \dim_k H^0(X, \L) - \dim_k H^0(X, \omega_X \otimes_{\struct{X}} \L^\vee) = \deg{\L} + 1 - g \]
Where $\deg{\L}$ is defined in the arithmetic case as follows. Choose a nonzero $s \in H^0(X, \L)$ and a local trivialization $\{ (U_i, s_i) \}$ with $s_i \in \L(U_i)$ such that $\struct{U_i} \xrightarrow{s_i} \L|_{U_i}$ is an isomorphism. Then define,
\[ \deg{\L} = \sum_{P \in X} [\kappa(P) : k] \: \ord_{P}(s/s_i) \] 
for some $i$ with $P \in U_i$. 
\end{theorem}

\begin{proof}
First, note that by Serre duality, $H^1(X, \L) \cong H^0(X, \omega_X \otimes_{\struct{X}} \L^\vee)^\vee$ so,
\[ \chi(\L) = \dim_k H^0(X, \L) - \dim_k H^1(X, \L) = \dim_k H^0(X, \L) - \dim_k H^0(X, \omega_X \otimes_{\struct{X}} \L^\vee) \]
Since $X$ is smooth, every line bundle $\L$ is $\struct{X}(D)$ for some divisor $D$. However,
\end{proof}

\end{document}
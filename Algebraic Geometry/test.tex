\documentclass[12pt]{article}
\usepackage{../General/macros}

\begin{document}



\thanks{I thank Ravi Vakil, Mihnea Popa, and Nathan Chen for many enlightening conversations and comments on this manuscript. The author would like to thank the Department of Mathematics at Harvard University for its hospitality during which some of this research was carried out. During the preparation of this article, the author was partially supported by an NSF Graduate Research Fellowship under grant DMS-2103099 DGE-2146755.}

%\subjclass[2020]{Primary: 14J10. Secondary: 14D06, 14K05.}

\title{Nowhere vanishing $1$-forms on varieties admitting a good minimal model.}
\author{Benjamin Church}


\begin{abstract}
We prove several conjectures relating the existence of nonvanishing $1$-forms to smooth morphisms over abelian varieties assuming the existence of good minimal models. The proof involves a decomposition result for families of Calabi-Yau varieties equipped with a surjective map to an abelian scheme. When $X$ is uniruled and its MRC base admits a good minimal model we also achieve a structure theorem for those varities admitting nowhere vanishing $1$-forms.
\end{abstract}

\maketitle
\tableofcontents

\section{Introduction}

$\Pic_A$ \Kollar $\Pic(X)$ $\Hom (A)$

A theorem of Popa and Schnell \cite{PS14} answering the conjectures of Luo and Zhang in \cite{LZ05} shows that if a smooth projective variety admits a $1$-form without zeros it cannot be of general type. This is a constraint on the geometry but we expect far more stringent constraints on those varieties of intermediate Kodaira dimension that actually do carry nowhere vanishing $1$-forms. For example, Hao and Schreieder exactly classified all $3$-folds with this property \cite{HS21(1)} and found they are always blowups of isotrivial smooth fibrations over an abelian variety. 
\par
Inspired by the case of $3$-folds, our guiding principle is that $1$-forms without zeros should be explained geometrically by the existence of a smooth map to an abelian variety. In the non-minimal case, this is not true on the nose (see Example~\ref{example:scheireder} {\color{red} PUT EXAMPLE SOMEWHERE}), and one must first pass to a birational model in order to obtain a \textit{smooth} map to an abelian variety. This principle was explored in \cite{Hao23} and \cite{CCH23} where results were obtained for large Kodiara dimension. We generalize those results (see \ref{section:iitaka}) and remove the assumption on the Kodaira dimension. We apply those results to prove several conjectures in the spirit of our main principle assuming the main conjectures of MMP hold. In particular, this resolves the conjectures for $4$-folds using the results of Fujino \cite{Fuj10} on irregular $4$-folds.
\par 
We verify conjectures of Hao and Schreieder \cite[Conjecture 1.7]{HS21(1)} (in the case $\kappa(X) \ge 0$) and \cite[Conjecture 1.8]{HS21(1)} assuming the existence of a good minimal model:
\end{document}
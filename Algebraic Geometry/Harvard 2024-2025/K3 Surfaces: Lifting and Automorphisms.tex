\documentclass[12pt]{article}
\usepackage{import}
\import{../}{AlgGeoCommands}

\newcommand{\dbar}{\bar{\partial}}
\newcommand{\HH}{\mathbb{H}}
\renewcommand{\gr}{\mathrm{gr}}
\newcommand{\R}{\mathrm{R}}

\renewcommand{\H}{\mathcal{H}}
\newcommand{\LL}{\mathbb{L}}

\newcommand{\Def}{\mathrm{Def}}
\newcommand{\Spf}[1]{\mathrm{Spf}\left( #1 \right)}
\newcommand{\Art}{\mathrm{Art}}
\newcommand{\Set}{\mathrm{Set}}
\newcommand{\At}{\mathrm{At}}
\newcommand{\ob}{\mathrm{ob}}

\begin{document}

\section{Deligne's Paper on Lifting}

Here let $X_0$ be a K3 surface over $k$ an algebraically closed field of cahracteristic $p > 0$. Let $W(k)$ be the ring of Witt vectors.

\begin{prop}
The sepectral sequence
\[ E_1^{i,j} = H^j(X_0, \Omega^i_{X_0/k}) \implies H_{\dR}^{i+j}(X_0/k) \]
degenerates at $E_1$ and the hodge numbers are as expected. Furthermore
\begin{enumerate}
\item $H^i(X_0, \T_{X_0}) = 0$ for $i = 0,2$ and $h^1(X_0, \T_{X_0}) = 20$
\item the crystaline cohomology $W$-modules are free of rank $1,0,22,0,1$ for $i = 0,1,2,3,4$.
\end{enumerate}
\end{prop}

\begin{cor}
The formal versal deformation space of $X_0$ is a $W$-algebra artian local with residue field $k$ is universal and is smooth of dimension $20$ meaning
\[ \Def_{X_0} = S := \Spf{W \dbrac{t_1, \dots, t_{20}}} \]
\end{cor}

From now on, let $\X \to S$ be the universal deformation of $X_0$.

\subsection{Line Bundles}

Let $L_0$ be an invertible sheaf on $X_0$. Write $\ul{\Def}(X_0, L_0)$ for the functor
\[ \Art_k \to \Set \]
which takes $A$ to deformations of the pair $(X_0, L_0)$ over $A$. There is a forgetful map
\[ \ul{\Def}(X_0, L_0) \to \ul{\Def}(X_0) \]

\begin{prop}
$\ul{\Def}(X_0, L_0)$ is pro-representable and the map
\[ \ul{\Def}(X_0, L_0) \to \ul{\Def}(X_0) \]
is a closed imersion defined by one equation.
\end{prop} 

This means there is a closed formal subscheme
\[ \Sigma(L_0) \subset S \]
such that $L_0$ extends over $\ul{X} \times_S \Sigma(L_0)$ and this extension is unique.

\begin{proof}
We verify that $F = \ul{\Def}(X_0, L_0)$ satisfies Schlessinger's condition for the existence of a hull. 
\begin{enumerate}
\item[(H1)] let $A'_1 \onto A$ be a small extension of Artin rings and $A'_2 \to A$ any map of Artin rings. Consider
\[ F(A'_1 \times_A A'_2) \to F(A_1') \times_{F(A)} F(A_2') \]
Notice that
\begin{center}
\begin{tikzcd}
\Spec{A} \arrow[r] \arrow[d] & \Spec{A_1'} \arrow[d]
\\
\Spec{A_2'} \arrow[r] &  \Spec{A_1' \times_A A_2'}
\end{tikzcd}
\end{center} 
is a pushout diagram of schemes. By affinianity we can always glue deformations of schemes to form the pushout in schemes. 

Given a line bundles on the reduction to $A_i'$ that are isomorphic over $A$ we need to glue them. This holds by representability of the Picard scheme but we can argue directly. In fact it is true for any coherent sheaves. We show there is a unique way to do it so we can argue locally. Given a fiber product of rings $A \times_R B$ and modules $M, N$ over $A,B$ with an isomorphism $\psi : M \ot_A R \iso N \ot_B R$ we can build a glued module
\[ M \times_\psi N = \{ (m,n) \in M \times N \mid \psi(m \ot 1) = n \ot 1 \} \]
it is clear how to endow this with an $A \times_R B$-module structure. The claim is that this is unique amoung modules equipped reductions to $M, N$. Indeed, given any other $G$ we construct an isomorphism as follows. The reduction maps give
\[ G \to M \quad G \to N \]
which are compatible with ring projections. Hence we get a map
\[ G \to M \times N \]
which is equivariant for the ring map $A \times_R B \to A \times B$. The image must land inside $M \times_{\psi} N$ because $G \ot A \iso M$ and $G \ot B \iso N$ so reducing mod $R$ the isomorphisms are required to be compatible with $\psi$. The map $G \to M \times_{\psi} N$ is then an isomorphism because this is true after tensoring with $A$ or $B$ which are scheme-theoretically dense.

\item[(H1)] We have shown that the underlying deformation category satisfies (RS) so the functor of isomorphism classes also satsifies (H2)

\item[(H3)] finiteness of the tangent space is clear by usual deformation theory and properness
\end{enumerate}
Therefore, we get a formal scheem $S' = \Spf{R'}$ with $R'$ a local $W$-algebra that is noetherian and complete with resoude field $k$ and there is a deformation $(X', L')$ over $S'$ hence there is a map
\[ \Hom{}{R'}{A} \to \ul{\Def}(X_0, L_0)(A) \]
which is surjective and bijective over $A = k[\epsilon]$. Let $R$ be the ring of $S$. Since $R$ pro-represents $\ul{\Def}(X_0)$ there is a composition
\[ \Hom{}{R'}{A} \to \ul{\Def}(X_0, L_0)(A) \to \\ul{\Def}(X_0)(A) = \Hom{}{R}{A} \]
of natural transformations there is a map $u : R \to R'$ of local $W$-algebras. To prove closedness, it suffices to show that $u$ is surjective since then the composition and hence the first map are injective and hence the first map is bijective. 
\par 
According to a well-known lemma [15, 1.1], it suffices to show that if $\m$ (resp.\ $\m'$) is the maximal ideal of $R$ (resp.\ $R'$) the map
\[ \m / (p R + \m^2) \to \m' / (p R' + \m'^2) \]
is surjective or equivalently (using Schlessinger H2) that
\[ \ul{\Def}(X_0, L_0)(k[\epsilon]) \to \ul{\Def}(X_0)(k[\epsilon]) \]
is injective. The Atiyah extension
\[ 0 \to \struct{X_0} \to \At(L_0) \to \T_{X_0} \to 0 \]
controlls the deformation theory in the sense that $H^i(X_0, \At(L_0))$ forms an automorphism-tangent-obstruction theory for $\ul{\Def}(X_0, L_0)$. The long exact sequence gives,
\[ H^1(X_0, \struct{X_0}) \to H^1(X_0, \At(L_0)) \to H^1(X_0, \T_{X_0}) \]
and $H^1(X_0, \struct{X_0}) = 0$ so the map on tangent spaces is injective. 
\par 
It remains to show that the closed immersion $S' \to S$ is defined by a single equation i.e.\ the ideal $\ker{u} = I$ is monogenic. To do this, consider $S'' = \Spf{R/\m I}$ which is a thickening of $S'$ inside $S$ by the square-zero ideal $I / \m I$. The obstruction to extend the sheaf $L'$ defined over $\X \times_S S''$ is an element
\[ \ob \in H^2(X_0, I / \m I) = H^2(X_0, \struct{X_0}) \ot I/\m I \]
which can be regarded as an element of $I / \m I$ given a choice of generator of $H^2(X_0, \struct{X_0})$.  Let $\Sigma = \Spf{(R / (\m I + (f))}$ where $f \in I$ lifts $a$. We then have
\[ S' \subset \Sigma \subset S'' \subset S \]
and by construction (and functoriality of the aobstruction) $L'$ lifts to $\X \times_S \Sigma$. By the universal property of $S'$ this means $S' = \Sigma$ menaing $\m I + (f) = I$. Then by Nakayama's lemma, $f$ generates $I$. 
\end{proof}

\begin{theorem}[A]
Let $L_0$ be a nontrivial sheaf on $X_0$. Then the formal scheme $\Sigma(L_0)$ pro-represents $\ul{\Def}(X_0, L_0)$ and is flat over $W$ of relative dimension $19$.
\end{theorem}

In other words, if $f$ is the equation defining $\Sigma(L_0)$ over $S$ then $p$ does not divide $f$. This still means that $\Sigma(L_0)$ is not contained in the reduction $S_0$ of $S$ mod $p$ hence $L_0$ does not lift to $\X \times_S S_0$.
\par 
We prove this is section $2$. We finish the section with some consequences of this theorem.

\begin{cor}
Let $L_0$ be a nontrivial invertible sheaf on $X_0$. There exists a trait $T$ finite over $W$, a deformation of $X_0$ to a formal scheme $X$ flat over $T$, and an extension of $L_0$ to an invertible sheaf $L$ on $X$.
\end{cor}


It suffices to show there is a $W$-morphism $T \to \Sigma(L_0)$ with $T$ a trait finite over $W$. Since $p$ is not a zero-divisor in $R'$, there exists elements $x_1, \dots, x_n \in \m \subset R'$ forming along with $p$ a system of parameters. The quotient $B = R'/(x_1, \dots, x_n)$ is quasi-finite over $W$, hence finite over $W$. There exists a local $W$-homomorphism $B \to C$ to a complete DVR finite over $W$ so the composition $R' \to B \to C$ answers the question.
\par 
Applying Grothendieck's algebraization theorem (EGA III, 5.4.5), we deduce from 1.7 the following theorem:

\begin{cor}
Let $L_0$ be an ample invertible sheaf on $X_0$. There exists a trait $T$ finite over $W$ and a deformation of $X_0$ to a proper smooth scheme $X \to T$ and an extension $L$ of $L_0$ over $X$.
\end{cor}

\begin{rmk}
We do not know whether any K3 surface over $k$ lifts to a proper smooth scheme over $W$. Ogus [13] shows that (a) every K3 surface over $k$ lifts over $W$ except perhaps the superspecial case, non elliptic, which actually should not exist if we accept Artin's conjecture [1]. (b) if $p > 2$ every K3 surface over $k$ lifts over $W[\sqrt{p}]$ therefore only the specal case of 1.8 for $p = 2$ and $X_0$ superspecial is not covered by other results. Let's point out that the other part of Ogus' article has interesting additions on the structure of $\Sigma(L_0)$, cf. also the following exposition in the ordinary case.
\end{rmk}

\begin{cor}
If $k$ is the algebraic closure of a finite field then on which $X_0$ is defined, the Frobenius has a semisimple action on $H^2(X_0, \Q_{\ell})$ for $\ell \neq p$.
\end{cor}

\subsection{de Rham Cohomology and Theorem A}

Use the same notation as the previous section. Let $X_0$ be a K3 surface over $k$, and $\X / S$ the universal $W$-deformation. The reader familar with de Rham cohomology is invited to skip the first section which develops standard material about the Gauss-Manin connection, the Hodge filtration, the action of Frobenius, and the Chern classes of invertible sheaves.
\par 
Let $\Omega_{\X/S}^\bullet$ be the de Rham complex of the formal scheme which is a relative scheme (by definition, it is the projective limit of the de Rham complex for infinitesimal neighbrohoods of $\Spec{k}$).

\begin{prop}
The spectral sequence
\[ E_1^{i,j} = H^j(\X, \Omega_{\X/S}^i) \implies H^\bullet_{\dR}(\X / S) \]
degenerates at $E_1$ and the Hodge cohomology groups are free of finite ype and the canonical maps
\[ H^j(\X, \Omega^i_{\X/S}) \ot k \to H^j(X_0, \Omega^i_{X_0/k}) \]
are isomorphisms. The $\struct{S}$-modules $H^i_{\dR}(\X/S)$ are free of finite type, and the canonical maps
\[ H^i_{\dR}(\X/S) \ot k \to H^i_{\dR}(X_0/k) \]
are isomorphims. The cup-product
\[ \smile : H^2_{\dR}(\X/S) \ot H^2_{\dR}(\X/S) \to H^4_{\dR}(\X/S) \]
is a perfect pairing. 
\end{prop}

\begin{proof}
Indeed, since the Hodge table is ``interlaced with zeros'' cohomology and base change applies to show these results. The last assertion follows from flatness and Poincare duality over $k$.
\end{proof}

\subsubsection{2.3}

Let $\Omega^\bullet_{S/W}$ be the de Rham complex of ``formal'' differentials of $S/W$ menaing
\[ \Omega^i_{S/W} = \bigwedge^i \Omega_{X/S} \quad \Omega_{S/W} = \ilim_n \Omega_{S_n/W_n} \]
where $\Omega_{S_n/W_n}$ is the module of compelte differentials of $S_n / W_n$ the mod $p^n$-reduction of $W \dbrac{t_1, \dots, t_n}$. This is free over $\struct{S}$ with basis $\d{t_1}, \dots, \d{t_{20}}$. Then $H^i_{\dR}(\X/S)$ is equipped with a canonical integrable connection, the Gauss-Manin connection
\[ \nabla : H^i_{\dR}(\X/S) \to H^i_{\dR}(\X/S) \ot \Omega^1_{S/W} \]
We can use the fact that $H^i_{\dR}(\X/S)$ is the value over $S$ of a cystal in $\struct{}$-modules on the crystalline site of $S_0/W$
\[ H^i_{\dR}(\X/S) = R^i (f_0)_{\text{crys}*} (\struct{\X_0/W})(S) \]


\section{Helene's Paper}

\renewcommand{\Aut}{\mathrm{Aut}}

Let $X$ be a K3 surface over an algebraically closed field $k$ of characteristic $p > 0$. We assume $p > 3$. Let $S$ be the formal deformation space of $X$ and $\Spec{R} \to S$ a morphism from a DVR such that $\Spec{R} \to \Spec{W}$ is dominant. Let $X_R \to \Spec{R}$ be a proper model of $X$. Let $K = \Frac{R}$.
\par 
There is a specialization homomorphism
\[ \iota : \Aut^e(X_{\ol{K}}/\ol{K}) \to \Aut(X/k) \]
where 
\[ \Aut^e(X_{\ol{K}}/\ol{K}) \subset \Aut(X_{\ol{K}}/\ol{K}) \]
is the subgroup of automorphisms that lift to some model. We say that $f \in \Aut(X/k)$ is not geometrically liftable to characteristic $0$ if it is not in the image of $\iota$. 

\section{Some Ideas}

The deformation theory of pairs $(X, \phi)$ is controlled by the complex
\[ C^\bullet = [\T_X \xrightarrow{\d{\phi} - \id} \T_X] \]
placed in degrees $0,1$ in the sense that $\HH^i(C^\bullet)$ is a automorphisms-tangent-obstruction theory. Indeed, by definition it fits into an exact triangle
\[ C \to \T_X \xrightarrow{\d{\phi} - \id} \T_X \to +1 \]
since $H^i(X, \T_X) = 0$ for $i \neq 1$ we get an exact sequence
\[ 0 \to H^1(C^\bullet) \to H^1(X, \T_X) \xrightarrow{\d{\phi} - \id} H^1(X, \T_X) \to H^2(C^\bullet) \to 0 \]
Hence the moduli space of deformations $(X, \phi)$ has virtual dimension zero. It looks like it also has a perfect obstruction theory. 

\section{Automorphisms of K3 surfaces and lifting}

\newcommand{\inner}[2]{\left< #1, #2 \right>}

Let $X / k$ be a K3 surface over a field $k$. Write $\phi : X \to X$ for an automorphism defined over $k$. Let $G \acts X$ be a group of automorphisms defined over $k$ acting on $X$.

\subsection{The representation on $H^1(X, \T_X)$}

\begin{lemma}
$G \acts H^1(X, \T_X)$ via the representation $H^1(X, \Omega_X)^{\vee} \ot H^0(X, \omega_X)^\vee$.
\end{lemma}

\begin{proof}
Consider the pairing
\[ \Omega_X \ot \T_X \xrightarrow{\inner{-}{-}} \struct{X} \]
this induces via the Yoneda pairing a morphism
\[ H^1(X, \Omega_X) \ot H^1(X, \T_X) \to H^2(X, \struct{X}) \iso H^0(X, \omega_X)^\vee \]
where the last map is Serre duality. These are equivariant maps. I claim that this pairing is perfect meaning that the map
\[ H^1(X, \T_X) \to H^1(X, \Omega_X)^{\vee} \ot H^0(X, \omega_X)^{\vee} \]
is an isomorphism. This would exhibit a natural $G$-equivariant isomorphism as required. Choose a generator $\omega \in H^0(X, \omega_X)$ this gives a commutative diagram
\begin{center}
\begin{tikzcd}
\Omega_X \ot \T_X \arrow[r] \arrow[d] & \struct{X} \arrow[d, "\omega"]
\\
\Omega_X \ot \Omega_X \arrow[r, "\wedge"] & \omega_X
\end{tikzcd}
\end{center}
where the downward map is given by
\[ \eta \ot \xi \mapsto \eta \ot \omega(\xi,-) \]
since the downward maps are isomorphisms this gives a diagram of pairings
\begin{center}
\begin{tikzcd}
H^1(\Omega_X) \ot H^1(\T_X) \arrow[r] \arrow[d] & H^2(\struct{X}) \arrow[d, "\omega"]
\\
H^1(\Omega_X) \ot H^1(\Omega_X) \arrow[r, "\wedge"] & H^2(\omega_X)
\end{tikzcd}
\end{center}
and the bottom is perfect by Poincar\'{e} duality [D, 2.3.13].
\end{proof}

We need one other ingredient:

\begin{lemma}
Let $X$ be Shioda supersingular (meaning $\rho = b_2$) and with Artin invariant $1$. Then the map
\[ c_1^{\text{Hodge}} : \NS(X) \ot k \to H^1(X, \Omega_X^1) \]
is a surjective map of $k[G]$-modules.
\end{lemma}

\begin{proof}
Theorem 11.10 of van der Geer and Katsura.
\end{proof}

\subsection{Computing representations}

\newcommand{\PGU}{\mathrm{PGU}}
\newcommand{\Nm}{\mathrm{Nm}}

These facts give us enough information to completely understand the representation $G \acts H^1(X, \T_X)$ using the information computed by [KS]. Let $X = X(3)$ be the Fermat quartic over $\FF_3$ and set $k = \FF_9$ over which the automorphism group and Neron-Severi groups are defined. We use the KS description of the automorphism group:
\[ \Aut(X) := \left< \tau_1, \tau_2, \Aut(X,h) \right> \]
where $\tau_i$ are involutions associated to generically finite $2$-to-$1$ covers
\[ \varphi_i : X \to \P^2 \]
Over the locus where $\varphi_i$ is finite (which is an open set in $\P^2$ whose complement has finitely many points) it induces an isomorphism $\varphi_i^* : H^0(\P^2, \omega_{\P^2}) \to H^0(X, \omega_X)^{\tau_i}$ via the trace map (here we use that $2$ is coprime to the characteristic) whch extends because it is defined away from codimension $2$ on the base. Since $H^0(\P^2, \omega_{\P^2}) = 0$ we must have $\tau^*_i \omega = -\omega$ since it is an involution that acts nontrivially on $H^0(X, \omega_X)$. 
Let $h = \iota^* \struct{\P^3}(1)$ under the standard embedding $\iota : X \embed \P^3$. Then, as in [KS], we identify
\[ \Aut(X,h) = \PGU_4(\FF_9) \embed \PGL_4(\FF_9) \]
induced by the embedding $\iota : X \embed \P^3$. 

This group is described as follows. First define,
\[ \U_4(\FF_9) = \{ A \in \GL_4(\FF_9) \mid A A^* = I \} \]
where $A^*$ is the conjugate transpose using the Frobenius $\sigma$ generating the Galois group of $\FF_9 / \FF_3$. 

\begin{rmk}
The identification $\Aut(X, h) = \PGU_4(\FF_9)$ arises from viewing the defining function
\[ F(X_0, X_1, X_2, X_3) = X_0^4 + X_1^4 + X_2^4 + X_3^4 \]
as the inner product of $(X_0, X_1, X_2, X_3)$ and $(X_0^{\sigma}, X_1^{\sigma}, X_2^{\sigma}, X_3^{\sigma})$. Hence the group of automorphisms of $\P^3_{\FF_9}$ preserving $F$ is exactly $\PGU_4(\FF_9)$. Any automorphism preserving $F$ up to scaling is represented by an element of $\PGU_4(\FF_9)$. Indeed, if $F(A \ul{X}) = \lambda F(\ul{X})$ then plugging in $\ul{X} = (1,0,0,0)$ we see that $\lambda$ is a sum of $4$-th powers of elements in $\FF_9$ and hence lies in $\FF_3^\times$. Either element has a $4$-th root $\xi$ in $\FF_9$ (since the Norm is surjective) and hence we can modify $A$ to $\xi^{-1} A$ so that $F(\xi^{-1} A \ul{X}) = F(\ul{X})$ hence $A \in \xi \cdot \U_4(\FF_9)$ so the image in $\PGL_4(\FF_9)$ lies in $\PGU_4(\FF_9)$. 
\end{rmk}

Note that for $A \in \U_4(\FF_9)$ we have
\[ (\det{A}) \, \sigma(\det{A}) = 1 \]
hence $\det{A} \in \ker{\Nm}$ where $\Nm(x) = x \sigma(x)$ is the norm for $\FF_9 / \FF_3$. Likewise, if $\lambda I \in \U_4(\FF_9)$ it satisfies the same condition so the central torus satisfies,
\[ \FF_9^\times \cdot I \cap U_4(\FF_9) = \ker{\Nm} \cdot I \]
Thus we define
\[ \PGU_4(\FF_9) = \U_4(\FF_9) / \ker{\Nm} \cdot I \]
Crucially the determinant
\[ \PGU_4(\FF_9) \xrightarrow{\det} \ker{\Nm} \subset \FF_9^\times \]
is well-defined since the kernel of the norm map has order $4$. The kernel of this map is $\PSL_4(\FF_9)$ by definition. Note that $\PSL_4(\FF_9)$ is simple (e.g.\ by \chref{https://kconrad.math.uconn.edu/blurbs/grouptheory/PSLnsimple.pdf}{these notes by Keith Conrad}) so the determinant map coincides with the abelianization of $\PGU_4(\FF_9)$. In particular, the representation
\[ \PGU_4(\FF_9) \acts H^0(\omega_X) \]
factors through the determinant.

\begin{prop}
The representation $\Aut(X) \acts H^0(\omega_X)$ is determined by the following data
\begin{enumerate}
\item $\tau_i$ acts via $-1$
\item $\Aut(X,h) = \PGU_4(\FF_9)$ acts via $\det$
\end{enumerate}
\end{prop}

\begin{proof}
Because $\PGU_4(\FF_9)$ must act factoring through the determinant, it suffices consider the action of diagonal matrices of the form $(1,\lambda,1,1)$ since $\det{A} = \det{(1,\det{A},1,1)}$. If
\[ F(X_0, \dots, X_n) = 0 \]
is the equation for a Calabi-Yau hypersurface in $\P^n$ then the top form can be written
\[ \omega = \frac{\d{\left( \tfrac{X_1}{X_0} \right)} \wedge \cdots \wedge \d{\left( \tfrac{X_{n-1}}{X_0} \right)}}{X_0^{-(n+1)} \partial_{X_n} F(X_0, \dots, X_n)} \]
For us, 
\[ F(X_0, \dots, X_3) = X_0^4 + \cdots + X_3^4 \]
and therefore $\partial_{X_4} F = 4 X_3^3$ so the above matrix acts by multiplication by $\lambda$ hence proving the claim.
\bigskip\\  
We can give an alternative argument that does not rely on the standard form for $\omega$. There is a natural $\GL_{n+1}$ linearization on $\struct{\P^{n}}(1)$ and a natural $\PGL_{n+1}$-linearization on $\omega_{\P^{n}}^\vee$ as the dual of the usual equivariant structure on the canonical bundle. These are not quite compatible. The discrepancy is exactly 
\[ \struct{\P^n}(n+1) := \struct{\P^n}(1)^{\ot (n+1)} \cong \omega_{\P^n}^{\vee} \ot \det \]
where $\det$ is the $1$-dimensional determinant representation of $\GL_{n+1}$ (or equivalently $\struct{\P^n}$ endowed with this nontrivial linearization). To see this, recall that the $\PGL_{n+1}$ linearization is defined by taking the induced $\SL_{n+1}$-linearization on $\omega_{\P^n}^\vee \cong \struct{\P^n}(n+1)$ and notincing that it kills $\mu_{n+1}$ hence factors through $\PGL_{n+1}$. Since any matrix can be written as $\lambda A$ for $A \in \SL_{n+1}$ (at the level of $\bar{k}$-points) we see that the action on $\struct{\P^n}(n+1)$ is via $\lambda^{n+1} A_*$ but $\lambda^{n+1} = \det{(\lambda A)}$ which demonstrates the claim. Consider the inclusion of a Calabi-Yau hypersurface
\[ X \embed \P^n \]
inducing $G = \Aut(X,\struct{X}(1)) \embed \GL_{n+1}$ these are automorphism of the pair $(X, \struct{X}(1))$ (this is a larger group than those automorphisms preserving the line bundle up to isomorphism, indeed it is an extension by $\Gm$ of $\Aut(X,h)$). The canonical construction of $\omega_X$, which gives it a $G$-equivariant structure is
\[ \omega_X = \shExt{1}{\struct{\P^n}}{\struct{X}}{\omega_{\P^n}} \]
this can be computed via a $G$-equivariant resolution
\[ 0 \to \struct{\P^{n+1}}(-(n+1)) \xrightarrow{f} \struct{\P^{n+1}} \to \struct{X} \to 0 \]
however, this is not quite $G$-equivariant. Since $G \acts X$ we see that $f$ is preserved up to scaling but the character $s_f : G \to k^\times$ is nontrivial and must enter into the exact sequence. The correct $G$-equivariant sequence is
\[ 0 \to \struct{\P^{n+1}}(-(n+1)) \ot s_f \xrightarrow{f} \struct{\P^{n+1}} \to \struct{X} \to 0 \]
where $\struct{\P^{n+1}}(-(n+1))$ is given a $G$-structure through $G \embed \GL_{n+1}$ and the others have the trivial $G$-structure.  Using this resolution,
\[ \omega_X \cong (\struct{\P^{n+1}}(n+1) \ot s_f^{\vee} \ot \omega_{\P^{n}})|_X = \det \ot s_f^{\vee} \]
Therefore, $\Aut(X, \struct{X}(1))$ acts on $H^0(\omega_X)$ via $\det \ot s_f^{\vee}$. Notice the action on $H^0(\omega_X)$ factors through 
\[ \Aut(X, \struct{X}(1)) \to \Aut(X, h) \]
as it must since elements of the kernel define trivial automorphisms of $X$. Indeed, this holds because for a scalar matrix $\lambda \cdot \id \in \Aut(X, \struct{X}(1)) \embed \GL_{n+1}$ acts via $\lambda^{n+1} \cdot \lambda^{-(n+1)} = 1$ because it scales $f$ by $\lambda^{n+1}$. However, this action is nontrivial on elements of $\Aut(X, h)$. Indeed, in our case of interest, $\U_4(\FF_9) \subset \Aut(X, \struct{X}(1))$ is the kernel of $s_f$ so on $\U_4(\FF_9)$ and hence on $\PGU_4(\FF_9) = \Aut(X, h)$ the action on $H^0(\omega_X)$ is via $\det$. 
\end{proof}


\subsection{The representation on $H^1(X, \Omega_X)$}

\newcommand{\magma}{\textsc{magma}\xspace}

We leverage the following trick to compute $G \acts H^1(X, \Omega_X)$.

\begin{lemma}
$G \acts \NS(X) \ot \FF_9$ has a unique rank $2$ submodule
\end{lemma}

\begin{proof}
In fact, a \magma computation shows that this is already true about the action $\left< \tau_1, \tau_2, \tau \right>$. Using the explicit integer matrices for the action on $\NS(X)$ we find a submodule lattice:
\begin{verbatim}
Submodule Lattice of GModule M of dimension 22 over GF(3^2)

Partially ordered set of submodule classes
------------------------------------------

---
[7]  Dim 22  Maximal submodules: 5 6
---
[6]  Dim 21  Maximal submodules: 4
[5]  Dim 3   Maximal submodules: 4
---
[4]  Dim 2   Maximal submodules: 2 3
---
[3]  Dim 1   Maximal submodules: 1
[2]  Dim 1   Maximal submodules: 1
---
[1]  Dim 0   Maximal submodules:
\end{verbatim}
so we see that [4] is the unique submodule of rank $2$ and it consists of two invariant lines. Call this submodule $U_2 \subset \NS(X) \ot \FF_9$. 
\end{proof}

Since $\ker{c_1^{\text{Hodge}}}$ is also rank $2$ because $c_1^{\text{Hodge}}$ is surjective onto a rank $20$ space and is $G$-invariant, we conclude that
\[ U_2 = \ker{c_1^{\text{Hodge}}} \]
Therefore, the action $G \acts H^1(X, \Omega_X)$ is isomorphic to the action
\[ G \acts (\NS(X)_{\FF_9} / U_2) \]
which is easily computed in \magma.

\subsection{Computations}

[KS] computed explicit matrices for $\tau_i$ and $\PGU_4(\FF_9)$ acting on $\NS(X)$.

\begin{rmk}
In [KS] the matrices act \textit{on the right} meaning $\tau_*(x) = \vec{x} T_{\tau}$ where $\vec{x}$ in row vector corresponding to $x \in \NS(X)$ the line basis specified by [KS]. This is consistent with \textsc{magma} conventions for right actions. The reader should be aware that we will adopt this convention throughout so $G \acts X$ is a right action and $\phi_1 \circ \phi_2$ is the automorphism given by first applying $\phi_1$ and then $\phi_2$. This will not change anything substantial about the calculation (even if the wrong convention were employed) because transpose does not change the rank or spectrum.
\end{rmk}


We let $\phi_i$ be the $i^{\text{th}}$-automorphism in $\PGU_4(\FF_9)$ according to the indexing of [KS, data files]. Also let $\tau$ be the special automorphism of order $28$ considered in [KS12, Ex. 3.4]. and in [EO].
Explicitly,
\begin{equation}
\tau :=
\begin{pmatrix}
i & 0 & i & -1 + i
\\
1 & 1 -i & -1 & 0
\\
1 & i & i & -i
\\
1 & -1 & -i & -1
\end{pmatrix}
\end{equation}
where $i \in \FF_9$ is a choice of square root of $-1$. Note that $i$ generates $\ker{\Nm}$. We also record two other elements of $\PGU_4(\FF_9)$
\begin{equation}
\phi_2 :=
\begin{pmatrix}
0 & 0 & 0 & 1
\\
0 & 0 & 1 & 0
\\
0 & 1 & 0 & 0
\\
-1 & 0 & 0 & 0
\end{pmatrix}
\end{equation}
\begin{equation}
\phi_5 :=
\begin{pmatrix}
0 & 0 & 0 & 1
\\
0 & 0 & 1 & 0
\\
0 & -1 & 0 & 0
\\
1 & 0 & 0 & 0
\end{pmatrix}
\end{equation}
Notice that:
\begin{enumerate}
\item $\det{\tau} = 1$
\item $\det{\phi_2} = -1$
\item $\det{\phi_5 } = -1$.
\end{enumerate}

Now we define
\[ \phi := \tau_1 \circ \tau \circ \phi_2 \circ \tau_2 \circ \tau \circ \phi_5 \circ \tau_1 \circ \tau \in \Aut(X) \]
Note that $\phi \acts H^0(\omega_X)$ by $-1$ since each $\tau_i$ acts by $-1$ and $\phi_2, \phi_5$ act through their determinant by $-1$ and $\tau$ acts through its determinant by $1$.

\begin{thm}
Let $\phi \acts X$ be the automorphism as above. Then
\begin{enumerate}
\item $\phi \acts H^1(X, \T_X)$ has no $1$-eigenspace meaning that $\phi_* - \id$ is an isomorphism

\item $\phi \acts \NS(X)$ has characteristic polynomial
\begin{align*}
p_{\phi}(x) &= x^{22}-30 x^{21}+15 x^{20}-14 x^{19}+16 x^{18}+7 x^{17}-19
   x^{16}
   \\
   & + 4 x^{15}-14 x^{14}+15 x^{13}-4 x^{12}+10 x^{11}-4
   x^{10}+15 x^9-14 x^8
   \\
   & + 4 x^7-19 x^6+7 x^5+16 x^4-14 x^3+15
   x^2-30 x+1
\end{align*}
which is irreducible

\item $\phi$ has positive entropy $h(\phi)$ equal to the logarithm of a Salem number $a$ of degree $22$ which is numerically
\[ a = 29.5071\cdots \, , \quad h(g) = 3.38463 \cdots \]
\item $\phi$ does not lift to any projective model $X_R \to \Spec{R}$

\item $\phi$ lifts to a canonically-defined formal scheme $\X \to \Spf{W(k)}$ lift of $X$.
\end{enumerate}
\end{thm}

\begin{proof}
Let $T$ be the matrix representing the action $\phi \acts H^1(X, \Omega_X)$. In the previous section we found a method to compute this. From the $G$-isomorphism
\[ H^1(X, \T_X) \iso H^1(X, \Omega_X)^{\vee} \ot H^0(\omega_X)^{\vee} \]
we see that $\phi \acts H^1(X, \T_X)$ via the matrix $-T^{\top}$ (using that $\phi \acts H^0(\omega_X)$ by $-1$). \magma computes the factorization of the characteristic polynomial of this action on the $20$-dimension $\FF_9$-vectorspace $H^1(X, \T_X)$ to be 
\begin{verbatim}
[
    <t + 1, 2>,
    <t + F.1^2, 1>,
    <t + F.1^6, 1>,
    <t^8 + 2*t^7 + F.1^2*t^6 + F.1^7*t^5 + F.1^6*t^4 + F.1^7*t^3 + F.1^2*t^2 + 
        2*t + 1, 1>,
    <t^8 + 2*t^7 + F.1^6*t^6 + F.1^5*t^5 + F.1^2*t^4 + F.1^5*t^3 + F.1^6*t^2 + 
        2*t + 1, 1>
]
\end{verbatim}
where $F.1$ is a generator of $\FF_9^\times$ in this case $1 + i$. Since $F.1^2 = -i$ and $F.1^6 = i$ this means that $\phi \acts H^1(X, \T_X)$ does not have any $1$-eigenspace (or generalized eigenspace). This proves (a). For (b) \magma performs the calculation using the explicit matrices defined in the data associated to [KS]. Then (c) follows immediately. (d) follows from the results of [EO] showing that no automorphism with entropy the log of a Salem number of degree $22$ can lift to a projective K3 surface over characteristic zero. (e) then follows from the subsequent discussion. 
\end{proof}

\subsection{Deformations of Automorphism}

\begin{lemma}
Let $X$ be a smooth proper $k$-scheme. Let $f : X \to X$ be an endomorphism. The complex
\[ C^\bullet = [\T_X \xrightarrow{f^* - \d{f}} f^* \T_X] \]
supported in degrees $[0,1]$ controls the deformation theory of the pair $(X, f)$. Explicitly, for any small extension of Artin rings $A' \onto A$ with residue field $A$ and given a deformation $(X_A, f_A)$ of $(X, f)$ over $A$ there is
\begin{enumerate}
\item an obstruction class $\ob \in \HH^2(C^\bullet)$
\item if $\ob = 0$ then the deformations of $(X_A, \phi_A)$ to $A'$ form a torsor over $\HH^1(C^\bullet)$
\item the automorphisms of any deformation to $A'$ is isomorphic to $\HH^0(C^\bullet)$.
\end{enumerate}
\end{lemma}

\begin{proof}
Let $\{U_i\}$ be an affine cover of $X$. Consider the Čech complex computing $\HH^*(C^\bullet)$:
\[ \check{C}^{p,q} = \bigoplus_{i_0 < \dots < i_p} \Gamma(U_{i_0\ldots i_p}, C^q) \]
with differentials $d_1 = \check{d}$ (the Čech differential) and $d_2 = f^* - \d{f}$. This is a double complex whose total complex computes $\HH^*(C^\bullet)$.
\bigskip\\
Given a deformation $(X_A, f_A)$ over $A$ and a small extension $A' \onto A$ with kernel $I$, let $\wt{U}_i$ be local lifts of $U_i$ and $\wt{f}_i : \wt{U}_i \to \wt{U}_i$ be local lifts of $f$. For compatibility, we need diagrams
\begin{center}
\begin{tikzcd}
\wt{U}_i |_{U_{ij}} \arrow[r, "\wt{f}_i + \eta_i"] \arrow[d, "\varphi_{ij} + \xi_{ij}"] & \wt{U}_i |_{U_{ij}} \arrow[d, "\varphi_{ij} + \xi_{ij}"]
\\
\wt{U}_j |_{U_{ij}} \arrow[r, "f_i + \eta_i"] & \wt{U}_j |_{U_{ij}}
\end{tikzcd}
\end{center}
for some choice of isomorphism $\varphi_{ij} : \wt{U}_i|_{U_{ij}} \iso \wt{U}_j |_{U_{ij}}$ such that $\varphi_{ij} = \id$ mod $I = \ker{(A' \to A)}$. This exists since there is a unique isomorphism class of deformation of a smooth affine scheme. Here $\xi_{ij} \in \check{C}^{1,0} \ot I$ and $\eta_i \in \check{C}^{0,1} \ot I$ and we can write any such deformation of $f_i$ in this form any any gluing as $\varphi_{ij} + \xi_{ij}$ defining a deformation of $X$. The key compatibility equation is derives from commutativity of the above diagram and reads:
\[ (\wt{f}_i + \eta_i) \circ (\varphi_{ij} + \xi_{ij})^\# = (\varphi_{ij} + \xi_{ij})^\# \circ (\wt{f}_j + \eta_j)^\# \]
which expands to
\[ \wt{f}_i^\# \varphi_{ij}^\# + \eta_i + f_i^\# \xi_{ij}^\# = \varphi_{ij}^\# \wt{f}_j^\# + \xi_{ij} f_j^\# + \eta_j \]
using that $I^2 = 0$ and that $\xi$ or $\eta$ land in $I$ so multiplying by either has the effect of reduction of the other term mod $I$. This means we need the equation
\[ \wt{f}_i^\# \varphi_{ij}^\# - \varphi_{ij}^\#\wt{f}_j^\# = f_j^\# + \eta_j - \eta_i + \xi_{ij} f_j^\#  -  f_i^\# \xi_{ij}^\# \]
The RHS is the total differential of $(\eta, \xi)$ in the total complex projected to $\check{C}^{1,1}$. Therefore, the space of solutions (if one exists) forms a torsor over $\HH^1(C^\bullet)$. We need to simulteneously be able to solve this equation along with the cocycle that says
\[ d(\xi_{ij}) = \varphi_{ij}^{\#} \circ (\varphi_{ik}^{\#})^{-1} \circ \varphi_{jk}^{\#} \]
This defines an element 
\[ (\varphi_{ij}^{\#} \circ (\varphi_{ik}^{\#})^{-1} \circ \varphi_{jk}^{\#}, \wt{f}_i^\# \varphi_{ij}^\# - \varphi_{ij}^\#\wt{f}_j^\#) \in \check{C}^{2,0} \oplus \check{C}^{1,1} \]
which is easily checked to be a cocycle and we are asking if it is a coboundary hence giving an obstruction
\[ \ob \in \HH^2(C^\bullet) \]
\end{proof}

\begin{rmk}
If we assume $\Def_X$ is unobstructed so we can choose $\varphi_{ij}$ to satisfy the cocycle condition then we obtain The obstruction class lies in
\[ \coker(H^1(\mathcal{T}_X) \xrightarrow{f^* - \d{f}} H^1(f^*\mathcal{T}_X)) \]
since we are asking if a class in $H^1(f^* \mathcal{T}_X)$ lies in the image of $f^* - \d{f}$ up to a boundary $\{ \eta_i \}$.
When this vanishes, the choices of compatible deformations form a torsor over $\HH^1(C^\bullet)$ as claimed.
\end{rmk}

\begin{rmk}
When $\phi  : X \to X$ is an automorphism then $\phi^* : \T_X \to \phi^* \T_X$ is an isomorphism of sheaves so we can form $\phi_* : \T_X \to \T_X$ as $(\phi^*)^{-1} \circ \d{\phi}$ which is the pushforward of vector fields in the sense used in differential geometry. It is clear that $C^\bullet$ is isomorphic to the complex
\[ [\T_X \xrightarrow{\id - \phi_*} \T_X] \]
supported in degrees $[0,1]$.
\end{rmk}

\begin{lemma}
If $X$ is a smooth proper $k$-variety with $H^0(X, \T_X) = H^2(X, \T_X) = 0$ then there is an exact sequence
\[ 0 \to \HH^1(C^\bullet) \to H^1(X, \T_X) \xrightarrow{\phi_* - \id} H^1(X, \T_X) \to \HH^2(C^\bullet) \to 0 \] 
\end{lemma}

\begin{cor}
If moreover, $\phi_* - \id$ is an isomorphism on $H^1(X, \T_X)$ then $(X, \phi)$ is unobstructed and has a trivial tangent space. 
\end{cor}

\begin{cor}
Suppose that $X$ is a smooth proper $k$-variety with an endomorphism $\phi : X \to X$. Let $k$ be a perfect field of characteristic $p > 0$. Suppose that 
\begin{enumerate}
\item $H^0(X, \T_X) = 0$
\item $H^2(X, \T_X) = 2$
\item $\phi_* - \id : H^1(X, \T_X) \to H^1(X, \T_X)$ is an isomorphism
\end{enumerate}
then there exists a canonical lift of $(X, \phi)$ to a formal scheme $\X \to \Spf{W(k)}$.
\end{cor}

\begin{proof}
Indeed, the above shows that the deformation space of $(X, \phi)$ is smooth over $W(k)$ of relative dimension $0$. Since $\Def_{(X,\phi)} \to \Spf{W}$ is an isomorphism over $k$ it is an isomorphism. 
\end{proof}

\section{Entropy}

\subsection{Spectral Radius Entropies}


Recall that for an action $\varphi : V \to V$ on a complex vectorspace the \textit{spectral radius} is 
\[ \rho(\varphi) := \sup_{\lambda} |\lambda| = \sup_{v \in V \sm \{ 0 \}} \frac{|| \varphi v||}{|| v ||} \]
which holds for any choice of norm on $V$. The \textit{entropy} of $\varphi$ is $h(\varphi) := \log{\rho(\varphi)}$.

\begin{defn}
Let $X$ be a finite CW complex (we just need the total ring $H^\bullet(X, \CC)$ finite dimensional over $\CC$) and $\varphi : X \to X$ an automorphism. We define the \textit{topological entropy} $h(\varphi)$ as the log of the spectral radius of $\varphi \acts H^\bullet(X, \CC)$. 
\end{defn}

\begin{defn}[Esnault-Srinivas]
Let $X$ be a smooth proper variety over a field $k$ and $\varphi : X \to X$ an endomorphism over $k$. Then we define
\begin{enumerate}
\item for a prime $\ell$ invertible on $k$, the characteristic polynomial of $\varphi \acts H^\bullet_{\et}(X_{\bar{k}}, \Q_{\ell})$ is independent of $\ell$ and has integer coefficients and algebraic integer roots so we made defien the spectral radius of the action as a real number and its logarithm as the \textit{topological entropy} $h(\varphi)$ 

\item Let $\CH^\bullet_{\text{num}}(X_{\bar{k}})$ be the Chow ring modulo numerical equivalence. The underlying abelian groups is a finite free $\Z$-module hence we can define the \textit{algebraic entropy} of $\varphi$ as the log of the spectral radius of $\varphi \acts \CH^\bullet_{\text{num}}(X_{\bar{k}})$.
\end{enumerate}
\end{defn}

\begin{rmk}
Note, if $k \embed \CC$ then by Artin comparison $h_{\top}(\varphi)$ can also be computed as the spectral radius on singular cohomology. 
\end{rmk}

\begin{theorem}
If $X$ is an algebraic surface then the algebraic entropy and topological entropy coincide. 
\end{theorem}

\subsection{Dinh and Sibony}

Let $X$ be a compact \Kahler manifold of dimension $n$ and $[\omega] \in H^2(X, \RR)$ a \Kahler class. Choose a norm $|| \bullet ||$ on $H^{p,p}(X, \RR) := H^{p,p}(X) \cap H^{2p}(X, \RR)$. For any endomorphism $f : X \to X$ we define
\[ d_{p,n} := || (f^n)^* [\omega^p] || \]
and 
\[ d_p := \limsup_{n \to \infty} d_{p,n}^{1/n} \quad H(f) := \sup_{0 \le p \le n} \log{d_p} \]
Note that $d_{p,n}$ does not depend on the choice of \Kahler class. Furthermore, $d_p$ and $H(f)$ do not depend on the the choice of form $\omega$ or on the choice of norm $|| \bullet ||$. Moreover, $d_p$ is the spectral radius of $f^*$ on $H^{p,p}(X)$ {\color{red} WHY?}. We say that $d_p$ is the dynamical degree of order $p$ of $f$. In particular, we can compute $d_p$ from the formula
\[ d_p := \lim_{n \to \infty} \left( \int_X (f^n)^* \omega^p \wedge \omega^{k-p} \right)^{1/n} \]

\subsubsection{Topological Entropy}

\newcommand{\dist}{\mathrm{dist}}

Let $\dist$ denote the metric distance on $(X, \omega)$ induced by the associated \Kahler metric. Then we denote
\[ \dist_n(x,y) := \max_{0 \le i \le n-1} \dist(f^i(x), f^i(y)) \]
Let $s_n(\epsilon)$ be the largest number of balls of radius $\epsilon / 2$ defined in the metric $\dist_n$ that can fit disjointly in $X$. The \textit{topological entropy} of $f$ is defined by the formula
\[ h(f) := \limsup_{\epsilon \to 0} \limsup_{n \to \infty} \frac{\log{s_n(\epsilon)}}{n} \]

\begin{theorem}[Yomdin-Gromov]
Let $f$ be a holomorphic endomorphism of a compact \Kahler manifold $X$. Then $h(f) = H(f)$.
\end{theorem}

\section{Salem Numbers}

\begin{defn}
A \textit{Salem} number is an algebraic integer $\lambda \in \CC$ such that 
\begin{enumerate}
\item $\lambda \in \R_{>1}$
\item every conjugate of $\lambda$ has $|\lambda| \le 1$
\item at least one conjugate of $\lambda$ lies on the unit circle: $|\lambda| = 1$.
\end{enumerate}
\end{defn}

\begin{prop}
The minimal polynomial of an algebraic integer on the unit circle (besides $1$) is reciprocal meaning its coefficients are a palindrome.
\end{prop}

\begin{proof}
Indeed, let $z \in \CC$ have $|z|$ and $p(x)$ the minimal polynomial. Let $n = \deg{p}$ then since $z \bar{z} = 1$,
\[ z^n \ol{p(1/\bar{z})} = z^n \ol{p(z)} = 0 \]
so $z$ is a root of $z^n \ol{p(1/\bar{z})}$ which is a nonzero polynomial of degree $\le n$ so we must have, by minimality,
\[ p^{\dagger} = c p \]
where $p^{\dagger}(z) = z^n \ol{p(1/\bar{z})}$ is the conjugate reciprocal. Note that because $p \in \Z[x]$ we have that $p^{\dagger}$ is the reciprocal, the polynomial whose coefficients are reversed. Thus it suffices to show $c = 1$. The above formula says that if 
\[ p(x) = \sum_i a_i x^i \]
then $c a_i = \ol{a_{n-i}} = a_{n-i}$ since the coefficients are integers. Taking the sum,
\[ c (a_0 + \cdots + a_n) = a_n + \cdots + a_0 \]
and the sum is nonzero because $p(1) \neq 0$ because it is irreducible and $z \neq 0$ so we conclude $c = 1$.
\end{proof}

\begin{cor}
The minimal polynomial of a Salem number is reciprocal meaning its coefficients are a palindrome.
\end{cor}

\begin{lemma}
Let $p$ be a conjugate-reciprocal polynomial. Then $p(\alpha) = 0$ iff $p(\alpha^{-1}) = 0$.
\end{lemma}

\begin{proof}
This is obvious from $p^{\dagger}(x) = x^n \ol{p(1/\bar{x})}$ and $p = p^{\dagger}$.
\end{proof}

\begin{cor}
Let $\lambda$ be a Salem number. Then the conjugates of $\lambda$ are exactly
\[ \lambda, \lambda^{-1}, \alpha_1, \dots, \alpha_{n-2} \]
where $|\alpha_i| \in \CC$. Since $\lambda^{-1}$ is also an algebraic integer, $\lambda$ is a unit in the ring of algebraic integers. 
\end{cor}

\begin{proof}
Since the minimal polynomial $p$ is reciprocal we see that $\lambda^{-1}$ is also a root. Furthermore, for any other root $\alpha$, by assumption $|\alpha| \le 1$ but since $p$ is reciprocal $\alpha^{-1}$ is also a root so either $\alpha^{-1} = \lambda$ hence $\alpha = \lambda^{-1}$ or $|\alpha^{-1}| \le 1$ so we conclude $|\alpha| = 1$.
\end{proof}

\subsection{Automorphisms of K3 surfaces over $\CC$}

Let $X$ be a complex (possibly non-algebraic) K3 surface. Then $H^2(X, \Z)$ is an even unimodular lattice of singnature $(3, 19)$. By the Torelli theorem $\Aut(X) \acts H^2(X, \Z)$ is faithful. For $f \in \Aut(X)$, define two invariants:
\begin{enumerate}
\item $\lambda(f)$ -- the spectral radius of $f \acts H^2(X)$
\item $\delta(f)$ -- the eigenvalue of $f$ on the line $H^{2,0}(X) = H^0(X, \omega_X) = \CC \cdot \omega$ inside $H^2(X, \CC)$
\end{enumerate}
The topological entropy of $f$ is 
\[ h(f) = \log{\lambda(f)} \ge 0 \]

\begin{rmk}
Since $f$ preserves the volume $\int_X \omega \wedge \ol{\omega}$ we see that $|\delta(f)| = 1$. Indeed, $f$ acts on $H^4(X,\Z)$ by the degree but $\deg{f} = 1$ so it must preserve this integral which is proportional to $[X]$ cap the generator of $H^4(X, \Z)$. We refer to $\delta(f)$ as the \textit{determinant} of $f$, since
\[ \det{D f_p} = \delta(f) \]
for any fixed-point $p \in X$. 
\end{rmk}

\begin{prop}
If $h(f) > 0$ then $\lambda(f)$ is the unique eigenvalue outside the unit circle and hence is a Salem number. Otherwise the eigenvalues of $f$ are all roots of unity.
\end{prop}

\begin{proof}
If $h(f) = 0$ this means $\lambda(f) = 1$ so all eigenvalues are within the unit disk. Since they are non-zero algebraic integers the product of all of them, the constant term of $p$ the characteristic polynomial of $f \acts H^2(X)$, is a nonzero integer $\le 1$ so it must equal $1$ meaning that all $|\lambda_i| = 1$ since their product equals $1$ and $|\lambda_i| \le 1$. If all conjugates of an algebraic integer lie on the unit circle then it is a root of unity so in this case we conclude.
\par 
Otherwise, $\lambda(f) > 1$ and let $\lambda$ be an eigenvalue with $|\lambda| > 1$. Since $f^*$ stabilizes $H^{1,1}(X) \subset H^2(X)$ which has signature $(1,19)$ it is conjugate to an element $T \in O(2,0) \times O(1,19)$. Use the following lemma to see that $T$ has a unique eigenvalue outside the unit disk. Since $\lambda$ is unique, it must be real.
\par 
Since $f^*$ preserves the \Kahler cone, the $O(1,19)$ part does not interchange the sheets of the light-cone in $H^{1,1}_{\RR}$, and thus $\lambda > 1$. Hence $\lambda$ is a Salem number since it is the unique root of $p$ outside the unit disk. Therefore $p$ is a product of at most one Salem polynomial and some number of cyclotomic polynomials.
\end{proof}

\begin{lemma}
A transformation $T \in O(p,q)$ has at most $\min{(p,q)}$ eigenvalues outside the unit circle, counted with multiplicities.
\end{lemma}

\begin{proof}
Consider the subspace
\[ S = \bigoplus_{|\lambda| > 1} E(\lambda) \]
which is isotropic and defined over $\RR$. Thus
\[ \dim{S} \le \min{(p,q)} \]
and $\dim{S}$ is at least as large as the number of eigenvalues outside $S^1$.
\end{proof}

\begin{prop}
If $X$ is projective then $\delta(f)$ is a root of unity so there cannot be a Siegel disk. 
\end{prop}

\begin{proof}
Since $X$ is projective, there is a $D \in \Pic(X) \subset H^2(X, \Z)$ with $D^2 > 0$. The supsace $H^{1,1}(X) \cap D^{\perp}$ is negative-definite, with signature $(0,19)$ by the Hodge index theorem and contains $\Pic{X} \cap D^{\perp}$. The intersection form on $\Pic{X} \ot \RR$ hence has signature $(1,n)$ or some $0 \le n \le 19$. Consequence, the rational $f^*$-invariant subspace
\[ S = \Pic{X}^{\perp} \supset H^{2,0}(X) \oplus H^{0,2}(X) \]
has signature $(2, 19-n)$. Now $f^*|_S$ preserves the signature $(2,0)$-subspace on the right, so it is conjugate to an element of $O(2) \times O(19-n)$. Thus all eigenvalues of $f^*|_S$, inclusing $\delta(f)$ lie on the unit circle (since it is conjugate to an orthogonal matrix). But $f^*|_S$ also preserves the lattice $S \cap H^2(X, \Z)$, so its characteristic polynomial lies in $\Z[t]$, and therefore the eigenvalues are all roots of unity.
\end{proof}

\begin{theorem}
Up to isomorphism, there are only countably many pairs $(X, f)$ where $\delta(f)$ is not a root of unity and these all have algebraic periods.
\end{theorem}

\begin{proof}
Assume $\delta = \delta(f)$ is not a root of unity. Since the characteristic polynomial $p$ of $f \acts H^2(X, \Z)$ has a unique Salem factor and the other factors are cyclotomic, $\delta$ is a root of the Salem factor and hence has multiplicity $1$ (since the Salem factor is irreducible). Therefore, $H^{2,0}(X)$ is an eigenspace for $f^*$, and therefore $f^*$ determines the Hodge structure on $H^2(X)$ up to finitely many choices (the choice of root of the Salem factor not on the unit circle). By the Torelli theorem, the Hodge structure on $H^2(X)$ together with $f^* \acts H^2(X, \Z)$ determines $(X, f)$ up to isomorphism. There are only countably many because there are only countably many $f$ acting on the lattice $H^2(X, \Z)$.  
\end{proof}

\section{Eratum}

I think [KS] has the wrong matrix for $\tau$. The automorphism of $X$ whose corresponding action on $\NS(X)$ corresponds to the action $T$ is given in the data set at line 3370320 (there is a shift by one where line $n$ in FQprojautS corresponds to line $n+1$ in FQprojaut because there is an extra header line. If we look in the data file:
\begin{verbatim}
sed -n '3370321p' FQprojaut.m
[1, 2, 6, 2, 2, 5, 5, 5, 2, 6, 8, 0, 3, 0, 5, 2], 
\end{verbatim}
unwinding, we get the matrix
\[ 
\tau = 
\begin{pmatrix}
i & 0 & i & -1 + i
\\
1 & 1 -i & -1 & 0
\\
1 & i & i & -1
\\
1 & -1 & -i & -1
\end{pmatrix} \]
which is different from the matrix given in [KS]. Indeed, that matrix was in $\PSL_4(\FF_9)$ but this one has $\det{\tau} = i$ whcih agrees with my calculations arising from $T$. Therefore, Helene's choice of automorphism actually will give a canonical lift. 


\end{document}

\documentclass[12pt]{article}
\usepackage{import}
\import{../}{AlgGeoCommands}

\newcommand{\dbar}{\bar{\partial}}
\newcommand{\HH}{\mathbb{H}}
\renewcommand{\gr}{\mathrm{gr}}
\newcommand{\R}{\mathrm{R}}
\renewcommand{\ch}{\mathrm{ch}}
\renewcommand{\F}{\mathcal{F}}


\begin{document}

\subsection{Notation}

Let $\omega = \alpha H$ and $B = \beta H$ where $H$ is an ample class. Let 
\[ \ch^B(E) = e^{-B} \ch(E) \]
therefore
\begin{align*}
\ch_0^B(E) &= \ch_0(E)
\\
\ch_1^B(E) &= \ch_1(E) - B
\\
\ch_2^B(E) &= \ch_2(E) - B \ch_1(E) + \tfrac{1}{2} B^2 \ch_0(E)
\\
\ch_3^B(E) &= \ch_3(E) - B \ch_2(E) + \tfrac{1}{2} B^2 \ch_1(E) - \tfrac{1}{6} B^3 \ch_0(E)
\end{align*}
Furthermore, we define the slope 
\[ \mu_{\omega, B}(E) := \frac{\omega^{n-1} \ch_1^B(E)}{\omega^n \ch_0^B(E)} = \frac{\omega^{n-1} \ch_1(E) - \omega^{n-1} \cdot B \ch_0(E)}{\omega^n \ch_0(E)} \]
when $E$ is a torsion-free sheaf then 
\[ \mu_{\omega, B}(E) = \frac{\omega^{n-1}c_1(E)}{\rank{E}} - \omega^{n-1} \cdot B \]
We define the central charge
\[ Z_{\omega,B}(E) = \int_X e^{-i\omega - B} \ch(E) \]
Define two subcategories of $\Coh{X}$. 

\begin{defn}
Let $\T_{\omega,B}$ be the subcategory generated, via extensions, by $\mu_{\omega, B}$-semistable sheaves of slope $\mu_{\omega, B} > 0$ where we set the torsion sheaves to have $\mu_{\omega,B} = \infty$. 
\end{defn} 

This is the category of sheaves $E$ such that $E / E_{\tors}$ has all HN-slopes strictly positive. 

\begin{defn}
Let $\F_{\omega, B}$ be the subcategory generated under extensions by $\mu_{\omega,B}$-semistable sheaves of slope $\mu_{\omega,B} \le 0$ (hence these are torsion-free).
\end{defn}

\begin{defn}
$\cB_{\omega,B} = \left< \F_{\omega,B}[1], \T_{\omega,B} \right> \subset D^b(X)$ what we call the \textit{tilt}. We define the slope function on $\cB_{\omega,B}$
\[ \nu_{\omega,B}(E) = \frac{\Im{Z_{\omega,B}(E)}}{\omega^2 \ch_1^B(E)} = \frac{\omega \ch_2^B(E) - \tfrac{1}{6} \omega^3 \ch_0^B(E)}{\omega^2 \ch_1^B(E)} \]
\end{defn}

\begin{conj}
For any tilt-stable object $E \in \cB_{\omega,B}$ satisfying $\nu_{\omega,B}(E) = 0$ meaning
\[ \tfrac{1}{6} \ch_0^B(E) = \omega \, \ch_2^B(E) \]
we have the inequality
\[ \ch_3^B(E) \le \tfrac{1}{18} \omega^2 \, \ch_1^B(E) \]
\end{conj}

\subsection{A few computations}

\newcommand{\Td}{\mathrm{Td}}

By Grothendieck-Riemann-Roch
\[ \ch(\struct{C}) = \iota_* (\Td_C \cdot \Td_X^{-1}) = \iota_* (1 - (g-1)[*] + \tfrac{1}{2} K_X \cdot C [*]) = [C] + \tfrac{1}{2}(K_X \cdot C - 2(g-1))[*] \]

\subsection{Tangent Bundle}

Let $X$ be not uniruled then Miyaoka's theorem shows that $\mu_{\max}(\T_X) \le 0$. Therefore, as long as $\beta \ge 0$ we have $\T_X \in \F_{\omega,B}$.
\begin{enumerate}
\item $\T_X[1] \in \cB_{\omega,B}$ for $\beta \ge 0$
\item if $\T_X$ satisfies $\mu_{\max}(\T_X) < -\epsilon$ then $\T_X \in \cB_{\omega,B}$ for $- H^3 \epsilon > \beta$
\item if $\mu_{\min}(\Omega_X) > \epsilon$ then $\Omega_X \in \cB_{\omega,B}$ for $\beta < \omega^3 \epsilon$
\item $\I_C[1] \in \cB_{\omega, B}$ for $\beta \ge 0$ and $\I_C \in \cB_{\omega,B}$ for $\beta < 0$
\item $\I_C^\vee[1] \in \cB_{\omega,B}$ for $\beta \ge 0$
\end{enumerate}


We want to compute
\[ \Ext{i}{X}{\struct{X}}{\I_C \ot \T_X} = \Ext{i}{X}{\I_C^\vee}{\T_X} = \Ext{i}{X}{\Omega_X}{\I_C} \]


\subsection{Case (a)}

We consider $X$ such that $\mu(\Omega_X) = \epsilon > 0$ and $\Omega_X$ is $\mu$-semistable. We need $0 < \beta H^3 < \epsilon$.  
\[ \Ext{2}{X}{\Omega_X}{\I_C} = \Hom{D^b(X)}{\Omega_X}{\I_C[2]} \]
We consider a nonzero element $\xi$ then we consider the extension defined by $(\xi, \xi, \xi)$
\[ \I_C[1]^{\oplus 3} \to E_\xi \to \Omega_X \] 
Now we compute 
\[ \ch^B(E_\xi) = 3 \ch^B(\I_C[1]) + \ch^B(\Omega_X) \]
likewise
\begin{align*}
\ch^B(\I_C[1]) &= e^{-B} \ch(\I_C[1]) = -e^{-B} (\ch(\struct{X}) - \ch(\struct{C})) = - e^{-B} (1 - [C] + \tfrac{1}{2} (K_X \cdot C - 2(g-1))[*])
\\
& = \left( -1, B, - \tfrac{1}{2} B^2 + [C], \tfrac{1}{6} B^3 - \tfrac{1}{2}(K_X \cdot C - 2(g-1)) \right)
\end{align*}
Therefore
\[ \ch(E_\xi) = \left( 0, c_1, \tfrac{1}{2} c_1^2 - c_2 + 3 [C], \tfrac{1}{6} (c_1^3 - 3 c_1 c_2 + 3 c_3) - \tfrac{3}{2}(c_1 \cdot C - 2(g-1)) \right) \]
hence
\begin{align*}
\ch^B(E_{\xi}) & = \Big(0, c_1, - c_1 \cdot B + \tfrac{1}{2} c_1^2 - c_2 + 3[C],
\\
& \quad \quad \tfrac{1}{6}(c_1^3 - 3 c_1 c_2 + 3 c_3) - \tfrac{3}{2} (c_1 \cdot C - 2(g-1)) + \tfrac{1}{2} B^2 \cdot c_1 - B \cdot (\tfrac{1}{2} c_1^2 - c_2 + 3 [C]) \Big)
\end{align*}
We want 
\[ H \cdot \left( \tfrac{1}{2} c_1^2 + c_1 \cdot B - c_2 + 3[C] \right) = 0 \]
i.e.
\[ \beta = \frac{H \cdot (c_2 - \tfrac{1}{2} c_1^2) - 3 \deg_H{C}}{c_1 \cdot H^2} \] 
but this is a series condition on the curves we can consider. Now we consider the conjectural inequality:
\[ \ch_3^B(E_\xi) \le \tfrac{1}{18} \omega^2 \, \ch_1^B(E_\xi) \]
This says
\[ \tfrac{1}{6}(c_1^3 - 3 c_1 c_2 + 3 c_3) - \tfrac{3}{2} (c_1 \cdot C - 2(g-1)) + \tfrac{1}{2} B^2 \cdot c_1 - B \cdot (\tfrac{1}{2} c_1^2 - c_2 + 3 [C]) \le \tfrac{1}{18} \omega^2 \, \ch_1^B(E_\xi) = \tfrac{1}{18} \alpha^2 H^2 \cdot c_1 \]
We are going to take the limit as $\alpha \to 0$. Therefore, $E_\xi$ is destabilized if the RHS is positive. Therefore
\[ B \cdot (\tfrac{1}{2} c_1^2 - c_2 + 3 [C]) = \beta H \cdot (\tfrac{1}{2} c_1^2 - c_2 + 3 [C]) = - \beta H \cdot B \cdot c_1 = - B^2 \cdot c_1 \]
Therefore we get
\[ \tfrac{1}{9} (c_1^3 - 3 c_1 c_2 + 3 c_3) - (c_1 \cdot C - 2(g-1)) + B^2 \cdot c_1 \le 0 \]

\subsection{(b)}

Alternatively, lets look at
\[ \Ext{1}{X}{\struct{X}}{\I_C \ot \T_X} = \Ext{1}{X}{\I_C^\vee}{\T_X} = \Ext{1}{X}{\I_C^\vee[1]}{\T_X[1]} \]
then we get an extension
\[ \T_X[1] \to E_\xi \to \I_C^\vee[1] \]

\end{document}
\documentclass[12pt]{article}
\usepackage{import}
\import{../}{AlgGeoCommands}

\newcommand{\dbar}{\bar{\partial}}
\newcommand{\HH}{\mathbb{H}}
\renewcommand{\gr}{\mathrm{gr}}
\newcommand{\R}{\mathrm{R}}

\begin{document}

\title{Nonvanishing $1$-forms on varieties admitting a good minimal model.}

\maketitle

\section{Talk}

We start with a theorem of Popa and Schnell that I will state via the contrapositive of the ususal fashion:

\begin{theorem}[Popa-Schnell '14]
If $X$ is a smooth projective variety carrying a $1$-form $\omega \in H^0(X, \Omega_X)$ with no zeros then $\kappa(X) \le n-1$. 
\end{theorem}

This shows that having a $1$-form with no zeros constrains the geometry of $X$. However, we expect there should be a much more stringent restriction on those varieties with $\kappa(X) \le n-1$ that actually do carry a $1$-form with no zeros. 
\bigskip\\
\textbf{Question:}
How do you produce $1$-forms, they always arises by pulling back along a map $f : X \to A$ (say to the Albanese). The $1$-form will have no zeros if $f$ is smooth. So we might guess that every nonvanishing $1$-form arises from the pullback along a smooth map to an abelian variety.

\begin{example}
This is not true: let $X = E \times C$ where $C$ is any curve of genus $g \ge 2$ such that $E$ is not an isogeny factor of $\Jac{C}$. Then the only smooth map to an abelian variety is $f : X \to E$. However $\pi_1 \omega_E + \pi_2 \omega_C$ are all nonvanishing for any nonzero $\omega_E$ and any $\omega_C$. The only one pulled back from $f$ are of the form $\omega_E$. Therefore, we have to be careful. It seems that having a nonvanishing $1$-form $\omega$ implied that some smooth map to an abelian variety exists but $\omega$ may not be pulled back along it. Indeed, you have to deform $\omega$ (by taking $\omega_C \to 0$ in this case) to get it as a pullback from a smooth map.
\end{example}

However, this is still not enough.

\begin{example}
Let $E_1, E_2$ are nonisogenous elliptic curves. Let $X$ be the blowup of $E_1 \times E_2 \times \P^1$ along $E_1 \times \{ 0 \} \times \{ 0 \}$ and $\{ 0 \} \times E_2 \times \{ \infty \}$. Then the pullback of $\pi_1 \omega_1 + \pi_2 \omega_2$ to $X$ has no zeros. However, there is no ``diagonal map'' to an elliptic curve since $E_1, E_2$ are not isogenous. Indeed, the only smooth maps to abelian varieties are (up to composition with an isogeny) the projections $X \to E_i$ and both are not smooth since they have reducible fiber along the exceptional. 
\end{example}

Therefore the best we could do is the following conjecture of Hao and Schreieder:

\begin{conj}[Hao-Schreieder '21, A]
Let $X$ be a smooth projective variety and $\omega \in H^0(X, \Omega_X)$ a $1$-form with no zeros. Then there is a diagram,
\begin{center}
\begin{tikzcd}
X \arrow[rr, dashed] \arrow[rd] & & X' \arrow[ld]
\\
& A
\end{tikzcd}
\end{center}
where $X \rat X'$ is a birational modification and $X' \to A$ is a smooth map to an abelian variety.
\end{conj}

Furthermore, they conjecture that when $\kappa(X) \ge 0$ we can choose $X' \to A$ to be isotrivial 

\begin{rmk}
When I say ``isotrivial'' I mean the stronger asusmption than $X \to Y$ is an analytic / \etale fiber bundle, I mean it is trivial by a \textit{finite} \etale cover $Y' \to Y$. This is always true for constant families of curves of genus $g \ge 1$ over a regular base. But it already fails for smooth conic bundles over a surface (e.g. any nontrivial Brauer class on a K3). 
\end{rmk}

\begin{conj}[Hao-Schreieder '21, B]
With the assumptions as above, if moreover, $\kappa(X) \ge 0$ then there is a diagram
\begin{center}
\begin{tikzcd}
X \arrow[rr, dashed] \arrow[rd] & & X' \arrow[ld]
\\
& A
\end{tikzcd}
\end{center}
where $X \rat X'$ is a birational modification and $X' \to A$ is a smooth isotrivial map (meaning it is an analytic fiber bundle and moreover is trivialized by an isogeny $A' \to A$).
\end{conj}

It turns out our work will also have applications to the case where, instead of asuming there is a nonvanishing $1$-form, we assume that we are given a map $f : X \to A$ that is close to smooth.

\begin{conj}[Meng-Popa '21, C]
Let $f : X \to A$ be an algebraic fiber space, with $X$ a smooth projective variety and $\kappa(X) \ge 0$ (equivalently by their work $\kappa(F) \ge 0$ for the general fiber). If $f$ is smooth away from codimension $2$ in $A$ then there there is a birational modification
\begin{center}
\begin{tikzcd}
X \arrow[rr, dashed] \arrow[rd] & & X' \arrow[ld]
\\
& A
\end{tikzcd}
\end{center}
so that $X' \to A$ is a smooth isotrivial fiber bundle. Equivalently $X \to A$ is birationally trivialized after an isogeny $A' \to A$. 
\end{conj}

From the example, we can see that we had to blow up to make a birational modification necessary. Therefore, Nathan, Hao, and I conjectured that:

\begin{conj}[Chen-C-Hao '23, D]
If $X$ has a nonvanishing $1$-form and moreover $X$ is minimal then there is a smooth isotrivial map $X \to A$.
\end{conj}


\begin{theorem}[C '24]
These conjectures hold under the assumption that $X$ admits a good minimal model (exists $X \rat X'$ such that $K_{X'}$ is semiample, in particular we must have $\kappa(X) \ge 0$). 
\end{theorem}

\begin{cor}
Suppose $\kappa(X) \ge 0$ if moveover one of 
\begin{enumerate}
\item $\dim{X} - \kappa(X) \le 4$
\item $f : X \to \Alb_X$ has generic fiber (over its image) of dimesnion $\le 3$
\end{enumerate} 
then the conjectures hold.
\end{cor}

Notice that because we had to assume $\kappa(X) \ge 0$ to get a minimal model, our theorem says nothing about Conjecture A when $X$ is uniruled. Our main technical theorem partially rectifies this issue.

\begin{theorem} \label{thm:main_MRC}
Let $X$ be a smooth projective variety equipped with a map $f : X \to A$ to an abelian variety satisfying and there are $1$-forms $\omega_1, \dots, \omega_g \in H^0(A, \Omega_A)$ such that $f^* \omega_1, \dots, f^* \omega_g$ are independent pointwise. Assume the base $Y$ of the MRC fibration $X \rat Y$ admits a good minimal model. Then there exists a quotient with connected kernel $q : A \to B$ to an abelian variety $B$ of dimension $\ge g$ and a birational map $Y \rat Z \times^G B'$ making the diagram
\begin{center}
    \begin{tikzcd}
        X \arrow[r, dashed] \arrow[rr, bend left, "f"] & Y \arrow[d, dashed] \arrow[r] & A \arrow[d, "q"]
        \\
        & Z \times^G B' \arrow[r] & B
    \end{tikzcd}
\end{center}
commute. Here, $B' \to B$ is an isogeny with kernel $G$, and $Z$ is a smooth projective variety with a $G$-action. 
\end{theorem}

\subsection{Proof of the Main Result}

There are two main ingredients in the proof. Here is a sketch of the argument:

\begin{enumerate}
\item the Iitaka fiber $F \to Y \rat S$ has image in $A$ an abelian variety $B$ of dimension $\ge g$
\item choose a good minimal model $Y'$ of $Y$ then the Iitaka fibration
\begin{center}
\begin{tikzcd}
Y' \arrow[d] \arrow[r, "f"] & B \times S \arrow[ld]
\\
S
\end{tikzcd}
\end{center}
over the locus $U \subset S$ where the fibers are at worst klt is a (weak) Calabi-Yau fibration. In this case prove that: $Y'_U \cong Z \times^G B'$ for an isogeny $B' \to B$ with kernel $G$ where $Z = f^{-1}(0) \cap Y_U'$
\item choose a $G$-equivariant smooth compactification $Z \embed \ol{Z}$ thus $Y'_U$ is an open set of $\ol{Z} \times^G B'$ which is a smooth variety with an obvious smooth isotrivial map
\[ \ol{Z} \times^G B' \to B \]
hence giving our diagram
\begin{center}
    \begin{tikzcd}
        X \arrow[r, dashed] \arrow[rr, bend left, "f"] & Y \arrow[d, dashed] \arrow[r] & A \arrow[d, "q"]
        \\
        & Z \times^G B' \arrow[r] & B
    \end{tikzcd}
\end{center}
\end{enumerate}

\subsection{Step (a)}

Suppose that $Y$ itself admits $g$ poinwise independent $1$-forms. Consider the following diagram,

\begin{center}
\begin{tikzcd}
\wt{Y} \arrow[r] \arrow[rd] & Y \arrow[d, dashed] \arrow[r] & \Alb_X \arrow[d] \arrow[r] & B
\\
& S \arrow[r, dashed] & Q
\end{tikzcd}
\end{center}
where $Q = \coker{(\Alb_F \to A)}$ (recall that since $\kappa(F) = 0$ Kawamata proves that $F \onto \Alb_F$)
since $\wt{Y} \to B$ contracts the general fiber of $\wt{Y} \to S$ by definition, rigidity shows that there is a rational map $S \rat Q$. Hence the map $Y \to Q$ factors birationally through the Iitaka fibration. Now we use the full power of Popa-Schnell

\begin{theorem}[PS '14]
Let $f : X \to A$ be a morphism from a smooth projective variety to an abelian variety. If $H^0(X, \omega_X^{\ot n} \ot f^* \L^{-1} ) \neq 0$ for some ample $\L \in \Pic{A}$ and some $n > 0$ then every $Z(f^* \omega) \neq \empty$ for all $\omega \in H^0(A, \Omega_A)$.
\end{theorem}

Let $W \subset H^0(A, \Omega_A)$ be spanned by the $\omega_1, \dots, \omega_g$. Then the above theoem shows that $W \cap H^0(Q, \Omega_Q) = \{ 0 \}$ so $\dim{Q} + g \le \dim{A}$ proving the claim.
\bigskip\\
But since $X \rat Y$ is a rational map, its' not actually clear that $\omega_1, \dots, \omega_g$ are independent everywhere on $Y$. We need a slight improvement of PS14.

\newcommand{\cN}{\mathcal{N}}

\begin{theorem} \label{thm:generalization_of_PS14}
Let $f : X \to A$ be in $\Var_A$. Consider the sheaf of $k$-forms killed by $-\wedge f^* \omega$ for all $\omega \in H^0(A, \Omega_A)$
\[ P\Omega_X^k := \ker{(\Omega_X^k \to \Omega_X^{k+1} \ot H^0(A, \Omega_A^{\vee}))}  \]
Suppose there is a line bundle $\cN \embed P\Omega_X^k$ and an ample $\L \in \Pic(A)$ so that $H^0(X, \cN^{\ot d} \ot f^* \L^{-1}) \neq 0$ for some $d \ge 1$. Then $f$ does not satisfy $(\ast)_1$ i.e.\ every $\omega \in H^0(A, \Omega_A)$ has nonempty $Z(f^* \omega) \neq \varnothing$.
\end{theorem}

I claim that $X \to Q$ will satisfy the assumption of this result. Indeed, $m^* \omega_Y \embed P \Omega^{\dim{Y}}_X$ and is positive for $Y \to Q$ because this factors through Iitaka. 


\subsection{Step (b)}

\begin{theorem} \label{thm:abelian_decomposition}
Let $g : (X, \Delta) \to S$ be a flat projective family of pairs over a locally noetherian reduced base scheme $S$ of pure characteristic zero whose fibers satisfy
\begin{enumerate}
\item $(X_s, \Delta_s)$ are klt pairs (in particular the fibers are integral with $K_{X_s} + \Delta_s$ a $\Q$-Cartier divisor) 
\item $K_{X_s} + \Delta_{X_s} \equiv_{\text{num}} 0$ 
\end{enumerate}
equipped with a surjective $S$-morphism $g : X \to \cA$ where $\cA \to S$ is a polarized abelian scheme. Let $Z = f^{-1}(0_A)$. Then there is an isogeny $\pi : \cB \to \cA$ such that in the diagram
\begin{center}
\begin{tikzcd}
Z \times_S \cB \arrow[rrd, bend left] \arrow[ddr, bend right, "\sigma"] \arrow[rd, dashed, "\tilde{\sigma}"]
\\
& X \times_{\cA} \cB \arrow[r] \arrow[d] & \cB \arrow[d, "\pi"]
\\
& X \arrow[r, "f"] & \cA 
\end{tikzcd} 
\end{center}
the unique map $\tilde{\sigma} : Z \times \cB \iso X \times_{\cA} \cB$ induced by the action is an isomorphism. Hence there is an $S$-isomorphism $X \cong Z \times^G_S \cB$ where $G = \ker{(\cB \to \cA)}$.
\end{theorem}

\end{document}
\documentclass[12pt]{article}
\usepackage{import}
\import{../}{AlgGeoCommands}

\newcommand{\dbar}{\bar{\partial}}
\newcommand{\HH}{\mathbb{H}}
\renewcommand{\gr}{\mathrm{gr}}
\newcommand{\R}{\mathrm{R}}

\DeclareMathOperator{\lct}{\mathrm{lct}}
\newcommand{\cO}{\mathcal{O}}
\renewcommand{\codim}{\mathrm{codim}}

\DeclareMathOperator{\mult}{\mathrm{mult}}
\newcommand{\LL}{\mathbb{L}}
\newcommand{\Zar}{\mathrm{Zar}}
\newcommand{\MT}{\mathrm{MT}}

\begin{document}



 
\section{Symbols of Differential Operators and Pseduodifferential Operators}

\newcommand{\cS}{\mathcal{S}}
\newcommand{\inner}[2]{\left< #1, #2 \right>}
\newcommand{\OP}{\mathrm{OP}}
\renewcommand{\Diff}{\mathrm{Diff}}


Every citation is to [1] Raymond Wells ``Differential Analysis on Complex Manifolds''

\subsection{Structures on Manifolds}

Let $K$ be a complete valued field (either $\RR$ or $\CC$ we will care about). For $D \subset K^n$ an open subset we have the following:
\begin{enumerate}
\item $K = \RR$
\begin{enumerate}
\item $\E(D)$ are the $C^\infty$-functions on $D$
\item $\cA(D)$ are the \textit{real-analytic} functions on $D$
\end{enumerate}
\item $K = \CC$
\begin{enumerate}
\item $\cO(D)$ is the complex-valued \textit{holomorphic functions} on $D$
\end{enumerate}
\end{enumerate}
In general, let $\cS$ be a subsheaf of the sheaf of continuous functions on $K^n$ in the standard topology.

\begin{defn} [1, Definition 1.1]
An $\cS$-\textit{structure} $\cS_M$ on a $K$-manifold $M$ is a subsheaf of the sheaf $\cC_M$ of $K$-valued continuous functions on $M$ such that for any chart $(U, \varphi)$ of $M$ the natural isomorphism
\[ \varphi^{\#} : \cC_{\varphi(U)} \iso \varphi_* \cC_M \]
identifies $\varphi_* \cS_M$ with $\cS_{\varphi(U)}$ defined via the open $\varphi(U) \subset K^n$.  
\end{defn}

For our three classes of functions we have defined for
\begin{enumerate}
\item $\cS = \E$ a \textit{differentiable} (or $C^\infty$) manifold and the function in $\E_M$ are called $C^{\infty}$-functions on (open subsets of) $M$
\item $\cS = \cA$ a \textit{real-analytic} (or $C^\omega$) manifold and the functions in $\cA_M$ are called \textit{real-analytic functions} on (open subsets) of $M$
\item $\cS = \cO$ a \textit{complex-analytic} (or \textit{holomorphic} or simply \textit{complex}) \textit{manifold}, and the functions in $\cO_M$ are called \textit{holomorphic} (or \textit{complex-analytic functions}) on (open subsets) of $M$.
\end{enumerate}

\begin{defn}
An $\cS$-\textit{morphism} $F : (M, \cS_M) \to (N, \cS_N)$ is a continuous map $F : M \to N$ suhc that
\[ f \in \cS_N(N) \implies f \circ F \in \cS_M(F^{-1}(U)) \]
equivalently the morphism of sheaves $F^{\#} : \cC_N \to F_* \cC_M$ induces a morphism of sheaves between the subsheaves $\cS_N$ and $F_* \cS_M$.
\end{defn}

\begin{defn}
Let $X$ be an $\cS$-manifold.
An $\cS$-structure on a topological $K$-vector bundle $\pi : E \to X$ is a $\cS$-manifold structure on $E$ such that $\pi : E \to X$ becomes an $\cS$-morphism and there exists local trivializations by $\cS$-isomorphisms. 
\end{defn}

\subsection{Sobolev Spaces}

Recall that there is a sobolev norm for compactly supported differentiable functions $f : \RR^n \to \CC^m$ defined by 
\[ || f ||_{s, \RR^n}^2 = \int | \hat{f}(y) |^2 (1 + |y|^2)^s \d{y} \]
where $\hat{f}$ is the Fourier transform
\[ \hat{f}(y) = (2 \pi)^{-n} \int e^{-i \inner{x}{y}} f(x) \d{x} \]
In $\RR^n$ this norm is equivalent to the norm
\[ \left[ \sum_{|\alpha| \le s} \int_{\RR^n} |D^\alpha f|^2 \d{x} \right]^{1/2} \]
Then we let $W(\RR^n, \CC^m)$ to be the completion of $\E(\RR^n, \CC^m)$ with respect to either norm. Note that $|| \bullet ||_s$ is defined for all $s \in \RR$ but we shall deal only with integral values in our applications.
\bigskip\\
Let $E$ be a Hermitian vector bundle on $X$. Let $\E_k(X, E)$ be the $C^k$-sections of $E$. Let $\cD(X, E) \subset \E(X, E)$ be the compactly supported section. Choosing a strictly positive smooth measure $\mu$ on $X$ (e.g. arising from a metric). Then we define an inner product on $\E(X, E)$ via
\[ \inner{\xi}{\eta} = \int_X \inner{\xi(x)}{\eta(x)}_E \d{\mu} \]
where $\inner{-}{-}_E$ is the Hermitian metric on $E$. Now we define a Sobolev norm $|| \bullet ||_s$ on $\E(X, E)$. To do this, we choose a partition of unity $\{ \rho_\alpha \}$ subordinate to a finite cover by charts $\{ (U_\alpha, \varphi_\alpha) \}$ over which $E$ has a trivialization 
\begin{center}
\begin{tikzcd}
E|_{U_\alpha} \arrow[d] \arrow[r, "\tilde{\varphi}_\alpha"] & \wt{U}_\alpha \times \CC^m \arrow[d]
\\
U_\alpha \arrow[r, "\varphi_\alpha"] & \wt{U}_\alpha
\end{tikzcd}
\end{center}
where $\varphi_\alpha : U_\alpha \to \wt{U}_\alpha \subset \RR^n$ is a diffeomorphism. 
\bigskip\\
Finally, we define, for $\xi \in \E(X, E)$
\[ || \xi ||_{s,E} = \sum_\alpha || \tilde{\varphi}_\alpha^* \rho_\alpha \xi ||_{s, \RR^n} \]
We let $W^s(X, E)$ be the completion of $\E(X, E)$ with respect to $|| \bullet ||_s$. The norm $|| \bullet ||_s$ defined on $\E(X, E)$ depends on the choice of paritions of unity, the local trivialization, and the Hermitian and metric structure. However, the topology on $W^s(X, E)$ is independent of these choices. 

\subsection{Differential Operators}

\begin{defn} [1, p. 113]
Let $E, F$ be differentiable $\CC$-vector bundles over a differentiable manifold $X$. Let
\[ L : \E(X, E) \to \E(X, F) \]
be a $\CC$-linear map. We say that $L$ is a \textit{differntial operator} if for any choice of local coordinates and local trivializations there exists a linear partial differential operator $\wt{L}$ such that the dagram
\begin{center}
\begin{tikzcd}
\E(U)^p \arrow[r, "\wt{L}"] \arrow[d, equals] & \E(U)^q \arrow[d, equals] 
\\
\E(U, U \times \CC^p) \arrow[r] & \E(U, U \times \CC^q)
\\
\E(X, E)|_U \arrow[u, hook] \arrow[r, "L"] & \E(X, F)|_U \arrow[u, hook] 
\end{tikzcd}
\end{center}
commutes. That is, for $(f_1, \dots, f_p) \in \E(U)^p$ we have
\[ \wt{L}(\ul{f})_i = \sum_{\substack{ i = 1 \\ |\alpha| \le k}}p a_\alpha^{ij} D^\alpha f_j \]
A differential operator is said to be of \textit{order} $k$ if $\wt{L}$ can be taken to be of the above form. 
\end{defn}

\begin{defn}
Suppose $X$ is a compact $C^\infty$-manifold. We define $\OP_k(E, F)$ as the the vector space of $\CC$-linear mappings
\[ T : \E(X, E) \to \E(X, F) \]
such that there is a continuous extension of $T$
\[ \ol{T}_s : W^s(X, W) \to W^{s-k}(X, F) \]
for all $s$. These are the \textit{operators of order} $k$ mapping $E$ to $F$. 
\end{defn}

\begin{prop} [1, Prop. 2.1]
Let $L \in \OP_k(E, F)$. Then $L^*$ exists and moreover $L^* \in \OP_k(F, E)$ and the extension
\[ (\ol{L}^*)_s : W^s(X, F) \to W^{s-k}(X, E) \]
is given by the adjoint map
\[ (\ol{L}_{k-s})^* : W^s(X, F) \to W^{s-k}(X, E) \]
\end{prop}
 
\begin{prop} [1, Prop. 2.2]
$\Diff_k(E, F) \subset \OP_k(E, F)$
\end{prop}

\begin{proof}
This is a local calculation and we use
\[ \wh{D^\alpha f}(\xi) = \xi^\alpha \hat{f}(\xi) \]
\end{proof}

\subsection{Symbol}

\newcommand{\Smbl}{\mathrm{Smbl}}

We review how [1] defines the symbol. Let $U \subset T^* X$ be the complement of the zero section and $\pi : U \to X$ the projection. For $k \in \Z$ we define
\[ \Smbl_k(E, F) = \{ \sigma \in \Hom{U}{\pi^* E}{\pi^* F} \mid \forall \rho > 0, (x,v) \in U : \sigma(x, \rho v) = \rho^k \sigma(x,v) \} \]
Then we define a linear map
\[ \sigma_k : \Diff_k(E, F) \to \Smbl_k(E, F) \]
where $\sigma_k(L)$ is called the $k$-\textit{symbol} of the differential operator $L$. For $(x,v) \in U$ we define a linear map
\[ \sigma_k(L)(x,v) : E_x \to F_x \]
as follows: let $e \in E_x$ be given and choose $g \in \E(X)$ and $f \in \E(X,E)$ such that $\d{g}_x = v$ and $f(x) = e$ then we define
\[ \sigma_k(L)(x,v) e = L \left( \frac{i^k}{k!} (g - g(x))^k f \right)(x) \in F_x \]

\begin{prop}
The symbol map $\sigma_k$ gives rise to an exact sequence
\[ 0 \to \Diff_{k-1}(E, F) \to \Diff_k(E, F) \xrightarrow{\sigma_k} \Smbl_k(E, F) \]
\end{prop}

\subsection{Algebraic Symbols}

In the algebraic category, a map $\varphi : \pi^* F \to \pi^* G$ extends to a map over all of $T^* X$ as long as $\dim{X} \ge 2$ since the zero section has codimension $\dim{X}$. Therefore, the data of $\varphi$ is equivalent to the data of 
\[ \varphi : F \to \pi_* \pi^* G = 
\begin{cases}
G \ot \bigoplus_{n \ge 0} \nSym{n}{T_X} & \dim{X} \ge 2
\\
G \ot \bigoplus_{n \in \ZZ} T_X^{\ot n} & \dim{X} = 1
\end{cases} \]
and the $k$-symbols are those maps that are homogeneous of degree $k$ i.e.
\[ \Smbl_k(E, F) = \Hom{}{E}{F \ot \nSym{k}{T_X}} \]
We can describe the symbol map as follows. The jet bundle or bundle of principal parts (or total symbols) satisfies
\[ \Diff_k(E, F) = \Hom{}{J^k(E)}{F} \]
and there is an exact sequence
\[ 0 \to E \ot \nSym{k}{\Omega_X} \to J^k(E) \to J^{k-1}(E) \to 0 \]
and therefore applying $\shHom{}{-}{F}$ we get and exact sequence of sheaves
\[ 0 \to \Diff_{k-1}(E, F) \to \Diff_k(E, F) \xrightarrow{\sigma_k} \Hom{}{E}{F \ot \nSym{k}{T_X}} \to 0 \]
the last map is identified with the symbol map. 

\subsection{Pseudodifferential Operators}

\begin{defn} [1, Def. 3.7]
A linear map $L : \cD(X, E) \to \E(X, E)$ is a \textit{pseduodifferential operator} on $X$ if for any coordinate chart $(U, \varphi)$ trivializing $E$ and $F$ and any open $U' \subset U$ with compact closure there is a $r \times p$ matrix $p^{ij} \in S^m_0(U)$ so that the induced 
\[ L_U : \cD(U')^p \to \E(U)^r \]
via extending by zero to $U$ and applying $L$ is a matrix of canonical pseudodifferential operators d
\end{defn}

\section{Pre-Talk for Helene}

\subsection{The Moduli Spaces}

\newcommand{\all}{\mathrm{all}}

Given a finitely presented group $\pi$ we consider the functor sending a ring $A$ to representations valued in $A$,
\[ \Rep_{\pi, r} : A \mapsto \{ \rho : \pi \to \GL_r(A) \} / \text{conjugation}. \]
This is not quite representable. Indeed, it is not even an \etale sheaf. 

\begin{example}
Suppose $\pi$ is finite and $K$ has characteristic zero. Then $M(\pi, r)$ satisfies the sheaf condition for $L/K$ exactly if all dimension $r$ representations over $L$ with traces in $K$ are defined over $K$. For example let $\pi = Q_8$ and consider the representation
\[ Q_8 \to \GL_2(\CC) \]
using the standard representation via Pauli matrices. It is a standard that
\[ \sigma_i \sigma_j = - \delta_{ij} I + \epsilon_{ijk} \sigma_k \]
So 
\[ \tr{\sigma_i} = 0 \quad \quad \tr{\sigma_i \sigma_j} = -2 \delta_{ij} \]
hence the traces are all real. However, there are not enough independent order four elements in $\GL_2(\RR)$ for this to descend. 
\end{example}

Let's first consider a framed version
\[ M^{\square}(\pi, r) : A \mapsto \{ \rho : \pi \to \GL_r(A) \} \]
This is clearly represented by an affine scheme (inside $\A^{nr^2}$ where $n$ is the number of generators and impose $\det \neq 0$ and the finite number of relations). Now we can form a stack
\[ [M^{\square}(\pi, r) / \GL_r] \]
which represents the groupoid version
\[ [M^{\square}(\pi, r) / \GL_r] : A \mapsto [\{ \rho : \pi \to \GL_r(A) \} / \text{conjugation}.] \]
by definition. To get the isomorphism classes, we use GIT to form a coarse space
\[ M^{\all}(\pi, r) := M^{\square}(\pi, r) // \GL_r \]
From the perspective of GIT stability conditions:
\begin{enumerate}
\item completely reducible (i.e. semisimple) $\iff$ polystable
\item irreducible $\implies$ stable (and usually the converse)
\end{enumerate}
Recall that $\varphi : X \to X // G$ identifies two points iff they have the same orbit closures and there is a unique polystable point in each fiber. Hence $M(\pi, r)$ ``parametrizes semisimple representations''. In fact, we can identify 
\[ M^{\sqcup}(\pi, r) \to M^{\all}(\pi, r) \]
on $\bar{k}$-points with the semisimplification. 

\subsection{Irreducibility}

\newcommand{\irr}{\mathrm{irr}}

We can form two subschemes of $M^{\all}(\pi, r)$. The first is functorial
\[ M^{\irr}(\pi, r) \subset M^{\all}(\pi, r) \]
which is the open determined by the open of $M^{\sqcup}(\pi, r)$ of \textit{absolutely irreducible} representations meaning $\pi : \pi \to \GL_r(A)$ such that for all geometric points $A \to \bar{k}$ the representation $\rho : \pi \to \GL_r(\bar{k})$ is irreducible. 
\bigskip\\
This will be too limiting for us. Instead, we consider $M^{\mathrm{gen-irr}}(\pi, r)$ to be the closure of $M^{\irr}(\pi, r)(\CC)$ inside $M^{\all}(\pi, r)$ which we write as $M(\pi, r)$. This is the natural space to work in if we want to consider only representations that deform to a an irreducible representation over characteristic zero. 

\subsection{Specialization and Tame Fundamental Groups}

\begin{theorem}
If $\pi = \pi_1(X)$ for $X$ a quasi-projective variety then $\epsilon : M \to \Spec{\Z}$ is surjective if and only if it is dominant.
\end{theorem}

\begin{rmk}
To make this true we need $M = M(\pi, r, \delta)$ to modify slightly our definitions to involve only representations such that $\det{\rho}^\delta = 1$. This technical condition will just come along for the ride at almost every step of the proof.
\end{rmk}

\begin{rmk}
Note the reason we passed to $M$ is so that each component hits $\Spec{\QQ}$ by definition. Therefore the above statemnt is equivalent to saying 
\[ M(\CC) \neq \empty \iff \forall \ell : M(\ol{\ZZ}_{\ell}) \neq \empty \]
In fact, Helene proves more: that these $\ol{\ZZ}_{\ell}$-points can be chosen to pass through $M^{\irr}_{\Q}$.
\end{rmk}

\begin{rmk}
In fact, this theorem is an obstruction to groups arising from geometry since not every character variety satisfies this property. For example, the groups
\[ \Gamma_{\ell} = \langle a,b\ \vert\ a^{\ell(\ell-1)}ba^{-\ell}b^{-2}\rangle \]
has structure map $\epsilon : M \to \Spec{\Z}$
with image $\Spec{\Z} \sm \{ \Spec{\FF_\ell} \}$.
\end{rmk}

How are we going to prove this? We are going to think about representations that factor as 
\begin{center}
\begin{tikzcd}
\pi \arrow[d] \arrow[r, "\rho"] & \GL_r(\CC) \arrow[d, "\tau"]
\\
\hat{\pi} \arrow[r, "\text{cont.}"] & \GL_r(\ol{\Q}_\ell)
\end{tikzcd}
\end{center}
If this exists then by continuity and compactness of $\hat{\pi}$, up to conjugation, $\hat{\pi} \to \GL_r(\ol{\Q}_\ell)$ lands in $\GL_r(\ol{\Z}_\ell)$ so we are done. However, the representation we started with probably does not fit into such a diagram. The game will be to ``approximate'' $\rho$ by -- for each $\ell$ -- a representation of the above form.
\bigskip\\
For $\ell \gg 0$ it turns out this is easy just by generic smoothness of $\epsilon : M \to \Spec{\ZZ}$. To get the other primes, we need some technology: companions for arithmetic representations. This technology is for representations of the fundamental group of a variety over $\FF_p$. Since $\hat{\pi}_1 = \pi_1^{\et}(X_{\CC})$ we can spread out $X$ over characteristic $p$ and use Grothendieck specialization maps to obtain a representation of a variety over $\FF_p$. However, the specialization map only exists when $X$ is proper. To handle the quasi-projective case, we need the tame fundamental group.

\subsection{Tame Fundamental Groups} 

\begin{defn}
Let $X$ be a smooth projective variety. A \textit{good compactification} $X \embed \ol{X}$ is an open immersion into a smooth projective variety such that $\ol{X} \sm X$ is an SNC divisor. 
\end{defn}

\begin{defn}
A finite \etale cover $Y \to X$ is \textit{tame} if it extends to $\ol{Y} \to \ol{X}$ over a good compactification $X \embed \ol{X}$ that is tamely ramified along the boundary.
\end{defn}

\begin{defn}
A finite map $Y \to X$ is tamely ramified if for any map $\pi : C \to X$ from a curve the associated map $C \times_X Y \to C$ is tamely ramified in the sense that the residue field extensions are separable and the ramification indices are prime to the characteristic. 
\end{defn}

\begin{defn}
We define $\pi_1^t(X, x)$ exactly as $\pi_1^{\et}(X, x)$ but replacing fintie \etale covers by tame finite \etale covers. 
\end{defn}


\begin{rmk}
There are surjections
\[ \pi_1^{\et}(X) \onto \pi_1^t(X) \onto \pi_1^{\et}(X)^{(p')} \]
where the last is the prime-to-$p$ completion.
\end{rmk}

\begin{theorem}[SGA1, Ex. XIII 2.10, Corollarie 2.12]
Let $S$ be a finite type scheme over $\Z$ and $X \to S$ a smooth morphism. Let $X \embed \ol{X}$ be a good compactification \textit{relative over} $S$ and $x_S$ an $S$-point of $X$. Then for $s \spto s'$ there is a specialization map
\[ \sp : \pi_1^t(X_{\bar{s}}, x_{\bar{s}}) \to \pi_1^t(X_{\bar{s}'}, x_{\bar{s}'}) \]
which is continuous and surjective.
\end{theorem}

\subsection{Fixing Monodromy at Infinity}

We now introduce a refinement of $M(\pi, r, \delta)$ which, unlike the others, does not just depend on $\pi$ but also on the geometry of $X \embed \ol{X}$.  Choose loops $\gamma_1, \dots, \gamma_n$ spanning the kernel of $\pi_1(X) \to \pi_1(\ol{X})$ then for some representation we call $\rho(\gamma_i)$ the ``monodromy at infinity''. Let $M^{\all}(\pi, r, \delta, \lambda_{ij})$ be the space of representations such that:
\begin{enumerate}
\item $\rho(\gamma_i)$ is quasi-unipotent (some power is unipotent)
\item its eigenvalues are $\lambda_{i1}, \dots, \lambda_{ir}$
\end{enumerate}
Note this only fixes (up to conjugacy) the semisimplification of $\rho(\gamma_i)$. As before we have subscheme $M^{\irr}(\pi, r, \delta, \lambda_{ij})$ and $M = M^{\text{gen-irr}}$. 

\begin{rmk}
Because it depends on the choice of loops $\gamma_1, \dots, \gamma_n$ at infinity, we cannot make sense of $M(\pi, r, \delta, \lambda_{ij})$ for an abstract group $\pi$ so it doesn't make sense to say that its properties could be obstructions to a group being geometric. 
\end{rmk}

\subsection{Arithmetic Representations}

\subsection{Companions}

\subsection{Local Structure: de Jong's Theorem}

Drinfelds solution to de Jong's conjecture allows us to approximate by arithmetic representations of $\pi_1^t(X_{\ol{\FF}_p})$ for $p \gg 0$. 

\section{Helene}

\subsection{Preliminaries}

Let $B$ be an effective divisor on a projective smooth variety $Y$ over $\CC$. We set
\[ B = \sum v_j E_j \]
where $E_j$ are the irreducible components of $B$. If we suppose that the divisors is assoicated to a positive power $\L^{\ot d}$ of a line bundle $\L$ meaning $\L^d = \struct{Y}(\sum v_j E_j)$ we write for $i \ge 0$
\[ \L^{(i)} := \L^i \ot \struct{Y}(-\sum \floor{ v_j \cdot i \cdot d^{-1}} \cdot E_j) \]
If $0 \le i < d$ the definition of $\L^{(i)}$ involves those $E_j$ for whcich $v_j \ge 2$. When we want to highlight the role of the reduced diviso $D$ of $B$ we write
\[ B = D + \sum_{v_j \ge 2} v_j \cdot E_j \text{ or } D = \sum_{v_j = 1} v_j \cdot E_j \]
We suppose that the divisor $B$ is strict normal crossings (SNC). The section $s$ of $\L^{\ot d}$ with support $B$ then defines a sheaf of $\struct{Y}$-modules
\[ \cA = \bigoplus_{i = 0}^{d-1} \L^{-\ot i} \]
which the strucrure of an algebra. The multiplication is defined by
\[ \L^{-i} \oplus \L^{-j} \to \L^{-i} \ot \L^{-j} \to \L^{-i -j} \]
and we identify $\L^{-d} \embed \struct{Y}$ by the dual of $s$. Let $W$ denote the normalization of $\rSpec{Y}{\cA}$ and $V \to W$ a resolution of singularities so we get a diagram
\begin{center}
\begin{tikzcd}
V \arrow[rd, "f"] \arrow[r, "g"] & W \arrow[d, "\tau"]
\\
& Y
\end{tikzcd}
\end{center}
The variety $W$ is called the $d$-th root of the divisor $B$. 

\begin{lemma}
$W$ has rational singularities. In particular $W$ is Cohen-Macaulay and $\tau$ is flat. Furthermore
\[ \tau_* \struct{W} = \bigoplus_{i = 0}^{d-1} (\L^{(i)})^{-1} \quad \R f_* \struct{V} = \bigoplus_{i = 0}^{d-1} (\L^{(i)})^{-1} [0] \]
\end{lemma}

\begin{theorem}
Let $Y, B, \L$ be as above and let $\omega_Y$ be the canonical bundle. Suppose that the Kodiara dimension of $\L$ satisfies $\kappa(\L) = \dim{Y} = n$ and $\L$ is is generated by global sections. Then
\[ H^q(Y, \omega_Y \ot \L^{(i)} \ot \L^k) = 0 \]
for all $k > 0, q > 0$ and $i \ge 0$.
\end{theorem}

\begin{proof}
Serre duality and Kawamata-Vieweg vanishing for $V$.
\end{proof}

\begin{theorem}
Let $Y,B,\L$ as above. Suppose that $\kappa(\struct{Y}(D)) = \dim{Y}$. Then
\[ H^q(Y, \omega_Y \ot \L^{(i)}) = 0\]
for all $q > 0$ and $d > i > 0$.
\end{theorem}

\begin{proof}
The proof is based on three facts:
\begin{enumerate}
\item the caluation of $\L^{(i)}$ in (2.2)
\item the symmetry of Hodge numbers on $V$
\item the formation of differential forms with logarithmc poles on a divisor with normal crossings.
\end{enumerate}
\end{proof}

\subsection{•}

Let $X^0$ be quasi-projective smooth subvariety of dimension $m \ge 1$ in $\P^n$. Let $Z$ be the closure and $\pi : X \to Z$ a birational map such that $X$ is smooth projective. In the proof of Theorem I, we construct sections of certain line bundles on $X$ which we want to identify with the restriction to $X^0$ of polynomial functions on $\P^n$. We do this using the ``section hunting'' proposition as follows. Let $X'$ be the normalization of $Z$ and write
\begin{center}
\begin{tikzcd}
X \arrow[rd, "\pi"] \arrow[d]
\\
X' \arrow[r, "\pi'"] & Z \arrow[r, hook] & \P^n
\end{tikzcd}
\end{center}
for the corresponding maps. Let $U$ be the smooth locus of $X'$. For any variety $U'$ with a mrophism $\varphi : U' \to \P^n$ we set $\struct{U'}(1) = \varphi^* \struct{\P^n}(1)$. For any $\ell$< call $\theta_\ell$ the composition of the canonical maps
\[ H^0(\P^n, \struct{\P^n}(\ell)) \to H^0(Z, \struct{Z}(\ell)) \to H^0(X', \struct{X'}(\ell)) \to H^0(U, \struct{U}(\ell)) \]

\begin{prop}
There is an injection
\[ j : \omega_U \to \struct{U}(\deg{X^0} - m - 2) \]
such that for all $k$, the image under $j$ of $H^0(U, \omega_U \ot \struct{U}(k))$ inside $H^0(U, \struct{U}(\deg{X^0} - m - 2 +k ))$ is contaied in the image of $\theta_{\deg{X^0} - m - 2 + k'}$. 
\end{prop}

\subsection{Proof of Theorem I}

We consider the situation of Part 1. Let $X^0$ be smooth and quasi-projective of dimension $m$ and $Z$ the closure. Let $\{ X_j^0 \}$ be (integral) subvarities of $X^0$ of dimensions $n_j$ and $Z_j$ the closures inside $Z$. 
\par 
We write $X'$ for the normalization of $Z$ {\color{red} (corrected from $X$)}. Then choose a resolution $\pi : X \to X'$. 
\par 
We construct a desingularization of the divisor $V(s)$ associated to the section $s$ in Theorem I. 

\section{Ample Line bundles on Abelian Varities}

The following is used implicitly in PS14.

\newcommand{\Td}{\mathrm{Td}}

\begin{prop}
Let $A$ be an abelian variety over a field of characteristic $p > \dim{X}$ and $\L$ an ample line bundle. Then $H^0(A, \L) \neq 0$.
\end{prop}

\begin{proof}
Note that $\Td_A = 1$ so by Grothendieck-Riemann-Roch
\[ \chi(\L) = \int_A \mathrm{\ch}(\L) = \frac{1}{n!} c_1(\L)^n > 0 \]
Note this also shows that $\deg_L(A)$ is divisible by $n!$ (hence also the degree of any embedding $A \embed \P^N$). Now since $A$ lifts over $W_2(k)$ (see Mumford's book ) and $p > \dim{A}$ Deligne-Illusie applies to get Kodaira vanishing (note we only need to lift $A$ not $\L$ to apply Deligne-Illusie) so 
\[ H^{>0}(A, \L) = H^{>0}(A, \L \ot \omega_A) = 0 \]
and hence
\[ H^0(X, \L) = \frac{1}{n!} c_1(\L)^n > 0 \]
\end{proof}



\section{Cubic Fourfolds}

$X \subset \P^5$ a smooth cubic four-fold. First we consider the Hodge diamond. By Lefschetz we just need to understand the middle row. 
\begin{enumerate}
\item $H^4(X, \struct{X}) = H^4(X, \omega_X(3)) = 0$ by Kodaira vanishing
\item for $H^3(X, \Omega_X^1)$ we use
\[ 0 \to \struct{X}(-3) \to \Omega^1_{\P^5}|_X \to \Omega_X^1 \to 0 \]
so by Kodaira vanishing we get
\[ H^3(X, \Omega_X^1) \iso H^4(X, \struct{X}(-3)) = H^4(X, \omega_X) = \CC \]
\item $\chi_{\text{top}}(X) = \deg c_4(\T_X)$. Thus we get
\[ h^{22} + 6 = \deg c_4 \]
and we use the SES
\[ 0 \to \T_X \to \T_{\P^5}|_X \to \struct{}(3) \to 0 \]
and hence
\[ c(\T_X) = c(\T_{\P^5}) / (1 + 3 H) = \frac{(1 + H)^6}{1 + 3 H} = 1 + 3 H + 6 H^2 + 2 H^3 + 9 H^4 \]
and $\deg{H^4} = 3$ so $\deg{c_4} = 27$ and thus $h^{22} = 21$. 
\end{enumerate}
The question is: when is $X$ rational? For $X_d \subset \P^{n+1}$ surface if $d = 1,2$ then it is rational. Therefore, $d = 3$ is the first interesting case. 
\begin{enumerate}
\item if $X_3$ is a curve it has genus $1$ so is not rational
\item if $X_3$ is a surface then it is rational
\item if $X_3$ is a 3-fold it is not rational (Clemens-Griffiths)
\item if $X_3$ is a 4-fold ... well this is interesting
\item if $X_3$ has $\dim{X_3} > 5$ or something it is rational
\end{enumerate}

\begin{example}
Fix two planes:
\[ P_1 = \{ u = v = w = 0 \} \quad P_2 = \{ x = y = z = 0 \} \]
in $\P^5$ and let $X$ be a cubic 4-fold containing $P_1, P_2$. Consider
\[ \varphi : P_1 \times \P^2 \rat X \]
given by
\[ (p, q) \mapsto (\ell_{p,q} \cap X) \sm \{ p, q \} \]
there is a unique extra intersection point since the line intersects in three points. More precisely, $\varphi$ is defined outside the locus at which $\ell_{p,q} \subset X$ which is a surface. We can always write $X = V(F_1 + F_2)$ where $F_1$ has bidegree $(2,1)$ and $F_2$ has bidegree $(1,2)$ (wrt the variables $x,y,z$ and $u,v,w$) usually there is a bidegree $(0,3)$ and $(3,0)$ part but these are zero if it contains the planes. Then the non-defined locus $S$ is a K3 surfaces $V(F_1, F_2) \subset P_1 \times P_2$.
\end{example}

\begin{example}
Suppose $X$ contains a plane then there is a map
\[ q : \Bl_P(X) \to \P^2 \]
projecting away from the plane. The fibers are quadric surfaces (these are the residuals of the intersection of a $3$-space containing $P$ with $X$). Then $X$ is rational if $q$ admits a rational section since then it is birational to $\P^2 \times \P^1 \times \P^1$. 
\end{example}

Consider $F_1(q)$ be the relative Fano scheme of lines for the map $q$. This is a fibration over $\P^2$. The map
\[ F_1(q) \to \P^2 \]
has general fiber a disjoint union of two lines. The stein factorization
\[ F_1(q) \to S \to \P^2 \]
gives a degree $2$ cover $S \to \P^2$ branched over a sextic and $S$ is a K3 surfaces. And $F_1(q) \to S$ is a smooth conic bundle. 

\begin{prop}
$q$ admits a rational section iff $r : F_1(q) \to S$ admits a rational section (i.e. it is a trivial Brauer class on $S$). 
\end{prop} 

\begin{defn}
A polarized K3 surface $(X, L)$ is associated with $X$ if there exists a surface $T$ on $X$ non-homologous to a complete intersection such that $\left< h^2, T \right>^{\perp} \subset H^4(X, \ZZ)$ is isomorphic to $\left< L \right>^\perp \subset H^2(S, \Z)(-1)$.
\end{defn}

\begin{conj}
Let $X$ be a cubic 4-fold. Then $X$ is rational iff it admits an associated K3 surface. 
\end{conj}

It is known that admitting an associated K3 surface is equivalent to $F_1(X)$ being birational to a Moduli space of stable sheaves on a K3. 

\section{Twisted Intermediate Jacobian Fibrations}

\newcommand{\cY}{\mathcal{Y}}

Setup: $X \subset \P^5$ smooth cubic $4$-fold. Let $B := \{ [H] \mid H \subset \P^5 \} \cong (\P^5)^\vee$. Then we get a fibration
\[ p : \cY \to B \]
whose fibers are $X \cap H_b$ for $b \in B$ called the universal hyperplane section. Recall: cohomology of the generic fiber which is a cubic $3$-fold
\[ H^i(Y, \Q) \cong \begin{cases}
\Q & i = 0,2,6
\\
\Q^{\oplus 10} & i = 4
\\
0 & i = \text{odd}
\end{cases} \]
The intermediate Jacobian associated a smooth cubic $3$-fold $Y$ is given by the Hodge filtration
\[ H^3(Y, \CC) \supset F^1 \supset F^2 \supset 0 \]
then we define
\[ J(Y) = \frac{(F^2 H^3)^\vee}{H^3(Y, \ZZ)} \]
is a ppav of dimension $5$. Goal to do this for the family $p : \cY \to B$. What about the singular fibers? 
\par 
Proposed candiate: $(R^2 p_* \Omega_{\cY}^1) / R^3 p_* \Z_{\cY}$

\begin{rmk}
Note that $R^2 p_* \Omega^1_{\cY/B} = R^2 p_* \Omega^1_{\cY}$ because $R^1 p_* \struct{\cY} = 0$ for all $i > 0$. 
\end{rmk} 

\subsection{•}

Over the smooth locus $U \subset B$ of $p$ we have a VHS 
\[ (\Lambda_U := R^3 p_* \Z_{\cY_U}, \Lambda_{\CC}, F^\bullet \Lambda_{\CC}) \]
and so we can associate an intermediate Jacobian
\[ J(\Lambda_U) := \frac{(F^2 \Lambda_{\CC})^\vee}{R^3 p_* \Z_{\cY}} \cong \frac{R^2 p_* \Omega_{\cY}}{R^3 p_* \Z_{\cY}} \]

\begin{prop}
The injection
\[ \Lambda \to \Lambda_{\CC} \to (F^2 \Lambda)^\vee \]
 extends to an injection
 \[ \Lambda := R^3 p_* \Z_{\cY} \to R^2 p_* \Omega^1_{\cY} \]
\end{prop}

\begin{rmk}
$R^2 p_* \Omega^1_{\cY}$ is a locally free sheaf isomorphic to $\Omega^1_B$.
\end{rmk}

\begin{proof}
Consider the exponential sequence
\[ 0 \to \Z_{\cY} \to \struct{\cY} \to \struct{\cY}^\times \to 0 \]
and we get
\[ R^3 p_* \Z_{\cY} \cong R^2 p_* \struct{\cY}^\times \xrightarrow{\d{\log}} R^2 p_* \Omega_{\cY}^1 \]
Step 2: need to show $R^3 p_* \Z_{\cY}$ is an irreducible sheaf. 
\end{proof}

Therefore we can define
\[ J := \frac{R^2 p_* \Omega^1_{\cY}}{R^3 p_* \Z_{\cY}} \]
is an abelian sheaf on $B$. 

\subsection{Hodge Modules}

Schnell: complex analytic Neron model. Recall, given a VHS of weight $2k + 1$ and level $1$ (meaning there is only two steps in the Hodge filtration $\Lambda \supsetneq F^k \supsetneq F^{k+1} \supsetneq 0$). Call it $(\Lambda, \Lambda_{\CC}, F^\bullet \Lambda_{\CC})$. On $U \subset B$ we have $(\Lambda_U, F^\bullet \Lambda)$ a VHS.
\[ J(\Lambda_U) = \frac{(F^{k+1} \Lambda_{\CC})^{\vee}}{\Lambda_U} \]
On $B$, let $\M$ be the minimal extension of $\Lambda_{\CC}$ as a Hodge module 
\[ J(\M) := \frac{(F_{-k-1} \M)^{\vee}}{j_* \Lambda_U} \]
Schnell shows:
\begin{enumerate}
\item total space is Hausdorff
\item its formation commutes with smooth base change $B' \to B$
\item Extends admissible normal functions without singularities w/o singularities (??) 
\end{enumerate}

\begin{prop}
Back to our VHS $(\Lambda_U, F^\bullet \Lambda_{\CC}$ 
\begin{enumerate}
\item $(F_{-k-1} \M)^\vee \cong R^2 p_* \Omega_{\cY}^1$
\item $j_* \Lambda_U \cong \Lambda \cong R^3 p_* \Z_{\cY}$
\end{enumerate}
\end{prop}

\newcommand{\DR}{\mathrm{DR}}

\begin{proof}
Main input: decomposition theorem
\[ \R p_* \Q_{\cY}[8] = \Q_B [5][3] \oplus \Q_B[5][1] \oplus \R^3 p_* \Z_{\cY}[5] \oplus \Q_B[5][-1] \oplus \Q_B[5][-3] \oplus K \]
we show that $K = 0$. Upshot $IC(\Lambda_U) = \Lambda[5]$. Moreover we get the Hodge-Module theoretic decomposition theorem
\[ p_+ \struct{\cY} = \struct{B}[3] \oplus \struct{B}(-1) [1] \oplus \M \oplus \struct{B}(-2)[-1] \oplus \struct{B}(-3)[-2] \]
Saito: 
\[ \gr^F_{-k} \DR(p_+ \struct{\cY}) \cong \R p_* \gr^F_{-k} \DR(\struct{\cY}) \]
For $k = 1$ we get
\[ \gr_{-1}^F \DR(\struct{\cY}) = \Omega_{\cY}^1[7] \]
and therefore by Saito
\[ \R p_* \Omega_{\cY}^1[7] \cong \Omega_B^1[7] \oplus \struct{B}[6] \oplus \gr^F_{-1} \DR(\M) \]
therefore
\[ \gr^F_{-1} \DR(\M) \cong R^2 p_* \Omega^1_{\cY} \]
\end{proof}



\section{Hodge Index Theorem}

\begin{lemma}
Let $X$ be a smooth projective surface and $D$ a divisor such that $D^2 > 0$. Then either $h^0(X, \struct{X}(nD)) \to \infty$ or $h^0(X, \struct{X}(-nD)) \to \infty$.
\end{lemma}

\begin{proof}
Note that we cannot have $h^0(X, \struct{X}(nD)) \neq 0$ and $h^0(X, \struct{X}(-nD)) \neq 0$ simultaneously unless $D \sim 0$ and we have $D^2 > 0$. Hence these are mutually exclusive. Using
\[ \chi(\struct{X}(nD)) = \tfrac{1}{2} (n^2 D^2 - n D \cdot K_X) + \chi(\struct{X}) \] 
we see that $\chi(\struct{X}(nD)) \to \infty$ since $D^2 > 0$. Thus either $h^0(X, \struct{X}(nD)) \to \infty$ or $h^2(X \struct{X}(nD)) = h^0(X, \struct{X}(K_X - nD)) \to \infty$. Thus, in the second case we just need to show that $h^0(X, \struct{X}(-nD)) \to \infty$. Indeed, $h^0(X, \struct{X}(-nD)) - h^0(X, \struct{X}(K_X - nD))$ is controlled by cohomology on curves and hence grows linearly in $n$ not quadratically. 
\end{proof}

\begin{theorem}[Hodge Index]
Let $X$ be a smooth projective surface and $L$ an ample class. For any divisor, if $L \cdot D = 0$ then $D^2 \le 0$.
\end{theorem}

\begin{proof}
Assume $D^2 > 0$ the lemma shows that ether $nD$ or $-nD$ is effective for $n \gg 0$ and hence either $L \cdot D > 0$ or $L \cdot D < 0$. We don't need to do the last step of the lemma. If $nD$ is not effective it gives $K_X - nD$ is effective for $n \gg 0$ so $K_X \cdot L - n D \cdot L > 0$. But the fact that the number of sections is going to $\infty$ means it cannot have bounded degree against $L$ so $D \cdot L < 0$. 
\end{proof}

\begin{corollary}
Let $N_1(X) = \Pic{X} / \sim_{\text{num}}$. This is a finite free $\Z$-module. On $N_1(X)$ the intersection product on $N_1(X)^{\perp L}$ is negative-definite.
\end{corollary}

\begin{proof}
Indeed, we have shown that $D^2 \le 0$ for all $D \in N_1(X)^{\perp L}$. Furthermore, by definition the pairing is nondegenerate. If $D^2 = 0$ suppose $D \cdot E \neq 0$ where we can assume $E \cdot L = 0$ since $D \cdot L = 0$. Then $(D + \lambda E)^2 = D^2 + 2 \lambda D \cdot E + \lambda^2 E^2 \le 0$ for all $\lambda$. But $D^2 = 0$ so this is impossible for small $\lambda$ unless $D \cdot E = 0$. Hence $D^2 = 0$ implies $D \sim_{\text{num}} 0$. 
\end{proof}

\begin{corollary}
If $L$ is nef then $L^2 D^2 \le (L \cdot D)^2$ with equality iff $D \sim_{\text{num}} \alpha L$
\end{corollary}

\begin{proof}
Taking limits, we may assume that $L$ is ample. Write $D = \alpha L + D^\perp$ then $L \cdot D = \alpha L^2$ and $D^2 = \alpha^2 L^2 + (D^{\perp})^2 \le \alpha L^2$. Hence 
\[ L^2 D^2 = \alpha^2 (L^2)^2 +  L^2 (D^{\perp})^2 \le \alpha^2 (L^2)^2 = (L \cdot D)^2 \]
which equality iff $(D^{\perp})^2 = 0$ iff $D^{\perp} \sim_{\text{num}} 0$ since the paring is negative-definite.
\end{proof}

\section{Intersection and Linking Form}

\subsection{Linking Form}

Using Poincare duality $H_i(M) \cong H^{n-i}(M)$ we always have a intersection pairing by taking the dual of cup product:
\[ H_{n-i}(M) \ot H_{n-j}(M) \to H_{n-(i+j)}(M) \]
Let $M$ be an oriented $2k$-manifold. Let $i = j = k$ then we get a map
\[ H_{k}(M) \ot H_{k}(M) \to H_0(M) \iso \Z \]
which is the intersection pairing. If $M$ is unoriented, then we still get a pairing on $H_k(M, \Z/2\Z)$.  
\bigskip\\
What about for $2k+1$-manifold? Let $M$ be oriented. Then consider $i = k + 1$ and $j = k$ we get a pairing
\[ H_k(M) \ot H_{k+1}(M) \to H_0(M) \iso \Z \]
We can also do the following: consider 
\[ H^k(M, \Q / \Z) \ot H^k(M, \Q / \Z) \to \Q / \Z \]
as follows, first apply the Bockstein for 
\[ 0 \to \Z \to \Q \to \Q / \Z \to 0 \]
to get
\[ \beta : H^k(M, \Q/ \Z) \to H^{k+1}(M, \Z) \]
and then apply cup product to obtain
\[ H^k(M, \Q / \Z) \ot H^k(M, \Q / \Z) \xrightarrow{\id \ot \beta} H^k(M, \Q / \Z) \ot H^{k+1}(M, \Z) \xrightarrow{\smile} H^{2k+1}(M, \Q / \Z) \iso \Q / \Z \]
where the last uses the universal coefficient theorem. Indeed, the universal coefficient theorem gives a sequence
\[ 0 \to \Ext{1}{\Z}{H_{i-1}(M)}{\Q/\Z} \to H^{i}(M, \Q/\Z) \to \Hom{\Z}{H_i(M)}{\Q/\Z} \to 0 \]
Hence, if $H_{i-1}(M)$ is torsion-free (which e.g. is true for $i = n$ or $i = 1$) then
\[ H^i(M, \Q/\Z) \iso \Hom{\Z}{H_i(M)}{\Q/\Z} \iso H_i(M,\Z)_{\tors}^\vee \]
so we can alternatively get a pairing on $H_i(M, \Z)_{\tors}$. 
\bigskip\\
More generally, we always get a pairing
\[ H_k(M,\Z)_{\tors} \ot H_k(M, \Z)_{\tors} \to \Q / \Z \]
To see this we use the universal coefficient theorem to see
\[ 0 \to \Ext{1}{\Z}{H_{k}(M, \Z)}{\Z} \to H^{k+1}(M, \Z) \to \Hom{\Z}{H_{k+1}(M)}{\Z} \to 0 \] 
therefore
\[ H^{k+1}(M, \Z)_{\tors} = \Ext{1}{\Z}{H_{k}(M, \Z)}{\Z} = \Hom{\Z}{H_{k}(M, \Z)}{\Q/\Z} \]
Then Poincare duality gives
\[ H_k(M, \Z)_{\tors} \cong H^{k+1}(M, \Z)_{\tors} = \Hom{\Z}{H_{k}(M, \Z)}{\Q/\Z} \]
and hence a ``perfect'' pairing
\[ H_k(M, \Z)_{\tors} \ot H_k(M, \Z)_{\tors} \to \Q / \Z \]

\section{Talk}

We start with a theorem of Popa and Schnell that I will state via the contrapositive of the ususal fashion:

\begin{theorem}[Popa-Schnell '14]
If $X$ is a smooth projective variety carrying a $1$-form $\omega \in H^0(X, \Omega_X)$ with no zeros then $\kappa(X) \le n-1$. 
\end{theorem}

This shows that having a $1$-form with no zeros constrains the geometry of $X$. However, we expect there should be a much more stringent restriction on those varieties with $\kappa(X) \le n-1$ that actually do carry a $1$-form with no zeros. 
\bigskip\\
How do you get $1$-forms, they always arises by pulling back along a map $f : X \to A$ (say to the Albanese). The $1$-form will have no zeros if $f$ is smooth. So we might guess that every nonvanishing $1$-form arises from the pullback along a smooth map to an abelian variety.

\begin{example}
This is not true: let $X = E \times C$ where $C$ is any curve of genus $g \ge 2$ such that $E$ is not an isogeny factor of $\Jac{C}$. Then the only smooth map to an abelian variety is $f : X \to E$. However $\\pi_1 \omega_E + \pi_2 \omega_C$ are all nonvanishing for any nonzero $\omega_E$ and any $\omega_C$. The only one pulled back from $f$ are of the form $\omega_E$. Therefore, we have to be careful. It seems that having a nonvanishing $1$-form $\omega$ implied that some smooth map to an abelian variety exists but $\omega$ may not be pulled back along it. Indeed, you have to deform $\omega$ (by taking $\omega_C \to 0$ in this case) to get it as a pullback from a smooth map.
\end{example}

However, this is still not enough.

\begin{example}
Let $E_1, E_2$ are nonisogenous elliptic curves. Let $X$ be the blowup of $E_1 \times E_2 \times \P^1$ along $E_1 \times \{ 0 \} \times \{ 0 \}$ and $\{ 0 \} \times E_2 \times \{ \infty \}$. Then the pullback of $\pi_1 \omega_1 + \pi_2 \omega_2$ to $X$ has no zeros. However, there is no ``diagonal map'' to an elliptic curve since $E_1, E_2$ are not isogenous. Indeed, the only smooth maps to abelian varieties are (up to composition with an isogeny) the projections $X \to E_i$ and both are not smooth since they have reducible fiber along the exceptional. 
\end{example}

Therefore the best we could do is the following conjecture of Hao and Schreieder:

\begin{conj}[Hao-Schreieder '21, A]
Let $X$ be a smooth projective variety and $\omega \in H^0(X, \Omega_X)$ a $1$-form with no zeros. Then there is a diagram,
\begin{center}
\begin{tikzcd}
X \arrow[rr, dashed] \arrow[rd] & & X' \arrow[ld]
\\
& A
\end{tikzcd}
\end{center}
where $X \rat X'$ is a birational modification and $X' \to A$ is a smooth map to a (nontrivial) abelian variety.
\end{conj}

Furthermore, they conjecture that when $\kappa(X) \ge 0$ we can choose $X' \to A$ to be isotrivial 

\begin{rmk}
When I say ``isotrivial'' I mean the stronger asusmption than $X \to Y$ is an analytic / \etale fiber bundle, I mean it is trivial by a \textit{finite} \etale cover $Y' \to Y$. This is always true for constant families of curves of genus $g \ge 1$ over a regular base. But it already fails for smooth conic bundles over a surface (e.g. any nontrivial Brauer class on a K3). 
\end{rmk}

\begin{conj}[Hao-Schreieder '21, B]
With the assumptions as above, if moreover, $\kappa(X) \ge 0$ then there is a diagram
\begin{center}
\begin{tikzcd}
X \arrow[rr, dashed] \arrow[rd] & & X' \arrow[ld]
\\
& A
\end{tikzcd}
\end{center}
where $X \rat X'$ is a birational modification and $X' \to A$ is a smooth isotrivial map (meaning it is an analytic fiber bundle and moreover is trivialized by an isogeny $A' \to A$).
\end{conj}

It turns out our work will also have applications to the case where, instead of asuming there is a nonvanishing $1$-form, we assume that we are given a map $f : X \to A$ that is close to smooth.

\begin{conj}[Meng-Popa '21, C]
Let $f : X \to A$ be an algebraic fiber space, with $X$ a smooth projective variety and $\kappa(X) \ge 0$ (equivalently by their work $\kappa(F) \ge 0$ for the general fiber). If $f$ is smooth away from codimension $2$ in $A$ then there there is a birational modification
\begin{center}
\begin{tikzcd}
X \arrow[rr, dashed] \arrow[rd] & & X' \arrow[ld]
\\
& A
\end{tikzcd}
\end{center}
so that $X' \to A$ is a smooth isotrivial fiber bundle. Equivalently $X \to A$ is birationally trivialized after an isogeny $A' \to A$. 
\end{conj}

From the example, we can see that we had to blow up to make a birational modification necessary. Therefore, Nathan, Hao, and I conjectured that:

\begin{conj}[Chen-C-Hao '23, D]
If $X$ has a nonvanishing $1$-form and moreover $X$ is minimal then there is a smooth isotrivial map $X \to A$.
\end{conj}


\begin{theorem}[C '24]
These conjectures hold under the assumption that $X$ admits a good minimal model (i.e. there is $X \rat X'$ such that $K_{X'}$ is semiample, in particular we must have $\kappa(X) \ge 0$). 
\end{theorem}

\begin{cor}
If both $\kappa(X) \ge 0$ and $\dim{X} - \kappa(X) \le 4$ or $f : X \to \Alb_X$ has generic fiber of dimesnion $\le 3$ then the conjectures hold.
\end{cor}

\begin{proof}
By [Lai] if the Iitaka fiber $F$ of $X \rat S$ has a good minimal model then so does $X$. We will prove that $F \to \Alb_X$ is not contracted so $F$ is irregular. Then Fujino's work shows that abundance holds for irregular varieties of dim $\le 4$. Similarly, [Lai] shows if the generic fiber of $X \to \Alb_X$ admits a good minimal model then so does $X$.
\end{proof}

Notice that because we had to assume $\kappa(X) \ge 0$ to get a minimal model, our theorem says nothing about Conjecture A when $X$ is uniruled. Our main technical theorem partially rectifies this issue.

\begin{theorem} \label{thm:main_MRC}
Let $X$ be a smooth projective variety equipped with a map $f : X \to A$ to an abelian variety satisfying and there are $1$-forms $\omega_1, \dots, \omega_g \in H^0(A, \Omega_A)$ such that $f^* \omega_1, \dots, f^* \omega_g$ are independent pointwise. Assume the base $Y$ of the MRC fibration $X \rat Y$ admits a good minimal model. Then there exists a quotient with connected kernel $q : A \to B$ to an abelian variety $B$ of dimension $\ge g$ and a birational map $Y \rat Z \times^G B'$ making the diagram
\begin{center}
    \begin{tikzcd}
        X \arrow[r, dashed] \arrow[rr, bend left, "f"] & Y \arrow[d, dashed] \arrow[r] & A \arrow[d, "q"]
        \\
        & Z \times^G B' \arrow[r] & B
    \end{tikzcd}
\end{center}
commute. Here, $B' \to B$ is an isogeny with kernel $G$, and $Z$ is a smooth projective variety with a $G$-action. 
\end{theorem}

\subsection{Proof of the Main Result}

The proof has the following steps:

\begin{enumerate}
\item the Iitaka fiber $F \to Y \rat S$ has image in $A$ an abelian variety $B$ of dimension $\ge g$
\item suppose that there is a diagram
\begin{center}
\begin{tikzcd}
X \arrow[d, "g"'] \arrow[r, "f"] & \cA \arrow[ld]
\\
S 
\end{tikzcd}
\end{center}
where $\cA \to S$ is an abelian scheme, $g$ has klt fibers with $K \sim 0$, and $f$ is surjective. Then $\cA$ ``splits off as a direct factor''. Really there is an isogeny $\cA' \to \cA$ with kernel $G$ and $X \cong f^{-1}(0) \times^G_S \cA'$ over $S$.
\item apply this to the Iitaka fibration of a good minimal model $Y'$ of $Y$
\begin{center}
\begin{tikzcd}
Y' \arrow[d] \arrow[r, "f"] & B \times S \arrow[ld]
\\
S
\end{tikzcd}
\end{center}
over the locus $U \subset S$ where the fibers are at worst klt we get $Y'_U \cong Z \times^G B'$ for an isogeny $B' \to B$ with kernel $G$ where $Z = f^{-1}(0) \cap Y_U'$
\item choose a $G$-equivariant smooth compactification $Z \embed \ol{Z}$ thus $Y'_U$ is an open set of $\ol{Z} \times^G B'$ which is a smooth variety with an obvious smooth isotrivial map
\[ \ol{Z} \times^G B' \to B \]
hence giving our diagram
\begin{center}
    \begin{tikzcd}
        X \arrow[r, dashed] \arrow[rr, bend left, "f"] & Y \arrow[d, dashed] \arrow[r] & A \arrow[d, "q"]
        \\
        & Z \times^G B' \arrow[r] & B
    \end{tikzcd}
\end{center}
\end{enumerate}

\subsection{Step (a)}

Suppose that $Y$ itself admits $g$ poinwise independent $1$-forms. Consider the following diagram,

\begin{center}
\begin{tikzcd}
\wt{Y} \arrow[r] \arrow[rd] & Y \arrow[d, dashed] \arrow[r] & \Alb_X \arrow[d] \arrow[r] & B
\\
& S \arrow[r, dashed] & Q
\end{tikzcd}
\end{center}
where $Q = \coker{(\Alb_F \to A)}$ (recall that since $\kappa(F) = 0$ Kawamata proves that $F \onto \Alb_F$)
since $\wt{Y} \to B$ contracts the general fiber of $\wt{Y} \to S$ by definition, rigidity shows that there is a rational map $S \rat Q$. Hence the map $Y \to Q$ factors birationally through the Iitaka fibration. Now we use the full power of Popa-Schnell

\begin{theorem}[PS '14]
Let $f : X \to A$ be a morphism from a smooth projective variety to an abelian variety. If $H^0(X, \omega_X^{\ot n} \ot f^* \L^{-1} ) \neq 0$ for some ample $\L \in \Pic{A}$ and some $n > 0$ then every $Z(f^* \omega) \neq \empty$ for all $\omega \in H^0(A, \Omega_A)$.
\end{theorem}

Let $W \subset H^0(A, \Omega_A)$ be spanned by the $\omega_1, \dots, \omega_g$. Then the above theoem shows that $W \cap H^0(Q, \Omega_Q) = \{ 0 \}$ so $\dim{Q} + g \le \dim{A}$ proving the claim.
\bigskip\\
But since $X \rat Y$ is a rational map, its' not actually clear that $\omega_1, \dots, \omega_g$ are independent everywhere on $Y$. We need a slight improvement of PS14.

\newcommand{\cN}{\mathcal{N}}

\begin{theorem} \label{thm:generalization_of_PS14}
Let $f : X \to A$ be in $\Var_A$. Consider the sheaf of $k$-forms killed by $-\wedge f^* \omega$ for all $\omega \in H^0(A, \Omega_A)$
\[ P\Omega_X^k \coloneq \ker{(\Omega_X^k \to \Omega_X^{k+1} \ot H^0(A, \Omega_A^{\vee}))}  \]
Suppose there is a line bundle $\cN \embed P\Omega_X^k$ and an ample $\L \in \Pic(A)$ so that $H^0(X, \cN^{\ot d} \ot f^* \L^{-1}) \neq 0$ for some $d \ge 1$. Then $f$ does not satisfy $(\ast)_1$ i.e.\ every $\omega \in H^0(A, \Omega_A)$ has nonempty $Z(f^* \omega) \neq \varnothing$.
\end{theorem}

I claim that $X \to Q$ will satisfy the assumption of this result. Indeed, $m^* \omega_Y \embed P \Omega^{\dim{Y}}_X$ and is positive for $Y \to Q$ because this factors through Iitaka. 


\subsection{Step (b)}

\begin{theorem} \label{thm:abelian_decomposition}
Let $g : (X, \Delta) \to S$ be a flat projective family of pairs over a locally noetherian reduced base scheme $S$ of pure characteristic zero whose fibers satisfy
\begin{enumerate}
\item $(X_s, \Delta_s)$ are klt pairs (in particular the fibers are integral with $K_{X_s} + \Delta_s$ a $\Q$-Cartier divisor) 
\item $K_{X_s} + \Delta_{X_s} \equiv_{\text{num}} 0$ 
\end{enumerate}
equipped with a surjective $S$-morphism $g : X \to \cA$ where $\cA \to S$ is a polarized abelian scheme. Let $Z = f^{-1}(0_A)$. Then there is an isogeny $\pi : \cB \to \cA$ such that in the diagram
\begin{center}
\begin{tikzcd}
Z \times_S \cB \arrow[rrd, bend left] \arrow[ddr, bend right, "\sigma"] \arrow[rd, dashed, "\tilde{\sigma}"]
\\
& X \times_{\cA} \cB \arrow[r] \arrow[d] & \cB \arrow[d, "\pi"]
\\
& X \arrow[r, "f"] & \cA 
\end{tikzcd} 
\end{center}
the unique map $\tilde{\sigma} : Z \times \cB \iso X \times_{\cA} \cB$ induced by the action is an isomorphism. Hence there is an $S$-isomorphism $X \cong Z \times^G_S \cB$ where $G = \ker{(\cB \to \cA)}$.
\end{theorem}

\section{Brigeland Stability}


\renewcommand{\Coh}{\mathrm{Coh}}
\renewcommand{\ch}{\mathrm{ch}}

Let $\omega$ be an ample and $B$ a divisor class. Then we define
\[ \Coh^{\omega, B}(X) := \left< \F_{\omega, B}[1], \J_{\omega,B} \right> \]
where $\F$ are the torsion-free objects with HN factors having $\mu_{\omega, B} \le 0$ and $\J$ contains all torsion objects and those whose torsion-free quoteient has all HS factors with $\mu_{\omega, B} > 0$ where
\[ \mu_{\omega,B}(x) = \frac{\omega \ch_1^B(x)}{\omega^2 \ch_0^B(x)} \]
where
\[ \ch_0^B := \ch_0 \quad \ch_1^B = \ch_1 - B \ch_0 \quad \ch_2^B = \ch_2 - B \ch_1 + B^2/2 \ch_0 \]
$E \in \Coh^{\omega, B}$ then 
\[ \nu_{\omega, B}(E) = \frac{\ch_2^B - \omega^2/2 \ch_0^B(E)}{\omega \ch_1^B(E)} \]
which is the real part over the imaginary part of the central charge. 
\[ Z_{\omega, B} = \int_X e^{i \omega} \cdot \ch^B \]

\begin{theorem}
Theorem, this is the heart of a $t$-structure on $D^b(\Coh)$. Maybe this holds in general??
\end{theorem}

The central charge $Z_{\omega, B} : K_0(X) \to \CC$ is determined by $\omega, B$. 

\begin{rmk}
We will often restrict to $\omega = \alpha H$ and $B = \beta H$. 
\end{rmk}

\begin{defn}
We say $E \in \Coh^{\omega, B}$ is semistable if for any exact $\triangle$ $F \to E \to Q$ all in the heart (this is the definition of an exact sequence in the heart) $\nu(F) \le \nu(E)$. 
\end{defn}

\begin{lemma}
If $\phi : E \to F$ is a map in $\Coh^{\omega, B}(X)$ of semistable objects and $\nu(E) > \nu(F)$ then $\phi = 0$.
\end{lemma}

\subsection{Correct Definitions}

Twisted Chern character
\[ \ch^B = e^{-B} \ch \]
Central charge
\[ Z = \int_X e^{-i \omega} \ch^B = \int_X e^{-B - i \omega} \ch \]

\subsection{Reider redux}

We want to show that $K_X \ot L$ is globally generated. Then $H^1(X, \omega_X \ot L \ot \I_x) = 0$ would suffice for each point $x \in X$. Then
\[ H^1(X, \omega_X \ot L \ot \I_x) = \Ext{1}{X}{\L \ot \I_x}{\struct{}}^\vee \]
We upgrade this to 
\[ \Hom{}{L \ot \I_x}{\struct{}[1]} = 0 \]
we want that both are in the heart and semistable and the slope of the first is greater than the slope of the second. 
\bigskip\\
They actually show the following
\[ H^1(X, \omega_X \ot L \ot \I_x \ot \I_y) = \Hom{}{L \ot \I_x}{\I_y^\vee [1]} = 0 \]
or more generally $\I_W$ and $\I_Z$ for $W,Z$ subschemes of the same length.


\begin{rmk}
$\mu_{\omega,B} = \mu_{\omega,0} - B \cdot \omega / \omega^2$ and therefore NH filtration is independent of $B$ and only depends on the numerical ray generated by $\omega$.
\end{rmk}

\subsubsection{Checking in the Heart}

We want to show that $L \ot \I_x$ and $\I_y^\vee[1]$ in the heart. 
\[ \cH^i(\I_y^\vee[1]) = 
\begin{cases}
\struct{X} & i = -1
\\
\text{torsion} & i = 0
\\
0 & \text{else}
\end{cases} \]
so this is in the heart. For the first one, $L \ot \I_x$ this is torsion-free and 
\[ \mu_{\omega, B}(L \ot \I_x) = \omega (c_1(L)  - B) / \omega^2 = \alpha (c_1(L) - \beta H)/\alpha^2 H^2 = \alpha(1 - \beta)/\alpha^2 \]
(where we set $H = c_1(L)$) so we must have $\beta < 1$ and then this has positive slope. 

\subsubsection{Computing the Slopes}

We get the inequality
\[ \alpha^2 + (\beta - \tfrac{1}{2})^2 < \tfrac{1}{4} - \tfrac{2}{H^2} \]
want to choose $(\alpha, \beta)$ in the regions such that both objects are semistable. 

\subsubsection{We need to show that the objects are semistable}

We choose the top point since usually this is the most stable 
\[ (\alpha, \beta) = (\tfrac{1}{2}, \sqrt{\tfrac{1}{4} - \tfrac{1}{H^2}}) \]
then 
\[ 0 \to K \to L \ot \I_x \to Q \to 0 \]
so we can show that $K$ is a sheaf if they are all in the heart (because everying is supported in $[-1,0]$ and use the long exact sequence). 
\bigskip\\
Pick some stable factor $\cH^{-1}(Q)$ called $E$ and apply Bogomolov then $D = c_1(E)$. So $\ch_2(E) \le $



\section{Kristen Exercises}

\subsubsection{(a)}

Weighted projective space $X = \P(a,b,c)$ has $K_X^2 = \frac{(a+b+c)^2}{abc}$ (see \chref{https://mathoverflow.net/questions/285123/volume-of-k-x-for-a-weighted-projective-variety}{Jason's answer}. Suppose
\[ K_X^2 = 9 \]
then 
\[ (a^2 + b^2 + c^2)^2 = 9 a^2b^2c^2 \]
hence we have $K_X^2 = 9$ if and only if $a^2 + b^2 + c^2 = 3 a^2 b^2 c^2$ (since both sides are positive). Now, we need to show that if $K_X^2 = 9$ for $X = \P(r,s,t)$ that $r,s,t$ are squares. In general we have
\[ (r + s + t)^2 = 9 r s t \]
and $(r + s + t)^2$ is, of course, a square. For any prime $p \divides r$ we have $p \divides (r + s + t)^2$ and hence $p$ divides it twice. For $p \neq 3$ this implies that $p^2 \divides rst$ but $r,s$ and $r,t$ are coprime so $p^2 \divides r$. Furthermore, if $p = 3$ then we must have an even power of $3$ on the right-hand side so we get $9 \divides r$. The same argument shows that $r,s,t$ are all squares so we conclude.

\subsubsection{(b)}

\newcommand{\ob}{\mathrm{ob}}
\newcommand{\Def}{\mathrm{Def}}
\renewcommand{\X}{\mathcal{X}}

We need to show that the singularities of $\P(a^2, b^2, c^2)$ are all smoothable if $a^2 + b^2 + c^2 = 3 abc$. Consider a singularity of $\P(p,q,w)$. These occur when $[x : y : z]$ is fixed by $(\lambda^p, \lambda^q, \lambda^w)$ for some $\lambda \neq 1$. Assume $z \neq 0$ then we can set $z = 1$ by choosing $\lambda = \zeta z^{1/w}$ for $\zeta^w = 1$ then this is a fixed point if $\zeta_w^p x = x$ and $\zeta_w^q y = y$ but since $(p,w) = (q,w) = 1$ we must have $x = y = 0$. Hence there are exactly three singular points $[1:0:0]$ and $[0:1:0]$ and $[0:0:1]$ which are also the torus fixed points. The singularity at $[0:0:1]$ is the finite quotient singularity $\frac{1}{w}(p,q)$ and similarly for the others. We need to show how to smooth it. Since $p, w$ are coprime, changing the root of unity we get the singularity $\frac{1}{w}(1,q/p)$. The singularities admitting a $\Q$-Gorenstein smoothing (those of class $T$) are exactly the du Val singularieties and the quotient singularities of type $\frac{1}{d n^2}(1 - dna - 1)$ where $n > 1$ and $d,a > 0$ are integers and $a,n$ are coprime. Consider the Markov singularity
\[ \tfrac{1}{c^2} (a^2, b^2) = \tfrac{1}{c^2} (1, b^2 r) \]
where $r$ is such that $a^2 r \equiv 1 \mod c^2$. Lets write $a^2 + b^2 + c^2 = 3 abc$ and hence $r b^2 \equiv -1 \mod c$ so we can write $r b^2 = c s - 1$ for some $s$. Indeed,
\[ s = (3ab - c) r - (ra^2 - 1)/c \]
Thus for $n = c$ and $d = 1$ we win as long as $c, s$ are coprime. Suppose $p$ is a common factor. Since $p$ does not divide $r$ but it does divide $(ra^2 - 1)/c$ (because $c^2 \divides r a^1 - 1$) we must have $p \divides (3ab - c)$ but $p \divides c$ so $p \divides 3 ab$ but $c$ is coprime to each of $a$ and $b$ so $p = 3$. However, $c$ cannot be divisible by $3$ otherwise $x^2 + y^2$ is divisble by $9$ but squares are $1$ mod $3$ so $x^2 + y^2$ is $2$ mod $3$ because $x,y$ cannot have $3$ as a factor if $c$ does.
\bigskip\\   
Now we consider the obstruction theory to deforming. There are three sucessive obstructions:
\[ \ob_1 \in H^0(X, \T^2_X) \quad \ob_2 \in H^1(X, \T^1_X) \quad \ob_3 \in H^2(X, \T^0_X) \]
and if these are zero then there is an exact sequence
\[ 0 \to H^1(X, \T^0_X) \to \Def_X \to H^0(X, \T^1_X) \to H^2(X, \T^0_X) \]
so there are three possible obstruction spaces. The first is an obstruction to deforming the singularities locally, the second is an obstruction to making these local deformations agree on overlaps, the third to making these coherent and hence gluing to a global deformation. Then $H^0(X, \T^1_X)$ specifies a local deformation and as long as it is coherent meaning zero n $H^2(X, \T^0_X)$ it produces a deformation up to a choice of purely global deformation $H^1(X, \T^0_X)$. 
\bigskip\\
Let $X = \P(a^2, b^2, c^2)$ satisfying the Markov equation. We want to compute the $\T^i_X$ sheaves. Now, since the singularities are isolated, we see that $H^{>0}(X, \T^{>0}_X) = 0$ for dimension reasons. Hence $\ob_2 = 0$ automatically. Moreover, $X$ is normal hence CM so Serre duality holds. This will be useful in computing $H^2$. The fact that we can choose a local smoothing means that $\ob_1 = 0$ over each extension of our base. Therefore, it suffices to show that $H^2(X, \T^0_X) = 0$ so that $\ob_3 = 0$ and that any local deformations globalize so we don't need to choose the different local deformations in a compatible way. In general,
\[ \T^0_X = \shHom{\struct{X}}{\Omega_X}{\struct{X}} \]
There is an Euler sequence
\[ 0 \to (\Omega_X)^{\vee \vee} \to \struct{X}(-a^2) \oplus \struct{X}(-b^2) \oplus \struct{X}(-c^2) \to \struct{X} \to 0 \]
given by the kernel of the canonical sections that generate the ring $k[x,y,z]$ graded in degrees $(a^2, b^2, c^2)$. On the stack $\X \to X$ this is an exact sequence with $\Omega_{\X}$ so pushing forward gives the above since the pushforward of the differentials is the reflexive differentials since it is a reflexive sheaf isomorphic to $\Omega_X$ in the smooth locus. Using Serre duality
\[ H^2(X, \T^0_X) = H^0(X, \Omega_X^{\vee \vee} \ot \omega_X)^\vee \]
and there is an inclusion
\[ H^0(X, \Omega_X^{\vee \vee} \ot \omega_X) \subset H^0(X, \struct{X}(-a^2 - (a^2 + b^2 + c^2)) \oplus \struct{X}(-b^2 - (a^2 + b^2 + c^2)) \oplus \struct{X}(-c^2 - (a^2 + b^2 + c^2))) = 0 \]
so we conclude.
\bigskip\\
Therefore, there exists a family $\X \to \Spec{R}$ smoothing $X$ and since around each singularity it is the $\Q$-Gorenstein smoothing we found before this implies that $\X$ is $\Q$-Gorenstein as well since we just need to check near the singular points of $\X$ which are contained in the singular points of $X$. Since $\X$ is $\Q$-Gorenstein $K_{\X_t}^2 = 9$ by constancy and $-K_{\X_t}$ is anti-ample. Hence the generic fiber is a smooth del Pezzo with degree $9$ and hence is isomorphic to $\P^2$ (at least over the algebraic closure). If we pass to $R^{\wedge}$ then since the base is complete we can lift any point in the smooth locus of $X$ to a section and thus $\X_{\eta}$ will have a point and hence is isomorphic to $\P^2$. 

\subsubsection{(3)}

Choose $X \subset \P^3$ to be the Fermat cubic $V(x^3 + y^3 + z^3 + w^3)$. This has an action of $G = S_4 \ltimes (\Z / 3 \Z)^3$ and $-K_X \sim H$ and there is no hyperplane invariant under $G$ (since to be invariant under the action on roots of unity one must not include a variable so it includes none). Hence $\alpha_G(X) = \infty$ so $X$ is $K$-polystable. Hence by openess, a generic cubic surface is $K$-semistable. 

\subsubsection{(4)}

If we can find $\X \to \A^1$ an isotrivial degeneration of smooth Fano 3-fold to $X = \X_0$ a $K$-polystable Fano $3$-fold then by openness of semistability $\X_t$ is $K$-semistable but by properness of the moduli space we cannot have $\X_t$ be $K$-polystable since it is isotrivial but has more than one limit. 
\bigskip\\

\subsection{4.3 Exercies}

\subsubsection{(1)}

Let $X = \P^2$ or $\P^1 \times \P^1$. We show there is no $K$-semistable degeneration. Indeed, the volume is constant. Suppose $\X$ is such a degeneration then $\X_0$ is $\Q$-factorial terminal so it has finite quotient singularities. Then we get
\[ (-K_X)^2 \le \frac{(n+1)^n}{|G|} = \frac{9}{|G|} \]
since $(-K_X)^2 \ge 8$ we must have $|G| = 1$ so $\X_0$ is smooth. Then we use rigidity of smooth Fanos of large index.


\subsubsection{(2)}

The singularities $\frac{1}{4}(1,1)$ and $\frac{1}{4}(1,3)$ are $\Q$-Gorenstein smoothable. We must have
\[ (-K_X)^2 \le \frac{9}{|G|} = \frac{9}{4} \]
so only index $1,2$ are possible.


\subsubsection{(3)}

Let $X = \P(a_0, \dots, a_n)$ be a weighted projective space not equal to $\P^n$. To show it is $K$-unstable we use the local volume
\[ (-K_X)^n \le \frac{(n+1)^n}{|G|} \]
but
\[ (-K_X)^n = \frac{(a_0 + \cdots + a_n)^n}{a_0 \cdots a_n} \]
at the coordinate points there are quotient singularities with $|G| = a_n$ since the action is faithful using the well-formedness condition. Hence if we order $a_0 \le \cdots \le a_n$ then
\[ (-K_X)^n = \frac{(a_0 + \cdots + a_n)^n}{a_0 \cdots a_n}  \]
By AM-GM
\[ \frac{a_0 + \cdots + a_n}{n+1} \ge (a_0 \cdots a_n)^{\frac{1}{n+1}} \]
with equality iff $a_i$ are all equal. If they are all equal then the result is clear by well-formedness. Hence 
\[ \frac{a_0 + \cdots + a_n}{n+1} > (a_0 \cdots a_n)^{\frac{1}{n+1}} \]
so 
\[ (-K_X)^n = \frac{(a_0 + \cdots + a_n)^{n}}{a_0 \cdots a_n} > (n+1)^n \frac{1}{(a_0 \cdots a_n)^{\frac{1}{n+1}}} \ge (n+1)^n \frac{1}{a_n} \]
since $(a_0 \cdots a_n)^{\frac{1}{n+1}} \le a_n$ where $a_n$ is the largest term proving that $X$ is not $K$-semistable.


\subsubsection{(4)}

\begin{lemma}
If $x \in X$ is a smooth point then
\[ \lim_{k \to \infty} \frac{\length{}{\stalk{X}{x} / \a_k}}{k^n / n!} = 1 \]
where $\a_k$ is the ideal of functions with valuation $\ge k$ for some valuation centered at $x$. 
\end{lemma}

\begin{proof}

\end{proof}


\subsubsection{(5)}


\subsubsection{(6)}

Let $X$ be a smooth degree $d$ del Pezzo surface. Then if $d > 2$ show that $-K_X$ is very ample and the linear system $|-K_X|$ embeds $X \embed \P^d$ as a degree $d$ surface.
\bigskip\\


\subsubsection{(7)}


\subsubsection{(8)}


\subsubsection{(9)}


\subsubsection{(10)}


 
\section{Anti-Iitaka}

\begin{theorem}
Let $f : X \to Y$ be an algebraic fiber space between normal projective $\Q$-Gorenstein varieties and $F$ a general fiber. Suppose $X$ has at worst klt singularities, and $-K_X$ is effective with stable base locus $B(-K_X)$ which does not dominate $Y$. Then we have
\[ \kappa(X, -K_X) \le \kappa(F, -K_F) + \kappa(Y, -K_Y) \]
\end{theorem}

\begin{rmk}
The strict inequality occurs for an elliptic K3. 
\end{rmk}

This relies on:

\begin{theorem}
Let $f : X \to Y$ be an algebraic fiber space between normal projective varieties such that $Y$ is $\Q$-Gorenstein. Let $\Delta$ be an effective $\Q$-Weil divisor on $X$ such that $(X, \Delta)$ is klt, and $D$ a $\Q$-Cartier divisor on $Y$. Suppose that $-(K_X + \Delta) - f^* D$ is $\Q$-effective with stable base locus not dominating $Y$. Then $-K_Y - D$ is $\Q$-effective.
\end{theorem}

\subsection{Use numerically flat bundles for isotriviality}

\begin{lemma}[Cao, Prop 2.8]
Let $f : X \to Y$ be a flat $\struct{X}$-connected morphism of smooth projective varities and let $L$ be $f$-very ample. Let $\E_m = f_* L^{\ot m}$ which is a vector bundle. If $\E_m$ is numberically flat for all $m \ge 1$ then $f$ is locally trivial.  
\end{lemma}

\subsection{In the $-K_X$ nef case we have many effectivity results}

\begin{prop}[Cao, 3.16]
Let $f : X \to Y$ be a fibration of smooth projective varieties with $-K_{X/Y}$ nef. Let $L$ be a pseudo-effective and $f$-ample line bundle. If $p_* L \neq 0$ then $\det{p_* L}$ is pseudo-effective.
\end{prop}

TODO

\begin{prop}[Cao, 3.15]
Let $f : X \to Y$ be a fibration of smooth projective varieties with $-K_{X/Y}$ nef. Suppose that $f$ is smooth in codimension $1$. Let $A$ be a $f$-ample line bundle on $X$ such that $f_* A$ is locally free. Then $A^{\ot r} \ot (f^* \det{f_* A})^\vee$ is pseudo-effective where $r = \rank{f_* A}$.
\end{prop}


TODO.

\subsection{Proof of Main Theorem}

\begin{theorem}[Cao 4.17]
Let $X$ be a smooth projective variety with $-K_X$ nef. Let $f : X \to Y$ be the albanese then $f$ is locally trivial.
\end{theorem}

\begin{proof}
Sketch:
\begin{enumerate}
\item $f$ is flat (he cites) so choose $A$ such that $f_* A^{\ot m}$ is locally free $m \ge 0$ and $A$ is very ample
\item after an isogeny $L := A \ot (f^* \det{p_*(A)})^{\ot - \frac{1}{}}$ makes sense
\item $f$ is smooth in codim $1$ (he cites) so $L$ is pseudo-effective by the lemma and $c_1(f_* L) = 0$ by cnstruction
\item show $f_*(L^{\ot m})$ is numerically flat for all $m \ge 1$
\item then we apply 2.18 to $L$ which is equivalent to $A$ over the base and hence relative very ample
\end{enumerate}
Now we need to justify why it is numerically flat. 
\end{proof}

\begin{lemma}
The vector bundle $f_* (L^{\ot m})$ is numerically flat for all $m \ge 1$.
\end{lemma}

\begin{proof}
We show $m = 1$ separately. Let $V = f_* L$ then because $L$ is $f$-very ample
\[ S^m(V) \onto f_* (L^{\ot m}) \]
so they are all nef and it suffices to show $c_1 = 0$. Prop 3.15 implies $m L - \frac{1}{r_m} f^* c_1(f_* mL)$ is pseudo-effective divide by $m$ and take an isogeny so that 
\[ \wt{L} = L - \frac{1}{m r_m} f^* c_1(p_*(mL)) \]
is represented by a pseudo-effective line bundle. Thus
\[ c_1(\det{f_* \wt{L}}) = c_1(\det{f_* L}) - \frac{1}{m r_m} c_1(f_* mL) \]
but $c_1(V) = 0$ so we get
\[  c_1(\det{f_* \wt{L}}) = - \frac{1}{m r_m} c_1(f_* mL) \]
but the RHS is anti-nef and the LHS is pseudo-effective by Prop 3.16 so they are both trivial.
\end{proof}

Now we have to do the hard work of actually showing that $V$ is nef and $c_1(V) = 0$. 

\begin{enumerate}
\item let $A_Y$ be ample enough that $r A_Y - \det{f_* A}$ is ample and satisfies condition in Theorem 2.10, show
\[ H^0(Y, 2 A_Y \ot V_n) \onto (2A_Y \ot V_n)_y \]
for generic $y \in Y$ where $V_n$ is the pullback of $V$ along multiplication ny $n$

\item for $n \gg 0$ and divisible
\[ H^0(Y, 3 A_Y \ot V_n) \to (3 A_Y \ot V_n)_y \]
is surjective for all $y \in Y$

\item Let $\P(V)$ (resp $\P(V_n)$) be the projecivizations. There is a diagram
\begin{center}
\begin{tikzcd}
\P(V_n) \arrow[d, "p_n"] \arrow[r] & \P(V) \arrow[d] 
\\
Y \arrow[r, "n"] & Y
\end{tikzcd}
\end{center}
since $A_Y^{\ot 3} \ot V_n$ is globally generated it is nef and hence
\[ c_(V_n) + 3 A_Y \]
is nef. Taking the average over translates by torsion points 
\[ c_1(\struct{\P(V)}(1)) + \frac{3}{n} A_Y \]
is nef and this holds for $n \to \infty$ so it is nef and hence $V$ is nef but also $c_1(V) = 0$ by construction so we win.
\end{enumerate}
The better way to say the step is that $\pi^*$ of this thing is nef and $\pi_n^* c_1(A_Y) = n c_1(A_Y)$ or better.

\subsection{Global Generation Results}

\begin{prop}
Let $f : X \to Y$ be a fibration of smooth projective varieties and $A_Y$ a very ample on $Y$ such that $A_Y - K_Y$ separates all $2\dim{Y}$-jets. If $-K_{X/Y}$ is nef then for every $f$-ample pseudo-effective $L$ on $X$
\[ H^0(X, L \ot f^* A_Y^{\ot 2}) \to H^0(X_y, L \ot f^* A_Y^{\ot 2}) \]
is surjective for generic $y \in Y$.
\end{prop}

How do we prove this. 

\begin{theorem}[BP10, PT14, Cao Thm. 2.10]
Let $f : X \to Y$ be a fibration between smooth projective varieties and $L$ pseudo-effective and $D \sim c_1(L)$. Let $m > 0$ be an integer such that $\J(h_L^{\frac{1}{m}}) = \struct{X_y}$ for a semipositive metric $h_L$ on $L$. If $f_* (m K_{X/Y} + L)$ is nonzero then its determinant is pseudo-effective on $Y$.
\par 
Moreover, if $A_Y$ is very ample on $Y$ such that $A_Y - K_Y$ separates all $2 \dim{Y}$-gets then 
\[ H^0(X, mK_{X/Y} + L + f^* A_Y) \onto H^0(X_y, m K_{X/Y} + L f^* A_Y) \]
is surjective for generic $y \in Y$.
\end{theorem}

Question: is there a version with algebraic multiplier ideals? Say $L = \struct{X}(D)$ is effective and we say $\J(\frac{1}{m}D|_{X_y}) = \struct{X_y}$ does this work?


\subsubsection{Important Lemmmas Proofs}

\begin{prop}
Let $f : X \to Y$ be a fibration between smooth projective varieties with $-K_{X/Y}$ nef. Suppose that $f$ is smooth in codim $1$. Let $A$ be a $f$-ample line bundle on $X$ with $f_* A$ locally free. Then $r A - f^* \det{f_* A}$ is pseudo-effective with $r$ the rank of $f_* A$.
\end{prop}

\begin{proof}
Let's just do the case $f$ is smooth. Consider the map
\[ s : \det{f_* A} \to \bigotimes^r f_* A \]
Let $f^r : X^r \to Y$ be the $r$-fold fiber product and $\pi_i : X^r \to X$ the projections. Let
\[ A_r = \bigotimes_{i = 1}^r \pi_r^* A \]
and 
\[ L := A_r - (f^r)^* \det{f_* A} \]
Since 
\[ (f^r)_* A_r = \bigotimes^r f_* A \]
the morphism $s$ induces a non-trivial section
\[ \tau \in H^0(X^r, L) \]
Biehweg's idea is: let $\Delta : X \to X^r$ be the total diagonal then
\[ L|_\Delta = r A - f^* \det{f_* A} \]
and $L$ on $X^r$ is effective thanks to $\tau$. If we can show that $L|_\Delta$ is pseudo-effective then we win. Corollary 2.14 gives us what we want.
\par 
Indeed, let $A_Y$ be the ample in Theorem 2.10. Then  $qL$ is effective and $f^r$-ample. Since $-K_{X^r/Y}$ is nef (it is the sum of the pullbacks of $-K_{X/Y}$ which are nef) we can apply Cor. 2.14 to see 
\[ H^0(X^r, q L + 2 (f^r)^* A_Y) \to H^0(X_y^r, q L + 2 (f^r)^* A_Y) \]
is surjective for a generic fiber. Restricting to the diagonal
\begin{center}
\begin{tikzcd}
H^0(X^r, q L + 2 (f^r)^* A_Y) \arrow[r] \arrow[d] & H^0(X^r_y, q L + 2 (f^r)^* A_Y) \arrow[d]
\\
H^0(X, qr A - q f^* \det{f_* A} + 2 f^* A_Y) \arrow[r] & H^0(X_y, qr A)
\end{tikzcd}
\end{center}
where the stuff pulled back from $Y$ is trivial on $X_y$. But the top map is surjective and the right map is nonzero (since $A$ is relatively very ample) so the bottom map is nonzero. Hence
\[ qr A - q f^* \det{f_* A} + 2 f^* A_Y \]
is effective. Dividing by $q$ and taking $q \to \infty$ we win. 
\end{proof}

\subsection{Guesses}

I think the following is true: as long as Corollary 2.14 holds then you win. Let me state it in the smooth case just to be careful.

\begin{defn}
For a morphism $f : X \to Y$ and an ample $M$ on $Y$ we say that $(\ast)$ holds if for all pseudo-effective $f$-ample $L$ on $X$ the map
\[ H^0(X, L \ot f^* M^{\ot 2}) \to H^0(X_y, L \ot f^* M^{\ot 2}) \]
is surjective for a generic $y \in Y$.
\end{defn}

\begin{theorem}
Let $f : X \to A$ be a smooth projective morphism over an abelian variety. Let $M$ be a very ample on $A$ separating all $2 \dim{A}$-jets. Suppose that $(\ast)$ holds. Then $f$ is isotrivial (I think in the strong sense of killed by the universal cover). 
\end{theorem}

\begin{rmk}
If $L$ is not torsion but is algebraically equivalent to $0$ then $L$ is not killed by any finite \etale cover but it is killed on the universal cover. What is the algebraic version of this stronger notion of isotriviality? Actually, every line bundle is trivialized on the universal cover since $\CC$ has no nontrivial holomorphic line bundles. Now I am a bit confused. 
\end{rmk}

\subsection{Analytic Glossary}

Let $L$ be a holomorphic line bundle on a smooth projective variety viewed a Kahler manifold $(X, \omega)$
\begin{enumerate}
\item $L$ is nef iff for all $\epsilon > 0$ there is a smooth hermitian metric $h_\epsilon$ such that $i \Theta_{h_\epsilon} \ge - i \omega$
\item $L$ is pseudo-effective if there is a singular hermitian metric $h$ such that current $i \Theta_h \ge 0$.
\end{enumerate}

\subsubsection{Multiplier Ideals}

\chref{https://www.math.purdue.edu/~murayama/minicourses2016/Multiplier\%20Ideals.pdf}{copied from here}

Let $X$ be a complex manifold and $\varphi : X \to \RR \cup \{ - \infty \}$ a function (usually psh) then we define the muliplier ideal via
\[ \J(\varphi)_x = \{ f \in \stalk{X}{x} \mid |f| e^{-\varphi} \in L^2_{\text{loc}}(x) \} \]
where being in $L^2_{\text{loc}}(x)$ means that there is some open $x \in U$ such that
\[ \int_U |f|^2 e^{-2 \varphi} \d{\lambda} < \infty \]
for the Lebesgue measure $\lambda$.

\begin{theorem}[Nadel]
If $\varphi$ is plurisubharmonic then $\J(\varphi)$ s a coherent sheaf of ideals.
\end{theorem}

\begin{theorem}
Let $D = \sum ai D_i$ be an effective $\Q$-divisor. Locally write $D_i = V(g_i)$ and $\varphi_D = \log{\left| \prod_i g_i^{a_i} \right|}$ then 
\[ \J(X, D)^{\an} = \J(\varphi_D) \] 
\end{theorem}

Note $\J(\varphi_D)$ is well-defined since any two local equations differ by a nonvanishing  function so the log is bounded and hence does not change local $L^2$-ness.

\subsubsection{Metrics on Line Bundles}

Let $X$ be a complex manifold and $L$ a holomorphic line bundle.

\begin{defn}
A metric on $L$ is a real-valued function $|| \bullet || : L \to \RR_{\ge 0}$ such that for all $x \in X$ and $\xi \in L_x$ we have
\begin{enumerate}
\item $|| \lambda \cdot \xi || = |\lambda| \cdot || \xi ||$ for $\lambda \in \CC$
\item $|| \xi || = 0$ iff $\xi = 0$
\end{enumerate}
\end{defn}

Giving a section $s$ of $L$ is equivalent to giving local sections $s_i \in \struct{X}(U_i)$ compatible $s_i = g_{ij} s_j$ with the transition maps on $U_i \cap U_j$. We can write 
\[ || s || = |s_i| \cdot e^{-\varphi_i} \]
for some $\varphi_i$ defined on $U_i$. Note that there is a compatiblity condition
\[ \varphi_i - \varphi_j = \log{|g_{ij}|} \]
on $U_i \cap U_j$. We call these $\varphi_i$ the \textit{local weights}. 

\begin{defn}
We say a metric $\varphi$ is \textit{singular} if $\varphi_i \in L^1_{\text{loc}}(U_i)$. In this case, we can define a curvature form
\[ \d \d^c \varphi := \frac{i}{\pi} \partial \bar{\partial} \varphi_i \]
in the distributional sense. If the $\varphi_i$ are $C^2$ then you get an actual $(1,1)$-form whatever extra regularity you have beyond $C^2$.
\end{defn}

\begin{example}
Let $s_1, \dots, s_N$ be global sections of $L^{\ot m}$ and $U_i$ a cover trivializing $L$. Define a metric on $L$ via
\[ \varphi_i = \frac{1}{m} \log{\left( \sum_{j = 1}^n |s^i_j|^2 \right)} \]
where $s^i_j$ is a representative local section of $s_j$ on $U_i$. The singular locus of this metric is the base locus of $s_1, \dots, s_N$. 
\end{example}

\begin{example}
Let $D = \sum \alpha_j D_j$ be a $\ZZ$-divisor and $L = \struct{X}(D)$. Given a local section $f$, let $|| f || = |f|$. To find the weights, write $D_j = V(g_j)$ locally and
\[ || f || = \left| f \cdot \prod_{j} g_j^{\alpha_j} \right| e^{- \sum \alpha_j \log{|g_j|}} \]
Notice 
\[ \varphi = \sum \alpha_j \log{|g_j|} \] 
has $\d \d^c \varphi \ge 0$ iff $D$ is effective.
\end{example}

\begin{defn}
A metric $\varphi$ is (semi)positive if $\d \d^c \varphi > 0$ (resp. $\d \d^c \varphi \ge 0$) that is $\d \d^c \varphi > \epsilon \omega$ (resp. $\d \d^c \varphi \ge \epsilon \omega$) where $\omega$ is the Kalher form (or any hermitian form) on $X$. $L$ is (semi)positive if it admits a (semi)positive metric. 
\end{defn}

\begin{defn}
Given a semipositive metric $\varphi$ on $L$ we define the multiplier ideal
\[ \J(\varphi) := \J(\varphi_i) \]
over each $U_i$. 
\end{defn}

\begin{rmk}
This is well-defined since on the overlap $\varphi_i - \varphi_j = \log{|g_{ij}|}$ and multiplication by $|g_{ij}|^2$ does not change being in $L^2_{\text{loc}}(x)$ since it has no poles.
\end{rmk}

\begin{theorem}[Demailly]
$L$ is pseudoeffective iff there exists a singular semipositive metric on $L$.
\end{theorem}


\section{Eliot's Talk - Monodromy Equidistribution and Class Groups}

\newcommand{\cM}{\mathcal{M}}

Cohen-Lenstra conjectures -> Friedman and Washington 1989 formulated similar conjectures with function fields. 
\par 
Let $\FF_q$ be a finite field of characteristic $p \neq 2$. Let $E / \FF_q$ be an elliptic curve (think of this an a quadratic number field with small discriminant since the function field given by $y^2 = f(x)$ i.e. $\FF_q(x)(\sqrt{f(x)})$ and $f$ has small degree).

We want to compute $\Cl(K)$ where $K = k(E)$ and this is the same as $\Jac(E)(\FF_q) = E(\FF_q)$. On $E(\ol{\FF}_q)$ there is a Galois action and we are looking for the fixed points. 
\par 
Let $\ell \neq p$ be a prime not dividing $q$ then $E[\ell](\ol{\FF}_q) \cong (\Z / \ell \Z)^2$ and the Frobenius action corresponds to some matrix
\[ F_E \in \GL_2(\Z / \ell \Z) \]
What is canonical is the conjugacy class of $F_E$. Since it preserves the Weil pairing, it lies in
\[ \mathrm{GSp}_2(\Z / \ell \Z) = \{ A \in \GL_2 \mid \exists m \in (\Z / \ell \Z)^\times : \inner{A x}{A y} = m \inner{x,y} \} \]
In fact, it lives in the particular $\Sp_2$ coset where $m = q$. Now we know $E(\FF_q)[\ell] \cong \ker{(1 - F_E)}$ and the idea of Friedman and Washington is that $F_E$ should behave like a random matrix.

\subsection{What is a Random Elliptic Curve}

Consider a family $\E \to \cM \to \Spec{\FF_q}$ require $\# \M(\FF_{q^n}) < \infty$ for all $n$. First fix $n$ and take $x$ to be uniformly distributed on $\M(\FF_{q^n})$ and take the fibers. 
\bigskip\\
Consider $\E[\ell] \to \M$ the $\ell$-torsion subgroup. This is an \etale cover of order $\ell^2$. The monodromy action will contain the action of all frobenii on the fibers.

\subsection{\etale Fundamental Group}

Let $X$ be a connected scheme and $\FEt_X$ the category of finite \etale covers. 

\begin{example}
For $X = \Spec{k}$ any finite \etale cover is a disjoint union of $\Spec{k'}$ for $k'/k$ finite separable.
\end{example} 

\begin{prop}
There exists a unique (up to almost canonical isomorphism) profinite group $\pi_1^{\et}(X)$ such that $\FEt_X \cong \pi_1(X)^{\et}\text{-Sets}$ where this is the category of finite continuous $\pi_1^{\et}(X)$-sets. Almost canonical means canonical up to inner automorphism.
\end{prop}

Given $\varphi : Z \to X$ then there is a pullback map $\varphi^* : \FEt_X \to \FEt_Z$ and hence a map $\pi_1(Z) \to \pi_1(X)$ unique up to conjugacy. 

\begin{example}
$\pi_1(\Spec{k}) = \Gal(k^{\sep}/k)$. For $k = \FF_q$ then $\pi_1(\Spec{k}) = \hat{\Z} \left< \Frob \right>$. Given any $\Spec{\FF_q} \to X$ we get a natural conjugacy class in $\pi_1(X)$ generated by the Frobenius. 
\end{example}

The sequence of maps $X_{k^\sep} \to X \to \Spec{k}$ induces a sequence
\[ 1 \to \pi_1(X_{k^{\sep}}) \to \pi_1(X) \to \pi_1(\Spec{k}) \to 1 \]
which is exact when $X$ is geometrically connected. 

For $Y \to X$ a finite \etale cover or a local system $\F$ on $X$ these are the same as a map
\[ \rho : \pi_1(X) \to G \]
up to conjugacy where $G$ is the automorphisms of some fiber $Y_{\bar{x}}$ or $\F_{\bar{x}}$.

\newcommand{\geom}{\mathrm{geom}}
\renewcommand{\Gal}{\mathrm{Gal}}

\begin{defn}
$\rho(\pi_1(X_{k^{\sep}}) \subset G$ is called the \textit{geometric monodromy group} denoted $G^{\geom}$.
\end{defn}

Therefore we get a sequence
\begin{center}
\begin{tikzcd}
1 \arrow[r] & \pi_1(X_{k^{\sep}}) \arrow[d, two heads] \arrow[r] & \pi_1(X) \arrow[d] \arrow[r] & \Gal_k \arrow[r] \arrow[d] & 1
\\
1 \arrow[r] & G^{\geom} \arrow[r] & G \arrow[r, "\pi"] & \Gamma \arrow[r] & 1
\end{tikzcd}
\end{center}

\begin{theorem}[Lang '59, Deligne '89, Katz]
Let $M$ be a geometrically irreducible variety and suppose we have a representation
\[ \rho : \pi_1(M) \to G \]
and from the diagram as above. Assume $G$ is finite and $q$ coprime to $|G|$ and $\Gamma$ is abelian. If $C \subset G$ stable under conjugacy and $n \in \Z_{\ge 1}$ then
\[ 
\frac{\# \{ x \in M(\FF_{q^n}) \mid \rho(\Frob_x) \in C \}}{\# M(\FF_{q^n})} = \frac{\# (|C| \cap \pi^{-1}(\gamma^n))}{\# G^{\geom}} + O_{M, \rho}(q^{-n/2}) \]
where $\gamma \in \Gamma$ is the image of $1$ along the downward map. 
\end{theorem}

\begin{rmk}
Note that $\pi(\rho(\Frob_x))$ is equal exactly to $\gamma^n$ since this depends on what field you are over only since it factors through $\pi_1(\Spec{\FF_q})$. Therefore, the best we can hope for is equidistribution in the coset of $G^{\geom}$.
\end{rmk}

\begin{rmk}
\begin{enumerate}
\item Deligne (1980): $G = G(\ol{\Q}_\ell)$
\item Katz-Sarnak (1998) can let $M$ vary in a family
\item Kowalski (2007) gave explicit description of the error bound
\item Feng-Landesman-Rains (2022) did Kowalski for families.
\end{enumerate}
\end{rmk}

\begin{proof}
\begin{enumerate}
\item write $1_C = \sum_{\chi} a_{\chi} \chi$ for a sum of characters of $G$ since these are class functions
\item if $\chi$ is trivial on $\pi^{-1}(\gamma^n)$ then we get the main term
\item $\chi$ is a character of $G$ then $\chi \circ \rho$ is a character of $\pi_1(X)$ hence get $\F$ an irreducible $\ol{\Q}_{\ell}$-local system
\item Lefschetz trace formula:
\[ \sum_{x \in M(\FF_{q^n})} \chi(\rho(\Frob_x)) = \sum_{i = 0}^{2n} (-1)^i \tr{\Frob_{\FF_{q^n}} | H^i_c(X, \F)} \]
\end{enumerate}
\end{proof}

\subsection{Application}

Let $C \to M$ be a family of smooth proper curves of genus $g$ over $\FF_q$. Choose a prime $\ell$ not dividing $q$ and $e \in \Z_{\ge 1}$. There is a finite \etale cover $\Jac(C)[\ell^e] \to M$ this gives a map
\[ \rho : \pi_1(M) \to \mathrm{GSp}_{2g}(\Z / \ell^e \Z) \]
Often we will have $G^{\geom} = \Sp_{2g}(\Z / \ell^e \Z)$ hence $\Gamma = (\Z / \ell^e \Z)^\times$. 

\begin{example}
\begin{enumerate}
\item if $E \to M$ is any non-isotrivial family of elliptic curves then $G^{\geom}(E[\ell^e]) = \Sp_2(\Z / \ell^e \Z)$ for all but finitely many $(\ell, e)$. 
\item $C \to M$ the universal family of hyperelliptic curves has $G^{\geom}(\Jac(C)[\ell^e]) = \Sp_{2g}(\Z / \ell^e \Z)$ [Achter-Pries, 2006]
\item universal curve $C \to M_g$ has $G^{\geom} = \Sp_{2g}$. 
\end{enumerate}

In Achter (2006) he computed everything when $G^{\geom} = \Sp_{2g}$. Consider the case $q \equiv 1 \mod \ell$. This forces $\gamma = 1$ so all the cosets are the same. Then
\[ \frac{\# \{ x \in M(\FF_{q^n}) \mid \Jac(C_x)[\ell](\FF_q) \cong (\Z / \ell \ZZ)^r \}}{\# M(\FF_{q^n})} = \frac{\# F \in \Sp_{2g}(\Z / \ell \Z) \dim \ker{(1 - F)} = r}{\# \Sp_{2g}(\Z / \ell \Z)} + O(q^{-n/2}) \]
In particular, if $\E \to M$ is non-isotrivial 
\[ \E_x[\ell](\FF_q) = 
\begin{cases}
1 & \text{prob } \frac{\ell^2 - \ell - 1}{\ell^2 - 1}
\\
(\Z / \ell \Z) & \text{prob } \frac{1}{\ell}
\\
(\Z / \ell \Z)^2 & \text{prob } \frac{1}{\ell (\ell^2 - 1)}
\end{cases} \] 
\end{example}

If you add level structure at a prime $\ell$ this will constrain the monodromy group to be smaller than $\Sp_{2g}$. 

\section{Noncommutative Prime Ideals}

First we will consider the notion of a ``prime'' (noncommutative) ring $R$. We have the analogous notion of a domain to commutative ring theory

\begin{defn}
A nonzero ring $R$ is a \textit{domain} if $ab = 0$ implies $a = 0$ or $b = 0$.
\end{defn} 

This is unsatisfactory since even matrix rings $M_n(F)$ are not domains for $n \ge 2$. Therefore, we decribe a wider class of rings.

\begin{defn}
Let $R$ be a nonzero ring. We say $R$ is \textit{prime} if $a R b = 0$ implies $a = 0$ or $b = 0$.
\end{defn}

Clearly, if $R$ is commutative and unital then $R$ is a domian if and only if it is prime. 

\begin{lemma}
$R = M_n(F)$ is prime for all $n$ and all fields (or commutative domains) $F$.
\end{lemma}

\begin{proof}
Suppose $a,b \neq 0$. Then $b$ has a nonzero column so $b e_i \neq 0$ for some $e_i$ basis vector. Likewise $a e_j \neq 0$. Now I claim there is a matrix $M$ sending $a e_i$ to $\lambda e_j$ for some $\lambda \in F$ thus $a M b \neq 0$. Indeed, choose the elementary matrix $E_{kj}$ where $k$ is a nonzero entry of $a e_i$ this sends $a e_i$ to its $k$-th coefficient times $e_j$. 
\end{proof}

\begin{defn}
A two-sided ideal $I \subset R$ is \textit{prime} if $R / I$ is.
\end{defn}

Notice we need $I$ to be two-sided to get a ring quotient. Note that $I$ is prime iff $I \neq R$ and $a R b \subset I$ implies $a \in I$ or $b \in I$. 

\begin{rmk}
There's something really nasty about this definition. Unlike the usual definition, $R / I$ is a domain, this notion of prime ideal is not functorial. Indeed, if $\varphi : R \to S$ is not surjective and $P \subset S$ is a prime ideal then $\varphi^{-1}(P)$ may not be prime because if $a R b \subset \varphi^{-1}(P)$ this just means that $\varphi(a) \varphi(R) \varphi(b) \subset P$ but if $\varphi(R) \neq P$ this does not imply that $\varphi(a) \in P$ or $\varphi(b) \in P$. The usual notion is functorial but too rare to be useful. This is one reason that noncommutative geometry is horrible. 
\end{rmk}

\section{Bogomolov Inequality and Projective FLatness}

\begin{defn}
Let $\E$ be a vector bundle on a smooth projective variety $X$. We say $\E$ is \textit{projectively flat} if $\P_X(\E)$ arises as a projective representation of the fundamental group
\[ \rho : \pi_1(X) \to \PGL_{r} \]
meaning $\P_X(\E)$ is trivialized on the universal cover.
\end{defn}

\begin{theorem}[Bogomolov-Gieseker]
Let $(X, H)$ be a smooth projective polarized variety of dimension $n$. Suppose that $\E$ is a $\mu_H$-semistable vector bundle of rank $r$. then
\[ \left( 2 r c_2(\E) - (r-1) c_1(\E)^2 \right) H^{n-2} \ge 0 \]
which equality if and only if $\E$ is projectively flat.
\end{theorem}

\begin{rmk}
Note that using the splitting principle, writing $\E$ as a sum of chern roots $\alpha_1, \dots, \alpha_r$ we see that
\[ \ch(\E) = e^{\alpha_1} + \cdots + e^{\alpha_r} \]
hence
\[ \ch_2(\E) = \tfrac{1}{2} (\alpha_1^2 + \cdots + \alpha_r^2) = \tfrac{1}{2} (c_1(\E)^2 - 2 c_2(\E)) \]
Therefore, noting that $\ch_1(\E) = c_1(\E)$ and $\ch_0(\E) = r$ we conclude that 
\begin{align*}
\Delta(\E) :&= \left( \ch_1(\E)^2 - 2 \ch_0(\E) \ch_2(\E) \right) H^{n-2} = \left( c_1(\E)^2 - r (c_1(\E)^2 - 2 c_2(\E)) \right) \cdot H^{n-2} 
\\
& = \left( 2 r c_2(\E) - (r-1) c_1(\E) \right) \cdot H^{n-2} \ge 0
\end{align*}
when $\E$ is $\mu_H$-semistable. From now on, this is the standard way to write the Bogomolov-Gieseker inequality.
\end{rmk}

\begin{cor}
If $X$ is simply connected and $\E$ is a vector bundle of rank $r$ satisfying the equality 
\[ \ch_1(\E)^2 \cdot H^{n-2} = 2 r \, \ch_2(\E) \cdot H^{n-2} \]
(do we need semistable) then $\E = \L^{\oplus r}$ for some $\L \in \Pic{X}$.
\end{cor}

\subsection{Vector Bundles of Rank $2$}

Let $\E$ be a vector bundle of rank $2$ on a smooth projective surface $X$. Let $\F = \shHom{\struct{X}}{\E}{\E}$. Notice that
\[ c(\F) = 1 + 4 c_2(\E) - c_1(\E)^2 \]
Grothendieck-Riemann-Roch shows
\[ \chi(\F) = \int_X \ch(\F) \Td_X = \tfrac{1}{3} (c_1^2 + c_2) - \Delta(\E) \]
Therefore, the deformation space of $\E$ has virtual dimension 
\[ h^0(\F) - \chi(\F) = h^0(\F) + \Delta(\E) - \tfrac{1}{12} (c_1^2 + c_2) \ge h^0(\F)  - \tfrac{1}{12} (c_1^2 + c_2) \]
if $\E$ is semistable.

\subsubsection{Pairings}

Recall the Mukai pairing
\[ \chi(\E_1, \E_2) := \sum_{i = 0}^n \mathrm{ext}(\E_1, \E_2) \]
On a surface of Picard rank $1$ we write Chern characters as triples (Mukai vectors?) $(r, \mu, \Delta)$ where $r$ is the rank, $\mu := \frac{\ch_1}{r}$ is the slope, and 
\[ \Delta := \frac{\mu}{2} - \frac{\ch_2}{r} \]
is the $\Delta$ from before divided by $2 r^2$.

\begin{rmk}
The advantage of slope and discriminant is that they are additive under tensor products
\[ \mu(E_1 \ot E_2) = \mu(E_1) + \mu(E_2) \quad \quad \Delta(E_1 \ot E_2) = \Delta(E_1) + \Delta(E_2) \]
\end{rmk}

For $X = \P^2$, Riemann-Roch becomes
\[ \chi(E) = r(E) \left[ P(\mu(E)) - \Delta(E) \right] \]
where
\[ P(x) = \tfrac{1}{2} x^2 + \tfrac{3}{2} x + 1 \]
is the Hilbert polynomial of $\struct{\P^2}$.

\subsubsection{Classification on $\P^2$}


The classification is due to Dr\'{e}zet and Le Potier (see \chref{https://arxiv.org/pdf/2008.10695}{refenerces here}).

\begin{lemma}
If $\E$ is semistable on $\P^2$ then $\Ext{2}{}{\E}{\E} = 0$.
\end{lemma}

\begin{proof}
By Serre duality
\[ \Ext{2}{}{\E}{\E} = \Hom{}{\E}{\E \ot \omega_X}^{\vee} \]
and these are semistable sheaves with
\[ \mu(\E \ot \omega_X) = \frac{c_1(\E \ot \omega_X) \cdot H}{\rank{\E}} = \frac{c_1(\E) \cdot H + (\rank{\E}) K_X \cdot H}{\rank{\E}} = \mu(\E) + K_X \cdot H = \mu(\E) - 3 < \mu(\E) \]
so there are no maps since the slope is smaller. 
\end{proof}

\begin{cor}
Any moduli space of semistable sheaves on $\P^2$ is smooth. 
\end{cor}

The same proof works as long as $K_X \cdot H < 0$. Mukai's theorem shows that for $X$ a K3 (and I think an abelian surface also) even though $\Ext{2}{}{\E}{\E} \neq 0$ by Serre duality the moduli spaces of sheaves still turn out to be smooth.

\begin{defn}
A bundle $\E$ is \textit{exceptional} if it is stable and $\Ext{1}{}{\E}{\E} = 0$ (hence it is rigid).
\end{defn}

Notice that on $\P^2$ we always have $\Ext{2}{}{\E}{\E} = 0$ and if $\E$ is stable then $\Hom{}{\E}{\E} = \C \id$ so if $\E$ is exceptional we have
\[ \chi(\E, \E) = 1 \]
Using the Riemann-Roch formula
\[ 1 = r(\E)^2 (1 - 2 \Delta(\E)) = r(\E)^2 - \ch_1(\E)^2 + 2 r(\E) \ch_2(\E) \]
and therefore $r(\E)$ and $\ch_1(\E) \cdot H$ must be coprime. Let $\alpha = \mu(\E)$ and $r_\alpha$ the smallest positive integer such that $r_\alpha \alpha \in \Z$. Then $r(\E) = r_\alpha$ since $\alpha = \frac{\ch_1(\E) \cdot H}{r(\E)}$ and the top and bottom are coprime. Hence from the formula we see that
\[ \Delta(\E) = \Delta_\alpha := \frac{1}{2} \left( 1 - \frac{1}{r_{\alpha}^2} \right) \]

\begin{theorem}[Dre87]
On $\P^2$, for each slope $\mu \in \Q$ there is at most one exceptional bundle with slope $\mu$. Furthermore, the exceptional slopes are given by an injection
\[ \varepsilon : \Z[\tfrac{1}{2}] \to \Q \]
defined inductively as: $\varepsilon(n) = n$ for $n \in \ZZ$ and
\[ \varepsilon\left( \frac{2 p + 1}{2^{q+1}} \right) = \varepsilon \left( \frac{p}{2^q} \right) \star \varepsilon \left( \frac{p+1}{2^q} \right) \]
where
\[ \alpha \star \beta = \frac{\alpha + \beta}{2} + \frac{\Delta_\beta - \Delta_\alpha}{3 + \alpha - \beta} \]
\end{theorem}

Since all the $\Ext{2}{\E}{\E} = 0$ for semistable $\E$ we have found all the zero-dimensional moduli spaces. The higher-dimensional spaces are controlled by the Dr\'{e}zet--Le Potier curve:

\[ \delta(\mu) = \sup_{x \in \Z[\tfrac{1}{2}] : |\mu - \varepsilon(x)| < 3} \big( P(-|\mu - \varepsilon(x)|) - \Delta_{\varepsilon(x)} \big) \]
This is a ``fractal'' curve (it is infinitely branched at certain limit points but probably has fractal dimension $1$ so isn't really a fractal).

\begin{theorem}
Let $\bf{v} = (r, \mu, \Delta)$ be an integral Chern character of positive rank. Then the moduli space $M(\bf{v})$ is positive dimensional if and only if $\Delta \ge \delta(\mu)$. If $r \ge 2$, then the general member of $M(\bf{v})$ is a vector bundle.
\end{theorem}

\section{Positivity of CM line bundle}

\subsection{Nefness results}

\begin{prop}[Thm 9.25 K-stab book]
Let $f : X \to C$ be a projective flat morphism from a normal variety to a smooth projective curve with reduced fibers. Let $\Delta$ be an effective $\Q$-Divisor and $L$ a Cartier divisor. Assume
\begin{enumerate}
\item $K_X + \Delta$ is $\Q$-Cartier and a general fiber $(X_t, \Delta_t)$ is klt
\item $L - K_{X/C} - \Delta$ is nef and $f$-ample.
\end{enumerate}
Then $f_* L$ is a nef vector bundle.
\end{prop}

We also need:

\begin{prop}
Let $f : X \to C$ be a projective flat morphism from a normal variety to a smooth projective curve with reduced fibers. Let $\Delta$ be an effective $\Q$-Divisor such that $K_X + \Delta$ is $\Q$-Cartier and $(X_t, \Delta_t)$ is klt for generic $t$. If $m(K_X + \Delta)$ is Cartier then $f_* \struct{X}(m(K_{X/C} + \Delta))$ is nef. 
\end{prop}

\begin{prop}
Let $f : X \to C$ be a projective flat morphism from a normal variety to a smooth projective curve with reduced fibers. Let $\Delta$ be an effective $\Q$-divisor such that $(X_t, \Delta_t)$ is klt for generic $t$ and $K_{X/D} + \Delta \sim_{\Q} f^* L$ then $L$ is pseduo-effective.
\end{prop}

\subsection{The CM line bundle}

$\lambda_{CM}$ on the stack descends to $\Lambda_{CM}$ as a $\Q$-line bundle on the GMS (Prop. 9.33).

\subsection{Invariants of Filtrations}

Let $X$ be an $n$-dimensional projective variety and $L$ a big $\Q$-line bundle. Fix $r$ such that $r L$ is Cartier. Let
\[ V_\bullet = \bigoplus_{m \in r \N} V_m \subset R = \bigoplus_{m \in r \N} H^0(X, m L) \]
be a graded linear series containing an ample series.

\begin{defn}
Fix a filtration $\F$ on $V_\bullet$. We define the \textit{concave transform} 
\[ G^{\F} : \Delta(V_\bullet) \to \RR \]
\end{defn}

\begin{defn}
Given a filtration $\F$ of $R$ and $\delta \in (0, \delta_{\max}]$ define the $\delta$-log canonical slope $\mu(\F, \delta)$ as
\[ \mu(\F, \delta) = \sup \{ t \in \RR \mid \lct(X, \Delta; I_\bullet^{(t)}(\F)) \ge \delta \} \]
\end{defn}
When $\delta = 1$, we call it the log canonical slope denoted denoted by $\mu(\F)$. Then we define the Ding invariant
\[ \D(\F, \delta) := \mu(\F, \delta) - S(\F) \]


\subsection{Nefness}

Now let $f : (X, \Delta) \to C$ be a family of log Fano pairs over a smooth projective curve $C$ and fix $r$ such that $r(K_{X/C} + \Delta)$ is Cartier. Assume $L = -K_X - \Delta$ is $f$-ample and $(X_t, \Delta_t)$ is klt for general $t \in C$. Fix some general $t$. 
\bigskip\\
We set up a HN filtration on the ring
\[ R = \bigoplus_{m \in r \N} R_m = \bigoplus_{m \in r \N} H^0(X_t, -m(K_{X_t} + \Delta_t)) \]
and define
\[ N_m = \dim{R_m} \quad V = (-K_{X_t} - \Delta_t)^n \]


\section{Maryland Thoughts}

Some questions:
\begin{enumerate}
\item $\P^1$ in the $K$-stable smooth locus? We already have:
\begin{enumerate}
\item $\P^1$ in the coarse space (blowup of type $(2,2)$)
\item $\P^1$ in the semistable locus mapping trivially to the coarse space (Junyan showed me such a polystable test configuration) that is not isotrivial (not all fibers are the same)
\item 
\end{enumerate}
\end{enumerate}

Can we run Popa-Schnell hyperbolicity argument when the fibers have MRC base admitting a GMM. I think we could use the higher MHMs in the decomposition theorem as I did previously. Maybe I need the MRC base to have maximal variation. Maybe if we use strong non-isotriviality (the top Kodaira-Spencer map is injective) this will hold.

\subsection{TODO list}

\begin{enumerate}
\item reimbursements
\item K3 bridgeland stability calculations
\item measures of irrationality for CY hypersurfaces
\item write down argument for degrees $d,2d$
\item write up $\mu_p$-cover rational surfaces
\item hyperbolicity questions
\end{enumerate}


\section{Constructing a Hodge Module}

Let $f : Y \to X$ be a surjective morphism with connected fibers between smooth projective varieties.

\subsection{The cotagent bundles}

Consider the diagram
\begin{center}
\begin{tikzcd}
Y \arrow[d, "f"] & T^* X \times_X Y \arrow[l, "p_2"'] \arrow[r, "\d{f}"] \arrow[d] & T^* Y 
\\
X & T^* X \arrow[l, "p"']
\end{tikzcd}
\end{center}
Inside the cotangent bundle of $X$ there is the singular set
\[ S_f = p_1(\d{f}^{-1}(0)) \subset T^* X \]
A cotangent vector $(x, \xi) \in T^* X$ belongs to $S_f$ iff $(f^* \xi)_x = 0$. Hence $x \in D_f$ whenever $(x, \xi) \in S_f$. 

\begin{lemma}
$\dim{S_f} \le \dim{X}$ and every irreducible component of $S_f$ of dimension $\dim{X}$ is the conormal variety of a subvariety of $X$.
\end{lemma}

\subsection{The interesting subsheaves}

\newcommand{\HM}{\mathrm{HM}}
\renewcommand{\R}{\mathbf{R}}

\begin{defn}
We consider the subsheaf
\[ P_f \Omega_{Y}^k = \ker{(\Omega_Y^k \to \Omega_Y^{k+1} \ot f^* \T_X)} \]
where the map is given by adjunction with pullback composed with wedge giving a map
\[ f^* \Omega_X \ot \Omega_Y^k \to \Omega_Y^{k+1} \]
\end{defn}

We fix the following data: a line bundle $\L$ on $X$ and an inclusion of a subsheaf
\[ \cN \embed P_f \Omega_Y^k \]
and define the line bundle
\[ B := \cN \ot f^* (\omega_X \ot \L)^{-1} \]
and an integer $m > 0$ and a section
\[ s \in H^0(Y, B^{\ot m}) \]
Now we form the cyclic cover,
\begin{center}
\begin{tikzcd}
Z \arrow[rrd, "h", bend right] \arrow[rr, bend left, "\varphi"] \arrow[r, "\mu"] & Y_m \arrow[r, "\pi"] & Y \arrow[d, "f"]
\\
& & X
\end{tikzcd}
\end{center}
set $n = \dim{X}$ and $d = \dim{Y} = \dim{Z}$. 
\bigskip\\
Let $\cH^{k-d} h_* \Q_Z^H[d] \in \HM(X, k)$ be the polarizable Hodge module obtained by taking the direct image of the constant Hodge module on $Z$ and taking the decomposition theorem. Restricting to the smooth locus of $h$, this is the polarizable VHS on the $(k-n)$-cohomology of the fiber {\color{red} check this}. Let $M \in \HM_X(X, k)$ be the summand with strict support $X$ in the decomposition by strict support. Let $\M$ denote the underlying regular holomoic left $\D_X$-modules and $F_\bullet \M$ the Hodge filtration. Since $F_\bullet \M$ is a good filtration, the associated graded $\cA_X$-module 
\[ \gr_\bullet^F \M = \bigoplus_{k \in \Z} \gr^F_k \M \]
is coherent over $\cA_X = \Sym{}{\T_X}$. We denote the corresponding coherent sheaf on the cotangent bundle by $\G^M$. 
\bigskip\\
One has the following concrete description of $\gr^F_\bullet \M$ on $Z$ consider the complex of graded $h^* \cA_X$-modules
\[ C_{Z, \bullet} := \Big[ h^* \cA_X^{\bullet - n} \to h^* \cA_X^{\bullet - n + 1} \ot \Omega_Z^1 \to \cdots \to h^* \cA_X^{\bullet - n + d} \ot \omega_Z \Big] \]

{\color{red} NOTE MY COMPLEX IS $\omega_Z$ TIMES PS's}

placed in cohomological degrees $-d, \dots, 0$. The differential is induced by the natural pullback, wedge, and contraction of $1$-forms and tangent fields. 

\begin{prop}
In the category of graded $\cA_X$-modules, $\gr_\bullet^F \cM$ is a direct summand of 
\[ R^{k-d} h_* C_{Z, \bullet} \ot \omega_X^{-1} \]
\end{prop}

\begin{proof}
The complex
\[ \R h_* C_{Z, \bullet} \ot \omega_X^{-1} \]
splits in the derived category of graded $\cA_X$-modules and its cohomology modules computes the associated graded of the Hodge module $\cH^i h_* \Q_Z^H[d]$. 
\end{proof}

\begin{cor}
We have $\gr^F_{\ell} \cM = 0$ for $\ell < n - k$ and $\gr^F_{n - k} \cM = h_* P_f \Omega_Z^k \ot \omega_X^{-1}$
\end{cor}

\begin{proof}
Notice that for $\ell \le n - k$ the complex is supported in degrees $[-(d-k), 0]$ so the $(k-d)$-derived pushforward is zero when $\ell < n - k$ and for $\ell = n - k$ we just get the bottom term
\[ R^{k-d} h_* C_{Z, n-k} = \ker{(h_* \Omega_Y^k \to h_* \Omega_Y^{k+1} \ot \T_X)} = h_* P_f \Omega_Z^k \]
Furthermore, the other summands are torsion (since $\cM$ is defined by strict support decomposition) and this bundle is torsion-free (it is the pushforward of a torsion-free sheaf) so we conclude {\color{red} check this}.
\end{proof}

\begin{prop}
The support of $C_Z$ is $\d{h}^{-1}(0) \subset T^* X \times_X Z$.
\end{prop}


\begin{proof}
It is the pullback of the dual of the Kozul complex. 
\end{proof}

\begin{cor}
The support of $\G^M$ is a union of irreducible components of $S_h$.
\end{cor}

\begin{proof}
We have
\[ \Supp{}{\G^M} \subset \Supp{}{R^{k-d} p_{1*} (p_2^* C_Z \ot \omega_X^{-1}))} \]
{\color{red} CHECK THIS CAREFULY}
and the support of $C_Z$ is $\d{h}^{-1}(0)$ so it follows that this support is contained in $S_h = p_1(\d{h}^{-1}(0))$. Then the support of $\G^M$ is the characteristic variety of the regular holonomic $\D$-module $\cM$ and hence of pure dimension $n = \dim{X}$. Thus it is a union of irreducible complenents of $S_h$ since we know from Lemma 8.2 that $\dim{S_h} \le n$.
\end{proof}

\section{Constructing the Graded Module}

\newcommand{\Rbf}{\mathbf{R}}

We produce in this section a graded $\cA_X$-submodule
\[ \G_\bullet \subset \gr^F_\bullet \cM \]
which, unlike $\gr_\bullet^F \cM$ itself, encodes information about the $f$-singular locus $D_f$ the of \textit{original} morphism $f : Y \to X$. In fact, the support of $\G_\bullet$ will be contained in the set of singular cotangent vectors $S_f$; the point is that $S_f$ is typically smaller than $S_h$, because both the covering and its resolution create additional singular fibers. 
\bigskip\\
There is a tautological section $\varphi^* B^{-1} \to \struct{Z}$ an isomorphism over the component of $Z(s)$. After composiing with $\varphi^* \Omega_Y^p \to \Omega_Z^p$ we obtain
\[ \iota_p : \varphi^* (B^{-1} \ot \Omega_Y^k) \to \Omega_Z^k \]
which is an isomorphism away from $Z(s)$. 

\begin{prop}
There is a morphism of complexes of graded $\cA_X$-modules
\[ \varphi : \Rbf f_*  (B^{-1} \ot C_{Y,\bullet} \ot f^* \omega_X^{-1}) \to \Rbf h_* (C_{Z,\bullet} \ot \omega_X^{-1}) \]
\end{prop}

\begin{proof}
This is basically obvious.
\end{proof}

The construction of $\G_\bullet$ goes as follows
\begin{align*}
\G_\bullet & := \im{\cH^{k-d} \varphi} = \im{\left( R^{k-d} f_*  (B^{-1} \ot C_{Y,\bullet} \ot f^* \omega_X^{-1}) \to R^{k-d} h_* (C_{Z,\bullet} \ot h^* \omega_X^{-1}) \to \gr_\bullet^F \cM \right)}
\\
& = \im{\left( \gr_\bullet^F \cH^{k-d} f_+ (\struct{Y}, F) \to \gr_\bullet^F h_+(\struct{Z}, F) \to \gr_\bullet^F \cM \right)}
\end{align*}

\begin{prop}
One has $\G_\ell = 0$ for $\ell < n - k$ and $L \ot f_* \struct{Y}\embed \G_{n-k}$
\end{prop} 

\begin{proof}
The first assertion is obvious from $\gr_{\ell}^F \cM = 0$ for $\ell < n - d$. For the second assertion, we need to consider the image of 
\[ R^{k-d} f_* (B^{-1} \ot C_{Y, n-d}) \ot \omega_X^{-1} \to R^{k-d} h_* C_{Z, n-d} \ot \omega_X^{-1} \to \gr^F_{k-d} \cM \]
but $R^{k-d} f_*$ of these complexes is just the pushforward of $\cH^{k-d}$ since they are supported in degrees $[-(d-k), 0]$. The inclusions
\[ f^* \L \embed \cN \ot B^{-1} \ot f^* \omega_X^{-1} \embed B^{-1} \ot P_f \Omega^k_Y \ot f^* \omega_X^{-1} = B^{-1} \ot \cH^{k-d} C_{Y, n-d} \ot f^* \omega_X^{-1} \]
remain injective after applying $f_*$ hence giving
\[ \L \ot f_* \struct{Y} \embed R^{k-d} f_* (B^{-1} \ot C_{Y, n-d}) \ot \omega_X^{-1} \to R^{k-d} h_* C_{Z, n-d} \ot \omega_X^{-1} \]
and the last map is
\[ f_* (B^{-1} \ot P_f \Omega_Y^k) \ot \omega_X^{-1} \to h_* P_h \Omega_Z^k) \ot \omega_X^{-1} \]
which is the pushforward of an injective map and hence injective. Furthermore, we know that this last sheaf equals $\gr^F_{k-n} \cM$ (by the torsion-freeness) so we win.
\end{proof}

\begin{prop}
We have $\Supp{}{\G} \subset S_f$
\end{prop}

\begin{proof}
Indeed, $\G_\bullet$ is a quotient of a complex formed out of $C_{Y, \bullet}$ which has support inside $S_f$.
\end{proof}

\subsection{Additional properties}


Except in trivial cases, $\G_\bullet$ is not the associated graded of a Hodge module. Nevertheless, we shall see that it inherts several good properties from $\gr_\bullet^F \cM$.

\begin{lemma}
Every irreducible component of $\Supp{}{\G}$ is the conormal variety of some subvariety of $X$.
\end{lemma}

\begin{proof}
Saito [Sch14, section 29] proves $\G^M$ is a Cohen-Macaulay sheaf on $T^* X$ of dimension $n$ hence is unmixed so every associated point has codimension $n$. This property is inherited by subsheaves so the components of $\Supp{}{\G}$ are all $n$-dimensional. Since $\Supp{}{\G} \subset S_f$ and the top dimensional subsets of $S_f$ are conormal varieties we conclude.
\end{proof}

Recall our notation: $D_f \subset X$ is the singular locus of the surjective morphism $f : Y \to X$. Being part of a Hodge module, the coherent sheaves $\gr^F_k \cM$ are locally free on the open $X \sm D_h$ where $M$ is a VHS. Surprisingly, the sheaves $\G_\ell$ are torsion-free on the much larger open set $X \sm D_f$.

\begin{prop}
For every $\ell$, the sheaf $\G_\ell$ is torison-free on $X \sm D_f$.
\end{prop}

\begin{proof}
Replacing $X$ by the open $X \sm D_f$ we may assume $S_f$ is the zero section so $\Supp{}{\G}$ is contained in the zero section. Torsion-freeness is equivalent to 
\[ \codim_X \Supp{}{R^i \shHom{\struct{X}}{\G_\ell}{\omega_X}} \ge i + 1 \]
for every $i \ge 1$ (see [PP09, Lemma A.5]). We can compute the dual directly apply Grothendieck duality to the projection $p : T^* X \to X$. Note first that
\[ p_* \G = \bigoplus_{k \in \Z} \G_k \]
is $\struct{X}$-coherent because the support of $\G$ is contained in the zero section (so it is proper over $X$) in particular $\G_k = 0$ for $k \gg 0$. Since the relative dualizing sheaf is $p^* \omega_X^{-1}$ (up to a shift) duality says
\[ \RHom{}{\G_\bullet}{\omega_X} = p_* \RHom{}{\G}{\struct{}[n]} \]
so 
\[ \Supp{}{R^i \shHom{}{\G_k}{\omega_X}} \subset \Supp{}{R^i \shHom{}{\G}{\struct{}[n]}} \]
Since the zero section has codimension $n$, this reduces the problem to proving that
\[ \codim_{T^* X} \Supp{}{R^i \shHom{}{\G}{\struct{}[n]}} \ge n + i - 1 \]
for all $i \ge 1$. On $T^* X$, we have a short exact sequence of coherent sheaves
\[ 0 \to \G \to \G^M \to \G^M / \G \to 0 \]
and hence a distinguished triangle
\[ \RHom{}{\G^M/\G}{\struct{}[n]} \to \RHom{}{\G^M}{\cO[n]} \to \RHom{}{\G}{\cO[n]} \to +1 \]
All three complexes are concentrated in nonnegative degrees, because the original sheaves are supported on a subset of codimension $\ge n$. Moreover, the middle term is a sheaf, because $\G^M$ is CM of dimension $n$. Hence we concude that
\[ R^i \shHom{}{\G}{\cO[n]} \cong R^{i+1} \shHom{}{\G^M/\G}{\cO[n]} \]
for $i \ge 1$ but $\G^M / \G$ is a sheaf so the support of the RHS has codimenson at least $n + i + 1$ proving the desired inequality. 
\end{proof}

\subsection{Proof of Theorem 9.2}

To fix a discrepancy in the indexing, we replace $M \in \HM_X(X, k)$ by its Tate twist $M(k-n) \in \HM_X(X, 2n-k)$ leaving the underlying regular holonomic $\D$-module $\cM$ unchanged, but replaces the filtration $F_\bullet \cM$ by the shift $F_{\bullet + n - k} \cM$. Similarly, we replace $\G_\bullet$ but the shift $\G_{\bullet + n - d}$. Then the assertions in (d) and (e) hold b construction. The assertion in (b) follows {\color{red} NEED TO REPLACE WITH AN INCLUSION} In prop 11.5 we showed $\Supp{}{\G} \subset S_f$ which proves (a). The remaining assertions in (c) has been established in Prop. 12.2. 


\subsection{Proof of Theorem A}

Note that since the conclusion is purely about $X$, we are allowed to change $Y$ as necessary.
\bigskip\\
\textbf{Step 1.} We reduce to the following situation: given a fiber space over $X$ as in the statement, and given any ample $A$ on $X$ we modify to get a new family $f' : Y' \to X$ so that there is $k_0 \ge 0$ so that $B = \omega_{Y' / X} \ot f'^* L^{-1}$ satisfies the necessary propety (some power has a section) with $L = A(-k_0 D_f)$. To this end, fix $m > 0$ so that 
\[ L_m := \det{f_* \omega_{Y/X}^{\ot m}} \]
is a big line bundle. Given any ample line bundle $M$ on $X$, we will produce a new family $f' : Y' \to X$, smooth over $U = X \sm D_f$ such that
\[ H^0(Y', \omega_{Y'/X}^{\ot m} \ot f'^* M^{-1}(k D_f)) \neq 0 \]
for some integer $k \ge 0$. In particular, we can take $M = A^{\ot m}$ and can increase $k$ to $k = k_0 m$ for some $k_0 \ge 0$ to win.
\par 
To show this, for $N$ big we can write
\[ L_m^{\ot N} \cong M \ot \struct{X}(B) \]
where $B$ is effective. Denote by $r_0$ the rank of $f_* \omega_{Y/X}^{\ot m}$ over $U$, where it is locally free (here we use Siu) and define $r := N \cdot r_0$. There is an inclusion of sheaves
\[ L_m^{\ot N} \embed (f_* \omega_{Y/X}^{\ot m})^{\ot r} \]
split over the locus where $f_* \omega_{Y/X}^{\ot m}$ is locally free. 
\par 
Use the fiber product trick: replace $Y$ be $Y^{(r)}$ a resolution of the main component of the $r$-fold fiber product which is also smooth over $U$. There is a morphism
\[ f_*^{(r)} \omega_{Y^{(r)}/X}^{\ot m} \to \left( (f_* \omega_{Y/X}^{\ot m})^{\ot r} \right)^{\vee \vee} \]
which is an isomorphism over $U$. Since $L_m^{\ot N}$ injects into the RHS and the morphism $f^{(r)}$ degenerates at most over $D_f$, it follows that there is an inclusion
\[ L_m^{\ot N}(-k D_f) \embed f_*^{(r)} \omega_{Y^{(r)} / X}^{\ot m} \]
for some integer $k \ge 0$. This implies that on $Y^{(r)}$ we have 
\[ H^0(Y^{(r)}, \omega_{Y^{(r)}/X}^{\ot m} \ot f^{(r)*} M^{-1}(k D_f)) \neq 0 \]
so we can replace $Y$ by $Y^{(r)}$.

\textbf{Step 2.} Fix now an ample $A$ on $X$. The considerations above show that, in order to prove the theorem, we can assume that there exists an integer $k_0 \ge 0$ such that some power of $\cN \ot f^* L^{-1}$ has a section for 
\[ L = A(-k_0 D_f) \]
But then Theorem 9.2 produces a graded $\cA_X$-module $\G_\bullet$ as in the statement which in particular is large with respect to $D_f$.

\begin{theorem}
Let $X$ be a smooth projective variety, and let $M$ be a pure Hodge module $M$ with strict support $X$ and underlying filtered $\D_X$-module $(\cM, F_\bullet \cM)$ which is generally a VGS of weight $k$. Assume that there exists a graded $\cA_X$-submodule $\G_\bullet \subset \gr_\bullet^F \cM$ which is large with respect ot a divisor $D$. Then at least of the followign holds
\begin{enumerate}
\item $D$ is big
\item there is $1 \le s \le k$ and $r \ge 1$ and a big coherent sheaf $\cH$ on $X$ and an inclusion
\[ \cH \embed (\Omega_X^1)^{\ot s} \ot \struct{X}(r D) \]
Moreover, if $X$ is not uniruled, then $\omega_X(D)$ is big. 
\end{enumerate}
\end{theorem}

\subsection{Questions for Mihnea}

\begin{enumerate}
\item why can't we prove the full Kovac's conjecture $\mathrm{Var}(f) \le \kappa(X,D_f)$ by taking the maximal variation quotient as in the proof of Vieweg's conjecture since you proved the general type case. 

\item 
\end{enumerate}


\section{Review of MHM}

\newcommand{\MF}{\mathrm{MF}}
\newcommand{\DD}{\mathbb{D}}
\newcommand{\MHM}{\mathrm{MHM}}

Recall that there is a category $\MF(X, \Q)$ of filtered holonomic $\D$-modueles with a $\Q$-structure (meaning a $\Q$-Perverse sheaf refining the derived solutions, i.e. De Rhram complex, perverse sheaf of the $\D$-module). Then we define a full subcategory 
\[ \HM_Z(X, w) \subset \MF(X, \Q) \]
of pure Hodge modules of strict support $Z$ and weight $w$. It turns out that the filtration by strict support is a direct sum so we get
\[ \HM(X, w) := \bigoplus_{Z \subset X} \HM_Z(X, w) \subset \MF(X, \Q) \]

\begin{theorem}
For any closed irreducible $Z \subset X$ restriction to a sufficiently small open subvarieties of $Z$ induces an equivalence
\[ \HM_Z(X, w) \cong \mathrm{VHS}_{\text{gen}}(Z, w - \dim{Z})^p \]
where the RHS is the category of polarizable variations of pure Hodge structures of weight $w - \dim{Z}$ define on smooth dense open subvarieties $U \subset Z$ (defined by taking the inductive limit over $U \subset Z$).
\end{theorem}

\subsection{Pushforwards}

Here $K$ is the perverse sheaf and $\alpha_M : \DR_X(\cM) \iso K \ot_{\Q} \CC$ is an isomorphism. Here ${}^{\m} \cH^i$ is the perverse cohomology:
\[ {}^{\m} \cH^i : D_c^b(X, A) \to \mathbf{HS}(X, A) \]
where Saito calls perverse sheaves ``hypersheaves''. 

\begin{theorem}
Let $f : X \to Y$ be a projective morphism of smooth projective varieties and $M = ((\M, F), K, \alpha_M) \in \HM_Z(X, w)$. let $\ell$ be the first Chern class of an $f$-ample line bundle. Then the direct image $f_*^{\D}(\M, F)$ as a filtered $\D$-module is strict and we have
\[ \cH^i f_* \M := (\cH^i f_*^{\D} (\M, F), {}^{\m} \cH^i f_* K, {}^{\m} \cH^i f_* \alpha_M) \in \HM(Y, w + i) \]
together wth the isomorphisms
\[ \ell^i : \cH^{-i} f_* M \iso \cH^i f_* \cM(i) \]
for $i > 0$. Moreover, if $S : K \ot K \to \DD_X(-w)$ is a polarization of $M$ then a polarization of the $\ell$-primative part 
\[ {}^P \cH^{-i} f_* M := \ker{\ell^{i+1}} \subset \cH^{-i} f_* M \]
is given by the restriction to the $\ell$-primative part of the induced pairing
\[ (-1)^{i (i-1)/2} {}^{\m} \cH f_* S \circ (\id \ot \ell^i) : {}^{\m} \cH^{-i} f_* K \ot {}^{\m} \cH^{-i} f_* K \to \DD_Y(i-w) \] 
\end{theorem}

\begin{rmk}
Note that usually people (e.g. Mihnea) writes $f_*^{\D} \cM$ as $f_+ \cM$.
\end{rmk}


\begin{theorem}[Saito, 3.6]
There are canonically defined functors $f_*, f_{!}, f^*, f^{!}, \psi_g, \varphi_{g,1}, \boxtimes, \DD, \ot, \mathrm{Hom}$ on the bounded derived categories $D^b \MHM(X)$ for smooth complex algebraic varieties $X$ so that we have the canonical isomorphisms $H^j f_* = \cH^j f_*$ etc where $f$ is a morphism of smooth complex algebraic varieties, $g \in \Gamma(X, \cO_X)$, and $H^j$ is the usual cohomology functor of the derived categories 
\end{theorem}

\begin{theorem}[Saito's decomposition theorem]
Let $f : X \to Y$ be a projective morphism between two smooth complex varieties. Let $M \in \HM(X, w)^p$ be a polarizable Hodge module of pure weight $w$. Then each $\cH^i f_* M \in HM(Y, w + i)^p$ is of pure weight $w + i$ and admits a non canonical isomorphism
\[ f_* M \cong \bigoplus_i H^i (f_* M) [-i] \]
\end{theorem}

Because of the strict support decomposition
\[ \HM(X, w)^p = \bigoplus_Z \HM_Z(X, w)^p \]
each $\cH^i(f_* M)$ admits futher decomposition by strict support. Due to the commutativity of $\gr^F_p \DR$ and $f_*$, we get $p(H^i f_* M) \ge p(f_* M) \ge p(M)$. 

\begin{rmk}
Note that if $a : X \to \Spec{\CC}$ is the structure map then $\Q^H_X := a^* \Q^H_{\mathrm{pt}}$ is called the trivial MHM but really this is not a Hodge module it lives in $D^b \MHM(X)$. When $X$ is smooth it is just a shift of pure Hodge module of weight $d := \dim{X}$ namely $\Q^H_X[d] \in \HM_X(X, d)$ is, supported in degree zero, the pure Hodge module of weight $d$ defined by
\[ (\struct{X}, F, \ul{\Q}[d], \alpha) \]
where
\[ \alpha : \ul{\Q}[d] \ot_{\Q} \CC \iso \Omega_X^{\bullet-d} = \DR_X(\struct{X}) \]
is the natural map. 
\end{rmk}



\section{Chudnovsky}

Let $S \subset \P^n$ be a finite set of points.

\begin{prop}
Suppose that $A \subset \P^n$ is a hypersurface of degree $d$ such that
\[ \mult_x A \ge k \quad \forall x \in S \]
then $S$ lies on a hypersurface of degree $\le \delta := \floor{\frac{dn}{k}}$.
\end{prop}

\begin{proof}
Consider the effective $\Q$-divisor $D = \tfrac{n}{k} A$ on $\P^n$. Then
\[ \mult_x D \ge n \quad \forall x \in S \]
and hence the multiplier ideal satisfies $\J(D) \subset \I_S$. Then
\[ \ell H - D \equiv (\ell - \tfrac{dn}{k}) H \]
is ample if $\ell \ge \delta + 1$. Since $K_{\P^n} = -(n+1) H$ we conclude that $\struct{\P^n}(\delta) \ot \J(D)$ is globally generated by Nadel vanishing. Indeed,
by Mumford regularity considerations, it suffices to show that
\[ H^i(\P^n, \struct{\P^n}(\delta - i) \ot \J(D)) = 0 \]
for $i > 0$. Nadel vanishing shows that
\[ H^i(\P^n, \struct{\P^n}(\delta - i) \ot \J(D)) = H^i(\P^n, \struct{\P^n}(\delta  + (n+1) - i) \ot \omega_{\P^n} \ot \J(D)) = 0 \]
whenever $(\delta + (n+1) - i) H - D$ is big and nef. This is true since $\delta + (n+1) - i \ge \delta + 1$. Hence there is a hypersurface in the linear series $\struct{\P^n}(\delta)$ that contains $S \subset Z(\J(D)) $. 
\end{proof}

\section{Deligne's Fixed Point Theorem}

Framework: let $f : X \to S$ be a smooth proper morphism and a good compactification $\bar{f} : \ol{X} \to \ol{S}$ so that the boundaries are snc divisors. Also assume that $S$ is a $K(\pi, 1)$. 


Rational version:
\begin{enumerate}
\item $E_2^{p,q} = H^q(S, R^p f_* \QQ) \implies H^{p+q}(X, \QQ)$ degenerates at $E_2$
\item If $\ol{X}$ is any smooth compactification of $X$ with $\ol{X} \sm X$ snc then
\[ H^n(\ol{X}, \QQ) \to H^n(X, \QQ) \to H^0(S, R^n f_* \QQ) \]
is surjective for all $n$. 
This is the same as saying that
\[ H^n(\ol{X}, \QQ) \to H^n(X_s, \QQ) \]
has image equal to the monodromy fixed points of $\pi_1(S)$. 
\end{enumerate}

We want an integral version: in the same setup let $\xi \in H^{2i}(X_s, \QQ)$ be an integral Hodge class 
\[ \xi \in H^{i,i}(X_s) \cap H^{2i}(X, \Z) \]
then the following are equivalent:
\begin{enumerate}
\item $\xi$ has finite orbit under the action $\pi_1(S, s) \acts H^{2i}(X_s, \QQ)$

\item $\xi$ extends to a class in $H^{2i}(X, \CC)$ after finite base change

\item $\xi$ extends to a Hodge class in $H^{2i}(X, \CC)$ after finite base change

\item $\xi$ extends to $H^{2i}(X_\Delta, \Omega_{X_\Delta}^{\ge i})$ where $\Delta$ is a contractible open neighborhood

\item the Gauss-Manin flat deformation $\xi$ lies in $F^i H^{2i}(\hat{X} / \hat{S})$ (meaning in the formal neighborhood).  
\end{enumerate}


\subsection{Rational version proof}

For (1) use relative Lefschetz. For (1) implies (2) consider the map
\[ H^n(\ol{X}, \QQ) \to H^i(X, \QQ) \to H^0(S, R^n f_* \QQ) = (R^n f_* \QQ)_s^{\pi_1(S)} = H^n(X_s, \QQ)^{\pi_1(S)} \]
From (1) the second map is surjective. Thus it is enough to show that
\[ H^n(\ol{X}, \QQ) \to H^n(X, \QQ) \xrightarrow{\psi} H^n(X_s, \QQ) \]
has the same image as $\psi$. The final thing is a pure Hodge structure of weight $n$. However, the weight $n$ parts of the first two terms are isomorphic by the construction of the mixed Hodge structure on $H^n(X, \QQ)$ hence they have the same image.

\subsection{Integral verison}

(1) is equivalent to (2) is easy. For (1) $\implies$ (3) we know 
\[ H^{2i}(\ol{X}, \QQ) \to H^{2i}(X_S, \QQ) \]
surjects onto monodromy invariants. Hence we can lift $\xi$ rationally to $H^{2i}(\ol{X}, \QQ)$ we want to know it is in the image of $H^{2i}(\ol{X}, \ZZ)$. The obstruction is $\bar{\xi} \in H^{2i}(\ol{X}, \QQ / \ZZ)$ but this is torison. But $\xi$ was originally integral so there is some $N$ such that 
\[ \bar{\xi} \in \ker{(H^{2i}(\ol{X}, \tfrac{1}{N} \Z / \Z) \to H^{2i}(X_s, \tfrac{1}{N} \Z / \Z)} \]
and then we will find a finite \etale cover to kill this kernel. Let $S' \to S$ be a finite \etale cover and $X' \to X$ the induced cover. Then we have a diagram,
\begin{center}
\begin{tikzcd}
H^{2i}(X', \QQ) \arrow[r] & H^{2i}(X', \tfrac{1}{N} \Z / \Z) 
\\
H^{2i}(X, \QQ) \arrow[r] \arrow[u] \arrow[d] & H^{2i}(X, \tfrac{1}{N} \Z / \Z) \arrow[u, "\varphi"] \arrow[d]
\\
H^{2i}(X_s, \QQ) \arrow[r] & H^{2i}(X_s, \tfrac{1}{N} \Z / \Z) 
\end{tikzcd}
\end{center}
To show $\varphi$ is zero it suffices to use the morphism of spectral sequences
\[ H^p(S, R^q f_* (\tfrac{1}{N} \Z / \Z)) \to H^p(S', R^q f'_* (\tfrac{1}{N} \Z / \Z)) \]
is zero. To do this, we need a lemma.

\begin{lemma}
Fix $q$, and $\LL$ a torsion local system. If $S$ is $K(\pi, 1)$ then there is a finite \etale cover such that $H^q(S, \LL) \to H^q(S', \LL)$ is the zero map
\end{lemma}

\section{A non-abelian version of Deligne's Fixed Point Theorem}

Essential here means the following: if $H \subset G$ normal subgroup say that a property holds \textit{essentially} for $(H, G)$ if it does for some finite index subgroup of $G$ containing $H$. 

\begin{example}
If $G = \pi_1(X)$ for $X$ smooth quasi-projective and there is a fibration $f : X \to S$ then $\pi_1(X_s) \to \pi_1(X)$ will have image a normal subgroup. Then a property holding ``essentially'' just means after a finite \etale cover of the base. 
\end{example}

Assume $f : X \to S$ is smooth projective over $\CC$ and $S$ is smooth quasi-projective then $f$ is a topological fibration so $\pi_1(S,s) \acts H^i(X_s, \ZZ)$. Classical definition:
\begin{enumerate}
\item $\zeta \in H^{2i}(X_s)$ is a \textit{Hodge cycle} is an element of the image
\[ H^{2i}(X_s, \ZZ) \to F^i H^{2i}(X, \CC) \cap W_i H^{2i}(X, \CC) \]

\item Artin $K(\pi, 1)$ we assume that $S$ is smooth quasi-projective and $K(\pi, 1)$ and there is a filtration
\[ \cdots \subset \Gamma_2 \subset \Gamma_1 \subset \pi_1(S, s) \]
and $\Gamma_i / \Gamma_{i+1}$. Artin proved that there is a basis of the topology of any smooth variety given by such $S$. 
\end{enumerate}

\subsection{Deligne's Fixed Point Theorem}

Let $f : X \to S$ be a smooth projective and $s \in S$. Let $S$ be an Artin $K(\pi,1)$ and $\xi \in H^{2i}(X_s, \CC)$ is Hodge. Then the following are equivalent
\begin{enumerate}
\item the oribit $\pi_1(S, s) \cdot \xi$ is finite
\item $\xi$ essentially extends as a Hodge class to $X$
\item on the formal scheme $\hat{X} / \hat{S}$ the Gauss-Manin flat deformation of $\xi$ lies in $F^i H^{2i}(\hat{X} / \hat{S})$
\end{enumerate}

Hint: if $S$ is an Artin $K(\pi, 1)$ then if $\LL$ is a local system with torsion coefficients then there is $S' \to S$ finite \etale killing the higher cohomology of $\LL$. 

\subsection{Nonabelian version}

The Hodge conjecture says that $\xi$ should come from a cycle class. This is the same as saying that $\xi$ dies when restricting away from something of codimension $i$. 

Replace $\xi$ by a $\CC$-local system $\LL$ underlying a $K$-polarized VHS where $K = \{ \QQ, \RR, \CC \}$ is a field and it carrying a $\ZZ$-structure. 

\begin{rmk}
If $S$ is a $K(\pi, 1)$ then $\pi_1(S, s)$ acts on the set of local systems on $X_s$. Indeed, the sequence
\[ \pi_2(S, s) \to \pi_1(X_s, x) \to \pi_1(X, s) \to \pi_1(S, s) \to 1 \]
so if $\pi_2(S, s) = 0$ 
means that we can lift an element $\sigma \in \pi_1(S, s)$ up to an element of $\pi_1(X_s, x)$ hence we get a conjugation $\sigma \cdot \LL$ which is well-defined up to isomorphism since they differ by an inner automorphism of $\pi_1(X_s, x)$. The construction only works if $\rho$ corresponding to $\LL$ factors through the image of $\pi_1(X_s, x) \to \pi_1(X, x)$.
\end{rmk}


Very nice remark: if $S$ is an Artin $K(\pi, 1)$ this will imply that the completion of the above sequence is actually also exact on the left. This is very interesting.

\begin{theorem}[EK'24]
Here $f : X \to S$ is a smooth good (there exists a relative compactification by a relative SNC divisor hence it is a topological fibration) holomorphic map. Let $S$ be an Artin $K(\pi, 1)$ and $X_s$ carries a $K$-PVHS $(\LL_s, F_s, Q_s)$ TFAE
\begin{enumerate}
\item $\pi_1(S, s) \cdot \LL_s$ is finite
\item $(\LL_s, F_s, Q_s)$ essentially extends to a $\Z$-$K$-PVHS on $X$
\item it extends to $X_{\Delta}$ for some contractible open $\Delta$
\item a formal version
\end{enumerate}
\end{theorem} 

\subsection{Main group theoretic tools}

Let $f : X \to S$ be good and $s \in S$. Consider
\begin{center}
\begin{tikzcd}
\pi_1(X_s, x) \arrow[r] \arrow[d, "\rho_s"] & \pi_1(X, s) \arrow[ld, "\rho"]
\\
\GL_r(\CC)
\end{tikzcd}
\end{center}
isomonodromic if: $\im{\rho_s} = \im{\rho}$. It is an algebraically isomonodromic deformation if
\[ \left( \ol{\rho_s(\pi_1(X_s, x))}^{\Zar} \right)^\circ = \left( \ol{\rho(\pi_1(X, x))}^{\Zar} \right)^\circ \]

\begin{theorem}
If $(\LL_s, F_s, Q_s)$ as in the main theorem then (1) is equivalent to (2') namely the group-theoretic condition: $\LL_s$ essentially extends as an algebraically isomonodromic deformation. Then (2') leads to a essentially ($\Z$) PVHS on $X$ (after a finite \etale cover of the base) 
\end{theorem}

This uses $\ell$-adic theory. We need only semisimplicity of the representation only not that it underlies a VHS. 

\subsection{Mumford-Tate Group}

Let $(\LL, F, Q)$ be a $(\Z)$ PVHS then we consider the smallest connected algebraic $\Q$-subgroup of $\GL(\LL_s)$ that contains $\lambda \mapsto \lambda^i \bar{\lambda}^{j}$ for $i + j = w$. 

For $f : X \to S$ good for each fiber we get a MT group. We want to glue these together to get a ``local system of mumford-tate groups''. 

Let $X$ be smooth quasi-projective / $\CC$ and $x \in X$. Let $(\LL, F, Q)$ be a ($\ZZ$) QPVHS. Then for each $x \in X$ we have $MT(\LL_x, F_x) \embed \GL(\LL_x)$. We also have the Tanakaian monodromy group
\[ M(\LL_x)^\circ \to \GL(\LL_x) \]
the Mumford-Tate group normalizes $M(\LL_x)$ so we can form the product of these groups. 

\begin{defn}
$MT((\LL, F), x) \embed \GL(\LL_x)$ is the product of $M(\LL_x)$ and $MT(\LL_x, F_x)$. 
\end{defn}

\begin{theorem}
This is a local system of groups as we move $x$. If $f : X \to S$ is a good holomorphic map and $s \in S$ and $(\LL_s, F_s, Q_s)$ is a ($\ZZ$) $\QQ$-PVHS then its monodromy group satisfies
\[ \MT( (\LL_s, F_s), x) = \MT( (\LL, F), x) \]
for an algebraically isomonodromic deformation. 
\end{theorem}

\section{Thoughts on Isotrivial fibrations}

If we can show that if $f : X \to S$ is smooth projective with fibers isomorphic then it is trivial over formal neighborhoods then we win. Indeed, then the isom scheme is locally trivial and hence flat as well. Since it is smooth, it is \etale locally trivial. 
\par 
Over $\CC$ the theorem that it is analytically locally trivial implies analytically locally trivial so we win. Does this hold more generally? Probably not think Moret-Bailly families. 


\section{Kodaira Vanishing In the Sense of Deligne-Illusie}

\begin{theorem}
Let $k$ be a field of characteristic $p$, and let $X$ be a smooth projective $k$-scheme. Let $L$ be an ample line bundle on $X$. If $X$ has pure dimension $d < p$ and lifts over $W_2(k)$ then,
\[ H^j(X, L \ot \Omega_X^i) = 0 \quad i + j > d \]
\end{theorem}

\begin{proof}
By Serre vanishing we have,
\[ H^j(X, L^{p^\ell} \ot \Omega_X^i) = 0 \]
for all $j > 0$ and $\ell \gg 0$. Therefore, by induction it suffices to prove the following
\begin{center}
if $M$ is a line bundle such that $H^j(X, M^{\ot p} \ot \Omega^i_X) = 0$ for all $i + j > d$ then $H^j(X, M \ot \Omega_X^i) = 0$ for all $i + j > d$.
\end{center}
Note that $\Frob_X^* M \cong M^{\ot p}$ via the $p$-linear map $M \to M^{\ot p}$ given by $m \mapsto m^{\ot p}$. Now $\Frob_X = F \circ F_k$ where $F_k : X^{(p)} \to X$ is the base change to $X$ of $\Frob_k : \Spec{k} \to \Spec{k}$. Therefore $F^* M' \cong M^{\ot p}$ where $M' = F_k^* M$. Therefore,
\[ M' \ot F_* \Omega_X^i = F_* (F^* M' \ot \Omega^i_X) = F_* (M^{\ot p} \ot \Omega^i_X) \]
hence since $F$ is finite (alternatively because it is a homeomorphism),
\[ H^j(X, M^{\ot p} \ot \Omega^i_X) = H^j(X^{(p)}, F_* (M^{\ot p} \ot \Omega^i_X)) = H^j(X^{(p)}, M' \ot F_* \Omega_X^i) \]
and by assumption these are zero for $i + j > d$. However, there is a spectal sequence computing the hypercohomology of $M' \ot F_* \Omega^\bullet_X$,
\[ E_1^{i,j} = H^j(X^{(p)}, M' \ot F_* \Omega_X^i) \implies H^{i+j}(X^{(p)}, M' \ot F_* \Omega_X^\bullet) \]
and the vanishing hence implies the vanishing of $H^n(X^{(p)}, M' \ot F_* \Omega^\bullet_X) = 0$ for $n > d$. But the Cartier operator induces a decomposition,
\[ F_* \Omega_X^\bullet \cong \bigoplus_i \Omega_{X^{(p)}}^i[-i] \]
therefore
\[ H^{n}(X^{(p)}, M' \ot F_* \Omega_X^\bullet) \cong \bigoplus_{i + j = n} H^j(X^{(p)}, M' \ot \Omega_{X^{(p)}}^i) \]
and hence we get the vanishing,
\[ H^j(X^{(p)}, M' \ot \Omega_{X^{(p)}}^i) = 0 \quad i + j > d \]
Hence it just remains to transfer this information to $X$. Since $F_k$ is flat, by flat base change, 
\[ F^*_k H^i(X, \G) \iso H^i(X^{(p)}, F_k^* \G) \]  
for any coherent sheaf $\G$. Furthemore, by compatibility of $\Omega_X^i$ with base change we get,
\[ F_k^* (M \ot \Omega_X^i) = M' \ot \Omega_{X^{(p)}}^i \]
and therefore we see,
\[ F_k^* H^j(X, M \ot \Omega_X^i) \iso H^j(X^{(p)}, M' \ot \Omega_{X^{(p)}}^i) = 0 \]
completing the proof.
\end{proof}

\subsection{Frobenius Splitting}

\begin{defn}
Let $X$ be a scheme of characteristic $p > 0$. Then $X$ is \textit{globally F-split} if the Frobenius map $\struct{X} \to F_{X*} \struct{X}$ is left split. 
\end{defn}

\begin{prop}
Let $X$ be a globally $F$-split scheme and $\L$ an ample line bundle on $X$. Then $H^i(X, \L) = 0$ for all $i > 0$.
\end{prop}

\begin{proof}
Indeed, by the projection formula the unit map $\L \to F_{X*} F_X^* \L$ is $- \ot \L$ applied to $F_X^{\#}$ and hence is split. Therefore, there are injections,
\[ H^i(X, \L) \embed H^i(X, F_{X*} F_X^* \L) = H^i(X, F_X^* \L) \]
because $F_X$ is affine. Furthermore $F_X^* \L \cong \L^{\ot p}$ via $s \mapsto s^{\ot p}$. Iterating we get injections,
\[ H^i(X, \L) \embed H^i(X, \L^{\ot p}) \embed H^i(X, \L^{\ot p^2}) \embed \cdots \]
By Serre vanishing we see that $H^i(X, \L^{p^n}) = 0$ for $n \gg 0$ and hence $H^i(X, \L) = 0$.
\end{proof}

\begin{rmk}
The same argument shows that if $X$ has Serre duality then $H^i(X, \L^{-1}) = 0$ for $i < \dim{X}$.
\end{rmk}

\begin{rmk}
If $f : X \to S$ is a morphism of schemes of characteristic $p$ then $F_X = F_S \circ F_{X/S}$ factors as relative Frobenius $F_{X/S} : X \to X'$ followed by the Frobenius on $S$ base changed to $F_S : X' \to X$. Therefore, 
\begin{center}
\begin{tikzcd}
\struct{X} \arrow[rr, "F_X^{\#}"] \arrow[d, equals] & & F_{X*} \struct{X} \arrow[d, equals]
\\
\struct{X} \arrow[r, "F_S^{\#}"] & F_{S*} \struct{X'} \arrow[r, "F_{S*} F_{X/S}^\#"] &  F_{S*} F_{X/S*} \struct{X}
\end{tikzcd}
\end{center} 
Therefore, if $F_S^{\#}$ and $F_{X/S}^{\#}$ are split then $X$ is globally $F$-split. Suppose that $F_{X/S}^{\#}$ is split and $S$ is globally $F$-split. Let $r : F_{S*} \struct{S} \to \struct{S}$ be a retraction on $S$ and consider,
\begin{center}
\begin{tikzcd}
X' \arrow[r, "F_S"] \arrow[d, "f'"] & X \arrow[d, "f"]
\\
S \arrow[r, "F_S"] & S
\end{tikzcd}
\end{center}
so $f^* r : f^* F_{S*} \struct{S} \to \struct{X}$ is a section of $f^* F_S^{\#} : \struct{X} \to f^* F_{S*} \struct{S}$ consider the diagram (the left square is just given by commutativity of the above square via adjunction $f$)

\begin{center}
\begin{tikzcd}
f^* \struct{S} \arrow[d, equals] \arrow[r, "f^* F_S^{\#}"] & f^* F_{S*} \struct{S} \arrow[d] \arrow[r, "f^* r"] & f^* \struct{S} \arrow[d, equals]
\\
\struct{X} \arrow[r, "F_{S*}^{\#}"] & F_{S*} \struct{X'} \arrow[r, dashed] & \struct{X}
\end{tikzcd}
\end{center}
the composition along the top row is $\id$ so if the dashed map exists then $X$ is globally $F$-split. This happens when the natural map $f^* F_{S*} \struct{S} \to F_{S*} \struct{X'}$ is an isomorphism which occurs when $f$ is flat or $S$ is regular so $F_S$ is finite locally free. 
\end{rmk}

\subsection{Some Homological Algebra}

\begin{defn}
A chain complex $(C_{\bullet}, \partial)$ is \textit{split} if there maps $s_n : C_n \to C_{n+1}$ such that
\[ \partial_n = \partial_n \circ s_{n-1} \circ \partial_n \]
\end{defn}

\begin{lemma}
$C_\bullet$ is split and exact if and only if $\id_C$ is chain homotopic to $0$.
\end{lemma}

\begin{proof}
Suppose $s$ is a chain homotopy between $\id_C$ and $0$ meaning
\[ \id_n = \partial_{n+1} \circ s_n + s_{n-1} \circ \partial_{n} \]
then composing with $\partial_n$ on the left we get
\[ \partial_n = \partial_n \circ s_{n-1} \circ \partial_n \]
hence $s$ forms a splitting. Furthermore, $\id = 0$ on homology so $C_\bullet$ is exact.
\par 
Conversely, let $s$ be a splitting and assume $C_\bullet$ is exact. We will modify $s$ to obtain a nullhomotopy $h$. First, consider
\[ e_n = \id_n - (\partial_{n+1} \circ s_n + s_{n-1} \circ \partial_n) \]
By splitness
\[ \partial_n \circ e_n = 0 \text{ and } \quad e_{n} \circ \partial_{n+1} = 0 \]
Thus
\[ \im{e_n} \subset \ker{\partial_{n}} = \im{\partial_{n+1}} \subset \ker{e_{n}} \]
using exactness of the complex. This implies $e_n^2 = 0$. Therefore,
\[ e_n^2 = [\id_n - (\partial_{n+1} \circ s_n + s_{n-1} \circ \partial_n)]^2 = 0 \]
this equals
\[ \id_n - 2 s_{n-1} \circ \partial_n - 2 \partial_{n+1} \circ s_n + s_n \circ \partial_n \circ s_{n-1} \circ \partial_n + \partial_{n+1} \circ s_n \circ s_{n-1} \circ \partial_n + \partial_{n+1} \circ s_n \circ \partial_{n+1} \circ s_n \]
using that $\partial^2 = 0$. 
Therefore, using the splitting relation this means,
\[ \partial_{n+1} \circ s_n + s_{n-1} \circ \partial_n = \id_n + \partial_{n+1} \circ s_n \circ s_{n-1} \circ \partial_n \]
Therefore, define
\[ h_n = s_n - \partial_{n+2} \circ s_{n+1} \circ s_n \]
is a map $C_n \to C_{n+1}$. 
Now we verify
\[ \partial_{n+1} \circ h_n + h_{n-1} \circ \partial_n = \partial_{n+1} \circ s_n + s_{n-1} \circ \partial_n - \partial_{n+2} \circ s_{n+1} \circ s_n \circ \partial_n = \id_n \]
\end{proof}

\subsection{Bott Vanishing}

The argument is adapted from \chref{https://projecteuclid.org/journals/tohoku-mathematical-journal/volume-49/issue-3/The-Frobenius-morphism-on-a-toric-variety/10.2748/tmj/1178225109.full}{``The Frobenius morphism on a toric variety''}. Fix a perfect field $k$ of characteristic zero.  Let $X$ be a smooth $k$-scheme and $\wt{X}$ a flat lift over $W_2(k)$. 

\begin{lemma}
If $X$ admits a global lift of $F$ over $W_2(k)$ there is a quasi-isomorphism of complexes
\[ \varphi : \bigoplus_{i \ge 0} \Omega_{X^{(p)}}^i[-i] \to F_* \Omega^\bullet_X \]
induced by split inclusions termwise.
\end{lemma}

\begin{proof}
The maps 
\[ \varphi^i : \Omega^i_{X^{(p)}} \to F_* \Omega^i_X \]
are defined be Deligne-Illuse that form a map of complexes (meaning their images lie in the cocycles). For $\varphi^1$ it was defined as follows: let $\wt{F}$ be the lift of Frobenius then
\[ \wt{F}(a') = a^p + p \cdot \psi(a') \]
where $a'$ is the pullback along $X^{(p)} \to X$ of a local section of $\struct{X}$. Then we define
\[ \varphi^1(\d{a'}) = a^{p-1} \d{a} + \d{\psi(a')} \]
which is closed and well-defined even before quotienting by the boundaries. Notice that this is actually a split inclusion. It is clear that it splits the Cartier operator in
\[ 0 \to B_X^i \to Z_X^i \xrightarrow{C} \Omega_{X^{(p)}}^i \to 0 \]
{\color{red} finish this}
\end{proof}

\begin{prop}
Let $X$ be a projective normal variety of  dimension $n$ over a perfect field $k$ of characteristic $p > 0$. Suppose that $X$ lifts to $W_2(k)$ and admits a global lift of $F$ to $W_2(k)$. Then for any $\L$ an ample line bundle on $X$ we have $H^j(X, \Omega_X^{[i]} \ot \L) = 0$ for all $j > 0$ and $i \ge 0$.
\end{prop}

\begin{proof}
Let $j : U \embed X$ be the inclusion of the smooth locus of $X$.
The lemma gives us split inclusions
\[ \Omega^i_{U^{(p)}} \embed F_* \Omega_U^i \]
Notice that
\[ j_* F_* \Omega_U^i = F_* \Omega_X^{[i]} \]
since $F$ and $j$ commute and $\Omega_X^{[i]} = j_* \Omega_U^i$. Therefore, applying $j_*$ gives a split inclusion
\[ \Omega^{[i]}_{X^{(p)}} \embed F_* \Omega_X^{[i]} \]
Let $\L'$ be the pullback of $\L$ along $X^{(p)} \to X$. Then $F^* \L' = \L^{\ot p}$. Tensoring with $\L'$ we get an injection
\[ F_k^* H^j(X, \Omega_X^{[i]} \ot \L) = H^j(X^{(p)}, \Omega_{X^{(p)}}^{[i]} \ot \L') \embed H^j(X^{(p)}, F_* \Omega_X^{[i]} \ot \L') = H^j(X, \Omega_X^{[i]} \ot \L^{\ot p} \]
where the last step we use the projection formula
\[ F_* \Omega_X^{[i]} \ot \L' = F_* (\Omega_X^{[i]} \ot F^* \L') = F_* (\Omega_X^{[i]} \ot \L^{\ot p}) \]
and the fact that $F : X \to X^{(p)}$ is finite so pushforward gives an isomorphism on cohomology. Iterating this process and applying Serre vanishing we see that $H^j(X, \Omega_X^{[i]} \ot \L) = 0$ for all $j > 0$ and all $i \ge 0$.
\end{proof}

\section{Obstructions for Subschemes}

\newcommand{\Ob}{\mathrm{Ob}}

Let $X$ be a smooth proper scheme over an artin ring $R$ and $Y \subset X$ a regularly embeded subscheme. Let $R' \onto R$ be a square-zero extension with kernel $J$. We want to understand the obstruction map
\[ \ob : \Def_X \to \Ob_{Y \subset X} \]
given by sending a deformation $X'$ of $X$ over $R'$ to the obstructions to deforming $Y$ inside $X'$.
\par 
This works as follows, let $U_i$ be an open affine cover of $X$ and $\xi \in H^1(\T_X)$ represented by a \cech cocycle $\xi_{ij} \in H^0(U_{ij}, \T_X \ot J)$ corresponding to the deformation as follows. Fix deformations $U'_i$ of $U_i$. Then because $U'_i |_{U_{ij}}$ and $U_j'|_{U_{ij}}$ are both deformations of $U_{ij}$ which is a smooth affine schemes they are isomorphic via
\[ \varphi_{ij} : U'_i |_{U_{ij}} \iso U_j'|_{U_{ij}} \]
If the $\varphi_{ij}$ satisfy the cocycle condition they define a particular deformation $X'$. The difference between two deformations is given by the different choices of $\varphi$ and the difference between two is a \cech cocycle $\xi_{ij} \in H^0(U_{ij}, \T_X)$. Write $X' + \xi$ for the deformation defined by
\[ \varphi_{ij} + \xi_{ij} :  U'_i |_{U_{ij}} \iso U_j'|_{U_{ij}} \]
where what this really means is we take the same homeomorphism and modify the ring map:
\[ \varphi_{ij}^{\#} + \xi_{ij} \circ \d : \struct{U_j'}|_{U_{ij}} \to \struct{U'_i} |_{U_{ij}} \]
and one checks that this is a ring map using the fact that $J^2 = 0$.
\par 
Now we involve the subscheme. Let $V_i = Y \cap U_i$ which is affine so we can fix a choice of deformation $\iota_i' : V_i' \embed U_i'$ locally (see Hartshorne Deformation Theory chapter 2). Writing $A$ and $B$ for the coordinate rings of $U_i$ and $V_i$ respectively the difference between two maps $V_i \embed U_i$ fit into a diagram
\begin{center}
\begin{tikzcd}
0 \arrow[r] & J \ot_R A \arrow[d] \arrow[r] & A' \arrow[d, "\psi"] \arrow[r] & A \arrow[d] \arrow[r] & 0
\\
0 \arrow[r] & J \ot_R B \arrow[r] & B' \arrow[r] & B \arrow[r] & 0 
\end{tikzcd}
\end{center}
then the difference between two maps $\psi$ is a derivation
\[ A' \to J \ot_R B \]
hence it factors through $\Omega_{A/R} \to J \ot_R B$. However, we get the same closed subscheme composing with any automorphism of $B'$ as a deformation. These modify the map by adding a term $\Omega_{B/R} \to J \ot_R B$ so the quotient lives in $I/I^2 \to J \ot_R B$ where $I = \ker{(A \to B)}$. The point is that two deformations of $V_i \embed U_i$ differ by a section $\eta_i \in H^0(U_i, \cN_{Y|X})$. 
\par 
Suppose we choose deformations $V_i' \embed U_i'$ then we compare the closed subschemes via the identiications $\varphi_{ij} + \xi_{ij}$. This produces sections $\eta_{ij} \in H^0(U_{ij}, \cN_{Y|X})$ satisfying the cocycle condition whose nontriviality measures the obstruction to modifying so that the subschemes glue. To fix notation, let $A$ and $B$ be the coordinate rings of $U_{ij}$ and $V_{ij}$ and $A_i'$ be the coordinate ring of $U'_i|_{U_{ij}}$ and $B'_i$ the coordinate ring of $V'_i|_{U_{ij}}$ and likewise for $j$. Write $\psi_i : A_i' \onto B_i'$ for the quotient maps and $\varphi_{ij} + \xi_{ij} \circ \d : A_j' \to A_i'$ for the identification arising from the deformation of $X$. The discrepancy is given by
\[ \psi_i \circ (\varphi_{ij} + \xi_{ij} \circ \d) - \psi_j : A'_j \to J \ot_R B \]
which is a derivation and kills $J$ so it defines a map $\eta : I/I^2 \to \Omega_{A/R} \ot_A B \to J \ot_R B$. Notice that $\psi_i$ and $\psi_j$ agree modulo $J$ (they are both deformations of $\psi : A \onto B$) so $\psi_i \circ \xi_{ij} \circ \d = \psi \circ \xi_{ij} \circ \d$ therefore
\[ \eta = (\psi_i \circ \varphi_{ij} - \psi_j)|_{I/I^2} + \psi \circ \xi_{ij}|_{I/I^2} \]  
and the first term is the obstruction when $\xi = 0$ so we get
\[ \ob(X' + \xi) = \ob(X') + \psi \circ \xi |_{I/I^2} \in H^1(Y, \cN_{Y|X}) \]
and the second term is exactly the image of $\xi$ under the composition of the natural maps
\[ H^1(\T_X) \to H^1(\T_X|_Y) \to H^1(\cN_{X|Y}) \]
where at the level of \cech cycles, the first map is post-composing with $\psi$ (restricting $\xi$ to $Y$) and the second map is restricting $\xi$ to the conormal bundle (or alternatively viewing it as a tangent vector and taking it modulo $\T_Y$). In particular, suppose $R' \to R$ is a split extension then taking $X'$ to be the trivial (split) deformation $\ob(X') = 0$. Hence we see that
\[ \Def_X \to \Ob_{Y \subset X} \]
is canonically identified with
\[ H^1(\T_X) \to H^1(\cN_{Y|X}) \]
where the $H^1(\T_X)$-torsor $\Def_X$ is pointed at the trivial deformation $X'$.

\section{Andrew's Talk}

\renewcommand{\sp}{\mathrm{sp}}

Let $f : \X \to B$ be a smooth proper family with $B$ over a field $k$ of characteristic zero. Consider
\[ B_{sb} = \{ b \in B \mid X_{\bar{b}} \text{ is stably rational} \} \]
is a subset of $B$. 

\begin{theorem}[Nicaise-Schinder]
$B_{sb}$ is a countable union of closed subvarieties of $B$.
\end{theorem}

\begin{proof}
By general arguments, $B_{sb}$ is a countable union of locally closed subvarieties the point is to show closure under specialization. After some reductions, we only need $f : \X \to \Spec{k[[t]]}$ smooth proper want to show if the generic fiber is stably rational then $\X_0$ is stably rational. To complete the proof we use the following.
\end{proof}

\begin{theorem}
There exists a specialization map
\[ \sp : K_0(\Var_{k((t))}) \to K_0(\Var_k) \]
unique amoung those group maps satisfying
\begin{enumerate}
\item if $X$ is smooth and proper over $k((t))$, choose a regular model $\X \to \Spec{k[[t]]}$ with general fiber $X$ and special fiber
\[ X_0 = \bigcup_{i \in I} d_i D_i \]
such that the reduction is an SNC divisor, then we require
\[ \sp([X]) = \sum_{\varnothing \neq J \subset I} (-1)^{\# J - 1} [D_J \times \P^{\# J -1}] \]
where
\[ D_J = \bigcap_{j \in J} D_j \]
\end{enumerate}
\end{theorem}

\newcommand{\SB}{\mathrm{SB}}
\newcommand{\Bir}{\mathrm{Bir}}

\begin{theorem}[Larsen-Lunts]
$K_0(\Var_k) / [\A^1] \iso \Z[\mathrm{SB}_k]$ where $\mathrm{SB}_k$ is the monoid of stable birational classes under product. 
\end{theorem}

\begin{cor}
Stable rationality specializes for smooth proper maps.
\end{cor}

\begin{proof}
Since $f : \X \to \Spec{k[[t]]}$ is smooth and proper $\X_0$ is irreduclbe so
\[ \sp([\X_\eta]) = [\X_0] \]
but since $[\X_{\eta}]$ is stably rational. Now we get a diagram,
\begin{center}
\begin{tikzcd}
K_0(\Var_{k((t))}) \arrow[r] \arrow[d] & K_0(\Var_k) \arrow[d]
\\
\Z[\SB_{k((t))}] \arrow[r] & \Z[\SB_k] 
\end{tikzcd}
\end{center}
but $\X_{\eta}$ is stably rational so its image is
\[ [\Spec{k((t))}] \mapsto [\Spec{k}] \]
and therefore $[\X_0] \mapsto [\Spec{k}]$ moding by $[\A^1]$ hence $[\X_0]$ is stably rational. 
\end{proof}

\begin{rmk}
\begin{enumerate}
\item Kontsevich-Tschinkel constructed
\[ \sp : \Z[\Bir_{k(t)}] \to \Z[\Bir_k] \]
to remove the adjective stable. 

\item Nicaise-Ottem unified these specializations with
\[ \sp : K_0(\Var^{\dim}_{k((t))}) \to K_0(\Var_k^{\dim}) \]
on the graded grothendieck group.
\end{enumerate}
\end{rmk}

\subsection{Graded Grothendieck Ring}

\newcommand{\PP}{\mathbb{P}}

\begin{defn}
Let $B$ be a variety. Let $K_0(\Var_B^{\dim})$ be the graded abelian group with generators
\[ [X \xrightarrow{f} B]_d \]
with $d \ge \dim{X} - \dim{B}$ where $f : X \to B$ is a morphism from a variety. The relations are
\[ [X \xrightarrow{f} B]_d = [D \xrightarrow{f} B]_d + [U \xrightarrow{f} B]_d \]
where $D \embed X$ is a subvariety and $U = X \sm U$. 
\end{defn}

Note that $K_0(\Var_B^{\dim})$ is a module over $K_0(\Var_k^{\dim})$ via
\[ [Y]_{e} \cdot [X \xrightarrow{f} B]_d = [X \times Y \to X \to B]_{d+e} \]

\begin{defn}
Let $K_0(\Var_B^{sp})$ be the graded subgroup generated by 
\[ [X \xrightarrow{f} B]_{\dim{X} - \dim{B}} \]
where $X$ is smooth and $f$ is proper. 
\end{defn}

\begin{rmk}
Then $K_0(\Var_B^{sp})$ is a graded $K_0(\Var^{sp}_k)$-module. Note that sp stands for ``smooth proper''.
\end{rmk}
\noindent
Let $\LL = [\A^1_k]_1$ and $\tau = [\Spec{k}]_1$ both live in $K_0(\Var_k^{\dim})$. Note that $\id = [\Spec{k}]_0$ so $\tau \neq \id$. 

\begin{rmk}
There is a surjective forgetful map
\[ K_0(\Var^{\dim}_k) \onto K_0(\Var_k) \]
given by $[X]_d \mapsto [X]$ where the kernel is $(\tau - 1)$. Also
\[ K_0(\Var_k^{\dim}) \to \Z[\Bir_k] \]
defined by
\[ [X]_d \mapsto \begin{cases}
[X] & \dim{X} = d
\\
0 & \dim{X} < d
\end{cases} \]
whose kernel is $(\tau)$. 
\end{rmk}

\begin{theorem}
There exists an isomorphism
\[ K_0(\Var_B^{\dim}) \iso K_0(\Var_k^{sp}) \ot_{\Z[\tau \LL, \tau + \LL]} \Z[\tau, \LL] \]
\end{theorem}

\begin{rmk}
Note that $\tau \LL = [\PP^1 \times \PP^1]_2 - [\PP^2]_2$ and $\tau + \LL = [\P^1]_1$. Therefore,
\[ K_0(\Var_B^{\dim}) \cong \frac{K_0(\Var_B^{sp})[\tau, \LL]}{(\tau + \LL = [\P^1]_1, \tau \LL = [\P^1 \times \P^1]_2 - [\P^2]_2)} \cong \frac{K_0(\Var_B^{sp})[\tau]}{(\tau^2 - \tau \cdot [\P^1]_1 + ([(\P^1)^2]_2 - [\P^2]_2))} \]
\end{rmk}

There are projections
\begin{center}
\begin{tikzcd}
& K_0(\Var_B^{\dim}) \arrow[ld] \arrow[rd]
\\
K_0(\Var_B^{sp}) & & K_0(\Var_B^{sp}) 
\end{tikzcd}
\end{center}
the maps send $\alpha + \beta \tau \mapsto (\alpha, \beta)$. There is an involution corresponding to the quadratic extension
\[ \DD : \alpha + \beta \tau \mapsto \alpha + \beta \LL \]

\begin{theorem}
$K_0(\Var_k) / (\LL) \cong \ZZ[\SB_k]$
\end{theorem}

\begin{proof}
$K_0(\Var_k)/(\LL) \cong K_0(\Var_k^{\dim}) / (\tau - 1, \LL)$ then applying $\DD$ we get
\[ K_0(\Var_k)/(\LL) \cong K_0(\Var_k^{\dim}) / (\tau - 1, \LL) \cong_{\DD} K_0(\Var_k^{\dim}) / (\LL - 1, \tau) \cong \ZZ[\Bir_k]/([\PP^1] - [\Spec{k}]) = \ZZ[\SB_k] \]
because as a monoid
\[ \SB_k = \Bir_k / ([\P^1] - [\Spec{k}]) \]
\end{proof}

\begin{prop}
Now consider the open and closed embeddings
\[ j : \Spec{k((t))} \to \Spec{k[[t]]} \leftarrow \Spec{k} : i \]
then there is a pushforward by composition
\[ j_! : K_0(\Var_{k((t))}^{\dim}) \to K_0(\Var^{\dim}_{k[[t]]}) \]
we form a diagram
\begin{center}
\begin{tikzcd}
K_0(\Var_{k((t))}^{\dim}) \arrow[rdd, "\sp"'] \arrow[r, "j_!"] &  K_0(\Var^{\dim}_{k[[t]]}) \arrow[d, "-\pi_2"]
\\
& K_0(\Var^{\dim}_{k[[t]]}) \arrow[d, "i^*"]
\\
& K_0(\Var_k^{\dim})
\end{tikzcd}
\end{center}
where $-\pi_2$ is the map $\alpha + \beta \tau \mapsto - \beta$ and then we include $K_0(\Var_{k[[t]]}^{sp}) \embed K_0(\Var_{k[[t]]}^{\dim})$. 
\end{prop}

\begin{proof}
For $X$ smooth and proper over $k((t))$ let $d = \dim{X}$. Then we get 
\[ j_! [X]_{d} = [\X]_{d} - [X_0]_d \]
for any regular model. Note that $[X_0]_d = [X_0]_{d-1} \tau$ and $[X_0] := [X_0]_{d-1} \in K_0(\Var^{Sp}_{k[[t]]})$ since $X_0$ has relative dimension $d-1$ over the DVR. Then we write
\[ \X_0 = \bigcup_{i \in I} d_i D_i \]
and therefore by inclusion-exclusion
\[ [\X] - [\X_0] \cdot \tau = [\X] - \sum_{\varnothing \neq J \subset I} (-1)^{\# J - 1} [D_J] \, \tau^{\# J} \]
where again we had to perform a shift: 
\[ [D_J] := [D_J]_{d - \# J} \quad \text{ thus } [D_J]_{d} = [D_J]_{d} \, \tau^{\# J} \]
so that $[D_J] \in K_0(\Var^{sp}_{k[[t]]})$. 
Notice
\[ \tau^{\# J} = \tau \cdot [\P^{\# J - 1}] - \tau \LL \cdot [\P^{\# J - 2}] \]
because 
\[ \tau^{k} = \left( \tau^{k} + \tau^{k-1} \LL + \cdots + \LL^{k} \right) - \LL \left(\tau^{k-1} + \tau^{k-2} \LL + \cdots + \LL^{k-1} \right) = [\PP^k] - \LL [\PP^{k-1}] \]
Recall $\tau \LL \in K_0(\Var_k^{sp})$ because
\[ \tau \LL = [\A^1]_2 = [\P^1 \times \P^1]_2 = [\P^2]_2 \]
and therefore 
\begin{align*}
[\X] - [\X_0] \cdot \tau & = [\X] - \sum_{\varnothing \neq J \subset I} (-1)^{\# J - 1} [D_J] \, \tau^{\# J} 
\\
& = [\X] + \tau \LL \sum(-1)^{\# J - 1} [D_J \times \P^{\# J - 2}] - \tau \sum_{\varnothing \neq J \subset I} (-1)^{\# J - 1} [D_J \times \P^{\# J - 1}]
\end{align*}
the first two terms lie in $K_0(\Var^{sp}_{k[[t]]})$ and the last term is the $\tau$-part so applying $-\pi_2$ recovers
\[ \sum_{\varnothing \neq J \subset I} (-1)^{\# J - 1} [D_J \times \P^{\# J - 1}]_{d-1} \]
in $K_0(\Var_{k[[t]]}^{\dim})$. Then applying the fiber product pullback
\[ i^* : K_0(\Var_{k[[t]]}^{\dim}) \to K_0(\Var_k^{\dim}) \]
which shifts degrees up by $1$ we get 
\[ \sum_{\varnothing \neq J \subset I} (-1)^{\# J - 1} [D_J \times \P^{\# J - 1}]_{d} = \sp([X]) \]
\end{proof}


\begin{defn}
Say $X$ has $\DD$-rational singularities if $\id = [X \to X]_0 \in K_0(\Var_X^{\dim})$ lies in $K_0(\Var_X^{sp})$ i.e. $\pi_2(\id) = 0$. This is the same as $\DD(\id) = \id$.
\end{defn}

\begin{example}
The following have $\DD$-rational singularities
\begin{enumerate}
\item $X$ smooth
\item $C$ a cuspidal curve because $\id = [\wt{C} \to C]$ since if $C^0$ is the component of the cusp then $[C] = [C^0] + [*]$ and $[\wt{C}] = [C^0] + [*]$ since the normalization map is a homeomorphism
\item $X = C(Z)$ the cone over smooth proper $Z$ then the blowup of the vertex $v$ gives a resolution $\wt{X} \to X$ then
\[ \id = [X \to X] = [\wt{X} \to X]_0 - [Z \to X]_{-1} \cdot \tau + [v \to X]_{-n} \cdot \tau^n \] 
but $[Z \to X]$ is in smooth and proper and so is $[v \to X]$. Now we write
\[ \tau^n = \tau [\PP^{n-1}] - \tau \LL \cdot [\PP^{n-2}] \]
since the second is in $K_0(\Var_X^{sp})$ applying $\pi_2$ gives
\[ 0 - [Z \to X]_1 + [\PP^{n-1} \to X]_1 \]
where both terms collapse to $v$. Therefore $X$ is $\DD$-rational iff $[Z] = [\PP^{n-1}]$ in $K_0(\Var_k)$. For example this is true if $Z = \P^{n-1}$ or $Z$ is an odd-dimensional quadric. 
\end{enumerate}
\end{example}

\textbf{Question}: if $X$ is smooth and $m \ge 1$ are the symmetric powers $S^m(X)$ $\DD$-rational. 


There is a map
\[ K_0(\Var_X^{\dim}) \to K_0(\MHM(X)) \]
given by
\[ [Y \xrightarrow{f} X]_d \mapsto [f_! \Q_Y^H] \]
If $X$ has $\DD$-rational singularities then $[\DD \Q_X^H] = [\Q_X^H]$. Recall $X$ is a rational homology manifold iff $\DD \Q_X^H = \Q_X^H$.

\textbf{Question}: if $X$ has $\DD$-rational singularities is it a rational homology manifold?

\newcommand{\St}{\mathbf{St}}

Furthermore, there is a map
\[ K_0(\MHM(X)) \to K_0(X)[t] \]
given by
\[ [M] \mapsto \sum_{j \ge 0} [\mathrm{gr}^F_j \DR(M)] t^j \]

\subsection{Grothendieck Ring of Stacks}

$K_0(\St_B^{\dim})$ is defined the same way for open substacks but we add one more relation. If $E \to X$ is a rank $d$ vector bundle on a stack $X$ we impose
\[ [E]_n = [X \times \A^d]_n \]
There is a morphism
\[ K_0(\Var_k^{\dim}) \to K_0(\St_k^{\dim}) \cong K_0(\Var_k^{\dim})[(\tau \LL)^{-1}, [\PP^k]^{-1})_{k \in \NN}] \]
We show
\[ S^m(X) = [ [X^m / S_m] ] \in K_0(\St^{sp}_k) \]
which relies on some representation theory of the symmetric group. Therefore, $\pi_2(\id_{S^m(X)})$ must map to zero in the localization so is killed by the elements we are localizing by.


\section{Weird Proper Varieties}

\begin{defn}
Recall for any variety $X$ we can consider $N_1 = Z_1(X)_{\RR} / \mathrm{Num}(X)$ where $\mathrm{Num}(X)$ are the numerically trivial real cycles meaning they have trivial intersection numbers with any Carier divisor $D$. There is a perfect pairing 
\[ N_1(X) \times (\NS(X) \ot_{\ZZ} \RR) \to \RR \]
The Theorem of the Base asserts that both are finite dimensional $\RR$-vector spaces. 
\end{defn}

\begin{rmk}
The theorem of the base holds for any proper scheme over a field $k$. If $X$ is not proper weird things can happen. For example, $X = (\A^2 \sm 0)_{k[\epsilon]} \to \Spec{k}$ has an exact sequence
\[ 0 \to \struct{X_{\red}} \to \struct{X}^\times \to \struct{X_{\red}}^\times \to 0 \]
where the first map is $x \mapsto 1 + \epsilon x$. Therefore, the sequence
\[ H^0(X, \struct{X}^\times) \to H^0(X, \struct{X_{\red}}^\times) \to H^1(X, \struct{X_{\red}}) \to \Pic{X} \to \Pic{X_{\red}} \to H^2(X, \struct{X_{\red}}) \]
is exact. Note that
\[ H^i(X_{\red}, \struct{X_{\red}}) = \bigoplus_{n \in \ZZ} H^i(\P^1, \struct{\P^1}(n)) \]
is infinite dimensional for $i = 0,1$ and zero for $i > 1$. This is because $\A^2 \sm \{ 0 \} \to \P^1$ is the total space of the $\Gm$-bundle associated to $\struct{\P^1}(-1)$ so the pushforward along $\pi : X \to \P^1$ gives
\[ \pi_* \struct{X} = \bigoplus_{n \in \ZZ} \struct{\P^1}(n) \] 
Furthermore, the map 
\[ H^0(X, \struct{X}^\times) \to H^0(X, \struct{X_{\red}}^\times) \]
is surjective because the left is just $k[x,y]$ any any such function lifts trivially to $k[\epsilon][x,y]$ which then restricts to give a function on $X$ via $X \embed \A^2_{k[\epsilon]}$. Likewise, since every line bundle extends along $X_{\red} \embed \A^2_k$ we see that $\Pic{X_{\red}} = 0$. Hence
\[ \Pic{X} = H^1(X_{\red}, \struct{X_{\red}}) = \bigoplus_{n \in \ZZ} H^1(\P^1, \struct{\P^1}(n)) \]
which is infinite dimensional in the strongest sense. 
\end{rmk}


Note that any variety $X$ is covered by subvarieties of each dimesnion $< \dim{X}$. This follows by taking the closures of complete intersections in an appropriate affine open. However, the next example shows that even if $X$ is proper we may have $N_1(X) = 0$. 


\subsection{A variety with no embedding in a smooth variety}

This is the argument given by David Rydh in \chref{https://mathoverflow.net/questions/201/is-there-an-example-of-a-variety-over-the-complex-numbers-with-no-embedding-into}{this} answer.

There exists a projective variety $X$ over any field (say $\CC$) such that
\begin{enumerate}
\item $\Pic{X} = 0$
\item $X$ does not embedd into any smooth variety.
\end{enumerate}

Indeed, (i) $\implies$ (2) because of the following theorem of Kleiman [H, III Ex. 6.8]
\begin{theorem}
Let $X$ be a noetherian integral, separated, locally factorial scheme. Then every coherent sheaf on $X$ is the quotient of a sum of line bundles (i.e.\ there is an ``ample family of line bundles'').
\end{theorem}

\begin{cor}
If $X$ is a proper variety of positive dimension and $\Pic{X} = 0$ then it does not embed in any locally factorial variety.
\end{cor}

\begin{proof}
Indeed, otherwise $X$ would admit an ample family of line bundles via pullback along the embedding. However, if $\Pic{X} = 0$ this is not possible. Indeed, consider the ideal sheaf $\I_x$ of a point $x \in X$. Since $\dim{X} > 0$ this is a nontrivial sheaf. There cannot be any section of $\I_x$ since $H^0(X, \struct{X})$ is a field (because it is a finite dimensional domain) the reduction map to $\kappa(x)$ is injective. Hence $\I_x$ is not the quotient of $\struct{X}^{\oplus n}$.
\end{proof}

\begin{theorem}
There exists a normal proper variety over $k$ with $\Pic{X} = 0$.
\end{theorem}

\begin{proof}
This is an example of Eikelberg \chref{https://www.semanticscholar.org/paper/The-Picard-Group-of-a-compact-toric-Variety-Eikelberg/fc442f2c6e1ac4bcccf8ead76bccf0496d171335}{Example 3.5 in The Picard Group of a compact toric Variety}. 
\end{proof}

\subsection{Invertible Sheaf Not Arising From Cartier Divisors}

\renewcommand{\Div}{\mathfrak{Div}}
\newcommand{\tot}{\mathrm{tot}}

Recall that there is a sheaf of Cartier divisors
\[ 0 \to \struct{X}^\times \to \K_X^\times \to \Div_X \to 0 \]
Recall that $\K_X$, the sheaf of meromorphic functions, is the sheafification of the presheaf
\[ U \mapsto \mathcal{S}(U)^{-1} \struct{X}(U) \]
where $\mathcal{S}(U) \subset \struct{X}(U)$ is the subset of \textit{regular} functions (those such that multiplication is injective, i.e. a nonzerodivisor in each $\stalk{X}{x}$). Since we only localize at nonzerodivisors the map $\struct{X} \to \K_X$ is injective. 
\par 
We must be careful to only localize at regular sections not all non zero divisors in $\struct{X}(U)$ which would give $(\stalk{X}(U))_{\tot}$ the total ring of fractions. Indeed, recall Kleiman's ``three misconceptions''
\begin{enumerate}
\item that $\K_X$ can be defined as the sheaf associated ot the presheaf of total fraction rings
\[ (\ast) \quad U \mapsto \Gamma(U, \struct{X})_{\tot} \]
\item that the stalks $\K_{X,x}$ equal the total fraction rings $\stalk{X}{x}$
\item if $X$ is a scheme and $U = \Spec{A}$ is an affine open then $\Gamma(U, \K_X)$ is $A_{\tot}$ (i.e.\ the presheaf $(\ast)$ is already a sheaf on affines).
\end{enumerate}
The problem arises from the fact that nonzerodivisors may be not regular and restrict to nonzerodivisors. Indeed, $(\ast)$ may not form a presheaf since we cannot restrict. Indeed, if $t \in \Gamma(X, \struct{X})$ is a nonzerodivisor whose restriction to $U$ becomes a zerodivisor in $\Gamma(U, \struct{X})$ then $1/t \in \Gamma(X, \struct{X})_{\tot}$ has no resction in $\Gamma(U, \struct{X})_{\tot}$. 

\begin{example}
Let $A$ be a domain with nonzero maximal ideal $\m$ and $P = \P^1_A$. Let $Y$ be the closed fiber over $\m$. Set
\[ X = \Spec{\struct{P} \oplus \struct{Y}(-1)} \]
where $\struct{Y}(-1)$ is viewed as a square-zero ideal. Then
\[ \Gamma(X, \struct{X}) = \Gamma(P, \struct{P}) \oplus \Gamma(Y, \struct{Y}(-1)) = A \]
hence any nonzero element $t \in \m$ is a nonzerodivisor. However, for any affine open $U \subset X$ containing a point $Y$, the restriction of $t \in \Gamma(U, \struct{X})$ is a zerodivisor since $t$ vanishes along $Y$ and $\struct{Y}(-1)|_U \cong \struct{Y}|_U$ so it has a nonzero section $s$ over $U$ and $ts = 0$.
\end{example}

Recall that we define the Cartier class group as
\[ \CaCl{X} = \coker{(H^0(\K_X^\times) \to H^0(\Div_X))} \]
and the long exact sequence gives a natural injection
\[ \CaCl{X} \embed \Pic{X} \]

\begin{theorem}
The map $\CaCl{X} \to \Pic{X}$ is an isomorphism when
\begin{enumerate}
\item $X$ is integral
\item $X$ is projective and $A = H^0(X,\struct{X})$ is noetherian.
\end{enumerate}
\end{theorem}

\begin{proof}
The first is easy since when $X$ is integral $\K_X$ is flasque (it is actually a constant sheaf) so $H^1(X, \K_X) = 0$ so the long exact sequence gives surjectivity. \chref{https://martapr.webs.uvigo.es/Investigacion/Divisors.pdf}{This paper} of Rodriguez proves the second assertion. Isn't this also proved in \chref{https://stacks.math.columbia.edu/tag/0AYM}{Tag 0AYM}? The basic idea should be to twist by the ample until globally generated then we can choose a section that is effective Cartier. 
\end{proof}

Section $3$ of \chref{https://martapr.webs.uvigo.es/Investigacion/Divisors.pdf}{the same paper} of Rodriguez exhibits an example of a non-projective proper scheme over a field $k$ on which the injection
\[ \CaCl{X} \embed \Pic{X} \]
is \textit{not} surjective. 

\subsection{Relationship to Effective Cartier Divisors}

\newcommand{\EffCart}{\mathrm{EffCart}}
\newcommand{\gp}{\mathrm{gp}}

An effective Cartier divisor $D \subset X$ is a subscheme such that $\I_D = \struct{X}(-D)$ is locally free. Recall that the data of an effective Cartier divisor is equivalent to a pair $(\L, s)$ of an invertible $\cO_X$-module and a regular section up to isomorphism of pairs. We set $\struct{X}(D) = \I_D^{\ot -1}$. This is a monoid under addition $D_1 + D_2$ which is the subscheme corresponding to the ideal $\I_{D_1} \I_{D_2}$ hence $\struct{X}(D_1 + D_2) = \struct{X}(D_1) \ot \struct{X}(D_2)$. Therefore, there is a map of monoids
\[ \EffCart(X) \to \Pic{X} \quad D \mapsto \struct{X}(D) \]
thus defining a group homomorphism
\[ \EffCart(X)^{\gp} \to \Pic{X} \]
In fact, this factors through
\[ \CaCl{X} \to \Pic{X} \]
and even through
\[ H^0(X, \Div_X) \to \CaCl{X} \to \Pic{X} \]
To see this, we must construct a map of monoids
\[ \EffCart(X) \to H^0(\Div_X) \]
refining the above map. Indeed, the section $1_D \in H^0(X, \struct{X}(D))$ defines an element of $H^0(X, \Div_X)$ by chosing an open cover $U_i$ trivializing $\struct{X}(D)$ and writing $1_D |_{U_i} = f_i s_i$ for $s_i : \struct{U} \iso \struct{X}(D)|_U$ the trivialization. Then $\{(U_i, f_i)\}_i$ is a well-defined section of $H^0(X, \Div_X)$ because the $f_i$ differ by units on the overlaps. 
\par
Hence there is a group homomorphism 
\[ \EffCart(X)^{\gp} \to H^0(X, \Div_X) \]

\textbf{Question}: when is this map an isomorphism?

\section{Endomorphisms of Varieties}

Let $X$ be a normal projective variety and $f : X \to X$ an dominant (hence surjective) endomorphism. 

\begin{prop}
If $\deg{f} = 1$, meaning $f$ is birational, then it is an isomorphism.
\end{prop}

\begin{proof}
Indeed, if $\deg{f} = 1$ this means $\struct{X} \to f_* \struct{X}$ is an isomorphism at the generic point so by normality it is an isomorphism. Hence $f$ has connected fibers. Since $f$ is proper, if we can show it is quasi-finite then it is finite and hence an isomorphism (since $\struct{X} \iso f_* \struct{X}$). Suppose $C \subset X$ is a curve contracted by $f$. If there is a nontrivial fiber it must be covered by curves. This means the pushforward map
\[ f_* : N_1(X) \to N_1(X) \]
is not injective since $C$ is a nonzero element because $X$ has an ample divisor $H$ and $C \cdot H > 0$. Since $N_1(X)$ is finite-dimensional this means that $f_*$ is not surjective. However, this is always surjective for any dominant map. Indeed, the fiber $f^{-1}(C')$ over some $C' \subset X$ is proper and surjective over $C'$ so there is a closed (hence proper) curve in $f^{-1}(C')$ surjecting onto $C'$. 
\end{proof}

\begin{prop}
Suppose $X$ is a smooth proper variety with $\kappa(X) \ge 0$ then if $f : X \to X$ is separable it is \etale.
\end{prop}

\begin{proof}
By separability, the pullback map $f^* \omega_X \to \omega_X$ is injective. Therefore
\[ \omega_X \cong f^* \omega_X \ot \cO_X(R) \]
where $R$ is the ramification divisor. This shows that pullback
\[ H^0(X, \omega_X) \embed H^0(X, g^* \omega_X) = H^0(X, \omega_X(-R)) \]
injects into the space of forms with zeros along $R$. Taking powers of the canonical bundle and iterating this process
\[ H^0(X, \omega_X^{\ot n}) \embed \bigcap_{i \ge 0} H^0(X, \omega_X^{\ot n}(-n (R + f^* R + \cdots + (f^*)^i R))) \]
Since $R$ is effective, any section in the intersection must vanish to arbitrarily large order (or along infinitely many divisors) and hence is zero. This holds for each $n \ge 0$ contradicting $\kappa(X) \ge 0$ unless $R = 0$ and hence $\phi$ is \etale. 
\end{proof}

\begin{rmk}
The Frobenius map shows that separability is essential. 
\end{rmk}


The nonproper case is also quite interesting. Recall the theorem of Ax
\begin{theorem}[Ax-Grothendieck]
Let $X$ be a finitely presented $S$-scheme and $f : X \to X$ an $S$-endomorphism. If $f$ is a monomorphism then it is an isomorphism.
\end{theorem}

\begin{prop}
Let $f : X \to Y$ be a morphism of varities with $Y$ normal. Suppose $f$ is birational and quasi-finite then $f$ is an open immersion. 
\end{prop}

\begin{proof}
By Zariski's main theorem there is a factorization $X \embed \wt{X} \to Y$ with $X \embed \wt{X}$ an open immersion and $\wt{X} \to Y$ finite. Hence $\wt{X} \to Y$ is birational and finite so an isomorphism by normality of $Y$. Hence $X \embed \wt{X} \cong Y$ is an open immersion.
\end{proof}

\begin{cor}
Let $f : X \to X$ be an endomorphism of a normal variety $X$. Suppose that $f$ is birational and quasi-finite then $f$ is an isomorphism.
\end{cor}

\begin{proof}
The proposition shows that $f$ is monic so by Ax's theorem it is an isomorphism.
\end{proof}

\chref{https://arxiv.org/pdf/2103.17130}{On endomorphisms of algebrac varities} by Das has some more results dropping the normality assumptions.

\section{Deformations of Vector Bundles}

\newcommand{\at}{\mathrm{at}}
\newcommand{\At}{\mathrm{At}}

The Atyiah class
\[ \at(\E) \in H^1(X, \shEnd{\struct{X}}{\E} \ot \Omega_X^1) \]
corresponds to the jet bundle extension
\[ 0 \to \E \ot \Omega_X^1 \to J^1(\E) \to \E \to 0 \]
but also 
\[ H^1(X, \shEnd{\struct{X}}{\E} \ot \Omega_X^1) = \Ext{1}{}{\T_X}{\shEnd{\struct{X}}{\E}} \]
giving the Atyiah extension
\[ 0 \to \shEnd{\struct{X}}{\E} \to \At(\E) \to \T_X \to 0 \]
Then the claim is that 
\[ T^i = H^i(X, \At(\E)) \]
gives a automorphism-deformation-obstruction theory for the pair $(X, \E)$. 

\subsection{Proof}

We form a stack $\X \to X_{\Zar}$ of deformations along an extension of Artin rings $R' \onto R$. We need to consider the automorphisms $\psi : (X', \E') \to (X', \E')$ meaning a pair $\psi : X' \to X'$ and $\psi_{\E} : \psi^* \E' \iso \E'$ compatible with the isomorphism $\varphi : (X', \E')_{R} \iso (X, \E)$. The map $\psi : X' \to X'$ is given by 
\begin{center}
\begin{tikzcd}
0 \arrow[r] & J \ot_R \struct{X} \arrow[d, equals] \arrow[r] & \struct{X'} \arrow[d, "\psi^{\#}"] \arrow[r] & \struct{X} \arrow[r] \arrow[d, equals] & 0 
\\
0 \arrow[r] & J \ot_R \struct{X} \arrow[r] & \struct{X'} \arrow[r] & \struct{X} \arrow[r] & 0 
\end{tikzcd}
\end{center}
and we see that
\[ \xi := \psi^{\#} - \id \in \Hom{X}{\Omega_{X/R}}{J \ot_R \struct{X}} \]
Now we consider the diagram for $\psi_{\E}$
\begin{center}
\begin{tikzcd}
0 \arrow[r] & J \ot_R \E \arrow[d, equals] \arrow[r] & \psi^* \E' \arrow[d, "\psi_{\E}"] \arrow[r] & \E \arrow[r] \arrow[d, equals] & 0 
\\
0 \arrow[r] & J \ot_R \E \arrow[r] & \E' \arrow[r] & \E \arrow[r] & 0 
\end{tikzcd}
\end{center}
so the set of such $\psi_{\E}$ is a torsor over $\shEnd{\struct{X}}{\E}$ we need to figure out which torsor. The sections of $\At(\E)$ are exactly for which the connecting map
\[ \delta : H^0(X, \T_X) \to H^1(X, \shEnd{\struct{X}}{\E}) \]
plus a section of $\shEnd{\struct{X}}{\E}$. We need to identify this torsor with the one given by the connecting map for the Atyiah sequence. Assume there is a trivialization $\ul{s} : \struct{X'}^{\oplus r} \iso \E'$ associated to such a trivialization is a canonical choice for $\psi_{\E}$ given by $\psi_{\E}^{s} := \ul{s} \circ (\psi^{\#})^{r} \circ \ul{s}^{-1}$. Therefore, we get an identification between the set of $\psi_{\E}$ and sections $\eta$ of $\shEnd{\struct{X}}{\E}$ via
\[ \eta \mapsto \psi_{\E}^s + \eta \]
depending $s$. Given two trivializations $s, s'$ with transtion matrix $t = \ul{s'}^{-1} \circ \ul{s}$ the difference between these two identifications is 
\[ \psi_{\E}^{s} - \psi_{\E}^{s'} = \psi_{\E}^{s} - \ul{s} \circ t^{-1} \circ (\psi^{\#})^r \circ t \circ \ul{s}^{-1}  \]
but $\psi^{\#} = \id + \xi \circ \d$ and therefore
\[ \psi_{\E}^{s} - \psi_{\E}^{s'} = \ul{s} \circ (t^{-1} \circ \xi(\d{t})) \circ \ul{s}^{-1} \]
which is exactly $\delta(\xi)$ as a local cocycle using the $\d{\log}$ description of the Atiyah class. 

\begin{lemma}
The stack $\X \to X_{\Zar}$ is an $\At(\E)$-gerbe. 
\end{lemma}

\begin{proof}
We have shown thay any two automorphisms differ by a section of $\At(\E)$ moreover any two objects are locally isomorphic and deformations exist locally. This implies it is a gerbe. 
\end{proof}

Note: everything works the same way if I consider the pair $f : (X, \E) \to S$ flat over a scheme $S$ and I fix a square-zero extension $S \embed S'$ by $\J$. Then there is a $(\At(\E) \ot f^* \J)$-gerbe $\X \to X_{\Zar}$ of lifts along $S \embed S'$.

\subsection{Calculations}

Therefore, the connecting map
\[ \delta : H^1(X, \T_X) \to H^2(X, \shEnd{\struct{X}}{\E}) \]
which is Yoneda pairing with $[\at(\E)]$ and contracting along $\T_X \ot \Omega_X \to \struct{X}$ gives the map from a deformation of $X$ to the obstruction of deforming $\E$ along that given deformation. 

\begin{example}
For $\E = \L$ a line bundle $\at(\L) \in H^1(X, \Omega^1)$ is the first Chern class. The map
\[ \delta : H^1(X, \T_X) \to H^2(X, \struct{X}) \]
is wedge with this Chern class. If $\omega_X \cong \struct{X}$ the ismorphism gives a commutative square 
\begin{center}
\begin{tikzcd}
H^1(X, \T_X) \arrow[d] \arrow[r, "\delta"] & H^2(X, \struct{X}) \arrow[d]
\\
H^1(X, \Omega_X^{n-1}) \arrow[r, "\smile c_1(\L)"] & H^2(X, \omega_X)
\end{tikzcd}
\end{center}
so we just need to understand the wedge operations on Hodge cohomology. For $X$ a surface the pairing
\[ H^1(X, \Omega_X^1) \times H^1(X, \Omega_X^1) \to H^2(X, \omega_X) \xrightarrow{\tr} k \]
is perfect by Poincare duality. Therefore, if $c_1(\L) \neq 0$ then $\delta$ is surjective. Now consider the long exact sequence
\begin{center}
\begin{tikzcd}[column sep = tiny]
0 \arrow[r] & H^0(X, \struct{X}) \arrow[r] &  H^0(X, \At(\L)) \arrow[r] & H^0(X, \T_X) \arrow[r] & H^1(X, \struct{X}) \arrow[r] & H^1(X, \At(\L)) \arrow[r] & H^1(X, \T_X) \connectingmap{llllld}
\\
 & H^2(X, \struct{X}) \arrow[r] &  H^2(X, \At(\L)) \arrow[r] & H^2(X, \T_X) \arrow[r] & H^3(X, \struct{X}) \arrow[r] &  H^3(X, \At(\L)) \arrow[r] & H^3(X, \T_X)
\end{tikzcd}
\end{center}
For a K3 surface we have $H^0(X, \T_X) = H^1(X, \struct{X}) = H^2(X, \T_X) = 0$ so this gives
\[ H^0(X, \At(\L)) = k \]
and an exact sequence
\[ 0 \to H^1(X, \At(\L)) \to H^1(X, \T_X) \xrightarrow{\delta} H^2(X, \cO_X) \to H^2(X, \At(\L)) \to 0 \]
Therefore, if $c_1(\L) \neq 0$ then $\delta$ is surjective so $H^2(X, \At(\L)) = 0$ so the deformation space of $(X, \L)$ is smooth with tangent space
\[ H^1(X, \At(\L)) = \ker{\delta} \]
which has dimension $19$. Otherwise, the deformation space has virtual dimension $19$ but may be singular. I would guess this can happen in positive characteristic. 
\end{example}

\subsection{Subschemes}

Here we run the same strategy for the stack $\X \to X_{\Zar}$ of deformations along $R' \onto R$ of a pair $\iota : Y \embed X$ consisting of a closed subscheme. The claim is that this stack is a gerbe over
\[ \G = \ker{(\T_X \to \iota_* \cN_{Y|X})} = \T_X(-\log{Y}) \]
The reason for the notation is when $Y$ is an SNC divisor it agrees with the usual notion of tangent fields with log zeros. Notice that the inclusions of sheaves,
\[ \T_X \ot \I_Y \embed \T_X(-\log{Y}) \embed \T_X \]
controll deformations that are trivial over $Y$, deformations of $Y \embed X$, and deformations of $X$ respectively.  

\begin{proof}
We need to show that autormorphisms of $\psi : (X', Y') \to (X', Y')$ compatible with the identification $\varphi : (X', Y')_{R} \iso (X, Y)$ correspond to sections of $\T_X(-\log{Y})$. Note that we consider two closed subschemes the same if they are isomorphic as objects equipped with a map to $X$. This is the same as considering pairs $(X, \I_Y)$ for the ideal sheaves. Thus $\psi$ consists of a diagram,
\begin{center}
\begin{tikzcd}
0 \arrow[r] & \I_{Y'} \arrow[d] \arrow[r] & \struct{X'} \arrow[d, "\psi^{\#}"] \arrow[r] & \struct{Y'} \arrow[d, "\psi_Y"] \arrow[r] & 0
\\
0 \arrow[r] & \I_{Y'} \arrow[r] & \struct{X'} \arrow[r] & \struct{Y'} \arrow[r] & 0
\end{tikzcd}
\end{center} 
which reduces to the identity over $R$. Note if we force $\psi_Y = \id$ (meaning we are considering deformations \textit{trivial along} $Y$) then
\[ \psi^{\#} - \id \in \Hom{}{\Omega_{X/R}}{J \ot_R \I_{Y}} = H^0(X, \T_{X/R} \ot \I_{Y} \ot_R J)  \]
hence the deformation theory is controlled by $\T_X \ot \I_{Y}$. If we loosen the requirement that $\psi_Y = \id$ to just ask that $\psi$ forms a morphism of complexes. This is the same as sayng that $\psi^{\#}$ preserves $\I_{Y'}$ which is the same as saying that the composition $\xi = \psi^{\#} - \id$ along
\[ \I_{Y} \xrightarrow{\d} \Omega_{X/R} \xrightarrow{\xi} \struct{X} \ot_R J \to \struct{X} \ot_R J \]
is zero since this map measures how much $\psi^{\#}$ maps $\I_{Y}$ outside $\I_{Y}$. But this is the image of $\xi$ under
\[ \Hom{}{\Omega_{X/R}}{\struct{X} \ot_R J} \to \Hom{}{\I_Y}{\struct{Y} \ot_R J} = \Hom{}{\I_Y / \I_Y^2}{\struct{Y} \ot_R J} \]
so the automorphisms correspond to the sections in the kernel of 
\[ \T_X \ot_R J \to \cN_{Y|X} \ot_R J \]
\end{proof}

\subsection{Morphisms}

Now we want to consider deformations of morphism $f : X \to Y$ of $S$-schemes. Again we can consider the stack $\X \to (X \to Y)_{\Zar}$ of deformations of $f|_U : U \to V$ over a pair of opens $U \subset X$ and $V \subset Y$ so that $f : U \to V$ along a fixed square-zero extension $S \embed S'$ by $\J$. Consider the automorphism $\psi : (f', X', Y') \to (f', X', Y')$ of a fixed deformation. These are diagrams,
\begin{center}
\begin{tikzcd}
f^{-1} \struct{Y'} \arrow[d, "f^{-1} \psi_Y^{\#}"] \arrow[r, "f'^{\#}"] & \struct{X'} \arrow[d, "\psi_X^{\#}"] 
\\
f^{-1} \struct{Y'} \arrow[r, "f'^{\#}"] & \struct{X'} 
\end{tikzcd}
\end{center}
and we have
\[ \xi_X := \psi^{\#}_X - \id \in \Hom{}{\Omega_{X/S}}{s_X^* \J} \quad \text{ and } \quad \xi_Y := \psi^{\#}_X - \id \in \Hom{}{\Omega_{Y/S}}{s_Y^* \J} \]
where $s_X : X \to S$ and $s_Y : Y \to S$ are the structure maps. The commutativity of the square is equivalent to the condition that
\[ \xi_X \circ (\d_X \circ f^{\#}) = f^{\#} \circ \xi_Y \circ \d_Y \]
as maps of sheaves since the $\xi$ kill terms proportional to $\J$ and $f'^{\#}$ mod $\J$ is $f^{\#}$. This condition is saying that the tangent field $\xi_Y$ is a pushforward of the tangent field $\xi_X$ along $f$. Because
\[ f^{\#} \circ \xi_Y \circ \d_Y = \xi_X \circ (\d_X \circ f^{\#}) = \xi_X \circ f^* \d_Y \]
so $f^{\#} = \xi_X \circ f^*$ by universality of the differential. 
Notice that this stack is not a gerbe because two maps are not locally isomorphic. 

\subsection{Interlude}

\begin{prop}
Let $A'$ be a square-zero extension of $A$ by $I$. Suppose that there is a pushout square
\begin{center}
\begin{tikzcd}
A' \arrow[r] \arrow[d] & A \arrow[d]
\\
B' \arrow[r] & B
\end{tikzcd}
\end{center}
meaning $B = B' \ot_{A'} A$ with $B' \to B$ a square-extension via $I \ot_A B$. If $A \to B$ is flat then $A' \to B'$ is flat. Hence if $A \to B$ is flat then $A' \to B'$ is a deformation of $A \to B$ along $A' \onto A$.
\end{prop}

\begin{proof}
This is an application of the local criterion for flatness: since $I$ is nilpotennt, if $M$ is an $A'$-module then $M$ is $A$-flat iff $M/I$ is $A = A'/I$-flat and $\Tor{A'}{1}{A}{M} = 0$. Since $B' \ot_{A'} A = B$ is $A$-flat we just need to show that $\Tor{A'}{1}{A}{B'} = 0$. Consider the sequence
\[ 0 \to I \to A' \to A \to 0 \]
then the long exact sequence associated to $- \ot_{A'} B'$ gives
\[ \Tor{A'}{1}{A'}{B'} \to \Tor{A'}{1}{A}{B'} \to I \ot_{A'} B' \to B' \to A \ot_{A'} B' \to 0 \]
but $A \ot_{A'} B' = B$ and $I \ot_{A'} B' = I \ot_{A} B$ since $I^2 = 0$ and since $B' \to B$ is a $I \ot_A B$-extension the sequence
\[ 0 \to I \ot_A B \to B' \to B \to 0 \]
is exact. Hence the connecting map $\Tor{A'}{1}{A}{B'} \to I \ot_{A'} B'$ is zero. Since $A'$ is free over $A'$ the first term also vanishes hence $\Tor{A'}{1}{A}{B'} = 0$ proving that $A' \to B'$ is flat.
\end{proof}

\section{Derived Setup}

\newcommand{\Alg}{\mathrm{Alg}}
\newcommand{\SAlg}{\mathrm{SAlg}}
\newcommand{\Map}{\mathrm{Map}}
\newcommand{\cQ}{\mathcal{Q}}

Note [Higher Algebra 1.1.2.17] Let $\C$ be a stable $\infty$-category containing a pair of objects $X$ and $Y$. We denote by $\Ext{n}{\C}{X}{Y}$ the abelian group $\Hom{h\C}{X[-n]}{Y} = \Hom{h\C}{X}{Y[n]}$. If $n$ is negative, this can be idenfitied with the homotopy group $\pi_{-n} \Map_{\C}(X,Y)$. More generally it is $\pi_{-n}$ of the mapping spectrum.
\bigskip\\
Let $k$ be a fixed (ordinary) base ring. Let $\SAlg_k$ be the $\infty$-category of derived simplicial commutative $k$-algebras and let $\Alg_k$ be the full subcategory of discrete objects. Given $A \in \SAlg_k$ we define the stable $\infty$-category $\Mod{A}$ of $A$-modules which is symmetric monoidal. This realizes the derived $\infty$-category for $A$ discrete. Note that for $M \in \Mod{A}$ and $N \in \Mod{B}$ with a map $A \to B$ there is a natural equivalent
\[ \Hom{\Mod{A}}{M}{N} \cong \Hom{\Mod{B}}{M \ot_A B}{N} \]
Tensor product, like all functors is here derived.
\bigskip\\
A $k$-linear derivation $A \to M$ for $M \in \Mod{A}$ will be, by definition a section of the split square-zero extension
\[ A \oplus M \to A \]
Let $\Der{k}{A}{M}$ denote the $\infty$-groupoid of such sections (given as the fiber of mapping spaces)

\begin{theorem}
The universal derivation $\d : A \to \LL_{A}$ induces a functorial equivalence
\[ \Hom{A}{\LL_A}{M} \cong \Der{k}{A}{M} \]
\end{theorem}

Note there is an exact (fiber/cofiber) sequence
\[ \LL_A \ot_A B \to \LL_B \to \LL_{B/A} \]
whose rotation
\[ \LL_{B/A}[-1] \to \LL_{A} \ot_A B \]
is the Kodara-Spencer class
\[ \kappa(A \to B) \in \Ext{1}{B}{\LL_{B/A}}{\LL_{A} \ot_A B} \]

\subsection{Lifting Problems}

We can globalize everything to the sheaf topos of a DM-stack (DM so that $\LL$ is still connective). Let $f : X \to Y$ be a morphism of DM-stacks and $M$ be a $\struct{Y}$-module. Given a derivation $D : \struct{Y} \to M$ we consider if the diagram
\begin{center}
\begin{tikzcd}
f^{-1} \struct{Y} \arrow[d] \arrow[r] & f^{-1} M \arrow[d]
\\
\struct{X} \arrow[r, dashed] & f^* M
\end{tikzcd}
\end{center}
can be filled in with a derivation.
Here $f^*$ is derived pullback and $f^{-1}$ is exact so it makes sense already at the $\infty$-category level. This is equivalent to filling in the diagram
\begin{center}
\begin{tikzcd}
f^* \LL_Y \arrow[r, "f^* D"] \arrow[d] & f^* M \arrow[d, equals]
\\
\LL_B \arrow[r, dashed] & f^* M
\end{tikzcd}
\end{center}
hence the space of lifts is the homotopy fiber of
\[ \Hom{X}{\LL_X}{f^* M} \to \Hom{X}{f^* \LL_Y}{f^* M} \]
over the element $D$. Using the rotated transitivity tiangle we see there is a fiber/cofiber sequence
\[ \Hom{X}{\LL_X}{f^* M} \to \Hom{X}{f^* \LL_Y}{f^* M} \to \Hom{X}{\LL_{X/Y}[-1]}{f^* M} \]
here these are the mapping spectra. Therefore, the image gives an obstruction
\[ \ob(f, D) \in \Ext{1}{X}{\LL_{X/Y}}{f^*M} \]
when it vanishes the the fiber is a torsor over the fiber over zero:
\[ \Hom{X}{\LL_{X/Y}[-1]}{f^* M}[-1] = \Hom{X}{\LL_{X/Y}}{f^* M} \]
meaning its path components (points up to homotopy) form a torsor over
\[ \Ext{0}{X}{\LL_{X/Y}}{f^* M} \]
and fundamental group at $0$ (automorphisms) is 
\[ \pi_1 \Hom{X}{\LL_{X/Y}}{f^* M} = \pi_0 \Omega \Hom{X}{\LL_{X/Y}}{f^* M} = \Ext{-1}{X}{\LL_{X/Y}}{f^* M} := \Ext{0}{X}{\LL_{X/Y}}{f^* M[-1]} \]
Notice that if we replace $M$ by $M[1]$ then a derivation $A \to M[1]$ corresponds to a square-zero extension of $A$ by $M$. Hence the above applies verbatim to tell us about lifting square-zero extensions. The flatness lemma shows that if $f$ is flat then any such lift will also be flat over the thickening.

\begin{example}
Consider the problem of deforming maps $f : X \to Y$ flat over $S$ with the target ``fixed''. What this means really is just that we are given extensions $S \embed S'$ and $Y \embed Y'$ by $M$ and we need to lift along $f : X \to Y$ which will automatically give an extension along $X \to S$. Therefore, if $D_Y : \struct{Y} \to f^* M[1]$ defines the extension we just need to take the fiber of
\[ \Hom{X}{\LL_X}{f^* g^* M[1]} \to \Hom{Y}{\LL_Y}{g^* M[1]} \]
over $D_Y$. Therefore, the same story applies and the problem is exactly controlled by $\Hom{X}{\LL_{X/Y}}{f^* g^* M}$.
If we moreover take the target to be $Y = Y_0 \times S$ and $D_Y$ defined via
\[ \LL_{Y} \to \pi_2^* \LL_S \xrightarrow{\pi_2^* D_S} \pi_2^* M[1] \]
then 
\[ \Hom{X}{\LL_X}{f^* g^* M[1]} \to \Hom{Y}{\LL_Y}{g^* M[1]} = \Hom{Y}{\pi_1^* \LL_{Y_0}}{\pi_2^* M[1]} \oplus \Hom{Y}{\pi_2^* \LL_{S}}{\pi_2^* M[1]} \]
and we are taking the fiber over $(0, \pi_2^* D_S)$.
\end{example}

\begin{example}
Let $\C \to S$ be a stable curve and $f : \C \to X \times S$ a stable map. Since $\C \to S$ is lci the cotangent complex is supported in $[-1,0]$ and it is represented by the Naive cotangent complex. Assume that $X$ is smooth then $\LL_f$ is defined by the cone over the following map of complexes
\begin{center}
\begin{tikzcd}
0 \arrow[r] & 0
\\
f^* \Omega_X \arrow[u] \arrow[r] & L^0_{\C/S} \arrow[u]
\\
0 \arrow[r] \arrow[u] & L^{-1}_{\C/S} \arrow[u]
\\
0 \arrow[u] \arrow[r] & 0 \arrow[u]
\end{tikzcd}
\end{center}
note that $\LL f^* \Omega_X = f^* \Omega_X$ since $\Omega_X$ is locally free. There is a long exact sequence taking homology ($\pi_{-n}$)
\[ 0 \to \cH^{-1}(\LL_{\C/S}) \to \cH^{-1}(\LL_f) \to f^* \Omega_X \to \Omega_{\C/S} \to \cH^0(\LL_f) \to 0 \]
Note that $\tau^{\ge 0} \LL_{\C/S} = \Omega_{\C/S}$. Therefore, we get a diagram of exact sequences
\begin{center}
\begin{tikzcd}
f^* \LL_X \arrow[d, equals] \arrow[r] & \LL_{\C/S} \arrow[r] \arrow[d] & \LL_f \arrow[d]
\\
f^* \LL_X  \arrow[r] & \tau^{\ge 0} \LL_{\C/S} \arrow[r] & C
\end{tikzcd}
\end{center}
here we really use that we are in a stable $\infty$-category so that cones are functorial! The complex $C$ is represented by
\[ [f^* \Omega_X \to \Omega_{\C/S}] \]
The claim is that if $\E$ is locally free sheaf then the map
\[ \Ext{i}{\C}{C}{\E} \to \Ext{i}{\C}{\LL_f}{\E} \]
is an isomorphism for $i \le 1$ and injective for $i = 2$. Note that the fiber 
\[ \cQ \to \LL_f \to C \]
satisfies $\tau^{\ge 0}(\cQ) = 0$ and $\tau^{< 0} \cQ = \cH^{-1}(\LL_{\C/S})[1]$ is supported in one degree. Therefore, the long exact sequence
\[ \Ext{i-1}{\C}{\cQ}{\E} \to \Ext{i}{\C}{C}{\E} \to \Ext{i}{\C}{\LL_f}{\E} \to \Ext{i}{X}{\cQ}{\E} \]
and $\Ext{i-1}{\C}{\cQ}{\E} = \Ext{i-2}{\C}{\cH^{-1}(\LL_{\C/S})}{\E}$ is zero for $i \le 2$ because $\cH^{-1}(\LL_{\C/S})$ is torsion and $\E$ is torsion-free. This proves the claim. Therefore, $\Ext{i}{X}{C}{\struct{X}}$ forms a automorphism-tangent-obstruction theory for stable maps.
\end{example}

\subsection{Defoming Maps}

Given a morphism $f : X \to Y$ of DM-stacks flat over $S$ we want to discuss deformations of $f : X \to Y$ over a thickening $S \embed S'$ given by $\I$. Let $D_S : \struct{S} \to \I[1]$ be the derivation corresponding to the thickening. The point is that all we need to do is lift this thickening along $g : Y \to S$ and then subsequently along $f : X \to Y$ which given a lift along $g \circ f$ which will hence be flat. Therefore, it is controlled by the object $F'$ defined in the diagram
\begin{center}
\begin{tikzcd}
F \arrow[d] \arrow[rr] \pullback & & * \arrow[d, "g^* D_S"] 
\\
G \pullback \arrow[r] \arrow[d] & \Hom{Y}{\LL_Y}{g^* \I[1]} \arrow[r] \arrow[d] & \Hom{Y}{g^* \LL_S}{g^* \I[1]}
\\
\Hom{X}{\LL_X}{f^* g^* \I[1]} \arrow[r] & \Hom{X}{f^* \LL_Y}{f^* g^* \I[1]}
\end{tikzcd}
\end{center}
then $F$ is the fiber of the following map over $(0, g^* D_S)$
\[ \Hom{X}{\LL_X}{f^* g^* \I[1]} \oplus \Hom{Y}{\LL_Y}{g^* \I[1]} \to \Hom{X}{f^* \LL_Y}{f^* g^* \I[1]} \oplus \Hom{Y}{g^* \LL_S}{g^* \I[1]} \]
where it is the difference of the natural maps to the first factor and the second is the section projection composed with the functoriality $g^* \LL_S \to \LL_Y$ map.


\section{Notes for Meeting}

\begin{enumerate}
\item argument using nonvanishing theorem

\item Can theorem 5.1 be true for $n = 2$, it seems unlikely

\item in corollary 4.2 why do we need $d$ to be an integer, you get more juice if it is slightly less than an integer. Is it so that the $\Q$-line bundle is actually a line bundle in the sense of
\[ M^{\ot \ell} \cong \struct{X}(\ell \alpha D) \]
for some $\ell$.

\item Did you expect thm 4.1 to be false or are you just sayng that you assume $Adj(D) = I_0^{W_1}$

\item where does the argument use this equality? Is it Theorem 3.5?

\item can you explain Rmk. 4.3?

\item 
\end{enumerate}

\section{Higher Kodaira-Spencer maps}

Let $f : X \to S$ be a smooth proper morphism with $f_* \struct{X} = \struct{S}$. Denote $\wedge^m \T_X$ by $\T_X^m$. There is a filtration
\[ T_X^p = \F^0 \supset \F^1 \supset \cdots \supset \F^p \supset \F^{p+1} = 0 \]
such that
\[ \F^p / \F^{p+1} \cong \T_{X/S}^i \ot f^* \T_B^{p - i} \]
this is a general fact about exact sequences
\[ 0 \to \T_{X/S} \to \T_X \to f^* \T_S \to 0 \]
Therefore, there is an exact sequence
\[ 0 \to \T_{X/S}^p \to \F^{p-1} \to \T_{X/S}^{p-1} \ot f^* \T_S \to 0 \]
which has connecting maps
\[ \psi_{p,q} : R^{q-1} f_* \T_{X/S}^{p-1} \ot \T_S \to R^{q} \T_{X/S}^p \]
and hence we can define
\[ \rho_f^{(p)} : R^{p-1} f_* \T_{X/S}^{p-1} \ot \T_S^{\ot (n - p + 1)} \to R^p f_* \T^p_{X/S} \ot \T_S^{\ot (n - p)} \]
obtained by tensoring $\psi_{p,p}$ by $\T_{S}^{\ot (n - p + 1)}$. The point of this tensoring is so that they become composable:
\[ \kappa(f)^{(p)} := \rho_f^{(p)} \circ \rho_f^{(p-1)} \circ \cdots \circ \rho_f^{(1)}  : \T_{S}^{\ot n} \to R^p f_* \T_{X/S}^p \ot \T_S^{\ot (n - p)} \]

\subsection{Linear Katz-Oda maps}

The Higgs bundle structure of the canonical VHS arises in a similar way. The exact sequence
\[ 0 \to f^* \Omega_S \to \Omega_{X} \to \Omega_{X/S} \to 0 \]
induces a filtration
\[ \Omega_X^p = \F^0 \supset \F^1 \supset \cdots \supset \F^{p} \supset \F^{p+1} = 0 \]
such that
\[ \F^i / \F^{i+1} = f^* \Omega_S^i \ot \Omega_{X/S}^{p-i} \]
therefore there is an exact sequence
\[ 0 \to f^* \Omega_S^1 \to \Omega_X^{p-1} / \F^2 \to \Omega_{X/S}^{p} \to 0 \]
which induces connecting maps
\[ \theta_{p,q} : R^q f_* \Omega_{X/S}^p \to R^{q+1} f_* \Omega_{X/S}^{p-1} \ot \Omega_S^1 \] 
what is the relation between these objects? More generally, we can define an $\L$-twisted variant using the exact sequence
\[ 0 \to f^* \Omega_S^1 \ot \L \to \Omega_X^{p-1} / \F^2 \ot \L \to \Omega_{X/S}^{p} \ot \L \to 0 \]
so the associated connecting maps are
\[ \theta_{p,q}^{\L} : R^q f_* (\Omega_{X/S}^p \ot \L) \to R^{q+1} f_* (\Omega_{X/S}^{p-1} \ot \L) \ot \Omega_S^1 \]
the point is for $\L = \omega_{X/S}$ we get that $\Omega_{X/S}^p \ot \L$ is the relative Serre dual of $\T_{X/S}^p$. Thus
\[ \theta^{\omega_{X/S}}_{p,q} = \psi_{p,n-q}^{\vee} \]
{\color{red} PROVE IT}




\section{TODO}

\begin{enumerate}
\item \chref{https://people.math.harvard.edu/~mpopa/571/}{Modern aspects of the topology of varieties}

\item 
\end{enumerate}

\section{Hochschild homology}

\newcommand{\coev}{\mathrm{coev}}
\newcommand{\xar}[1]{\xrightarrow{#1}}
\newcommand{\perf}{\mathrm{perf}}

\newcommand{\Fun}{\mathrm{Fun}}
\newcommand{\qc}{\mathrm{qc}}
\newcommand{\cHH}{\mathcal{HH}}

Let $D_{\perf}(S)$ denote the category of perfect complexes (a strong version of the bounded derived category) and $D_{\qc}(S)$ the unbounded derived category of quasi-coherent sheaves. An $S$-linear category $\C$ is a smal idempotent-complete stable $\infty$-category equipped with a module structure over $D_{\perf}(S)$. 

\begin{defn}
Given a symmetric monoidal $\infty$-category $(\cA, \ot, 1)$ and object $A \in \cA$ is called \textit{dualizable} if there exists an object $A^{\vee} \in \cA$ and morphisms
\[ \coev_{A} : 1 \to A \ot A^{\vee} \quad \quad \ev_{A} : A^{\vee} \ot A \to 1 \]
such that the compositions
\[ A \xar{\coev_A \ot \id_A} A \ot A^{\vee} \ot A \xar{\id_A \ot \ev_A} A \]
\[ A^{\vee} \xrightarrow{\id_{A^{\vee}} \ot \coev_A} A^{\vee} \ot A \ot A^{\vee} \xrightarrow{\ev_A \ot \id_{A^{\vee}}} A^{\vee} \]
are equivalent to the identity morphisms of $A$ and $A^{\vee}$.
\end{defn}

\begin{rmk}
If $A,B$ are dualizable objects then so it $A^{\vee}$ (with dual $A$) and $A \ot B$ (with dual $A^{\vee} \ot B^{\vee}$). 
\end{rmk}

Given a symmetric monoidal $\infty$-category $(\cA, \ot, 1)$ and a dualizable object $A \in \cA$, the \textit{trace} of an endomorphism $F : A \to A$ is the map $\tr{} \in \Hom{A}{1}{1}$ given by
\[ 1 \xar{\coev_A} A \ot A^{\vee} \xar{F \ot \id_{A^{\vee}}} A \ot A^{\vee} \cong A^{\vee} \ot A \xar{\ev_A} 1 \]

\begin{defn}
Let $\C$ be a dualizable presentable $S$-linear category and $\Phi \in \Fun_{S}(\C, \C)$ is an endomorphism, the \textit{Hochschild homology} of $\C$ over $S$ with coefficients in $\Phi$ is the complex
\[ \HH_*(\C/S, \Phi) = \Tr{\Phi}(\struct{S}) \in D_{\qc}(S) \]
\end{defn}

\begin{defn}
Let $\C$ be a small or presentable $S$-linear category, and let $F : \C \to \C$ be an endomorphism. Then the \textit{Hochscild cohomology of $\C$ over $S$ with coefficients in $F$} is the complex
\[ \cHH^\bullet(\C/S,F) := \shHom{S}{\id_{\C}}{F} \in D_{\qc}(S) \]
computed in the $S$-linear category $\Fun_{S}(\C, \C)$. We write
\[ \cHH^\bullet(\C/S) := \cHH^\bullet(\C/S, \id_{\C}) \]
Furthermore, we define
\[ HH^\bullet(\C/S, F) := \R\Gamma(\cHH^\bullet(\C/S, F)) \]
and likewise the cohomology groups
\[ HH^i(\C/S, F) := H^i(\cHH^\bullet(\C/S, F)) \]
\end{defn}

\begin{theorem}
Let $f : X \to S$ be a smooth morphism of relative dimension $n$, where $n!$ is invertible on $S$. Let $F \in D_{\perf}(S)$. Then there is an equivalent
\[ HH_\bullet(X/S, -\ot f^* F) \cong \bigoplus_{p=0}^n F \ot R f_* \Omega_{X/S}^p [p] \]
and
\[ HH^\bullet(X/S, -\ot f^* F) \cong \bigoplus_{i = 0}^n F \ot R f_* \left( \bigwedge^i \T_{X/S} \right) [-i] \]
\end{theorem}

\subsection{Topological $K$-theory}

\renewcommand{\top}{\mathrm{top}}
\newcommand{\Cat}{\mathrm{Cat}}

Blanc constructed a lax symmetric monoidal topological $K$-theory functor
\[ K^{\top} : \Cat_{\CC} \to \Sp \]
from $\CC$-linear categories to the $\infty$-category of spectra. It is such that
\[ K^{\top}(D_{\perf}(X)) \cong K^{\top}(X^{\an}) \]

\section{Sasha - Chern classes of v.b. on the Fargues-Fontaine curve}

Let $X$ be a smooth projective variety over a number field $F / \Q$. Let $E$ be a vector bundle of rank $n$ on $X$. Then the Chern classes
\[ c_i^{CH}(E) \in \CH^i(X) \]
live in Chow and using cycle class maps we get Chern classes in cohomology theories. Since $X$ is defined over a number field $F$ we can get a class
\[ c_i(E) \in H^{2i}_{\et}(X, \Z_p(i)) \]
in absolute \etale cohomology. If $E$ caries a flat connection $\nabla$ then for any $\sigma : F \embed \CC$ then
\[ c_i(E) \in H^{2i}(X(\CC), \Z) \]
is torsion by Chern-Weil theory. Moreover, if we factor through Deligne cohomology
\[ H_D^{2i}(X, \Z(i)) \to H^{2i}(X(\CC), \Z) \]
which depends on the complex structure not just on the topology. 

\begin{theorem}
If $E$ has a flat connection then $c_i^D(E) \in H^{2i}_D(X, \Z(i))$ it torsion for $i > 1$ (for $i = 1$ it remembers line bundles exactly so it need not be torsion).
\end{theorem}

\begin{conj}[Beilinson, Esnault-Harris]
For any $E / X$ with a flat connection $c_i^{CH}(E) \in \CH^i(X)$ is torsion for $i > 1$. 
\end{conj}

This would imply that $c_i(E) \in H^{2i}_{\et}(X, \Z_p(i))$ is also torsion. Equivalently $c_i(E) \in H^{2i}_{\et}(X, \Q_p(i))$ is zero.


\subsection{Characteristic classes of \etale local systems}

\renewcommand{\cont}{\mathrm{cont}}
\newcommand{\Isom}{\mathrm{Isom}}

Let $X$ be any scheme and $\LL$ an \etale $\Z_p$-local system on $X$ i.e. a continuous representation $\rho_{\LL} : \pi_1^{\et}(X) \to \GL_n(\ZZ_p)$. There are characteristic classes
\[ H^*_{\cont}(\GL_n(\ZZ_p), \Q_p) \xrightarrow{\rho_{\LL}^*} H^*_{\cont}(\pi_1^{\et}(X), \Q_p) \to H^*_{\et}(X, \Q_p) \]
moreover we can compute
\[ H^*_{\cont}(\GL_n(\ZZ_p), \Q_p) \cong \bigwedge^*_{\Q_p}(\ell_1, \dots, \ell_n) \]
with $\deg{\ell_i} = 2i - 1$ and $\ell_i \mapsto \ell_i(\LL)$ are the characteristic classes.

Consider
\[ \wt{X} = \Isom_X(\LL, \ZZ_p^{\oplus n}) \to X \]
which is an inverse limit of schemes and forms a $\GL_n(\ZZ_p)$-torsor trivializing $\LL$. Furthermore
\[ R \Gamma_{\et}(X, \Z_p) = R \Gamma_{\cont}(\GL_n(\ZZ_p), R \Gamma_{\et}(\wt{X}, \Z_p)) \]
There is a $\GL_n(\Z_p)$-equivariant inclusion of the constant functions into $H^0(\wt{X}, \Z_p)$
\[ \epsilon : \Z_p \to \R \Gamma_{\et}(\wt{X}, \Z_p) \]
If $\epsilon$ had a splitting in the derived category of representations of $\GL_n(\Z_p)$ then the map
\[ R \Gamma_{\cont}(\GL_n(\Z_p), \Z_p) \to R \Gamma_{\cont}(\GL_n(\Z_p), \Z_p) \]
would be injective on cohomology. But usually this is false. 

\begin{theorem}[Petrov-Pan]
Let $X / K$ be a smooth proper variety over a $p$-adic field (say a finite extension $K / \QQ_p$) and a de Rham local system $\LL$ so there is an associated vector bundle $D_{\dR}(\LL)$ with flat connection $\nabla$ and a filtration (e.g. consider $\LL = R^n f_* \Z_p$ for some smooth proper $f : Y \to X$) then
\[ \ell_{d+1}(\LL) \in H^{2d+1}_{\et}(X, \Q_p) = H^1(G_k, H^{2d}_{\et}(X_{\ol{K}}, \Q_p)) \cong K \]
equals
\[ d! \sum m \cdot \mathrm{ch}_d(\gr^m D_{\dR}(\LL)) \in \Z \subset K \]
assuming on the following holds
\begin{enumerate}
\item $K = \Q_p$
\item $d = 1$ and $X$ has good reduction at $p$ and the residue field of $K$ is $\FF_p$
\item Newton slopes of $\LL$ are constant for all points $x \in X$
\end{enumerate}
\end{theorem}

\begin{rmk}
Notice that $H^{2d}_{\et}(X_{\ol{K}}, \Q_p) = \Q_p(-d)$ and $p$-adic Hodge theory tells us $H^1(G_K, \Q_p(-d)) = K$. 
\end{rmk}

\subsection{Fargues-Fontaine curve}

Let $R$ be a perfectoid $\Q_p$-algebra so it contains $R^+ \subset R$ which is $p$-adically complete and integrally closed in $R$ and $\Frob : R^+ / p \to R^+ / p$ is surjective. Let $R^{\flat}$ be the tilt, a perfect $\FF_p$-algebra, so that 
\[ (R^{\flat})^+ = \ilim_{x \mapsto x^p} R^+ / p \]
These perfectoid $R$ satisfy the ``almost purity'' property:
\begin{enumerate}
\item $\{ \text{finite \etale } R\text{-algebras} \} \cong \{ \text{finite \etale } R^{\flat}\text{-algebras}\}$ which is given by $(R \to S) \mapsto (R^{\flat} \to S^{\flat})$
\item $R \to R_i$ finite \etale for each $i \in I$ let $R_{\infty} = (\colim_{i \in I} R_i)_p^{\wedge}$ then $R \to R_{\infty}$ satisfies descent for projective modules
\end{enumerate}

\begin{rmk}
The second is very surprising: indeed if $R$ is not perfectoid, say $R = \Q_p$ we could take $R_{\infty} = \CC_p$ and it is not true that $\Q_p$-modules are the same as $\CC_p$-modules with a Galois action. However, $\Q_p(\mu_{\infty})_p^{\wedge}$ is perfectoid. If $V$ is a f.d.\ $\CC_p$-vectorspace with an action of the Galois group for $\CC_p / \Q_p(\mu_{\infty})_p^{\wedge}$, this descends to a $\Q_p(\mu_{\infty})_p^{\wedge}$-vectorspace but it does not work down to $\Q_p$. 
\end{rmk}

This descent shows that for $R$ perfectoid we get that $\Z_p$-local systems on $R^{\flat}$ are $\Z_p$-local systems on $R$ which produce vector bundles on $R$ using descent for an extension $R \to R_{\infty}$ trivializing the local system. 

\begin{example}
Tilting is not injective. Indeed, $\Q_p(\mu_{\infty})_p^{\wedge}$ and $\Q_p(1/p^{\infty})_p^{\wedge}$ both have tilt $\FF_p((t^{1/p^{\infty}}))$. However, they have the same category of $\Z_p$-local systems using the tilting equivalence.
\end{example}

\newcommand{\Spa}[1]{\mathrm{Spa}\left(#1\right)}

For all $R'$ perfectoid $\Q_p$-algebras with $(R')^{\flat} = R^{\flat}$ then we get a projective $R'$-module using the same construction. If we bundle this information together this gives us enough data to recover the local system.
\bigskip\\
Let $X / K$ be an adic space over $K$ with $K / \Q_p$ finite. Then there is a Fargues-Fontain curve 
\[ FF_X : \{ \text{perfectoid } \FF_p\text{-algebras} \} \to \mathbf{Set} \]
this takes the diamond $(X^{\diamond} \times \Q_p^{\diamond}) / \varphi_X^{\diamond}$ more explicitly
\[ FF_X(S) = \{ S^{\#}, S^{\#'} \text{ perfectoid algebras}, \Spa{S^{\#}} \to X \, \, S^{\#'} / \Q_p \, S^{\# \flat} \cong S \cong S^{\# ' \flat} \} \]
and we mod out by composing the first isomorphism with Frobenius.


\[ \R \Gamma_{\et}(X, \LL) \cong \R \Gamma(FF_X, \E_{\LL}) \]
and $\LL \mapsto \E_{\LL}$ is fully faithful. 

\section{The noncommutative MMP}

Motivation: the ordinary MMP we want to find the simplest variety birationally equivalent to a given $X$. For curves: there is no nontrivial birational equivalences. For surfaces it is easy, you just blow down $(-1)$-curves. This terminates when $K_X$ pairs nonegatively with each curve i.e. $K_X$ is nef.

Goal: find a ``minimal'' ($K_X$ nef) variety in each birational equivalence class.  MMP:
\begin{enumerate}
\item if $K_X$ is nef, we're done
\item otherwise by the cone theorem, there is an extremal ray generated by some rational curve $C$ on which $K_X \cdot C < 0$ then we contract $C$
\begin{enumerate}
\item if the contraction $X \to X'$ is a Fano fibration then $\dim{X'} < \dim{X}$ so we can induct on dimension
\item if the contraction $X \to X'$ is a divisorial contraction then the rank of the Picard group decreases so we can also induct
\item if the contraction $X \to X'$ is small then we are out of luck: we need to ``flip'' $X$ so that $K_X \cdot C > 0$. 
\end{enumerate}
\end{enumerate}

\section{Neron Model}

\newcommand{\Neron}{N\'{e}ron\xspace}

We are oven interested in finding integral models for objects such as abelian varieties. Meaning let $X_K \to \Spec{K}$ be some nice object over a number field we want to construct a model $X_{\struct{K}} \to \Spec{\struct{K}}$. Usually for nice objects (curves, abelian varieties, etc) there is an obvious unique choice at the primes of good reduction. The issue is chosing the model at the points of bad reduction.
\\
It suffices to construct models $X_R \to \Spec{R}$ for each dvr $R \subset K$ (in the number field case for each $\p \subset \struct{K}$ a model over $\struct{K,\p} \subset K$). So we reduce to the following problem:
\begin{center}
Given a dvr $R$ with fraction field $K$ and $X_K \to \Spec{K}$ find a canonical model $X_R \to \Spec{R}$.
\end{center}

There are many different particular sintuations and choices of solutions to this general setup (e.g. stable reduction of curves). 

\subsection{The N\'{e}ron model}

\newcommand{\sh}{\mathrm{sh}}


Specialize to the case that $X_K \to \Spec{K}$ is smooth and separated. Then we define the {(lft)-\Neron model} $X_R \to \Spec{R}$ to be the unique smooth separated (locally) of finite type $R$-scheme equipped with an isomorphism $\varphi (X_R)_K \iso X_K$ satisfying the N\'{e}ron mapping property:
\begin{center}
If $T \to \Spec{R}$ is a smooth separated $R$-scheme then the map
\[ \Hom{R}{T}{X_R} \to \Hom{K}{T_K}{X_K} \]
is a bijection. We are asking that any $K$-morphism $T_K \to X_K$ can be extended to a unique $R$-morphism $T \to X$. 
\end{center} 

If the N\'{e}ron model exists, it is unique up to unique isomorphism by Yoneda. 

\begin{rmk}
If $X_K$ is a smooth separated group scheme then so is the \Neron model. Indeed, via
\[ \Hom{R}{T}{X_R} \iso \Hom{K}{T_K}{X_K} \]
the functor $\Hom{R}{-}{X_R}$ on smooth separated $R$-schemes takes values in groups. Since $X_R \to \Spec{R}$ is smooth and separated, $X_R$ is a group object by the Yoneda lemma.  
\end{rmk}

\begin{example}
For $X_K = \Spec{K}$ the \Neron model is $X_R = \Spec{R}$. More generally, if $K' / K$ is a finite extension and $X_K = \Spec{K'}$ then if $K'/K$ is unramified $X_R = \Spec{R'}$. However, when $K'/K$ is ramified $X_R = \Spec{K'}$. This shows that the \Neron model of a smooth proper $K$-scheme may not be proper. This is the usual behavior in the case of ``bad reduction''. As we will see later, in many cases the \Neron model can be analogously computed by removing bad points from a natural ramified model.
\end{example}

In general it is quite difficult to prove the existence / non-existence of \Neron models. We need a few lemmas before continuing that revolve around the notion of boundedness.

\begin{defn}
Let $R$ be a DVR with fraction field $K$. Let $R \subset R'$ be a faithfully flat extension of DVRs and $K'$ the fraction field of $K'$. These fields are equipped with compatible absolute values $| \bullet |$ arising from the valuation. For a point $x \in X_K(K')$ and secton $g \in \Gamma(X_K, \struct{X_K})$ view $g(x) \in K'$ so we have a well-defined real number $|g(x)| \in \RR_{>0}$. Hence we can ask that a function $g$ is \textit{bounded} on (a subset of) $X_K(K')$. 
\begin{enumerate}
\item we say $S \subset \A^n_K(K')$ is \textit{bounded} if the coordinate functions are bounded on $S$
\item if $X_K$ is affine then $E \subset X_K(K')$ is \textit{bounded in} $X_K$ if there is an embedding $X_K \embed \A^n_K$ mapping $E$ to a bounded subset
\item if $X_K$ is any finite type $K$-scheme then $E \subset X_K(K')$ is \textit{bounded in} $X_K$ if there is a covering by finitely many affine opens $U_1, \dots, U_n$ a decomposition $E = E_1 \cup \cdots \cup E_n$ into subsets $E_i \subset U_i(K')$ such that $E_i$ is bounded in $U_i$.
\end{enumerate}
\end{defn}

\begin{example}
\begin{enumerate}
\item $\A^1(R) \subset \A^1(K)$ is bounded but $\A^1(K)$ is not bounded
\item $\Gm(K)$ is not bounded in $\Gm$
\item $\P^1(K) = \P^1(R)$ is bounded $\P^1_K$
\item if $X_K$ is proper, any subset of $X_K(K')$ is bounded in $X_K$ (use Chow's lemma and map to projective space)
\item if $T$ is an anisotropic torus (not containing $\Gm$) then $T(K)$ is bounded 
\end{enumerate}
\end{example}

\begin{theorem}[Neron Models, Thm 1.3/1, Cor 6.5/4]
Let $R$ be a DVR with fraction field $K$, with a strict henselization $R^{\sh}$ with field of fractions $K^{\sh}$. Let $G_K$ be a smooth $K$-group scheme of finite type. Then $X_K$ admits a \Neron model $X$ over $R$ if and only if $G_K(K^{\sh})$ is bounded. More generally, if $X_K$ is a torsor under $G_K$ then the following are equivlant
\begin{enumerate}
\item $X_K$ admits a \Neron model over $R$
\item $X_K(K^{\sh})$ is bounded in $X_K$
\item either $X_K$ is ramified (meaing $X_K(K^{\sh}) = \empty$) or $G_K(K^{\sh})$ is bounded in $G_K$
\end{enumerate}
\end{theorem}

\begin{cor}
Abelian varieties (and their torsors) admit \Neron models.
\end{cor}

\begin{example}
$\Ga$ and $\Gm$ do not admit \Neron models. However, $\Gm$ (unlike $\Ga$) admits an lft-\Neron model $G$ obtained by gluing $\Z$ copies of $(\Gm)_R$ along $(\Gm)_K$ so that
\[ G(R) = \Z \times R^\times \iso \Gm(K) \quad (n, r) \mapsto \pi^n r \]
\end{example}

\begin{example}
Let $R$ be a DVR of equicharacteristic $p > 0$ and $\pi$ a uniformizer. Consider the subgroup $G$ of $\Ga \times_R \Ga$ given by the equation
\[ x + x^p  + \pi y^p = 0 \]
Then $G$ is a smooth $R$-group scheme of finite type. The following criterion shows that it is the \Neron model of $G_K$ (hence $G_K$ is bounded). $G_K$ is an example of a so-called $K$-wound unipotent group, i.e.\ a connected unipotent algebraic $K$-group which does not contain $(\Ga)_K$ as a subgroup. 
\end{example}

\begin{prop}
Let $G$ be a smooth $R$-group scheme of finite type and $X$ a torsor under $G$ (e.g.\ $X = G$). Then the following are equivalent:
\begin{enumerate}
\item $X$ is a \Neron model of $X_K$
\item $X$ (hence $G$) is separated and the canonical map $X(R^{\sh}) \to X(K^{\sh})$ is surjective
\item the canonical map $X(R^{\sh}) \to X(K^{\sh})$ is bijective.
\end{enumerate}
\end{prop}

\begin{prop}
The \Neron model is compatible with \etale base change. 
\end{prop}

\subsection{Motivation: minimal regular models}

For a scheme $X_K \to \Spec{K}$ a \textit{model} over $R$ is a scheme $X_R \to \Spec{R}$ equipped with an isomorphsm $\varphi : (X_R)_K \iso X_K$.

Recal, given a regular proper curve $C_K \to \Spec{K}$ the minimal regular model of $C_K$ is the regular flat proper model $\C_{\min} \to \Spec{R}$ which is minimal in the sense of the following lemma.

{\color{red} FIND CITATION}

\begin{lemma}
Let $X \to \Spec{R}$ be a regular scheme which is a flat proper regular curve over $R$. Then the following are equivalent:
\begin{enumerate}
\item $X$ is \textit{minimal} in the sense that for any regular flat proper model $X' \to \Spec{R}$ of $X_K$ there is a unique morphism of models $X' \to X$
\item $X$ does not contain any $(-1)$-curve in its fibers and $p_a(X_K) > 0$
\item $K_{X/R}$ is nef 
\end{enumerate} 
\end{lemma}

\begin{rmk}
$\P^1$ does not have a minimal model in the above sense. This is intimately connected with the fact that $\P^1_K$ does not admit a \Neron model over $R$. The fundamental problem is that $\P^1$ has a non-proper automorphism group so there are automorphisms of $\P^1_K$ that do not extend to automorphisms of $\P^1_R$. For example, if $\pi$ is a uniformizer then 
\[ \begin{pmatrix}
1 & 0
\\
0 & \pi
\end{pmatrix} \cdot [x,y] = [x, \pi y] \]
is an automorphism of $\P^1_K$ but its specialization is not invertible so it cannot be extended to an automorphism of $\P^1_R$. In fact, by rigidity, it cannot be extended to a morphism at all (it can be extended away from $[0:1]$ on the special fiber). This shows that $\P^1_R$ does not respect the \Neron mapping property (with $\P^1_R$ as the test object) and it shows that $\P^1_R$ is not a minimal model of $\P^1$ otherwise $\P^1_R$ as a model of $\P^1_K$ identified through this automorphism whould extend to a morphism of models. 
\end{rmk}

\subsubsection{Relatonship to \Neron Models}

The main claim of this section is that, for curves, we can actually compute the \Neron model since it is related to the minimal regular model.

\begin{thm}[Tong-Liu]
Let $C \to \Spec{R}$ be a smooth proper geometrically connected curve with $g > 0$ and $\C_{\min}$ the minimal regular model. Then $(\C_{\min})^{\text{smooth}}$ is the \Neron model of $C$.
\end{thm}

It is difficult (and the subject of \chref{https://arxiv.org/pdf/1312.4822}{this} paper) to show that the \Neron model $\cN$ exists. Granting this, we can show how to identify it with $X = (\C_{\min})^{\text{smooth}}$. By the \Neron mapping property, there is a canonical map
\[ X \to \cN \]
since $X$ is, by definition, smooth over $R$. Now to use the universal property of the minimal regular model, compactify $\cN \embed \ol{\cN}$ to get a flat proper model (via Nagata)  

\begin{lemma}
Let $f : Y \to X$ be a map of $S$-schemes. Suppose that
\begin{enumerate}
\item $S$ is regular and noetherian
\item $Y \to S$ is smooth
\item $X \to S$ is flat and finite type
\item $X$ is a regular scheme
\end{enumerate}
then $f(Y)$ is contained in the smooth locus of $X \to S$.
\end{lemma}

\begin{proof}
Choose $y \in Y$ we need to show that $f(y)$ is contained in the smooth locus of $X \to S$. This question is \etale local on $S$ so, via the structure theorem for smooth morphisms, we may replace $S$ by an \etale cover such that $Y \to S$ has a section passing through $y$. This produces a section of $X \to S$ passing through $f(x)$. Therefore, it suffices to show that following: let $X \to S$ be a flat finite type map of noetherian regular schemes then any section is contained in the smooth locus of $X \to S$. Indeed, the section gives a section to the structure map on regular local rings so that the composite
\[ \stalk{S}{s} \to \stalk{X}{x} \to \stalk{S}{s} \]
is an isomorphism. Therefore, if we choose elements $x_1, \dots, x_d \in \m_s \sm \m_s^2$ giving a basis of $\m_s / \m_s^2$ then $x_i \notin \m_x^2$ otherwise we would have $x_i \in \m_s^2$ via the second map. This proves that 
\[ \stalk{X_s}{x} = \stalk{X}{x} / \m_s \stalk{X}{x} \]
is regular hence the fiber $X_s$ is regular at $x \in X_s$. Since $X \to S$ is flat, it suffices to prove that moreover $X_s$ is geometrically regular at $x$. Indeed, this is because $x$ is a $\kappa(s)$-point. If $R$ is a regular local $k$-algebra equipped with a $k$-map $R \to k$ (i.e.\ its residue field is $k$) then $R$ is geometrically regular. Indeed, the sequence
\[ 0 \to \m_R \ot_k \ol{k} \to R \ot_k \ol{k} \to \ol{k} \to 0 \]
is exact so $\m_R \ot_k \ol{k}$ is still a maximal ideal. However, it is generated by $\dim{R}$ elements so $R \ot_k \ol{k}$ is regular. 
\end{proof}

In particular, this applies to form \Neron models for elliptic curves and they are related closely to minimal models. It turns out there is a similar minimality property in higher-dimensions although (to by knowledge higher-dimensional \Neron models do not have nice canonical regular compactifications)

\begin{theorem}[Lorenzini, Theorem 3.2]
Let $R$ be a Dedekind domain and $A_K / K$ an abelian variety. Consider the category of smooth finite type group scheme $G \to S$ models of $A_K$ (meaning equipped with an isomorphism $G_K \iso A_K$). Then the \Neron model $A / S$ of $A_K / K$ is the unique terminal object of this category of models.
\end{theorem}

\begin{example}
$(\Gm)_R$ is the terminal model of $(\Gm)_K$ as above. However, it is not the \Neron model since $\Gm(R) \subsetneq \Gm(K)$ in particular we do not get a reduction map $\Gm(K) \to \Gm(k)$. 
\end{example}

\begin{example}
Let $X \to \Spec{R}$ be a regular elliptic surface whose special fiber is a multiple elliptic curve. Then $X_K$ is a torsor over an elliptic curve with good reduction so it has a proper \Neron model. However, the \Neron model of $X_K$ is $X^{\text{smooth}}$ by the above which deletes the entire special fiber. Hence the \Neron model of $X_K$ is not surjective over $\Spec{R}$.
\end{example}

\subsubsection{References}

\begin{enumerate}
\item \chref{http://alpha.math.uga.edu/~lorenz/LorenziniNeronModels1.pdf}{Lorenzini}
\item \Neron Models by Bosch
\end{enumerate}


\section{Ming Hao Feb. 3}

\begin{enumerate}
\item Principalization = sequences of smooth blowups are cofinal amoung all modifications of smooth $X$
\item Local uniformization
\item Fibration by curves
\end{enumerate}

Aside: weak/strong factorization (even stronger control on the system of modifications of fixed smooth $X$). 

\begin{defn}
Let $\pi : X" \to X$ be a modification (proper birational map) of regular varities. A \textit{factorization} of $\pi$ into a composition of regular blowups and blowdowns is called
\begin{enumerate}
\item \textit{weak} if the order is arbitrary
\item \textit{strong} if all the blowups preceed all the blowdowns.
\end{enumerate}
\end{defn}

\begin{theorem}[Morelli-toric case, Wtodarczyk-general case, Abramovich-Karu-Matuski-Wtodarczky]
weak factorization holds in characteristic $0$.
\end{theorem}

Question: does strong factorization hold? 

\begin{rmk}
Weak factorization in dimension $d$ is implied by equivariant resolution of singularities in dimension $d + 1$. Proof uses ``birational cobordisms''. 
\end{rmk}

Temkin's challenge: can we prove the other direction ``weak factorization in dimension $d$'' implies ``resolution of singularities in dimension $d+1$''. 

\subsection{Local uniformization}

\newcommand{\reg}{\mathrm{reg}}

The strategy that's dominant in positive characteristic. Local counterpart of the statement of resolution of singularities: let $X$ be a variety (integral, separated, finite type over $k$). To resolve $X$ locally, we also need to care about the local behavior on a blowup $X_1$ etc so it's not sufficient to work locally on just $X$ because of the step-by-step structure of the process. You want to also localize on the intermediate modifications as one builds a resolution. 
\\
Need to work in some ``birational topology'' of $X$ generated by Zariski covers and modifications of $X$. 

Zariski-Riemann spaces: 
\[ ZR(X) := \ilim_{\alpha} X'_\alpha \]
in the category of locally ringed spaces where $\alpha$ ranges over all modifications $X_\alpha' \to X$. Set of modifications form a filtered inverse system: form the ``join''
\begin{center}
\begin{tikzcd}
& X'_{\alpha\beta} \arrow[ld] \arrow[rd]
\\
X_{\alpha}' \arrow[rd] & & X_{\beta}' \arrow[ld]
\\
& X
\end{tikzcd}
\end{center}
where we define $X'_{\alpha\beta}$ as the graph of the induced birational map $X'_\alpha \rat X'_\beta$ which is the scheme-theoretic closure in $X'_\alpha \times X_\beta$ of the graph of some representative on an open set of $X'_\alpha \rat X_\beta'$. 

Concretely,
\begin{enumerate}
\item points $x_\alpha \in X_\alpha'$ inverse system
\item topology: smallest topology such that $\pi_\alpha : ZR(X) \to X_\alpha'$
\item stalks:
\[ \stalk{ZR(X)}{(x_\alpha)} = \dlim_{\alpha} \stalk{X_\alpha}{x_\alpha} \]
is a valuation ring of the function field $K(X)$ of $X$ 
\end{enumerate}
Consequently, if $X$ is separated we have the following alternative description of $ZR(X)$:
\begin{enumerate}
\item as a set $ZR(X)$ is the set of valuations of $X$ has a center on $X$
\item as a topological space has a bass
\[ \mathcal{B} = \{ U(x_1, \dots, x_k) : x_1, \dots, x_k \in K(X)^\times \} \]
where
\[ U(x_1, \dots, x_k) := \{ v \in ZR(X) \mid x_1, \dots, x_k \in R_v \} \]
where $R_v$ is the valuation ring for a valuation $v$ of $K(X)$.
\end{enumerate}
In this description:
\begin{enumerate}
\item the map
\[ ZR(X) \to X'_\alpha \]
is the map $v$ to the center of $v$ on $X_\alpha'$
\item if $X$ is proper over $k$ then $ZR(X)$ is the set of all $k$-trivial valuations on $K(X)$
\item $ZR(X)$ is quasi-compact. 
\end{enumerate}

\begin{defn}
A local unifromation of $X$ along a valuation $v \in ZR(X)$ is a \textit{separated} birational morphism $\pi : X' \to X$ such that $v$ is centered on the regular locus of $X'$. 
\end{defn}

\begin{rmk}
replacing separated with proper gives the same definition (by Nagata compactification). 
\end{rmk}

State of the art:
\begin{enumerate}
\item Zariski: char = 0, (global resolution follows in dim = 3)
\item Abhyankar: char > 0, dim = 2 and dim = 3, char > 3
\item Cossart-Pittant: dim = 3, char > 0
\end{enumerate}

How does one go from local uniformation to global resolution? 

Zariski's strategy: since $ZR(X)$  is quasi-compact, local uniformization implies $\exists : \pi_i : X_i' \to X$ for $i=1,2,\dots,m$ such that $\forall v \in ZR(X)$ there is $i$ such that $v$ is centered on $(X_i')_{\reg}$.

\subsubsection{Zariski Patching}

Modify $X_i$ further to some common modfication $X' \xrightarrow{\varphi_i} X_i' \xrightarrow{\pi_i} X$ such that
\[ \varphi_i^{-1}((X_i')_{\red}) \subset X_{\reg}' (\ast) \]

Need some good control on the system of nonsingular modification of a non-singular schemes e.g. principalization factorization/

Only have good control in low dimensions - as demonstrated by: Zariski's parthcing holds in dimension $3$. Crucial case: $m = 2$. A natural candidate for patching $X_1'$ and $X_2'$ is the join but no reason that property $(\ast)$ should hold. 
\\
A sufficient condition for $(\ast)$ is 
\begin{center}
$(\dagger)$ for every $v \in ZR(X)$ with centers $x_i$ on $X_i'$ either one of $x_i$ such satisfy:
\begin{enumerate}
\item $x_i \in (X_i')_{\reg}$
\item $x_i \notin$ indet. locus of $X_1' \rat X_2'$.
\end{enumerate}
\end{center}

For $i = 1$ this is asking that $x_1$ is regular point of $X_1'$ and $\stalk{X_1}{x_1}$ dominates $\stalk{X_2'}{x_2}$.

\section{Feb. 5}

Zariski patching, $m = 2$.

Last time: a sufficent condition for this to hold is the followng:
\begin{center}
for all $v \in ZR(X)$ with centers $x_i$ on $X_i'$ at least one of $x_1, x_2$ satisfy
\begin{enumerate}
\item $x_i \in (X_i')_{\reg}$
\item $x_2$ not in the base locus of $X_1' \rat X_2'$
\end{enumerate}
\end{center}

\begin{lemma}[Lemma 1, Elimination of regular fundamental points while preserving the regular locus]
$\pi : Y \rat Z$ for $3$-dimensional projective over $k$ with $F$ the ill-defined locus then there exists a sequence of regular blowups so that no point of $\pi^{-1}(F \cap Y_{\reg})$ is fundamental for $T \circ \pi$. 
\end{lemma}

This lemma is fundamental for Zariski's approach. Modern approach is to use Hironaka's principalization. 

\begin{theorem}[Cossart's thesis]
log principalization of ideals on regular $3$-folds hold in characteristic $p > 0$
\end{theorem}

Uses Hironaka's characteristic polyhedron. 

\begin{rmk}
Cossart's principalization is not smooth functorial so we cannot glue to get global resolution.
\end{rmk}

\subsection{Cossart's proof of Lemma 1}

Consider the join $W$
\begin{center}
\begin{tikzcd}
& Y \arrow[dd, dashed]
\\
W \arrow[rd] \arrow[ru]
\\
& T
\end{tikzcd}
\end{center}
here $W \to Y$ is projective so $W \to Y$ is the blowup of some $I \subset \struct{Y}$. We apply principalization to $I|_{Y_{\reg}} \subset \struct{Y_{\reg}}$ so there exists
\[ q : U_r \to U_{r-1} \to \cdots \to U_0 = Y_{\reg} \]
such that $I|_{U_0} \cdot \struct{U_r}$ is an invertible exceptional divisor. 
\\
Note: if $Y = Y_{\reg}$ we are done since by the universal property of blowups:
\begin{center}
\begin{tikzcd}
& U_r \arrow[d, "q"]
\\
& Y \arrow[dd, "T", dashed]
\\
W \arrow[ru] \arrow[rd]
\\
& Z
\end{tikzcd}
\end{center}

then $q \circ T$ is $U_r \to W \to Z$ is a morphism. 
\\
What happens in general, $Y_{\reg} \subset Y$ we need to extend the sequence of regular blowups $U_r \to Y_{\reg}$ to a sequence of regular blowups of $Y$. Then we are done by the previous case (since we want the morphism to be well-defined over the regular locus). How? Since $\dim = 3$, each blowup locus $Z_i \subset U_i$ is either a closed point or a regular curve. For the point, it is obvious what to do. For a curve, take the closure and then do embedded resolution and blowup along the strict transform. 
\\
Inductively, define $Y_{i+1} \to Y_{i}$ with $Y_0 = Y$:
\begin{enumerate}
\item if the blowup locus $Z_i \subset U_i$ of $U_{i+1} \to U_i$ is a closed point then $Y_{i+1} \to Y_i$ is the blowup at $Z_i$
\item if $Z_i \subset U_i$ is a regular curve then let $\ol{Z}_i$ be the closure in $Y_i$ and let $Y_{i+1} \to Y_i' \to Y_i$ be the composition of the embedded resolution of $\ol{Z}_i$ (via blowups at points) and then the blowup at the strict transform of $\ol{Z}_i$. 
\end{enumerate}


\begin{rmk}
Principalization only implies local resolution not global resolution unless it is functorial. This is because we can only embed our singular variety locally in a smooth variety. 
\end{rmk}

\begin{lemma}[Zariski-Abhyankar factorization lemma]
Let $R, R'$ be regular $2$-dimensional local rings with $K(R) = K(R')$ and assume $R'$ dominates $R$ then $R'$ cab be obtained from $R$ by a sequence of closed point blowups (and localizations).  
\end{lemma}

\subsection{Stratgies of local uniformization (in positive characteristic)}

\begin{enumerate}
\item Abhyankar: first generalizes a method of Albanese: $V$ projective variety of dimension $d$ then project from points of high multiplicity on $V$ and normalize in $K(V)$ -> get a projective variety $Y$ birational to $V$ with singularities of small ($\le d!$) multiplicity.
\item  
\end{enumerate}

\section{\Neron models of Lagrangian fibrations}

Everything over $\CC$. 

\subsection{\Neron models}

Let $B$ be smooth over $\CC$ and $B_0 \subset B$ a Zariski open. Given a smooth morphism $N_0 \to B_0$ I want to find an extension
\begin{center}
\begin{tikzcd}
N_0 \arrow[d] \arrow[r] \pullback & N \arrow[d]
\\
B_0 \arrow[r] & B
\end{tikzcd}
\end{center}
A \textit{\Neron model} of $N_0 / B_0$ is a smooth extension $N / B$ such that the \Neron mapping property holds: for any smooth $Z \to B$ the restriction map
\[ \Hom{B}{Z}{N} \iso \Hom{B_0}{Z_0}{N_0} \]
is bijective. 

\begin{rmk}
\begin{enumerate}
\item \Neron defined it only when $B$ is a curve
\item unique up to unique isomorphism
\item extistence is usually quite difficult 
\item Need algebraic spaces instead of schemes in general for existence
\item \Neron model, when it is exists, may be nonseparated and non-quasi-compact.
\end{enumerate}
\end{rmk}

\begin{prop}
Construction of \Neron model is (\etale) local on $B$.
\end{prop}

\begin{proof}
It glues because of the uniqueness of \Neron models and the fact that \Neron models are compatible with \etale base change. This is where we might need algebraic spaces.
\end{proof}

\begin{prop}
Start with a group scheme $G_0 \to B_0$. The \Neron model, if it exists, is also a group scheme.
\end{prop}

\begin{proof}
Automatic from the smoothness using the \Neron mapping property to extend the maps defining a group object.
\end{proof}

\begin{example}
Say $A_0 \to B_0$ is a family of abelian varieties and assume there is an extension to a family of abelian varities $A \to B$. Then $A \to B$ is the \Neron model.
\end{example}

\begin{proof}
Suppose we have a diagram
\begin{center}
\begin{tikzcd}
Z \arrow[rr, dashed] \arrow[rd] & & A \arrow[ld]
\\
& B
\end{tikzcd}
\end{center}
with $Z \to B$ smooth. Consider the graph $\Gamma \subset Z \times_B A$ suppose $\Gamma \to Z$ is not an isomorphism. Since $Z \to B$ is smooth, the non-isomorphic part is uniruled and these are contracted in the map to $A$ so we must have $\Gamma \to Z$ is an isomorphism so the map extends. 
\end{proof}

\begin{example}
$\pi : S \to \Delta$ minimal elliptic fibration, then \Neron and Raynaud shows that $S' = S \sm \mathrm{Sing}(\pi) \xrightarrow{\pi} \Delta$ is the \Neron model of $S_0 \to \Delta_0$ a smooth proper morphism. 
\end{example}

\begin{rmk}
\begin{enumerate}
\item \Neron proved that if $B = \Delta$ is a curve and $N_0 \to \Delta_0$ is an abelian-torsor then there exists a \Neron model
\item (Holmes 2019) gave counterexamples to existence of \Neron model for most abelian varieties if $\dim{B} \ge 2$
\end{enumerate}
\end{rmk}

\subsection{Lagrangian Fibrations}

\begin{defn}
A \textit{symplectic variety} is a smooth variety with an everywhere nondegenerated closed holomorphic $2$-form $\sigma \in H^0(X, \Omega_X^2)$.
\end{defn}

\begin{defn}
A \textit{Lagrangian fibration} is a proper fibration $\pi : X \to B$ such that all smooth fibers are Lagrangian subvarieties (meaning $\omega$ restricts to zero).
\end{defn}

\begin{example}
\begin{enumerate}
\item Hitchin: let $C$ be a smooth projective curve $g \gw$ the moduli space of stable Higgs bundles with rank $r$ and degree $d$ and $\gcd(r,d) = 1$. It comes with a Lagrangian fibration $\pi : X \to B$ called the Hitchin map
\item A compact hyperkahler $X$ is a compact smooth symplectic variety. Every nontrivial fibration $\pi : X \to B$ is always Lagrangian.
\end{enumerate}
\end{example}

From now on, $X$ is a smooth symplectic variety, $B$ is a smoth variety, and $\pi : X \to B$ is a projective Lagrangian fibration. Assume for simplicity there exists a section $s : B \to X$. 

\begin{theorem}
The restriction $\pi_0 : X_0 \to B_0$ over the smooth locus $B_0 \subset B$ admits a \Neron model $P = X^n$ which group algebraic space.
\end{theorem}

This shows that Lagrangian fibrations are very special. 
\\
Consider $X' := X \sm \Sing(\pi)$. Is $\pi' : X' \to B$ equal to the \Neron model? No! There is a locus $B_1 \subset B$ whose complement has codimension $2$ called the $\delta$-regular subset where this holds. 

\begin{theorem}
\begin{enumerate}
\item $X^n = \bigcup Y_U'$ where $U \to B$ is \etale and consider all smooth symplectic varieties with a proper Lagrangian fibration $\tau : Y_U \to U$ birational to $X_U$ over $U$ then set $Y'_U = Y_U \sm \Sing(\tau)$ glued over the largest open subset where $X_U \to Y_U$ is an isomorphism
\item there is a prefered choice of a codimension $\ge 2$ complement open set $B_1 \subset B$ such that $X^n|_{B_1} = X'|_{B_1}$ (this means all birational automorphisms over $B$ is an isomorphism over $B_1$). 
\end{enumerate}
\end{theorem}

\begin{example}
Let $\dim{X} = 4$ and $X \to \B$ a Lagrangian fibration with a point $b \in B$ such that $X_b$ has three components and exactly one is $\P^2$. Doing a Mukai flop to $\P^2$ we get a new model $\check{X}$. The complement of the singular loci of both $X$ and $\check{X}$ must both embed into $X^n$ this forces $X^n$ to be nonseparated. 
\end{example}

\begin{theorem}
\begin{enumerate}
\item The \Neron model as a group space $P \to B$ is the moduli space of $\pi$-birational automorphisms: for all $U \to B$ \etale then $P(U)$ is the collection of birational automorphisms $X_U \rat X_U$ over $U$ which are translations on the generic fiber.  
\item there exists a quasi-projective open subgroup scheme $P^a \subset P$ parametrizing automorphisms in the same above sense
\item $P^a$ is $\delta$-regular.
\end{enumerate}
\end{theorem}

\section{Arithmetic Hyperbolicity}

Let $K$ be a number field. Conjectures due to Bombieri-Lang:

\begin{conj}[Weak form]
Let $X$ be a smooth surface of general type then $X(K)$ is not Zariski dense.
\end{conj}

\begin{conj}[general form]
Let $X$ be a smooth general type variety then $X(K)$ is not Zarisiki dense.
\end{conj}

\begin{conj}[refined form]
For any $X / K$ smooth of general type there exists $U \subset X$ Zariski open such that for any finite $K'/K$ the set $U(K')$ is finite. 
\end{conj}

\begin{theorem}[Faltings]
For $\dim{X} = 1$, the above conjectures are true, in fact we can take $X = U$.
\end{theorem}

Analytic hyperbolicity: connection with arithmetic is that $K$-points could correspond to entire curves $\C \to X(\C)$.
\\
Goals: survey the following theorem

\begin{theorem}[Diverio-Merker-Rousseau, 2010]
Let $X_d \subset \P^{n+1}$ be a smooth hypersurface. If $X_d$ is very general and $d \ge 2^{n^5}$ then $X_d$ is hyperbolic
\end{theorem}

\begin{rmk}
$X$ general type, then every entire curve satisfies a differential equation. 
\end{rmk}

\begin{conj}
Let $X \subset \P^{n+r}$ be a general complete intersection and $d_i \gg 0$ and $r \ge n$ then $\Omega_X$ is ample. 
\end{conj}

This is now a theorem of Brotbeck.

\subsection{Kobayashi hyperbolicity}

$\eta \in \Delta \subset \CC$ unit disk then the Poincar\'{e} metric 
\[ \d{s}^2 = \frac{\d{\eta} \ot \d{\bar{\eta}}}{(1 - |\eta|^2)^2} \]
This is the unique complete metric of constant curvature (scaled to lie in $\{ 0, 1, - 1\}$)
Starting point: want intrinsic metric built from $\Delta \to X$. For $a, b \in \Delta$ the integrated distance is
\[ \rho(a,b) = \tanh^{-1} \left| \frac{a - b}{1 - a \bar{b}} \right| \]
Then on $X$ we define a length function by taking the infimum over all patching by chains of disks 
\[ \ell(\alpha) = \rho(p_1, p_2) + \rho(p_2, p_3) + \cdots + \rho(p_{k-1}, p_k) \]
and then the Kobayashi pseduodistance:
\[ d_X(p,q) = \inf_\alpha \ell(\alpha) \]

\begin{defn}
Let $X$ be a complex manifold. Then $X$ is \textit{Kobayashi hyperbolic} if $d_X$ is a distance.
\end{defn}

\begin{lemma}
$d_\Delta = \rho$
\end{lemma}

\begin{rmk}
$d_{\CC} = 0$.
\end{rmk}

Property: $d_X$ is contracted under holomorphic maps $f : X \to Y$ then for all $x, y \in Y$
\[ d_Y(f(x), f(y)) \le d_X(x,y) \]

\begin{defn}
Let $X$ be a complex manifold and $v \in T_{x_0} X$ then let
\[ k_x(v) = \inf \{ \lambda > 0 \mid \exists f : \Delta \to X : f(0) = x_0, \lambda f'(0) = v \} \]
\end{defn}

This intrinsic notion of length agrees with the Poincare metric for $X = \Delta$.

\begin{theorem}[Royden]
Let $X$ be a complex manifold then 
\[ d_X(p,q) = \inf_\gamma \int_\gamma k_X(\gamma'(t)) \d{\gamma} \]
for Piecewise smooth paths $\gamma : [0,1] \to X$ from $p \mapsto q$
\end{theorem}

\begin{defn}
Say $X$ is \textit{infinitesimally Kobayashi hyperbolic} $k_X$ is uniformly bounded below compared to some hermitian metric. 
\end{defn}

\begin{cor}
Infinitesimally hyperbolic implies Kobayashi hyperbolic.
\end{cor}

\begin{prop}
If $X$ is Kobayashi hyperbolic then every $f : \CC \to X$ is constant.
\end{prop}

\begin{proof}
$d_{\CC} = 0$ and use the length decreasing property.
\end{proof}

\begin{defn}
A \textit{entire} curve $f : \CC \to X$ (meaning holomorphic and nonconstant) is a \textit{Brody} curve if it has bounded derivative with respect to some choice of Hermitian metrics. 
\end{defn}

\begin{theorem}
Let $X$ be a compact complex manifold. If $X$ is not infinitesimally hyperbolic then then there exists a Brody curve. 
\end{theorem}

\begin{cor}
$X$ compact manifold then hyperbolicitiy = infinitesimal hyperbolicity = no entre curves. 
\end{cor}

Some applications:

\begin{theorem}
Let $X \subset T$ be a complex submanifold with $T$ a complex torus then $X$ is hyperbolic iff it does not contain any translates of subtori. 
\end{theorem}

\section{Bend-and-Break Stacks}

Let $\X$ be a smooth proper separated DM-stack. Let $\pi : \X \to X$ be the coarse space and assume it is projective. Let $L$ be an ample line bundle on $X$. 

\end{document}
\documentclass[12pt]{article}
\usepackage{import}
\import{../}{AlgGeoCommands}

\newcommand{\dbar}{\bar{\partial}}
\newcommand{\HH}{\mathbb{H}}
\renewcommand{\gr}{\mathrm{gr}}
\newcommand{\R}{\mathrm{R}}

\renewcommand{\H}{\mathcal{H}}
\newcommand{\LL}{\mathbb{L}}

\begin{document}
\author{Benjamin Church}
\title{\Huge Measures of Irrationality and Degrees of Curves on Random Algebraic Varieties}

\maketitle

\section{Introduction: Measures of Irrationality}

My research focuses on what are called ``measures of irrationality''. These are quantitative ways of extending the classical rationality question in algebraic geometry to get a sort of ``distance'' or metric on the set of algebraic varieties considered up to birational equivalence. 
\par 
Rationality is one of the oldest and most fundamental ideals in algebraic geometry. Fundamentally, it concerns the solvability of a polynomial function $F(x_1, \dots, x_n) \in \CC[x_1, \dots, x_n]$ in $n$ variables. The zero set of this polynomial $X \subset \CC^n$ is called the affine variety cut out by the equation $F$. We say that $X$ is \textit{rational} if (almost all) of the solutions of $F$ can be parametrized by rational functions. This means that there exist rational functions $G_1, \dots, G_n \in \CC(t_1, \dots, t_{n-1})$ in $n-1$ indeterminants that solve $F$ menaing 
\[ F(G_1(t_1, \dots, t_{n-1}), \dots, G_n(t_1, \dots, t_{n-1})) = 0 \]
and moreover (almost every) point $(x_1, \dots, x_n) \in X$ is $(G_1(t_1, \dots, t_{n-1}, \dots, G_n(t_1, \dots, t_{n-1})$ for exactly one point $(t_1, \dots, t_{n-1}) \in \CC^{n-1}$. 

\begin{example}
Euclid's parametrization of pythagorean triples is the most famous example of a variety having such a rational parametrization. In this case 
\[ F(x,y) = x^2 + y^2 - 1 \]
so the set of solutions $X$ is the unit circle (for $x,y \in \RR$ but we can also consider complex solutions). By performing steriographic projection from the point $(0,1)$ we obtain a map
\[ t \mapsto \left( \frac{2t}{t^2 + 1}, \frac{t^2 - 1}{t^2 + 1} \right) \]
that hits every solution to $F(x,y) = 0$ exactly once except for $(0, 1)$. This shows that $F$ defines a rational variety. This parametrization allows produces all pythagorean tiples by plugging in $t \in \QQ$.
\end{example}

Algebraic geometers write this by saying there is a \textit{birational map} $\A^{n-1} \to X$ where $\A^{n-1}$ is the affine variety corresponding to the complex space $\CC^{n-1}$ and a birational map is the same as saying there is a polynomially defined isomorphism between an open set of $\A^{n-1}$ and an open set of $X$. 

\begin{defn}
We say that algebraic varieties $X$ and $Y$ are \textit{birationally equivalent} if they contain isomorphic open sets equivalently if there is a birational map $X \birat Y$.
\end{defn}

\newcommand{\irr}{\mathrm{irr}}
\newcommand{\covgon}{\mathrm{cov.gon}}

Suppose $X$ is not rational, we can then ask: how far is $X$ from being rational? The goal of measures of irrationality are to give quantitative answers to these questions. The most popular measures are the \textit{degree of irrationality}
\[ \irr(X)\ :=\ \min\big\{\delta>0\ |\ \exists\textup{ rational dominant map } X \rat \A^n\textup{ of degree }\delta\big\}; \]
which is the minimal degree of a map to projective space defined on some open set of $X$. Conversely, the \textit{covering gonality} is defined by how complicated the $1$-dimensional subvarieties (curves) that cover $X$ are. We define
\[ \covgon(X)\ :=\ \min\big\{c>0\ |\ \exists\textup{ a curve of gonality } c \textup{ through a general point } x\in X\big\}.\]
where the gonality of a curve is the minimal degree of a rational map $C \to \A^1$. More generally, these can be extended to so called ``measures of association'' between any two varieties hence giving a metric on the set of birational equivalence classes. 

\subsection{Measures of Irrationality for ``Random Varieties''}

One source of algebraic varieties that has generated significant interest are those cut out by a sequence $F_1, \dots, F_r$ of polynomial in $n + r$ variables (so that they define a variety of dimension $n$) whose coefficients are chosen at random. We call a variety obtained this way a \textit{general complete intersection}. Numerical properties should not depend on the specific polynomials chosen, only on their degrees, as long as the coefficients are chosen sufficiently generally. 
\par 
Bastianelli--De Poi--Ein--Lazarsfeld--Ullery {\color{red} CITE} computed the measures of irrationality of a general hypersurface (i.e.\ the case $r = 1$) and found that the naive upper bounds are, asymtotically in the degree $F$, obtained. One interpretation of this result is that a variety defined by a random polynomial 
is ``as far from'' being rational as possible. In \cite{BDELU} the authors conjecture that the same should hold for $r > 1$. With Nathan Chen and Junyan Zhao we settled this conjecture. 
 {\color{red} CITE} 

\subsection{Future Work}


\end{document}
\documentclass[12pt]{article}
\usepackage{import}
\import{../}{AlgGeoCommands}

\begin{document}

\section{Sheaves Stuff}

Question: does there exists an fpqc torsor for a reasonable group not representable by an algebraic space?

\newcommand{\Shv}{\mathrm{Shv}}

\begin{lemma}
Descent holds along a $\tau$-cover for sheaves in the $\tau$-topology. Explicitly, let $\C_\tau$ be a site and consider the natural map
\[ \Shv_S(\C_\tau) \to \mathrm{DD}_{S'/S}(\Shv_{S'}(\C_\tau)) \]
is an equivaence of categories. 
\end{lemma}

\begin{rmk}
Note that $\Shv_S(\C_\tau)$, the slice category of sheaces on $\C_\tau$ over the representable $h^S$ (in presheaves if $h^S$ is not a $\tau$-sheaf), is equivalent to $\Shv(\C_{\tau / S})$ the sheaves on the slice category over $S$. Indeed, the map $\varphi : F \to S$ gives a map $F(T) \to \Hom{}{T}{S}$ so it lives over the slice category already. Conversely, given a sheaf $G$ on the slice category we define $F$ via 
\[ T \mapsto \{ (\alpha, \beta) \mid \alpha : T \to S \text{ and } \beta \in G(\alpha : T \to S) \} \]
\end{rmk}

\begin{proof}
This is just unwinding definitions. For full faithfulness, we need to show that
\[ \Hom{S}{F}{G} \to \Hom{S'}{F_{S'}}{G_{S'}} \rightrightarrows \Hom{S' \times_S S'}{F_{S' \times_S S'}}{G_{S' \times_S S'}} \]
is an equalizer. This is exactly the sheaf condition for $\Hom{}{F}{G}$. Indeed, let's prove it. Let $\varphi, \psi : F \to G$ be $S$-morphisms that become equal upon pulling back to $S'$. For any $T \to S$ consider the cover $T_{S'} \to T$ then $\varphi_{T_{S'}} = \psi_{T_{S'}}$ so by local uniqueness: $\varphi_T = \psi_T$. Now supppose that $\varphi' : F_{S'} \to G_{S'}$ is equalized. Let $\varphi$ be defined as follows: $\varphi_T(x) \in G(T)$ is obtained by gluing $\varphi_{T_{S'}}(x|_{T_{S'}})$ along $T_{S'} \to T$ which exists because of the overlap condition on $\varphi_T$.
\bigskip\\
Now we prove essential surjectivity. Let $(G, \alpha)$ be a descent datum. We produce a sheaf $F$ as follows. Base changing along $T \to S$ we can replace $S$ by any $T$ so it suffices to produce $F(S)$. Define $F(S)$ as the limit (equalizer) of the diagram
\begin{center}
\begin{tikzcd}
& & G(\pi_1 : S' \times_S S' \to S') \arrow[dd, "\alpha"]
\\
F(S) \arrow[r] & G(S') \arrow[ru] \arrow[rd]
\\
& & G(\pi_2 : S' \times_S S' \to S')
\end{tikzcd}
\end{center}
\end{proof}

\section{Accessible Categories}

Lurie works only with $\infty$-categories that are sets basically by definition since an $\infty$-category is a simplicial set. To differentiate between ``small'' and ``large'' he fixes a regular cardinal $\kappa$ (meaning it is not a limit over less than $\kappa$ smaller cardinals, eg. an inaccessible limit cardinal) and lets the the ``small'' simplicial sets be those in the corresponding Grothendieck universe of sets of rank $\le \kappa$ in the Von Neumann hierarchy.

\begin{defn}
An $\infty$-category $\C$ is $\kappa$-accessible if it is closed under $\kappa$-filtered colimits and there exists a $\kappa$-small subcategory $\C^0 \subset \C$ such that the natural map
\[ \mathrm{Ind}_{\kappa}(\C^0) \to \C \]
is an equivalence.
\end{defn}

Usually people say ``$\C$ is accessible if it is generated by $\C^0$ under $\kappa$-filtered colimits'' which is true but confusing since it is really a stronger property than ``everything is a colimit''. The natural map being an equivalence says that $\C$ really is the category of Ind-objects not just a quotient of it. For example, the category of free $R$-modules is not accessible. It is obviously generated under colimits by the trivial module $R$ but it is not isomorphic to the ind-objects since filtered colimits produce all flat $R$-modules. These is a filtered colimit of frees that gives a non-free finite projective and we are required to have this as well. The definition is equivalent to:

\begin{lemma}
An $\infty$-category $\C$ is $\kappa$-accessible if and only if it is
\begin{enumerate}
\item locally small 
\item closed under $\kappa$-filtered colimits
\item the full subcategory $\C^\kappa \subset \C$ of $\kappa$-compact objects is essential small
\item $\C^\kappa$ generates $\C$ under small, 
$\kappa$-filtered colimits
\end{enumerate}
\end{lemma}

\section{Stable Motivic Homotopy Theory}

\newcommand{\Cat}{\mathrm{Cat}}

Stable category: natural home for compatible sequences of spaces. Natural source for cohomology theories. 
\par 
Naive way: sequential spectra: a sequence of spaces $\{ X_n \}_{n \ge 0}$ and bonding maps $\sigma_n : \Sigma X_n \to X_{n+1}$.

\begin{defn}
If $\C$ is a category and $F : \C \to \C$ is a functor then define
\[ \Sp^{\N}(\C, F) := \colim{( \C \xrightarrow{F} \C \xrightarrow{F} \C \to \cdots)} \]
\end{defn}

The problem is it is hard to preserve nice properties of $\C$ under this construction. Nice properties:
\begin{enumerate}
\item presentable
\item symmetric monoidal structure
\end{enumerate}

Issue with presentability: $\Pr^L \subset \Cat_{\infty}$ is not closed under colimits. However, there is a hacky trick.

\begin{prop}
If $\C$ is presentable and $G$ is a right adjoint to $F$ then
\[ \Sp^{\NN}(\G, F) \iso \lim{(\cdots \to \C \xrightarrow{G} \C \xrightarrow{G} \C)} \]
in particular it is presentable since $\Pr^R \subset \Cat_{\infty}$ is limit-closed. 
\end{prop}

\begin{example}
Say we want to invert $\Sigma$ on Spaces. Instead of the colimit of iterating $\Sigma$ we use the right adjoint $\Omega$ to form spectra via a limit. 
\end{example}

More generally if $\C$ is pointed and has limits then there is an endofunctor
\[ \Omega : \C \to \C \]
given by taking the limit of the diagram
\begin{center}
\begin{tikzcd}
& * \arrow[d]
\\
* \arrow[r] & X 
\end{tikzcd}
\end{center}
Then 
\[ \Sp(\C) = \lim{( \cdots \to \C \xrightarrow{\Omega} \C \xrightarrow{\Omega} \C)} \]

\begin{defn}
An object $X \in \C$ in a symmetric monoidal category is \textit{symmetric} if for some $n \ge 2$ the $n$-cycle 
\[ (1 \, 2 \, \dots \, n) : X^{\ot n} \to X^{\ot n} \]
is homotopic to the identity. 
\end{defn}

\begin{theorem}
If $\C$ is presentably symmetric monoid, and $X \in \C$, then there is a natural functor
\[ \Sp^{\N}(\C, X \ot -) \to \C[X^{-1}] \]
is an equivalence if $X$ is symmetric. 
\end{theorem}

\begin{cor}
The category of spectra, as a presentably symmetric monoidal category, can be modeled in three equivalent ways:
\begin{enumerate}
\item $\colim{(S_* \xrightarrow{\Sigma} S_* \xrightarrow{\Sigma} S_* \to \cdots)}$
\item $\lim{(\cdots \to S_* \xrightarrow{\Omega} S_* \xrightarrow{\Omega} S_*)}$
\item $S_*[(S^1)^{-1}]$ 
\end{enumerate}
\end{cor}

\begin{proof}
The first two are by adjunction. For the last, we need check that $S^1$ is symmetric. Indeed,
\[ (1 \, 2 \, 3) : S^1 \wedge S^1 \wedge S^1 \to S^1 \wedge S^1 \wedge S^3 \]
is homotopic to the identity as a self-map of $S^3$. 
\end{proof}

We get a natural adjunction:
\[ \Sigma^{\infty} : \C \to \Sp(\C) : \Omega^\infty \]


\subsection{Motivic Spectra}

\newcommand{\PSh}{\mathrm{PSh}}
\newcommand{\Sm}{\mathrm{Sm}}
\newcommand{\Fun}{\mathrm{Fun}}
\newcommand{\Nis}{\mathrm{Nis}}

Could take $\PSh(\Sm_k)$ and stabilize it, we would get $\Fun(\Sm_k^{\op}, \Sp)$. Could look at the presheaves that are Nisnevich sheaves of spectra. Denote this by
\[ \Sp(k) = \Shv_{\Nis}(\Sm_k, \Sp) = \Sp(\Shv_{\Nis}(\Sm_k)) \]

\begin{example}
For any sheaf of abelian groups $A$, get a reoresenting object $H A \in \Sp(k)$, defined as $\{ K(A, n) \}_{n \ge 0}$ along with the maps $K(A, n) \iso \Omega K(A, n+1)$. 
\end{example}

\begin{prop}[representability of cohomology]
For $X \in \Sm_k$ and $n \ge 0$
\[ H^n_{\Nis}(X, A) = [ \Sigma^{-n} \Sigma^\infty_+ X, H A]_{\Sp(k)} \]
\end{prop}

\begin{proof}
\begin{align*}
[\Sigma_+^\infty \Sigma^n X, HA]_{\Sp(k)} & \cong [\Sigma^n X_+, \Omega^\infty HA]_{\Shv}
\\
& \cong [\Sigma^n X_+, K(A, 0)] 
\\
& \cong [ X_+, \Omega^n K(A, 0)]_{\Shv_*}
\\
& \cong [X_+, K(A,n)]_{\Shv_*} 
\\
& = [X, K(A,n)]_{\Shv}
\\
& = H^n(X, A)
\end{align*}
\end{proof}

\begin{defn}
For $E \in \Sp(k)$, can define $\pi_n(E)$ to be the sheafifcation of the presheaf
\[ U \mapsto [\Sigma_+^\infty \Sigma^n U, E]_{\Sp(k)} \]
\end{defn}

\begin{example}
\[ \pi_n HA = 
\begin{cases}
0 & n \neq 0
\\
A & n = 0
\end{cases} \]
\end{example}

This induces a $t$-structure such that
\[ \Sp(k)^{\heart} = \Ab(\Shv(\Sm_k)_{\le 0}) \]

Notation: Denote by $\Sp_{\A^1}(k) \subset \Sp(k)$ the full subcategory of $\A^1$-invariant sheaves of spectra, i.e. those $E \in \Sp(k)$ for which $X \times \A^1 \to X$ induces an equivalence
\[ E(X) \iso E(X \times \A^1) \]
for every $X$. 

\begin{rmk}
In the literature $\SH^{S^1}(k)$ or $\SH_{S^1}(k)$ or $\SH_{S^1}^s(k)$ for $\Sp_{\A^1}(k)$ called motivic $S^1$-spectra. Here only $S^1$ has been inverted not $\Gm$. 
\end{rmk}

\begin{rmk}
$\Sp_{\A^1}(k)^{\heart} = HI(k)$ the strongly invariant sheaves. 
\end{rmk}

Want: invert all motivic spheres not just $S^1$. 

\begin{prop}
$\P^1$ is symmetric
\end{prop}

\begin{proof}
I can identify $\P^1 \wedge \P^1 \wedge \P^1 \cong \A^3 / (\A^3 \sm 0)$ and the cycle $(1 \, 2 \, 3)$ becomes the map
\[ \A^3 / (\A^3 \sm 0) \to \A^3 / (\A^3 \sm 0) \quad (x,y,z) \mapsto (y,z,x) \]
hence given by the matrix
\begin{center}
\begin{pmatrix}
0 & 0 & 1
\\
1 & 0 & 0
\\
0 & 1 & 0
\end{pmatrix}
this is a product of elementary matrices and any elementary matrix is homotopic to the identity through invertible maps. 
\end{proof}


\begin{defn}
The \textit{stable motivic category} is 
\[ \SH(k) = \Spc(k)_* [(\P^1)^{-1}] \]
This is presentably symmetric monoidal because $\P^1$ is symmetric. 
\end{defn}

\subsection{Eilenberg-Maclane Spaces}

How to build $HA$. If we have some $A = K(A, 0)$ we need a delooping of it
\[ \Omega_{\P^1} K(\A^1, 1) = K(A, 0) \]
but this is 
\[ \Omega^{1,1} \Omega^{1,0} K(A', 1) = \Omega^{1,1} K(A', 0) = K((A')_{-1}, 0) \]
Therefore, we want $A'$ should be the ``decontraction''. Hence we need $A$ to be an infinite contraction. 

In other words, need $A_*$ to be a homotopy module. 

\begin{prop}
Any homotopy module gives rise to an eilenberg-Maclane spectrum in $\SH(k)$ and in fact
\[ \SH(k)^{\heart} \cong \HM(k) \]
\end{prop}

Now for any $A_*$ and $k,n$
\[ H^n_{\Nis}(X, A_{-n}) = [\Sigma^\infty X, \Sigma^{n+k, k} HA]_{\SH(k)} \]

\begin{defn}
For $a,b \in \ZZ$ can define homotopy groups of $E \in \SH(k)$ to be the sheafifcation of
\[ U \mapsto [\Sigma^\infty \Sigma^{a,b} U, E]_{\SH(k)} \]
\end{defn}

\subsection{Representability of $K$-Theory}

Goal: show algebraic $K$-theory is represented by $\KGL \in \SH(k)$. 


What is group completetion? Given a monoid $M$ then its group completetion $M^{\gp}$ is the initial group with a map from $M$. 

\begin{example}
$\N^{\gp} = \Z$. 
\end{example}

Given a monoid $M$ in a category $\C$, then it has some data $M \times M \to M$ and a unit $1 \to M$ and there is associativity relations. 
\par 
Let $\Fin$ be the category of finite sets, and $\Span(\Fin)$ the category with
\begin{enumerate}
\item objects: finite sets
\item morphisms are roofs $X \leftarrow Z \rightarrow Y$ maps of finite sets
\end{enumerate}
to form compositions we take fiber products. 

In a monoid $M$, can
\begin{enumerate}
\item add $x + y$
\item perform iterated addition $x + \cdots + x = n \cdot x$
\end{enumerate} 
hence can build, evaluate, and compose systems of linear mulivariate polynomials. 

We can encode these operations in spans and composition of spans. We use a set with $n$ elements to mean $M^{n}$ and use repedition along the first map and grouping along the second map to represent addition. 

\begin{defn}
If $\C$ is an $\infty$-category, the category of commutative monoids
\[ \CMon(\C) = \Fun^\times(\Span(\Fin), \C) \]
is the product-preserving functors $\Span(\Fin) \to \C$. 
\end{defn}

\begin{example}
The span $\{ x, y \} \leftarrow \{ x, y \} \to \{ f \}$ maps to the multiplication map $M^2 \to M$. 
\end{example}

\section{Nov. 21 - Monoids}

\begin{defn}
We define the full subcategory 
\[ \Ab(\C) \subset \CMon(\C) \]
of ``abelian group objects'' as those for which the distinguished span $z \leftarrow z \rightarrow z$ is an equivalence.
\end{defn}

\begin{prop}
If $\C$ is presentable then
\begin{enumerate}
\item $\CMon(\C)$ and $\Ab(\C)$ are presentable
\item the inclusion $\Ab(\C) \subset \CMon(\C)$ preserves limits and filtered colimits, and admits a right adjoint
\[ (-)^{\gp} : \CMon(\C) \to \Ab(\C) \]
\end{enumerate}
\end{prop}

\begin{example}
For $\C = \Set$ then this is classical group completion.
\end{example}

\begin{example}
If $F : \C \to \D$ preserves finite products get a diagram
\begin{center}
\begin{tikzcd}
\Ab(\C) \arrow[r] \arrow[d] & \CMon(\C) \arrow[d]
\\
\Ab(\D) \arrow[r] & \CMon(\D)
\end{tikzcd}
\end{center}
\end{example}

Main case: category $S$ of spaces. In this case a commutative monoid is a commutative group iff it is a loop space.

\begin{example}
$B \N = B \Z$.
\end{example}

Given $M \in \CMon(\C)$, a candidate for its delooping is $B M$ and we consider
\[ M \to \Omega B M \]
this is a super interesting map, we can study it at the level of homology (McDull-Segal). 

\begin{defn}
Take a collection of generators for $\pi_0 M$ and denote by $M_\inty$ the colimit of multiplying by these generators infinitely many times on $M$. More precisely,
\[ \left< I \right> = \pi_0 M \]
then for any $S \subset I$ finite we produce
\[ M_\infty = \colim_{S \subset I} \colim{(M \xrightarrow{\prod S} M \xrightarrow{\prod S} M \to \cdots)} \]
In this setting
\[ M_{\infty} = M[ (\pi_0 M)^{-1}] \]
The map
\[ M \to \Omega B M \]
has target a group and therefore we get a factorization
\[ M_\infty \to \Omega B M \]
\end{defn}

\begin{theorem}
The map $M_{\infty} \to \Omega B M$ is a plus construction.
\end{theorem}

\begin{rmk}
The plus construction
\begin{enumerate}
\item abelianizes $\pi_1$ (since if it is a group it must be abelian)
\item fixes homology to agree with the input space
\item totally messes up $\pi_\bullet$.
\end{enumerate}
\end{rmk}

\begin{example}
$(B \GL_\infty(R) \times \ZZ)^+ = K(R)$. 
\end{example}

There is a natural map
\[ B \Sigma_n \to (M^{\times n})_{h \Sigma_n} \to M \]
inducing a homomorphism
\[ \Sigma_n = \pi_1(B \Sigma_n) \to \pi_1(M) \]

\begin{theorem}
The following are equivalent:
\begin{enumerate}
\item the natural map $M_\infty \to B \Omega M$ is an equivalence
\item the cyclic permutation $(1 \, 2 \, 3)$ is in the kernel of 
\[ \Sigma_3 \to \pi_1(M) \to \pi_1(M_\infty) \]
\end{enumerate}
\end{theorem}

\begin{defn}
For a ring $R$, let $\Vect(R)$ be the groupoid of finitely generated projective $R$-modules. Then \textit{algebraic K-theory} is the group completion in $S$ 
\[ K(R) = \Vect(R)^{\gp} \]
\end{defn}

Note: $\Vect(-)$ as a functor can be extended to
\[ \Vect(-) : \Sm^{\op}_k \to \Gpd \embed S \]
is a fppf sheaf.

Recall: finitely generated modules of rank $r$ are classified by maps into $\BGL_r$ in the sheaf topos. There is an equivalence of categories of groupoids 
\[ \Vect(0) \to \sqcup_{r \ge 0} \BGL_r \]
in $\Shv_{\Nis}(\Sm_k)$. 
Idea: do the group completion operation as above
\begin{center}
\begin{tikzcd}
\Vect \arrow[rr] \arrow[ld] & & \Vect^{\gp} = K
\\
& \Vect_\infty \arrow[ru]
\end{tikzcd}
\end{center}
Notation: 
\[ \BGL = \colim_{r \to \infty} \BGL_r \]
along the stabilization maps $E \mapsto E \oplus \struct{}$. Each $\BGL_r$ is connected so
\[ \pi_0 \Vect = \po_0 \sqcup_{r \ge 0} \BGL_r = \N \]
The $+1$ map is given by the stabilization maps inducing ``shifts''
\[ \sqcup_{r \ge 0} \BGL_r \to \sqcup_{r \ge 1} \BGL_r \subset \sqcup_{r \ge 0} \BGL_r \]

\begin{prop}
$\Vect_\infty = \BGL \times \Z$
\end{prop}

\begin{proof}
Can pull disjoint union out of the colimit of shift maps
\[ \colim{\left( \sqcup_{r \ge 0} \BGL_r \to \sqcup_{r \ge 0} \BGL_r \to \cdots \right)} = \sqcup_{n \in \ZZ} \colim{(\BGL_n \to \BGL_n \to \cdots)} = \sqcup_{n \in \Z} \BGL = \BGL \times \Z \]
\end{proof}

Since $K = \Vect^{\gp}$ the factorization
\[ \Vect \to \Vect_{\infty} \to \Vect^{\gp} \]
gives
\[ \Vect \to \BGL \times \Z \to K \]

\begin{theorem}
$\BGL \times \Z \to K$ is a motivic equivalence. 
\end{theorem}

\begin{proof}
Since $L_{\mot}$ prserves finite product it also preserves commutative monoids and abelian group objects. Also $L_{\mot}$ is a left adjoint so it commutes with limits and therefore it commues with $(-)_{\infty}$ and as a left adjoint preserving monoids and groups so it commutes with $(-)^{\gp}$. We're trying to show that
\[ L_{\mot} (\Vect_\infty \to \Vect^{\gp}) \]
is an equivalence. This is the same as showing that
\[ (L_{\mot} \Vect)_{\infty} \to (L_{\mot} \Vect)^{\gp} \]
is an equivalence. We apply the theorem to $L_{\mot} \Vect$. This means we need to show that the permutation $(1 \, 2 \, 3)$ on the bundle $\struct{}^{\oplus 3}$ is homotopic to the identity. This is true because the associated matrix is a product of elementary matrices. 
\end{proof}

\subsection{Algebraic $K$-theory is a Nisnevich sheaf}

\begin{theorem}[Thomason-Trobaugh]
Algebraic $K$-theory i a Nisnevich sheaf of spectra. 
\end{theorem}

Sketch: given a Nisnevich distinguished square
\begin{center}
\begin{tikzcd}
W \arrow[r] \arrow[d] \pullback & V \arrow[d, "p"]
\\
U \arrow[r] & X
\end{tikzcd}
\end{center}
such that $p$ is \etale and $p^{-1}(X \sm U) \iso X \sm U$ is an isomorphism. We want to show that
\begin{center}
\begin{tikzcd}
K(W) \arrow[r] \arrow[d] \pullback & K(V)
\\
K(U) \arrow[r] & K(X) 
\end{tikzcd}
\end{center}
is a pullback square of spectra. Taking the category of perfect complexes
\[ \Perf :  \Sm_k^{\op} \to \Cat^{\text{st}}_\infty \]
is an \fppf sheaf. Then there is a diagram
\begin{center}
\begin{tikzcd}
\Perf_Z(V) \arrow[d, equals] \arrow[r] & \Perf(W) \pullback \arrow[r] \arrow[d] & \Perf(V) \arrow[d]
\\
\Perf_Z(X) \arrow[r] & \Perf(U) \arrow[r] & \Perf(X)
\end{tikzcd}
\end{center}
where the kernels are complexes ``supported on $Z$'' and the equivalence comes from the square being a pullback. In order to show $K$-theory is a Nisnevich sheaf, we have to argue it ``preserves fiber sequences''
\[ K : \Cat_\infty^{\st} \to \Sp \]
this follows from $K$ being a \textit{localizing invariant}.

\subsection{Algebraic $K$-theory is $\A^1$-invariant}

Fundamental theorem of algebraic $K$-theory
\begin{theorem}[Quillen]
If $R$ is a regular Noetherian ring, then
\[ K(R) \to K(R[t]) \]
is an equivalence.
\end{theorem}

Sketch: $G$-theory ($\Mod^{fg}(-)^{\gp}$) is $\A^1$-invariant on Noetherian rings, and exploit a devisage argument and commutative algebra to show $K(R) = G(R)$ for $R$ regular noetherian. 

\begin{theorem}
If $X$ is a regular noetherian scheme then
\[ K(X) \to K(X \times \A^1) \]
is an equivalence.
\end{theorem}

\begin{cor}
$K : \Sm_k^{\op} \to \Sp$ is $\A^1$-invariant. 
\end{cor}

\subsection{Projective Bundle Formula}

$K(\P^n_R) = K(R)[x]/(x^{n+1})$ can use this to get $\P^1$-bonding maps of $\BGL \times \Z$ to itself to get a $\P^1$-spectrum $\KGL \in \SH(k)$



\end{document}
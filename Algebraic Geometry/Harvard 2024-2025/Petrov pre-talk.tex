\documentclass[12pt]{article}
\usepackage{import}
\import{../}{AlgGeoCommands}

\newcommand{\dbar}{\bar{\partial}}
\newcommand{\HH}{\mathbb{H}}
\renewcommand{\gr}{\mathrm{gr}}
\newcommand{\R}{\mathrm{R}}

\renewcommand{\H}{\mathcal{H}}
\newcommand{\LL}{\mathbb{L}}
\newcommand{\can}{\mathrm{can}}

\begin{document}
\author{Benjamin Church}
\title{\Huge Pre-Talk}

\section{Introduction}

\begin{thm}[Hodge-to-de Rham-degeneration]
Let $X / k$ be a smooth proper scheme with $k$ a field of characteristic zero and $\Omega^\bullet_{X/k}$ is de Rham complex. Then, the Hodge-to-de Rham spectral sequence,
\[ E_1^{p,q} = H^{q}(X, \Omega^p_{X/k}) \implies H_{\dR}^{p+q}(X) \]
degenerates at the $E_1$-page.
\end{thm}

\begin{rmk}
This is equivalent to the numerical equality
\[ \dim{H^n_{\dR}(X)} = \sum_{p + q = n} \dim{H^q(X, \Omega^p_{X/k})} \]
\end{rmk}

\begin{theorem}[Kodaira-Nakano-vanishing]
Let $X / k$ be a smooth proper scheme with $k$ a field of characteristic zero and $\L$ an ample. Then $H^q(X, \Omega_X^p \ot \L) = 0$ for $p + q > \dim{X}$. 
\end{theorem}


\begin{theorem}[Deligne-Illusie]
Let $X / k$ be a smooth proper scheme of pure dimension $n$. Let $k$ be a perfect field of characteristic $p$. Suppose
\begin{enumerate}
\item $p > n$
\item $X$ lifts to $W_2(k)$
\end{enumerate}
then the Hodge-to-de Rham spectral sequence degenerates at $E_1$ and Kodaira-Nakano vanishing holds for any ample $\L$.
\end{theorem}


\begin{cor}
The theorems also hold over $k$ characteristic zero. 
\end{cor}

\begin{proof}
Both are completely numerical statements. For $X / k$ in characteristic zero, we can spread out to a finite type $\ZZ$-algebra $A \subset k$ and smooth proper morphism $f : \X \to S = \Spec{A}$ with a relatively ample $\L$ so that the the dimensions of all the relevant cohomology groups are constant. Hence we just need to prove degeneration and Nakano vanishing for some fiber. Shrinking, we may assume $S \to \Spec{\Z}$ is smooth. By Chevallay, there is a prime $p > \dim{X}$ in the image of $S \to \Spec{\Z}$ so choose $s \mapsto p$ and by smoothness there is a map $\Spec{W_2(\kappa(s))} \to S$ hence $\X_s$ satisfies the hypothesis and we apply Deligne-Illusie's results to win.
\end{proof}



\subsection{Recall: the Frobenius}


\newcommand{\Fr}{\mathrm{Fr}}

\begin{defn}
Let $X$ be a scheme of characteristic $p$ (meaning $p \struct{X} = 0$). Then there is a natural map $\Fr : X \to X$ via $\id$ on topological spaces and $\struct{X} \to \struct{X}$ via $x \mapsto x^p$. This is natural, in the sense that for any map $f : X \to Y$ there is a commutative diagram,
\begin{center}
\begin{tikzcd}
X \arrow[r, "\Fr_X"] \arrow[d, "f"] & X \arrow[d, "f"]
\\
Y \arrow[r, "\Fr_Y"] & Y
\end{tikzcd}
\end{center}
Therefore, we can define via pullbacks,
\begin{center}
\begin{tikzcd}
X \arrow[rd, "f"'] \arrow[r, "F_{X/Y}", dashed] & X^{(p)} \arrow[d] \arrow[r] & X \arrow[d, "f"]
\\
& Y \arrow[r, "\Fr_Y"] & Y
\end{tikzcd}
\end{center}
giving the relative Frobenius $F_{X/Y} : X \to X^{(p)}$. 
\end{defn}

\begin{prop}
If $Y$ has characteristic $p$ and $f : X \to Y$ is smooth of relative dimension $n$ then $F_{X/Y} : X \to X^{(p)}$ is finite and flat of degree $n$. Therefore, $F_* \struct{X}$ is locally free of rank $n$ as a $\struct{X^{(p)}}$-module.
\end{prop}


\subsection{The Incredible Trick: Cartier Isomorphisms}

Let $F = F_{X/k}$ for $X$ smooth over a perfect field $k$ of characteristic $p$. 
\par 
The following is a crucial remark. The differentials $(\Omega_X^\bullet, \d)$ are nonlinear so it does not form an element of $D^b(X)$. However, $F_* \Omega_X^\bullet \in D^b(X^{(p)})$ because the differentials are $\struct{X^{(p)}}$-linear! This is because 
\[ \d{s^p} = p \, s^{p-1} \d{s} = 0 \]

The incredible observation is that under our hypotheses $F_* \Omega_X^\bullet$ decomposes into its cohomology in the derived category.

\begin{theorem}
If $X$ lifts to $W_2(k)$ then there is a quasi-isomorphism in $D^b(X^{(p)})$ 
\[ \varphi : \tau^{<p} F_* \Omega_X^\bullet \iso \bigoplus_{i < p} \Omega_{X^{(p)}}^i [-i] \]
In particular, if $p > \dim{X}$ then $\tau^{<p} F_* \Omega_X^\bullet = F_* \Omega_X^\bullet$ decomposes.
\end{theorem}

Let's assume this and prove the main theorem.

\subsection{Hodge-to-de Rham degeneration}

Let $X / k$ be satisfying the hypotheses so $\varphi$ exists and $p < \dim{X}$. Then
\[ \HH^n(X, \Omega_X^\bullet) = \HH^n(X^{(p)}, F_* \Omega_X^\bullet) \iso \bigoplus_i H^{n-i}(X^{(p)}, \Omega^i_{X^{(p)}}) = \bigoplus_{i} \Fr_k^* H^{n-i}(X, \Omega_X^i)  \]
The first map is because $F$ is a homeomorphism\footnote{Since it is not a complex of $\struct{X}$-modules, because the maps are nonlinear, affine is not enough. However, the Leray spectral sequence is completely general so quasi-finite is enough because then the higher derived pushforwards vanish when the fibers are zero dimensional.}, the second is $\varphi$ on cohomology the last one uses flat base change for the pullback diagram
\begin{center}
\begin{tikzcd}
X^{(p)} \arrow[r] \arrow[d] & X \arrow[d]
\\
\Spec{k} \arrow[r, "\Fr_k"] & \Spec{k}
\end{tikzcd}
\end{center}
Since $F_k^*$ does not change the dimension of a vector space (only the $k$-action) we conclude using that $E_1$-degeneration is equivalent to the numerical equality given by taking dimensions above.

\subsection{Kodaira Vanishing}

We will use $\varphi$ to prove the following inductive step. 
\begin{defn}
We say that $\M \in \Pic{X}$ satisfies (NV) if 
\[ H^q(X, \Omega_X^p \ot \M) = 0 \quad \text{ for all } p + q > \dim{X} \]
\end{defn}
We will prove
\[ (\ast) \quad \quad \M^{\ot p} \text{ satisfies (NV)} \implies \M \text{ satisfies (NV)} \]
Why does $(\ast)$ suffice. By downward induction, we just need to show that if $\L \in \Pic{X}$ is ample then $\L^{\ot p^k}$ satisfies (NV) for $k \gg 0$. But this is clear: large enough powers of $\L$ kill \textit{all} higher cohomology of anything by Serre vanishing. 


\begin{lemma}
For any invertible module $\M$,
\[ F_X^* \M \iso \M^{\otimes p} \]
\end{lemma}

\begin{proof}
The map is defined by adjunction of $\M \to (F_X)_* \M^{\otimes p}$ via $m \mapsto m^{\otimes p}$ which is linear because,
\[ am \mapsto (am)^p = a^p m^p = a \cdot m^p \]
We check $F_X^* \M \iso \M^{\ot p}$ locally.
\end{proof}

\begin{cor}
Let $\M'$ be the pullback of $\M$ under $X^{(p)} \to X$. Then $F^* \M' = \M^{\otimes p}$. 
\end{cor}

\begin{rmk}
The point of this is that the $p$-th power of any line bundle is pulled back from a line bundle on $X^{(p)}$.
\end{rmk}

\begin{proof}[Proof of induction]
Assume (NV) holds for $\M^{\ot p}$. By the projection formula,
\[ F_*(\M^{\otimes p} \otimes \Omega^i_X) \cong F_*(F^* \M' \otimes \Omega^i) \cong \M' \otimes F_* \Omega_X^i \] 
Consider the hypercohomology spectral sequence computing the cohomology of $\M' \ot F_* \Omega_X^\bullet$,
\[ E^{i,j}_1 = H^j(X^{(p)}, \M' \otimes F_* \Omega^i_X) \implies \mathbb{H}^{i+j}(X^{(p)}, \M' \otimes F_* \Omega^\bullet_X) \]
However,
\[ H^j(X^{(p)}, \M' \otimes F_* \Omega^i_X) = H^j(X^{(p)}, F_*(\M^{\ot p} \ot \Omega_X^i)) = H^j(X, \M^{\ot p} \otimes \Omega^i_X) = 0 \]
for $i + j > \dim{X}$ by the induction hypothesis. Therefore, we conclude from the spectral sequence,
\[ \mathbb{H}^{k}(X^{(p)}, \M' \otimes F_* \Omega^\bullet_X) = 0 \]
for $k > \dim{X}$. Now we use the decomposition
\[ \M' \ot F_* \Omega^\bullet_X \iso \bigoplus_{i} \M' \otimes \Omega^i_X [-i] \]
so the hypercohomology is given by,
\[ \mathbb{H}^k(X^{(p)}, \M' \ot F_* \Omega_X^\bullet) = \bigoplus_{i + j = k} H^j(X^{(p)}, \M' \ot \Omega^i_{X^{(p)}}) = \bigoplus_{i + j = k} \Fr_k^* H^j(X, \M \ot \Omega^i_X) \]
and thus by vanihsing of the hypercohomology for $n > \dim{X}$ we get vanishing,
\[ H^j(X, \M \ot \Omega^i_{X}) = 0 \]
for $i + j > \dim{X}$ proving (NV) for $\M$ thus completing the induction.
\end{proof}


\subsection{The Cartier Operator}

We need to construct $\varphi$. The first step is to understand the Cartier operator. There is a graded isomorphism,
\[ C^{-1} : \bigoplus_{i} \Omega^i_{X^{(p)}} \iso \bigoplus_i \mathcal{H}^i(F_* \Omega^\bullet_{X}) \]
such that,
\begin{enumerate}
\item in $i = 0$ the map $\struct{X^{(p)}} \to F_* \struct{X}$ is exactly $F^{\#}$
\item in $i = 1$,
\[ C^{-1}(1 \otimes \d{s}) = s^{p-1} \d{s} \in \mathcal{H}^1(F_* \Omega^\bullet_{X}) \]
think of this as ``$\frac{F^*(\d{s})}{p}$''.  
\end{enumerate}
\noindent
To prove the theorem, we will exhibit a quasi-isomorphism
\[ \varphi : \bigoplus_{i < p} \Omega^i_{X^{(p)}}[-i] \to F_* \Omega^\bullet_{X} \]
that induces $C^{-1}$ on cohomology for $i < p$ (and thus is a quasi-isomorphism to the trucation). We want to reduce to constructing $\varphi^1$ where $\varphi^i$ are the components of the map from the direct sum. For $\varphi^0$ we just define,
\[ \varphi^0 : \struct{X^{(p)}} \xrightarrow{C^{-1}} F_* \struct{X} = \mathcal{H}^0(F_* \Omega^\bullet_{X}) \embed F_* \Omega^\bullet_{X} \]
Now assume we have constructed,
\[ \varphi^1 : \Omega^1_{X^{(p)}} [-1] \to \wt{F}_* \Omega^\bullet_{X} \]
inducing $C^{-1}$ on $\mathcal{H}^1$.
Then there exists,
\[ \left( \Omega^1_{X^{(p)}} \right)^{\otimes i} \to \Omega^i_{X^{(p)}} \]
by sending,
\[ w_1 \otimes \cdots \otimes w_i \mapsto w_1 \wedge \cdots \wedge w_i \]
If $i < p$ (or in characteristic zero) then there exists a section to this map,
\[ a(w_1 \wedge \cdots \wedge w_i)  = \frac{1}{i!} \sum_{\sigma \in S_i} \mathrm{sign}(i) w_{\sigma(1)} \otimes \cdots \otimes w_{\sigma(i)} \]
Therefore we get,
\begin{center}
\begin{tikzcd}
(\Omega^1_{X^{(p)}})^{\otimes i} \arrow[r, "\varphi_1^{\otimes i}"] & \left( F_* \Omega_{X}^\bullet \right)^{\otimes^{\LL} i} \arrow[d]
\\
\Omega^i_{X^{(p)}} \arrow[u] \arrow[r, dashed, "\varphi^i"] & F_* \Omega^\bullet_{X}
\end{tikzcd}
\end{center}
Because this construction agrees with the product structure and the Cartier isomorphism is determined (using the product structure) by its values in degree $1$ this means that $\varphi^i$ must induce $C^{-1}$ in degree $i$.


\subsection{Construction of $\varphi^1$}

First we consider the case when $F$ admits a global lift over $W_2(k)$. This means there is a diagram,
\begin{center}
\begin{tikzcd}
X \arrow[d, "F"'] \pullback \arrow[r] & \wt{X} \arrow[d, "\wt{F}"] 
\\
X^{(p)} \arrow[r] & \wt{X}^{(p)}
\end{tikzcd}
\end{center}
where $\wt{X} \to \Spec{W_2(k)}$ and $\wt{X^{(p)}} \to \Spec{W_2(k)}$ are smooth lifts of $X$ and $X^{(p)}$ over $W_2(k)$. 
\bigskip\\
Now to perform the construction notice that,
\[ \im{(\wt{F}^* : \Omega^1_{\wt{X^{(p)}}/\wt{S}} \to \wt{F}_* \Omega^1_{\wt{X}/\wt{S}})} \subset p \cdot \wt{F}_* \Omega^1_{\wt{X}/\wt{S}} \]
because pulling back differentials by Frobenius introduces a factor of $p$. Therefore, we get a diagram,
\begin{center}
\begin{tikzcd}
\Omega^1_{\wt{X^{(p)}}/\wt{S}} \arrow[d, two heads] \arrow[r, "\wt{F}"] & p \cdot \wt{F}_* \Omega^1_{\wt{X}/\wt{S}} 
\\
\Omega^1_{X^{(p)}} \arrow[r, dashed, "\varphi^1"] & F_* \Omega^1_{X} \arrow[u, "p \cdot (-)"]
\end{tikzcd}
\end{center}
which exists because the right upward map is an isomorphism and the kernel of the left downward map is the multiples of $p$ which are sent to zero. 
I claim that
\[ \im{\varphi^1} \subset Z^1(F_* \Omega^\bullet_{X}) \]
and $\varphi^1$ induces $C^{-1}$ in degree $1$. For local section $a'\in \Gamma(U^{(p)}, \struct{\wt{X^{(p)}}})$ pulled back from $a \in \Gamma(U, \struct{X})$, the differential $\d{a}$ is acted on via
\[ \wt{F}^*(\d{a'}) = \d{\,\wt{F}^\# a'} = p a^{p-1} \d{a} + p \, \d{b} \]
where $\wt{F}^\# a' = a^p + p \, b$ where $p \, b$ is the error term. Hence
\[ \varphi^1(\d{a'}) = a^{p-1} \d{a} + \d{b} \]
which is clearly an exact form (lies in $Z^1$). But notice that the second term is exact and therefore dies in the quotient
\[ Z^1(F_* \Omega^\bullet_{X}) \to \cH^1(F_* \Omega^\bullet_{X}) \]
so the induced map is exactly given by the Cartier isomorphism in degree $1$.

\subsection{What about if $F$ doesn't lift?}

From smoothness, we know that lifts exist locally. We need to compare the outputs of different lifts. 

\begin{lemma}
Given flat lifts $\wt{X}_i$ of $X$ and $G_i : \wt{X} \to \wt{X^{(p)}}$ of $F$ over $\wt{S}$ there is a canonical element,
\[ h(G_1, G_2) : \Omega^1_{X^{(p)}} \to F_* \struct{X} \]
such that,
\[ \varphi^1_{G_2} - \varphi^1_{G_1} = \d{h(G_1, G_2)} \] 
and if $G_3 : \wt{X}_3 \to \wt{X^{(p)}}$ is a third lifting then
\[ h(G_1, G_2) + h(G_2, G_3) = h(G_1, G_3) \]
\end{lemma}

\begin{proof}
Choose an isomorphism $u : \wt{X}_1 \iso \wt{X}_2$ of lifts (which may only exist locally) then 
\[ u^* G_2 - G_1 : \struct{X^{(p)}} \to F_* \struct{X} \]
is a derivation which does not depend on the choice of isomorphism $u$. Indeed, given $u' : \wt{X}_1 \iso \wt{X}_2$ the difference is a derivation or equivalently a map
\[ \delta : \Omega_{X}^1 \to \struct{X} \]
Then $u^* G_2$ and $u'^* G_2$ differ by the composition of $\delta$ with the pullback $F^* \Omega^1_{X^{(p)}} \to \Omega^1_{X}$ which is zero. Hence $u^* G_2 = u'^* G_2$. Therefore, working locally on $X$ so that an isomorphism $u$ exists, we get a well-defined derivation 
\[ h(G_1, G_2) : \Omega^1_{X^{(p)}} \to \wt{F}_* \struct{X} \]
via the difference above. Then 
\[ \varphi_{G_2}^1 - \varphi_{G_1}^1 = \d{h(G_1, G_2)} \]
from the formula for $\varphi^1$ since $G_2^{\#}(a') - G_1^{\#}(a') = b_2 - b_1$ in $F_* \struct{X} = p \cdot \wt{F}_* \struct{\wt{X}}$ then
\[ \varphi^1_{G_2}(a') - \varphi^1_{G_1}(a') = \d{(b_2 - b_1)} \]
\end{proof}

This is exactly enough data to modify the local lifts in such a way that the $\varphi_{G}^1$ glue to
\[ \varphi^1 : \Omega^1_{X^{(p)}} \to Z^1(F_* \Omega_X^\bullet) \]
using the Cech description.


\section{Talk}

\newcommand{\Zar}{\mathrm{Zar}}

Let $X / \CC$ be smooth and proper. Then 
\[ H^n(X/\CC, \CC) = \bigoplus_{i + j = n} H^j(X, \Omega_X^i) \]
by the Hodge decomposition. This is because by Grothendieck
\[ H^n(X, \CC) = H^n_{\dR}(X/\CC) = \HH^n_{\Zar}(X, \Omega_X^\bullet) \]
we can interpret the Hodge decomposition by saying that we can replace the differentials by $0$ and the cohomology does not change. 
\par 
We also have Kodaira-Akizuki-Nakano vanishing: if $\L$ is ample then
\[ H^i(X, \L \ot \Omega_X^j) = 0 \]
for all $i + j > \dim{X}$. 

\subsection{The positive characteristic scenario}

Let $p$ be a prime and $X / \FF_p$ be a smooth projective variety. Is there a natural decomposition?
\[ H^n_{\dR}(X / \FF_p ) = \bigoplus_{i + j = n} H^j(X, \Omega^i_X) \]
but we cannot hope to have a natural such decomposition. Indeed, if $F : X \to X$ is the Frobenius
\[ f \in \struct{X} \quad F^*(f) = f^p \]

\begin{example}
Let $E / \FF_p$ be a supersingular elliptic curve then $F^* = 0$ on $H^1(E, \struct{E})$. Then
\[ F^* \acts H^1_{\dR}(E/\FF_p) = \FF_p^{\oplus 2} \]
acts via
\[ F^* = 
\begin{pmatrix}
0 & 1
\\
0 & 0
\end{pmatrix} \]
and hence cannot respect any decomposition.
\end{example}

\begin{example}
Mumford gave an example of a surface such that 
\[ \d : H^0(X, \Omega_X^1) \to H^0(X, \Omega^2_X) \]
is nonzero and thus
\[ \dim{H^1_{\dR}} < \dim{H^0(X, \Omega_X^1)} + \dim{H^1(X, \struct{})} \]
\end{example}

\begin{theorem}[Deligne-Illusie]
Let $X / \FF_p$ be a smooth variety endowed with a lift $\wt{X}$ over $\Z / p^2$ then 
\[ H^n_{\dR}(X) = \bigoplus_{i + j = n} H^j(X, \Omega^i) \]
for $n < p$ which is natural in $\wt{X}$ (so not completely canonical only functorial for morphisms endowed with a lift). 
\end{theorem}

This fixes the issues. In the first example, $F$ cannot be lifted, in the second the variety does not lift at all. 

\begin{theorem}
If $X$ is smooth projective and $\dim{X} \le p$ then
\[ H^i(X, \Omega^j \ot \L) = 0 \]
for $i + j > \dim{X}$ if $X$ admits a lift.
\end{theorem}

\begin{theorem}[P]
There exist smooth projective $X / \FF_p$ liftable to $\ZZ_p$ such that
\[ \dim{H_{\dR}^p(X / \FF_p)} < \dim \bigoplus_{i + j = p} H^j(X, \Omega^i_X) \]
\end{theorem}

So the first case possible fails. 

\subsection{Method of Deligne-Illusie}

For $X / \FF_p$ smooth the complex
\[ F_* \Omega_X^\bullet := [F_* \struct{X} \xrightarrow{F_* \d} F_* \Omega^1_X \xrightarrow{F_* \d} F_* \Omega_X^2 \to \cdots]  \]
is a \textit{linear} complex of $\struct{X}$-modules. Indeed,
\[ \d{f^p \omega} = p f^{p-1} \d{f} \wedge \omega + f^p \omega = f^p \omega \]
But note that 
\[ \R \Gamma_{\dR}(X) = \R \Gamma_{\Zar}(X, F_* \Omega_X^\bullet) \]
Cartier isomorphism
\[ \cH^i(F_* \Omega_X^\bullet) \iso \Omega^i_X \]
The key question is whether this lifts to the derived category? This means is $F_* \Omega_X^\bullet$ quasi-isomorphic to a complex with zero differentials? Question: is there a quasi-isomorphism
\[ F_* \Omega^\bullet_X =\iso \bigoplus_{i \ge 0} \Omega^i_X[-i] \]
If yes, we get a decomposition
\[ H^n_{\dR}(X) \cong \bigoplus_{i + j = n} H^j(X, \Omega^i_X) \]
for all $n$ and we get Kodaira-Akizuki-Nakano vanishing. 

Consider
\[ \tau^{\le 1} F_* \Omega_X \iso [\struct{X} \xrightarrow{0} \Omega_X^1] \]
exists iff $X$ lifts to $\Z / p^2$. Moreover if you are precise about the equivalence of quasi-isomorphisms they are in bijection with isomorphism classes of lifts. It turns out that such a quasi-isomorphism implies 
\[ \tau^{< p} F_* \Omega_X^\bullet \iso \bigoplus_{i = 0}^{p-1} \Omega^i_X [-i] \]
We want to figure out some classes of varieties for which this quasi-isomorphism exists.

\subsection{Frobenius Splitting}

\begin{defn}[Mehta-Ramanathan]
Let $X / \FF_p$ be any scheme. We say it is \textit{Frobenius split} if there exists
\[ \sigma : F_* \struct{X} \to \struct{X} \]
such that the composition
\[ \struct{X} \to F_* \strut{X} \xrightarrow{\sigma} \struct{X} \]
is the identity.
\end{defn}

Note that $F_* \struct{X}$ is, for $X$ smooth, a vector bundle of rank $p^{\dim{X}}$.

\begin{rmk}
In characteristic zero, if $f : X \to Y$ is a finite flat morphism then $\struct{Y} \to f_* \struct{X}$ always has a splitting. Indeed, if $\deg{f}$ is invertible on $Y$ then there is a splitting by taking the trace map which gives multiplication by $p$. Thus failure of splitting is a strictly characteristic zero phenomenon. 
\end{rmk}

\begin{prop}
If $X$ is any projective variety over $\FF_p$ and is $F$-split then if $\L$ is an ample line bundle then $H^i(X, \L) = 0$ for $i > 0$. 
\end{prop}

\begin{proof}
For any line bundle $\L$ we see that if $H^i(X, \L^{\ot p}) = 0$ then $H^i(X, \L^{\ot p}) = 0$. Indeed,
\[ H^i(X, \L^{\ot p}) = H^i(X, F^* \L) = H^i(X, F_* F^* \L) = H^i(X, \L \ot F_* \struct{X}) \]
there is always a map $\L \to F_* F^* \L$ but we have a splitting so 
\[ H^i(X, \L) \to H^i(X, \L \ot F_* \struct{X}) \]
is a direct summand. Hence $H^i(X, \L^{\ot p}) = 0$ implies $H^i(X, \L) = 0$. Since $\L$ is ample, some power will kill higher cohomology so we win by downward induction. 
\end{proof}

Analogously we can prove
\[ H^i(X, \L \ot \omega_X) = 0 \]
for $i > 0$ if $F$-split and smooth projective.

\begin{example}
The following are $F$-split
\begin{enumerate}
\item ordinary elliptic curves
\item flag varities $G / P$ for $P \subset G$ parabolic in a reductive group
\item toric varieties?
\item $B G$ for any reductive group $G$ 
\end{enumerate}
non $F$-split varieties: 
\begin{enumerate}
\item curve of genus $g > 1$
\item most varieties
\end{enumerate}
\end{example}

\begin{theorem}[P]
If $X$ is smooth and $F$-split then 
\[ F_* \Omega_X^\bullet = \bigoplus_{i \ge 0} \Omega_X^i[-i] \]
so there is a decomposition in all degrees. Moreover, if $X$ is smooth projective then \[ H^i(X, \L \ot \Omega_X^j) = 0 \]
for $i + j > \dim{X}$. 
\end{theorem}

\begin{prop}[Vologodsky, Bhatt]
If $X / \FF_p$ is smooth then
\[ F^* F_* \Omega_X^\bullet \cong \bigoplus_{i \ge 0} F^* \Omega_X^\bullet [-i] \]
Any possible extension thus die under $F^*$.
\end{prop}

Then the theorem follows quickly. We get
\[ F_* \struct{X} \ot_{\struct{X}} F_* \Omega_X^\bullet \cong \bigoplus_{i \ge 0} F_* \struct{X} \ot \Omega_X^i[-i] \]
there is always a map
\[ F_* \Omega_X^\bullet \to F_* \struct{X} \ot_{\struct{X}} F_* \Omega_X^\bullet \]
but when you are $F$-split there is a section
\[ F_* \struct{X} \ot_{\struct{X}} F_* \Omega_X^\bullet \to F_* \Omega_X^\bullet \]
so $F_* \Omega_X^\bullet$ is a direct summand of a decomposition complex. 

\subsection{The de Rham stack (Simpson, Drinfeld, Bhatt-Lurie, Ogus-Vologodsky}

The proof of these claims uses the de Rham stack. Let $X / \FF_p$ be a smooth variety. There is a stack $X^{\dR} / \FF_p$ which is a \etale stack in groupoids such that
\begin{enumerate}
\item $\R \Gamma(X^{\dR}, \struct{X^{\dR}}) \cong \R \Gamma_{\dR}(X/\FF_p)$
\item $\QCoh(X^{\dR}) \cong D_X$-modules with locally nilpotentent $p$-curvature. These are pairs $(\E, \nabla)$ where $\E$ is a vector bundle with $\nabla$ a flat connection with locally nilpotent $p$-curvature. 
\end{enumerate}

There is a map $s : X \to X^{\dR}$ such that $s^* (\E, \nabla) = \E$. In characteristic $p$ there is a map $\pi : X^{\dR} \to X$ such that $\pi^* \E = (F^* \E, \nabla^{\text{can}})$ where $\nabla^{\text{can}}(f \ot s) = \d{f} \ot s$ which is well-defined because $p$-th powers have zero differential. The composition $\pi \circ s = F$ because $s^* \pi^* \E = F^* \E$. 

\begin{prop}
$F_* \Omega_X^\bullet = \R \pi_* \struct{X^{\dR}}$
\end{prop} 

Key property: $\pi : X^{\dR} \to X$ is a gerbe for a group scheme on $X$. 

\begin{lemma}[Ogus-Vologodsky]
Assume $X$ has a lift $\wt{X}$ over $\Z / p^2$ equipped with a lift of Frobenius $\wt{F} : \wt{X} \to \wt{X}$. Then the category of quasi-coherent sheaves on $X$ with flat locally nilpotent $p$-curvature connections is equivalent to the category of nilpotent Higgs sheaves $(\E, \theta)$ on $X$. 
\end{lemma}

In particular, when we have $F$-lifts, $X^{\dR} \cong B_X T_X^{\#}$ for a certain group scheme $T_X^{\#}$. This is induced by the above equivalence of categories.

\begin{proof}
Let $\theta : \E \to \E \ot \Omega_X^1$ be a Higgs sheaf. We want to produce $(\E, \nabla)$. The lift $\wt{F}$ produces
\[ F^* \Omega^1_X \xrightarrow{\frac{1}{p} \d{wt{F}}} \Omega_X^1 \]
given by
\[ \omega \ot 1 \mapsto \frac{\d{\wt{F}(\wt{\omega})}}{p} \]
for some lift $\wt{\omega}$ of $\omega$. 
Then I can consider $(F^* \E, \nabla^{\can} + \\frac{\d{\wt{F}}}{p} \cdot F^* \theta)$
where we need this curious map or else $F^* \theta$ has target in $F^* \Omega_X^1$. This map of categories works in general, it is an equivalence if we impose nilpotency conditions.
\end{proof}

\begin{cor}
Given a lift 
\begin{center}
\begin{tikzcd}
Y \arrow[rd] \arrow[r, "\wt{f}"] & X^{\dR} \arrow[d, "\pi"] 
\\
& X
\end{tikzcd}
\end{center}
then there is a quasi-isomorphism
\[ \wt{f}^* F_* \Omega_X^\bullet \bigoplus_{i \ge 0} \Omega_X^i[-i] \]
\end{cor}

In particular, we can apply this to Frobenius given a lifting. 
\end{document}

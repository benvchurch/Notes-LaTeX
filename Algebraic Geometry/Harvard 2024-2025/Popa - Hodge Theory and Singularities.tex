\documentclass[12pt]{article}
\usepackage{import}
\import{../}{AlgGeoCommands}

\newcommand{\dbar}{\bar{\partial}}
\newcommand{\HH}{\mathbb{H}}
\renewcommand{\gr}{\mathrm{gr}}
\newcommand{\R}{\mathrm{R}}


\begin{document}

 
\section{Sep. 04 - Review of Hodge Theory}

Varities are over $\CC$.


\begin{theorem}[Hodge Decomposition]
$X$ compact Kahler manifold then the singular cohomology groups have a decomposition
\[ H^i(X, \CC) = \bigoplus_{p + q = i} H^{p,q} = \bigoplus_{p + q = i} H^{p,q}(X) = \bigoplus_{p + q = i} \cH^{p,q} \]
where $H^{p,q}(X)$ is the cohomology of $(p,q)$-forms with respect to $\dbar$ and $\cH^{p,q}$ are the $(p,q)$-Harmonic forms with respect to some choice of Kahler metric. This splitting does not depend on the choice of Kahler metric. We can say $H^{p,q}$ is the space of $\d$-closed forms containiing a class of a $(p,q)$-form. Such that,
\begin{enumerate}
\item $H^{p,q}(X) = \ol{H^{q,p}(X)}$ using the structure $H^i(X, \CC) = H^i(X, \RR)$
\item Dolbeaut Theorem: $H^{p,q}(X) \cong H^q(X, \Omega^p_X)$ 
\end{enumerate} 
\end{theorem} 

Consequences:
\begin{enumerate}
\item Betti numbers: $b_i(X) := \dim_{\CC} H^i(X, \CC)$ is even for $i$ odd
\item Hodge numbers: $b_i(X) = \sum_{p+q = i} h^{p,q}(X)$ where $h^{p,q}(X) := \dim_{\CC} H^{p,q}(X)$
\end{enumerate}

Serre duality: if $E$ is a vector bundle then $H^q(X, E) \cong H^{n-q}(X, E^\vee \ot \omega_X)^\vee$. In particular, for $\Omega_X^p$ we get
\[ H^q(X, \Omega_X^p) \cong H^{n-q}(X, \Omega_X^{n-p})^\vee \]
and hence $h^{p,q}(X) = h^{n-p,n-q}(X)$. 
\bigskip\\
Moreover, assume $X$ is projective. Then every subvarity contributes to $H^{p,p}(X)$. Say $Z \subset X$ is codimenion $p$ then $\eta_X = PD[Z] \in H^{2p}(X, \Z) \cap H^{p,p}(X)$. Therefore $b_{2p}(X) \ge h^{p,p}(X) > 0$. 

\begin{example}
The Hodge diamond with $1$ on the middle column and 0 elsewhere exists and is $\P^n$. 
\end{example}

If $X$ is projective, then $H^(X, \Z)$ caries a polarization so we get a polarized Pure hodge structure of weight $i$. Let $H_{\Z} := H^i(X, \Z)$ is a finitely generated free abelian group. Such that
\begin{enumerate}
\item there is a decomposition
\[ H_{\Z} \ot_\Z \CC = \bigoplus_{p+q = i} H^{p,q} \]
\item $H^{p,q} = \ol{H^{q,p}}$
\end{enumerate}
And there is a pairing:
\[ Q : H_\Z \times H_\Z \to \Z \] 
such that
\begin{enumerate}
\item $Q$ is symmetric when $i$ is even, anti-symmetric when $i$ is odd
\item decomposition is orthogonal with respect to $(\sqrt{-1})^i S(\alpha,\beta)$ where $S(\alpha, \beta) = Q_{\CC}(\alpha, \bar{\beta})$. 
\item $(\sqrt{-1})^{i(i+1)/2} S(\alpha, \alpha) > 0$
\end{enumerate}

When it is only compact Kahler we can do this over $\RR$ but not over $\QQ$ or $\ZZ$. 

\subsection{Filtration}

Alternatively consider the decreasing filtration:
\[ F^\ell H_{\CC} = \bigoplus_{p \ge \ell} H^{p,q} \]
This satisfies
\[ F^\ell \cap \ol{F^{i - \ell + 1}} = 0 \]
and thus 
\[ F^\ell \oplus \ol{F^{i - \ell + 1}} = H_{\CC} \]
and these properties determine the Hodge structure. 

\subsection{Degeneration of the Hodge-to-De Rham Spectral sequence}

Somewhat weaker than the full Hodge decomposition but works algebraically so has arithmetic information. 
 
On the complex $\Omega_X^\bullet$ there is a decreasing Filtration
\[ F^p \Omega_X^\bullet := \Omega_X^{\ge p} := [0 \to \Omega_X^p \to \Omega_X^{p+1} \to \cdots \to \Omega_X^n \to 0] \] 
The associated graded is
\[ \gr^p_F \Omega_X^\bullet = \Omega_X^p[-p] \]
the sheaf $\Omega_X^p$ supported in degree $p$. The assocated spectral sequence of the Filtered complex 
\[ E^{p,q}_1 := \HH^{p+q}(X, \gr^p_F \Omega_X^\bullet) = H^q(X, \Omega_X^p) \implies \HH^{p+q}(X, \Omega_X^\bullet) \cong H^{p+q}(X, \CC) \]

\begin{theorem}
The spectral sequence degenerates at $E_1$. 
\end{theorem}


\subsection{Lefschetz Theorems}

\begin{theorem}[Hard Lefschetz]
$(X, \omega)$ compact Kahler manifold. Let $L = \omega \wedge -$ be the Lefschetz operator. It acts on cohomology by the de Rham theorem. Then for all $i \le n$
\[ L^{n-i} : H^i(X, \CC) \to H^{2n - i}(X, \CC) \]
is an isomorphism. 
\end{theorem}

Poincare dualiy: nondegenerate pairing
\[ H^i(X, \CC) \times H^{2n-i}(X, \CC) \to \C \quad \quad (\alpha, \beta) \mapsto \int_X \alpha \smile \beta \]

\begin{theorem}[Weak Lefschetz]
If $X$ is smooth projective of dimension $n$ and $D$ is any ample effective divisor on $X$ then 
\[ H^i(X, \ZZ) \to H^i(D, \ZZ) \]
is an isomorphism for $i < n-1$ and injective for $i = n-1$. 
\end{theorem}

One approach: when $D$ is smooth it follows from Kodaira vanishing. Consider $U = X \sm D$ which is an affine variety then our theorem is just about the cohomology of a affine. 

\begin{theorem}[Andreotti-Frankel]
if $Y$ is a closed subvariety of $\CC^n$ then $H_j(Y, \Z) = H^j(Y, \Z) = 0$ for $j > n$. In fact, $Y$ has the homotopy type of a CW complex of real dimension $n$. 
\end{theorem}
 
 
Now we can prove Lefschetz: using the relative cohomology exact sequence
\[ H^i(X, D; \Z) \to H^i(X, \Z) \to H^i(D, \Z) \to H^{i+1}(X, D; \Z) \] 
Then by Alexander duality
\[ H^i(X, D;\Z) \cong H_i(U, \Z) = 0 \]
so we win by Andreotti-Frankel. Notice if $D$ is smooth then $H^i(X, \Z) \to H^i(D, \Z)$ is a map of pure hodge structures. 

\begin{cor}
$H^{p,q}(X) \to H^{p,q}(D)$ are isomorphisms for $p + q < n - 1$ and injections for $p + q = n - 1$. 
\end{cor}

\begin{theorem}[Lefschetz (1,1)]
If $X$ is smooth projective then every integral $(1,1)$-class $\alpha \in H^2(X, \Z) \cap H^{1,1}(X)$ is $c_1(\L)$ for some line bundle and hence is a $\Z$-linear combination of classes of effective divisors. 
\end{theorem}

Where does the $c_1$ map come from. The exponential map:
\[ 0 \to \ul{\Z}_X \to \struct{X} \xrightarrow{\exp} \struct{X}^\times \to 0 \]
is exact (where we look at $X$ as an analytic variety) so there is an exact sequence
\[ 0 \to H^1(X, \Z) \to H^1(X, \struct{X}) \to H^1(X, \struct{X}^\times) \xrightarrow{c_1} H^2(X, \Z) \to H^2(X, \struct{X}) \]
then the boundary map is $c_1$ viewing $H^1(X, \struct{X}^\times) = \Pic{X}$ and its kernel is
\[ \mathrm{Pic}^0(X) \cong H^1(X, \struct{X}) / H^1(X, \Z) \] 
 

\begin{theorem}
If $f : X \to B$ is a proper submersion with smooth compact Kahler fibers then the Hodge numbers $h^{p,q}(X_b)$ are constant for $b \in B$. 
\end{theorem}

\begin{proof}
By Ehresmann's theorem, it is $C^\infty$-locally trivial and hence $b_i(X_b)$ is constant. However
\[ b_i(X_b) = \sum_{p+q = i} h^{p,q}(X_b) \]
and the $h^{p,q}(X_b)$ are upper semi-continuous since they are cohomology groups of coherent sheaves. Therefore they must be constant. 
\end{proof}
 
 
 
 
\section{Sep. 09 - Rational Singularities}

\subsection{Resolution}

\begin{defn}
Let $X$ be a smooth variety, a \textit{simple normal crossings (SNC)} divisor $E \subset X$ is a divisor of the form
\[ E = \sum_{i = 1}^m E_i \]
where $E_i$ is a smooth irreducible hypersurface in $X$ and such that the intersections are transverse i.e. analytically locally $E$ is cut out by $x_1 \dots x_s = 0$ for some $s \le n := \dim{X}$.
\end{defn} 

\begin{example}
A nodal curve in $\P^2$ is not SNC since its components are not smooth. However it is a normal crossings divisor (NC) which is just the local condition. 
\end{example}
 
 
\begin{defn}
A resolution of singularities of a variety $X$ is a proper birational map $f : \wt{X} \to X$ with $\wt{X}$ smooth 
\begin{enumerate}
\item a log resolution of $X$ is a resolution $f : \wt{X} \to X$ such that (the support of) $\Exc{f}$ is an SNC divisor 
\item more generally, for any effective $\Q$-divisor $D$ on $X$ a log resolution $(X, D)$ is a resolution $f : \wt{X} \to X$ with $\wt{D} = f^{-1}(D) \cup \Exc{f}$ has SNC support.
\end{enumerate}
\end{defn}

\begin{theorem}[Hironaka]
For any $X$ (or pair $(X, D)$) in characteristic zero, log resolutions exist. Moreover, they can be chosen with the properties
\begin{enumerate}
\item $f$ is an isomorphism away from the singular locus of $X$ (or the singular locus of $X$ union the non SNC locus of $(X, D)$)
\item $f$ is a composition of blowups with smooth centers.
\end{enumerate}
\end{theorem}

\begin{example}
Let $Y \subset \P^n$ be a smooth subvariety. Then there is an affine cone $X := C(Y)$ given by taking the homogeneous equations and viewing them in $\A^{n+1}$. To resolve, we blowup the origin of $\A^{n+1}$. The strict transform of $E$ intersects $E \cong \P^n$ at the projectivization of the tangent cone which is again isomorphic to $Y$ making $X$ the total space of a vector bundle over $Y$. This gives a log resolution of $(\A^{n+1}, X)$. 
\end{example}

\begin{rmk}
An ``ordinary singularity'' here was a singularity whose projectivization of its tangent cone is smooth.
\end{rmk}
 
\subsection{Rational Singularities}

\begin{defn}
A variety $X / \CC$ has \textit{rational singularities} if for any resolution of singularities $f : \wt{X} \to X$ we have $R f_* \struct{\wt{X}} \cong \struct{X}$.
\end{defn}

\begin{rmk}
We split this into two conditions:
\begin{enumerate}
\item $f_* \struct{\wt{X}} \cong \struct{X}$ i.e. $X$ is normal
\item $R^i f_* \struct{\wt{X}} = 0$ for $i > 0$ which is the real condition. 
\end{enumerate}
\end{rmk}

\begin{exercise}
If $f : Y \to X$ is a proper birational map of smooth varities then $\R^i f_* \struct{Y} = 0$ for $i > 0$. 
\end{exercise}

\begin{rmk}
The above is also true in characteristic $p$ but is much harder to show.
\end{rmk}

\begin{exercise}
Grauert-Riemenschneider Theorem: if $f : Y \to X$ is a generically finite surjective morphism (in characteristic zero) and $Y$ is smooth then $R^i f_* \omega_Y = 0$ for $i > 0$. 
\end{exercise}
 
\begin{prop}
Let $f : \wt{X} \to X$ be a resolution of singularities. Then the following are equivalent:
\begin{enumerate}
\item $\R f_* \struct{\wt{X}} \cong \struct{X}$ 
\item $X$ is Cohen-Macaualy and $f_* \omega_{\wt{X}} \cong \omega_X$ where $\omega_X$ is the dualizing sheaf of $X$ (which exists for $X$ CM and irreducible)
\end{enumerate}
\end{prop}

\begin{rmk}
There is a less fancy duality theory proof in Kollar-Mori section 5.1. 
\end{rmk}

\begin{prop}
every variety $X$ has a dualizing complex $\omega_X^\bullet \in D^\flat(X)$. One way to get it is to embed $X \embed Y$ in a smooth variety $Y$ (we can always do this locally) then $\omega_X^\bullet := \RHom{\struct{Y}}{\struct{X}}{\omega_Y}[\dim{Y}]$ note that $\omega_Y[\dim{Y}] = \omega_Y^\bullet$ since $Y$ is smooth. 
\end{prop} 
 
\begin{rmk}
Always we let $\omega_X := \cH^{-\dim{X}}(\omega_X^\bullet)$ is a sheaf but the issue is that there may be terms in larger degree. The commutative algbera problem we are solving is (recall $\omega_Y$ is a line bundle)
\[ \Ext{R}{i}{R/I}{R} = 0 \]
for $R$ a regular ring and the smallest $i$ for which this does not hold is $\depth{R}{I,R}$ and $R/I$ is CM iff $\depth{R}{I,R} = \height{I}$ i.e. only one cohomology group so the dualizing complex is a dualizing sheaf. 
\end{rmk}
 
\begin{lemma}
$X$ is CM iff $\omega_X^\bullet = \omega_X[\dim{X}]$. Note that $\omega_X$ is a line bundle iff $X$ is Gorenstein. 
\end{lemma}

Grothendieck Duality: for $f : Y \to X$ proper
\[ \R f_* \RHom{Y}{\F}{\omega_Y^\bullet} \cong \RHom{X}{\R f_* \F}{\omega_X^\bullet} \] 
 
\begin{proof}[Proof of proposition]
Let $\F = \omega_{\wt{X}}[\dim{X}]$ and apply Grothendieck duality to $f : \wt{X} \to X$ then 
\[ \R f_* \struct{\wt{X}} = \R f_* \RHom{\wt{X}}{\omega_{\wt{X}}[\dim{X}]}{\omega_{\wt{X}}[\dim{X}]} = \RHom{X}{\R f_* \omega_{\wt{X}}[\dim{X}]}{\omega_X^\bullet} \]
But by Grauert-Riemenschneider: $\R f_* \omega_{\wt{X}} = f_* \omega_{\wt{X}}$ and thus, setting $n := \dim{X}$ we get,
\[ \R f_* \struct{\wt{X}} = \RHom{X}{f_* \omega_{\wt{X}}[n]}{\omega_X^\bullet} \]
Now applying the functor $\RHom{X}{-}{\omega_X^\bullet}$ we get 
\[ \RHom{X}{\R f_* \struct{\wt{X}}}{\omega_X^\bullet} \cong f_* \omega_{\wt{X}}[n] \]
Therefore $\R f_* \struct{\wt{X}} \cong \struct{X}$ iff $\omega_X^\bullet \cong f_* \omega_{\wt{X}}[n]$. 
\end{proof}
 
 
\begin{cor}
The following are equivalent:
\begin{enumerate}
\item $X$ has rational singularities
\item there exists a resolution $f : \wt{X} \to X$ such that $\R f_* \struct{\wt{X}} \cong \struct{X}$. 
\end{enumerate}
\end{cor}

\begin{proof}
(a) implies (b) is obvious so assume (b). By the proposition, $X$ is CM and $f_* \omega_{\wt{X}} \cong \omega_X$. Given any other resolution $g : Y \to X$ we can find a roof i.e. a resolution $h : Z \to X$ that dominates both. We need to show that $g_* \omega_Y = \omega_X$ but in the diagram
\begin{center}
\begin{tikzcd}
& Z  \arrow[dd, "h"] \arrow[ld, "\varphi"] \arrow[rd, "\psi"]
\\
\wt{X} \arrow[rd, "f"] & & Y \arrow[ld, "g"]
\\
& X 
\end{tikzcd}
\end{center}
since $\varphi, \psi$ are birational maps of smooth varities we have $\varphi_* \omega_Z = \omega_{\wt{X}}$ and $\psi_* \omega_Z = \omega_Y$ and thus $h_* \omega_Z = f_* \omega_{\wt{X}} = \omega_X$ by assumption so $h_* \omega_Z = g_* \omega_Y$ so we win. 
\end{proof}

\begin{rmk}
We of course could have made the same argument with $\R f_* \struct{X}$ and it is just as easy except that we need to use composition of derived functors / Leray spectral sequence. But this way we can just work the level of sheaves. Also it will be useful to use the dualizing sheaf characterization of rational singularities. 
\end{rmk}

\begin{example}
\begin{enumerate}
\item
If $C$ is a curve, rational singularities implies $C$ normal so $C$ is smooth.
\item 
$(x_1^2 + x_2^2 + x_3^2 = 0) \subset \A^3$ the quadric cone has rational singularities
\item if $Y \subset \P^n$ is a smooth hypersurface of degree $d$ then $X = C(Y)$ is a rational singularity iff $d \le n$. Indeed, consider the resolution $\wt{X} \to X$ given by blowing up the cone point. Since $X$ is a hypersurface in $\A^{n+1}$ it is lci and hence Gorenstein so we just need to check if $f_* \omega_{\wt{X}} = \omega_X$. 
\end{enumerate}
\end{example}

\section{Sept. 11 - Rational Singularities II}

\subsection{Some Advantages of Rational Singularities}

Let $X$ be a projective variety with only rational singularities then
\begin{enumerate}
\item if $f : \wt{X} \to X$ is a resolution of singularities: $H^i(X, \struct{X}) = H^i(\wt{X}, \struct{\wt{X}})$ for all $i$. Proof: both equal $H^i(X, R f_* \struct{\wt{X}})$ 
\item Kodaira vanishing holds: if $L$ is an ample line bundle on $X$ then $H^i(X, \omega_X \ot L) = 0$ for all $i > 0$. Proof: $\omega_X = \R f_* \omega_{\wt{X}}$ and therefore by the projection formula
\[ H^i(\wt{X}, \omega_{\wt{X}} \ot f^* L) = H^i(X, \omega_X \ot L) \]
and the LHS is zero by Kawamata-Viehweg vanishing. 
\end{enumerate}

\begin{rmk}
For a variety over $\CC$ having rational singularities is the same notion in the algebraic or analytic categories. Indeed if $f : Y \to X$ is proper then $(R^i f_* \F)^\an = R^i f^{\an}_* \F^{\an}$ for any coherent sheaf $\F$ on $Y$.
\end{rmk}

\begin{example}
We continue our calculation of the cone over a hypersurface from last time. Let $X = C(Y) \subset \A^{n+1}$ be the cone over a smooth projective variety $Y \subset \P^n$. Since $X$ is a hypersurface, it is CM so we need only understand when $f_* \omega_{\wt{X}} = \omega_X$ where $f : \wt{X} \to X$ is the log resolution of $(\A^{n+1}, X)$ givien by blowing up the origin. Recall
\[ K_{\wt{\A}^{n+1}} = f^* K_{\A^{n+1}} + n E \]
and likewise
\[ f^* X = \wt{X} + d E \]
since the multiplicity of the point is $d$. Now we compute
\[ K_{\wt{X}} = (K_{\wt{\A}^{n+1}} + \wt{X})|_{\wt{X}} = f^* (K_{\A^{n+1}} + X) + (n-d) E \]
Therefore, by the projection formula,
\[ f_* \omega_{\wt{X}} = \omega_{X} \ot f_* \struct{\wt{X}}((n-d)E|_X) = \begin{cases}
\omega_X & d \le n
\\
\omega_X \ot \I_0^{(d - n)} & d > n
\end{cases} \]
where $\I_0$ is the ideal sheaf of the origin of the cone. Therefore, $X$ has rational singularities if and only if $d \le n$. 
\end{example}

\begin{example}
Abstract cones: fix $Y$ a variety and let $L$ be an ample line bundle. Let, 
\[ X = \Spec{ \bigoplus_{m \ge 0} H^0(X, L^{\ot m}) } \to Y \]
If $Y$ is normal and $L = \struct{Y}(1)$ for an embedding $Y \embed \P^n$ then $X = C(Y,L)$ is the normalization of $C(Y)$. Exercise: $C(Y, L)$ has rational singularities if and only if $H^i(Y, L^{\ot m}) = 0$ for all $i > 0$ and $m \ge 0$. In particular, this requires $H^i(Y, \struct{Y}) = 0$ for $i > 0$. This fails for all Calabi-Yau but holds for Fano if we take $L = -K_Y$ by Kodaira vanishing:
\[ H^i(Y, L^{\ot m}) = H^i(Y, L^{\ot (m+1)} \ot K_Y) = 0 \]
since $m + 1 > 0$ so $L^{\ot (m+1)}$ is ample. 
\end{example}

\subsection{More Flexible Interpretation of Rational Singularities}

\begin{theorem}[Kov\'{a}cs]
Let $f : Y \to X$ be a proper morphism of complex varities such that $Y$ has rational singularities (e.g. $Y$ smooth) and the morphism $\varphi : \struct{X} \to \R f_* \struct{Y}$ has a splitting in $D^\flat(X)$ (i.e. $\exists : \psi : \R f_* \struct{Y} \to \struct{X}$ such that $\psi \circ \varphi$ is a quis) then $X$ has rational singularities. 
\end{theorem}

\begin{proof}
Consider compatible resolutions:
\begin{center}
\begin{tikzcd}
\wt{Y} \arrow[r, "\tilde{f}"] \arrow[d, "\sigma"] & \wt{X} \arrow[d, "\pi"]
\\
Y \arrow[r, "f"] & X
\end{tikzcd}
\end{center}
Get a diagram of complexes of sheaves,
\begin{center}
\begin{tikzcd}
\struct{X} \arrow[r, "\varphi"] \arrow[d, "\alpha"] & \R f_* \struct{Y} \arrow[d, "\beta"]
\\
\R f_* \struct{\wt{X}} \arrow[r, "\gamma"] & \R (f \circ \sigma)_* \struct{\wt{Y}} 
\end{tikzcd}
\end{center}
We are assuming $\beta$ is an isomorphism and $\varphi$ has a splitting $\psi$. Therefore $\gamma \circ \alpha$ has a splitting meaining $\psi \circ \gamma \circ \alpha = \id$ so $\alpha$ has a splitting. Therefore, we reduce to the same statement but applied to the resolution $\pi : \wt{X} \to X$. 
\bigskip\\
Consider the splitting
\[ \struct{X} \xrightarrow{\varphi} \R f_* \struct{\wt{X}} \xrightarrow{\psi} \struct{X} \]
composing to the identiy. Dualize to obtain
\[ \omega_X^\bullet \to \R f_* \omega_Y[n] \to \omega_X^\bullet \]
and the composition is still a quasi-isomorphism. This is because
\[ \RHom{Y}{\R f_* \struct{\wt{X}}}{\omega_X^\bullet} = \R f_* \RHom{X}{\struct{\wt{X}}}{\omega_{\wt{X}}[n]} = \R f_* \omega_{\wt{X}} [n] \]
since $\wt{X}$ is smooth. By G-S $\R f_* \omega_{\wt{X}}[n] = (f_* \omega_{\wt{X}})[n]$ and therefore taking cohomology sheaves we get splittings
\[ \cH^i(\omega_X^\bullet) \to R^{i+n} f_* \omega_{\wt{X}} \to \cH^i(\omega_X^\bullet) \]
since the middle term is zero for $i \neq -n$ we get $\cH^i(\omega_X^\bullet)$ is zero for $i \neq -n$ so $X$ is CM. Furthermore, for $i = -n$ we get maps
\[ \omega_X \to f_* \omega_{\wt{X}} \to \omega_X \] 
the map $f_* \omega_{\wt{X}} \to \omega_X$ always exists (it is the trace map) and it is injective but this composition shows surjective and hence an isomormphism. Therefore we conclude $X$ has rational singularities. 
\end{proof}

\begin{cor}
If $f : Y \to X$ is a finite surjective morphism of normal varities and $Y$ has rational singularities then $X$ has rational singularities.
\end{cor}

\begin{proof}
Indeed, there is a trace map $f_* \struct{Y} \to \struct{Y}$ giving a splitting so the theorem applies (recall $R f_* \struct{Y} = f_* \struct{Y}$ since $f$ is finite).
\end{proof}

\begin{cor}
Finite quotient singularities are rational. 
\end{cor}

\begin{example}
All quotients by reductive groups have rational singularities because there is a splitting $f_8 \struct{Y} \to \struct{X}$ given by the Reynolds operator (defined via an averaging procedure)
\end{example}

It is important to know that quotient singularities are rational because then coarse moduli spaces of smooth Artin stacks are rational. In particular, the moduli spaces of curves or of vector bundles on a smooth curve. 

\begin{exercise}
Kov\'{a}c's paper: let $f : Y \to X$ be a surjective morphism with connected fibers of projective varities over $\CC$ and $Y$ has rational singularities. Then $X$ has rational singularities $\iff R^{\dim{X} - \dim{Y}} f_* \omega_Y = \omega_X$.
\end{exercise}

Koll\'{a}r noted that if $f : Y \to X$ is a map whose \textit{general} fiber has dimenson $k$ then $R^i f_* \omega_Y = 0$ for $i > k$. He also showed that if $X$ is smooth then $R^k f_* \omega_Y = \omega_X$. The above exercise generalizes this. 

\begin{exercise}
Other examples of rational singularities (references in the notes):
\begin{enumerate}
\item toric varieties
\item generic determinantal varieties: fix integers $n,m$ and $k = \min \{ n, m \}$ then consider the vanishing of the $k \times k$ minors of a $m \times n$ matrix of indeterminants
\item the previous implies that $\Theta$-divisors of Jacobians have rational singularities.
\end{enumerate}
\end{exercise}

\section{Sep. 16 Singularities from the point of view of the MMP}

Let $X$ be a complex variety and $f : \wt{X} \to X$ a log resolution of singularities. Let $E = \sum E_i$ be the support of the exceptional locus which is SNC. We assume
\begin{enumerate}
\item $X$ is normal (in particular $X_{\text{sing}} \ge 2$) so that Weil divisors work nicely 
\item $K_X$ is $\Q$-Cartier (more flexible than assuming $X$ is Gorenstein) meaning $n K_X$ is Cartier for some integer $n \ge 1$. 
\end{enumerate}

\begin{exercise}
Let $X = \CC^3 / (\Z / 2 \Z)$ with the action $(x,y,z) \mapsto (-x,-y,-z)$ then $K_X$ is \textit{not} Cartier but it is $\Q$-Cartier. Hence $X$ has a non lci quotient singularity. 
\end{exercise}

We can always write 
\[ K_{\wt{X}} \sim_{\Q} f^* K_X + \sum a_i E_i \quad a_i \in \Q \]
which just means
\[ m K_{\wt{X}} \sim f^* (m K_X) + \sum b_i E_i \quad b_i \in \Z \]
where $m K_X$ is integral. 

\newcommand{\reg}{\mathrm{reg}}

\begin{rmk}
We can talk about 
\begin{enumerate}
\item for $\omega_X^\bullet$ the dualizing complex $\omega_X := \cH^{-\dim{X}}(\omega_X^\bullet)$ (if $X \embed Y$ with $Y$ smooth then this is $\shExt{c}{}{\struct{X}}{\omega_Y}$) or
\item for $\iota : U = X_{\reg} \embed X$ we let $\omega_X = \iota_* \omega_{U}$ which is a reflexive sheaf since $U$ has codimension at least $2$.
\end{enumerate}
In fact these two give the same sheaf. However $\omega_X$ is not necessarily a line bundle it is a ``$\Q$-line bundle'' in the sense that $\omega_X \not\cong \struct{X}(K_X)$ in general but $\omega_X^{\ot m}$ is a line bundle so $\omega_X = \omega_X^{[m]}$ and we write $\omega_X^{[m]} = (\omega_X^{\ot m})^{\vee \vee} \cong \struct{X}(m K_X)$ 
\end{rmk}

\begin{defn}
We say $X$ is 
\begin{enumerate}
\item \textit{terminal} if $a_i > 0$
\item \textit{canonical} if $a_i \ge 0$
\item \textit{log terminal} if $a_i > -1$
\item \textit{log canonical} if $a_i \ge -1$
\end{enumerate}
\end{defn}

\begin{rmk}
If $X$ is Gorenstein then canonical = log terminal.
\end{rmk}

\begin{theorem}[Elkik]
If $X$ has klt singularities then $X$ has rational singularities. 
\end{theorem}

\begin{rmk}
KMM, Fujita: in $(X, D)$ is a pair with dlt singularities then $X$ has rational singularities. 
\end{rmk}

\begin{rmk}
Say $X$ is Gorenstein then rational implies canonical singularities since for any resolution $f : \wt{X} \to X$ we have $f_* \omega_{\wt{X}} = \omega_X$ which implies $a_i \ge 0$ because in the Gorenstein case we have
\[ \omega_{\wt{X}} = f^* \omega_X \ot \struct{\wt{X}}(\sum a_i E_i) \]
\end{rmk}

\begin{proof}[Proof of Elkik's theorem]
Step (1) reduce to the Gorenstein case. Say $m K_X$ is Cartier with $m > 1$ (note the smallest such $m$ is called the Gorenstein index of $X$) then locally on $X$ there is an index $1$-cover $Y \to X$ menaing $Y$ is Gorenstein and $Y \to X$ is finite. We construct it as follows. Let $L = \struct{X}(mK_X)$ so locally on $X$ this has a section and we take the cyclic cover for this section.
\bigskip\\
Exercise: $Y \to X$ is an index $1$ cover and $X$ klt then $Y$ has canonical singularites. If we can show $Y$ has rational singularities then we are done by Kov\'{a}cs' theorem. 
\bigskip\\
Step 2: assume $X$ has canonical Gorenstein singularities i.e.
\[ \omega_{\wt{X}} = f^* \omega_X \ot \struct{\wt{X}}(\sum a_i E_i) \]
with $a_i \ge 0$. Thus there is an injection $f^* \omega_X \embed \omega_{\wt{X}}$. Using the trace map we get a splitting sequence
\[ \omega_X \to f_* \omega_{\wt{X}} \xrightarrow{\tr} \omega_X \]
and the composition is an isomorphism because it is an isomorphism away from codimension $2$ and $\omega_X$ is a line bundle. Therefore $X$ has rational singularities by the argument in Kov\'{a}cs' result on splitings\footnote{We need $X$ to be CM for this to work. This is implicit in the fact we already know it is Gorenstein}.
\end{proof}

\begin{example}
Let $X$ be the cone over $Y \subset \P^2$ is a smooth plane cubic (an elliptic curve after chosing a base point). Let $f : \wt{X} \to X$ be the standard resolution. We saw that $f_* \omega_{\wt{X}} \embed \omega_X$ is \textit{not} an isomorphism so this is not a rational singularity and indeed $R^1 F_* \struct{\wt{X}} \neq 0$. 
\bigskip\\
However, what happens if instead of taking the strict transform we took the proper transform $G := f^{-1}(X)_{\red} = \wt{X} \cup E \subset \wt{A}^3$. What about $\R f_* \struct{G}$? We use the inclusion-exclusion sequence
\[ 0 \to \struct{G} \to \struct{\wt{X}} \oplus \struct{E} \to \struct{F} \to 0 \]
where $F \cong Y$ is the intersection of $\wt{X}$ and $E$. Pushing forward gives an exact triangle
\[ \R f_* \struct{G} \to \R f_* \struct{\wt{X}} \oplus \R f_* \struct{E} \to \R f_* \struct{F} \to + 1 \]
and $\R f_* \struct{E} = \struct{0}$ since $E$ maps to a point and $E \cong \P^2$ so it has no higher cohomology. Likewise $\R f_* \struct{F} \cong [ \struct{0} \to \struct{0}]$ in degrees $[0,1]$ with the zero map since $F$ is an elliptic curve mapping to a point so explicitly
\begin{enumerate}
\item $f_* \struct{E} = \struct{0} \quad R^i f_* \struct{E} = H^i(\struct{\P^n}) = 0 \text{ for } i > 0$
\item $f_* \struct{F} = \struct{0} \quad R^1 f_* \struct{F} = H^1(F, \struct{F}) \ot \struct{0} = \struct{0}$ 
\end{enumerate}
Therefore the long exact sequence gives
\[ 0 \to f_* \struct{G} \to \struct{X} \oplus \struct{0} \to \struct{0} \to R^1 f_* \struct{G} \to \struct{0} \to \struct{0} \to R^2 f_* \struct{G} \to 0 \]
therefore we conclude $\R f_* \struct{G} = \struct{X}$ since the maps between $\struct{0}$ are nontrivial. 
\end{example}

\textbf{Preliminar Definition}: $X$ has du Bois singularites if there exists (equivalently for all) locally an embedding $X \embed Y$ with $Y$ smooth and a log resolution $f : \wt{Y} \to Y$ of $(Y, X)$ such that $G = f^{-1}(X)_{\red}$ satisfies $\R f_* \struct{G} = \struct{X}$.  

\begin{rmk}
We will identify $\R f_* \struct{G}$ for any log resolution with the du Bois complex $\ul{\Omega}^0_X$ which is intrinsic.  
\end{rmk}

\begin{exercise}
If $Y \subset \P^n$ is a smooth hypersurface of degree $d$ and $X \subset \A^{n+1}$ is the cone over $Y$ then $X$ has du Bois iff $d \le n + 1$. 
\end{exercise}

To do things the right way: study the filtered de Rham complex of a complex variety. 

\subsection{Motivation}

If $X$ is smooth projective: we truncated the de Rham complex
\[ 0 \to \struct{X} \to \Omega_X^1 \to \Omega_X^2 \to \cdots \to \Omega_X^n \to 0 \]
to get the ``stupid filtration'' $F^\ell \Omega_X^\bullet := \Omega^{\ge \ell}_X \subset \Omega^\bullet_X$. This gives the Hodge filtration $F^\ell H^i(X, \C) = \bigoplus_{p \ge \ell} H^q(X, \Omega_X^p)$ and we got
\[ \gr_F^\ell H^i = H^{i - \ell}(X, \Omega_X^\ell) \]
If $X$ is singular, we no longer have degeneration at $E_1$ of this spectral sequence. 
\bigskip\\
Want: construct some gadget $\ul{\Omega}_X^\ell \in D^\flat_{\mathrm{Coh}}(X)$ such that
\[ \gr_F^\ell H^i(X, \CC) \cong H^{i - \ell}(X, \ul{\Omega}_X^\ell) \]
for the Hodge filtration on the mixed Hodge structure. 
\end{document}

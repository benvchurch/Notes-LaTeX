\documentclass[12pt]{article}
\usepackage{import}
\import{./}{AlgGeoCommands}
\renewcommand{\U}{\mathfrak{U}}

\DeclareMathOperator{\fib}{\mathrm{fib}}
\DeclareMathOperator{\cofib}{\mathrm{cofib}}
\newcommand{\sq}{\mathrm{sq}}
\DeclareMathOperator{\dlog}{\mathrm{dlog}}
\newcommand{\st}{\mathrm{st}}

\begin{document}

\section{Sept 30}

\newcommand{\Nis}{\mathrm{Nis}}

Make all the homotopy equivalences invertible.  


Nisnvictch topology on smooth schemes $\Nis_S$ which send all 


No tubular neighborhood theorem. 



Hilbert spaces of infinite affine spaces: show there are $\A^1$-equivalences,
\[ \Hilb_d(\A^{\infty}) \iso \mathrm{FFlat}_d \iso \mathrm{Vect}_{d-1} \iso \mathrm{Gr}_{d-1}(\A^{\infty}) \]
by sending,
\[ \{ X \to \A^{\infty}_S \} \mapsto (X \to S) \mapsto \mathrm{etc} \]
for any $X \to S$ there is, up to homotopy, a unique embedding into infinite affine space. In the case $X \to S$ is $A \to B$ finite flat map then we send 


Motivic homotopy theory, algebraic cobordism, other theories. 

Universal six functor formalism given by motivic homology theory. (VERY COOL)

analogue of how Cohomology theories factor throught stable homotopy category.


Primer for unstable motivic homotopy theory.

Purity is replacement for tubular neighborhood theorem (hmmm!)

\section{Oct 7}

\newcommand{\Map}[3]{\mathrm{Map}_{#1}\left(#2, #3 \right)}
\newcommand{\Fun}[2]{\mathrm{Fun}\left( #1, #2 \right)}

The data of an $\infty$-category $\C$,
\begin{enumerate}
\item $X, Y \in \C$ a space $\Map{\C}{X}{Y}$
\item replace strict composition with a homotopy class of composition maps,
\[ \Map{\C}{X}{Y} \times \Map{\C}{Y}{Z} \to \Map{\C}{X}{Z} \]
\end{enumerate}

\begin{defn}
$h\C$ is the homotopy category of $\C$ whose objects are the same but,
\[ \Map{h\C}{X}{Y} = \pi_0(\Map{\C}{X}{Y}) = [X,Y]_{\C} \]
\end{defn}

\begin{example}
\begin{enumerate}
\item $1$-categories are discrete $\infty$-categories
\item Spaces
\item $\infty$-category of $\infty$-categories
\item $\Fun{\C}{\D}$ and there is a morphism,
\[ h \Fun{\C}{\D} = \Fun{h \C}{h \D} \]
\end{enumerate}
\end{example}

\begin{prop}
\begin{enumerate}
\item $\Fun{*}{\C} \iso \C$
\item a morphism $\alpha \in \Fun{\C}{\D}$ is an equivalence iff its image in $\Fun{h \C}{h \D}$ is an equivalence iff $\alpha_X$ si an equivalence in $\D$ for all $X \in \C$. 
\end{enumerate}
\end{prop}

\begin{defn}
Let $F : \C \to \D$ be an $\infty$-functor. A colimit of $F$ is an object $d \in \D$ and a morphism $\alpha : F \to d*$ that induces an equivalence,
\[ \Map{\D}{d}{X} \iso \Map{\Fun{\C}{\D}}{d^*}{X^*} \xrightarrow{ \circ \alpha} \Map{\Fun{\C}{\D}}{F}{X^*} \]
is an equivalence where $d^*$ is the constant map $\C \to * \to \D$ at $d$. 
\end{defn}

\begin{rmk}
Here, uniqueness up to unique isomorphism is replaced with ``the space of colimits is contractible'' capturing the higher uniqueness. 
\end{rmk}

\subsection{Commutative Monoid OBjects in $\C$}

How do we capture the notion of a group object in an $\infty$-category? The associativity and inverse diagrams need higher coherences.

\newcommand{\CMon}[1]{\mathrm{CMon}\left( #1 \right)}
\newcommand{\Fin}{\mathrm{Fin}}
\newcommand{\Space}{\mathbf{Space}}
\newcommand{\Set}{\mathbf{Set}}
\newcommand{\gp}{\mathrm{gp}}

\begin{defn}
Let $\Fin_*$ be the $1$-category of finite pointed sets. Let $[\ul{n}] = \{ *, 1, \dots, n \}$ where $*$ is the base point. There are $n$ maps $p_k [\ul{n}] \to [\ul{1}]$ sending $i \mapsto 1$ and the rest to $*$.
\end{defn}

\begin{defn}
Let $\C$ be an $\infty$-category with finite products. Define $\CMon{\C} \subset \Fun{\Fin_*}{\C}$ to be the functors such that for each $n \ge 0$ the map,
\[  \prod_{k = 1}^n (p_k)_* : M([\ul{n}]) \to M([\ul{1}])^n \] 
is an equivalence.
\end{defn}

\begin{rmk}
Think of $M([\ul{0}]) \cong *$ and $M([\ul{1}]) = M$ the ``underlying object''. Then we obtain an addition,
\[ M([\ul{1}]) \times M([\ul{1}]) \iso M([\ul{2}]) \to M([\ul{1}]) \]
where the second map $[\ul{2}] \to [\ul{1}]$ sending $1,2 \mapsto 1$. Furthermore, there is a unit,
\[ M([\ul{0}]) \to M([\ul{1}]) \]
given by the unique map $[\ul{0}] \to [\ul{1}]$. 
\bigskip\\
Futhermore, in $\Fin_*$ the map $[\ul{2}] \to [\ul{1}]$ is invariant under swapping which gives homotopy coherent commutativity. Homotopy coherent associativity holds similarly. 
\end{rmk}

\begin{exercise}
If $\C$ is discrete then this is the usual definition. 
\end{exercise}

\begin{rmk}
If $\C = \Space$ and $M$ is in $\CMon{\Space}$ then $\pi_0(M)$ is a commutative monoid in $\Set$ meaning a usual commutative monoid. 
\end{rmk}

\begin{defn}
A cmonoid $M \in \CMon{\Space}$ is \textit{group-like} if $\pi_0(M)$ is a group. Consider the full subcategory,
\[ \CMon{\Space}^{\gp} \subset \CMon{\Space} \]
\end{defn}

\begin{defn}
The group completion of $M \in \CMon{\Space}$ is an object $M^{\gp} \in \CMon{\Space}^{\gp}$ along with a map $M \to M^{\gp}$ which is initial in the sense,
\[ \Map{\CMon{\Space}^\gp}{M^\gp}{N}  \iso \Map{\CMon{\Space}}{M}{N} \]
\end{defn}

\begin{rmk}
Given a symmetric monoidal $1$-category $(\C, \ot, 1)$ there is a functor $\Fin_* \to \text{1-Cat}$ defining this symmetric monoidal structure. 
\end{rmk}

Given a $1$-groupoid $G$ we can take the classifying space $BG$ and get a functor $B : \mathrm{Grpd}_1 \to \Space$. If $\C$ is a symmetric monoidal $1$-category then we apply maximal groupoid functor $\C \to \C^{\simeq}$ compose with $B$ to get $B \C^{\simeq} \in \CMon{\Space}$.


\section{Oct 14}


\subsection{Nisnevich Sheaves}

\newcommand{\Sm}{\mathrm{Sm}}
\newcommand{\Spc}{\mathrm{Spc}}
\newcommand{\PSh}{\mathrm{PSh}}
\newcommand{\pr}{\mathrm{pr}}

Let $\PSh(\Sm_S) = \Fun{\Sm_S^\op}{\Spc}$ 


For the Zariski site Gerstein property, a presheaf $\F$ is a zariski sheaf iff,
\begin{enumerate}
\item $\F(\empty)$ is contractible
\item for all opens $U, V \subset U \cup V$ we have,
\begin{center}
\begin{tikzcd}
\F(U \cup V) \arrow[r] \pullback \arrow[d] & \F(V) \arrow[d]
\\
\F(U) \arrow[r] & \F(U \cap V)
\end{tikzcd}
\end{center}
is a homotopy pullback diagram.
\end{enumerate}



\subsubsection{Why not Take the Zariski Site}

We want local models, for $x \in U$
\[ X / (X \sm \{ x \}) \simeq U / (U \sm \{ x \}) \]
this should be good. But moreover, for small enough $U$ we want,
\[ U / (U \sm \{ x \}) \simeq \A^n / (\A^n \sm \{ 0 \}) \]
This works in the \etale topology because smooth schemes are \etale locally affine space, 
\begin{center}
\begin{tikzcd}
X \arrow[d] \arrow[from=r, hook] & U \arrow[d, "\et"]
\\
\Spec{k} \arrow[from=r] & \A^n
\end{tikzcd}
\end{center}
Therefore, this property should work in the \etale topology. It will also work in the coarser topology called the Nisnevich topology.

\subsubsection{Why not Take the \etale Site}

$K$-theory does not satisfy \etale descent. 

\subsubsection{Criterion for a presheaf of spaces to be a Nisnevich sheaf}

\begin{defn}
Let $X$ be a smooth scheme over $S$. Nisnevich squares are diagrams of the form,
\begin{center}
\begin{tikzcd}
W \arrow[d] \arrow[r, hook] \pullback & V \arrow[d, "\et"] \arrow[from=r, hook'] & V \sm W \arrow[d, "\sim"] 
\\
U \arrow[r, hook, "\iota"] & X \arrow[from=r, "j"', hook'] & X \sm U
\end{tikzcd}
\end{center}
where $\iota$ is an open embedding and $j$ is the closed embedding. 
\end{defn}

\begin{example}
\begin{enumerate}
\item 
The Gerstein square gives a Nisnevich square,
\begin{center}
\begin{tikzcd}
U \cap V \arrow[d] \arrow[r, hook] \pullback & V \arrow[d, "\et"] \arrow[from=r, hook'] & V \sm (U \cap V) \arrow[d, "\sim"] 
\\
U \arrow[r, hook, "\iota"] & U \cup V \arrow[from=r, "j"', hook'] & V \sm U
\end{tikzcd}
\end{center}

\item 

\begin{center}
\begin{tikzcd}
U \cap \{ x \} \arrow[d] \arrow[r, hook] \pullback & V \arrow[d, "\et"] \arrow[from=r, hook'] & V \sm (U \cap V) \arrow[d, "\sim"] 
\\
U \arrow[r, hook, "\iota"] & U \cup V \arrow[from=r, "j"', hook'] & V \sm U
\end{tikzcd}
\end{center}

\item 
\end{enumerate}
\end{example}

\begin{defn}
$\F \in \PSh(\Sm_S)$ is a Nisnevich sheaf if,
\begin{enumerate}
\item $\F(\empty)$ is contractible
\item for any Nisnevich square,
\begin{center}
\begin{tikzcd}
\F(X) \pullback \arrow[r] \arrow[d] & \F(U) \arrow[d] 
\\
\F(V) \arrow[r] & \F(W)
\end{tikzcd}
\end{center}
is a homotopy pullback. 
\end{enumerate}
\end{defn}

\begin{rmk}
The diagram,
\begin{center}
\begin{tikzcd}
\empty \arrow[r] \arrow[d] \arrow[d] & U \arrow[d]
\\
V \arrow[r] & U \sqcup V
\end{tikzcd}
\end{center}
shows that,
\[ \F(U \sqcup V) \simeq \F(U) \times \F(V) \]
\end{rmk}

\subsection{$\A^1$-invariance}

\begin{defn}
$F \in \PSh(\Sm_S)$ is $\A^1$-invariant if for all $X \in \Sm_S$  the maps,
\[ \F(\pr_1) : \F(X) \to \F(X \times \A^1) \]
are homotopy equivalences.
\end{defn}

\begin{rmk}
We use the notation,
\begin{enumerate}
\item $\L_{\Nis}(\PSh(\Sm_S)) \embed \PSh(\Sm_S)$
\item $\L_{\A^1}(\PSh(\Sm_S)) \embed \PSh(\Sm_S)$
\end{enumerate}
for the full subcategory of Nisnevich sheaves and $\A^1$-invariant presheaves respectively. Why do we use this notation? This is because there is a left adjoint to the inclusions which we denote by $\L$.
\end{rmk}

\begin{defn}
The unstable motivic homotopy category is,
\[ \Spc(S) = \L_{\Nis}(\PSh(\Sm_S)) \cap \L_{\A^1}(\PSh(\Sm_S)) \]
\end{defn}

\begin{defn}
Let $R : \D \to \C$ be a functor of $\infty$-categories. A left adjoint for $R$ is a pair $(L : \C \to \D, \alpha : L \circ R \to \id_{\D})$ such that for all $X \in \C$ and $Y \in \D$ the natural map,
\[ \Hom{\C}{X}{R(Y)} \xrightarrow{L} \Hom{\D}{L(X)}{LR(Y)} \xrightarrow{\alpha_Y \circ -} \Hom{\D}{L(X)}{Y} \]
is an equivalence. 
\end{defn}

\begin{rmk}
There is a notion of uniqueness (HTT 5.2.6.2)
\end{rmk}


\begin{rmk}
In the $\infty$-category setting: left adjoint gives colimits right adjoints are limits.
\end{rmk}

\begin{defn}
An $\infty$-category $\C$ is presentable if,
\begin{enumerate}
\item $\C$ is locally small
\item $\C$ has small colimits
\item there exists regular cardinal $\kappa$ and $S \subset \mathrm{Obj}(\C)$ such that for all $X \in S$ is $\kappa$-compact (meaning $\Hom{\C}{X}{-}$ preserves $\kappa$-filerted colimit) and every $X \in \mathrm{Obj}(\C)$ can be presented as a colimits $\F : \left< S \right> \to \C$.
\end{enumerate}
\end{defn}

\begin{prop}
Let $\C$ be a presentable $\infty$-category,
\begin{enumerate}
\item $\C$ has colimits [HTT, 5.5.2.9]
\end{enumerate}
\end{prop}

\begin{prop}
The following $\infty$-categories are presentable,
\begin{enumerate}
\item $\Spc$ is presentable [HTT 5.5.18]
\item if $\C$ is small then $\Sh(\C)$ is presentable

\item $\Spc(S)$ is presentable
\item $\L_{\Nis}(\PSh(\Sm_S))$ and $\L_{\A^1}(\PSh(\Sm_S))$ are presentable.
\end{enumerate}
\end{prop}

\begin{theorem}
Let $\F : \C \to \D$ be a functor of presentable $\infty$-categories. Then,
\begin{enumerate}
\item $\F$ has a right adjoint iff $\F$ preserves small colimits
\item $\F$ has a left adjoint iff $\F$ preserves small limits and $\kappa$-filtered colimits 
\end{enumerate}
\end{theorem}

\begin{proof}
[HTT 5.5.2.9]
\end{proof}

\subsection{Explicit Construction of $\L_{\A^1}$}

\begin{defn}
Standard cosimplicial schemes,
\[ \Delta^\bullet : \Delta \to \Sm_S \]
given by 
\[ [n] \mapsto \Delta^n = V(T_0 + \cdots + T_N = 1) \subset \A^{n+1}_S \]
and given a monotone map $f : [m] \to [n]$ send it to the restriction of,
\[ T_i \mapsto T_{f(j) = i} T_j' \]
giving a map,
\[ \A^{m+1}_S \to \A_S^{n+1} \]
with coordinates $T_j'$ on the first and $T_i$ on the second. 
\end{defn}

\begin{theorem}[A25 of Primer]
Let $\F \in \PSh(\Sm_S)$ and $X \in \Sm_S$ then,
\[ (\L_{\A^1} \F)(X) = \colim_{[n] \in \Delta^\op} \F(X \times \Delta^n) \]
We can also set,
\[ \F_{\A^1} : \Delta^\op \to \PSh(\Sm_S) \quad [n] \mapsto \F(- \times \Delta^n) \]
\end{theorem}

\begin{defn}
A morphism $\alpha : \F \to \G$ in $\PSh(\Sm_S)$ is an $\A^1$-equivalence if $\L_{\A^1}(f)$ is an equivalence. 
\end{defn}

\begin{rmk}
For $\pi_X : X \times \A^1 \to X$ then $h^{\pr_X} : h^{X \times \A^1} \to h^X$ is an $\A^1$-equivalence. 
\end{rmk}

\begin{defn}
$\alpha, \beta : F \to G$ are $\A^1$-homotopic if ther eixsts,
\begin{center}
\begin{tikzcd}
F \times h^{\{0\}} \arrow[rd, "\alpha"] \arrow[d]
\\
F \times h^{\A^1} \arrow[r] & \G
\\
F \times h^{\{1 \}} \arrow[ru, "\beta"] \arrow[u]
\end{tikzcd}
\end{center}
\end{defn}

\begin{theorem}
Let $S$ be a regular noetherian scheem of finite dimension. There exists $K \in \Spc(S)$ such that for all $X = \Spec{A} \in \Sm_S$ we have $K(X) \simeq K(A)$ where $K(A)$ is the algebraic $K$-theory space of the ring $A$.
\end{theorem}

\begin{rmk}
on affine $A \mapsto K(A)$ is an $\A^1$-invariant Nisevich sheaf. In particular,
\[ K(A) \simeq K(A[t]) \]
For general $X \in \Sm_S$ not affine then $K(X)$ is the Thomason-Traubagh $K$-theory. 
\end{rmk}

\section{Oct 21}

\subsection{Infinite dimensional Hilbert schemes}

\newcommand{\FFlat}{\mathrm{FFlat}}
\newcommand{\Vect}{\mathrm{Vect}}
\newcommand{\Aff}{\mathrm{Aff}}

Fix $S$ noetherian scheme. Let $\PSh(\Sch_S)$ be presheaes over schemes. We will be proving that some presheaves are either motivically equivalent or $\A^1$-equivalent.

\begin{defn}
Let $\FFlat_S$ be the stack of finite flat schemes over $S$,
\[ T \to S \mapsto \{ \text{groupoid of finite finitely presented flat morphisms } X \to T \} \]
\end{defn}

\begin{rmk}
A finite flat map $\varphi : X \to S$ is finitely presented iff $\varphi_* \struct{X}$ is a finitely presented $\struct{T}$-module. (This is like saying that a finite algebra is finitely presented as an algebra iff finitely presented as a module). 
\end{rmk}

\begin{rmk}
Then we have the decomposition,
\[ \FFlat = \sqcup_{d \in \N} \FFlat_d \]
where $\FFlat_d \ subset \FFlat$ is the open and closed substack where $\varphi_* \struct{X}$ is a vector bundle of rank $d$.
\end{rmk}

\begin{defn}
Consider the stack,
\[ \Vect_S = \sqcup_{d \in \NN} \Vect_d \]
which is the stack of vector bundles over $S$. There is a forgetful morphism,
\[ \FFlat \to \Vect \]
sending $(X \to T) \mapsto \varphi_* \struct{X}$ which is degree preserving meaning it induces,
\[ \FFlat_d \to \Vect_d \]
\end{defn}

We refine this forgetful morphism (using the local) criterion for flatness,
\[ \nu : \FFlat_d \to \Vect_{d-1} \]
given by,
\[ (X \to T) \mapsto \varphi_* (\struct{X}) / (\struct{T} \cdot 1) \]

\begin{rmk}
Locally $T = \Spec{R}$ and $X = \Spec{A}$ then the inclusion $R \to A$ gives a sequence,
\begin{center}
\begin{tikzcd}
0 \arrow[r] & R \arrow[r] & A \arrow[r] & A / R \arrow[r] & 0
\end{tikzcd}
\end{center}
Since $A / R$ is finitely presented, it suffices to prove it is $R$-flat. By the local criterion for flatness we may take $(R, \m, \kappa)$ to be local and we just need to show that $\Tor{1}{R}{\kappa}{A/R} = 0$. We have a sequence,
\begin{center}
\begin{tikzcd}
0 \arrow[r] & \Tor{1}{R}{\kappa}{A} \to \Tor{1}{R}{\kappa}{A/R} \arrow[r] & \kappa \ot R \arrow[r] & \kappa \ot A \arrow[r] & \kappa \ot (A/R) 
\end{tikzcd}
\end{center}
but then $\kappa \ot R \to \kappa \ot A$ is injective because it is the inclusion of $\kappa$ into a $\kappa$-algebra and $\Tor{1}{R}{\kappa}{A} = 0$ since $A$ is $R$-flat giving the required vanishing.
\end{rmk}

There is also a morphism in the other direction,
\[ \delta : \Vect_{d-1} \to \FFlat_{d} \] 
given by sending,
\[ \E \mapsto \Sym{\struct{T}}{\E} / (\E^{\ot 2}) = \struct{T} \oplus \E \]
where the multiplication sends $\E \ot \E \to 0$. 

\begin{prop}
The morphisms $\nu$ and $\delta$ are $\A^1$-homotopy inverses. 
\end{prop}

\begin{proof}
We need to show that $\nu \circ \delta \simeq \id_{\Vect_{d-1}}$. Actually, they are canonically isomorphic because of the trivial sequence,
\begin{center}
\begin{tikzcd}
0 \arrow[r] & \struct{T} \arrow[r] & \struct{T} \oplus \E \arrow[r] & \E \arrow[r] & 0
\end{tikzcd}
\end{center}
The other direction $\delta \circ \nu \simeq \id_{\FFlat_d}$ follows from the Rees construction. We want a map,
\[ h : \A^1_S \times_S \FFlat_d \to \FFlat_d \]
such that $h(0, -) \cong \delta \circ \nu$ and $h(1, -) \cong \id_{\FFlat_d}$. 
\bigskip\\
A map $\A^1_S \times_S \FFlat_d \to \FFlat_d$ amounts to a finite flat algebra on $\A^1_S \times_S \FFlat_d$ or equivalently a finite flat $\struct{\A^1_S} \ot \struct{\FFlat_d}$-algebra,
\[ \cA_{\text{Rees}} = \bigoplus_{i = 0}^\infty \E_i t^i \quad \text{ where } \quad \E_i = 
\begin{cases}
\struct{\FFlat_d} & i = 0
\\
\cA_{\text{univ}} & i \ge 1
\end{cases} \]
where $\A^1_S = \rSpec{S}{\struct{S}[t]}$ and $\cA_{\text{univ}}$ is the universal flat algebra on $\FFlat$. Claim: this is a finite flat $\struct{T}[t] \ot \struct{\FFlat_d}$-algebra. The restriction to $\{ 1 \} \times \FFlat_d$ yields the map,
\[ \FFlat_d \to \FFlat_d \]
given by,
\[ \colim_i \E_i = \cA_{\text{univ}} \]
as a flat $\struct{\FFlat_d}$-algebra which thus gives the identity $\id : \FFlat_d \to \FFlat_d$. The restriction to $\{ 0 \} \times \FFlat_d$ yields the associated graded,
\[ \cA_{\text{Rees}} / (t) = \bigoplus_{i = 0}^{\infty} (\E_i / \E_{i-1}) \cong \struct{\FFlat_d} \oplus (\cA_{\text{univ}} / \struct{\FFlat_d}) \]
which by definition is the algebra corresponding to $\delta \circ \nu$. 
\end{proof}

\begin{prop}
Let $f : X \to Y$ be a morphism of presheaves on $\Aff_S$ satisfying the closed gluing condition. Then if for all algebras $\cA$ over $S$ and finitely generated ideals $\I \subset \cA$ and diagrams,
\begin{center}
\begin{tikzcd}
\Spec{A / I} \arrow[d] \arrow[r] & X \arrow[d]
\\
\Spec{A} \arrow[r] \arrow[ru, dashed] & Y
\end{tikzcd}
\end{center}
there is a lift up to homotopy meaning the morphism,
\[ \pi_0(X(\Spec{A})) \to \pi_0( X(\Spec{A/I}) \times_{Y(\Spec{A/I})} Y(\Spec{A})) \]
is surjective then $f$ is a universal $\A^1$-equivalence on $\Aff_S$. 
\end{prop}

\begin{rmk}
$\L_{\A^1}$ does not generally commute with base change so being a universal $\A^1$-equivalence is strictly stronger than being an $\A^1$-equivalence.
\end{rmk}

\begin{rmk}
Note that $\pi_0$ does not commute with homotopy pullbacks. Consider,
\begin{center}
\begin{tikzcd}
\Omega S^1 \arrow[r] \arrow[d] & * \arrow[d]
\\
* \arrow[r] & S^1
\end{tikzcd}
\end{center}
then $\pi_0(\Omega S^1) = \Z$ but $\pi_0(*) = 1$. 
\end{rmk}

\begin{defn}
A morphism $f : X \to Y$ satisfies the \textit{closed gluing condition} if for all schemes $T$ over $S$ and morphisms $T \to Y$ the fiber product $X \times_Y T$ satisfies
\begin{enumerate}
\item sends the empty scheme to a point
\item sends pushouts of closed immersions of schemes to pullbacks.
\end{enumerate}
\end{defn}

\begin{defn}
The infinite Hilbert scheme $\Hilb_d(\A^\infty_S)$ is te functor from $\Aff_S$ given by,
\[ (T \to S) \mapsto \colim_{n \to \infty} \Hilb_d(\A^n_S)(T \to S) \]
where the inclusions are induced by the sequence,
\[ \A^1 \embed \A^2 \embed \A^3 \embed \cdots \]
\end{defn}

\begin{rmk}
This functor classified finite flat finitely presented $S$-schemes of degree $d$ along with a closed embedding into affine space of $S$ of \textit{some} dimension. 
\end{rmk}

\begin{defn}
The funcor $\Hilb_d(\A^\N_S)$ is the functor that sends,
\[ (T \to S) \mapsto \{ Z \embed \rSpec{T}{\struct{T}[t_n]_{n \in \NN}} \} \]
the usual definition of the Hilbert functor of the actual scheme $\A_S^{\NN}$ without taking a colimit. 
\end{defn}

\begin{prop}
The forgetfull functors,
\begin{enumerate}
\item $\Hilb_d(\A_S^{\infty}) \to \FFlat_d$
\item $\Hilb_d(\A_S^{\NN}) \to \FFlat_d$
\end{enumerate}
are universal $\A^1$-equivalences. 
\end{prop}

\begin{proof}
First, these morphisms satisfy the closed gluing condition
\end{proof}

\section{Oct 28}

\newcommand{\Gr}{\mathrm{Gr}}

Recall we defined,
\begin{center}
\begin{tikzcd}
\Hilb_d(\A^\infty) \arrow[d, hook] \arrow[r] & \FFlat_d \arrow[r] & \Vect_{d-1} \arrow[from=r] & \Gr_{d-1}(\A^{\infty}
\\
\Hilb_d(\A^\N) \arrow[ru]
\end{tikzcd}
\end{center}
All of these maps will turn out to be $\A^1$-equivalences. 

\begin{defn}
A presheaf $\F \in \PSh(\Sch_S)$ satisfies the \textit{closed gluing condition} if,
\begin{enumerate}
\item $\F(\empty) \cong *$
\item for every pushout diagram,
\begin{center}
\begin{tikzcd}
Z \arrow[d, hook] \arrow[r, hook] & X \arrow[d, hook]
\\
Y \arrow[r, hook] & W
\end{tikzcd}
\end{center}
comprised of closed embeddings we have,
\[ \F(W) \cong \F(X) \times_{\F(Z)} \F(Y) \]
\end{enumerate}
\end{defn}

\begin{prop}
Let $\phi : Z \embed Y$ and $\psi : Z \embed X$ be closed immersion of schemes. Then the pushout $W = X \sqcup_{Z} Y$,
\begin{center}
\begin{tikzcd}
Z \arrow[r, "\phi", hook] \arrow[d, "\psi", hook] & Y \arrow[d, "b"]
\\
X \arrow[r, "a"] & W
\end{tikzcd}
\end{center} 
exists in the category of schemes and,
\begin{enumerate}
\item $a$ and $b$ are closed immersions
\item $\struct{W} = \ker{(a_* \struct{X} \oplus b_* \struct{Y} \to \iota_* \struct{Z})}$.
\end{enumerate}
\end{prop}

\newcommand{\RS}{\mathrm{RS}}

\begin{proof}
The pushout exists in the category of ringed spaces and has the above description with $|W| = |X| \sqcup_{|Z|} |Y|$ since then the ring structure is universal. Claim: if $W$ is a scheme then it is a pushout in $\Sch$. Then,
\begin{center}
\begin{tikzcd}
\Hom{\Sch}{X}{S} \times_{\Hom{\Sch}{Z}{S}} \arrow[d, hook] \Hom{\Sch}{Y}{S} \arrow[from=r] & \Hom{\Sch}{W}{S} \arrow[d, hook]
\\
\Hom{\RS}{X}{S} \times_{\Hom{\RS}{Z}{S}} \Hom{\RS}{Y}{S} \arrow[from=r, "\sim"'] & \Hom{\RS}{W}{S}
\end{tikzcd}
\end{center}
We want to show that the top map is an isomorphism. Via the diagram it is an injection. To check that the map is surjective we can work locally on $S$. Assume $S = \Spec{A}$ then get the diagram,
\begin{center}
\begin{tikzcd}
\Hom{}{A}{H^0(\struct{X})} \times_{\Hom{}{A}{H^0(\struct{Z})}} \arrow[d, hook] \Hom{}{A}{H^0(\struct{Z})} \arrow[from=r] & \Hom{}{A}{H^0(\struct{W})} \arrow[d, hook]
\\
\Hom{\RS}{X}{S} \times_{\Hom{\RS}{Z}{S}} \Hom{\RS}{Y}{S} \arrow[from=r, "\sim"'] & \Hom{\RS}{W}{S}
\end{tikzcd}
\end{center}
An element on the left is the same data as a map,
\begin{align*}
A \to H^0(\struct{X}) \times_{H^0(\struct{Z})} H^0(\struct{Y}) & = \ker{(H^0(\struct{X}) \oplus H^0(\struct{Y}) \to H^0(\struct{Z}))} 
\\
& = H^0(\ker{(a_* \struct{X} \oplus a_* \struct{Y} \to \iota_* \struct{Z})} = H^0(\struct{W})
\end{align*}
so the map is surjective. Now we need to show that $W$ is a scheme and that $a$ and $b$ are closed immersions. Away from $\iota(Z)$ this is clear because then we are taking disjoint union. Choose $z \in Z$ we need an open affine neighborhood of $\iota(z) \in W$. Claim, there are $U \subset X$ and $V \subset Y$ such that $\phi^{-1}(U) = \psi^{-1}(V)$ then $L = U \sqcup_{\phi^{-1}(U)} V$ is an open of $W$. Therefore, we reduce to the case that all $X,Y,Z$ are affine. Consider the pushout in the category of affine schemes,
\[ W_{\Aff} = \Spec{H^0(\struct{X}) \times_{H^0(\struct{Z})} H^0(\struct{Y})} \]
By the universal property, there is a map $W \to W_{\Aff}$. It suffices to show that this is a homeomorphism since then the rings argree by the definition of $W$. This is elementary commutative algebra [Schwede, Thm 3.4]. 
\end{proof}

\begin{lemma}
If $f : \F \to \G$ is a morphism in $\PSh(\Sch_S)$ such that $f$ satisfies the closed gluing condition (meaning the fibers over schemes do) and for any ring $A$ over $S$ and finitely generated ideal $I \subset A$ the morphism,
\[ \F(A) \to \F(A/I) \times_{\G(A/I)} \G(A) \]
is surjective on $\pi_0$. Then $f$ is a universal $\A^1$-equivalence on affine schemes.
\end{lemma}

\begin{rmk}
A morphism $f : \F \to \G$ is an $\A^1$-equivalent on affine schemes if $\L_{\A^1}(f)_{R}$ is an equivalence for any ring $R$. 
\end{rmk}

\begin{rmk}
$\L_{\A^1}$ does not preserve fiber products so being a universal $\A^1$-equivalent is important. For example consider a family $f : X \to \A^1$. Then the fibers $X_t$ are base changes of $f$ but $\L_{\A^1} (*) \times_{\L_{\A^1} \A^1} \L_{\A^1} X$ are all the same since $\L_{\A^1} \A^1 = *$. Therefore, $\L_{\A^1} (X_t)$ would all be equivalent. This is not true. Consider a map $f : \A^1 \to \A^1$ with different numbers of points in the fibers. Since a finite collection of points is $\A^1$-invariant and they are not equivalent as presheaves they are not $\A^1$-equivalent. 
\end{rmk}

\begin{proof}
We proved if $[Z = \Spec{B} \to \Spec{A}] \in \FFlat_d(A)$ and $Z_{A/I} \embed \A^n_{A/I}$ for some $n$ then there is a lift $Z \embed \A_{A}^{n+m}$. 
\bigskip\\
Let $Z_{A/I} \embed \A^{\N}_{A/I}$ equivalently a collection $(\bar{b}_i)_{i \in \N}$ in $B/IB$ such that,
\[ \Sym{\bullet}{\bigoplus_{i = 1}^\infty A b_i} \onto B / I B \]
First WLOG can take $\bar{b}_1, \dots, \bar{b}_r$ generate $B/I$ as an $A/I$-algebra. For every other $\bar{b}_i$ with $i > r$ we can write,
\[ \bar{b}_i = \bar{p}_i(\bar{b}_1, \dots, \bar{b}_r) \]
Lift to elements $b_i \in B$ for $i \le r$ and lift $p_i \in A[x_1, \dots, x_N]$ and make the definition. The problem is the lifted symmetric algebra may not surject. However, we do know,
\[ \Sym{\bullet}{\bigoplus_{i = 1}^r A b_i} \onto B / I B \]
is surjective. Take some $h_1, \dots, h_m$ generating $IB$ as an $A$-module. set,
\[ b_i = 
\begin{cases}
b_i & i \le r
\\
p_i(b_1, \dots, b_R) + h_{i-m} & r+1 \le i \le r+ m
\\
p_i(b_1, \dots, b_r) & i > m + r
\end{cases} \]
and we can check that this works. 
\bigskip\\
Closed gluing: consider the diagram for some test scheme $T$,
\begin{center}
\begin{tikzcd}
F_T \arrow[r] & \Hilb_d(\A^n) \arrow[d]
\\
T \arrow[r] & \FFlat_d
\end{tikzcd}
\end{center}
we need to show that $F_T : \Sch_T^\op \to \Set \to \Spc$ satisfies closed gluing. This is the functor,
\[ [W \to T] \mapsto \Hom{W\text{-closed}}{B_W}{A_W} \]
where $B \to T$ is the finite flat scheme defined by $T \to \FFlat_d$ and $A$ is the affine space. Thus we need to show that,
\[ \Hom{\text{cl}}{B_W}{A_W} \to \Hom{\text{cl}}{B_X}{A_X} \times_{\Hom{\text{cl}}{B_Z}{A_Z}} \Hom{\text{cl}}{B_Y}{A_Y} \]
is an isomorphism.
Notice that base change preserves the pushout squares of closed embeddings and thus if we remove the closed conditions the above is an isomorphism. Therefore, it suffices to show that closed embeddings on the right glue to a morphism $B_W \to A_W$ which is a closed embedding. Topologically, this is clear. Properness is clear, radicial can be checked using functor of points (DO THIS!!)
\end{proof}

\begin{defn}
$\Gr_d(\A^\infty) : \Sch_S^\op \to \Set \to \Spc$ is the functor,
\[ \Gr_D(\A^\infty) : (T \to S) \mapsto \colim_n \Gr_{n,d}(T) \]
\end{defn}

\begin{defn}
There is a forgetful functor,
\[ f : \Gr_d(\A^{\infty}) \to \Vect_d \]
given by sending $[\struct{T}^{\oplus n} \onto \F] \mapsto \F$.
\end{defn}

\begin{prop}
$f$ is an $\A^1$-equivalent.
\end{prop}

\begin{proof}
Using the lemma, we need to check closed gluing and lifts. For lifting, let $A$ be a ring over $S$ and fix a finitely generated ideal $I \subset A$. Let $\F = \wt{M}$ be a rank $d$ vector bundle on $\Spec{A}$. Want to show that given $(A/I)^{\oplus n} \onto M/IM$ we can lift to a surjection $A^{\oplus (n+m)} \onto M$. This surjection corresponds to elements $\bar{m}_i \in M / I M$. Lift these to $m_i \in M$ then the map $A^{\oplus n} \to M$ may not be surjective but taking generators $h_1, \dots, h_m \in IM$ the map $A^{\oplus (n + m)} \onto M$ is surjective.
\end{proof}

\section{Integral Crystalline-Tate Conjecture}

\begin{conj}[Tate]
Let $X$ be a smooth projective variety over $\CC$ and $z \in H^{2r}(X, \ZZ)$ and also Hodge $(Z \to H^{2r}_{\dR}(X)$ lies in $H^{r,r}(X)$ is an algebraic cycle.
\end{conj}

\begin{theorem}[Atiyah-Hirzebruch]
There exists $X$ smooth projective variety over $\CC$  and a class $z \in H^{2r}(X, \ZZ)$ is Hodge but not algebraic.
\end{theorem}

\begin{example}
$X \sim B((Z / p)^{\times 3})$ and $Z$ is $p$-torsion.
\end{example}

\begin{conj}
Let $X$ be smooth projective geometrically connected over $\FF_q$ and assume $z \in H^{2r}_{\text{crys}}(X/W)$ which is Tate then,
\[ \Im{(H^r_{\text{proet}}(X, W \Omega^r_{\log}) \to H^{2r}_{\text{crys}}(X/W))} \]
is algebraic.
\end{conj}

\begin{theorem}
The above conjecture is false. We can do this for $X \sim B \mu_p^{\times 3}$ with $p = \mathrm{char}(\FF_q)$.
\end{theorem}

\subsection{Atiyah-Hirzebruch counter examples}

\newcommand{\mot}{\text{mot}}

What A-H proved is that any cycle $z \in H^{2r}(X, \ZZ)$ which is algebraic the AHSS $H^\bullet \implies KU^\bullet$ then $\d_r(z) = 0$. These give obstructions to being algebraic. 

\begin{proof}
Consider motivic cohomology $H^i_{\mot}(X, \Z(j))$. Then it satisfies the following properties,
\begin{enumerate}
\item $H^{2j}_{\mot}(X, \Z(j)) \cong \CH^j(X)$
\item $H^p_{\mot}(X, \Z(q)) = 0$ for $p > 2 q$
\item there is a spectral sequence,
\[ E_2^{i,j} = H^{i-j}_{\mot}(X, \Z(-j)) \implies K_{-i-j}(X) \]
\end{enumerate}
There is a cycle class map giving a morphism from the above spectral sequence to the spectral sequence,
\[ H^{i-j}_{\text{sing}}(X^\an, \Z) \implies KU_{-i-j}(X^{\an}) \]
Then the differentials on the Chow part satisfy,
\[ \d_r^{j,-j} : H_{\mot}^{2j}(X, \Z(j)) = \CH^j(X) \to H_{\mot}^{j+r - (-j-r+1}(X, \Z(j+r-1)) = 0 \]
since $j+r - (-j-r+1) = 2j+2r - 1$ is bigger than $2(j + r - 1)$. Therefore, using the commutativity the topological $\d_r$ must also vanish for any algebraic cycle. 
\end{proof}

We want to understand $\d_r : H^\bullet \to H^\bullet$. Baby case $M \in D(\Lambda)$ we get a sequence,
\[ H^{a-1}(M)[-q+1] \to \tau^{[q-1,q]} M \to H^q(M)[-q] \xrightarrow{\delta} H^{q-1}(M)[-q+2]  \]
For simplicitly, weork with mod $p = 2$ coefficients. To provide counter example suffices to produce a class,
\[ z \in H^{2r}(X, \FF_p) \]
and $z$ lifts to $\ZZ$ iff $\beta(z) = 0$ (bokstein) not in the image of $\CH^r(X) / p$. In topology,
\[ (\Z/2)[2] \to \tau_{[2,0]} KU/2 \to \Z/2 \xrightarrow{Q_i} \Z/2[3] \]
where the maps $Q_i$ are Milnor's operations:
\begin{align*}
Q_0 & = \beta
\\
Q_1 = \beta \sq^2 - \sq^2 \beta 
\\
\vdots
\end{align*}
Lesson learned: cohomology operations are,
\begin{enumerate}
\item Steenrod operations
\item instructions to build cohomology theories
\item formal group laws.
\end{enumerate}

Let's explain the formal group laws:
\[ E^*(\CP^\infty) = E^*(pt)[[x]] \]
and then the multiplication $\CP^\infty \times \CP^\infty \to \CP^\infty$ gives the structure of a formal group law. Somehow,
\[ Q_i \iff K(j) \]
Morava's $K$-theory

\begin{prop}
For any cycle $z \in H^{2r}(X, \Z/2)$ which is algebraic then $Q_j z = 0$ for all $j \ge 0$.
\end{prop}

\begin{proof}
Consider the diagram,
\begin{center}
\begin{tikzcd}
H^{2r,r}_{\mot}(X, \Z/2) \arrow[r] \arrow[d] & H^{2r}(X, \Z/2) \arrow[d]
\\
0 \arrow[r] & H^{2r + 2, 2r-1}(X, \Z/2)
\end{tikzcd}
\end{center}

\subsection{Characteristic $p$}

For $X$ a smooth projective scheme over $\FF_q$ we have $H^r_{\text{crys}}(X/W)$ with $W = W(\FF_q)$ is a $W$-module. This is a good Weil cohomology theory in the absence of a lift of $X$ to characteristic zero.

Construction of cycle class maps: a theorem of Geisser and LEvine:
\[ \CH^r(X) / p^n = H^{2r}_{\mot}(X, \Z/p^n(r)) = H^r_{\text{Zar}}(X, W_n \Omega^r_{\log}) \]
where $W_n \Omega^r_{\log} / p = \Omega^r_{\log}$ is the abelian subsheaf of $\Omega^r$ generated by $\dlog{f_i}$ for units $f_i$.
\end{proof}


\begin{rmk}
$H^r_{\et}(X, W \Omega^r_{\log})[p^{-1}] \iso H^{2r}_{\text{crys}}(X/W)[p^{-1}]^{\varphi_X = p^r}$
\end{rmk}

\begin{defn}
$X$ is \textit{integrally crys-Tate} in degree $r$ level $n$ if the cycle class map surjective on to the image of,
\[ H^r_{\et}(X, W_n \Omega^r_{\log}) \to H^{2r}_{\text{crys}}(X/W_n) \]
\end{defn}

\newcommand{\Tate}{\mathrm{Tate}}

\begin{conj}[Tate]
$\CH^r(X)_{\Q_p} \onto H^{2r}_{\text{crys}}(X/W)[p^{-1}]^{\varphi_X = p^r}$
\end{conj}

There is also of course the $\ell$-adic Tate conjecture for Galois-fixed points of $\ell$-adic cohomology. 

\begin{prop}
$\Tate^1_\ell \iff \Tate^1_p$
\end{prop}

\begin{theorem}[de Jong, Marrow]
$\Tate^1_{\ell}(X)$ for $X$ a surface implies $\Tate^r_\ell(X)$ for all $X$ and all $r$. 
\end{theorem}


\section{Nov 4}

\newcommand{\mrk}{\mathrm{mrk}}
\newcommand{\tnu}{\mathrm{nu}}

\begin{defn}
\[ \FFlat^{\text{mrk}} : X \to \{ (f, s) \mid f : Z \to X \text{ and } s : X \to Z \text{ section} \} \]
and similarly,
\[ \FFlat^{\text{nu}} = \text{ fflat but non-unital} \]
There are maps between these given by unitalization and taking the augmentation ideal $\ker{(\struct{Z} \to \struct{X})}$.
\end{defn}

\begin{prop}
$\nu$ is an $\A^1$-equivalence.
\end{prop}

\begin{proof}
$\pi \circ \nu = \id$. Define $\FFlat^{\tnu} \to \Hom{}{\A^1}{\FFlat^{\tnu}}$ sending $A \mapsto t A[t]$ 
\end{proof}

\begin{lemma}
$\theta$ is an $\A^1$-equivalence.
\end{lemma}

\begin{proof}
Consider the diagram,
\begin{center}
\begin{tikzcd}
\Vect \arrow[r, "\alpha"] \arrow[d, "\nu"] & \FFlat_{\ge 1}
\\
\FFlat^{\tnu} \arrow[r, "\sim"] & \FFlat^{\mrk} \arrow[u, "\theta"]
\end{tikzcd}
\end{center}
The 2-out-of-3 after $\L_{\A^1}$ shows that $\theta$ is an $\A^1$-equivalent. 
\end{proof}

\begin{rmk}
Both the map $\alpha$ from last time and the bottom unitalization equivalent increase the degree by $1$. 
\end{rmk}

\begin{lemma}
\begin{center}
\begin{tikzcd}
Z \arrow[r, "\iota"] \arrow[d, "j"] & X
\\
Y
\end{tikzcd}
\end{center}
$Z, Y, X$ finite local free $S$-schemes then $X \sqcup_Z Y$ exists and is finite locally free over $S$.
\end{lemma}

\begin{proof}
Pushout exists by Stacks OE25 need to check if $M \onto P$ and $N \onto P$ are maps of finite loc free modules then $M \times_P N$ is finite locally free. From the sequence,
\begin{center}
\begin{tikzcd}
0 \arrow[r] & N \arrow[r] & M \times_P N \arrow[r] & \ker{(M \onto P)} \arrow[r] & 0
\end{tikzcd}
\end{center}
this follows by splitting since these are projective. 
\end{proof}

\begin{theorem}
$\eta^{\st}$ is an $\A^1$-equivalence. 
\end{theorem}

\begin{proof}
Consider the following schemes,
\begin{align*}
R &= \Spec{\Z[x,t]/((x - t) x)}
\\
R_0 &= \Spec{\Z[x]/(x^2)}
\\
R_1 &= \Spec{\Z} \sqcup \Spec{\Z} 
\end{align*}
Then we note,
\[ \epsilon : (f : Z \to X, s : X \to Z) \mapsto Z \sqcup_S (R_0)_S \]
define,
\[ H : \FFlat^{\mrk} \to \Hom{}{\A^1}{\FFlat^{\mrk}} \]
as follows,
\[ H : (f, s) \mapsto \A^1_Z \sqcup_{\A^1_S} R_S \]
Then $H_0 = \epsilon$ and $H_1 = \sigma^{\mrk}$. 
\end{proof}

\subsection{The Group Structures}

$\oplus \otimes$ on $\Vect$ and $\sqcup, \times$ on $\FFlat$ give $E_{\infty}$-semiring structures.

\begin{theorem}
$\eta^{\gp} : \FFlat^{\gp} \to \Vect^{\gp}$ is an $\A^1$-equivalence.
\end{theorem}


%\begin{proof}
%Denote inverting $\struct{}$ by $[-1]$. This gives a diagram,
%\begin{center}
%\begin{tikzcd}
%\FFlat^{\st} \arrow[r] \arrow[d, "\eta^{\st}"] & \FFlat[-1] \arrow[d, "\eta[-1]"]
%\\
%\Vect^{\st} \arrow[r] & \Vect[-1]
%\end{tikzcd}
%\end{center}
%Then $\L_{\A^1}$ commutes with group completion so we conclude.
%\end{proof}

\begin{rmk}
Now we prove the lemma that Andres keeps skipping. 
\end{rmk}

\begin{lemma}[4.1]
Let $f : \F \to \G$ be a map of presheaves on $\Sch_S$ such that the fibers ofver any scheme $T \to S$ satisfies closed gluing. And suppose that for anny affine $\Spec{A}$ with finite generated ideal $I \subset A$ the map,
\[ \F(A) \to \F(A/I) \times_{\G(A/I)} \G(A) \]
is surjective on $\pi_0$. Then $f$ is a universal $\A^1$-equivalence on affine schemes.
\end{lemma}

\begin{proof}
It suffices to consider $\G = *$ for some reason (WHY).
Recall $\Delta^n_A = \Spec{A[t_0, \dots, t_n]/( t_0 + \cdots + t_n - 1)}$ and,
\[ \L_{\A^1} \F(\Spec{A}) = \colim \F(\Delta_A^n) \]
We need to show that this is contractible. Then closed gluing implies,
\[ \Hom{}{\partial \Delta^n_A}{\F(\Delta_A^\bullet)} \cong \F(\partial \Delta_A^n) \]
where $\F(\Delta_A^\bullet)$ is the simplicial space $[n] \mapsto \F(\Delta_A^n)$. Then,
\[ \Hom{}{\Delta^n}{\F(\Delta^\bullet_A)} \to \Hom{}{\partial \Delta_A^n}{\F(\Delta_A^\bullet)} \]
is surjective on $\pi_0$ because it is the same as,
\[ \F(\Delta_A^n) \to \F(\partial \Delta_A^n) \]
which is induced by the closed immersion $\partial \Delta^n_A \embed \Delta^n_A$ and thus we apply the assumption. Therefore,
\[ | \F(\Delta_A^\bullet) | = \colim_n \F(\Delta_A^n) \] 
is contractible (WHY).
\end{proof}


\section{My Talk}

\newcommand{\Aconn}[1]{\A^1\text{-}(#1)\text{-connected}}
\newcommand{\cV}{\mathcal{V}}

\begin{theorem}
Let $n \ge d \ge 0$ and $k$ a field. The morphism,
\[ \Sigma \Hilb_d(\A^n_k) \to \Sigma \Hilb_d(\A^\infty_k) \]
induced by the inclusion is $\Aconn{n-d+1}$.
\end{theorem}

\subsection{Connectivity}

\begin{defn}
We say a space $X$ is $n$-connected if $\pi_i(X, x) = 0$ for $i \le n$ and $x \in X$.
\end{defn}

\begin{defn}
We say a morphism $f : X \to Y$ is $n$-connected if for each $x \in X$,
\begin{enumerate}
\item $f_* : \pi_i(X, x) \to \pi_i(Y, f(x))$ is an isomorphism for $i \le n$
\item $f_* : \pi_i(X, x) \to \pi_i(Y, f(x))$ is an epimorphism for $i = n+1$.
\end{enumerate}
\end{defn}

\begin{rmk}
From the Puppe sequence, a morphism $f : X \to Y$ is $n$-connected iff the homotopy fiber $F_f$ is $n$-connected. 
\end{rmk}

\begin{defn}
Let $\F \in \PSh(\Sm_S)$ be a presheaf. We define the homotopy groups $\pi_i(F)$ as the Nisnevich sheafification of the presheaf,
\[ U \mapsto \pi_i(\F(U)) \]
\end{defn}

\begin{rmk}
Strictly we should consider pointed presheaves $* \to \F$.
\end{rmk}

\begin{defn}
We say a presheaf $\F \in \PSh(\Sm_S)$ is $n$-connected if $\pi_i(\F) = 0$ for $i \le n$.
\end{defn}

\begin{defn}
We say a morphism $f : \F \to \G$ in $\PSh(\Sm_S)$ is $n$-connected if
\begin{enumerate}
\item $f_* : \pi_i(\F) \to \pi_i(\G)$ is an isomorphism for $i \le n$
\item $f_* : \pi_i(\F) \to \pi_i(\G)$ is an epimorphism for $i = n+1$.
\end{enumerate} 
\end{defn}

\begin{defn}
A presheaf $\F \in \PSh(\Sm_S)$ is $\Aconn{n}$ if $\L_{\mot}(\F)$ is $n$-connected. A morphism $f : \F \to \G$ in $\PSh(\Sm_S)$ is $\Aconn{n}$ if $\L_{\mot}(f)$ is $n$-connected. 
\end{defn}

\begin{rmk}
Equivalently, we can say a morphism $f : \F \to \G$ is $n$-connected if on each Nisevich stalk $f_x : \F_x \to \G_x$ is $n$-connected in the ordinary sense. Again $f$ is $\Aconn{n}$ if $\L_{\mot}(f)$ is $n$-connected in this sense. 
\end{rmk}

\begin{rmk}
We can furthermore define,
\[ \pi_i^{\A^1}(\X) := \pi_i(\L_{\mot} \X) \]
and then $\A^1$-connectivity of a space or of a map corresponds to the expected definition in terms of $\pi_i^{\A^1}$.
\end{rmk}

\begin{defn}
Recall that we define the loop space and suspension of a space or presheaf as the pullback (pushout),
\begin{center}
\begin{tikzcd}
\Omega X \pullback \arrow[d] \arrow[r] & * \arrow[d]
\\
* \arrow[r] & X
\end{tikzcd}
\quad \quad 
\begin{tikzcd}
X \arrow[r] \arrow[d] & * \arrow[d]
\\
* \arrow[r] & \Sigma X
\end{tikzcd}
\end{center}
\end{defn}

\begin{rmk}
Since $\L_{\mot}$ preserves homotopy colimits $\L_{\mot}(\Sigma \X) = \Sigma \L_{\mot}(\X)$. In particular, if $\X$ is motivic then $\Sigma \X$ is also motivic. However, $\L_{\mot}$ and $\Omega$ do not generally commute. This means although $\pi_i(\Omega \X) = \pi_{i+1}(\X)$ is immediate from the definitions, $\pi_i^{\A^1}(\Omega \X) \neq \pi_{i+1}^{\A^1}(\X)$ in general. However, if $\X$ is motivic then $\Omega \X$ is also motivic meaning that if $\X$ is motivic then,
\[ \pi_i^{\A^1}(\Omega \X) = \pi_{i+1}^{\A^1}(\X) \]
from the result from the unstabilized $\pi_i$.
\end{rmk}

\begin{lemma}
If $\X \in \PSh(\Sm_S)$ is motivic then $\Omega \X$ is motivic.
\end{lemma}

\begin{proof}
Since $\Omega \X$ is a sheaf (limits in sheaves and presheaves agree) it suffices to show that $\Omega \X$ is $\A^1$-invariant meaning,
\[ \Hom{}{Y \times \A^1}{\Omega \X} = \Hom{}{Y \times \A^1}{\Omega \X} \]
Indeed,
\begin{align*}
\Hom{}{Y \times \A^1}{\Omega \X} &= \Hom{}{\Sigma(Y \times \A^1)}{\X} = \Hom{}{\L_{\mot} \Sigma(Y \times \A^1)}{\X}
\end{align*}
but $\L_{\mot} \Sigma = \Sigma \L_{\mot}$ and hence $\L_{\mot} \Sigma(Y \times \A^1) = \Sigma \L_{\mot} (Y \times \A^1) = \Sigma \L_{\mot} Y$ and hence,
\[ \Hom{}{Y \times \A^1}{\Omega \X} = \Hom{}{\Sigma \L_{\mot} Y}{\X} = \Hom{}{\L_{\mot} Y}{\Omega \X} = \Hom{}{Y}{\Omega \X} \]
\end{proof}

\begin{lemma}[8.8]
Let $k$ be a perfect field, $f : Y \to X$ a morphism in $\PSh(\Sm_k)$ and $n \ge -1$. If $f$ is $\Aconn{n}$ then $\cofib(f)$ is $\Aconn{n+1}$. The coverse holds if $X$ and $Y$ are $\Aconn{1}$.
\end{lemma}


\begin{proof}
Since $\L_{\mot}$ is a left-adjoint, it preserves homotopy colimits. Thus $\L_{\mot}(\cofib f) = \cofib(\L_{\mot} f)$ and hence we can assume that $X, Y$ are motivic. Then it reduces to the corresponding lemma for ordinary spaces 
\end{proof}

{\color{red} We have to be careful: $\L_{\mot}(\cofib f) = \cofib(\L_{\mot} f)$ but the second is computed in motivic spaces which is not the same as the colimit computed on sheaves. In particular, its stalks are not the cofibers of the maps on stalks so we cannot direct apply topological arguments to it. Instead we have to use the connectivity theorem.}

\subsection{Connectivity Theorem}

(FIX THIS COMPLETELY)

\newcommand{\Y}{\mathcal{Y}}
\newcommand{\V}{\mathbb{V}}

\begin{theorem}[Mor12, 1.18]
Let $\X$ be a pointed presheaf and $n \ge 0$ an integer. If $\X$ is $n$-connected then it is $\Aconn{n}$ meaning $\pi_i^{\A^1}(\X) = \pi_i(\L_{\mot}(\X)) = 0$ for $i \le n$.
\end{theorem}

\begin{theorem}[Mor12, 6.56]
Let $f : \X \to \Y$ be a morphism with $\Y$ pointed and $0$-connected. Assume that the sheaf of groups $\pi_0^{\A^1}(\Omega^1(\Y)) = \pi_0(\L_{\mot}(\Omega^1(\Y)))$ is strongly $\A^1$-invariant (WHAT DOES IT MEAN). Let $n \ge 1$ be an integer and assume $f$ is $(n-1)$-connected, then $\L_{\mot}(f)$ is also $(n-1)$-connected. 
\end{theorem}

\begin{rmk}
Notice that $\L_{\A^1}$ may not preserve $\Omega^1$ because $\L_{\A^1}$ is a left adjoint but not a right adjoint. Therefore $\pi_i^{\A^1}(\Omega(\Y))$ need not equal $\pi_{i+1}(\Y)$.
\end{rmk}

\subsection{Purity}

\newcommand{\Th}{\mathrm{Th}}
\newcommand{\cN}{\mathcal{N}}

\begin{defn}
The Thom space of a vector bundle $\E$ over $X$ is,
\[ \Th(\E) = \V(\E)  / (\V(\E) \sm \iota(X)) \]
where $\V(\E)$ is the total space and $\iota : X \to \V(\E)$ is the zero section.
\end{defn}

\begin{lemma}
Consider the natural closed embedding $\P_X(\E) \to \P_X(\E \oplus \struct{X})$. Then the canonical morphism of pointed sheaves,
\[ \P(\E \oplus \struct{X}) / \P(\E) \to \Th(\E) \]
is an $\A^1$-equivalence. 
\end{lemma}

\begin{proof}
Consider the open covering, $\P(\E \oplus \struct{X}) = \V_X(\E) \cup (\P_X(\E \oplus \struct{X}) \sm X)$ which gives a gluing diagram,
\begin{center}
\begin{tikzcd}
\V_X(\E) \arrow[r] & \P_X(\E \oplus \struct{X})
\\
\V_X(\E) \sm X \arrow[u] \arrow[r] & \P_X(\E \oplus \struct{X}) \sm X \arrow[u]
\end{tikzcd}
\end{center}
and hence a homotopy pushout. Therefore, we complete it to, 
\begin{center}
\begin{tikzcd}
\V_X(\E) \arrow[r] & \P_X(\E \oplus \struct{X}) \arrow[r] & \Th(\E) 
\\
\V_X(\E) \sm X \arrow[u] \arrow[r] & \P_X(\E \oplus \struct{X}) \sm X \arrow[u] \arrow[r] & * \arrow[u]
\end{tikzcd}
\end{center}
where each square is a homotopy pushout so,
\[ \Th(\E) = \frac{\P_X(\E \oplus \struct{X})}{\P_X(\E \oplus \struct{X}) \sm X} \]
Therefore, since the map $\P_X(\E) \to \P_X(\E \oplus \struct{X}) \to \Th(\E)$ factors through $\P_X(\E \oplus \struct{X}) \sm X$ giving a map,
\[ \P(\E \oplus \struct{X}) / \P(\E) \to \Th(\E) \]
It suffices to show that $\P(\E) \to \P(\E \oplus \struct{X}) \sm X$ is an $\A^1$-equivalence. Indeed, it is the embedding of $\P(\E)$ as the zero section of the total space of the tautological bundle on $\P(\E)$ so it is a vector bundle over $\P_X(\E)$ and hence an $\A^1$-equivalence.  
\end{proof}

\begin{lemma}
If $Z \subset X$ is a closed subscheme and $Z, X$ are smooth $k$-schemes then,
\[ X / (X \sm Z) \cong_{\mot} \Th( \cN_{Z | X} )\]
\end{lemma}

\begin{proof}
DO THIS ONE!!!
\end{proof}

\begin{example}
If $Z = \{ p \}$ is a point then this shows that,
\[ X / (X \sm \{ p \}) \cong_{\mot} \A^d / (\A^d \sm \{ 0 \}) \]
where $d = \dim{X}$. 
\end{example}


\begin{lemma}
Let $\E$ be a vector bundle of rank $r$. Then $\Sigma \Th(\E)$ is $\Aconn{r}$. 
\end{lemma}

\begin{proof}
DO THIS!!!
\end{proof}

\begin{rmk}
If $U \to X$ is an open immersion of schemes then $\cofib(U \to X) = X / U$. Indeed this holds more generally ... (DO THIS).
\end{rmk}

\begin{lemma}[8.9]
Let $k$ be a perfect field, $X$ a smooth $k$-scheme, and $Z \subset X$ a closed subscheem of codimension $\ge r$. Then $\Sigma(X/(X \sm Z))$ is $\Aconn{r}$. 
\end{lemma}

\begin{proof}
$X$ is a disjoint union of quasi-compact smooth schemes so we may assume that $X$ is quasi-compact. If $Z$ is smooth then $X / (X \sm Z)$ is $\Aconn{r-1}$ by purity.
\bigskip\\
Notice that I can assume $Z$ is smooth because the sheaf represented by $Z$ and $Z_{\red}$ on $\Sm_k$ are the same since a morphism from a reduced scheme factors through the reduction.
\bigskip\\
Since $k$ is perfect, generic smoothness gives a filtration,
\[ \empty = Z_0 \subsetneq Z_1 \subsetneq \cdots \subsetneq Z_n = Z \]
of closed subschemes such that $Z_j \sm Z_{j-1}$ is smooth. We prove the result by induction on $n$. Consider the cofiber sequence,
\[ \frac{X \sm Z_{n-1}}{X \sm Z} \to \frac{X}{X \sm Z} \to \frac{X}{X \sm Z_{n-1}} \]
Therefore because homotopy colimits commute we get a cofiber sequence,
\[ \Sigma \left( \frac{X \sm Z_{n-1}}{X \sm Z} \right) \to \Sigma \left( \frac{X}{X \sm Z} \right) \to \Sigma \left( \frac{X}{X \sm Z_{n-1}} \right) \]
by the induction hypothesis $X / (X \sm Z_{n-1})$ is $\Aconn{r}$ (indeed it should be $\Aconn{r+1}$ because $\codim{Z_{n-1}, X} \ge r + 1$ but we don't need this) and hence by Lemma 8.8 the morphism,
\[ \Sigma \left( \frac{X \sm Z_{n-1}}{X \sm Z} \right) \to \Sigma \left( \frac{X}{X \sm Z} \right)  \]
is $\Aconn{r-1}$. Since, $X \sm Z_{n-1}$ and $X \sm Z$ are smooth, by the smooth case we know that $\Sigma ((X \sm Z_{n-1})/(X \sm Z))$ is $\Aconn{r}$ and hence $\Sigma( X / (X \sm Z))$ is $\Aconn{r}$.
\end{proof}

\subsection{Lemmas}

\newcommand{\cZ}{\mathcal{Z}}
\newcommand{\Rees}{\mathrm{Rees}}

\begin{defn}
Let $\cZ_d \to \FFlat_d$ be the universal finite locally free scheme of degree $d$ and let $\cV_d(\A^n)$ be the vector bundle over $\FFlat_d$ defined by,
\[ \cV_d(\A^n) = \Hom{\FFlat_d}{\cZ_d}{\A^n_{\FFlat_d}} \]
Then, $\Hilb_d(\A^n) \embed \cV_d(\A^n)$ is the open substack of closed immersions $\cZ_d \to \A^n_{\FFlat_d}$. 
\end{defn}

\begin{rmk}
Since $\cZ_d \to \FFlat_d$ and $\A_d^n \to \FFlat_d$ are both relative schemes $\Hom{\FFlat_d}{\cZ_d}{\A^n_{\FFlat_d}}$ is automatically a stack. 
\end{rmk}

\begin{lemma}
The closed complement of $\Hilb_d(\A^n)$ in $\cV_d(\A^n)$ has codimension at least $n-d+2$ in every fiber over $\FFlat_d$. 
\end{lemma}

\begin{proof}
Let $k$ be a field and $S = \Spec{R}$ for a finite $k$-scheme of degree $d$. We need to show that the closed cimplement of $\mathrm{Emb}_k(S, \A^n_k)$ in $\Hom{k}{S}{\A^n_k} \cong \A^{nd}_k$ has codimension at least $n - d + 2$. A $k$-morphism $S \to \A^n_k$ can be viewed as a $k$-algebra map,
\[ k[x_1, \dots, x_n] \to R \]
which is an embedding iff it is surjective. It suffices for the images of $x_1, \dots, x_n$ to generate $R / (k \cdot 1)$ as a $k$-module. This is the open set characterized by where a map $k^n \to k^{r}$ (with $r = d-1$) is surjective whose codimension (in the space of matrices) is $n - r + 1 = n - d + 2$. Indeed, being nonsurjective is equivalent to having its image lie in some hyperplane $H \in \P^{r-1}$ so has dimension \[ n(r-1) + (r-1) = nr - (n - r + 1) \] corresponding to a surjective map $k^n \to H$ and a choice of $H$.
\end{proof}

\begin{rmk}
This bound is optimal when $n \ge d - 2$. The worse case is for the square-zero extension algebra,
\[ R = k[x_1, \dots, x_{d-1}]/(x_i x_j) \]
in which case a map $k[x_1, \dots, x_n] \to R$ is surjective iff $x_1, \dots, x_n$ generate $R / (k \cdot 1)$ as a $k$-module. Therefore, the non-surjective maps $k[x_1, \dots, x_n] \to R$ have codimension exactly $n - d + 2$. 
\end{rmk}

\begin{rmk}
For our purposes we say that a subpresheaf $U \subset X$ is \textit{open} if the map $U \to X$ is representable by schemes and for each scheme $T \to X$ we have that $U \times_X T \to T$ is an open immersion of schemes.
\end{rmk}

\begin{prop}[8.10]
Let $k$ be a perfect field and $X \in \PSh(\Sm_k)$ a presheaf. Let $V \to X$ a vector bundle, and $r \ge 0$. Let $U \subset V$ be an open subpresheaf such that, for each finite field extension $k'/k$ and $\alpha \in X(k')$ the closed complement of $\alpha^*(U)$ in $\alpha^*(V)$ has codimension at least $r$. Then, $\Sigma^2 U \to \Sigma^2 X$ is $\Aconn{r}$. If $U$ and $X$ are $\A^1$-connected, then the morphism $\Sigma U \to \Sigma X$ is $\Aconn{r-1}$.
\end{prop}

\begin{proof}
By Lemma 8.8, it suffices to show that $\Sigma \cofib(U \to X)$ is $\Aconn{r}$. Colimits of pointed objects preserve connectivity (WHY), hence $\A^1$-connectivity by the $\A^1$-connectivity theorem, so we are reduced by universality of colimits to the case $X \in \Sm_k$ since every object is a colimit of representables. Since $V \to X$ is an $\A^1$-equivalence (because it is a vector bundle) and $V \sm U$ has codimension $\ge r$ in $V$ the result follows from Lemma 8.9. 
\end{proof}

\begin{lemma}
Let $X \in \PSh(\Sm_S)$ 
\end{lemma}

\begin{lemma}[8.11]
Let $k$ be a field and $n \ge d - 1 \ge 0$. Then $\Hilb_d(\A^n_k)$ is $\A^1$-connected.
\end{lemma}

\begin{proof}
It suffices to show that $\Hilb_d(\A^n)(k)$ is nonempty and that points are connected by $\A^1$-curves. Explicitly, for every separable finitely generated field extension $F / k$ we say that $x, y \in \Hilb_d(\A^n)(F)$ are connected if there is a map $\A^1_F \to \Hilb_d(\A^n)_F$ such that $0 \mapsto x$ and $1 \mapsto y$. We need to show that all $F$-points are connected in a chain menaing equivalent by the equivalence relation generated by being connected. By some results of Morel this proves $\A^1$-connectedness (it certainlly does in a moral sense). 
\bigskip\\
For $\A^n_F = \Spec{F[x_1, \dots, x_n]}$. Consider an $F$-point $[A] \in \Hilb_d(\A^n)(F)$ corresponding to a surjection $\pi : F[x_1, \dots, x_n] \onto A$. We claim that $\pi$ can be connected by a chain to a surjection such that the images of $1, x_1, \dots, x_{d-1}$ are linearly independent. Indeed, otherwise there is some $x_i$ in the span of $1, x_1, \dots, x_{i-1}$ then,
\[ \pi|_i : F[x_1, \dots, x_{i-1}, x_{i+1}, \dots, x_n] \onto A \]
is a surjection so we consider the $F[t]$-algebra map $\rho : F[x_1, \dots, x_n, t] \to A[t]$ defined by $\rho(x_j) = \pi(x_j)$ for $j \neq i$ and $\rho(x_i) = t a + (1 - t) \pi(x_i)$ where $a \in A$ is some element not in the span of $1, x_1, \dots, x_{i-1}$. This defines a homotopy between $\pi$ and $\pi' = \rho(1)$ where the images of $1, x_1, \dots, x_{i}$ are linearly independent. Since $\dim_F(A) = d$ this will make $1, x_1, \dots, x_{d-1}$ span $A$ and then we use another homotopy to set $\pi(x_{j}) = 0$ for $j \ge d$ just by scaling $\pi(x_j)$ by $t$.
\bigskip\\
Now we use the Rees algebra,
\[ \Rees(F) := F \oplus t A[t] \subset A[t] \]
which is finithe locally free over $F[t]$. Define an $F[t]$-algebra map $\tilde{\pi} : F[x_1, \dots, x_n, t] \to \Rees(F)$ by $\tilde{\pi}(x_i) = t \pi(x_i)$ for all $i$. Since we can assume $1, x_1, \dots, x_{d-1}$ span $A$ as an $F$-module, the image of $\tilde{\pi}$ contains $t$ and $t A$ and so is surjective. Geometrically $\tilde{\pi}$ corresponds to the morphism 
\[ \tilde{\pi} : \A^1_F \to \Hilb_d(\A^n) \]
that links $[A] = \tilde{\pi}(1)$ with $\tilde{\pi}(0)$ and $\Rees(F)/(t) = F \oplus F^{d-1}$ is the square-zero extension equppied with the canonical map $\tilde{\pi}(0) : F[x_1, \dots, x_n] \to \Rees(F)/(t)$ given by sending $x_1, \dots, x_{d-1}$ to the standard basis of $F^{d-1}$ and $x_j \mapsto 0$ for $j \ge d$. This is independent of $[A]$ so we conclude that all points are chain connected.
\end{proof}


\subsection{Proof of the Main Theorem}

Since the coclusions are preserved by aessentially smooth base change (WHAT DOES THIS MEAN) we may replace $k$ be a perfect subfield. 
\bigskip\\
Recall that the forgetful map $\Hilb_d(\A^{\infty}) \to \FFlat_d$ is a motivic equivalence. Note that both $\Hilb_d(\A^n)$ and $\FFlat_d$ are $\A^1$-connected by Lemma 8.11. Then we apply Prop 8.10 to the inclusion, $\Hilb_d(\A^n) \embed \cV_d(\A^n) \to \FFlat_d$ to conclude that $\Sigma \Hilb_d(\A^n) \to \Sigma \FFlat_d$ is $\Aconn{n-d-1}$ since the codimension of the complement inside $\cV_d(\A^n)$ is $n - d + 2$.


\begin{rmk}
Note that the assumptions of Prop 8.7 and 8.10 are preserved by any base change $X' \to X$. For example, by considering the open substack $\mathrm{FSyn}_d \subset \FFlat_d$ of syntomic finite flat covers, we see that Theorem 8.2 holds with $\Hilb_d$ replaced by $\Hilb_d^{\text{lci}}$.
\end{rmk}

(REMARKS, BLAKERS-MASSY DO BETTER, REWRITE LAST TWO PAGES)

\section{Dec 2}


Let $\E = \Sh(\C)$ sheaves of spaces on a site (not necessarily hypersheaves) we want to define the homotopy sheaves $\pi_n(X)$ of $X \in \Sh(\C)$. We have some options,
\begin{enumerate}
\item define the homotopy presheaves $\pi_n \circ X$ sending $U \mapsto \pi_n(X(U))$ and then sheafify
\item have a inclusion $\Sh_{\Set}(\C) \embed \E$ of discrete sheaves and $\pi_0$ is a left adjoint. If $X$ is a pointed sheaf then we can define $\pi_n(X) = \pi_0(\Omega^n X)$
\end{enumerate}

These agree because they agree on $\pi_0$ and then $\Omega$ behaves well with sheafification. 

\begin{defn}
a pointed sheaf $X$ is $n$-connected if $\pi_k(X) = 0$ for $k \le n$.
\end{defn}

\begin{rmk}
The condition $\pi_k(X) = 0$ means, for any $c \in \C$ and map $\alpha : S^k \to X(c)$ there a cover $\{ c_i \to c \}$ such that $S^k \to X(c) \to X(c_i)$ are nullhomotopic. 
\end{rmk}

\begin{rmk}
The problem is that $X$ may not have very many points since points correspond to global sections. Therefore we need a better notion. If the topos has enough points then we can check connectedness stalkwise. We need to choose a base point of each stalk but this is much more general than choosing a global point.
\end{rmk}

\begin{rmk}
Sheaves are not determined by maps from a point (unlike spaces). Therefore, to talk about $n$-connectedness of an unpointed sheaf, we don't expect it to suffice to check all basepoints. 
\end{rmk}

We have some solutions,
\begin{enumerate}
\item stalks
\item if $X$ is connected then $\pi_n(X, x) = 0 \iff \pi_n(X) = 0$ the second being Lurie's version but still its not clear that $X$ has any global point.
\end{enumerate}

\begin{rmk}
The reason for the second is that if $X$ is connected then $* \to X$ is an effective epimorphism. 
\end{rmk}

\begin{rmk}
Lurie $\pi_n(X) = 0 \iff \forall c \in C$ and $S^n \to X(c)$ there is a cover $\{ c_i \to c \}$ s.t. the composite $S^n \to X(c) \to X(c_i)$ are nullhomotopic (not based!). 
\end{rmk}


\begin{example}
\begin{enumerate}
\item if $X$ is a scheme then $\pi_0(X) = X$ and $\pi_n(X) = 0$ for $n > 0$ since it is a discrete sheaf (but its motivicification will not be)
\item if $G$ is a sheaf of groups then $\pi_0(*/G) = 0$ and $\pi_1(*/G) = G$
\item $\pi_k(\L_{\mot} \P^1)$ consider the Nisnevich square,
\begin{center}
\begin{tikzcd}
\Gm \arrow[r] \arrow[d] & \A^1 \arrow[d]
\\
\A^1 \arrow[r] & \P^1
\end{tikzcd}
\end{center}
is a pushout in the category of Nis sheaves since every Nis sheaf takes it to a pullback. Therefore $\L_{\mot} \P^1 \cong \L_{\mot} \Sigma \Gm$. If $X$ is pointed then $X \to *$ is an effective epi and thus $* \to \Sigma X$ is an effective epi because the diagram,
\begin{center}
\begin{tikzcd}
X \arrow[r] \arrow[d] & * \arrow[d]
\\
* \arrow[r] & \pushout \Sigma X
\end{tikzcd}
\end{center}
is a pushout
\end{enumerate}
\end{example}

\begin{rmk}
Taking stalks and homotopy groups commutes. 
\end{rmk}


Long exact sequence: if $f : (X, x) \to (Y, y)$ is a pointed map then consider the homotopy fiber $(F, x) \to (X, x) \to (Y, y)$. Then we get a LES of pointed homotopy sheaves.
\bigskip\\
Connectedness of a map: $f : (X, x) \to (Y, y)$ is $n$-connected iff the homotopy fiber is $n$-connected. By LES this is equivalent to $\pi_k(X, x) \to \pi_k(Y, y)$ is an equivalence for $k \le n$ and an effective epi for $k = n +1$. 
\bigskip\\
To get the HTT version for unpointed sheaves either use stalks or only consider connected $X$. 

\begin{prop}
\begin{enumerate}
\item $\pi_n$ preserves products
\item a map of presheaves $X \to Y$ is $n$-connected implies its sheafification is also $n$-connected. 
\end{enumerate}
\end{prop}

\begin{prop}[Blakers-Massey]
Consider the pushout diagram,
\begin{center}
\begin{tikzcd}
U \arrow[r] \arrow[d] & B \arrow[d]
\\
A \arrow[r] & \pushout X 
\end{tikzcd}
\end{center}
and $U \to B$ is $n$-connected and $U \to A$ is $m$-connected then $U \to A \times_X B$ is $(m+n)$-connected.
\end{prop}

\begin{cor}[Freudenthal Suspension]
When $A = B = *$ then get that $U \to \Omega \Sigma U$ is $2n$-connected 
\end{cor}

\begin{cor}
In the case that $B = *$ and $X = \cofib{f}$ we find that $X \to \fib{(Y \to \cofib{X})}$ is $(m+n)$-connected where $f$ is $n$-connected and $X$ is $m$-connected. In particular, take the fiber over a point in $Y$ gives $\fib(f) \to \Omega \cofib{f}$ is $(m+n)$-connected.
\end{cor}

\section{Vector Bundles}

In topology, if $Z \embed Z$ is a clsoed embedding I always have an open neighborhood of $Z$ in $M$ given by the normal bundle,
\begin{center}
\begin{tikzcd}
H_Z(M) \arrow[d, equals] \arrow[r, equals] & H_Z(N_{Z/M}) \arrow[d, equals] 
\\
\wt{H}(M/M \sm Z) & \wt{H}(\Th_Z(N_{Z/M}))  
\end{tikzcd}
\end{center}



Can give a similar description of the map induced by the inclusion,
\[ \mathrm{RFun}(\Spc_S^\op, \C) \to \Fun(\Sm_S^\op, \C) \]
RHS is limit preserving functors. Given such a $H : \Sm_S^\op \to \C$ and a closed immersion $Z \to X$ define $H_Z(X)$ as the fiber of $H(X) \to H(X \sm Z)$. We interpret exxcision given $Z \to X$ is $U$ is a neighborhood of $Z$ in $X$ then ...


Then see, if $V$ is a vector bundle on $X$ then,
\[ \Th_X(V) = \P(V \oplus \struct{X}) / \P(V) \]
We also see,
\[ \Th_X(V) = \Sigma (\A(V) \sm 0) = \A(V) / \A(V) \sm 0 \]
Indeed we see that $\Th_X$ is the pushout,
\begin{center}
\begin{tikzcd}
\A(V) \sm 0 \arrow[r] \arrow[d] & \A(V)
\\
* \arrow[r] & \pushout \Th_X(V)
\end{tikzcd}
\end{center}

\subsection{Deformation to the Normal Cone}

Given a closed immersion $Z \to X$ then define,
\[ D(X, Z) = \Bl_{Z \times 0} (X \times \A^1) \sm \Bl_{Z \times 0} (X \times 0) \]
Then the map,
\begin{center}
\begin{tikzcd}
Z \times \A^1 \arrow[rr] \arrow[rd] & & D(X, Z) \arrow[ld]
\\
& \A^1
\end{tikzcd}
\end{center}
The fiver over $0$ is,
\[ Z \embed \P(\cN_{Z / X} \oplus \struct{Z}) \sm \P(\cN_{Z/X}) = \A(\cN_{Z/X}) \]
Fibers away from $0$ are $Z \embed X$. 

\begin{defn}
Category of smooth pairs in $\Sm_S$ is the category of closed immersions and cartesian squares.
\end{defn}

\begin{defn}
$(X, Z) \to (X', Z')$ is \textit{weakly excisive} if,
\[ \L_{\mot} (X / (X \sm Z) \to X'/(X'/\sm Z')) \]
is an equivalence. 
\end{defn}

\begin{defn}
We say that $U \to X$ is an \etale neighborhood of $Z \to X$ if $U \times_X Z \to Z$ is an isomorphism. 
\end{defn}

\begin{rmk}
If $U \to X$ is an \etale neighborhood of $Z \to X$ then $U / (U \sm Z) \to X / (X \sm Z)$ is an equivalence by Nisnevich excision and thus $(U, U \cap Z) \to (X, Z)$ is weakly excisive. 
\end{rmk}


\begin{prop}
For any $t : \A^0 \to \A^1$ we have,
\[ (D_t(X, Z), Z) = t^* (D(X, Z), Z \times \A^1) \to (D(X, Z), Z \times \A^1) \]
is weakly excisive. 
\end{prop}

\begin{rmk}
if $S$ is qcqs then any smooth pair $(X, Z)$ in $\Sm_S$ satisfies the following: Zarisik-locally on $X$ there is a Cartesian square,
\begin{center}
\begin{tikzcd}
Z \arrow[d, "\et"] \arrow[r, hook] & X \arrow[d, "\et"]
\\
\A^m \arrow[r, hook] & \A^n
\end{tikzcd}
\end{center}
in $\Sm_S$. 
\end{rmk}

\begin{lemma}
Suppose that $P$ is a property of smooth pairs over qcqs $S$ such that,
\begin{enumerate}
\item $P$ has excision: if $U \to X$ is an \etale neighborhood of $Z \to X$ then $P(U, Z) \iff P(X, Z)$

\item $P$ has descent: if $U_0 \to X$ is a Zariski cover and $(X, Z)$ is a smooth pair then write $(U_\bullet, Z_\bullet)$ for the induced Cech nerve and if $P(U_n, Z_n)$ for all $n$ then $P(X, Z)$

\item $P$ holds for all pairs $(Z \times \A^c, Z \times \{ 0 \})$ for any $Z \in \Sm_S$ 
\end{enumerate}
then $P$ holds for all smooth pairs $(X, Z)$. 
\end{lemma}

\begin{rmk}
We're going to apply this lemma to the property $P(X, Z)$ holds if for any $t : \A^0 \to \A^1$ the map $(D_t(X, Z), Z) \to (D(X, Z), Z \times \A^1)$ is weakly excisive. Descent follows from the fact that being weakly excisive is Zariski local on the target. 
\bigskip\\
If $U' \to X'$ is an \etale neighborhood of $Z'$ write $U \to Z$ for the induced \etale neighborhood of $Z$. So we obtain a diagram,
\begin{center}
\begin{tikzcd}
(U, Z) \arrow[r] \arrow[d] & (X, Z) \arrow[d]
\\
(U', Z') \arrow[r] & (X', Z') 
\end{tikzcd}
\end{center} 
so $(U, Z) \to (U', Z')$ is weakly excisive iff $(X, Z') \to (X', Z')$ is weakly excisive. 
\end{rmk}

\begin{rmk}
Use descent to reduce to showing that have \etale $(X, Z) \to (B \times \A^1, B)$ we have $P(X, Z)$.
\[ Z \times_B Z \embed X \times_{B \times \A^1} (Z \times \A^1) \]
closed immersion. Then $Z \to B$ is \etale implies $Z \to Z \times_B Z$ is open immersion. Call its component $K$ call the complement,
\[ U := (X \times_{B \times \A^c} (Z \times \A^c)) \sm K \]
Then $U$ is an \etale neighborhood of $Z \to Z \times_B Z \to X \times_{B \times \A^c} (Z \times \A^c)$ 
then we have that,
\[ U \to X \times_{B \times \A^c} (Z \times \A^c) \to X \]
and with the last map to $Z \times \A^c$ are \etale. So $U \to X$ is an \etale neighbrohood of $Z$ and $U \to Z \times \A^c$ is an \etale neighborhood of $Z$. Thus $P(Z \times \A^1, Z) \implies P(U, Z) \implies P(X, Z)$,
\begin{center}
\begin{tikzcd}
& Z \arrow[r] & X 
\\
Z \arrow[ru, equals] \arrow[rd, equals] \arrow[r] & Z \times_B Z \arrow[u] \arrow[d] \arrow[r] \arrow[d] & X \times_{B \times \A^c} (Z \times \A^c) \arrow[d] \arrow[u]
\\
& Z \arrow[r] & Z \times \A^c
\end{tikzcd}
\end{center}
\end{rmk}

\begin{defn}
$[X, Y] = \pi_0(\Hom{\Spc{S}}{\L_{\mot} X}{\L_{\mot} Y)}$
\end{defn}

\begin{theorem}
If $S$ is qcqs, $X \in \Sm_S$ is noetherian, $\dim{S} \le d$. If $Y \to Y'$ is a $d$-connected map in $\Spc_S$ then,
\[ [X, Y]_{\mot} \to [X, Y']_{\mot} \]
is bijective.
\end{theorem}

\begin{theorem}
If $X$ is a smooth affine scheme of $\dim{X} = d$ then any vector bundle $\E$ with $\rank{\E} > d$ has a trivial summand. 
\end{theorem}

\newcommand{\B}{\mathbf{B}}

\begin{rmk}
This follows from the connectivity of the map $\B \GL_n \to \B \GL_{n+1}$.
\end{rmk}


\newpage

\section{Appendix}


To understand for writing the appendix:

\begin{enumerate}
\item \chref{https://arxiv.org/pdf/2012.13304}{Nisnevich motive of a stack}

\item \chref{https://rezk.web.illinois.edu/freudenthal-and-blakers-massey.pdf}{Blakers-Massey} need to find a reference in $\A^1$-homotopy theory

\item \chref{https://www.math.columbia.edu/~magenroy/motivicseminar.html}{here is Roy's master list}

\item \chref{Kirsten's intro to motivic homotopy theory}{https://arxiv.org/pdf/1902.08857}
\end{enumerate}

\subsection{TODO}

\begin{enumerate}
\item connectivity arguments find and prove
\item are Ravi's stacks Exhaustive as in \chref{https://arxiv.org/pdf/1711.11072}{here}
\item consider homotopy groups of $\P^n$ to see potential counterexamples
\end{enumerate}


\subsection{Blakers-Massey}

The main question concerns the connectivity of the inclusion map $j : U \to X$ of a open set of a smooth variety with large codimension.

\begin{defn}
Let $f : X \to Y$ be a morphism of topological spaces. $f$ is $n$-connected if for all $x \in X$ the induced map $f_* : \pi_k(X, x) \to \pi_k(Y, f(x))$ is an isomorphism for all $k \le n$ and an epimorphism for $k = n+1$. 
\end{defn}

\begin{rmk}
$X \to *$ is $n$-connected iff $X$ is $n$-connected. Likewise, if $X$ is nonempty then $X$ is $(n+1)$-connected iff $* \to X$ is $n$-connected. 
\end{rmk}

\begin{rmk}
From the long exact sequence of a fibration, $f : X \to Y$ is $n$-connected if and only if the homotopy fiber of $f$ is $n$-connected.
\end{rmk}

\begin{rmk}
Note how this convention plays with connectivity of a pair $(X, A)$. We say that $(X,A)$ is $n$-connected if $\pi_k(X,A) = 0$ for $k \le n$. This is equivalent, by the long exact sequence, to the inclusion map $A \to X$ being $(n-1)$-connected.
\end{rmk}

 Over $\CC$ and in the traditional homotopy theory this is taken care of by the classical Blakers-Massey theorem and manipulation of tubular neighborhoods. First we recal.

\begin{prop}[Blakers-Massey]
Consider a homotopy pushout diagram of topological spaces,
\begin{center}
\begin{tikzcd}
U \arrow[r] \arrow[d] & B \arrow[d]
\\
A \arrow[r] & \pushout X 
\end{tikzcd}
\end{center}
Suppose that the map $U \to A$ is $m$-connected and $U \to B$ is $n$-connected. Then
\begin{enumerate}
\item the map of pairs $(A, U) \to (X, B)$ is $(m + n + 1)$-connected (meaning the identical definition in terms of maps of homotopy groups of pairs) {\color{red} GET THE INDEXING RIGHT}
\item the map to the homotopy pullback $U \to A \times^h_X B$ is $(m+n)$-connected.
\end{enumerate} 
\end{prop}

\begin{proof}
\chref{https://rezk.web.illinois.edu/freudenthal-and-blakers-massey.pdf}{See a proof here}. 
\end{proof}

\begin{rmk}
In fact, these statements are equivalent. I claim there is a natural equivalence between certain homotopy fibers
\[ \fib(U \to A \times^h_X B) \iso \fib(\fib(U \to A) \to \fib(B \to X)) \]
Indeed, for any commutative square
\begin{center}
\begin{tikzcd}
U \arrow[r] \arrow[d] & B \arrow[d]
\\
A \arrow[r] & X
\end{tikzcd}
\end{center}
Then the homotopy groups of the fiber of $U \to A$ are (by the long exact sequence and 5-lemma) equal to the homotopy groups of the pair $(A, U)$ and likewise for $(X, B)$. Furthermore, the map $U \to A \times^h_X B$ is $k$-connected iff $\fib(U \to A \times^h_X B)$ is $k$-connected. This gives the equivalence between the formulations. 
\end{rmk}

\begin{rmk}
This theorem is a vast generalization of Freudenthal suspension theorem. In that case $A = B = *$ so if $U$ is $n$-connected then the maps $U \to *$ are $n$-connected so $U \to \Omega \Sigma U$ is $2n$-connected.
\end{rmk}

\begin{rmk}
Usually Blakers-Massey, in this context called ``homotopy excision'', is stated for an \textit{excisive triple} $(X; A, B)$. This is {\color{red} TODO} which ensures that $X$ is the homotopy colimit of $A \leftarrow A \cap B \to B$ hence we can apply the theorem to conclude that if $(A, A \cap B)$ is $m$-connected and $(B, A \cap B)$ is $n$-connected then $(A, A \cap B) \to (X, B)$ is $(m+n)$-connected. {\color{red} check indexing}
\end{rmk}

This gives, for example, May's or Hatcher's version of homotopy excision.

\begin{theorem}
Let $(X; A,B)$ be an excisive triad such that $C = A \cap B$ is nonempty and connected. Assume that $(A,C)$ is $m$-connected and $(B,C)$ is $n$-connected, for $m, n \ge 0$. Then the inclusion $(A, C) \to (X,B)$ is $(n+m-1)$-connected.
\end{theorem}

\begin{prop}
Let $X$ be a manifold and $Z \subset X$ a submanifold of (real) codimension $r$. Let $U = X \sm Z$. Then the inclusion $j : U \to X$ is $(r-2)$-connected meaning 
\[ \pi_k(U) \to \pi_k(X) \]
is an isomorphism for $k \le r-2$ and is surjective for $k = r - 1$.
\end{prop}

\begin{proof}
We apply Blakers-Massey to the excisive triple $(X; N_Z, U)$ where $N_Z$ is a disk bundle variant of the normal bundle which is isomorphic to a tubular neighborhood of $Z$ inside $X$. Notice that the intersection $C := N_Z \cap U = S(N_Z)$ is homotopy equivalent to the sphere bundle of the normal bundle. Since $N_Z$ has rank $r$, the bundle $S(N_Z) \to Z$ is a $S^{r-1}$-bundle and hence is $(r-2)$-connected. Furthermore, $S(N_Z) \to U$ is $-1$-connected because both are connected. Therefore, $(N_Z, S(N_Z)) \to (X, U)$ is $(r-2)$-connected meaning
\[ \pi_k(N_Z, S(N_Z)) \to \pi_k(X,U) \]
is an isomorphism for $k \le r - 2$ and surjective for $k = r - 1$. However, the pair $(N_Z, S(N_Z))$ is also $(r-1)$-connected (since $S(N_Z) \to Z$ is $(r-2)$-connected) meaning $\pi_k(N_Z, S(N_Z)) = 0$ for $k \le r - 1$ thus $\pi_k(X, U) = 0$ for $k \le r - 1$.  
\end{proof}

\begin{rmk}
The example of $S^n$ shows that this is optimal for all $r$. Indeed, let $U \embed S^n$ be the complment of one point then $U \cong \RR^n$ 
\end{rmk}

\begin{cor}
Let $X$ be a complex manifold (in particular a smooth $\CC$-variety) and $Z \subset X$ a submanifold of (complex) codimension $r$. Let $U = X \sm Z$. Then the inclusion $j : U \to X$ is $(2r-2)$-connected meaning 
\[ \pi_k(U) \to \pi_k(X) \]
is an isomorphism for $k \le 2r-2$ and is surjective for $k = 2r - 1$.
\end{cor}

\begin{rmk}
This is also optimal. For example, for $r = 1$ consider $j : E \sm \{ p \} \embed E$ the complement of a point in an elliptic curve. This is surjective on $\pi_1$ but not an isomorphism. For $\A^1 \to \P^1$, the complement of the point at $\infty$, the induced map on $\pi_2$ is not surjective. 
\end{rmk}

\begin{cor}
Let $X$ be a smooth $\CC$-variety and $Z \subset X$ a (not necessarily smooth) subvariety. Let $U = X \sm Z$. Then the inclusion $j : U \to X$ is $(2r - 2)$-connected meaning
\[ \pi_k(U) \to \pi_k(X) \]
is an isomorphism for $k \le 2r - 2$ and is surjective for $k = 2r - 1$.
\end{cor}

\begin{proof}
Consider a filtration
\[ Z = Z_0 \supsetneq Z_1 \supsetneq \cdots \supsetneq Z_k = \empty \]
where $Z_i \sm Z_{i+1}$ is smooth. This exists by generic smoothness. We proceed by induction on $r$. For $r = \dim{X}$ the subvariety consists of a finite set of points and hence is automatically smooth. Now we factor
\[ \pi_k(X \sm Z) \to \pi_k(X \sm Z_1) \to \pi_k(X) \]
since $Z_1$ is codimension $\ge r + 1$ in $X$ by the induction hypothesis $X \sm Z_1 \to X$ is $2r$-connected. Now $X \sm Z \to X \sm Z_1$ is the complement of the smooth subvariety $Z \sm Z_1$ so by our previous results it is $(2r-2)$-connected. Hence both maps are isomorphism for $k \le 2r-2$ and surjective for $k = 2r - 1$.
\end{proof}


\subsection{Motivic Spaces}

$\A^1$-connectivity results for $j : U \embed X$ do not carry over exactly as for complex manifolds. One manifestation is that for varieties defined over $\RR$ the motivic spaces have an $\RR$-realization so the best we could hope for is that if the codimension of $Z = X \sm U$ is $r$ then $j$ is $(r-2)$-connected. 

I do not know if this holds. I have some reason to doubt it. The reson the proof fails is that we don't have access to tubual neighborhoods or sphere bundles in motivic homotopy theory. Instead, we have the purity isomorphism which will tell us that $\cofib(U \to X)$ is highly $\A^1$-connected.

\subsection{Connectivity and cofibers}

Note the following consequences of Blakers-Massey,

\begin{prop}
Consider the homotopy cofiber
\begin{center}
\begin{tikzcd}
X \arrow[d] \arrow[r, "f"] & Y \arrow[d, "r"]
\\
* \arrow[r] & \cofib(f)
\end{tikzcd}
\end{center}
If $X$ is $m$-connected and $f$ is $n$-connected then $X \to \fib(r)$ is $(m+n)$-connected. Therefore, we get a diagram from the long exact sequence for the homotopy fiber of $r$.
\begin{center}
\begin{tikzcd}
\cdots \arrow[r] & \pi_k(\fib(r)) \arrow[r] & \pi_k(Y) \arrow[r] & \pi_k(\cofib(f)) \arrow[r] & \cdots
\\
& \pi_k(X) \arrow[u] \arrow[ru, "f_*"']  
\end{tikzcd}
\end{center}
since $\pi_k(X) \to \pi_k(\fib(r))$ is an isomorphism for $k \le m+n$ and an epi for $k = m + n + 1$ and $f_*$ is an isomorphism $k \le n$ and surjective for $k = n+1$. Thus $\pi_k(\cofib(f)) = 0$ for $k \le n+1$ as long as $m \ge 0$ {\color{red} apparently I don't need this [Lur09, Proposition 6.5.1.20]}.
\end{prop}

\begin{cor}
Consider the homotopy cofiber
\begin{center}
\begin{tikzcd}
X \arrow[d] \arrow[r, "f"] & Y \arrow[d, "r"]
\\
* \arrow[r] & \cofib(f)
\end{tikzcd}
\end{center}
we can ``rotate'' as in the formation of the Puppe sequence via taking pushouts to get a diagram consisting of homotopy pushout squares
\begin{center}
\begin{tikzcd}
X \arrow[d] \arrow[r, "f"] & Y \arrow[d, "r"] \arrow[r] & * \arrow[d]
\\
* \arrow[r] & \cofib(f) \arrow[r] \arrow[d] & \Sigma X \arrow[d, "-\Sigma f"] \arrow[r] & * \arrow[d]
\\
& * \arrow[r] & \Sigma Y \arrow[r] & \Sigma \cofib(f)
\end{tikzcd}
\end{center}
Applying the above result to the ``rotated'' homotopy pushout square
\begin{center}
\begin{tikzcd}
\cofib(f) \arrow[r] \arrow[d] & \Sigma X \arrow[d, "-\Sigma f"] 
\\
* \arrow[r] & \Sigma Y
\end{tikzcd}
\end{center}
since $\Sigma X$ is $0$-connected, $\cofib(f) \to \Sigma X$ is $-1$-connected\footnote{Here I probably assumed $X$ is nonempty so $Y$ is nonempty so $\cofib(f) \to \Sigma X$ is surjective on path-components since $\Sigma X$ is connected and $\cofib(f)$ is nonempty. This implies $\cofib(f) \to \Sigma X$ is $-1$-connected.} Hence by Blakers-Massy: if $\cofib(f)$ is $m$-connected we conclude that $\Sigma f : \Sigma X \to \Sigma Y$ is $(m-1)$-connected.
\end{cor}

\begin{rmk}
The signs in the Puppe sequence come from the standard identification of the iterated mapping cones with $\Sigma X$ etc. See for example,
\begin{enumerate}
\item \chref{https://mathoverflow.net/questions/2077/how-to-determine-the-homotopy-groups-of-the-suspension-of-a-space}{this mathoverflow answer}
\item \chref{https://math.colorado.edu/~agbe5088/math6280/classnotes/Class10.pdf}{MATH 6280 at Colorado class 10}
\item ipad notes from Dec.13 
\end{enumerate}
\end{rmk}

\begin{rmk}
This is the optimal result. Indeed, consider $S^1 \to *$ which has cofiber $\Sigma S^1 = S^2$ which is $1$-connected. Indeed, $S^1 \to *$ is $0$-connected but not $1$-connected.
\end{rmk}

We can form this in a different way using the alternative version of Blakers-Massey.

\begin{prop}
Let $f : X \to Y$ be $n$-connected and $X$ be $m$-connected. Then the natural map 
\[ \fib(f) \to \Omega \cofib(f) \]
is $(n+m)$-connected.
\end{prop}

\begin{proof}
Consider the pushout square
\begin{center}
\begin{tikzcd}
X \arrow[r, "f"] \arrow[d] & Y \arrow[d]
\\
* \arrow[r] & \cofib(f)
\end{tikzcd}
\end{center}
then using the first form we see that
\[ \fib(f) \to \fib(* \to \cofib(f)) = \Omega \cofib(f) \]
is $(n + m)$-connected. 
\end{proof}

These above results carry over completely to the world of motivic spaces.

\newcommand{\Hen}{\mathrm{Hen}}

\begin{lemma}
Let $F \in \PSh(\Sm_S)$. Then for $x \in X \in \Sm_S$ the Nisnevich stalk of $\pi_i(F)$ at $x$ is $\pi_i(F_x)$.
\end{lemma}

\begin{proof}
Indeed, 
\[ \pi_i(F)_x = \colim_{Y \in \Hen_{X,x}} \pi_i(F(Y)) = \pi_i(\colim_{Y \in \Hen_{X,x}} F(Y)) = \pi_i(F_x) \] 
because $\pi_i$ commutes with filtered homotopy colimits {\color{red} NEED REFERENCE} (e.g. look at \chref{https://mathoverflow.net/questions/56166/do-homotopy-groups-always-commute-with-filtered-colimits}{this answer}). 
\end{proof}

\begin{cor}
To check if a (pointed) object $F$ or morphism $f : F \to G$ is $n$-connected it suffices to check on Nisnevich stalks. 
\end{cor}

\begin{proof}
Indeed, the morphism of sheaves $\pi_i(F) \to \pi_i(G)$ (with $Y = *$ in the first case) is an isomorphism it suffices to check this on stalks which is the same as checking the corresponding map of spaces $F_x \to G_x$ is $n$-connected. 
\end{proof}

\begin{lemma}
Let $k$ be a perfect field, $f : Y \to X$ be a morphism in $\PSh(\Sm_k)$, and $n \ge -1$. If $f$ is $\A^1$-$n$-connected, then $\cofib(f)$ is $\A^1$-$(n+1)$-connected. The coverse holds if $X$ and $Y$ are $\A^1$-$1$-connected.
\end{lemma}

\begin{proof}
Since $L_{\mot}$ is a left-adjoint it preserves homotopy colimits. Hence it suffices to show that $L_{\mot}(\cofib(f)) = \cofib(L_{\mot}(f))$ is $(n+1)$-connected. By assumption $L_{\mot}(f))$ is $n$-connected so the map on stalks is $n$-connected but $\cofib$ commutes with colimits hence with taking stalks so by the result for spaces we conclude that $\cofib(L_{\mot}(f))$ is $(n+1)$-connected.
\par 
However, for the converse we need to be a little careful since it does not preserve homotopy fibers. Let $C = \cofib(L_{\mot}(f))$ and $F = \fib(L_{\mot}(f))$. Our assumption is that $C$ is $\Aconn{n+1}$. 
\end{proof}




\begin{lemma}[8.9]
Let $k$ be a perfect field, $X$ a smooth $k$-scheme, and $Z \subset X$ a closed subscheem of codimension $\ge r$. Then $\Sigma(X/(X \sm Z))$ is $\Aconn{r}$. 
\end{lemma}

\begin{proof}
$X$ is a disjoint union of quasi-compact smooth schemes so we may assume that $X$ is quasi-compact. If $Z$ is smooth then $X / (X \sm Z)$ is $\Aconn{r-1}$ by purity.
\bigskip\\
Notice that I can assume $Z$ is smooth because the sheaf represented by $Z$ and $Z_{\red}$ on $\Sm_k$ are the same since a morphism from a reduced scheme factors through the reduction.
\bigskip\\
Since $k$ is perfect, generic smoothness gives a filtration,
\[ \empty = Z_0 \subsetneq Z_1 \subsetneq \cdots \subsetneq Z_n = Z \]
of closed subschemes such that $Z_j \sm Z_{j-1}$ is smooth. We prove the result by induction on $n$. Consider the cofiber sequence,
\[ \frac{X \sm Z_{n-1}}{X \sm Z} \to \frac{X}{X \sm Z} \to \frac{X}{X \sm Z_{n-1}} \]
Therefore because homotopy colimits commute we get a cofiber sequence,
\[ \Sigma \left( \frac{X \sm Z_{n-1}}{X \sm Z} \right) \to \Sigma \left( \frac{X}{X \sm Z} \right) \to \Sigma \left( \frac{X}{X \sm Z_{n-1}} \right) \]
by the induction hypothesis $X / (X \sm Z_{n-1})$ is $\Aconn{r}$ (indeed it should be $\Aconn{r+1}$ because $\codim{Z_{n-1}, X} \ge r + 1$ but we don't need this) and hence by Lemma 8.8 the morphism,
\[ \Sigma \left( \frac{X \sm Z_{n-1}}{X \sm Z} \right) \to \Sigma \left( \frac{X}{X \sm Z} \right)  \]
is $\Aconn{r-1}$. Since, $X \sm Z_{n-1}$ and $X \sm Z$ are smooth, by the smooth case we know that $\Sigma ((X \sm Z_{n-1})/(X \sm Z))$ is $\Aconn{r}$ and hence $\Sigma( X / (X \sm Z))$ is $\Aconn{r}$.
\end{proof}


\subsection{Purity}



\subsection{Motivic Spectra}

\newcommand{\SH}{\mathrm{SH}}

Let $S$ be a scheme and $\Sm_S$ the category of smooth schemes over $S$. Then $\Spc(S)_*$ be the $\infty$-category of pointed motivic spaces over $S$. $\Spc(S)$ is the category of $\A^1$-invariant Nisnevich sheaves on $\Sm_S$ valued in spaces (or equivalently simplicial sheaves). The inclusion
\[ i : \Spc(S) \to \PSh(\Sm_S) \]
admits a left-adjoint $L_{\mot} : \PSh(\Sm_S) \to \Spc(S)$. 

\begin{defn}
The category \textit{of motivic spectra} is the localization $\SH(S) := \Spc(S)_*[(\P^1)^{-1}]$. Denote by
\[ \Sigma^\infty : \Spc(S)_* \to \SH(S) \]
the canonical functor. 
\end{defn}

\begin{prop}
Let $j : U \to X$ be the inclusion of an open set of a smooth $k$-scheme whose complement has codimension $r$. Then
\[ \Sigma^\infty U \to \Sigma^\infty X \]
is $(r-2)$-connected. In fact, it is $(r-2)$-very effective. 
\end{prop}

\begin{proof}
{\color{red} TODO}
\end{proof}

\subsection{Stacks}

Furthermore, if X admits a good moduli space (in the sense of [Alp]), then by a theorem of Alper,
Hall and Rydh, it is Nisnevich locally of the form [Spec (A)/GLn] (see [AHR, Theorem 13.1]), and
hence a cd-quotient stack2
.

Remark 2.11. In Proposition 2.10, the GLn-presentation Y• and the Cech nerve ˇ U• are simplicial
objects in the category of algebraic spaces. This is because a smooth presentation p : U → X of an
algebraic stack need not be representable by schemes, but only algebraic spaces. However, as any
algebraic space admits a Nisnevich presentation by a scheme [Knu, Theorem II.6.4], we can refine the
Cech nerve ˇ U• to a generalised hypercovering V• such that each Vi
is a scheme. Then M(V•) computes
the motive M(X) (see [DHI] for details).


\begin{rmk}
Since a nice stack is therefore Nisnevich locally a global quotient stack, we just need to show that a vector bundle over a global quotient stack is a motivic equivalence.
\end{rmk}

\section{Note about homotopy theory}


\newcommand{\Shv}{\mathrm{Shv}}


The reason that the homotopy category $h C$ of a homotopy theory (whatevery that means) is ``bad'' is that homotopy limits and colimits are not categorically defined (i.e. not intrinsic to the category $h C$) and certainly do not agree with the internal notion of limits or colimits. 
\par 
This is what $\infty$-categories fixes, it gives a framework in which homotopy limits or colimits are actual internal limits or colimits in the category theory itself. 
\par 
Model structures, on the other hand, give a $1$-categorical framework in which homotopy (co)limits can be computed and manipulated through judicious use of (co)fibrant replacement. They are still not intrinsic to the category (i.e.\ they depend on the model structure) and very different model structures may be Quillen equivalent so it seems less intrinsic than the $\infty$-categorical notions of homotopy (co)limits. Furthermore, the heavy use of replacements makes functoriality of the homotopy (co)limits somewhat more cumbersome (think of how anoying it can be to define things in the derived category). 



\section{Representing Torsors}

Here we use the notation of Morel-Voevodsky. 

{\color{red} MV uses simplicial objects in sheaves, isn't this not the same as sheaves of simplicial sets in the $\infty$-categorical sense? Maybe this is reflected in the model structure?}

Let $\Delta^{\op} \Shv(T)$ be the topos of simplicial objects in sheaves on a site $T$. There is a cosimplicial object 
\[ \Delta \to \Delta^{\op} \Shv(T) \quad n \mapsto \Delta^n \]
where $\Delta^n$ is the constant simplicial sheaf valued in the $n$-simplex $\Delta^n = h^{[n]}$ representable simplicial set.

\begin{defn}
On $\Delta^{\op} \Shv(T)$ we define a Quillen model structure $(W_s, C, F_s)$ 
\begin{enumerate}
\item $W_s$ is the class of (simplicial) weak equivalences: $f : \X \to \Y$ such that for any point $x$ of $T$
\[ x^*(t) : x^*(\X) \to x^*(\Y) \]
is a weak equivalence
\item $C$ is the class of cofibrations: monomorphisms $f : \X \to \Y$ 
\item $F_s$ is the class of fibrations: morphisms with the right lifting property with respect to cofibrations which are also weak equivalences.
\end{enumerate}
\end{defn}

\begin{theorem}
For any small site $T$, the triple $(W_s, C, F_s)$ is a model structure. 
\end{theorem}

Let $\H_s(T)$ be the corresponding homotopy category. 

\subsection{Homotopical classification of torsors}

Let $T$ be a site and $G$ a sheaf of simplicial groups on $T$. A right (resp.\ left) action of $G$ on a simplicial sheaf $\X$ is a morphism $a : \X \times G \to \X$ (resp.\ $G \times \X \to \X$) such that the usual diagrams commute. A (left) action is called (categorically) free ({\color{red} shouldn't it be free-transitive}) if the morphism $G \times \X \to \X \times \X$ of the form $(g, x) \mapsto( a(g,x), x)$ is an isomorphism.
\par 
For any right action $G$ on $\X$ define the quotient $\X / G$ as the coequalizer of the two morphisms $\X \times G \to \X$. 
\par 
A principal $G$-bundle (or equivalenctly $G$-torsor) over $\X$ is a morphism $\Y \to \X$ together with a free (right) action $G$ on $\Y$ such that the canonical morphism $\Y / G \to \X$ is an isomorphism. 

\begin{rmk}
If $T$ is the \etale site on a scheme $X$ then this gives the usual definition restricted to sheaves of sets. Indeed, if $G$ is a sheaf of groups, $\F$ is a sheaf of sets satisfying
\begin{enumerate}
\item $G \times \F \to \F \times \F$ is an isomorphism
\item $\F / G \to X$ is an isomorphism. Note that $\F \to \F / G$ is surjective by definition meaning the canonical section of $\F / G \iso X$ lifts locally on $X$ to a section of $\F$.
\end{enumerate} 
\end{rmk}

Denote by $P(\X, G)$ the set of isomorphism classes of principal $G$-bundles over $\X$ pointed by the trivial bundle.

\begin{rmk}
The base change maps make $\X \mapsto P(\X, G)$ into a contravariant functor on simplical sheaves to pointed sets. 
\end{rmk}

\begin{defn}
Let $X$ be a sheaf of sets on $T$. Denote by $E(X)$ the simplicial sheaf of sets whose $n$-th term is $X^{n+1}$ and with faces (resp.\ degeneracies) induced by partial projections (resp.\ diagonals). It is unqiuely defined by the property that
\[ \Hom{\Delta^{\op} \Shv(T)}{\Y}{E(X)} \to \Hom{\Shv(T)}{\Y_0}{X} \]
is bijective for any simplicial sheaf $\Y$.
\end{defn}

When $G$ is a sheaf of groups then $E(G)$ becomes a simplicial sheaf of groups (using $E(X \times Y) \cong E(X) \times E(Y)$ naturally) whose subgroup of vertices is $G$. It has both left and right $G$-actions. 


{\color{red} what is the definition of $B(G)$}

The morphism
\[ E(G) \to B(G) \quad (g_0, g_1, \dots, g_n) \mapsto (g_0  g_1^{-1}, g_1 g_2^{-1}, \dots, g_{n-1} g_n^{-1}, g_n) \]
induces an isomorphism
\[ E(G) / G \iso B(G) \]
If $G$ is a simplicial sheaf of group, then taking the diagonal of the bisimplical group $(n,m) \mapsto E(G_n)_m$ defines a sheaf of simplicial groups denoted $E(G)$ which again contains $G$ as a subgroup. 
\par 
Again taking the diagonal defines $E(G) \to B(G)$ inducing $E(G)/G \iso B(G)$.

\begin{rmk}
Recall a trivial local fibration $\Y \to \X$ is a morphism of simplicial sheaves whose stalks are both Kan fibrations and weak equivalences. 
\end{rmk}

\begin{lemma}
Let $G$ be a simplicial sheaf of groups, and let $P$ a $G$-torsor over a simplicial sheaf $\X$. Then there is a trivial local fibration $\Y \to \X$ and a morphism $\Y \to B(G)$ wuch that $P \times_{\X} \Y$ is isomorphic to the pull-back of $E(G) \to B(G)$ along $\Y \to B(G)$.
\end{lemma}

\begin{proof}
Let $\Y = (P \times E(G))/G$ via the right diagonal action. The projection $\Y \to \X$ is a trivial local fibration (whose fibers are the locally fibrant and weakly constractibl simplicial sheaf $E(G)$). By construction, the pullback of $P$ to $\Y$ is isomorphic to the pullback of $E(G)$ via $\Y \to B(G)$.
\end{proof}

\begin{defn}
A simplicial sheaf $\X$ is said to be of \textit{simplical dimension} $\le n$ if $\X_n \times \Delta^n \to \X$ is an epimorphism, or equivalently if $\sk_n(\X) = \X$. 
\end{defn}

\begin{lemma}
Assume $G$ has simplicial dimenson zeor and $f : \X \to \Y$ is a trivial local fibration. Then the pullback $P(\Y, G) \to \P(\X, G)$ is a bijection.
\end{lemma}

The following lemma shows that for $G$ of dimension $0$ the set of torsors is ``homotopy invariant''.

\begin{lemma}
Assume that $G$ has simplicial dimension zero, then for any simplicial sheaf $\X$, the map $P(\X, G) \to P(\X \times \Delta^1, G)$ is a bijection. 
\end{lemma}

There is a map
\[ P(\X, G) \to \Hom{\mathcal{H}_s(T)}{\X}{BG} \]
given by sending the torsor $P$ to the roof $\X \leftarrow \Y \to BG$ constructed above. Let $BG \to \mathcal{B} G$ be a trivial cofbiration such that $\mathcal{B} G$ is fibrant.

\begin{lemma}
Assume that $G$ has simplicial dimension zero the natural map
\[ P(\X, G) \to \Hom{\mathcal{H}_s(T)}{\X}{BG} \]
is a bijection. 
\end{lemma}

\begin{prop}
For any $G$ of simplicial dimension zero and any object $U$ of $T$ one has
\[ \pi_i(\mathcal{B} G(U), *) = 
\begin{cases}
H^i(U, G) := P(U, G) &  i = 0
\\
G(U) & i = 1
\\
0 & i > 0
\end{cases} \]
\end{prop}

\subsection{The \etale classifying space $B_{\et} G$}

From now on, let $S$ be a noetherian scheme with finite Krull dimension. Let $G$ be a sheaf of groups (actual groups, hence simplicial dimension $0$) on $(\Sm_S)_{\et}$. There is a natural morphism of sites
\[ \pi: (\Sm_S)_{\et} \to (\Sm_S)_{\Nis} \]
via the inclusion of topologies. This allows us to define
\[ B_{\et} G = \RR \pi_* \pi^* BG \]
as an object of $\H_s((\Sm_S)_{\Nis})$.
\par 
Note that if $\B_{\et} G$ is a fibrant model of $B(G_{\et})$ in the category of simplicial \etale sheaves, then $B_{\et}(G) \cong \B_{\et} G$ (where we not consider $\B_{\et} G$ as a (fibrant) simplicial Nisnevch sheaf). Then we conclude
\[ \Hom{\H^s_\bullet((\Sm_S)_{\Nis}}{\Sigma_s^n U_+}{(BG, *)} = 
\begin{cases}
H^1_{\Nis}(U, G) & n = 0
\\
G(U) & n = 1
\\
0 & n > 1
\end{cases} \]
and likewise
\[ \Hom{\H^s_\bullet((\Sm_S)_{\Nis}}{\Sigma_s^n U_+}{(B_{\et}G, *)} = 
\begin{cases}
H^1_{\et}(U, G) & n = 0
\\
G_{\et}(U) & n = 1
\\
0 & n > 1
\end{cases} \]
In particular, this gives a criterion for $BG \to B_{\et} G$ to ne an isomorphism in $\H_s((\Sm_S)_{\Nis})$

\begin{lemma}
The canonical morphism $BG \to B_{\et} G$ s an isomorphism if and only if $G$ is a sheaf in the \etale topology and one of the following equivalent conditions holds:
\begin{enumerate}
\item for any scheme $U$ smooth over $S$, one has $H^1_{\Nis}(U, G) = H^1_{\et}(U, G)$
\item for any scheme $U$ smooth over $S$, and a point $x \in X$ 
\[ H^1_{\et}(\Spec{\stalk{X}{x}^h}, G) = * \] 
\end{enumerate}
\end{lemma}

\subsection{$\A^1$-contractibility}

Recall the $\A^1$-homotopy category $\H_\bullet(S)$ is (DEFINE)

See p. 133 for geometric models. 

\subsection{\etale group schemes}

\begin{prop}
Let $G$ be a finite \etale group scheme over $S$ of order prime to the characteristic of $S$. Then $B_{\et} G \in \Delta^{\op} \Shv_{\Nis}(\Sm_S)$ is $\A^1$-local.
\end{prop}

\begin{cor}
Let $G$ be as above. Then for any scheme $U$ smooth over $S$
\[ \Hom{\H_\bullet(S)}{\Sigma^m_t \Sigma^n_s U_+}{(B_{\et} G, *)} = 
\begin{cases}
H^1_{\et}(U, G) & n,m = 0
\\
G(U) & m = 0, n = 1
\\
\ker{(H^1_{\et}(\Gm \times S, G) \to H^1_{\et}(S, G))} & m = 1, n = 0
\\
0 & \text{ else }
\end{cases} \]
\end{cor}

\subsection{Algebraic $K$-theory}

\newcommand{\BGL}{\mathrm{BGL}}

\begin{prop}
There are canonical isomorphisms in $\H(S)$ 
\[ B \GL_n \iso B_{\et} \GL_n \to \mathrm{Gr}(n, \infty) \]
\end{prop}

Using the homotopy invariance of units and Picard group we condude 
\begin{prop}
Let $S$ be regular. The for any scheme $U$ smooth over $S$ 
\[ \Hom{\H_\bullet(S)}{\Sigma^n_t \Sigma^m_t U_+}{(\P^\infty, *)} = 
\begin{cases}
\Pic{U} & m,n = 0
\\
\struct{}^\times(U) & m = 0, n = 1
\\
H^0(U, \Z) & m = 1, n = 1
\\
0 & \text{ else}
\end{cases} \]
\end{prop}

For $n > 1$ the objects $B \GL_n = B_{\et} \GL_n$ are not know to be $\A^1$-local. {\color{red} do we know this now? Shouldn't it be false, since vector bundles up to isomorphism are not $\A^1$-invariant?}

Morel-Voevodsky prove $K$-theory in representable in $\H_\bullet(S)$ by $\BGL_\infty \times \Z$. 

\begin{theorem}
For any smooth scheme $X$ over a regular scheme $S$ and any $n,m \ge 0$ there is a canonical isomorphism
\[ \Hom{\H_\bullet(S)}{\Sigma_t^m \Sigma_s^n X_+}{(\BGL_\infty \times \Z, *)} = K_{n-m}(X) \]
where for $n < m$ the groups $K_{n-m}$ are zero. 
\end{theorem}

\end{document}
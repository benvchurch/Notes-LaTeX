\documentclass[12pt]{article}
\usepackage{import}
\import{./}{Includes}

\begin{document}

\section{Feb 11}

\subsection{Line Bundles}

There exists a map,
\[ \Gamma(X, \L^{\otimes a}) \otimes \Gamma(X, \L^{\otimes b}) \to \Gamma(X, \L^{\otimes ab}) \]
since we have an isomorphism $\L^{\otimes a} \otimes \L^{\otimes b} = \L^{\otimes ab}$. Furthermore, since $\L$ is rank $1$ this map is commutative since $s \times s' = s' \otimes s$ since they only differ by a section of $\struct{X}$. This alows us to define the following graded ring structure. 

\begin{definition}
Let $\L$ be an invertable $\struct{X}$-module, $\F$ any $\struct{X}$-module and $s \in \L(X)$ a global section. Then we define the following graded ring.
\[ \Gamma_*(X, \L) = \bigoplus_{n \ge 0} \Gamma(X, \L^{\otimes n}) \]
and then the following module,
\[ \Gamma_*(X, \L, \F) = \bigoplus_{n \ge 0} \Gamma(X, \F \otimes_{\struct{X}} \L^{\otimes n}) \]
which is a graded $\Gamma_*(X. \L)$-module. Furthermore, there is a map,
\[ \Gamma_*(X, \L, \F)_{(s)} \to \F(X_s) = \Gamma(X_s, \F) \]
sending $\frac{t}{s^n} \mapsto t|_{X_s} \otimes (s |_{X_s})^{\otimes -n}$. 
\end{definition}

\begin{proposition}
Let $X$ be a quasi-compact, quasi-seperated scheme and $\F$ be quasi-coherent. Then the above map is an isomorphism. 
\end{proposition}

\begin{proof}
Tag OB5K. (Compare with that Hartshorne Excercise).
\end{proof}

\begin{example}
Let $A$ be a graded ring such that $A$ is generated by $A_1$ as a $A_0$-algebra (e.g. $A = k[X_0, \dots, X_n]$). Let $X = \Proj{A}$ and consider the graded module $M = A(n)$ which is the graded module $M_{k} = A_{k + n}$. Then we can construct the Serre twists,
\[ \struct{X}(n) = \wt{M} = \wt{A(n)} \]
which is an invertable $\struct{X}$-module. Furthermore,
\[ \struct{X}(n) \otimes_{\struct{X}} \struct{X}(m) = \struct{X}(n + m) \]
\end{example}

\begin{remark}
This will not be invertible and these maps will not be isomorphisms in general when $A$ does not satisfy the required conditions.
\end{remark}

\begin{proof}
We can decompose,
\[ X = \bigcup_{f \in A_1} D_{+}(f) = \bigcup_{f \in A_1}  \Spec{A_{(f)}} \]
via the given assumptions. We know that,
\[ \struct{X}(n) |_{D_+(f)} = \wt{A(n)} |_{D_+(f)} = \wt{A[f^{-1}]_{n}} \]
However it is clear that $A[f^{-1}]_{n} = A[f^{-1}]_{0} \cdot f^n$ so this sheaf is free of rank $1$. 
\end{proof}

\begin{remark}
For $n = 1$ any element $f \in A_1$ gives a global section $f \in \Gamma(X, \struct{X}(1))$ such that $D_{+}(f) = X_s$ and hence,
\[ \struct{X}|_{D_+(f)} \xrightarrow{\sim} \struct{X}(1)|_{X_s} \]\end{remark}

\begin{corollary}
In the setting above, further assume that $A$ is generated by finitely many $f \in A_1$ as an $A_0$-algebra. Then for any quasi-coherent $\struct{X}$-module $\F$ if we set,
\[ M = \Gamma_*(X, \struct{X}(1), \F) \]
as a graded $A$-module via the map,
\[ A \to \Gamma_*(X, \struct{X}(1)) = \bigoplus_{n \ge 0} \Gamma(X, \struct{X}(n)) \]
Then we get, $\F = \wt{M}$. 
\end{corollary}

\begin{proof}
Tag 
\end{proof}

\section{Feb. 13}


\begin{definition}
Let $X$ be a scheme and $\L$ an invertivle $\struct{X}$-module . We say $\L$ is \textit{ample} if $X$ is quasi-compact and $\forall x \in X \exists n > 0, s\ in \Gamma(X, \L^{\otimes n})$ such that $X_{s}$ is affine and $x \in X_s$. 
\end{definition}

\begin{example}
Let $X = \Proj{A}$ where $A$ is generated by $A_1$ as a $A_0$-algebra and $A_1 = f_1 A_0 + \cdots + f_r A_0$. Then $\struct{X}(1)$ is invertible and $X$ is covered by $D_{+}(f_i)$ and is quasi-compact, and $D_+(f_i) = X_{s_i}$ where $s_i \in \Gamma(X, \struct{X}(1))$ is a section corresponding to $f_i$. 
\end{example}

\begin{proposition}
Let $X$ be quasi-compact and quasi-seperated for $\L \in \Pic{X}$ the following are equivalent,
\begin{enumerate}
\item $\L$ is ample
\item for all $\struct{X}$-modules $\F$ locally of finite type there exists $n > 0$ s.t. $\F \otimes_{\struct{X}} \L^{\otimes n}$ is generated by global sections.
\end{enumerate}
\end{proposition}

\begin{proof}
TAG 01Q3.
\end{proof}

\begin{lemma}
$\L$ is ample iff $\L^{\otimes n}$ is ample for any $n > 0$. 
\end{lemma}

\begin{lemma}
If $X$ is affine, and $L$ is invertible, and $s \in \Gamma(X, \L)$ then $X_s$ is affine. 
\end{lemma}

\begin{definition}
A scheme is noetherian if it has a finite open cover by spectra of noetherian rings. 
\end{definition}

\begin{remark}
It is equivalent to require that $X$ is quasi-compact and $\struct{X}(U)$ is noetherian. 
\end{remark}

\begin{lemma}
A locally noetherian scheme is quasi-seperated.
\end{lemma}

\begin{proof}
If $U, V$ are affines then $U \cap V$ is quasi-compact since every subspace of a noetherian space is quasi-compact.
\end{proof}

\begin{definition}
Let $X$ be a neotherian scheme. An $\struct{X}$-module $\F$ is \textit{coherent} if it is quasi-coherent and locally of finite type. 
\end{definition}

\begin{remark}
It is equivalent to require that locally on affine opens $\F |_U = \wt{M}$ for a finitely-generated module $M$. 
\end{remark}

\begin{remark}
The inclusion functors,
\[ \Coh{\struct{X}} \subset \QCoh{\struct{X}} \subset \shMod{\struct{X}} \]
are exact and preserved under extensions i.e. given a short exact sequence,
\begin{center}
\begin{tikzcd}
0 \arrow[r] & \F_1 \arrow[r] & \F_2 \arrow[r] & \F_3 \arrow[r] & 0
\end{tikzcd}
\end{center}
if $\F_1, \F_2$ are (quasi)-coherent  then $\F_2$ is also (quasi)-coherent. 
\end{remark}

\begin{lemma}
A scheme of finite type over a noetherian scheme is noetherian.
\end{lemma}

\begin{proof}
Since $f : X \to Y$ is finite type $f$ is quasi-compact but $Y$ is quasi-compact open so its preimage $X$ is also quasi-compact. Furthermore, for any affine opens $\Spec{A} = U \subset X$ and $\Spec{B} = V \subset Y$ such that $f(U) \subset V$ we get a ring map $B \to A$ of finite type so $B[x_1, \dots, x_n] \onto A$ and since $B$ is noetherian we see that $A$ is noetherian so $X$ is quasi-compact and covered by $\Spec{A}$ for noetherian rings $A$. 
\end{proof}

\begin{remark}
We want to prove the following theorem. Let $R$ be a noetherian ring, $X$ a projective (or proper) scheme over $R$ (then $X$ is noetherian), and $\F$ a coherent sheaf on $X$, then,
\[ H^i(X, \F) \]
is a finite $R$-module for any $i$ and $H^i(X, \F) = 0$ for $i > \dim{X}$. 
\end{remark}

\end{document}
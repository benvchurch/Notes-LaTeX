\documentclass[12pt]{article}
\usepackage{import}
\import{./}{Includes}

\begin{document}

\atitle{2}

\section{02CJ}

Let $k$ be a field.
We need to show that the following $k$-algebras,
\begin{enumerate}
\item 
The $k$-algebras $k[x_1, \dots, x_n]$ and $k[x_1, \dots, x_{n+1}]$ have dimension $n$ and $n + 1$ respectively. Therefore, these $k$-algebras can not be isomorphic.
\item  
Consider the variety $V = V(ab + cd + ef) \subset \A^6_k$. Then $V$ is singular at the origin (since the partial derivatives vanish). Therefore, $V \not\cong \A^5_k$ since $\A^5_k$ is smooth. Thus their coordinate rings, $k[a,b,c,d,e,f] / (ab + cd + ef)$ and $k[x_1, \dots, x_5]$ cannot be isomorphic. 

\item 
Consider the $k$-algebras $A = k[a, b, c, d, e, f]$ and $I = (ab + cd + ef)$ and $B = A / I = k[a,b,c,d,e,f] / (ab+ cd + ef)$. Then, $\dim{A} = 6$. We use the fact that
\[ \dim{A} \ge \dim{A / I}  + \height{I} \] 
$A$ is a domain so $(0)$ is prime. Since $I \supsetneq (0)$ any prime above $I$ has height at least $1$ so $\height{I} > 0$. Therefore,
\[ \dim{B} = \dim{A / I} \le \dim{A} - \height{I} < \dim{A} \] 
Thus $A \not\cong B$. 
\end{enumerate} 

\renewcommand{\C}{\mathbb{C}}

\section{0E9D}

Consider he affine curve $X$ given by $t^2 = s^5 + 8$ in $\C^2$ with coordinates $s,t$. Let $x \in X$ be the point $(1, 3)$ and let $U = X \setminus \{x\}$. Consider the function on $U$,
\[
f(s, t) =
\begin{dcases}
\frac{t + 3}{s - 1} & (s,t) \in D(s - 1)
\\
\frac{s^4 + s^3 + s^2 + s + 1}{t - 3} & (s,t) \in D(t - 3) 
\end{dcases}
\]
Because the sets $D(s - 1)$ and $D(t - 3)$ intersected with $U$ are standard opens, $f$ is a regular function as long as its case agree on the overlap. This is the case because on the variety $X$,
\[ t^2 = s^5 + 8 \implies t^2 - 9 = s^5 - 1 \implies (t - 3)(t + 3) = (s - 1)(s^4 + s^3 + s^2 + s + 1) \]
Therefore, on $D(s - 1) \cap D(t - 3)$ we have $s \neq 1$ and $t \neq 3$ so we can write,
\[ \frac{t + 3}{s - 1} = \frac{(t + 3)(t - 3)}{(s - 1)(t - 3)} = \frac{(s - 1)(s^4 + s^3 + s^2 + s + 1}{(s - 1)(t - 3)} = \frac{s^4 + s^3 + s^2 + s + 1}{t - 3} \]
Therefore, $f$ is regular on $U \subset X$. However, $f$ cannot be extened to a regular function on $\A^2_\C$ since it diverges approaching the point $(s, t) = (1,3)$. All global regular functions on $\A^1_\C$ are polynomials but $f$ cannot be written without a denominator. 

\section{0E9E}


Let $E \subset \C^n$ be a finite subset. Then since points are closed (in the Zariski topology) the set $U = \C^n \setminus E$ is open. Consider a regular function $f$ on $U$. Then there exists an open cover $\mathcal{V} = \{V_\alpha\}$ of $U$ such that on $V_\alpha$ we can write $f = g_\alpha / h_\alpha$ with $V_\alpha \subset D(h_\alpha) = \{p \in \C^n \mid h_\alpha(p) \neq 0 \}$. Since all open sets intersect (because $\C^n$ is irreducible), $U$ must be connected. Let $U_\alpha = \{ p \in D(h_\alpha) \mid f(p) = g_\alpha(p) / h_\alpha(p) \}$. Since $U_\alpha \supset V_\alpha$ we know that the sets $U_\alpha$ cover $U$. Furthermore if $W = U_\alpha \cap U_\beta$ is nonempty, then $f = g_\alpha / h_\alpha = g_\beta / h_\beta$ on the open set $W$ which implies that $W \subset V(g_\alpha b_\beta - g_\beta h_\alpha)$. However, since $\C^n$ is irreducible no nonempty open can be contained in a proper closed. Since the vanishing locus of a polynomial is, by definition, closed, either $V(g_\alpha h_\beta - g_\beta h_\alpha) = \C^n$, implying that $g_\alpha h_\beta = g_\beta h_\alpha$, or $W = \varnothing$. In the first case, $U_\alpha = U_\beta$ and in the second case $U_\alpha$ and $U_\beta$ are disjoint. Therefore, the sets $\{ U_\alpha \}$ form a partition of $U$. Since $U$ is connected there must be a single $U = U_\alpha$ meaning that $f = g / h$ with $h \neq 0$ on all of $U$.  
\bigskip\\
It remains to show that $h$ is a constant. We know that $D(h) \supset U$ and thus $V(h) \subset E$ which is a finite set of points. However, by Lemma \ref{poly_num_zeros}, any nonconstant polynomial over $\C^n$ for $n \ge 2$ has infinitely many zeros. Thus $h$ is constant so $f = g / h$ is a polynomial. 
\section{0E9F}

Let $X \subset \C^n$ be an affine cone. Consider $I(X) \subset \C[x_1, \dots, x_n]$ the ideal of functions vanishing on $X$. For each $p \in I(X)$ we can write it as a sum of homogeneous terms. Define the ideal $[p] = (t_1) + \cdots + (t_\ell)$ where $t_1, \dots, t_\ell$ are the homogeneous terms of $p$. Then $[p] \supset (p)$ and $[p]$ is a homogeneous ideal. Consider the ideal,
\[ I = \sum_{p \in I(X)} [p] \]
Clearly, $I \supset I(X)$ and $I$ is homogeneous. It remains to show that $I \subset I(X)$ in order to prove that $I(X)$ is a homogeneous ideal. Take $p \in I(X)$ then $p = t_1 + \cdots + t_\ell$. Also take some $a \in X$ then for all $\lambda \in \C$ we must have $\lambda a \in X$ and since $p \in I(X)$ we have,
\[ p(\lambda a) = \sum_{i = 1}^\ell t_i(\lambda a) = \sum_{i = 1}^\ell t_1(a) \lambda^{d_i} = 0 \]
Where $d_i$ is the degree of the homogeneous polynomial $t_i$. Thus, for fixed $a$, the function $p(\lambda a)$ is a polynomial in $\lambda$ so $p(\lambda a) = 0$ for all $\lambda \in \C$ implies that it is the zero polynomial. Therfore $t_i(a) = 0$ for each $i$. Since $a \in X$ was arbitrary, we have shown that $t_i \in I(X)$ for each $i$ and thus $[p] \subset I(X)$ which implies that $I = I(X)$ so $I(X)$ is homogeneous.  

\section{Additional Problem}

Let $f : \R \to \R$ be given by $f(x) = \frac{1}{1 + x^2}$. Then $f$ is a quotient of polynomials but not a polynomial (since its tylor series does not terminate at $x = 0$). 

\section{Lemmas}

\begin{lemma} \label{poly_num_zeros}
Suppose that $f : \C^n \to \C$ is a nonconstant polynomial. If $n \ge 2$ then $f$ has uncountably infinitely many roots.
\end{lemma}

\begin{proof}
Since $f$ is nonconstant it must have positive degree in some variable, WLOG let that variable be $x_1$. Then write,
\[ f(x_1, \dots, x_n) = \sum_{i = 0}^d q_i(x_2, \dots, x_n) x_1^i \]
If $q_1, \dots, q_n$ have a common zero at $(\tilde{x}_2, \dots, \tilde{x}_n)$ then $f(x_1, \tilde{x}_2, \dots, \tilde{x}_n) = 0$ for all uncountably infinitely many $x_1 \in \C$. Otherwise, for each $x_2, \dots, x_n$ the polynomial 
\[g_{(x_2, \dots, x_n)}(x_1) = f(x_1, x_2, \dots, x_n)\] has nonzero degree and thus, since $\C$ is algebraically closed, has a root at some $x_1 \in \C$. Thus, there exists a root of $g_{(x_2, \dots, x_n)}$ for each choice of $x_2, \dots, x_n \in \C^{n-1}$ and therefore $\forall x_2, \dots, x_n \in \C^{n-1} : \exists x_1 \in \C : f(x_1, x_2, \dots, x_n) = 0$ so the set of roots is at least the size of $\C^{n-1}$ which is uncountably infinite.
\end{proof}
\end{document}

\documentclass[12pt]{article}
\usepackage{import}
\import{../}{AlgGeoCommands}


\begin{document}

\newcommand{\can}{\mathrm{can}}
\newcommand{\cO}{\mathcal{O}}

\section{Mar. 30}

\begin{rmk}
Let $G$ be an $S$-group with structure map $\pi : G \to S$ and $e : S \to G$ the identity section. Then $\Omega_{G/S} \cong \pi^* e^* \Omega_{G/S}$ canonically. Indeed, consider the diagram,
\begin{center}
\begin{tikzcd}
G \times_S G \arrow[rdd, bend right, "\pi_1"'] \arrow[rd, dashed] \arrow[rrd, bend left, "m"]
\\
& G \times_S \pullback G \arrow[r] \arrow[d] & G \arrow[d]
\\
& G  \arrow[r] & S
\end{tikzcd}
\end{center}
The morphism $(\pi_1, m)$ is an isomorphism because it has $(\pi_1, m \circ (\iota \circ \pi_1, \pi_2))$ is an inverse. 
\end{rmk}

Let $E$ be an elliptic curve over a ring $R$ with $p = 0$. Then let $\omega$ be a basis for $\Omega^1_{E/R}$ last time: we defined the Hasse invariant $A(E, \omega)$ which is a modular form of 1evel $1$ and weight $p-1$. And $A(E,\omega) = 0$ iff $E$ is super singular. Furthermore, $A(\mathrm{Tate}(q), \omega_{\can}) = 1$. 

\subsection{Lifting to Characteristic zero}

For $p \ge 5$ the $q$-expansion of $E_{p-1}$ is $1$ mod $p$ and $E_{p-1}$ is a lift of $A(E, \omega)$ to $\Z_p$ any other lift differs by multiples of $p$. 


\subsection{Katz-Lubin Canonical Subgroup}

$E[p]$ is a finite flat group scheme over $\cO$ of order $p^2$ with $k$-resiude field.  Then consider the connected-\etale sequence,
\begin{center}
\begin{tikzcd}
0 \arrow[r] & E[p]^0_k \arrow[r] & E[p]_k \arrow[r] & E[p]^{\et}_k \arrow[r] & 0
\end{tikzcd}
\end{center}
If $E$ is ordinary then $E[p]^{\et}_{\bar{k}} \cong \Z / p \Z$ and $E[p]^0_{\bar{k}} \cong (\Z / p \Z)^\vee \cong \mu_p$. Then,
\[ E[p]^0 = \ker{(F_{\text{rel}} : E \to E^{(p)})} \] 

\section{April 1}

\subsection{Finishing the Calculation for the Formal Group}

\newcommand{\Spf}[1]{\mathrm{Spf}\left( #1 \right)}

Recally that $E / \cO$ for $\cO$ the ring of integers of a finite extension of $\Q_p$. We want the valuation of the points in $\hat{E}[p]$ therefore we need to draw the Newton polygon of $[p](T)$ then we want the coeffints of $[p](T)$.

\begin{lemma}
$[p](T) \equiv f(T^p) \mod p$. 
\end{lemma}

\begin{proof}
Here denote by $E$ the reduction on $E$ modulo $\lambda$ so that it is an elliptic curve in characteristic $p$.
Consider the relative Frobenius $F : E \to E^{(p)}$ is an isogeny of order $p$. Therefore it has a dual isogeny $V : E^{(p)} \to E$ (the Verschebung) such that $V \circ F = [p]$. Taking the formal completion $\hat{E} \cong \Spf{\cO[[T]]}$ we get that $F$ is given by $T \mapsto T^p$ and therfore,
\[ [p](T) = V(F(T)) = V(T^p) \]
so we win. 
\end{proof}

\begin{lemma}
The coefficient of $T^p$ in $[p](T)$ is congruent to $A(E, \omega)$ modulo $p$.
\end{lemma}

\begin{proof}
We have $[p](T) \equiv V(T^p) \mod p$. Therefore the coefficient of $T^p$ in $[p](T)$ is the coefficient of $T$ in $V$ which is $V'(0)$. 
\bigskip\\
Now $A(E, \omega) \omega^\vee = F_{\text{abs}}^* \omega^\vee$ this gives the same answer as $V^* \omega = A(E, \omega) \omega$. And we can compute this directly (DO THIS).
\end{proof}

Therefore letting $h(E) = v(A(E,\omega))$ we get that the newton polygon is controlled by the points $(1,1)$ and $(p, h(E))$ and $(p^2, 0)$ (WHY IS THE LAST ONE ZERO). Therefore the two cases are,
\[ h(E) < \frac{p}{p+1} \]
and otherwise.

\begin{thm}[Katz]
Let $R$ be any $p$-adically complete $\cO$-algebra and $E / R$ and elliptic curve with 
\[ h(E) \le r < \frac{p}{p+1} \]
at all $\bar{\cO}$-points. Then $[p](T) \in R[[T]]$ has factor $T^p - a T$ for $a$ topologically nilpotent in $R$ given maximal valuation roots we found for each $\bar{\cO}$-points. 
\end{thm}

\begin{rmk}
Therefore, over families we have a canonical subscheme,
\[ \Spf{R[[T]]/(T^p - a T)} \]
We want to show this is a subgroup. Writing $G(X,Y)$ for the formal group law of $\hat{E}$ we need to show that,
\[ G(X,Y)^p - a G(X, Y) \in R[[X,Y]]/(X^p - a X, Y^p - a Y) \]
This is finite free ove r$R$ of rank $p^2$ with basis $X^i Y^j$ for $0 \le i,j \le p-1$. Expaning in this basis,
\[ G(X,Y)^p - a G(X,Y) = \sum_{i,j} g_{ij} X^i Y^j \]
we need the $g_{ij}$ to vanish in $R$. It suffices to consider the universal case $X = X_1(N)$ the modular curve of level $\Gamma_1(N)$ whose points are pairs $(E, \psi_N)$ where $\psi_N : \mu_N \to E[N]$ is an embedding. Let $R$ be the rigid analytic function on the subset $\{ h(E) \le r \}$ of $X$. Then $R \embed R'$ where $R'$ is the rigid analytic functions on $\{ h(E) = 0 \}$. However, over $R'$,
\[ \Spf{R[[T]]/(T^p - a T)} = \hat{E}[p] \]
is a group because $E$ is ordinary.
\end{rmk}

\section{Rigid Analytic Geometry}

\begin{rmk}
Reference: Brian Conrad's AWS notes.
\end{rmk}

Consider $L / \Q_p$ and $\cO$ its ring of integers with maximal ideal $\m$ and residue field $k = \cO / \m$. The rigid analytic unit disk is $\mSpec{L\left< t \right>}$ where,
\[ L \left< T \right> = \{ \sum_n a_n t_n \in L[[t]] \mid | a_n | \to 0 \text{ as } n \to \infty \} \]
this means we consider power series that are $p$-adically convergent when $|t| \le 1$. Similarly, the polydisk is,
\[ \mSpec{L \left< t_1, \dots, t_n \right>} \]
Anything of the form $L \left< t_1, \dots, t_n \right> / I$ for an ideal $I$ is called an $L$-affinoid algebra. 
If $A$ is an $L$-affinoid algebra then $\mSpec{A}$ is an affinoid rigid analytic space. 

\subsection{Topologies}

We have the $p$-adic topology but it is far too fine (because then all spaces would be totally disconnected). Therefore we restrict possible open sets and open coversing to get a Grothendieck topology. 

\begin{example}
Consider $L \left< t \right> \left< X \right> /(r X - t)$ for $0 < |r| \le 1$. These are power series which are convergent when $\left| \frac{t}{r} \right| \le 1$ i.e. $|t| \le |r|$ so it represents a smaller unit disk called $D(r)$. This is an alowable open. 
\end{example}

\begin{example}
$L\left< t \right> \left< X \right> /(t X - r)$ converges when $|t| \ge |r|$ giving an anulus.
\end{example}

In general, if $A$ is $L$-affinoid and $a_1, \dots, a_n, a' \in A$ are elements with no common zero then,
\[ R =  A \left< \tfrac{a_1}{a'}, \dots, \tfrac{a_n}{a'} \right> = A \left< X_1, \dots, X_n \right> / (a' X_1 - a_1, \dots, a' X_n - a_n) \]
are affinoid and,
\[ \mSpec{R} = \{ x \in \mSpec{A} \mid \forall i : |a_i(x)| \le | a'(x) | \} \]
Now consider the following,
\[ D^\circ = \bigcup_n \{ |t| \le |r|^{\frac{1}{n}} \} \]
gives the open unit disk which is not affinoid.

\begin{defn}
$U \subset \mSpec{A}$ is \textit{admissible} if it has a set-theoretic cover by $U_i \subset \mSpec{A}$ affinoid s.t. every affinoid $U' \subset \mSpec{A}$ such that $U' \subset U$ is covered by finitely many $U_i$. 
\end{defn}

\begin{rmk}
$D^\circ \subset D$ is admissible by the maximum modulus principle: if $U' \subset D$ affinoid and $|t| < 1$ on $U'$ then there exists $|r| < 1$ such that $|t| \le |r|$ on $U'$.
\end{rmk}

\section{April 4}

Last time we defined:

\begin{enumerate}
\item affinoid $L$-algebras e.g. $L\left< t \right>$
\item affinoid rigid spaces, e.g. $M(L\left< t \right)) = D$
\item affinoid subdomains: closed subdisk of $D$ and anuli
\item admissible open sets include affinoid subdomains and also open subdisks.
\end{enumerate}

\subsection{Admissible Covers}

Let $\{ V_i \}$ be a collection of admissible open subsets. We say that $\{ V_i \}$ forms an admissible open cover of $V$ if,
\begin{enumerate}
\item $\bigcup V_i = V$ (we think this implies that $V$ is admissible)
\item for every affinoid subdomain $V' \subset V$ the induced cover of $V'$ has a finite refinement. 
\end{enumerate}

\begin{example}
Some positive examples:
\begin{enumerate}
\item finite union of affinoids is admissible and form an admissible cover of the union
\end{enumerate}
Here is a nonexample, consider,
\[ D^\circ \cup C = D \]
where,
\[ C = \{ |t| \le 1 \} \cap \{ |t| = 1 \} = \{ |t| = 1 \} \]
is affinoid. I claim this is not an admissible cover of $D$. Otherwise $D$ would have a finite cover by affinoids contained in $D^\circ$ and $C$ implies that $D^\circ$ is covered by finitely affinoids inside $D^\circ$ which is impossible by the maximum modulus principle. 
\end{example}

\begin{rmk}
This does not form a topology because we see that the union of admissible opens does not give an admissible cover so instead this is a $G$-topology (version of a Grothendieck topology on the category of subsets where coverings are a subcollection of the set-theoretic covers).
\end{rmk}


\subsection{Rigid Spaces}

\begin{defn}
A rigid space is a locally ringed $G$-space which is locally isomorphic to an affinoid space with its sheaf of rigid analytic functions (arising from the Tate coordinate ring). 
\end{defn}

\begin{example}
$\A^1_L$ is given by gluing balls of increasing radius (or alternatively anuli $|a| \le r \le |b|$ for sequences $a \to 0$ and $b \to 1$ to get the anulus $0 < r < 1$. 
\end{example}

\subsection{Raynaud's Generic Fiber}

Let $X / L$ be a scheme over $L$ a finite extension of $\Q_p$. Then we want to define a rigid space $X^\an$ with a map $X^\an \to X$ of locally ringed spaces with an $L$-algebra structure sheaf an a universal property. 
\bigskip\\
For example we want $\A^1_L \mapsto \A^1_L$ as a rigid space. 
\bigskip\\
Let $X$ be a $\cO$-scheme then the generic fiber $X_L$ has $X_L^\an$ but alternatively we can form a formal scheme $\X$ by compleing at $\m$ this gives the Raynaud generic fiber.

\newcommand{\rig}{\mathrm{rig}}

\begin{example}
$X = \A^1_{\cO} = \Spec{\cO[t]}$ and $\X = \Spf{\widehat{\cO[t]}} = \Spf{\cO\left< t \right>}$ completed at the maximal ideal $\m \subset \cO$.
Therefore,
\[ X^{\rig} := \mSpec{L \ot_{\cO} \cO \left< t \right>} = \mSpec{L \left< t \right>} = D \]
which is not all of $\A^1_L$. 
\end{example}

\begin{thm}
If $X$ is proper over $\cO$ then $X^{\an}_L = X^{\text{rig}}$. 
\end{thm}

\begin{example}
Consider $X = \P^1_{\cO}$ then $X^\an_L$ is the gluing of two copies of $(\A^1_L)^\an$ but $X^{\rig}$ is two copies of $(\A^1_{\cO})^\rig = D$ glued and these give the same answer.
\end{example}


\begin{defn}
There is a specialization map $\mathrm{sp} : X^\rig \to \X$ defined as follows. Cover $X^\rig$ by $\mSpec{L \ot_{\cO} A_i}$ where $\Spf{A_i}$ is a finite affine open covering of $\X$ with $A_i$ an $\m$-adically complete $\cO$-algebra. Then we map $\mSpec{L \ot_{\cO} A_i} \to \Spf{A_i}$ via for a point $x \in \mSpec{L \ot_{\cO} A_i}$ this fives $L \ot_{\cO} A_i \to L(x)$ (the residue field) then $A_i \to \cO(x)$ (a valuation ring) then we get $A_i \m \to \cO(x) / \m = k(x)$ which is a point of $\Spec{A_i / \m}$ giving a point of $\Spf{A_i}$. 
\end{defn}

\begin{prop}[Berthelot]
Let $\xi \subset \X$ be a closed subscheme. Then $\hat{X}^\xi$ the formal completion of $\X$ along $\xi$ we have $\mathrm{sp}^{-1}(\xi) \subset \X^\rig$ is admissible and equals $(\hat{\X}^\xi)^\rig$. 
\end{prop}

\begin{example}
Let $X = \A^1_{\cO}$  or $\P^1_{\cO}$ with coordinate $t$ at $0$ let $\xi \subset \X$ from $\overline{0} \in X_k$ then locally $\X$ is $\Spf{\cO\left< t \right>}$ and,
\[ \hat{X}^\xi = \Spf{\cO\left< t \right> \text{ completed at } t} = \Spf{\cO[[t]]} \]
Then,
\[ (\hat{X}^\xi)^\rig = \mSpec{L \ot_{\cO} \cO[[t]]} = D^\circ \]
because $L \ot_{\cO} \cO[[t]]$ is power series whose coefficients have bounded below valuations. Then $\mathrm{sp}^{-1}(\overline{0})$ is points of distance $<1$ from $0$ which is the same points that reduce to $\overline{0}$ mod $\overline{\m}$. 
\end{example}

\section{April 6}

We want to apply these results to modular curves. Let $X_1(N)$ be the scheme theoretic modular curve over $\Z_p$ for $p \ndivides N$ for $N \ge 5$ where the points parametrize,
\[ (E, \psi_E : \mu_N \embed E[N]) \]
Then write $X_0(p)$ for the scheme theoretic modular curve over $\Z_p$ whose points parametrize,
\[ (E, \psi_N, C) \text{ quad } C \subset E[p] \]
[LOOK AT KATZ-MAZUR Ch. 12-13]  

\subsection{The Story mod $p$}

We reduce modulo $p$ to get $X_1(N)$ and $X_0(p)$. Then there are two irreducible components of $X_0(p)$,
\[ X_0(p)^{\can} = \{ (E, \psi_N, \ker{(F : E \to E^{(p)})}) \} \]
and the \etale locus
\[ X_0(p)^{\et} = \{ (E, \psi_N, \ker{V}) \} \] 
these intersect at the supersingular locus. The intersections are normal crossings with completed local ring,
\[ k[[x,y]] / (xy) \] 
The natural map $\pi_1 : X_0(p) \to X$ given by $(E, \psi_N, C) \mapsto (E, \psi_N)$ is bijective on each component and is ramified at the supersingular locus. However, there is a second map $\pi_2 : (E, \psi_N, C) \mapsto (E/C, \bar{\psi}_N)$. Then $\pi_1$ has degree $1$ on $X_0(p)^\can$ and degree $p$ on $X_0(p)^\et$. 
\bigskip\\
On the other hand, $\pi_2$ on $X_0(p)^\can$ is $E \mapsto E / \ker{F}$ which is degree $p$ i.e. $F : E \to E^{(p)}$ on universal $E$. Then $\pi_2$ on $X_0(p)^\et$ is degree $1$. This is because for $D \neq \ker{F}$ we want to recover $(E, D)$ from $E' = E / D$. Consider,
\begin{center}
\begin{tikzcd}
E \arrow[r, "\pi"] \arrow[d, "F"] & E' \arrow[d, "F"]
\\
E^{(p)} \arrow[r] & (E')^{(p)} 
\end{tikzcd}
\end{center} 
Then $\pi(\ker{F_E}) \subset \ker{F_{E'}}$ and actually $\pi(\ker{F_E}) = \ker{F_{E'}}$ because $D$ is disjoint from $\ker{F_E}$ (because it is a distinct prime-order subgroup). Then $\ker{F_E}$ and $D$ then span $E[p]$ so we see that $\ker{F_E'} = \ker{F_E} / D = E[p] / D \subset E'[p]$. Therefore, \[ E' / \ker{F_{E'}} = (E/D) / (E[p] / D) \xrightarrow{p} E \]
is an isomorphism. Furthermore,
\[ E'[p] / \ker{F_{E'}} = (E/D) [p] / (E[p] / D) \iso D \]
(WHY)
Therfore each component of $X_0(p)$ is isomorphic to $X$ but via different projections,
\begin{align*}
\pi_1 : X_0(p)^\can \iso X
\\
\pi_2 : X_0(p)^\et \iso X 
\end{align*} 
and each $\pi_i$ is degree $p$ on the other component so each is degree $p+1$. 

\begin{rmk}
We can describe $X_0(p) = X_0(N;p)$ as elliptic curves $E$ with $N$ level structure and a degree $p$ isogeny $E \to E'$ then $\pi_1$ and $\pi_2$ are forgetting this map and mapping to the source and target respectively.
\end{rmk}

\renewcommand{\ss}{\mathrm{ss}}

\subsection{How to Draw the Rigid Spaces}

Let $k = \overline{\FF}_p$ and $\cO = W(\overline{\FF}_p)$. Given $X_k$ we have $X^\rig$ which is the inverse of $X_k$ under the specialization map. Then the preimage of each point is an open disk. Then $X^{\ss}$ is a union of open disks and $X^{\ord}$ is affinoid. 
\bigskip\\
For $X_0(p)$ we have two components $X_0(p)^\can$ and $X_0(p)^\et$ then the inverse of the specialization map gives two rigid spaces $X_0(p)^\can$ and $X_0(p)^\et$ which are affinoid after removing the supersingular points  (both are isomorphic to $X^{\ord}$). Then the local ring at the supersingular points $k[[x,y]] /(xy)$ gives the local ring $\cO[[x,y]]/(xy-p)$ which is an open anulus $p^{-1} < r < 1$. Therefore, $X_0(p)^\can$ and $X_0(p)^\et$ are glued together by ``tubes'' i.e. anuli.  

\begin{rmk}
We could consider the same construction for,
\[ X_0(p^n) = \{ (E, \psi_N, C^n) \text{ with } C^n \subset E[p^n] \text{ cyclic of order } p^n \} \]
Then the special fiber has $n+1$ components each isomorphic to $X$ that all collide at the supersingular locus. Then the $a^{\text{th}}$ component is,
\[ X_a = \{ C^n \supset \ker{F^a} \text{ but } C^n \not\supset \ker{F^{a+1}} \} \]
Then the local ring in characteristic $p$ at the supersingular points is in Katz-Mazur. However, the generic fiber construction to get a rigid space is not so nice (although it probably is still smooth whatever that means).  
\end{rmk}

\begin{rmk}
Remember we are writing $X = X_1(N)$ so $X_0(p)$ has level structure $\Gamma_1(N) \cap \Gamma_0(p)$ (global functions are modular forms of this level). 
\end{rmk}

\section{April 8}

\subsection{Coordinates on supersingular disc in $X$}

Under the inverse of the specialization map $e \in X_k$ is mapped to some open disk $D_e$ in $X^{\rig}$. If the local ring at $e$ is $k[[t]]$ and $t \mod p$ has a simple root at $e$ then the local ring for the disk is $\cO[[t]]$. 
\bigskip\\
Any other $t'$ defined over $W(\overline{\FF}_p)$ is $p a(t) + t u(t)$ for $u \in \cO[[t]]^\times$ and $a(t) \in \cO[[t]]$. Then consider,
\[ v(p a(t) + t u(t)) = v(t) \]
if $v(t) < 1$ since $v(u) = 0$. Therefore, $h : \overline{D}_e \to [0, 1]$ defined by $(E, \psi_N) \mapsto \min \{1 , v(t(E, \psi_N)) \}$ is well-defined. 
\bigskip\\
Fix a generator $\omega$ of $\Omega^1_{\E / X}$ locally near $D_e$ then any lift of $A(E, \omega)$ is a function which is defined over $W(\overline{\FF}_p)$ reduces to $0 \mod p$ at $e$. 

\begin{rmk}
For $p = 2,3$ no lift of $A(E, \omega)$ to modular form integrally. Fine, pick any $t$ coordinate on $D_e$ then $X^{\ord}$ define $h(E) = \{ 1, v(t(E)) \}$. 
\end{rmk}

There is a section of ,
\[ \pi_1^{-1}(X_0(p)) \to X(< \tfrac{p}{p+1}) \]
sending $(E, c) \mapsto E$ given by $E \mapsto (E, H_{\can})$. 

\subsection{Coordinates on supersingular anulus in $X_0(p)$}

Consider $X_0(p)^{\can}_k$ and $X_0(p)^{\et}_k$ then the inverse specialization map takes an intersection point $e$ to an anulus (we saw this last time) $A_e$ connecting the canonical rigid locus to the \etale rigit locus. 

\begin{lemma}[Goren-Kassae]
with some conditions,
\[ \pi_1 : X_0(p) \to X \]
in coordinates is $\cO[[t]] \to \cO[[x,y]] / (xy - p)$ where $u,v \in (\cO[[x,y]] / (xy - p))^\times$ given by $t \mapsto u x + v y^p$. 
\end{lemma}

\begin{defn}
$v(E, C) = v_p(x(E, C)0$ for any such $x$ and $v(x) < \frac{p}{p+1}$ if and only if $v(y) > \frac{1}{p+1}$ if and only if $v(y^p) > \frac{p}{p+1} > v(x)$. Therefore,
\[ v(t) = v(ux + v y^p) = v(x) \]
\end{defn}

\subsection{Interpretation of the Valuation}

Let $G / S$ be a group scheme then let $\omega_{G/S} = \id_G^* \Omega^1_{G/S}$. For example $\omega_{E/S}$ is a line bundle. 

\begin{prop}
$C \subset E[p]$ be order $p$ defined over $\cO = \overline{\cO}$. Then,
\[ \omega_C \cong 
\begin{cases}
(\cO / p^{1- h(E)} \cO) \d{T} & C \text{ canonical }
\\
(\cO / p^{h(E) / p} \cO & C \text{ not canonical} 
\end{cases}\] 
\end{prop}

\begin{rmk}
If $C$ is \etale then $h(E) = 0$ because it means that $E$ must be ordinary and also $C$ is not canonical so we indeed get $\omega_C = 0$ as it should be. 
\end{rmk}

\begin{proof}
Katz construction of the canonical subgroup $H_{\can}$ goes through,
\[ \hat{H_{\can}} = \Spf{\cO[[T]] / (T^p - a T)} \]
Then $v(n) = 1 - h(E)$ and $v(\text{nonzero point}) = \frac{1 - h(E)}{p-1}$. Oort-Tate show that any finite flat group scheme $C / \cO$ of order $p$ is isomorphic to
\[ \Spec{\cO[T] / (T^p - a T)} \]
for some $a$. Then,
\[ \Omega^1_{C/\cO} = \cO[T] / (p T^{p-1} - a) \d{T} \]
and thus,
\[ \id^*_C \Omega^1_{C/\cO} \cong \cO / a \]
Then we apply the conormal exact sequence,
\begin{center}
\begin{tikzcd}
0 \arrow[r] & \omega_E \arrow[r, "p"] & \omega_E \arrow[r] & \omega_{E[p]} \arrow[r] & 0
\end{tikzcd}
\end{center}
where $\omega_{E[p]} = \omega_E / p \omega_E$. Likewise we have a sequence,
\begin{center}
\begin{tikzcd}
0 \arrow[r] & \omega_{E[p]/C} \arrow[r, "\times a"] & \omega_{E[p]} \arrow[r] & \omega_C \arrow[r] & 0
\end{tikzcd}
\end{center}
Then the image is $a \omega_E / p \omega_E \cong \omega_E / a \omega_E$. HMMMM 
\end{proof}

\section{April 11}

Continuing from last time






$A_e$ has completed local ring $\struct{}[[x,y]]/(xy - p)$

\begin{prop}
The valuation,
\[ v(E, C) = v_p(x(E, C)) = 
\begin{cases}
h(E) & C \text{ is canonical}
\\
1 - \frac{h(E)}{p} & C \text{ not canonical} 
\end{cases} \]
\end{prop}

\subsection{Atkin-Lehener Involution}

\begin{defn}
The automorphism $w : X_0(p) \to X_p(p)$ taking,
\[ w(E, C) = (E/C, E[p]/C) \]
We showed this is an involution when we computed the degree of the projection $\pi_2$. Explicitly,
\[ (E/C)/(E[p]/C) \cong E \quad \text{ and } \quad (E/C)[p]/(E[p]/C) \cong C \]
\end{defn}

\begin{prop}
$w$ satisfies,
\begin{enumerate}
\item $(E, C) \in X_0(p)^{\can, \ord} \iff w(E,C) \in X_p(p)^{\et, \ord}$
\item $v(w(E, C)) = 1 - v(E, C)$ (so why does Buzzard need the analytic continuation arguments!)
\item $\pi_1 \circ w = \pi_2$ and $\pi_2 \circ w = \pi_1$.  
\end{enumerate}
\end{prop}

\subsection{Serre $p$-adic Modular Forms}

\begin{defn}
Let $\omega = e^* \Omega^1_{\E/X}$ with $X = X_1(N)$. A classical modular form of weight $k$ and level $N$ is an element of,
\[ H^0(X, \omega^{\ot k}) \]
\end{defn}

\begin{example}
The Hasse invariant $h \in H^0(X_{\FF_p}, \omega^{p-1})$ is a modular form. For $p \ge 5$ there is a lift,
\[ E_{p-1} \in H^0(X_{\Z_p}, \omega^{p-1}) \]
\end{example}

\begin{prop}
The $p$-adic closure of,
\[ \bop_k H^0(X, \omega^k) \]
is,
\[ R = \widehat{\bop}_k H^0(X^{\ord}, \omega^k) \]
\end{prop}

\begin{rmk}
For a $\Z$-algebra $R$ the $p$-adic closure is the completion of $R$ at the ideal $p R$. 
\end{rmk}

\begin{proof}
If $f \in R$ then,
\[ f \mod p \in \bop_k H^0(X_{\overline{\FF}_p}^{\ord}, \omega^k) \]
is a rational section over $X_{\overline{\FF}_p}$ with finite poles at supersingular points. Let $A \in H^0(X, \omega^{(p-1)})$ lifting the Hasse invariant. Then $A \equiv 0 \mod p$ on supersingular points so $f A^{p^n}$ has no poles for $n$ sufficiently large. But $A \equiv 1 \mod p$ on $X^{\ord}$ and therefore $A^{p^n} \equiv 1 \mod p^n$ on $X^{\ord}$ so we see that $| f - f A^{p^n}|$ is small. 
\end{proof}

\begin{defn}
Serre $p$-adic modular forms of weight $k$ is an element of $H^0(X^{\ord}, \omega^k)$ and hence on $X_0(p)^{\can, \ord}$. 
\end{defn}

\subsection{Hecke $U_p$ Operator}

$U_p \acts H^0(X_0(p), \omega^k)$ via,
\[ U_p f(E, \psi_N, C) = \frac{1}{p} \sum_{D \neq C} \pi^*_{E \to E/D} f(E/D, \bar{\psi}_N, C/D) \]
where $f(E/D, \bar{\psi}_N, C/D)$ is an invariant differential on $E/D$ and thus we can pull it back to $E$. I think this does not make sense in characteristic $p$ we need $U_p = \pi_1^* \tr{\pi_2}$. 

\begin{prop}
On $q$-expansions at cusps in $X_0(p)^{\can}$,
\[ U_p : \sum_n a_n q^n \mapsto \sum_n a_{np} q^n \]
\end{prop}

\section{April 13}

\subsection{The $U_p$ operator}

\begin{defn}
$U_p \acts H^0(X_0(p), \omega^k)$ given by,
\[ U_p f(E, \psi_N, C) = \frac{1}{p} \sum_{D \neq C} \pi^* f(E/D, \bar{\psi}_N, C/D) \]
\end{defn}

\begin{prop}
On cusps in $X_p(p)^{\can}$,
\[ U_p : \sum_n a_n q^n \mapsto \sum_n a_{np} q^n \]
\end{prop}

\newcommand{\Tate}{\mathrm{Tate}}

\begin{proof}
Let $T = \mathrm{Tate}(q)$ the order-$p$ subgroups with $\zeta \neq 1$ but $\zeta^p = 1$ implies $\left< \zeta \right> = \mu_p$ get the canonical subgroup $\ker{[p]}$ and $\left< \zeta^i q^{\frac{1}{p}} \right>$. Then,
\[ \mathrm{Tate}(q) / \left< \zeta^i q^{\frac{1}{p}} \right> \cong \Gm / \left< \zeta^i q^{\frac{1}{p}} \right> \cong \mathrm{Tate}(\zeta^i q^{\frac{1}{p}}) \]
Then $\pi_1 : \Tate(q) \to \Tate(\zeta^i q^{1/p})$ preserves canonical subgroups. Then,
\[ f(\Tate(q), \psi_N, \left< \zeta \right>) = \left( \sum_n a_n q^n \right) \omega_{\can}^k \]
Therefore,
\begin{align*}
U_p f(\Tate(q), \psi_N, \left< \zeta \right>) & = \frac{1}{p} \sum_{i = 0}^{p-1} f(\Tate(\zeta^i q^n), \bar{\psi}_N, \left< \zeta^i \right>) = \frac{1}{p} \sum_{n = 0}^{p-1} \sum_{n} a_n (\zeta^i q^{1/p})^n \omega_{\can}^k 
\\
& = \sum_n a_n \left( \frac{1}{p} \sum_{i = 0}^{p-1} \zeta^{in} \right) q^{n/p} \omega_{\can}^k
\\
& = \sum_k a_{pk} q^k 
\end{align*}
\end{proof}

If $(E, C) \in X_0(p)^{\can, \ord}$ and $D \neq C$ then $C/D$ is canonical in $E / D$ (because Frob commutes with mod $D$). Therefore $(E/D, C/D) \in X_0(p)^{\can, \ord}$. Therefore,
\[ U_p \acts H^0(X_0(p)^{\can, \ord}, \omega^k) \cong H^0(X^{\ord}, \omega^k) \]
because $\pi_1 : X_0(p)^{\can, \ord} \to X^{\ord}$ gives an isomorphism.

\subsection{An Eigenvalue Issue}

\begin{defn}
For $f \in H^0(X^{\ord}, \omega^k)$ define,
\[ V_p f(E, \psi_N) = \pi^*_{E \to E/H_{\can}} f(E/H_{\can}, \bar{\psi}_N) \]
Then,
\[ V_p (\sum_n a_n q^n) = \sum_n a_n q^{np} \implies U_p V_p = \id \]
because for the tate curve $\Tate(q) / \left< \zeta \right> \cong \Tate(q^p)$. 
\end{defn}

For $f$ weight $k$ then define
\[ g = (1 - V_p U_p) f \]
implies that,
\[ U_p g = (U_p - V_p V_p U_p) f = 0 \]
Then for any $\lambda \in \CC_p$ for $v(\lambda) > 0$ then we can define,
\[ f_\lambda = \sum_{n = 0}^\infty \lambda^n V_p^n g \]
but applying $U_p$ we get,
\[ U_p f_\lambda = \sum_{n = 0}^\infty \lambda^n U_p V_p^n g = U_p g + \sum_{n = 1}^\infty \lambda^n V_{p}^{n-1} g = 0 + \lambda \sum_{n = 0}^\infty \lambda^n V_p^{n-1} g = \lambda f_\lambda \]
and therefore $U_p$ has an eigenform for every eigenvalue $\lambda \in \CC_p$. 

\subsection{Katz $p$-adic modular forms}

\begin{rmk}
To fix this problem we introduce overconvergent modular forms. 
\end{rmk}

\begin{defn}
A Katz \textit{overconvergent} modular form of radius $r$ and weight $k$ is an element of $H^0(X(\le r), \omega^k)$ where,
\[ X(\le r) = h^{-1}(\{ t \le r \}) \]
where this extends into the supersingular locus with Hasse invariant bounded by $r$. 
\end{defn}

\begin{rmk}
When we say a form is overconvergent without specifying $r$ we mean overconvergent for some $r > 0$. 
\end{rmk}

\begin{prop}
Let $H$ be the canonical subgroup of $E$ if it exists. Then,
\begin{enumerate}
\item $h(E) < \frac{1}{p+1}$ then $h(E/H) = p h(E) < \frac{p}{p+1}$ so $E/H$ has a canonical subgroup and is not equal to $E[p]/H$
\item $h(E) = \frac{1}{p+1}$ then $h(E/H) \ge \frac{p}{p+1}$ and $E/H$ has no canonical subgroup
\item if $\frac{1}{p+1} < h(E) < \frac{p}{p+1}$ then $h(E/H) = 1 - h(E) < \frac{p}{p+1}$ so $E/H$ has a canonical subgroup and it is $E[p]/H$
\item if $h(E) \ge \frac{p}{p+1}$ and $C \subset E$ has order $p$ then $h(E/C) = \frac{1}{p+1}$ and $E[p]/C$ is the canonical subgroup in $E/C$
\item $h(E) < \frac{p}{p+1}$ and $C \neq H$ has order $p$ then $h(E/C) = h(E)/p$ and $E[p]/C$ is canonical in $E/C$. 
\end{enumerate}
\end{prop}

\begin{proof}
Pages of newton polygon calculations (Katz 73 p-adic properties of modular forms and modular schemes). Choose coordinates $\wh{E} \cong \Spec{\cO[[T]]}$ and $\wh{E/C} \cong \Spec{\cO[[T']]}$. Then,
\[ \wh{E} \to \wh{E/C} \]
is given by,
\[ T \mapsto \prod_{c \in C} G(T, c) \]
where $G(X,Y)$ is the formal group law on $\wh{E}$. This is zero on anything in $C$ since invariant undertranlation by $C$. Compute vals of roots of $[p]$ on $\wh{E/C}$ in terms of vals of roots of $[p]$ on $\wh{E}$. 
\end{proof}

\begin{prop}
When $r < \frac{p}{p+1}$ we have $U_p : H^0(X(\le r), \omega^k) \to H^0(X(\le pr) \cap X(<\tfrac{p}{p+1}), \omega^k)$
\end{prop}

\begin{proof}
In our notation let $X = X_0(p)$. Then,
\[ U_p f(E, C) = \frac{1}{p} \sum_{D \neq C} \pi^* f(E/D, C/D) \]
Then $C$ is canonical and $D$ is not canonical implies $h(E/D) = h(E) / p$ and $C/D$ is canonical in $E/D$. So if $h(E,C) \in X(\le pr)$ then $h(E/D, C/D) \in X(\le r)$ for all $D \neq C$. 
\end{proof}

\begin{rmk}
Now we have,
\[ \bigcup_{n \ge 0} H^0(X(\le \tfrac{1}{p^n(p+1)}), \omega^k) \]
and $U_p :  H^0(X(\le \tfrac{1}{p^n(p+1)}), \omega^k) \to  H^0(X(\le \tfrac{1}{p^{n-1}(p+1)}), \omega^k)$ is compact (I THINK!!)
it will follow that nonzero eigenvalues of $U_p$ ofrm a countable sequence $\lambda_1, \lambda_2, \lambda_3, \dots$ and $|\lambda_i| \to 0$ as $i \to \infty$. If $U_p f = \lambda_i f$ then $f$ has ``finite-slope'' if $\lambda \neq 0$ called a slope from the newton polygon of $\det{(I - \lambda U_p)}$.
\end{rmk}

\section{April 15}

First here is why the two definitions of a modular form agree. If $F(E, s)$ is a function on $E$ with a generator of $\omega_E$ with $F(E,\lambda s) = \lambda^{-s} F(E, s)$ then $F(E, s) s^k$ is a well-defined section because $F(E, \lambda s) (\lambda s)^k = F(E, s)$. 

\subsection{Eigenfunctions as Overconvergent Forrms}

\begin{prop}
If $f$ is overconvergent and $U_p f = \lambda f$ for $\lambda \neq 0$ then $f$ extends to $X(< \frac{p}{p+1})$.
\end{prop}

\begin{proof}
We write,
\[ f = \frac{1}{\lambda} U_p f = \frac{1}{\lambda^n} U_p^n f \]
for all $n$ but $U_p^n f$ converges on all of $X(< \frac{p}{p+1})$ as $n \to \infty$. 
\end{proof}

\begin{rmk}
Here $X$ refers to $X_1(N)$. However, using canonical subgroups there is an isomorphism,
\[ X(< \tfrac{p}{p+1}) \cong X_0(p)(< \tfrac{p}{p+1}) \]
\end{rmk}

\begin{prop}
We have $H^0(X_0(p^m), \omega^k) \embed H^0(X(<\frac{1}{p^{n-2}(p+1)}), \omega^k)$
\end{prop}

\begin{proof}
Let $h(E) < \frac{1}{p^{m-2}(p+1)}$ for $m \ge 1$ then $E$ has $H^1 \subset E$ canonical. We want to find higher canonical subgroups. Then $h(E/H') = p h(E) < \frac{1}{p^{m-3}(p+1)}$ so if $m \ge 2$ then $E/H^1$ has a canonical subgroup $H' \subset E / H^1$ and let $H^2$ be its preimage in $E$ under $E \to E / H^1$ which is cyclic because $H' \neq E[p] / H^1$. Continue inductively to produce subgroups for all $i \le m$ we set $H^i$ the preimage of $H^{i-1}(E/H^1)$ under $E \to E/H^1$. 
\bigskip\\
Therefor we get a component of $X_0(p^m)^{\can, \ord}$ given by $(E, C^m)$ and,
\[ X(< \frac{1}{p^{m-2}(p+1)}) \to X_0(p^m) \]
so for $f \in H^0(X_0(p^m), \omega^k)$ we restrict along this embedding $X^{\ord} \to X_0(p^m)^{\can, \ord}$. 
\end{proof}

\subsection{Going Past the Canonical Locus}

If $v(E, C) < \frac{p}{p+1}$ we know $v(E, C) = h(E)$ and $C$ is canonical by definition. If $D \neq C$ then $v(E/D, C/D) = h(E/D) = h(E)/p = v(E,C)/p$. If $v(E, C) = \frac{p}{p+1}$ then $h(E) \ge \frac{p}{p+1}$ and thus $v(E/D, C/D) = \frac{1}{p+1}$ for $D \neq C$. 
\bigskip\\
Now suppose,
\[ \frac{p}{p+1} < v(E, C) < 1 - \frac{1}{p(p+1)} \]
then $C$ is not canonical and,
\[ h(E) = p (1 - v(E, C)) \]
therefore,
\[ \frac{1}{p+1} < h(E) < \frac{p}{p+1} \]
so have $H \subset E$ canonical $H \neq C$. Then $v(E/H, C/H) = v(E/H, E[p]/H) \in (\tfrac{1}{p+1}, \tfrac{p}{p+1})$. For $D \neq C$ we have $v(E/D, C/D) = h(E)/p < \frac{1}{p+1}$. Consider the interval,
\[ 1 - \frac{1}{p^n (p+1)} \le v(E, C) < 1 - \frac{1}{p^{n+1}(p+1)} \]
for $n \ge 1$. Then $C$ is not canonical so,
\[ \frac{1}{p^n (p+1)} < h(E) \le \frac{1}{p^{n-1}(p+1)} \]
therefore either $H \subset E$ is canonical $H \neq C$ so then mutiplying by $p$,
\[ h(E/H) > \frac{1}{p^{n-1}(p+1)} \]
and $C/H$ is not canonical so,
\[ v(E/H, E/H) = 1 - \frac{h(E/H)}{p} < 1 - \frac{1}{p^n(p+1)} \]
so the valuation decreases into the next interval. Similarly, if $D \neq H,C$ then diving by $p$
\[ h(E/D) \le \frac{1}{p^{n}(p+1)} \] 
and $C/D$ is canonical so,
\[ v(E/D, C/D) = h(E/D) < h(E/D) = \frac{1}{p^n(p+1)} \]
is much smaller than these intervals so we get the following.

\renewcommand{\succ}{\mathrm{succ}}
\newcommand{\Qbar}{\overline{\Q}}

\begin{prop}
Let $v \in (0, 1)$ define,
\[ \succ(v) = 
\begin{cases}
p v & 0 < v \le \frac{1}{p+1}
\\
1 - \frac{1}{p^n(p+1)} & 1 - \frac{1}{p^{n-1}(p+1)} \le v < 1 - \frac{1}{p^n(p+1)}
\end{cases} \]
Then,
\[ U_p : H^0(X_0(p)(\le v), \omega^k) \to H^0(X_0(p)(\le \succ(v)), \omega^k) \]
Therefore, every application of $U_p$ increases the radius of overconvergence since $\succ(v) > v$ and furthermore $\succ^n(v) \to 1$ as $n \to \infty$. 
\end{prop}

\begin{cor}
If $f$ is overconvergent and $U_p f = \lambda f$ for $\lambda \neq 0$ then $f$ extends to 
\[ X_0(p)(\le 1) = X_0(p) \sm X_0(p)^{\et, \ord} \]
\end{cor}

\begin{rmk}
The reason Buzzard needs this is to make things work at higher level.
\end{rmk}

\subsection{Higher Levels at $p$}

Consider 
\[ X_0(p^m)_{\overline{\FF}_p} = \{ (E, C^m) \mid E \text{ ord}, C^m \supset \ker{F^a} \text{ not } \ker{F^{a+1}} \} \]
Then $C^a = C^m[p^a]$ and 
\[ X_0(p)^{a, \ord}_{\rig} = \{ (E, C^m) \mid C^a \text{ level-a canonical subgroup of } E \text{ a maximal} \} \]
Consider the map  $\pi_1 : X_0(p^m) \to X_0(p)$ sending $(E, C^m) \mapsto (E, C^1)$ 


\section{April 18}

Consider $X_0(p^m)$ has an Atkin-Lehner involution $w^m : (E, C^m) \mapsto (E/C^m, E[p^m]/C^m)$. The Atkin-Lehner involution satisfies $\pi_2 \circ w^m = w \circ \pi_1$ where,
\begin{align*}
\pi_1 : & (E, C^m) \mapsto (E, C')
\\
\pi_2 : & (E, C^m) \mapsto (E/C^{m-1}, C^m /C^{m-1}) 
\end{align*}

--------------------------

but just need for $i = m,\dots, 1$
\[ U_i = X_0(p^m)^{\ss} \cup \bigcup_{a = m, \dots, m-i} X_0(p^m)^{a, \ord} \]
admissible and connected,
\[ U_i^{\ord} = \{ C^m \text{ canonical up to at least } m - i \} \]
then $U_i \subset U_{i+1}$. Buzzard: continue forms to $U_{m-1} = X_0(p^m) \sm X_0(p^m)^{0, \ord}$. 

\begin{defn}
$U_p = H^0(X_0(p^m), \omega^k)$ which is acted on by $U_p$ where,
\[ U_p f = \frac{1}{p} \sum_{D \neq C^1} \pi^* f (E/D, C^m/D) \]
\end{defn}

\begin{prop}
For $i = 1, \dots, m-1$ and $(E, C^m) \in U_p$ then $D \neq C^1 \implies (E/D, C^m/D) \in U_{i-1}$. 
\end{prop}

\begin{proof}
Let $H^a(E)$ be the level $a$ canonical subgroup. We only need $E$ ordinary. Consider $(E, C^m) \in U_1$ which implies $E^1 = H^1(E) \neq D$. Then have $C^{m-i} = H^{m-i}(E)$ and need $C^{m-i+1}/D = H^{m-i+1}(E/D)$ and then $C^1/D = H^1(E/D)$ implies that $H^{m-i+1}(E/D)$ is the preimage of $H^{m-i}((E/D)/(C^1/D))$ under $E/D \to (E/D)/(C^1/D)$. We know $H^{m-i}(E) = C^{m-i}$ and,
\[ H^{m-i}((E/D)/(C^1/D)) \xrightarrow{\times p} C^{m-i} \]
\end{proof}

\begin{cor}
If $U_p f = \lambda f$ for $\lambda \neq 0$ and $f \in H^0(X_0(p)(\le r), \omega^k)$ for $r < \frac{1}{p^{m-2} (p+1)}$ then $\pi^*f$ extends to $U_{m-1}$.
\end{cor}

\subsection{Buzzard Classicality}

Use $X_1(p^m) = \{ (E, \psi_N, P) \}$ for $P$ a point of order $p^m$ (not just the data of the subgroup it generates). Then we get a diagram,
\begin{center}
\begin{tikzcd}
X_1(p^m) \arrow[r] \arrow[d] & X_0(p^m) \arrow[d]
\\
X_1(p) & X_0(p)
\end{tikzcd}
\end{center}

\begin{defn}
For $f \in h^0(X_1(p^m)(\le r), \omega^k)$ a classical or overconvergent modular form we say $f$ \textit{nebentypus character} $\chi : (\Z / p^m \Z)^\times \to \Qbar_p^\times$ if,
\[ f(E, \psi_N, aP) = \chi(a) f(E, \psi_N, P) \]
for all $a \in (\Z / p^m \Z)^\times$. 
\end{defn}

\begin{thm}
Let $f,g$ be overconvergent cusp forms of weight $k$ and level $\Gamma_1(N p^m)$ for $m \ge 2$ and $U_p f = a_p f \neq 0$ and $U_p g = b_p g \neq 0$. Suppose $f, g$ have nebentupus $\chi$ and $\chi^{-1}$ respectively for $\chi$ cond $p^m$. Let $\zeta$ be a primitive $p^m$-th root. Then the $q$-expansions of $f$ and $g$ at $(\Tate(q), \zeta)$ are,
\[ f(q) = \sum_n a_n q^n \quad g(q) = \sum_n b_n q^n \]
such that $a_1 = b_1 = 1$ and $a_n = \chi(n) b_n$ for all $p \ndivides n$. Then $f$ and $g$ are classical. 
\end{thm}

\begin{proof}
By previous analytic continuation, $f,g$ extend to $U_{m-1} \subset X_1(p^m)$. Look at cusp $c = (\Tate(q), q^{\frac{1}{p}} \zeta)$ in $X_1(p^m)^{\ord}$ because $\left< q^{\frac{1}{p}} \zeta \right> \neq H^m(\Tate(q)) = \left< \zeta \right>$ but,
\[ \left< (q^{\frac{1}{p}} \zeta)^p \right> = \left< \zeta^p \right> = H^{m-1}(\Tate(q)) \]
Therefore $f$ and $(w^m)^* f$ are well-defined at $c$. Then,
\[ U_{m-1} \cup w^m(U_{m-1}) \]
is an admissible cover of $X_1(p^m)$ so it suffices to show that $f(c) = r (w^m)^* g(c)$ for some constant $r$ then they agree on some neighbrohood of $c$ and their domains of definition cover the entire modular curve so they glue to a rigid analytic section of $\omega^k$ on $X_1(p^m)$ which hence is classical by rigid GAGA. 
\end{proof}

\begin{rmk}
Next time we will compute $f(c)$ and $g(c)$ to complete the proof. 
\end{rmk}

\section{April 20}

\subsection{Buzzard Classicality}

For $f \in H^0(X_1(p^m)^{m,\ord}, \omega^k)$ overconvergent with nebentypus $\chi : (\Z / p^m \Z)^\times \to \Qbar_p^\times$ cand $p^m$. For $U_p f = a_p f \neq 0$ 




Let $\mu = \zeta^{p^m - 1}$ be a primitive $p$-th root of $1$ let $f(c) = \frac{1}{a_p} U_p f(c)$ with,
\[ U_p f(c) = \frac{1}{p} \sum_{i = 0}^{p-1} \pi^* f(\Tate(\mu^i q^{\frac{1}{p}}), q^{\frac{1}{p}} \zeta (\text{mod} \, \mu q^{\frac{1}{p}})) \]
This gives,
\begin{align*}
\sum_{i = 0}^{p-1} & \chi(1 - i p^{m-1}) \pi^* f(\Tate(\mu^i q^{\frac{1}{p}}, \zeta \mu^{-i}) = \sum_{i = 0}^{p-1} \chi(1 - i p^{m-1}) \pi^* f(\Tate(\mu^i q^{\frac{1}{p}}), \zeta)
\\
& = \sum_{i = 0}^{p-1} \chi(1 - i p^{m-1}) \sum_n a_n(\mu^i q^{1/p})^n \omega_\can^k
\\
& = \sum_n a_n q^{n/p} \sum_{i = 0}^{p-1} \chi(1 - i p^{m-1}) \mu^m \omega_\can^k 
\end{align*}

\section{Totally Real Fields}

\begin{defn}
A \textit{totally real field} is an extension $L / \Q$ of degree $g$ such that every $\sigma : L \embed \CC$ factors through $\RR \embed \CC$. We call the embeddings $\sigma_1, \dots, \sigma_g : L \embed \RR$.
\end{defn}

\begin{example}
Some totally real fields,
\begin{enumerate}
\item $\Q$
\item $\Q(\sqrt{D})$ for $D > 0$
\item $\Q(\zeta_m + \bar{\zeta}_m) \subset \Q(\zeta_m)$ the subfield fixed by complex conjugation.
\end{enumerate}
\end{example}

\begin{defn}
We say that $s \in L$ \textit{totally positive} if $\sigma_i(s) > 0$ for all $i$. We write $s \gg 0$. For $S \subset L$ we write $S^+$ for the subset of totally positive elements.
\end{defn}

\begin{defn}
Let $\cO_L \sub L$ be the ring of integers and $\D_L$ the differenti ideal,
\[ \D_L = \{ \ell \in L \mid \tr_{L/\Q} (\ell r) \in \cO_L \forall r \in \cO_L \}^{-1} \]
and $d_L = N_{L/\Q}(\D_L)$ is the discriminant. Then let,
\[ \Cl{L}^+ = \{ \text{fractional ideals} \} / (\text{priciple ideals generated by tot. positive elements} \} \]
\[ \Cl{L} = \{ \text{factional ideals} \} / \{ \text{principle ideals} \} \]
where the first is the \textit{narrow class group}. 
\end{defn}

There is,
\begin{center}
\begin{tikzcd}
1 \arrow[r] & L^\times / (\cO_L^\times \cdot L^\times)^{+} \arrow[r] & \Cl{L}^+ \arrow[r] & \Cl{L} \arrow[r] & 1
\end{tikzcd}
\end{center}

\begin{rmk}
$\Cl{L}^+ = \mathrm{Pic}^+(\cO_L)$ is isomorphic classes of porjective $\cO_L$-modules $M$ of rank-$1$ along with a choice of positivity meaning for each $i$ choose an orientation on $M \ot_{\sigma} \RR$ and isomorphisms are required to preserve this orientation. 
\bigskip\\
Indeed, if $\a \subset L$ is a fractional ideal then $\sigma_i |_{\a} : \a \to \RR$ gives an orientation on $\a \ot_{\sigma_i} \RR$ but we only declare equivalence when an isomorphism preserves orientation i.e. is given by a totally real element. 
\end{rmk}

\subsection{Complex Abelian Varieties with Real Multiplication}

\begin{defn}
A \textit{complex abelian variety with real multiplication by} $\cO_L$ is a $g$-dimensional abelian variety $A / \CC$ with a fixed embedding of rings $\iota : \cO_L \embed \End{A}$.
\end{defn}

\begin{example}
\begin{enumerate}
\item elliptic curves always have real multiplication by $\Z$
\item for $E$ an elliptic curve and $L = \Q(\sqrt{D})$ for $D > 1$ square-free then construct $\cO_L \acts E \times E$ by, if $D \equiv 2,3 \mod 4$, declaring $\cO_L \acts E \times E$ via sending
\[ \sqrt{D} \mapsto M \text{ with characteristic polynomial }\lambda^2 - D \]
and if $D \equiv 1 \mod 4$ sending,
\[ \frac{1 + \sqrt{D}}{2} \mapsto M \text{ with characteristic polynomial } \lambda^2 - \lambda + \frac{1-D}{4} \]
\item If $X_0(p)$ has no tame level structure then its Jacobian is isogenous to $\Pi$ simple abelian variety with real multiplication.
\end{enumerate}
\end{example}


Some facts:
\begin{enumerate}
\item Suppose $A$ has $\iota : \cO_L \to \End{A}$ then $A^\vee$ has $\iota^\vee : \cO_L \to \End{A^\vee}$ via $\ell \mapsto \iota(\ell)^\vee$ (which is still a ring map because $\cO_L$ is commutative). 

\item $- \ot \Q$ gives $\iota : L \embed \End{A} \ot_{\Z} {\Q}$

\item tangent space of $A$ at $0$ is a free $\cO_L \ot_{\Z} \C$-module of rank $1$

\item Poincare: $A$ is siogenous to $A_1^{n_1} \times \cdots \times A^{n_k}_k$ for $A_i$ simple then $A$ has RM implies $A$ is isogenous to $B^r$ for $B$ simple.
\end{enumerate}

\begin{proof}
Let $B^r$ be a maximal factor of $A$ and $\cO_L \acts B^r$ nontriviall so get $L \to \End{B^r} \ot \Q$ but the tangent space to $B^r$ is a free $\cO_L \ot_{\Z} \CC$ so $B^r$ has dimension $g$ so it is the only factor. 
\end{proof}

\subsection{Constructing Abelian Varieties with Real Multiplication}

\renewcommand{\b}{\mathfrak{b}}
\newcommand{\h}{\mathfrak{h}}

Choose fractional ideals $\a, \b \sub L$ and $L \acts \CC^g$ via for $\ell \in L$ and $t = (t_1, \dots, t_g)$ set,
\[ \ell \cdot t = (\sigma_1(\ell) t_1, \dots, \sigma_g(\ell) t_g) \]
For the upper half plane,
\[ \h = \{ z \in \CC \mid \imag{z} > 0 \} \]
choose a tuple $z \in \h^g$ Then construct the lattice,
\[ \Lambda_z = \a \cdot z + \b \cdot 1 = \{ (\sigma_1
(a) z_1 + \sigma_1(b), \dots, \sigma_g(a) z_g + \sigma_g(b)) \mid a \in \a, b \in \b \} \]
Then let,
\[ A_z = \CC^g / \Lambda_z \]
with the action $L \acts \CC^g$ gives an action $\cO_L \acts A_z$ which is an embedding,
\[ \cO_L \embed \End{A_z} \]
The set of $A_z$ that appear should only depend on $\a \b^{-1} \in \Cl{L}^+$. 

\subsection{Riemann Forms on $\Lambda_2$}

\begin{defn}
Let $r \in L$ consider $E_r : (\a \oplus \b) \times (\a \oplus \b) \to \Q$ defined by,
\[ ((x_1, y_1), (x_2, y_2)) \mapsto \tr_{L/\Q} r(x_1 y_2 - x_2 y_1) \]
This is alternating, bilinear and $\im \subset \Z \iff r \in (\D_L \a \b)^{-1}$.
\end{defn}

\begin{rmk}
Recall $\D_L^{-1} = \{ \ell \in L \mid \forall r \in \cO_L : \tr_{L/\Q}(\ell v) \in \Z \}$.
\end{rmk}

\begin{rmk}
The induced pairing on $\Lambda_z = \a z + \b$ extends $\RR$-linearly to $\CC^g$ to $E_{r,2}$ antisymmetric $\RR$-bilinear perfect pariting if $r \neq 0$.
\end{rmk}

\begin{defn}
Let,
\[ H_{r,z}((x_1, \dots, x_g), (y_1, \dots, y_g)) = \sum_{i = 1}^g \frac{x_i \bar{y}_i \sigma_i(r)}{\im{z_i}} \]
which is a Hermitian form on $\CC^g$. Recall, $\im{H_{r,z}} = E_{r,2}$ and,
\[ \tr_{L/\Q}(\ell) = \sum_{i = 1}^g \sigma_i(\ell) \]
Then $H_{r,z}$ is positive definite iff $r \gg 0$. Therefore, $H_{r, z}, E_{r,z}$ gives a Riemann form on $A_z$ so it is an abelian variety with real multiplication. 
\end{defn}

\subsection{Polarization Classes}

\begin{defn}
Let $(A, \iota : \cO_L \embed \End{A})$ be an Abelian Variety with RM by $\cO_L$ and,
\[ M_A = \{ \lambda : A \to A^\vee \mid \lambda = \lambda^\vee \text{ where } \lambda \text{ is } \cO_L\text{-linear} \} \]
and write,
\[ M^+_A = \{ \lambda \in M \mid \lambda \text{ polarization} \} \]
\end{defn}

\begin{lemma}
\begin{enumerate}
\item $M_A$ is a projective $\cO_L$-module of rank $1$
\item there exists a notion of posititivity on $M_A$ such that $M_A^+$ is the set of positive elements in each $M_A \ot_{\sigma_i} \RR$. 
\end{enumerate}
\end{lemma}

\newcommand{\Lie}{\mathrm{Lie}}

\begin{proof}
We have $\cO_L \acts M_A$ via $\ell \lambda = \lambda \circ \iota(\lambda)$. We need to show that $\ell \lambda \in M_A$ meaning it is symmetric,
\[ (\lambda \circ \iota(\ell))^\vee = \iota(\ell)^\vee \circ \lambda^\vee = \iota(\lambda)^\vee \circ \lambda = \lambda \circ \iota(\ell) \]
where the last step is $\cO_L$-linearity. Because $\cO_L$ is a Dedekind domain, to prove projectivity, it suffices to show that $M_A$ is torsion-free. If $\lambda \in M_A$ then is nonzero then,
\[ \lambda : \Lie(A) \to \Lie(A^\vee) \]
is nonzero and both are free $\cO_L \ot_{\Z} \CC$-modules of rank $1$ so this map must be an isomorphism and so multiplying by nonzero $\ell$ cannot make it zero (it is a map of $1$-dimensional free modules) so we see $M_A$ is torsion-free.
\bigskip\\
If $M^+_A \neq \empty$ assume $\lambda \in M_A^+$ then consider $C_{\End{A} \ot \Q}(L)$ the centralizer. And consider, $C_{\End{A} \ot \Q}(L)^{\text{sym}}$ which is the subalgebra of elements fixed by the Rosati involution $\nu \mapsto \lambda^{-1} \nu^\vee \lambda$. Then there is an embedding,
\[ M_A \embed C_{\End{A} \ot \Q}(L)^{\text{sym}} \]
via sending $\mu \mapsto \lambda^{-1} \mu$. Indeed, applying the Rosati involution,
\[ \lambda^{-1} (\lambda^{-1} \mu)^\vee \lambda = \lambda^{-1} \mu^\vee (\lambda^{-1})^\vee \lambda = \lambda^{-1} \mu \]
\end{proof}


\begin{rmk}
Any division algebra over $\Q$ with Rosati involution is one,
\begin{enumerate}
\item a central simple algebra over $L$
\item a central simple algebra over a totally imaginary quadratic extension of $L$.
\end{enumerate}
There exists $(c, c^+) \in \Cl{L}^+$ such that,
\[ (M_A, M_A^+) \iso (c, c^+) \]
and thus abelian varities with RM by $\cO_L$ are partitioned into components corresponding to $\Cl{L}^+$. 
\end{rmk}

\begin{rmk}
Also $A = \CC^g / \Lambda$ is an ABRM implies that $\Lambda$ is a projective $\cO_L$-module of rank $2$. Then $\Lambda \cong \Lambda'$ if and only if they have the same Steinitz class $s$,
\[ \bigwedge^2_{\cO_L} \Lambda \]
Then if $\Lambda \cong \a \oplus \b$ the Steinitz class is,
\[ \bigwedge_{\cO_L}^2 \Lambda \cong \a \b \]
\end{rmk}

\begin{rmk}
We can identify,
\[ (M_{A_z}, M_{A_z}^+) \iso ((\D_L \a \b)^{-1}, ((\D_L \a \b)^{-1})^{-1}) \]
where the map sends,
\[ H_{r,z} \mapsto r \]
\end{rmk}

\subsection{Families of AB with RM}

\begin{defn}
Consider 
\[ \GL(\a \oplus \b) = \left\{
\begin{pmatrix}
a & b 
\\
c & d
\end{pmatrix} :
b \in \a^{-1} \b, c \in \a \b^{-1}, ad - bc \in (\cO_L^\times)^+ \right\} \]
which are the matrices with totally positive determinant and that preserve $\a \oplus \b$ (by multiplication on the right for some reason).  
\end{defn}

\begin{defn}
There is an action on the upper half plane,
\[ \GL(\a \oplus \b) \acts \h^g \]
via,
\[ \begin{pmatrix}
a & b
\\
c & d
\end{pmatrix}
\cdot (z_1, \dots, z-g) = \left( \frac{\sigma_i(a) z_i + \sigma_i(b)}{\sigma_i(c) z_i + \sigma(d)} \right)_{i = 1, \dots, g} \]
\end{defn}

\begin{thm}
\begin{enumerate}
\item Isomorphism classes of $(A, \iota) / \CC$ such that there exists,
\[ (M_A, M_A^+) \iso (c, c^+) \quad \text{ where } \quad c = (\D_L \a \b)^{-1} \]
are parametrized by 
\[ \GL(\a \oplus \b)^+ \backslash \h^g \]
\item the isomorphism classes of $(A, \iota) / \CC$ with fixed isomorphism,
\[ m : (M_A, M_A^+) \iso (c, c^+) \]
are parametrized by
\[ \SL(\a \oplus \b) \backslash \h^g \]
\end{enumerate}
\end{thm}

\section{April 25}

\begin{thm}
\begin{enumerate}
\item Isomorphism classes of $(A, \iota) / \CC$ such that there is an isomorphism
\[ (M_A, M_A^+) \iso (c, c^+) \]
where,
\[ cc = (\D \a \b)^{-1} \]
Are parametrized by $\GL(\a \oplus \b)^+ \backslash \h^g$
\item Isom classes of $(A, \iota) / \CC$ with fixed isomorphism,
\[ m : (M_A, M_A^+) \iso (c, c^+) \]
are parametrized by,
\[ \SL(\a \oplus \b)^+ \backslash \h^g \]
\end{enumerate}
\end{thm}

\begin{proof}
Suppose that,
\[ \mu = \begin{pmatrix}
a & b 
\\
c & d
\end{pmatrix} \in \GL(\a \oplus \b)^+ \quad \text{ and } \quad z = (z_1, \dots, z_g) \in \h^g \]
Check $f : \CC^g \to \CC^g$ sending,
\[  x \mapsto x \, \mathrm{diag}((\sigma_i(c) z_i + \sigma_i(d))_i) \]
gives a map,
\[ f : \Lambda_{\mu z} \iso \Lambda_z \]
where,
\[ (\mu z)_i = \left( \frac{\sigma_i(a) z_i + \sigma_i(b)}{\sigma_i(z) z_i + \sigma_i(d)} \right) \]
Then an element of $\Lambda_{\mu z}$ is $(\alpha, \beta) \in \a \oplus \b$ such that,
\[ \left( \sigma_i(\alpha) \left( \frac{\sigma_i(a) z_i + \sigma_i(b)}{\sigma_i(c) z_i + \sigma_i(d)} + \sigma_i(\beta) \right) (\sigma_i(c) + \sigma_i(d) \right) \]



Need to show that if $A_z \cong A_{z'}$ are AVRM with $z, z' \in \h^g$ then there exists $\mu \in \GL(\a \oplus \b)^+$ such that $\mu z = z'$ and also if $(A, \iota)$ has $((M_A, M_A^+) \cong (c, c^+)$ then $(A, \iota) \cong (A_z, \iota)$ for some $z \in \h^g$. 
\bigskip\\
To show this, consider $f : \CC^g / \Lambda_{z'} \iso \CC^g / \Lambda_z$ via some matrix $M : \CC^g \to \CC^G$ but it commutes with,
\[ \ell = \mathrm{diag}(\sigma_1(\ell), \dots, \sigma_n(\ell) \]
for all $\ell \in L$ which generate all diagonal matrices by linear independence of characters and therefore $M$ commutes with all diagonal matrices so $M$ is diagonal. Write,
\[ M = \mathrm{diag}(m_1, \dots, m_g) \]
We can write $m_i = \sigma_i(c) z_i + \sigma_i(d)$ for $c, d \in L$ then $M$ takes $\b \cdot 1 \subset \Lambda_z'$ into $\Lambda_z$
\end{proof}

\section{April 27}

\subsection{Compactification}

Add to $\GL(\a \oplus \b)^+ \backslash \h^g$ a finite set of points $\GL(\a \oplus \b)^+ \backslash \P^1(L)$ ``cusps''.

\begin{prop}
$\GL(\a \oplus \b)^+ \backslash \P^1(L) \to \Cl{L}$ taking $[\alpha, \beta] \mapsto \alpha \a + \beta \b$ so there are $h = \# \Cl{L}$ cusps.
\end{prop}

Topology of the cusps,
\[ \coprod \GL(\a \oplus \b)^+ \backslash \h^g \]
is at $\infty \in \P^1(L)$ fundamental system of neighborhoods,
\[ U_r = \{ z \in \h^g \mid \im{z_i} > r \} \]
for all $r \in \RR$. Then $\mu u_r$ is a fundamental system of neighborhoods of $\mu \infty \in \P^1(L)$ for $\mu \in \GL_2(L)$. The compact complex variety is called the Satake/minimal compactification.
\bigskip\\
Notice that in every dimension, we are only adding a finite number of points meaning that the codimension of the boundary strata increases with dimension. Thus, in higher dimensions, modular forms automatically extend to the cusp. 
\bigskip\\
The cusps (ignoring level structure) represent semi-abelian varities with real multiplication by $\struct{L}$.

\begin{defn}
$A/S$ is a semi-abelian variety if $\pi : A \to S$ is smooth of relative dimension $g$ whose gemertric fibers are extensicons of abelian varities by a torus. Furthermore, it has real multiplication if there is $\iota : \struct{L} \to \End[S]{A}$ such that $\Lie(A/S)$ is a locally free sheaf of $\struct{L} \ot_{\Z} \struct{S}$-modules of rank $1$.
\end{defn}

\begin{rmk}
A semi-abelian variety is the equivalent of an elliptic curve with multiplicative (or good) reduction rather than additive reduction. 
\end{rmk}

\begin{lemma}
Semi-abelian varities with RM has geometric fibers that are either tori or abelian varieties.
\end{lemma}

\begin{proof}
Let $A_s$ be a fiber that is not an abelian variety. Then choose a maximal torus $T \subset A_s$ (which is unique because $A_s$ is an abelian group) then $\struct{L} \acts T$ nontrivially (because $\Lie(A_s)$ is rank $1$ over $\struct{L}$) thus $L \acts X^*(T) \ot \Q$ nontrivially so this is an $L$-vectorspace so $\rank{X^*(T)} \ge [L : \Q] = g$ and hence $\dim{T} \ge g$ so $T = A_s$. 
\end{proof}

\begin{rmk}
Furthermore, if $T$ is split then $X^*(T)$ is a projective $\struct{L}$-module of rank $g$ and thus rank $1$ over $\struct{L}$ and thus isomorphism classes of split tori with RM by $\struct{L}$ correspond to element of $\Cl{L}$ correspond to cusps in $\GL(\a \oplus \b)^+ \backslash \P^1(L)$. 
\end{rmk}

\subsection{Hilbert Modular Forms}

\begin{defn}
Consider,
\[ \mu = \begin{pmatrix}
a & b 
\\
c & d
\end{pmatrix} \in \GL_2(\RR)^+ \]
then we define the factor of automorphy,
\[ j(\mu, z) = (c z + d) (\det{\mu]})^{-\frac{1}{2}} \]
Then for multiple coordinates $z \in \h^g$ and $\mu \in \GL_2(L)^+$ we define,
\[ j_{\underline{k}}(\mu, z) = \prod_{i = 1}^g j(\sigma_i(\mu), z_i)^{k_i} \]
for $\underline{k} \in \Z^g$. Then let $f : \h^g \to \CC$ and define,
\[ (f|_{\underline{k}} \mu)(z) = j_{\underline{k}}(\mu, z)^{-1} f(\mu z) \]
For a finite index subgroup $\Gamma \subset \GL(\struct{L} \oplus \a)^+$ a \textit{Hilbert modular form} of weight $\underline{k}$ and level $\Gamma$ is a holomorphic function $f : \h^g \to \CC$ such that,
\[ \forall \mu \in \Gamma : f|_{\underline{k}} \mu = f \]
\end{defn}

\begin{rmk}
Explicitly, this becomes,
\[ f \left( \left( \frac{\sigma_i(a) z_i + \sigma_i(b)}{\sigma_i(c) z_i + \sigma_i(d)} \right)_i \right) = \left[ \prod_{i = 1}^g \left( \sigma_i(c) z_i + \sigma_i(d) \right)^{k_i} \det{(\sigma_i(\mu))}^{-\frac{k_i}{2}} \right] f((z_i)_i) \]
\end{rmk}

\begin{rmk}
Equivalently, $f$ is a section of some line bundle on $\Gamma \backslash \h^g$ given by $\Gamma \backslash \h^g \times \CC$ through the action for $\mu \in \Gamma, z \in \h^g, s \in \CC$,
\[ \mu \cdot (z, s) = (\mu z, j_{\underline{k}}(\mu, z) s) \] 
So why are these line bundles interesting? This line bundle is generated by $(\d{z_1})^{k_1/2} \cdots (\d{z_g})^{k_g/2}$. Indeed,
\[ \d{(\mu z_i)} = \frac{\sigma_i(ad - bc)}{(\sigma_i(c) z_i + \sigma_i(d))^2} \d{z_i} = j(\sigma_i(\mu, z_i)^{-2} \d{z_i} \]
Furthermore, if $A$ is an AVRM then $\omega_A$ is free $\struct{L} \ot_{\Z} \CC$-module of rank $1$ and we write,
\[ \struct{L} \ot_{\Z} \CC \cong \prod_{i = 1}^g \CC \]
with $\struct{L} \acts \CC$ in the $i$-th component by $\sigma_i$. Then we write,
\[ \omega_A = \prod_{i = 1}^g \omega_{A, i} \]
where $\struct{L} \acts \omega_{A, i}$ by $\sigma_i$ so each $\omega_{A,i}$ is a line bundle. Then if $A$ is the universal abelian variety on $\Gamma \backslash \h^g$ we get,
\[ \bigotimes_{i = 1}^g \omega_{A,i}^{k_i} \]
which is the correct line bundle (by the Kodaira-Spencer isomorphism). 
\end{rmk}

\subsection{$q$-Expansions}

Consider,
\[ M = \left\{ b : \begin{pmatrix}
1 & b
\\
0 & 1
\end{pmatrix} \in \Gamma \right\} \]
where $\Gamma \subset \GL(\struct{L} \oplus \a)^+$ is finite index (or maybe congruence?). Suppose,
\[ M \supset n \a \]
with $n$ sufficiently large so that $M$ is projective $\struct{L}$-module of rank $1$ then consider,
\[ \left( f\bigg|_{\underline{k}} \begin{pmatrix}
1 & b 
\\
0 & 1
\end{pmatrix} \right) (z_1, \dots, z_g) = f(z_1 + \sigma_1(b), \dots, z_g + \sigma_g(b)) = f(z_1, \dots, z_g) \]
and thus $f$ is $(\sigma_1(b), \dots, \sigma_g(b))$-periodic for all $b \in M$. Therefore, there is a Fourier expansion,
\[ f(z) = \sum_{v \in M^\vee} a_v e^{2 \pi i \tr{(v \cdot z)}} \]
where,
\[ M^\vee = \{ \ell \in L \mid \tr_{L / \Q}(\ell m) \in \Z \text{ for all } m \in M \} \]
and likewise,
\[ \tr{(v \cdot z)} = \sum_{j} \sigma_j(v) z_j \]
This is periodic because for any $b \in M$ we have,
\[ \tr{(v \cdot (z + b))} = \sum_j \sigma_j(v) (z_j + \sigma_j(b)) = \tr{(v \cdot z)} + \sum_j \sigma_j(vb) \]
and $v b \in \Z$ so $\sigma_j(vb) \in \Z$. 

\section{May 6}

\subsection{Tate Periods}

\newcommand{\TT}{\mathbb{T}}

For $\a,\b \in \Cl{L}^+$. 
Define,
\[ \TT_{\a,\b} := \Gm \ot_{\Z} \D_L^{-1} \a^{-1} / q(\b) \]
Over $S = \Z((\a \b, \Delta))$ whre,
\[ q : \b \to \Gm \ot_{\Z} \D_L^{-1} \a^{-1} \]
and
\[ q : \a \b \to \Hom{\Z}{\struct{L}}{\a \b} \]
via,
\[ v \mapsto (\ell \mapsto v \ell) \]
Then,
\[ \Hom{\Z}{\struct{L}}{\a \b} = \struct{L}^\vee \ot_{\Z} \a \b = \D_L^{-1} \ot_{\Z} \a \b \]
generated by,
\[ \ell \mapsto \tr_{L/\Q} (\delta \ell) v \]
some $\delta \ot v \in \D_L^{-1} \ot_{\Z} \a \b$. So,
\[ q : \a \b \to \D_L^{-1} \ot_{\Z} \a \b \]
is 
\[ v \mapsto \sum_j \delta_j \ot v_j \]
such that,
\[ \forall \ell \in \struct{L} : \sum_j \tr_{L/\Q}(\delta_j \ell) v_j = v \ell \]
Then we get,
\[ q : \a \b \to \D_L^{-1} \ot_{\ZZ} \Gm(S) \]
defined via,
\[ v \mapsto \sum_j \delta_j \ot q^{v_j} \]
Therefore,
\[ (\tr_{L/\Q} \ot 1)(q(v)) = (\tr_{L/\Q} \ot 1) \left( \sum_j \delta_j \ot q^{v_j} \right) = \sum_j \tr_{L/\Q}(\delta_j) \ot q^{v_j} = \sum_j 1 \ot q^{\tr_{L/\Q}(\delta_j) v_j} = 1 \ot q^{\sum_j \tr_{L / \Q}(\delta_j) v_j} \] 
which therefore lies in $\Z \ot_{\Z} \Gm(S)$.

\subsection{$q$-Expansions}

Let M be a HMV over $\Z_p$ with $\mu_N$-level structure with universal object $\cA$.

\begin{defn}
A \textit{Hilbert modular form} $f$ of weight $k$ and level $\mu_N$, pol $c \in \Cl{L}^+$, coeffs in $B$ is a function on tuples $(A/R, \iota, \lambda, \beta_N, \omega)$ with $(A/R, \iota, \lambda, \beta_N) \in M(R)$ where,
\begin{enumerate}
\item $A / R$ is a semi-abelian scheme
\item $\iota : \struct{E} \to \End{A}$
\item $\lambda$ is a polarization
\item $\beta_N$ is a $\mu_N$-level structure
\item $\omega$ is a nonvanihsing section of $\bigwedge^g \omega_{\cA}$
\end{enumerate}
such that,
\[ f(A/R, \iota, \lambda, \beta_N, \mu \omega) = \sigma_1(\mu)^{-k_1} \cdots \sigma_g(\mu)^{-k_g} f(A/R, \iota, \lambda, \beta_N, \omega) \]
for all $\mu \in (\struct{L} \ot_{\ZZ} R)^\times$.
\end{defn}

\begin{defn}
The $q$-expansion at $\TT_c(q)$ is,
\[ f_c(q) = f(\TT_c(q) \ot_S S_B, \iota_{\can}, \lambda_{\can}, \beta_{N, \can}, \omega_{\can}) \in S_B = \ZZ((c^{-1}, \Delta)) \ot_{\Z} B \]
\end{defn}

\begin{rmk}
Here $B$ is a $\Z_p$-algebra and we define our embeddings $\sigma_i : L \embed \overline{\Q_p}$ so that $\struct{L}$ lands in $\overline{\Z_p}$. 
\end{rmk}

\begin{theorem}
\begin{enumerate}
\item $f_c(q)$ is independent of $\Delta$ in $\Z[[(c^{-1})^+]]^{U_N^2}$ where 
\[ U_N = \{ u \in \struct{L}^\times \mid u \equiv 1 \mod N \} \]
and $\Z[[(c^{-1})^+]]^{U_N^2}$ is the completion of the local ring of the cusp at $(\struct{L}, c^{-1})$ on $M_B$. 
\item $M(B, k, c, \mu_N) \embed \Z[[ \dots ]] \ot_{\Z} \RR$

\item if $f$ is a $\CC$-HMF then $f_C(q)$ is the analytic $q$-expansion. 
\end{enumerate}
\end{theorem}

\subsection{Constructing Toroidal Compactifications}

Universal semi-abelian varieties, with boundary given by a normal crossing divisor (codim 1 unlike the minimal compactification) and are equipped with maps to the minimal compactification (which are a series of blowups). Consider $\TT_c(q) / \Z((c^{_1}, \Delta))$ any rational subcone $\Delta$ of $\ell((c^{-1})^+) \ge 0$ in $L^+$. Conver $\{ \ell((c^{-1})^+) \ge 0 \}$ by $\infty$ rational subcones take Tate objects, glie, but need rational cones to be,
\begin{enumerate}
\item disjoint
\item preserved by action of $U_N^2$
\item finite after modding out by $U_N^2$.
\end{enumerate}

\end{document}
\documentclass[12pt]{article}
\usepackage{hyperref}
\hypersetup{
    colorlinks=true,
    linkcolor=blue,
    filecolor=magenta,      
    urlcolor=blue,
}

\usepackage{import}
\import{./}{AlgGeoCommands}

\begin{document}

\section{The Analytic Setting}

\renewcommand{\F}{\mathcal{F}}
\renewcommand{\G}{\mathcal{G}}

\begin{defn}
An $F$-valued \textit{local system} $\L$ on a topological space $X$ is a locally-constant sheaf of finite dimensional $F$-vector spaces.
\end{defn}

\begin{prop}
Suppose that $X$ is connected and admits a universal cover. Then the map,
\[ \{ F\text{-valued local systems on } X \} \to \{ \pi_1(X, x)\text{-representations} \} \]
Given by sending a local system to its monodromy representation,
\[ \F \mapsto \rho_\F : \pi_1(X,x) \to \mathrm{Aut}_F(\F_x) \cong \GL{n}{F} \]
is an equivalence of categories.
\end{prop}

\begin{proof}
[https://people.maths.ox.ac.uk/liu/seminars/s20-category-o/cailan-notes2-part1.pdf]
\end{proof}

\subsection{Local Monodromy}

\begin{rmk}
For the rest of the section, let $X$ be a compact Riemann surface and $S \subset X$ a finite set of points. Let $U = X \setminus S$ and $j : U \embed X$ the open immersion. For each $s \in X$ let $D(s) \subset $ be a small disk about $s$ and $D^*(s) = D(s) \cap U$. Let $I(s) = \pi_1(D^*(s))$ and choose a generator $\gamma_s$ such that $I(s) = \Z \gamma_s$.
\end{rmk}

\begin{defn}
Let $\F$ be a local system on $U$. The \textit{local monodromy representation} at $s \in S$ is,
\[ I(s) := \pi_1(D^*(s)) \to \pi_1(U) \xrightarrow{\rho_\F} \GL{n}{F} \] 
considered up to isomorphism. Explicitly, this is a conjugacy class $\gamma_s \mapsto A_s \in \GL{n}{F}$.
\end{defn}

\begin{defn}
We say that a local system $\F$ is \textit{physically rigid} if for every local system $\G$ on $U$ such that for each $s \in S$ the local monodromy data of $\F$ and $\G$ at $s$ are equal. Explicitly, for each $s \in S$ there is an isomorphism of local systems $\F |_{D^*(s)} \cong \G |_{D^*(s)}$ or equivalently an isomorphism of representations $\rho_{\F}|_{I(s)} \cong \rho_{\G}|_{I(s)}$.
\end{defn}

\begin{rmk}
For $X = \P^1$ this is extremely explicit. For $\# S = r$ the fundamental group is $\pi_1(U) \cong F_{r-1}$ generated by $C_1, \dots, C_r$ sending $C_i \mapsto \gamma_i$ with one relation $C_1 \cdots C_r = 1$. A local system is a choice of matrices $A_1, \dots, A_{r} \in \GL{n}{F}$ subject to $A_1 \dots A_r = I$ (and hence just the choice of $A_1, \dots, A_{r-1}$) up to overall conjugacy. The local monodromy is the conjugacy class $I(s_i) = [A_i]$. Given local monodromy data, $[B_i]$ we ask if there exists a local system $A_1, \dots, A_r$ such that $[A_i] = [B_i]$ and this is rigid if there is a unique such choice up to overall conjugacy.
\end{rmk}

\begin{rmk}
If $X = \P^1$ and $S = \{0, \infty\}$ then every local system $\F$ on $U$ is phyiscally rigid because $\F$ is completely determined by its monodromy data $I(0)$ since $D^*(0) \to U$ is a homotopy equivalence. Furthermore, rank $1$ local systems on $\P^1 \setminus S$ are rigid because the monodromy directly determines the representation (there is no conjugacy).
\end{rmk}

\begin{rmk}
NONRIGID EXAMPLE
\end{rmk}

\begin{prop}
If $g(X) \ge 1$ there are no phyiscally rigid local systems.
\end{prop}

\begin{proof}
Let $\F$ be a local system on $U$ and $\L$ a rank $1$ nontorsion (meaning no tensor power is trivial) local system on $X$ which exists because $\pi_1(X) \neq 0$. Then $j^* \L$ is nontorsion because $j_* : \pi_1(U, u) \to \pi_1(X, u)$ is surjective. Therefore $j^* \L$ has trivial local monodromy so $\F \ot j^* \L$ and $\F$ have the same local monodromy but are not isomorphic because $\det{\F}$ and $\det{(\F \ot j^* \L)} = \det{\F} \ot (j^* \L)^{\rank{\F}}$ are nonisomorphic. 
\end{proof}

\subsection{Cohomological Rigidity}

\begin{prop}
Let $X$ be a manifold and $\F$ a local system. Then,
\[ \chi(X, \F) = \chi(X) \cdot \rank{\F} \quad \text{and} \quad \chi_c(X, \F) = \chi_c(X) \cdot \rank{\F} \]
\end{prop}

\begin{proof}
DO MAYER VIETOREZ
\end{proof}

\begin{prop}
Now we use our previous notation with a Riemann surface $X$. Let $\F$ be a local system on $U$ then,
\[ \chi(X, j_* \F) = \chi(X) \cdot \rank{\F} + \sum_{s \in S} \dim \F_s^{I(s)} \]
\end{prop}

\begin{proof}
The Leray spectral sequence gives,
\[ \chi(U, \L) = \chi(X, j_* \L) - \chi(X, R^1 f_* \L) \]
Then $R^1 f_* \L$ is supported on $S$. For each disk $D^*(s)$ 
\end{proof}

\begin{prop}
Let $X = \P^1$ and $\F$ an irreducible local system on $U$. Then $\F$ is phyiscally rigid if and only if $H^1(X, j_* \End{\F}) = 0$.
\end{prop}

\begin{proof}
Apply the previous calculation to $\L = \End{\F}$ and $\L = \Hom{}{\F}{\G}$ which have isomorphic local monodromy. Therefore,
\[ \chi(X, j_* \Hom{}{\F}{\G}) = \chi(X, j_* \End{\F})) = 2 \]
Therefore,
\[ h^0(X, j_* \Hom{}{\F}{\G}) + h^2(X, j_* \Hom{}{\F}{\G}) \ge 2 \]
Furthermore,
\[ h^2(X, j_* \Hom{}{\F}{\G}) = h^2_c(U, \Hom{}{\F}{\G}) = h^0(U, \Hom{}{\G}{\F}) \]
Therefore one of $\Hom{}{\F}{\G}$ or $\Hom{}{\G}{\F}$ has a nonzero global section. Because $\F$ and $\G$ are irreducible this must be an isomorphism. 
\end{proof}

\begin{rmk}
This justifies thinking of $H^1(X, j_* \End{\F})$ as the deformation space of local systems with fixed monodromy on $S$ at $\F$. This is an idea we will explore further now. 
\end{rmk}

DO THE MOTIVATION (3.2.2) IN THIS SETTING.

\section{The \etale Setting}

\newcommand{\Loc}{\mathrm{Loc}}
\newcommand{\Qbar}{\overline{\Q}}

\begin{rmk}
For now, let $k$ be any field and let $U$ be a finite type scheme over $k$.
\end{rmk}

\begin{defn}
A \textit{local system} on $U_{\et}$ is a lisse $\overline{\Q}_{\ell}$-sheaf. The category $\Loc(U)$ is surprisingly difficult to define. First we define $\Loc(U, \Z/\ell^n\Z)$ as the category of locally-constant finite locally-free \etale sheaves of $\Z / \ell^n\Z$-modules. Then a lisse $\Z_\ell$-sheaf is a projective system $\{ \F_n \}$ of $\Z / \ell^n \Z$-local systems such that,
\[ \F_n \ot \Z / \ell^{n-1} \Z \to \F_{n-1} \]
is an isomorphism. Thus we write,
\[ \Loc(U, \Z_\ell) = \varprojlim \Loc(U, \Z / \ell^n \Z) \]
Now the category of lisse $\Q_\ell$-sheaves is,
\[ \Loc(U, \Q_\ell) = \Loc(U, \Z_\ell) \ot_{\Z_\ell} \Q_\ell \]
where we invert $\ell$ in the Hom. Similarly, if $L / \Q_\ell$ is a finite extensions we define $\Loc(U, \struct{L})$ and $\Loc(U, L)$ in the same way. Finally, we define,
\[ \Loc(U) := \Loc(U, \overline{\Q}_\ell) = \varinjlim \Loc(U, L) \]
\end{defn}

\begin{thm}
Let $U$ be normal and connected and $\bar{u} \in U$ a geometric point. Then there is an equivalence of categories,
\[ \Loc(U) \iso \{ \rho : \pi_1^{\et}(U, \bar{u}) \to \GL{n}{\Qbar_\ell} \text{ continuous} \} \]
defined by evaluating on the fiber over $\bar{u}$,
\[ \F \mapsto \rho_\F : \pi_1(U, \bar{u}) \to \mathrm{Aut}_{\Qbar_\ell}(\F_{\bar{u}}) \cong \GL{n}{\Qbar_\ell} \]
\end{thm}

\begin{rmk}
The ``correct`` statement is PROETALE AND FINITENESS
\end{rmk}

\newcommand{\geom}{\mathrm{geom}}

\begin{rmk}
Let $\pi_1^{\geom}(U, \bar{u}) = \pi_1(U_{\bar{k}}, \bar{u})$. Then there is a short exact sequence,
\begin{center}
\begin{tikzcd}
1 \arrow[r] & \pi_1^{\geom}(U, \bar{u}) \arrow[r] & \pi_1(U, \bar{u}) \arrow[r] & \Gal{k^\sep/k} \arrow[r] & 1
\end{tikzcd}
\end{center}
\end{rmk}

\subsection{$H$-Local Systems}

\begin{rmk}
Local systems correspond to continuous representations,
\[ \rho : \pi_1(U, \bar{u}) \to \GL{n}{\Qbar_\ell} \]
Given an affine algebraic group $H$, we want a geometric object that corresponds to a continuous homomorphism,
\[ \rho : \pi_1(U, \bar{u}) \to H(\Qbar_\ell) \]
which form a category under intertwining by elements of $H(\Qbar_\ell)$. 
\end{rmk}

\renewcommand{\Rep}{\mathrm{Rep}}

\begin{defn}
Let $\Rep(H)$ be the tensor category of algebraic repreesntations of $H$ on finite-dimensional $\Qbar_\ell$-vector spaces. An $H$-local system is a tensor-preserving functor $\F : \Rep(H) \to \Loc(U)$. Thus the category of $H$-local systems is,
\[ \Loc_H(U) = \mathrm{Fun}^{\otimes}(\Rep(H), \Loc(U)) \]
\end{defn}

\begin{thm}
Let $U$ be normal and connected. Then there is an equivalence of categories,
\[ \Loc_H(U) \iso \{ \rho : \pi_1(U, \bar{u}) \to H(\Qbar_\ell) \} \]
Defined by sending $\rho$ to the functor,
\[ \F_\rho : V \in \Rep(H) \mapsto [\rho_V : \pi_1(U, \bar{u}) \xrightarrow{\rho} H(\Qbar_\ell) \to \GL{}{V} ] \]
Conversely, $\F \in \Loc_H(U)$ can be viewed as a functor $\F : \Rep(H) \to \Rep(\pi_1(U, \bar{u}))$ and hence defines a continuous homomorphism $\rho_\F : \pi_1(U, \bar{u}) \to H(\Qbar_\ell)$ well-defined up to conjugacy.
\end{thm}

\begin{defn}
Let $\F \in \Loc_H(U)$ with corresponding $\rho_\F : \pi_1(U, \bar{u}) \to H(\Qbar_\ell)$. The \textit{global geometric monodromy group} $H^{\geom}_{\F}$ is the Zariski closure,
\[ H^{\geom}_\F = \overline{\rho(\pi_1^{\geom}(U, \bar{u}))} \subset H \]
\end{defn}

\begin{thm}
DELIGNE??
\end{thm}

\subsection{Local Monodromy}

\newcommand{\ur}{\mathrm{ur}}

\begin{rmk}
In this section, we let $X$ be a projective, smooth geomertically connected curve over a perfect field $k$ and $S \subset X(k)$ a finite set of rational points. Let $U = X \setminus S$   be the open complement and $j : U \embed X$ the open immersion. 
\end{rmk}

\begin{rmk}
We require that $k$ is perfect so that the residue fields of $X$ are also all perfect which leads to good behavior of the unramified extensions of the local fields.
\end{rmk}

\begin{defn}
Let $x \in X$ be a closed point let $\widehat{\stalk{X}{x}}$ be the completed local ring and $F_x$ its fraction field and $k_x$ its residue field. Choose a separable algebraic closure $F_x^\sep$ which defines a geometric generic point,
\begin{center}
\begin{tikzcd}
\eta_x : \Spec{F_x^\sep} \arrow[r] & \Spec{F_x}  \arrow[drr] \arrow[r] &  \Spec{\widehat{\stalk{X}{x}}} \arrow[r] & X
\\
& & & U \arrow[u, hook]
\end{tikzcd}
\end{center}
This map gives a homomorphism of fundamental groups,
\[ \Gamma_x = \Gal{F_x^\sep / F_x} \xrightarrow{\eta_x} \pi_1(U, \eta_x) \cong \pi_1(U, \bar{u}) \]
where the second isomorphism is well-defined upt to conjugacy. 
\end{defn}

\begin{prop}
If $x \in S$ then $\eta_x : \Gamma_x \to \pi_1(U, \bar{u})$ is injective.
\end{prop}

\begin{proof}
WHYYYY
\end{proof}

\begin{defn}
Consider the diagram,
\begin{center}
\begin{tikzcd}
& \Spec{k_x}  \arrow[d]
\\
\Spec{F_x} \arrow[r] & \Spec{\widehat{\stalk{X}{x}}}
\end{tikzcd}
\end{center}
which induces a diagram of fundamental groups,
\begin{center}
\begin{tikzcd}
& \Gal{\bar{k}_x/k_x} \arrow[d] \arrow[rd, "\sim"]
\\
\Gal{F_x^\sep/F_x} \arrow[r, two heads] \arrow[ru, dashed] & \pi_1(\Spec{\widehat{\stalk{X}{x}}}, \eta_x) \arrow[r, equals] & \Gal{F_x^{\ur}/F_x}
\end{tikzcd}
\end{center}
using that $k_x / k$ is finite and hence $k_x$ is perfect. Then because $F_x$ is a local field with perfect residue field $k_x$ the map $\Gal{F_x^\ur/F_x} \to \Gal{\bar{k}_x / k_x}$ is an isomorphism. We define the kernel,
\begin{center}
\begin{tikzcd}
1 \arrow[r] & I_x \arrow[r] & \Gal{F^\sep_x / F_x} \arrow[r] & \Gal{\bar{k}_x / k_x} \arrow[r] & 1
\end{tikzcd}
\end{center} 
to be the \textit{inertia group} at $x \in U$. 
\end{defn}

\begin{prop}
Under the map $\Gamma_x \to \pi_1(U, \bar{u})$ the subgroup $I_x$ lands in $\pi_1^{\geom}(U, \bar{u}) \triangleleft \pi_1(U, \bar{u})$.
\end{prop}

\begin{proof}
This is immediate from the fact that the previous diagram is in the category of $k$-schemes. Explicitly, 
\begin{center}
\begin{tikzcd}
\Spec{F_x} \arrow[d] \arrow[r] & \Spec{\widehat{\stalk{X}{x}}} \arrow[d] & \Spec{k_x} \arrow[l] \arrow[d]
\\
U \arrow[r, hook] & X \arrow[r] & \Spec{k}
\end{tikzcd}
\end{center}
commutes. Therefore, we get a diagram of exact sequences,
\begin{center}
\begin{tikzcd}
1 \arrow[r] & I_x \arrow[r] \arrow[d, dashed] & \Gal{F^\sep_x / F_x} \arrow[d] \arrow[r] & \Gal{\bar{k}_x / k_x} \arrow[d, hook] \arrow[r] & 1
\\
1 \arrow[r] & \pi_1^{\geom}(U, \bar{u}) \arrow[r] & \pi_1(U, \bar{u}) \arrow[r] & \Gal{\bar{k}/k} \arrow[r] & 1
\end{tikzcd}
\end{center} 
\end{proof}

\begin{rmk}
Furthermore, if $x \in U$ then $\eta_x : \Gamma_x \to \pi_1(U, \bar{u})$ factors through $\Spec{\widehat{\stalk{X}{x}}} \to U$ which means it factors through $\Gal{F^\ur_x/F_x}$ and hence sends the monodromy to zero. 
\end{rmk}

\begin{defn}
When $\ch{k} = p$ is positive there is a normal subgroup $I^w_x \triangleleft I_x$ called the \textit{wild interia} subgroup suhc that its quotient $I_x^t = I_x / I_x^w$ the \textit{tame inertia group} is the maximal prime-to-$p$ quotient of $I_x$. 
\end{defn}

\begin{prop}
There is a canonical isomorphism of $\Gal{\bar{k}_x / k_x}$-modules,
\[ I^t_x \iso \varprojlim_{(n,p) = 1} \mu_n(\bar{k}) = \hat{\Z}^{(p)}(1) \]
\end{prop}

\begin{defn}
Let $\rho : \pi_1(U, \bar{i}) \to H(\Qbar_\ell)$ be an $H$-local system. The \textit{local monodromy} of $\rho$ at $x \in S$ is the homomorphism $\rho_x := \rho|_{I_x}  I_x \to H(\Qbar_\ell)$. The local system $\rho$ is \textit{tame} at $x \in S$ if $\rho_x(I_x^w) = 0$ and hence if $\rho_x$ factors through the tame inertia group $I^t_x$. 
\end{defn}

\begin{rmk}
In the case $H = \mathrm{GL}_n$ the map $\rho_x$ is just the representation of $\pi_1(U, \bar{u})$ restricted to the subgroup $\eta_x(I_x) \subset \pi_1(U, \bar{u})$. For some reason, Zhiwei intermittently calls this the ``local geometric monodromy''. 
\end{rmk}

\subsection{Ramification Conductors}

\newcommand{\Sw}{\mathrm{Sw}}

\begin{defn}
Let $\sigma : I_x \to \mathrm{GL}(V)$ be a continuous representation of inertia on a $\overline{\Q}_\ell$-vector space $V$ such that $D = \sigma(I_x)$ is finite\footnote{This will be the case for those arising from Galois representations (WHY!??)}. There is some finite Galois extension $L / F^\ur_x$ such that $D = \Gal{L/F^\ur_x}$ and then we define a filtration,
\[ D = D_0 \triangleright D_1 \triangleright D_2 \triangleright \cdots \]
where,
\[ D_i = \{ \sigma \in D \mid \forall x \in \struct{L} : \sigma(x) \equiv x \mod \m_L^{i+1} \} \]
is the subgroup of $D$ acting trivially on $\struct{L} / \m_L^{i+1}$. Then the Swan conductor is defined as,
\[ \Sw(\sigma) = \sum_{i \ge 1} \frac{\dim{(V / V^{D_i})}}{[D : D_i]} \]
Likewise, the Artin conductor is,
\[ a(\sigma) := \sum_{i \ge 0} \frac{\dim(V/V^{D_i})}{[D : D_i]} = \dim(V/V^{I_x}) + \Sw(\sigma) \]
\end{defn}

\begin{rmk}
I think there is a typo in Zhiwei's notes here with $i$ and $i+1$. 
\end{rmk}

\begin{rmk}
Since $D_1 = \sigma(I^w_x)$ if $\sigma$ is tamely ramified then $\Sw(\sigma) = 0$ because there is no $i = 0$ term in $\Sw(\sigma)$. Indeed $\sigma$ is tamely ramified if and only if $\Sw(\sigma) = 0$. Likewise, $\sigma$ is unramified (i.e. trivial because we are only considering $\sigma = \rho|_{I_x}$) if and only if $a(\sigma) = 0$. 
\end{rmk}


\subsection{Rigidity}

\begin{rmk}
In this section, we assume that $S$ is nonempty so that $U$ is nonproper. 
\end{rmk}

\begin{defn}
An $H$-local system $\F \in \Loc_H(U)$ is \textit{physically rigid} if for any other $\F' \in \Loc_H(U)$ such that for each $x \in S$ the local 
\end{defn}

\newcommand{\Rig}{\mathrm{Rig}}
\newcommand{\Ad}{\mathrm{Ad}}
\newcommand{\der}{\mathrm{der}}
\renewcommand{\GL}{\mathrm{GL}}
\newcommand{\h}{\mathfrak{h}}

\begin{defn}
Let $\F \in \Loc_H(U)$ be an $H$-local system and $n = \dim{H}$. We define a $\GL_n$-local system (i.e. a local system in the standard sense) $\Ad(\F)$ via,
\[ \Ad(\F) = \F_{\Ad} \in \Loc(U) \]
Furthermore, $\Ad^\der(\F)$ is the $\GL_{n-1}$-local system,
\[ \Ad(\F) = \F_{\Ad^\der} \]
where $\Ad^\der$ is the representation of $H$ on $\h^\der = \ker{(\h \to \h^\ab)}$ is the Lie algebra of the derived subgroup.
\end{defn}

\begin{rmk}
Notice that if $H = \GL_n$ then $\Ad(\F) = \End{\F}$ and $\Ad^\der{\F} = \mathrm{End}^0(\F)$ the subsheaf of traceless endomorphisms.
\end{rmk}

\begin{rmk}
Following Zhiewei, we denote by $j_!$ and $j_*$ the \textit{derived} extension by zero and pushforward respectively. Furthermore we denote by $j_{!*}$ the usually pushforward operation on sheaves (what sane people would call $j_*$) because for a local system $\F$ the sheaf $j_{!*} \F$ agrees with the middle extension of the perverse sheaf $\F[1]$. 
\end{rmk}

\begin{defn}
An object $\F \in \Loc_H(U)$ is \textit{cohomolocally rigid} if,
\[ \Rig(\F) := H^1(X, j_{!*} \Ad^\der(\F)) = 0 \]
\end{defn}

\begin{rmk}
Since $\h^\der$ carries the $\Ad$-invariant symmetric bilinear Killing form then $j_{!*} \Ad^\der(\F)$ is Verdier self-dual and $\Rig(\F)$ is a symplectic space and hence has even dimension. Furthermore,
\[ \dim{H^0(X, j_{!*} \Ad^\der(\F))} = \dim{H^2(X, j_{!*} \Ad^\der(\F))} \]
which says that $\F$ is unobstructed if and only if it has no automorphisms.  
\end{rmk}

\begin{rmk}
EXPLAIN FIXING THE CHARACTER!!!
\end{rmk}

\begin{rmk}
Because $j_{!*} \Ad^{\der}(\F)$ does not change if we shrink $U$ and pull back $\F$ we see that cohomological rigidity is also insensitive to $U$ (there is of course a largest $U$ on which $\F$ is defined). 
\end{rmk}

\begin{lemma}
For any local system $\L$ on $U$ there is an exact sequence,
\begin{center}
\begin{tikzcd}[column sep = small]
0 \arrow[r] & H^0(U, \L) \arrow[r] & \bigoplus\limits_{s \in S} (\L_x)^{I_x} \arrow[r] & H^1_c(U, \L) \arrow[r] & H^1(U, \L) \arrow[r] & \bigoplus\limits_{s \in S} (\L_x)_{I_x}(-1) \arrow[r] & H^2_c(U, \L) \arrow[r] & 0
\end{tikzcd}
\end{center}
\end{lemma}

\begin{proof}
This should follow from an exact sequence of sheaves,
\begin{center}
\begin{tikzcd}
0 \arrow[r] & j_! \L \arrow[r] & j_{!*} \L \arrow[r] & \bigoplus_{x \in S} \L_x \arrow[r] & 0
\end{tikzcd} 
\end{center}
Taking the associated long exact sequence gives the desired result noting that $H^q(X, j_! \L) = H^q_c(U, \L)$ and $H^0_c(U, \L) = 0$ for $S \neq \empty$ along with the following identifications,
\begin{align*}
H^0(X, j_{!*} \L) & = H^0(U, \L) \cong (\L_{\bar{u}})^{\pi_1(U, \bar{u})}
\\
H^1(X, j_{!*} \L ) & = \im{(H^1_c(U, \L) \to H^1(U, \L))} 
\\
H^2(X, j_{!*} \L) & = H^2_c(U, \L) \cong (\L_{\bar{u}})_{\pi_1(U, \bar{u})}(-1)
\end{align*}
\end{proof}

\begin{thm}[Grothendieck-Ogg-Shafarevich]
Let $\L$ be a local system. Then,
\[ \chi_c(U, \L) = \chi_c(U) \cdot \rank{\L} - \sum_{x \in S} \Sw_x(\L) \]
\end{thm}

\begin{example}
DO THE ARTIN-SCRIER COVER!!
\end{example}

\begin{prop}
Let $\F \in \Loc_H(U)$. Then $\F$ is cohomologically rigid if and only if,
\[ \frac{1}{2} \sum_{x \in S} a_x(\Ad^{\der}(\F)) = (1 - g_X) \dim{\h^\der} - \dim H^0(U, \Ad^\der(\F)) \]
where $a_x$ is the Artin conductor at $x \in S$ and $g_X$ is the genus of $X$.
\end{prop}

\begin{proof}
We apply the Grothendieck-Ogg-Shafarevich formula,
\[ \chi_c(U, \L) = \chi_c(U) \cdot \rank{\L} - \sum_{x \in S} \Sw_x(\L) \]
And $\chi_c(U) = 2 - 2 g_X - \# S$. However, by the previous lemma,
\[ \dim{H^1_c}(X, j_{!*} \L) = \dim{H^1_c(U, \L)} - \sum_{x \in S} \dim(\L_x)^{I_x} + \dim{H^0(U, \L)} \]
Adding the RHS - LHS of the GOS formula on the RHS we get \footnote{The first term comes from \[ \# S \cdot \rank{\L} - \sum_{x \in S} \dim{\L_x^{I_x}} = \sum_{x \in S} \dim(\L_x / \L_x^{I_x}) \] and $\chi_c(X, \L) + \dim{H_c^1(U, \L)} = \dim{H^0_c(U, \L)} + \dim{H^2_c(U, \L)} = \dim{H^2_c(U, \L)}$ since $H^0_c(U, \L) = 0$.}
\[ \dim{H^1_c(X, j_{!*} \L)} = \sum_{x \in S} \left( \dim (\L_x / \L_x^{I_x}) + \Sw_x(\L) \right) + (2 g_X - 2) \cdot \rank{\L} + \dim{H^2_c(U, \L)} + \dim{H^0(U, \L)} \]
By the definition of the Artin condutor and Poincare duality if $\L$ is self-dual,
\[ \dim{H^1_c(X, j_{!*} \L)} = \sum_{x \in S} a_x(\L) + (2 g_X - 2) \cdot \rank{\L} + 2\dim{H^0(U, \L)} \]
Applying this to $\L = \Ad^\der(\F)$ we conclude that,
\[ \frac{1}{2} \Rig(\F) = \frac{1}{2} \sum_{x \in S} a_x(\Ad^{\der}(\F)) - \left[ (1 - g_X) \dim{\h^\der} - \dim H^0(U, \Ad^\der(\F)) \right] \]
proving the claim.
\end{proof}

\begin{cor}
Cohomologically rigid $H$-local systems exist only when $g_X \le 1$. When $g_X = 1$ and $\F \in \Loc_H(U)$ is cohomologically rigid then $\Ad^\der(\F)$ must be everywhere unramified and have no global sections.
\end{cor}

\begin{proof}
For $g_X > 1$ the RHS of the above is negative but the LHS is by definition non-negative giving a contradiction. For $g_X = 1$ the RHS is only non-negative if $H^0(U, \Ad^\der(\F)) = 0$ in which case both sides are zero and thus each Artin conductor $a_x(\Ad^{\der}(\F)) = 0$ meaning that $\Ad^{\der}(\F)$ is everywhere unramified. 
\end{proof}

\begin{thm}[Katz]
For $X = \P^1$ and $H = \mathrm{GL}_n$ the notions of physical rigidty and cohomological rigidity coincide. 
\end{thm}

\end{document}

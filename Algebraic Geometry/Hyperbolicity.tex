\documentclass[12pt]{article}
\usepackage{import}
\import{./}{AlgGeoCommands}


\newcommand{\cO}{\mathcal{O}}
\newcommand{\V}{\mathbb{V}}

\begin{document}

\section{The Kobayashi Pseudodistance}

\begin{defn}
A \textit{directed pair} $(X, V)$ is a pair of a complex mnifold $X$ and a holomorphic subbundle $V \subset T_X$. 
\end{defn}

Here let $\Delta$ be the unit disk in $\CC$ and $\rho$ the Poincare metric on $\Delta$.

\begin{defn}
Let $X$ be a complex manifold. The \textit{Kobayashi pseudodistance} is the pseduometric defined,
\[ d_X(p,q) = \inf_\alpha \ell(\alpha) \]
where $\alpha$ is a chain of holomorphic disk $f_i : \Delta \to X$ and points $p = p_0, p_1, \dots, p_k = q$ of $X$ and pairs $a_1, b_1, \dots, a_k, b_k \in \Delta$ such that,
\[ f_i(a_i) = p_{i-1} \quad f_i(b_i) = p_i \]
and the length $\ell(\alpha)$ of the chain is defined as,
\[ \ell(\alpha) := \rho(a_1, b_1) + \cdots + \rho(a_k, b_k) \]
where $\rho$ is the Poincare metric on $\Delta$.
\end{defn}

\begin{example}
Let $X = \CC$ then $d_X = 0$. Indeed, by choosing larger and larger discs containing $p,q$ their pullback to the unit disk is then closer and closer to the origin and hence have vanishing Poincare distance.  
\end{example}

\begin{rmk}
Recall the Schwartz-Pick lemma says that any holomorphic map $f : \Delta \to \Delta$ is a contraction for the Poincare metric. Therefore, $d_{\Delta} = \rho$.
\end{rmk}

\begin{lemma}
Let $f : X \to Y$ be holomorphic. Then $d_Y(f(x), f(y)) \le d_X(x,y)$ 
\end{lemma}

\begin{proof}
Indeed, choosing any chain of disks $g_i : \Delta \to X$ computing $d_X(x,y)$ we see that $f \circ g_i$ is a chain of disks connecting $f(x)$ and $f(y)$ of the same length. Therefore,
\[ d_Y(f(x), f(y)) = \inf_{\alpha} \ell(\alpha) \le d_X(x, y) \] 
\end{proof}

\begin{cor}
If $f : \CC \to X$ is an entire curve then for $x, y \in f(\CC)$ we have $d_X(x,y) = 0$ meaning if $f$ is nonconstant then $d_X$ is degenerate along the image of $f$. 
\end{cor}

\begin{proof}
Indeed, let $z_1, z_2 \in \CC$ map to $x,y$ respectively. Then,
\[ d_X(x,y) \le d_{\CC}(z_1, z_2) = 0 \]
\end{proof}

\newcommand{\kk}{\mathbf{k}}

Brody's theorem is a converse to this result. We start by considering an infinitesimal anlogue of the Kobayashi pseudodistance. Let $v \in T_{X, x_0}$ be a holomorphic tangent vector at $x_0 \in X$ and define,
\[ \kk_X(v) = \inf \{ \lambda > 0 \mid \exists f : \Delta \to X \text{ such that } f(0) = x_0 \text{ and } \lambda f'(0) = v \} \]
where $f : \Delta \to X$ is holomorphic. It is easy to check that holomorphic maps contract this pseduometric and for $X = \Delta$ it agrees with the Poincare metric.

\begin{theorem}
Let $X$ be a complex manifold. Then,
\[ d_X(p,q) = \inf_\gamma \int_\gamma \kk_X(\gamma'(t)) \d{t} \]
where the infimum is taken over all piecewise smooth curves joining $p$ and $q$.
\end{theorem}

\begin{defn}
A \textit{Brody curve} $f : \CC \to X$ is an entire curve which has bounded derivative (wrt to some/any hermitian metric).
\end{defn}

\begin{theorem}[Brody]
Let $X$ be a compact complex manifold. If $d_X$ is degenerate then there exists a Brody curve in $X$.
\end{theorem}

\begin{rmk}
Of course, in the case that $X$ is compact any entire curve is automatically Brody.
\end{rmk}



\section{Definitions}

\begin{defn}
We say a directed pair $(X, V)$ is,
\begin{enumerate}
\item \textit{Brody hyperbolic} if there does not exist a nonconstant entire map $f : \CC \to X$ tangent to $V$
\item \textit{Kobyashi hyperbolic} if the Kobayashi pseudodistance is nondegenerate (i.e. it is a metric).
\end{enumerate}
\end{defn}

\begin{theorem}[Brody]
Let $X$ be a compact complex manifold. Then $X$ is Kobayashi hyperbolic if and only if it is Brody hyperbolic.
\end{theorem}

\begin{rmk}
Therefore, we will call manifolds with this property just ``hyperbolic'' or ``analytically hyperbolic'' for emphasis. 
\end{rmk}

\begin{defn}
If $X$ is a complex projective algebraic variety we say $(X, V)$ is
\begin{enumerate}
\item \textit{algebraically hyperbolic} if there exists $\epsilon > 0$ such that for every complete intergral curve $C \subset X$ we have,
\[ 2 g(C) - 2 \ge \epsilon \deg_H{C} \]
where $g(C)$ is the geometric genus of $C$
\item \textit{algebraically quasi-hyperbolic} if $X$ contains finitely many genus $0$ and genus $1$ curves.
\end{enumerate}
\end{defn}

\begin{theorem}[Demailly]
Let $X$ be a smooth projective varitety. Then the following hold,
\[ X \text{ is hyperbolic} \implies X \text{ is algebraically hyperbolic} \] 
\end{theorem}

\begin{theorem}
If $X$ is algebraically hyperbolic then $X$ admits no nonconstant morphisms from an abelian variety.
\end{theorem}


Some references:
\begin{enumerate}
\item \chref{https://arxiv.org/pdf/math/0103084.pdf}{Xi Chen}
\item \chref{https://arxiv.org/pdf/1807.03665.pdf}{Javanpeykar}
\end{enumerate}

\subsection{The Green-Griffiths Locus and Jets}

\begin{theorem}[\chref{https://www-fourier.ujf-grenoble.fr/~demailly/manuscripts/hyperbolic.pdf}{Demilly's Notes} Theorem 7.9]
Let $(X, V)$ be a direct projective manifold and $A$ an ample line bundle. Then for any entire curve $f : \CC \to X$ tangent to $V$ and any $P \in H^0(X, E^{GG}_{k,m}(V^*) \ot A^{-1})$  we have $P(f', f'', \dots, f^{(k)}) = 0$ identically.
\end{theorem}

Therefore, if we fix an ample line bundle we can consider the locus cut out by all these differential equations.

\newcommand{\GG}{\mathrm{GG}}
\newcommand{\Exc}{\mathrm{Exc}}


\begin{defn}
The \textit{Green-Griffiths locus} $\GG_A(X, V)$ is the set $x \in X$ such that for all $k > 0$ there exists a $k$-jet $\varphi_k : (\CC, 0) \to (X, x)$ tangent to $V$ so that for all $m > 0$ every global jet differential $P \in H^0(X, E^{GG}_{k,m}(V^*) \ot A^{-1})$ satisfies $P(\varphi_k) = 0$.
\end{defn}

\begin{rmk}
The locus $\GG_A(X, V)$ is independent of the choice of ample line bundle (see \chref{https://arxiv.org/pdf/1302.4756.pdf}{Diverio and Rousseau} Lemma 2.2. This paper also gives many examples showing that $\Exc(X)$ can be strictly smaller than $\GG(X)$. However, it is conjectured that if $X$ is general type then $\GG(X) \subsetneq X$.
\end{rmk}


LOOK AT THE HILBERT MODULAR SURFACES FOR WHICH THE GG LOCUS IS EVERYTHING


\section{Conjectures}

\begin{conj}[Kobayashi]
For $n \ge 2$ and $D \subset \P^n$ a very general hypersurface of degree $\deg{D} \ge 2n + 1$ then,
\begin{enumerate}
\item $D$ is hyperbolic
\item $\P^n \sm D$ is hyperbolic.
\end{enumerate}
\end{conj}

\begin{conj}[Green-Griffiths-Lang]
Let $X$ be a projective variety of general type. Then there exists a proper algebraic subvariety containing all non-constant entire curves $f : \CC \to X$.
\end{conj}

\begin{conj}[Demailly]
If $X$ is algebraically hyperbolic then $X$ is hyperbolic.
\end{conj}

\begin{prop}
The Green-Griffiths-Lang conjecture implies the Demailly conjecture.
\end{prop}

WHY?
\begin{proof}
Suppose $X$ is algebraically hyperbolic. If $X$ is not of general type then $X$ has a fibration over its canonical model by varities of Kodaira dimension $0$. (I NEED THAT IF NOT GENERAL TYPE THEN NOT ALGEBRAICALLY HYPERBOLIC DOES THIS FOLLOW FROM MMP)
\end{proof}

\section{Theorems}

\begin{theorem}[Bogomolov]
Let $X$ is a smooth projective surface with $s_2(X) = c_1(X)^2 - c_2(X) > 0$ then $X$ has finitely many genus $0$ or genus $1$ curves (i.e. it is algebraically quasi-hyperbolic).
\end{theorem}

\begin{theorem}[McQuillian]
Let $X$ is a smooth projective surface with $s_2(X) = c_1(X)^2 - c_2(X) > 0$ and $X$ has \textit{no} genus $0$ or genus $1$ curves then $X$ is hyperbolic.
\end{theorem}



\section{Bogomolov's Theorem}


The notion of stability of a point on a space of linear representations of a reductive group, due to Mumford [10], leads to a notion of stabilite for fiber bundles over a curve, whose properties were studied in [13] and [19].

\begin{defn}
Over a smooth proper integral curve, a vector bundle $E$ of rank $r(E)$ and degree $d(E)$ is \textit{stable} (resp. \textit{semistable}) is for every nonzero proper subbundle $F \subsetneq E$ we have,
\[ \frac{d(F)}{r(F)} < \frac{d(E)}{r(E)} \quad \left( \text{resp.} \frac{d(F)}{r(F)} \le \frac{d(E)}{r(E)} \right) \]
A vector bundle is \textit{unstable} if it is not semistable.
\end{defn}

Now let $X$ be a smooth proper surface over a field $k$, and $E$ a vector bundle over rank $2$ over $X$. Then a linear representation $\rho : \GL_2 \to \GL(V)$ produces an associated bundle $E^{(\rho)} := E \times_{\GL_2} V$ of rank $\dim{V}$. 

\begin{defn}
We say a rank $2$ vector bundle is \textit{instable} if there exists a representation $\rho : \GL_2 \to \GL(V)$ with $\det{\rho} = 1$ such that $E^{(\rho)}$ admits a nonzero section which vanishes at some point. 
\end{defn}

If the characteristic of $k$ is zero, which we will assume for the remainder, then Bogomolov's instabilite criterion is simply expressed in terms of devissage of bundles of rank $2$ (WHAT?). It is interesting to note that we can here short-circuit the theory and prove directly using these simpler methods. 
\par
There are many applications. We quote from memory a proof, elegant and algebraic, of the vanishing theorem of Kodaira-Ramanujan. In the remaning section we prove the following:

\begin{theorem}[0.3]
Let $X$ be a proper smooth surface of general type. Then $\Omega_X$ is not unstable.
\end{theorem}

As a consequence, we obtain the inequality $c_1^2 \le 4 c_2$ where $c_1, c_2$ are the Chern classes of the sheaf $\Omega_X^1$ -- improved by Miyaoka [9] which is the best form possible $c_1^2 \le 3 c_2$ - and a geometric result that we will develop here.
\par 
The problem is the following: can we show 'bound'' the familly of curves of bounded geometric genus on a smooth proper surface $X$? We construct easily examples where the answer is negative. Bogomolov provides a partial solution in the case that $X$ is a surface of general type. We summarize briefly the method.
\par 
Let $\pi : P = \P(\Omega_X^1) \to X$ be the canonical projection from the projectiviation of the canonical bundle. We construct on $P$ a good linear system of divisors alowing it to be mapped to the projective space $\P^N$. If $C$ is a smooth proper curve and $f: C \to X$ is a nonconstant morphism there is a lift $t_f : C \to P$ via the differential defined over points $\alpha \in P$ where $f$ is unramified as $t_f(\alpha) = (f(\alpha), f(v_\alpha))$ where $v_\alpha$ is a nonzero tangent vector to $C$ at $\alpha$. We apply this to the normalizations of curves embedded in $X$ and study their images in $\P^N$.
\bigskip\\
We prove the following result:

\begin{theorem}
Let $X$ be a smooth proper surface minimal of general type.
\begin{enumerate}
\item If $c_1^2 > c_2$ then the curves of bounded geomeric genus on $X$ form a bounded family.
\item If $c_1^2 \le c_2$ and $\rank \NS{X} \ge 2$ then there exists a nonempty open cone $C \subset \NS{X}_{\RR}$ containing the cone $\{ z \mid z \in \NS{X}_{\RR} , z^2 \le 0 \}$ such that for any closed cone $C'$ contained in $C$ the family of curves of bounded geometric grnus on $X$ have image in $\NS{X}_{\RR}$ contained in $C'$ forms a bounded family. Moreover, any translate of $C$ parallel to $K_X$ has the same property. 
\end{enumerate} 
\end{theorem}

As a corollary, we obtain finiteness of curves with negative self-intersection and bounded geometric genus on surfaces of general type. In particular a solution to Mordell's problem.
\par 
Let's point out finally that Bogomolov uses a powerful result of Deidenberg on differential equations [18] but a recent paper of Jouanalou [5] alows us to avoid the use of this sledgehammer.

\subsection{Criteria for instability of vector bundles of rank $2$ on surfaces}

Considering the form of representations of $\PGL_2$ we give a definition equivalent to above.

\begin{defn}
A vector bundle $E$ of rank $2$ on a surfaces is \textit{unstable} if and only if there exists $n > 0$ such that $S^{2n} E \ot (\det{E})^{-n}$ has a nonzero section vanishing at some point of $X$.
\end{defn}

\subsubsection{Remark: devissage of vector bundles of rank $2$}

Let $E$ be a vector bundle of rank $2$ and $L$ an invertible sheaf and $s : L \to E$ a nonzero map. The bidual $M$ of $E/L$ is reflexive and hence invertible (since $X$ is a smooth surface), and the kernel $L_1$ of the homomorphism $E \to M$ is a larger invertible subsheaf of $E$ contining $L$. We say that it is a saturated line bundle of $E$. The cokernel $E/L_1$ is torsion-free in rank $1$, and hence of the form $I_Z \ot M$ for $M$ an invertible sheaf and $I_Z$ a sheaf of ideals defining a closed subscheme $Z$ of dimension $0$ outside of which $L'$ is a subbundle of $E$. We have a diagram of exact sequences,
\begin{center}
\begin{tikzcd}
0 \arrow[r] & L \arrow[d] \arrow[r] & E \arrow[r] \arrow[d, equals] & E/L \arrow[d] \arrow[r] & 0
\\
0 \arrow[r] & L_1 \arrow[r] & E \arrow[r] & I_Z \ot M \arrow[r] & 0
\end{tikzcd}
\end{center} 
We will say that the second line is a devissage of $E$. We can deduce the Chern classes of $E$,
\[ c_1(E) = c_1(L_1) + c_1(M) \quad \quad c_2(E) = c_1(L_1) \cdot c_1(M) + \deg{Z} \]

\begin{theorem}[Bogomolov-Mumford]
A vector bundle $E$ of rank $2$ over a surface $X$ is unstable if and only if there exists a devissage,
\[ 0 \to L \to E \to I_Z \ot M \to 0 \]
such that if $L' = L \ot M^{-1} = L^2 \ot (\det{E})^{-1}$ then either,
\begin{enumerate}
\item $L'$ is in the cone $C_+ \subset \NS{X}_{\Q}$ generated by positive divisors (IS THIS NEF?)
\item or $L' = \struct{X}$ and $Z$ is nonempty
\end{enumerate}
Moreoverm the devissage is unique.
\end{theorem}

We will prove this using only Mumford's theory of instability.
\par 
Let $P = \P(E)$ and $p : P \to X$ the projection and $\struct{P}(1)$ the canonical relatively ample bundle on $P$. A nonzero section $s \in H^0(X, S^{2n} E \ot (\det{E})^{-n})$ corresponds to a nonzero section $t \in H^0(P, \struct{P}(2n) \ot p^* (\det{E})^{-n}))$ . Let $\xi \in X$ be the generic point and $K = \kappa(\xi)$. If we chose a basis of $E_K$ then $s(\xi)$ corresponds to a homogeneous polynomial $F$ of degree $2n$ in two variables. Since $s$ is zero t some point of $X$, we know $s(\xi)$ is unstable for the action of $\PGL_2$ on $S^{2n} E_K \ot (\det{E_K})^{-n}$ (WHY?). We deduce from the stability criterion using 1-parameter subgroups [11] that $F$ has a root of order greater than $n$ in the algebraic closure of $K$, so also in $K$ (WHAT? WHY?), that's to say there exists an integer $r \ge 1$ and two polynomials $G, H$ homogeneous of degrees $1$ and $n-r$ respectively such that $F = G^{n+r} H$. Let $D$ be the divisor of $t$ and $\Delta$ the closure of the divisor defined over a generic point by $G$. We can write $D = (n + r) \Delta + \Delta'$ where has degree $1$ and $\Delta'$ has degree $n - r$ on $P$. Therefore, there exist invertible modules $L, L'$ on $X$ such that,
\[ \struct{P}(\Delta) = \struct{P}(1) \ot p^* L \quad \cO_P(\Delta') = \cO_P(n-r) \ot p^* L' \]
and hence,
\[ (\det{E})^{-n} = L^{n+r} \ot L' \]
The divisor $\Delta$ corresponds to a section of $E \ot L$ and thus an injection $L^{-1} \embed E$ which by construction is saturated in $E$. We verify that it provides the desired devissage. 


\subsection{Operations on unstable bundles}

Instability is preserved by passage to the dual and tensor product with a line bundle. 

\begin{enumerate}
\item Let $f : Y \to X$ be a surjective morphism of surfaces, $E$ a vector bundle of rank $2$ over $X$. Then $E$ is unstable if and only if $f^* E$ is. 

\item Let $f : Y \to X$ be a finite faithfully flat morphism of surfaces, $F$ a fiber bundle of rank $2$ on $Y$. Then if $F$ is unstable so is $f_* F$.
\end{enumerate}


\subsection{Proof of Theorem 0.3}

Suppose that $\Omega_X^1$ is unstable. Then there exists a devissage:
\[ 0 \to L \to \Omega^1_X \to I_Z \ot M \to 0 \]
and an integer $n > 0$ such that there is an injection $\cO_X \embed (L \ot M^{-1})^{\ot n}$. Note yhat $L \ot M^{-1}  = L^2 \ot (\det{\Omega^1_X})^{-1} = L^2 \ot (\Omega_X^2)^{\ot -1}$. Also, for $m \gg 0$,
\[ h^0(L^{2m}) = h^0((L \ot M^{-1})^{\ot m} \ot (\Omega^2_X)^{\ot m}) \ge h^0((\Omega_X^2)^{\ot m}) \in O(m^2) \]
Therefore, the theorem is a consequence of the following.

\begin{theorem}[Bogomolov]
Let $X$ be a smooth proper surface and $L \embed \Omega_X^1$ an invertible subsheaf. Then $h^0(L^n) \in O(n)$. 
\end{theorem}

First recall the pretty result of Castelnuovo and of Franchis which we will need for the proof.

\begin{lemma}[4, 12]
Let $\omega_1, \omega_2$ be two holomorphic $1$-forms on $X$ which are linearly independent over $k$ such that $\omega_1 \wedge \omega_2 = 0$. Then there exists a curve $C$ which is proper and smooth over $k$ of genus $g \ge 2$ and two holomorphic $1$-forms $\theta_1, \theta_2$ on $C$ and a morphism $u : X \to C$ such that $\omega_i = u^* \theta_i$ for $i = 1,2$.
\end{lemma}

There exists a meromorphic function $f : X \rat \P^1$ such that $\omega_2 = f \omega_1$. This defines a morphism $f : X' \to \P^1$ where $X'$ is a blowup of $X$. Let $u : X' \to C \to \P^1$ be the Stein factorization.  We have an exact sequence of modules of differentials,
\[ 0 \to u^* \Omega^1_C \to \Omega^1_{X'} \to \Omega^1_{X'/C} \to 0 \]
We know $\omega_2 = f \omega_1$ and $0 = \d{\omega_2} = \d{f} \wedge \omega_1$ (since $\omega_i$ are global holomorphic forms they are closed by Hodge theory). 

{\color{red} WHY DOES IT WORK ON AN OPEN}

But $\d{f}$ is pulled back from an open of $U$ so $\omega_1$ is also as it is parallel to $\d{f}$ hence also $\omega_2 = f \omega_1$. So above an open $U \subset C$ the forms $\omega_1, \omega_2$ are in the image of,
\[ H^0(u^{-1}(U), u^* \Omega_C^1) = H^0(U, \Omega^1_C) \to H^0(u^{-1}(U), \Omega^1_{X'}) \]
so we choose $\theta_1, \theta_2$ holomorphic forms on $U$ which pull back to $\omega_1, \omega_2$. However, $u_* \cO_{X'} = \cO_{C}$ so $\theta_1, \theta_2$ extend to global sections of $\Omega_C$ because $\omega_1, \omega_2$ are global sections of $\Omega_{X'}$. Indeed,
{\color{red} (WHY DOES IT EXTEND??) THIS SEEMS WRONG}


Since $\omega_1, \omega_2$ are $k$-independent so are $\theta_1, \theta_2$. Hence $g(C) \ge 2$ and therefore the map $u : X' \to C$ contracts all rational curves and hence factors through $X' \to X$ giving the requried map. 

\subsubsection{Interlude: regularizing meromorphic 1-forms via covers}

{\color{red} WHATIS THE CORRECT DEFINITION OF TAME?}

\begin{lemma}
Let $f : X \to Y$ be a morphism of locally noetherian schemes. If $Z \subset Y$ is an irreducible subset of codimension $\le r$ then either $f$ does not dominate $Z$ or there is some closed $Z' \subset X$ of codimension $\le r$. 
\end{lemma}

\begin{proof}
Using that $\codim{Z,Y} = \dim{\stalk{Y}{\xi}}$ where $\xi \in Z$ is the generic point we immediately reduce to the affine case. Either $\xi \notin f(X)$ and we are done or we can choose $f : U \to V$ a mp of affine schemes sending $\xi' \in U$ to $\xi \in V$. Let $\varphi : A \to B$ be a map of noetherian rings and $\p \subset A$ a prime of height $\le r$ in the image of $\Spec{B} \to \Spec{A}$. Passing to $A_\p \to B_\p$ we need to find a prime $\q$ of $B_\p$ of height $\le r$ with $\varphi^{-1}(\q)$ maximal. Then $\p$ is the unique minimal prime over an ideal of definition $(x_1, \dots, x_r) \subset A_\p$ generated by at most $r$ elements by \chref{https://stacks.math.columbia.edu/tag/00KQ}{Tag 00KQ}. Since $B_\p / \p B_\p$ is nonzero (the fiber is nonempty) the ideal $(x_1, \dots, x_r) B_\p$ is proper hence, by the Krull height theorem, there exists a prime $\q$ containing it of height $\le r$. Then each $x_i \in \varphi^{-1}(\q)$ so $\p \subset \varphi^{-1}(\q)$ and we conclude. 
\end{proof}

\begin{example}
Noetherianity is essential in the above. Indeed, we could take a domain $D$ with every nonzero prime of infinite height (as constructed in ``Anti-archimedean rings and power series rings'' by D.D. Anderson). Then for any nonzero nonunit $t \in D$ the map $k[t] \to D$ certainly falisfies the claim that the divisor $V(t)$ is in the image of a divisor (since there are none) although it is in the image of some prime.
\end{example}

\begin{prop}
Let $f : X \to Y$ be a proper dominant morphism of locally noetherian integral $S$-schemes that are smooth over $S$ at the generic points of all divisors. If $f$ is tame and $\omega \in (\Omega_{Y/S})_{\eta}$ is a meromorphic differential such that $f^* \omega \in H^0(X, \Omega_{X/S}^{\vee \vee})$ is a global reflexive differential then $\omega \in H^0(Y, \Omega_{Y/S}^{\vee \vee}$ is a global reflexive differential.
\end{prop}

\begin{proof}
Since $Y$ is regular in codimension $1$ it suffices to show that for each $\xi \in Y$ of height $1$ that $\omega_\xi \in (\Omega_Y)_{\xi}$. Since $f$ is proper and dominant it is surjective so we may choose $\xi' \in X$ mapping to $\xi$. The fiber over a divisor must contain a divisor of $X$ so we can choose  $\xi'$ in the smooth locus. locus hence $f^* \omega$ is a well-defined differential form over $\stalk{X}{\xi'}$.  Since $\stalk{X}{\xi'}$ is a noetherian local domain by [Hartshorne, Ex.4.11] there exists a DVR $R \subset \Frac{\stalk{X}{\xi'}}$ dominating $\stalk{X}{\xi'}$. 

{\color{red} FINISH}

\end{proof}

\begin{rmk}
For example, this holds for any tame dominant map of normal proper varities over a perfect field. 
\end{rmk}

{\color{red} COUNTEREXAMPLES}


\subsubsection{Completion of the Theorem}

Either, for all $n > 0$ we have $h^0(X, L^{\ot n}) \le 1$ or there exists $n > 0$ such that $h^0(X, L^{\ot n}) \ge 2$. In the second case, there is a standard method of extracting an $n^{\text{th}}$-root {\color{red} (WHAT THE HELL DOES THIS MEAN)} to get $h^0(X, L) \ge 2$. In this case, there are two forms $\omega_1, \omega_2 \in H^0(X, \Omega^1_X)$ such that $\omega_1 \wedge \omega_2 = 0$ since they arise from the same subsheaf of rank $1$. Therefore, by the lemma, there exists a curve $C$ and a morphism $u : X \to C$ and an invertible sheaf $L_0$ on $C$ such that,
\[ L \subset u^*(L_0) \]
{\color{red} AGAIN WHY?}
so we can conclude that,
\[ h^0(X, L^n) \le h^0(C, L_0^n) \in O(n) \]

\begin{cor}
If $c_1$ and $c_2$ are the Chern classes of $\Omega^1_X$ then $c_1^2 \le 4 c_2$. 
\end{cor}

\subsubsection{Curves of bounded genus on a minimal surface of general type}

We provide a few examples showing that $X$ being general type plays an essential role, and that in the contrary case, there can be unbounded families of curves of fixed geometric genus.

\begin{example}
Let $X = \P^2$ then $\NS{X} = \Z$. There exist in the projective plane curves of bounded geometric genus but arbitrarily large degree.
\end{example} 

\begin{example}
Let $E$ be an elliptic curve without complex multiplication and let $X = E \times E$. Then $\NS{X} = \Z f_1 \oplus \Z f_2 \oplus \Z \Delta$ where $f_i$ are the fiber classes and $\Delta$ is the diagonal. For every pair of integers $(m, n)$ the image in $X$ od the morphism $f_{m,n} : E \to X$ given by $f_{m,n}(\alpha) = (m \alpha, n \alpha)$ is a curve of class $m^2 f_1 + n^2 f_2 + (m-n)^2 \Delta$ and genus $1$.
\end{example}

\begin{example}
Let $B$ be a smooth proper curve and $\pi : X \to B$ a nonisotrivial (that is to say it does not become trivial after some finite base change $B' \to B$) minimal elliptic fibration admitting a section $\sigma : B \to X$ of infinite order. Let $\omega$ be the conormal bundle of $\sigma$. Then there exist global sections $g_2 \in H^0(X, \omega^4)$ and $g_3 \in H^0(X, \omega^6)$ such that $X$ is the minimal resolution of the surface $Y \subset \P_B(\omega^2 \oplus \omega^3 \oplus \cO_B)$ defined by the Weierstrass equation,
\[ y^2 z = x^3 - g_2 x z^2 - g_3 z^3 \]
Morover, $\omega$ is independent of the section $\sigma$ as has degree $- \sigma(B)^2$. If the degree is zero, then $g_2$ and $g_3$ are constant and the fibration $\pi$ is isotrivial. There exist infinitely many sections of negative self-intersection and the classes are algebraically distict. 
\end{example}

Notation: we write $K_X$ for a canonical divisor of $X$ and $T_X$ the tangent bundle and $\pi : \P(\Omega_X) \to X$ the canonical projection and $L = \cO_P(1)$ the relatively ample bundle for $\pi$.
\par 
Let $F$ be a invertible bundle on $X$. We note that $F$ is a divisor of some linear system (DOES HE MEAN $F$ IS THE ZERO LOCUS OF SOME SECTION). Moreover, for any rational number $\ell \in \QQ$, we allow ourselves to form the sheaf $\ell F$, extending that we consider the tensor powers $(\ell F)^{\ot m}$ for which $m$ is such that $m \ell$ is an integer.


\subsection{COnstruction of a good linear system of divisors on $P$}

\begin{prop}
Let $F$ be an invertible sheaf on $X$ and $\ell$ a rational positive number such that,
\begin{enumerate}
\item $K \cdot F \ge 0$
\item $(K + 2 \ell F)^2 > 0$
\item $c_1^2(\Omega_X \ot \ell F) - c_2(\Omega_X \ot \ell F) > 0$
\end{enumerate}
Then for $m \gg 0$ the linear system $(L \ot \pi^* (\ell F))^m$ defines a rational map $u_F : \P(\Omega_X^1) \rat \P^N$ birational onto its image.
\end{prop}

{\color{red} IT SEEMS WRONG THAT $\ell F$ IS INSIDE THE $S^m$ THIS GIVES $(\ell F)^{2m}$ NOT $(\ell F)^{m}$ AS SHOULD BE FROM PROJECTION FORMULA}

\begin{proof}
By the theorem of Iitaka [20], it suffices to show that for $m \gg 0$,
\[ h^0(P, (L \ot \pi^* (\ell F))^m) = h^0(X, S^m(\Omega_X \ot \ell F)) \ge O(m^3) \]
The Riemann-Roch formula for $E$ shows that,
\[ \chi(S^m E) = \frac{m (m+1)(m+2)}{24} (c_1^2(E) - 4 c_2(E)) + \frac{m+1}{2} \left[ \frac{m^2}{4} c_1^2(E) - \frac{m}{2} K_X \cdot c_1(E) \right]  + (m + 1) \chi(\cO_X) \]
and hence for $m \gg 0$,
\[ h^0(S^m(\Omega_X^1 \ot \ell F)) + h^2(S^m(\Omega_X^1 \ot \ell F)) \sim h^1(S^m(\Omega_X^1 \ot \ell F)) + \frac{m^3}{6} \left[ c_1^2(\Omega_X \ot \ell F) - c_2(\Omega_X^1 \ot \ell F) \right] \ge O(m^3) \]
By Serre duality, {\color{red} HOW DO I FIX THE DUAL AND $S^m$ IN POSITIVE CHAR}
\[ h^2(S^m(\Omega_X \ot \ell F)) = h^0(K \ot S^m(T_X \ot -\ell F)) \]
Chosing some divisors $D$ and $D'$ ample and smooth such that,
\[ \cO_X(-D') \subset K \subset \cO_X(D) \]
we find that,
\[ \bigg| h^0(K \ot S^m(T_X \ot -\ell F)) - h^0(S^m(T_X \ot -\ell F)) \bigg| \in O(m^2) \]
Therefore, we conclude by appealing to the following lemma.
\end{proof}

\begin{lemma}
For any $m > 0$ we have $H^0(S^m(T_X \ot - \ell F)) = 0$.
\end{lemma}

{\color{red} THE $m$ VS $2m$ DOESNT MAKE SENSE}

\begin{proof}
We showed that $T_X \ot -\ell F$ is not unstable. Hence, the only sections of $H^0(S^{2m}(T_X \ot -\ell F) \ot (\det{(T_X \ot -\ell F)})^{-m})$ are nowhere vanishing. If we show for $m \gg 0$ that $H^0(\det{(T_X \ot - \ell F})^{-m})$ has a nonzero section with a zero at some point $x \in X$ then its product with a section $H^0(S^{2m}(T_X \ot - \ell F))$ will give a contradiction.
Thus, the result will follow from the definition,
\[ \det{(T_X \ot - \ell F)^{-m}} = m (K + 2 \ell F) \]
and Riemann-Roch,
\[ \chi(m(K + 2 \ell F)) \sim \frac{m^2}{2} (K + 2 \ell F)^2 \in O(m^2) \]
and therefore,
\[ h^0(m(K + 2 \ell F)) + h^2(m(K + 2 \ell F)) \ge O(m^2) \]
by Serre duality,
\[ h^0(m(K + 2 \ell F)) = h^0(K - m(K + 2 \ell F)) \]
Since $K \cdot (K - m(K + 2 \ell F)) = K^2 - m K \cdot (K + 2 \ell F) < 0$ and $K$ is nef (we assumed that $X$ is minimal) $h^0(K - m (K + 2 \ell F)) = 0$ for $m \gg 0$ giving the result.
\end{proof}

Our any bundle $F$ verifying the conditions of the properosition, we fix, once and for all, $m$ and $\ell$ and let $Z_F$ be the closed subset of $\P(\Omega_X)$ outside of which $u_F$ is defined.

\begin{defn}
Let $C$ be a curve embedded in $C$ and $f : \wt{C} \to C$ its normalization. If $t_f(\wt{C})$ is not contained in (resp. is contained in) $Z_F$, we say that $C$ is $F$-regular (resp. $F$-irregular).
\end{defn}

\subsection{Proof of Theorem 0.4}

We suppose that $\L \embed \Omega_X^1$ is a invertible subsheaf. If $h^0(X, \L^{\ot n}) \le 1$ for all $n$ then we are done. Otherwise, there is some $n > 0$ such that $h^0(X, \L^{\ot n}) \ge 2$. In this case, by passing to a cyclic cover we may assume that $h^0(X, \L) \ge 2$. Therefore, there are two independent $1$-forms $\omega_1, \omega_2 \in H^0(X, \L) \subset H^0(X, \Omega^1_X)$ such that $\omega_1 \wedge \omega_2 = 0$ because they lie in the same $1$-dimensional subspace at the generic point $\L_{\eta} \subset \Omega_{X, \eta}^1$. Therefore, we may apply Castelnuovo's lemma to produce a morphism $f : X \to C$ to some curve of genus $g \ge 2$ with $\omega_1, \omega_2$ pulled back along $f$. By the proof of this lemma, we see that any local section of $\L$ is pulled back along $f$ hence $\L \embed f^* \Omega_C$.  

\subsubsection{Ramified Cyclic Covers}

Let $X$ be a scheme and $\L \in \Pic{X}$ a line bundle and $s \in H^0(X, \L^{\ot n})$ a nonzero section of some tensor power. Then we may form a finitely-presented sheaf of $\struct{X}$-algebras,
\[ \cA = \struct{X} \oplus t \L^{\ot -1} \oplus \cdots \oplus t^{n-1} \L^{\ot -(n-1)} \]
where multiplication is defined in the obvious manner,
\[ (t^a f_1) (t^b t_2)  = 
\begin{cases}
t^{a+b} f_1 f_2 & a + b < n
\\
t^{a + b - nk} [(s^\vee)^{\ot k} \ot \id](f_1 f_2) & nk \le a + b < (n+1)k
\end{cases}  \]
where $[(s^\vee)^{\ot k} \ot \id] : \L^{\ot -(a+b)} \to \L^{\ot -(a + b - nk)}$. Then we define $X_{\L, s} := \rSpec{X}{\cA}$. Over the locus where $s$ is nonvanishing it is clear that $X_{\L, s} \to X$ is a degree $n$ cyclic cover which is \etale for $n$ nonzero in the base scheme. 
\bigskip\\
Note that $\cA$ can also be described as follows. Consider the symmetric algebra,
\[ \Sym{\bullet}{\L^\vee} = \bigoplus_{n = 0}^\infty t^n \L^{\ot -n} \onto \cA \]
which is the quotient as a sheaf of algebras by the ideal generated by $(t^n f - s^\vee(f))$ for local sections $f$ of $\L^{\ot -n}$. Therefore, $X_{\L, s} \embed \V_X(\L)$ is a closed subscheme of the total space of the line bundle $\L$ which can be described as the locus of points $(x, v)$ such that $v^n = s(x)$.
\bigskip\\
Note that under $\pi : \V_X(\L) \to X$ we get a canonical section $t \in H^0(\V_X(\L), \pi^* \L)$ and hence for $f : X_{\L, s} \to X$ there is a canonical section $t \in H^0(X_{\L, s}, f^* \L)$ such that $t^n = f^* s$. 
\bigskip\\
Now suppose that $s_1, s_2 \in H^0(X, \L^{\ot n})$ are two independent sections. Then by passing to the iterating cyclic cover, $X' = (X_{\L, s_1})_{f^* \L, f^* s_2} \to X_{\L, s_1} \to X$ we get $\L' = f^* \L$ and two canonical sections $t_1, t_2 \in H^0(X', \L')$ such that $t_i^n = f^* s_i$ for $i = 1,2$. 
\bigskip\\
Furthermore, suppose that $n$ is invertible on the base and there is an injection $\L \embed \Omega_X^1$. Then passing to the cyclic cover (which is generically \etale) we get $f^* \L \embed f^* \Omega_X^1 \embed \Omega_{X'}^1$ which is injective because it is at the generic point. Hence we reduce to the situation that $h^0(X, \L) \ge 2$.


\section{Semple Jets}


\begin{defn}
A \textit{directed variety} $(X, \E)$ is a pair of a variety $X$ with a subbundle $\E \subset \T_X$. A morphism of directed varities $f : (X, \E) \to (Y, \E')$ is a morphism $f : X \to Y$ such that under $f_* \T_X \to \T_Y$ we have $f_* \E \to \E'$.
\end{defn}

\begin{rmk}
Demailly's philosophy is that it is usefull to study this ``relative notion'' even for the absolute case $\E = \T_X$ since it has better functoriality properties.
\end{rmk}

\begin{rmk}
Here our convention is that $\P(\E) := \rProj{X}{\Sym{}{\E^\vee}}$ so that $\cO(-1)$ is the universal subbundle. Hence $\cO(1)$ on $\P(\T_X)$ is what I usually call $\cO(1)$ on $\P(\Omega_X)$.
\end{rmk}

\begin{defn}
To a directed pair $(X, \E)$ we introduce the \textit{projectivization} to produce a new pair $\P(X, \E) := (\wt{X}, \wt{\E})$ where $\wt{X} := \P(\E)$ and $\wt{\E}$ is defined via the diagram,
\begin{center}
\begin{tikzcd}[row sep = small, column sep = large]
0 \arrow[r] & \T_{\wt{X}/X} \arrow[dd] \arrow[r] & \wt{\E} \arrow[dd] \pullback \arrow[r] & \cO(-1) \arrow[d, hook] \arrow[r] & 0
\\
& & & \pi^* \E \arrow[d, hook]
\\
0 \arrow[r] & \T_{\wt{X}/X} \arrow[r] & \T_{\wt{X}} \arrow[r] & \pi^* \T_{X} \arrow[r] & 0
\end{tikzcd}
\end{center}
Then we have,
\[ \dim{\wt{X}} = \dim{X} + \rank{\E} - 1 \quad \quad \rank{\wt{\E}} = \rank{\E} \] 
\end{defn}

\begin{rmk}
Note that the Euler exact sequence takes the form,
\begin{center}
\begin{tikzcd}
0 \arrow[r] & \cO \arrow[r] & \pi^* \E \ot \cO(1) \arrow[r] & \T_{\wt{X}/X} \arrow[r] & 0
\end{tikzcd}
\end{center}
\end{rmk}

\begin{prop}
Given a morphism of directed varities $f : (X, \E) \to (Y, \F)$ we get a rational map $\wt{f} : (\wt{X}, \wt{\E}) \rat (\wt{Y}, \wt{\F})$ such that the diagram,
\begin{center}
\begin{tikzcd}
(\wt{X}, \wt{\E}) \arrow[d,"\pi"] \arrow[r, "\wt{f}", dashed] & (\wt{Y}, \wt{\F}) \arrow[d, "\pi"]
\\
(X, \E) \arrow[r, "f"] & (Y, \F)
\end{tikzcd}
\end{center} 
commutes in the category of directed manifolds (with rational maps). Moreover, if $f$ is ``immersive along $\E$'', meaning $f_{\#} : \E \to f^* \F$ is injective, then $\wt{f}$ is a morphism.
\end{prop}

\begin{defn}
Let $(X, V)$ be a directed manifold. The \textit{projectivized Semple $k$-jet bundle} $P_k V = X_k$ is defined iteratively via,
\[ (X_0, V_0) := (X, V) \quad \quad (X_{k+1}, V_{k+1}) := (\wt{X_k}, \wt{V_k}) \]
and we have,
\[ \dim{P_k V} = \dim{X} + k (\rank{V} - 1) \quad \quad \rank{V_k} = \rank{V} \]
\end{defn}

\begin{rmk}
We can alternatively think of the Semple construction in the dual sense,
\begin{center}
\begin{tikzcd}[row sep = small, column sep = large]
0 \arrow[r] & \pi^* \Omega_X \arrow[r] \arrow[d, two heads] & \Omega_{\wt{X}} \arrow[dd] \arrow[r] & \Omega_{\wt{X}/X} \arrow[r] \arrow[dd, equals] & 0
\\
& \pi^* \E^\vee \arrow[d, two heads]
\\
0 \arrow[r] & \struct{\wt{X}}(1) \arrow[r] & \wt{\E}^\vee \arrow[r] & \Omega_{\wt{X}/X} \arrow[r] & 0
\end{tikzcd}
\end{center}
This will be our standard perspective although we retain the dual notation to remain in agreement wit hthe complex geometry literature. Now the Euler sequence
\begin{center}
\begin{tikzcd}
0 \arrow[r] & \Omega_{\wt{X}/X} \arrow[r] & \pi^* \E^\vee \ot \struct{\wt{X}}(-1) \arrow[r] & \struct{\wt{X}} \arrow[r] & 0
\end{tikzcd}
\end{center}
gives $\pi_* \nSym{d}{\Omega_{\wt{X}/X}} = 0$ and $R^1 \pi_* \Omega_{\wt{X}/X} = \struct{X}$. Furthermore, applying Sym to the botom row gives,
\begin{center}
\begin{tikzcd}
0 \arrow[r] & \nSym{d-1}{\wt{\E}^\vee} \ot \cO_{\wt{X}}(1) \arrow[r] & \nSym{d}{\wt{\E}^\vee} \arrow[r] & \nSym{d}{\Omega_{\wt{X}/X}} \arrow[r] & 0
\end{tikzcd}
\end{center}
so applying $\pi_*$ gives,
\[ \pi_* \nSym{d}{\wt{\E}^\vee} = \pi_* [ \nSym{d-1}{\wt{\E}} \ot \cO_{\wt{X}}(1)] \]
\end{rmk}

\begin{example}
For the directed manifold $(X, \T_X)$ we set $P_k = X_k$ and set $\cP^{k,d} = \pi_{k*} \cO_{P_k}(d)$. Notice that there are exact sequence,

{\color{red} DO THIS}
\end{example}

The semple tower is defined so that the following holds. Suppose that $f : C \to X$ is an immersed curve such that $\d{f} : \T_C \to f^* \T_X$ factors through $f^* \E \subset f^* \T_X$. Since $\d{f}$ is a subbundle this gives a subbundle $\T_X \embed \pi^* \E$ and hence a lift $f' : C \to \wt{X}$ such $\d{f} : \T_C \to f^* \E \to f^* \T_X$ is $f'^* [\struct{\wt{X}}(-1) \to \pi^* \E \to \pi^* \T_X]$. Therefore, consider $\d{f'} : \T_C \to f'^* \T_{\wt{X}}$. Since this map lifts $\d{f}$ we see that $\d{f'} : \T_X \to f'^* \wt{\E}$.

Hence, if we start with an immersed curve $f : C \to X$ then there are lifts $f_k : C \to P_k$ for all $k$.

\subsection{Arc Spaces and Hasse-Schmidt Derivations}

\begin{defn}
Let $X$ be an $S$-scheme. Then $\ell^{\text{th}}$-order \textit{arc} of $X$ is a $S$-morphism $\Delta^\ell_S \to X$ where 
\[ \Delta^{\ell}_S = \rSpec{S}{\struct{S}[t]/(t^{\ell+1})} = S \times_{\ZZ} \Spec{\ZZ[t]/(t^{\ell + 1})} \]
If it exists, the $\ell^{\text{th}}$-order arc space is $J_\ell(X) = \Hom{S}{\Delta^\ell_S}{X}$ which represents the functor,
\[ T \mapsto \Hom{T}{\Delta_\ell \times_k T}{X_T} \]
\end{defn}

When $X$ is a $k$-scheme we let $S = \Spec{k}$ and let $\Delta^\ell = \Spec{k[t]/(t^{\ell + 1})}$ without adornment.

\newcommand{\HSDer}[4]{\mathrm{Der}^{#2}_{#1}\left(#3, #4\right)}
\newcommand{\HS}{\mathrm{HS}}

\begin{defn}
Let $R$ be a ring and $A, B$ be $R$-algebras. Then the group of $m^{\text{th}}$-order \textit{Hasse-Schmidt derivations} $\HSDer{R}{m}{A}{B}$ is the group of sequences $(D_0, D_1, \dots, D_m)$ of $R$-linear maps $D_i : A \to B$ such that,
\[ D_k(xy) = \sum_{p + q = k} D_p(x) D_q(y) \]
for all $k \le m$ and $x,y \in A$.
\end{defn}

\begin{prop}
For any $R$-algebra $A$ we have,
\[ \Hom{R}{\Delta^m_R}{A} = \Hom{R}{A}{R[t]/(t^{m+1})} = \HSDer{R}{m}{A}{R} \]
\end{prop}

\begin{proof}
The correspondence sends $\varphi : A \to R[t]/(t^{m+1})$ writen as,
\[ \varphi(x) = \sum_{i = 0}^m \varphi_i(x) t^i \]
to the HS derivation $(\varphi_0, \varphi_1, \dots, \varphi_m)$.
\end{proof}

\begin{prop}
Let $A$ be an $R$-algebra. Then there exists an $A$-algebra $\HS_{A/R}^m$ equipped with a universal HS-derivation $D : A \to \HS_{A/R}^m$ representing $\HSDer{R}{m}{A}{-}$ menaing,
\[ \Hom{A}{\HS_{A/R}^m}{B} = \HSDer{R}{m}{A}{B} \]
functorially in $R$-algebras $B$. Furthermore, this has an explicit presentation,
\[ \HS_{A/R}^m = A [ \d_i{x} ]_{x \in A, 0 \le i \le m} / \left< \d_i(x+y) = \d_i{x} + \d_i{y} \, \d_i{r} = 0 \, \d_i(xy) = \sum_{p + q = i} \d_p(x) \d_q(y) \right>_{r \in R} \]
Clearly, $\HS_{A/R}^m$ is graded by $A$-modules $\HS_{A/R}^{m,d}$ where we put $\d_i{x}$ in degree $i$ and the degree $k$ part consists of sums of monomials of total degree $k$.
\end{prop}

\begin{rmk}
The map $D_0 : A \to \HS^m_{A/R}$ makes $\HS^m_{A/R}$ into an $A$-algebra. Furthermore, if $B$ is an $A$-algebra then $\Hom{A}{\HS^m_{A/R}}{B} \subset \Hom{R}{\HS^m_{A/R}}{B}$ is identified with the sub $\HSDer{R}{m}{A}{B}_0 \subset \HSDer{R}{m}{A}{B}$ of HS-derivations $\varphi$ with $\varphi_0 : A \to B$ equal to the structure map. It is clear that representing $\HSDer{R}{m,B}{A}{-}_0$ on the category of $A$-algebras uniquely determines $\HS^m_{A/R}$ with its $A$-algebra structure and universal HS-derivation whose zeroth term agrees with the structure map.   
\end{rmk}

\begin{prop}
Let $f : X \to S$ be an $S$-scheme. Then these glue together to give a sheaf $\HS_{X/S}^m$ representing,
\[ \Hom{f^{-1} \struct{S}}{\HS_{X/S}^m}{\cA} = \HSDer{f^{-1} \struct{S}}{m}{\struct{X}}{\cA} \]
where $\cA$ is any sheaf of $\struct{X}$-algebras.
\end{prop}

\begin{lemma}
If $A \to B$ is a map of $R$-algebras then there is an exact sequence,
\begin{center}
\begin{tikzcd}
\HS_{A/R}^m \ot_A B \arrow[r] & \HS_{B/R}^m \arrow[r] & \HS^m_{B/A} \arrow[r] & 0
\end{tikzcd}
\end{center}
\end{lemma}

\begin{proof}
Surjectivity is immediate from the presentation. Thus we need to show that the kernel is generated by $\HS_{A/R}^m$. To show this, it suffices to show that,
\[ 0 \to \Hom{B}{\HS_{B/A}^m}{C} \to \Hom{B}{\HS_{B/R}^m}{C} \to \Hom{B}{\HS_{A/R}^m \ot_A B}{C} \] 
is exact for any $C$. But this is exactly,
\[ 0 \to \HSDer{A}{m}{B}{C}_0 \to \HSDer{R}{m}{B}{C}_0 \to \HSDer{R}{m}{A}{C}_0 \]
and the kernel is exactly those HS-derivations which vanish on the image of $A$ and hence correspond exactly to $A$-linear derivations by definition.
\end{proof}

\begin{lemma}
If $A \to B$ is an \etale map of $R$-algebras then $\HS_{A/R} \ot_A B \to \HS_{B/R}$ is an isomorphism.
\end{lemma}

\begin{proof}
By localizing we can assume that $A \to B$ is standard \etale meaning $B = A[x]_g/(f(x))$ where $f'(x)$ is a unit. From the exact sequence, it suffices to show injectivity and $\HS_{B/A}^m = 0$. Indeed, $f(x) = 0$ so $\d_i{(f(x))} = 0$ but $\d_1(f(x)) = f'(x) \d{x}$ so $\d_1{x} = 0$ since $f'(x)$ is a unit. Now assume that $\d_i(x) = 0$ for $i < k$ we will show that $\d_{k}(x) = 0$. First compute,
\[ \d_k(x^n) = n x^{n-1} \d_k(x) \]
because $\d_k(x^m) = \d_k(x^{m-1}) x + x^{m-1} \d_k(x)$ since the intermediate terms are zero so the claim is true by induction. Therefore, we see that $\d_k(f(x)) = f'(x) \d_k(x)$ but $f'(x)$ is a unit and thus $\d_k{x} = 0$ so we win. Now to show injectivity we need to show that if $C$ is a $B$-algebra then the map
\[ \HSDer{R}{m}{B}{C}_0 \to \HSDer{R}{m}{A}{C}_0 \]
is surjective. Given $\varphi : A \to C$ it suffices to specify $\varphi'(x)$ such that it becomes a HS-derivation. Because $f'(x) \d_k(x) = p$ for $p$ a polynomial $\d_i(x)$ for $i < k$ and $\d_i(a)$ for $a \in A$ we can specify $\varphi'_k(x) = - \varphi'_{<k}(p) \cdot \varphi_0(f'(x))^{-1}$ where $\varphi_0 : B \to C$ is the struture map and $f'(x)$ is a unit so this makes sense. Then it is elementary to check this defines a HS-derivation.	 
\end{proof}

\begin{prop}
If $X/S$ is locally of finite type then $\HS_{X/S}^m$ is graded by coherent $\struct{X}$-algebra. It is graded by vector bundles if $X/S$ is smooth.
\end{prop}

\begin{proof}
This immediately reduces to the corresponding property for $\HS_{A/R}$. If $R[x_1, \dots, x_n] \onto A$ then we claim that the natural map $\HS_{R[x_1, \dots, x_n]/R} \to \HS_{A/R}$ is surjective then the finite generation is obvious from examining the structure of the Hasse-Schmidt algebra of a polynomial ring. For smoothness we use the \etale-local structure to reduce to the polynomial ring. Furthermore,
\[ \HS_{R[x_1, \dots, x_n]/R}^{m,d} = \bigoplus R \, \d_{i_1}(x_{j_1}) \cdots \d_{i_r}(x_{j_r}) \]
where we sum over all monomials $\d_{i_1}(x_{j_1}) \cdots \d_{i_r}(x_{j_r})$ such that $i_1 + \cdots + i_r = k$ and $i_{\ell} \le m$. 
\end{proof}

\begin{example}
$\HS^0_{A/R} = A$ and $\HS^1_{A/R} = \Sym{R}{A}$.
\end{example}

{\color{red} DO I NEED SMOOTHNESS FOR THE FILTRATION??}

\begin{prop}
There are exact sequences,
\end{prop}


\subsection{Jets a la Jason Starr}

\newcommand{\pr}{\mathrm{pr}}

\begin{theorem}[FGA IV.3 p.267]
Let $p : Y \to X$ be flat and projective and $q : Z \to Y$ finitely-presented quasi-projective morphism then the functor,
\[ T \to \{ (f : T \to X, g : T \times_X Y \to Z) \mid q \circ g = \pr_2 \} \]
(i.e. to each $X$-scheme $T$ a $Y$-morphism $T \times_X Y \to Z$)
is representable by a universal pair,
\[ (r : \Pi_{Z/Y/X} \to X, \, s : \Pi_{Z/Y/X} \times_X Y \to Z) \]
\end{theorem}

\begin{rmk}
In the case $Z = W \times_X Y$ we just get the Hom scheme $\Hom{X}{Y}{W}$. Furthermore, if $q : Z \to Y$ is a bundle then this represents the functor of sections of $q$ because  the functor can be identified with, an $X$-scheme $T \to X$ and a morphism $g : T \times_X Y \to T \times_X Z$ such that,
\begin{center}
\begin{tikzcd}
& T \times_X Z \arrow[dd, "\id \times q"]
\\
T \times_X Y \arrow[ru, "g"] \arrow[rd, equals] & 
\\
& T \times_X Y
\end{tikzcd}
\end{center}  
\end{rmk}

Let $S$ be a scheme and $f : X \to S$ be a smooth separated morphism and let $\Delta_{X/S} : X \to X \times_S X$ be the relative diagonal which is a closed embedding defined by an ideal sheaf $\I$. Let $\Delta_e : X_e \embed X \times_S X$ be the closed embedding corresponding to $\I^{e+1}$. The associated projections $\pr_i : X_e \to X$ are finite flat (hence proper). 

\begin{defn}
Let $\pi : Z \to X$ be finitely presented and quasi-projective then so is the base change,
\[ B \times_{X, \pr_1} (X \times_X X) \to X \times_S X \]
thus the pullback $\pi_e : B_e \to X_e$ over $\Delta_e$ is also finitely presented and quasi-projective. Then the ``relative jets'' parameter space is the universal pair,
\[ (r : \Pi_{B_e / X_e / X} \to X, \, s : \Pi_{B_e/X_e/X} \times_{X, \pr_2} X_e \to B_e) \]
representing the functor, defined via $\pr_2 : X_e \to X$,
\[ f : T \to S, \,\, g : T \times_X X_e \to B_e \quad \text{such that} \quad \pi_e \circ g = \pr_2 \]  
\end{defn}

\begin{rmk}
We think of $\pi : B \to X$ as a bundle and $J^e(\pi) := \Pi_{B_e/X_e/X}$ is then the bundle of jets of sections of $\pi$. Note, a map $T \times_{X, \pr_2} X_e \to B_e$ over $X_e$ is the same as a map $T \times_{X, \pr_2} X_e \to B$ over $X$ (where we view the $X$-structure of $T \times_{X, \pr_2} X_e$ through $\pr_1$ on $X_e$) since $B_e = B \times_{X, \pr_1} X_e$. Consider the case, $B = Z \times_S X$ where $Z$ is an $S$-scheme. This case $\Pi_{B_e/X_e/X}$ is the space of jets of morphisms $f : X \to Z$. Indeed, in this case, $B_e = Z \times_S X_e$ and hence a $X_e$-morphism $g : T \times_X X_e \to B_e$ is just as $S$-morphism $T \times_X X_e \to Z$.
\end{rmk}

\begin{rmk}
Associated to the space of jets $\Pi = J^e(\pi)$ and a point $x : S \to X$ we get the space of jets at the point is $\Pi_x := \Pi \times_X S$.
\end{rmk}

\begin{example}
For $X = \A^1_S$ then $X_e = X \times \Delta^e$ where,
\[ \Delta^e = \Spec{\ZZ[t]/(t^{e+1})} \]
and we take $B = \A^1_S \times_S Z$ then we get,
\[ \Pi := \Pi_{B_e/X_e/X} = \A^1_S \times J_e(Z) \]
since it represents, as an $\A^1_S$-scheme, morphisms $T \times \Delta^e \to Z$ over $S$. Therefore, $\Pi_0 = J_e(Z)$ is the arc scheme in the usual sense.
\end{example}

\newcommand{\VV}{\mathbb{V}}

\begin{example}
Conversely, suppose that $Z = \A^1_S$ so we consider jets of maps $X \to \A^1_S$. Then,
\[ \{ T \to \Pi \} = \{ (T \to X, T \times_X X_e \to \A^1_S) \} = \{ (f : T \to X, s \in \Gamma(T \times_X X_e)) \} \] 
However, $\pi_2 : X_e \to X$ is affine corresponding to the algebra $\pr_{2*} (\struct{X \times_S X} / \I^{e+1}) = J^e(X)$ and hence $\Gamma(T \times_X X_e) = \Gamma(T, f^* J^e(X))$ since cohomology along an affine map commutes with base change. Therefore,
\[ \Gamma(T \times_X X_e) = \Gamma(T, f^* J^e(X)) = \Hom{\struct{T}\text{-alg}}{f^* \Sym{\bullet}{J^e(X)^\vee}}{\struct{T}} = \Hom{X}{T}{\VV_X(J^e(X))} \]
\end{example}

\begin{rmk}
To any section $s : X \to B$ of $\pi : B \to X$ we get a corresponding $X$-point of $J^e(\pi)$ (i.e. a section of the bundle of jets corresponding to the $e^{\text{th}}$-jet of $s$). Indeed, consider $X_e \to B_e = B \times_{X, \pr_1} X_e$ defined by $(s \circ \pr_1, \id)$ which is an $X_e$-morphism. However, to a $T$-point $s : T \to B$   of $B$ (which we can think of a section of $B \times_X T \to T$) we cannot associate a $T$-point of $J^e(\pi)$ meaning a morphism $T \times_{X, \pr_2} X_e \to B_e$ over $X_e$ because to write $s \times \id$ we need that the projections to $X$ commute with $\id : X \to X$ which they do not since these are $\pi_1, \pi_2 : X_e \to X$. This shows that the map $\{ \text{sections of } \pi : B \to X \} \to \{ \text{sections of } \pi : \Pi \to X \}$ is nonlinear. For the case $T = X$ the fact we use is that $X \times_{X, \pr_1} X_e \cong X \times_{X, \pr_2} X_e$ over $X_e$. In general, an isomorphism $T \times_{X, \pr_1} X_e \cong T \times_{X, \pi_2} X_e$ over $X_e$ is a sort of higher-order connection on $T$ over $X$.
\end{rmk}

\begin{rmk}
Notice that in the definition of $B_e$ we use $\pi_1$ while in the definition of the functor we form $T \times_X X_e$ through $\pi_2$. This is essential to get the jets of nontrivial bundles correct. It is analogous to how in the definition: $J^e(\E) := \pr_{2*} \pr_1^* \E$ for the projections $\pr_i : X_e \to X$ it is essential we use the two different projections. This means that the diagram,
\begin{center}
\begin{tikzcd}
B_e \arrow[d] \arrow[r, "\pi_e"] & X_e \arrow[d, "\pr_1"]
\\
B \arrow[r, "\pi"] & X
\end{tikzcd}
\end{center}
commutes for $\pr_1$ but \textit{not} for $\pr_2$ while we use $\pr_2$ for the construction of $T \times_X X_e$. 
\end{rmk}

\begin{example}
Let $\pi : B \to X$ be a vector bundle $\VV_X(\E) \to X$. A morphism $T \times_{X, \pr_2} X_e \to B$ over $X$ (through $\pr_1 : X_e \to X$) given $f : T \to X$ corresponds to a morphism of algebras,
\[ \pr_1^* \Sym{\bullet}{\E^\vee} \to \struct{T \times_{X, \pr_2} X_e} \]
and hence a section,
\[ s \in \Gamma(T \times_{X, \pr_2} X_e, \pr_1^* \E) = \Gamma(T, f^* \pr_{2*} \pr_1^* \E) = \Hom{X}{T}{\VV_X(J^e(\E))} \]
where we used that $\pr_2 : X_e \to X$ is affine so pushforward commutes with any base change. 
\end{example}

HOW TO MAKE THE ARCS TANGENT TO SOMETHING?

THE DERIVATIVE OPERATOR ON GG-JETS

REPARAMETRIZATION OF ARCS

\subsection{Semple Jets are Invariant Hasse-Schmidt Jets}

\newcommand{\aff}{\mathrm{aff}}

Construction: given a vector bundle $\E$ on $X$ note that $\struct{\V(\E)}$ is canonically identified with the graded ring
\[ \nSym{\bullet}{\E^\vee} = \bigoplus_{n \ge 0} \struct{\P(\E)}(n) \]
via the $\Gm$-equivariant rational map
\begin{center}
\begin{tikzcd}
\V(\E) \arrow[rd] \arrow[rr, dashed] & & \P(\E) \arrow[ld]
\\
& X
\end{tikzcd}
\end{center}
whose indeterminancy locus is in codimension $\rank{\E}$ and therefore functions extend over all of $\V(\E)$ by Harthogs' theorem (note that the case $\rank{\E} = 1$ is trivial for other reasons). Suppose we have a pair $(X, \E)$ where $\E$ is a vector bundle equipped with a map $\E \to \T_X$ (not assumed to be injective) and we construct the Semple tower $(X_k, \E_k)$. We can interpret this construction in terms of ``physical'' vector bundles as well. On $\V(\E)$ there is a map $\struct{\V(\E)}(-1) \to \pi^* \E$ of $\Gm$-equivariant coherent sheaves on $\V(\E)$ (or equivalently of graded $\cA_\E := \nSym{\bullet}{\E^\vee}$-modules where the $(-1)$ corresponds to the grading) given by the canonical cocontraction map
\[ \nSym{n-1}{\E^\vee} \to \nSym{n}{\E^\vee} \ot \E \] 
\[ s_1 \cdots s_{n-1} \mapsto \sum_{i = 1}^r s_1 \cdots s_{n-1} e_i \ot e^i \]
where $e_i$ is a local basis of $\E^\vee$ and $e^i$ is the dual basis. Therefore, setting $\wt{X}^{\aff} = \V(\E)$ we can create a diagram
\begin{center}
\begin{tikzcd}[row sep = small]
0 \arrow[r] & \T_{\wt{X}^\aff /X} \arrow[dd, equals] \arrow[r] & \wt{\E}^\aff \arrow[dd] \pullback \arrow[r] & \struct{\wt{X}^{\aff}}(-1) \arrow[d] \arrow[r] & 0
\\
& & & \pi^* \E \arrow[d]
\\
0 \arrow[r] & \T_{\wt{X}^{\text{aff}}/X} \arrow[r] & \T_{\wt{X}^\aff} \arrow[r] & \pi^* \T_{X} \arrow[r] & 0
\end{tikzcd}
\end{center}
the only difference to the projective case being that the downward maps are now not injective over the zero section. We now iterate this construction to produce a tower of directed affine bundles along with $\Gm$-equivariant maps to the ordinary Semple tower,
\begin{center}
\begin{tikzcd}
\vdots \arrow[d] & \vdots \arrow[d]
\\
(X^{\aff}_2, \E_2^\aff) \arrow[d] \arrow[r, "\varphi_2", dashed] & (X_2, \E_2) \arrow[d]
\\
(X^{\aff}_1, \E_1^\aff) \arrow[d] \arrow[r, "\varphi_1", dashed] & (X_1, \E_1) \arrow[d]
\\
(X, \E) \arrow[r, equals] & (X, \E)
\end{tikzcd}
\end{center}
Now the claim is that the $\Gm$-equivariant maps induce canonical injections of $\struct{X}$-algebras
\[ \cP^{k, \bullet} = \bigoplus_{d \ge 0} \pi_{k*} \struct{X_k}(d)\embed \pi_{k*} \struct{X_k^{\aff}}  \]
Indeed, consider the diagram
\begin{center}
\begin{tikzcd}[column sep={4em,between origins},row sep=1em]
& 0 \arrow[rr] & & \varphi_k^* \T_{X_k / X_{k-1}} \arrow[rr] \arrow[dd, equals] & & \varphi_k^* \E_k \arrow[rr] \arrow[dd] & & \varphi_k^* \struct{X_k}(-1) \arrow[rr] \arrow[dd] & & 0
\\
0 \arrow[rr] & & \T_{X_k^\aff / X_{k-1}^\aff}|_{U_k} \arrow[rr, crossing over] \arrow[ru] & & \E^\aff_k |_{U_k} \arrow[ru] \arrow[ru] \arrow[rr, crossing over] & & \struct{X_k^\aff}(-1)|_{U_k} \arrow[ru, equals] \arrow[rr, crossing over] & & 0
\\
& 0 \arrow[rr] & & \varphi^*_k \T_{X_k/X_{k-1}} \arrow[rr] & & \varphi^*_k \T_{X_k} \arrow[rr] & & \varphi^*_k \pi^* \T_{X_{k-1}} \arrow[rr] & & 0
\\
0 \arrow[rr] & & \T_{X_k^\aff / X_{k-1}^\aff} |_{U_k} \arrow[rr] \arrow[from=uu, equals, crossing over] \arrow[ur] & & \T_{X_k^\aff} |_{U_k} \arrow[from=uu, crossing over] \arrow[ur] \arrow[rr] & & \pi^* \T_{X_{k-1}^\aff}|_{U_k} \arrow[from=uu, crossing over] \arrow[rr] \arrow[ur] & & 0
\end{tikzcd}
\end{center}
Thus, given the map $\varphi_k : (X_k^\aff, \E_k^\aff) \rat (X_k, \E_k)$ we can build $\varphi_{k+1} : (X_{k+1}^\aff, \E_{k+1}^\aff) \rat (X_{k+1}, \E_{k+1})$. 
\bigskip\\
Indeed, given a $\Gm$-equivariant rational map $f : X \rat Y$ and $\Gm$-equivariant vector bundles $\E_X$ and $\E_Y$ and a $\Gm$-equivariant morphism of vector bundles $\varphi : \E_X|_U \embed f^* \E_Y$ then we produce a $\Gm$-equivariant rational map $f' : \V(\E_X) \rat \P(\E_Y)$ which is defined on $U' = \pi^{-1}(U) \sm V(\varphi)$ where $V(\varphi)$ is the locus
\[ V(\varphi) = \{ x \in U \mid v \in \ker{\varphi_x} \} \] 
The map $f' : U' \to \P(\E_Y)$ is defined by 
\[ \struct{\V(\E_X)}(-1)|_{U'} \to \pi^* \E_X|_{U'} \xrightarrow{\pi^* \varphi} f^* \E_Y \]
which is a subbundle over $U'$ because over $U'$ the composite is fiberwise injective.
\bigskip\\
In the case of the Semple tower, $\E_0^{\aff} = \E_0$ with rank $r$ and then $\rank{\E_k} = 1 + \rank{\T_{X_k/X_{k-1}}} = \rank{\E_{k-1}}$ and $\rank{\E_k^{\aff}} = 1 + \rank{\T_{X_k^\aff / X_{k-1}^\aff}} = 1 + \rank{\E_{k-1}^{\aff}}$ so $\rank{\E_k} = k + r$. Now $U_1 = X_1 \sm V(0)$ has codimension $r$. Furthermore, $\varphi_k \E_k^{\aff}|_{U_k} \onto \varphi^*_k \E_k$ is surjective with kernel of rank $k$ inside $\E_k^{\aff}$ which has rank $r + k$ so $V(\varphi_k)$ has codimension $r$. Therefore, we can build the morphisms in the Semple tower and each $\varphi_k$ is naturally defined away from codimension $r$. Since $r \ge 2$ sections extend and therefore there is an injective pullback map,
\[ \cP^{k, \bullet} = \bigoplus_{d \ge 0} \pi_{k*} \struct{X_k}(d)\embed \pi_{k*} \struct{X_k^{\aff}}  \]

\begin{defn}
Consider the projectivized Semple tower $(X_m, \E_m)$ where $\E_0 = \T_X$ then the \textit{projectivied Semple $m$-jet space} is defined as $P_k \E = X$ and the \textit{projectivied Semple $m$-jet bundle} is defined as $\cP^{m,d}_X = \pi_{m*} \struct{X_m}(d)$. Likewise, consider the affine Semple tower $(X_m^\aff, \E_m^\aff)$ where $\E_0 = \T_X$. Then the \textit{affine Semple $m$-jet space} is defined $J_m X = X_m^{\aff}$ and the \textit{affine Semple $m$-jet bundle} is $\E^{m,d} = [\pi_{m*} \struct{X^\aff_m}]_d$ where we take the degree $d$ part induced by the $\Gm$-action.
\end{defn}

\begin{prop}
Let $\cP^{m,d}_X = \pi_{m*} \struct{X_m}(d)$ where $(X_m, \E_m)$ is the projectivied Semple $m$-jet bundle $P_k \E = X_k$ with $\E_0 = \T_X$. Then there is a canonical doubly graded injection,
\[ \cP^{m,d} \embed \HS^{m,d}_{X} \] 
of $\struct{X}$-algebras.
\end{prop}

\begin{proof}
To illustrate, for $m = 0$ we set,
\[ \cP^{0,d} = \HS^{0,d}_{X} = \begin{cases}
\struct{X} & d = 0
\\
0 & d > 0
\end{cases} \]
Now for $m = 1$ there are canonical isomorphisms,
\[ \cP^{1,d} = \nSym{d}{\Omega_X} = \HS^{1,d}_{X} \]
To prove the claim, it suffices for each quasi-coherent $\struct{X}$-algebra $\cA$ to produce a functorial degree-preserving surjection
\[ \Hom{\struct{X}}{\HS_{X/S}^m}{\cA} \onto \Hom{\struct{X}}{\cP^{m,\bullet}}{\cA} \]
Note that
\[ \Hom{\struct{X}}{\HS_{X/S}^m}{\cA} = \Hom{S}{\Delta_{\cA}^m}{X}_0 \]
where $\Delta^m_{\cA} = \rSpec{X}{\cA[t]/(t^{m+1})}$ and the zero denotes that we are only considering maps compatible with the structure map $\rSpec{X}{\cA} \to X$. Given $f : \Delta^m_{\cA} \to X$ there is a differential
\[ \d{f} : f^* \Omega_X \to \Omega_{\Delta^m_{\cA}/\cA} = [\struct{\Delta_{\cA}^m} \d{t}]/((m+1) t^m \d{t}) \to \struct{\Delta^{m-1}_{\cA}} \]
where the last map takes $\d{t} \mapsto 1$ which is well-defined since $t^m \d{t} \mapsto t^m = 0$. This produces a morphism $f' : \Delta^{m-1}_{\cA} \to \V(\T_X) = \wt{X}^{\aff}$ lifting $f$. Note that if $\d{f}$ factors through $f^* \Omega_X \to f^* \E^\vee$ then the induced map $f'$ satisfies
\[ \d{f'} : f'^* \Omega_{\wt{X}^{\aff}} \to \Omega_{\Delta^{m-1}_{\cA}/\cA} \]
factors through $f'^* \Omega_{\wt{X}^{\aff}} \to \wt{\E}^\vee$ because, by definition, the following diagram commutes
\begin{center}
\begin{tikzcd}
f^* \Omega_X \arrow[r] \arrow[d] & f'^* \struct{\wt{X}^\aff}(1) \arrow[d] \arrow[rdd, bend left]
\\
f'^* \Omega_{\wt{X}^{\aff}} \arrow[rrd, bend right] \arrow[r, "\d{f'}"] & \wt{\E}^\aff \arrow[rd, dashed]
\\
& & \struct{\Delta^{m-1}_{\cA}}
\end{tikzcd}
\end{center}
Iterating this process produces a map $\rSpec{X}{\cA} \to X_m^{\aff}$ lifting $\rSpec{X}{\cA} \to X$. The pullback map of sections then gives the required map of algebras 
\[ \cP^{m, \bullet} \embed \pi_{k*} \struct{X_m^{\aff}} \to \cA \]
It suffices to prove that the obtained map
\[ \Hom{\struct{X}}{\HS_{X/S}^m}{\cA} \onto \Hom{\struct{X}}{\cP^{m,\bullet}}{\cA} \]
is surjective and graded. It is graded because everything so constructed is $\Gm$-equivariant for the obvious $\Gm$-action on $\Delta_{\cA}^m$ which corresponds to the grading on $\HS_{X/S}^m$. To check surjectivity, since $X$ is smooth, using the \etale-local structure, we reduce to cheking this property for $\A^n_S$. In this case we can directly compute. There is a presentation
\[ \HS_{X/S}^m = \struct{S}[\d_i(x_j)]_{\substack{0 \le i \le m \\ 0 \le j \le n}} \]
We now consider the map $\varphi_{ij} : \HS^m_{X/S} \to \struct{S}$ sending $\d_i(x_j) \mapsto 1$ and all other to zero. This corresponds to the Hasse-Schmidt differential $(D_0, \dots, D_m)$ where $D_j(x_i) = 1$ and $D_{j'}(x_i') = 0$ for all other $i' \neq i$ and $j' \neq j$. Now we consider the lift of the map
\[ \varphi_{ij} : \Delta^m_{S} \to X \]
to the Semple tower. We construct
\[ X_1 = \rSpec{S}{\struct{S}[x_1, \dots, x_n][\d{x_1}, \dots, \d{x_n}]} \]
and then $\pi^* \Omega_X \to \struct{\V(\T_X)}(1)$ is given by $\d{x_1} \mapsto$ 
\[ X_2 = \rSpec{S}{\struct{S}[x_1, \dots, x_n][\d{x_1}, \dots, \d{x_n}][\d_2{x_1}, \dots, \d_2{x_n}][s]} \]
\end{proof}


 
\end{document}
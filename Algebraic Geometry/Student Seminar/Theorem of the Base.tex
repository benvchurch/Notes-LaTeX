\documentclass[12pt]{article}
\usepackage{import}
\import{../}{AlgGeoCommands}

\begin{document}

\section{Theorem of the Base}

\begin{theorem}
Let $X$ be a projective variety over a field $k$. Then $\NS{X}$ is finitely generated. Moreover, if $X$ varies in a flat family over a connected Noetherian scheme $S$ then $\rank{\NS{X}}$ and $\# \NS{X}_{\tors}$ in the fibers are bounded. 
\end{theorem}

Over $\CC$ we can use the exponential sequence
\[ 0 \to \Z \to \struct{X} \to \struct{X}^\times \to 0 \]
which gives a long exact sequence
\[ H^1(X, \struct{X}) \to \Pic{X} \to H^2(X, \Z) \]
Therefore, if we can show that $H^1(X, \struct{X})$ map to algebraically trivial cycles then $\NS{X} \embed H^2(X, \Z)$ and hence it is finitely generated. 

\subsection{The Picard Scheme}

\begin{defn}
Let $X$ be a scheme over $S$ then we define the \textit{Picard stack} $\cPic_{X/S}(T)$ is the groupoid of invertible sheaves on $X_T$. 
\end{defn}

\begin{defn}
Let $X$ be a scheme over $S$ then we define 
\begin{enumerate}
\item \textit{Picard presheaf} $p \fPic_{X/S}(T)$ to be the isomorphism classes of $\cPic_{X/S}(T)$ this is usually not a sheaf, for example any line bundle arising from the base $T$ is locally trivial but does not have to be trivial
\item Let $\sigma : S \to X$ be a section. Then the rigidified Picard sheaf is the sheaf,
\[ T \mapsto \{ (\L, \alpha) \mid \L \in \Pic{X_T} \quad \alpha : \sigma_T^* \L \iso \struct{T} \} \]
\end{enumerate} 
\end{defn}

\begin{defn}
The \textit{Picard scheme} is the coarse moduli space for $\cPic_{X/S}$ and this represents the fppf sheafification of $p \fPic_{X/S}$.
\end{defn}

\begin{prop}
Let $X$ be a scheme over $S$ such that $p_* \struct{X} = \struct{S}$ holds universaly (i.e. for any $T \to S$ the map $\struct{T} \iso (p_T)_* \struct{X_T}$ is an isomorphism) then
\begin{enumerate}
\item for any geometric point $\Spec{K} \to S$ the geometric points of the functor are
\[ p \fPic_{X_K/K}^{\fppf}(K) = \Pic{X_K} \] 
\item if $X/S$ has a section $\sigma$, then forgetting $\alpha$ induces an isomorphism
\[ \Pic{X \times_S T} / \Pic{T} \iso \fPic_{(X, \sigma)/S}(T) \]
Moreover, pairs $(\L, \sigma)$ have trivial automorphism group.
\end{enumerate}
\end{prop}

\begin{proof}
For (b) given $\L$ on $X \times_S T$ we alter it to get the rigidifed bundle $\L \ot p^* \sigma^* \L$.
\end{proof}

\subsection{Over a field}

\begin{theorem}
Let $X$ be a projective variety over a field $k$. Then
\begin{enumerate}
\item $p\fPic_{X/k}^{\et}$ is represetnable by a group scheme lfp over $k$
\item $\fPic^\circ_{X/k}$ makes sense and is of finite type
\item If $X$ is geometrically normal then $\fPic_{X/k}^\circ$ is proper
\item there is an isomorphism $T_0 \fPic_{X/k} \iso H^1(X, \struct{X})$ hence $\dim{\fPic_{X/k}} \le \dim_k H^1(X, \struct{X})$ which equality iff $\fPic_{X/k}$ is smooth.
\end{enumerate}
\end{theorem}

\begin{rmk}
If $k$ has characteristic zero then $\fPic_{X/k}$ is always smooth by Cartier's theorem. However, if $H^2(X, \struct{X}) = 0$ then $\fPic_{X/k}$ is smooth.
\end{rmk}

\subsection{Over a Base}

\newcommand{\cN}{\mathcal{N}}

\begin{theorem}
Let $X \to S$ be a flat projective scheme over a locally noetherian scheme $S$. Then
\begin{enumerate}
\item if $X$ has integral geometric fibers then $p \fPic_{X/S}^{\fppf}$ is representable by a separated group scheme locally of finite type over $S$
\item If in addition $X$ has geometrically normal fibers then there exists a closed subscheme $\fPic^\circ_{X/S} \embed \fPic_{X/S}$ which is fiberwise $\fPic^\circ$ and it is proper over $S$
\item if $\fPic_{X/S}$ exists then there is an isomorphism $\cN_{0} \iso R^1 p_* \struct{X}$ \item if in addition $S$ is a reduced scheme of characteristic zero then $\fPic_{X/S}$ is smooth or if $s \in S$ such that $H^2(X_s, \struct{X_s}) = 0$ then $\fPic_{X/S}$ is smooth over a neighborhood of $s$. In both of these cases $\fPic^\circ_{X/S}$ is an open group subscheme of $\fPic_{X/S}$.  
\end{enumerate}
\end{theorem}


[EGA IV, vol 3, prop 15.6.8]

and [EGA IV, vol 3, 15.6.4]. 

\subsection{Notions of Equivalence}

\begin{defn}
Let $X$ be a projective variety over $k$. Let $\L \in \Pic{X}$ say that $\L$ is algebraically trivial if there exists a connected scheme $T$ over $k$ and $x,x' \in T(k)$ and $\Xi \in p \fPic_{X/k}(T)$ such that $\Xi|_x \cong \struct{X}$ and $\Xi|_{x'} \cong \L$. Furthermore,
\begin{enumerate}
\item $\L$ is algebraically torsion, if $\exists m \neq 0$ such that $\L^{\ot m}$ is algebraically trivial
\item numerically trivial if for every curve $C \subset X$ we have $\deg{\L|_C} = 0$.
\end{enumerate}
\end{defn}

\begin{theorem}
The following are equivalent
\begin{enumerate}
\item $\L$ is algebraically torsion
\item $\{ \L^m \}$ is bounded (meaning it lies in a quasi-compact open of $\fPic_{X/k}$)
\item $\L$ is numerically trivial
\end{enumerate}
\end{theorem}

\begin{proof}
$(a) \implies (b) \implies (c)$ are easy. We will spend the next section proving the coverse.
\end{proof}

\begin{defn}
Let $X$ be a projective variety over $k$. Let $\F$ be a coherent sheaf on $X$. We say that $\F$ is $m$-regular if for all $i > 0$ we have $H^i(X, \F(m-i)) = 0$. 
\end{defn}

\begin{rmk}
Notice that $m$-regularity is independent of the field so we may assume that $k = \bar{k}$. Then there always exsits a hyperplane section avoiding all the associated points of $\F$ so there always exists a sequence of the form
\begin{center}
\begin{tikzcd}
0 \arrow[r] & \F(-1) \arrow[r] & \F \arrow[r] & \F|_H \arrow[r] & 0
\end{tikzcd}
\end{center}
\end{rmk}

\begin{prop}
Let $X$ be a projective variety over $k$ and $\F$ is an $m$-regular coherent sheaf then,
\begin{enumerate}
\item $\F|_H$ is $m$-regular.
\item $\F$ is $(m+1)$-regular.
\item the map $H^0(X, \struct{X}(i)) \ot H^0(X, \F(m)) \to H^0(X, \F(m + i))$ is surjective
\item $\F(m)$ is generated by global sections.
\end{enumerate}
\end{prop}

\begin{proof}
Using the exact sequence,
\begin{center}
\begin{tikzcd}
0 \arrow[r] & \F(m-1) \arrow[r] & \F(m) \arrow[r] & \F|_H(m) \arrow[r] & 0
\end{tikzcd}
\end{center}
to get the LES
\begin{center}
\begin{tikzcd}
H^i(X, \F(m-i-1)) \arrow[r] & H^i(X, \F(m-i)) \arrow[r] & H^i(H, \F|_H(m-i)) \arrow[r] & H^{i+1}(X, \F(m-i-1)) 
\end{tikzcd}
\end{center}
the terms $H^i(X, \F(m-i)) = H^{i+1}(X, \F(m-i-1)) = 0$ by hypothesis and hence $H^i(H, \F|_H(m-i)) = 0$.

For (b) we induct on dimension and the previous sequence with $m \mapsto m + 1$. Then 


For (c) because $H^1(X, \F(m-1)) = 0$ we can lift sections of $H^0(H, \F|_H(m))$ to $H^0(X, \F(m))$ and therefore we can proceed by induction. 

For (d) we know that $\F(n)$ is generated by global sections for $n \gg 0$. However, we can write,
\[ H^0(X, \struct{X}(n-m)) \ot H^0(X, \F(m)) \onto H^0(X, \F(n)) \]
and therefore the map $\struct{X}(n-m) \ot H^0(X, \F(m)) \to \F(n)$ is surjective hence the map $\struct{X} \ot H^0(X, \F(m)) \to \F(n)$ is surjective proving the claim.
\end{proof}

\begin{lemma}
Let $X$ be a projective variety over $k$. Let $\F$ be a coherent sheaf with $\dim{\Supp{}{\F}} \le r$ then there exists a constant $A(\F)$ such that for all $\L \sim_{\text{num}} 0$ then $h^0(\F \ot \L(n)) \le A(\F) { n + r \choose r}$.
\end{lemma}

\begin{proof}
For sequences
\begin{center}
\begin{tikzcd}
0 \arrow[r] & \F \arrow[r] & \G \arrow[r] & \H \arrow[r] & 0 
\end{tikzcd}
\end{center}
then for $\G$ we get the constant $A(\F) + A(\H)$. Hence we reduce to sheaves of the form $\F = \struct{Z}$ for $Z \subset X$ an irreducible closed subscheme. Consider a hyperplane section $H \embed Z$ and consider the sequence
\begin{center}
\begin{tikzcd}
0 \arrow[r] & \L(j - 1) \arrow[r] & \L(j) \arrow[r] & \L(j)|_H \arrow[r] & 0
\end{tikzcd}
\end{center}
therefore
\[ h^0(Z, \L(j)) \le h^0(Z, \L(j-1)) + A(\struct{H}) { j + r - 1 \choose r - 1 } \]
and hence taking the sum
\[ h^0(Z, \L(n)) \le A(\struct{H}) { n + r \choose r } + h^0(Z, \L) \]
but $\L$ is numerically trivial so $h^0(Z, \L) \le 1$.
\end{proof}

\begin{prop}
Let $X$ be a projective variety then there exists $m(X)$ such that for all $L \sim_{\text{num}} 0$ then $\L$ is $m(X)$-regular. Also $\chi(X, \L(n)) = \chi(X, \struct{}(n))$. 
\end{prop}

\begin{proof}
Let $\L$ be numerically trivial and $\L(d)$ very ample. Let $F$ be an effective divisor in the class of $\struct{X}(d)$ and $G$ in the class of $\L(d)$. Consider the sequences
\begin{center}
\begin{tikzcd}
0 \arrow[r] & \struct{X}(-d) \arrow[r] & \struct{X} \arrow[r] & \struct{F} \arrow[r] & 0
\\
0 \arrow[r] & \L^{-1}(-d) \arrow[r] & \struct{X} \arrow[r] & \struct{G} \arrow[r] & 0
\end{tikzcd}
\end{center}
and apply $- \ot \L^{p}(n+d)$ and take Euler characteristics and subtract,
\[ \chi(X, \L^p(n)) - \chi(X, \L^{p-1}(n)) = \chi(G, \L^p(n+d)) - \chi(F, \L^p(n+d)) + \chi(X, \L^p(n))  \]
For induction we assume that both are known on $F$ and $G$ and therefore the RHS is a polynomial independent of $\L$ so by summation we get
\[ \chi(X, \L^p(n)) = \varphi_1(n) p + \varphi_0(n) \]
for some polynomials $\varphi_1, \varphi_0$ independent of $\L$.
Consider the sequence
\begin{center}
\begin{tikzcd}
0 \arrow[r] & \L^p(n) \arrow[r] & \L^p(n+d) \arrow[r] & \L^p|_F (n + d) \arrow[r] & 0
\end{tikzcd}
\end{center}
For all $n \ge m$ we know $H^i(F, \L^p(n-i)) = 0$ so $H^i(X, \L^p(n)) = H^i(X, \L^p(n+d))$ for all $i \ge 2$ and $n \ge m, p$. But by Serre vanishing this must be zero since we can take the twist very large. Then
\[ h^0(\L^p(n)) - h^1(\L^p(n)) = \varphi_1(n) p + \varphi_0(n) \]
and therefore $h^0(\L^p(n)) \ge p_1(n) p + p_0(n)$ taking some limit. But this contradicts what we previously proved so it must be $p$-independent.
\bigskip\\
To complete the induction we need to also prove the first part. Let $H$ be a hyperplane section then consider
\begin{center}
\begin{tikzcd}
0 \arrow[r] & \L(n) \arrow[r] & \L(n+1) \arrow[r] & \L|_H(n+1) \arrow[r] & 0
\end{tikzcd}
\end{center}
therefore
\[ \chi(H, \L(n+1)) = \chi(X, \L(n+1)) - \chi(X, \L(n)) = \chi(X, \struct{X}(n+1)) - \chi(X, \struct{X}(n)) \]
by what we just proved. Then there exists $m(H)$ such that $\L|_H$ is $m(H)$-regular and for $n \ ge m(H) - 2$ then from the long exact sequence we get vanishing $H^i(X, \L(n-i)) = 0$ for $i \ge 2$. For $i = 1$ we need a different argument. Claim: $h^1(\L(n))$ is strictly decreasing. Indeed, otherwise there is some $n$ such that $h^1(\L(n)) = h^1(\L(n+1))$ but then 
\[ H^0(H, \L(n+1)) \ot H^0(\struct{}(1)) \onto H^0(H, \L(n+2)) \]
is surjecive so $h^1(\L(n)) = h^2(\L+2)$ and so on then we see that $h^1(\L(n))$ never decreases again but it must go to zero by Serre vanishing. Therefore, we know that
\[ h^1(\L(m)) \le h^1(\L(m-1)) = h^0(\L(m-1)) - \chi(\L(m-1)) \le A(H) { n + m \choose r - 1 } \]
\end{proof}

This proves the main theorem because if $\L \sim_{\text{num}} 0$ is a quotient of $\struct{X}(-m)^{\oplus A(X) {m + r \choose r}}$ with hilbert polynomial $\chi(X, \L(n)) = \chi(X, \struct{X}(n))$ independent of $\L$. Therefore, all $\L$ live in a qc component of the Quot scheme.


\section{Talk 2}

\begin{theorem}
Let $X$ be projective variety ove $k$ then $\L \in \Pic{X}$ the following are equivalent
\begin{enumerate}
\item $\L$ is algebraically torsion
\item $\L$ is numerically trivial
\item $\{ \L^n \}$ is bounded
\item $\chi(X, \L^p(n)) = \chi(X, \struct{X})$ for a zariski dense set of integers $(p,n) \in \Z^2$
\end{enumerate}
\end{theorem}

\begin{theorem}
The family of all numerically trivial line bundles on $X$ is bounded.
\end{theorem}

\begin{cor}
$\NS(X)_{\tors}$ is finite.
\end{cor}

\begin{proof}
$\NS(X)_{\tors}$ is numerically trivial mod algebraically trivial line bundles which is the group of connected components of a quasi-compact group scheme which is hence finite. 
\end{proof}

\subsection{Alterations}

\begin{defn}
Let $X$ be a noetheiran scheme. An alteration $Y \to X$ is a proper generically finite map with $Y$ regular.
\end{defn}

\begin{prop}
If $X$ is a variety over a field $k$ then there exists a finite extension $k' / k$ purely inseparable such that $X_{k'}$ has an alteration by a smooth $k'$-variety. 
\end{prop}

For $X / k$ we have $\NS(X) \to \NS(X_{k'}) \to \NS(X)$ is multiplication by $[k' : k]$. 

\newcommand{\Num}{\mathrm{Num}}

\begin{defn}
Let $\Num(X) = \Pic{X} / \sim_{\text{num}}$.
\end{defn}

\begin{prop}
If $f : X \to Y$ is a surjective map of projective varities then $f^* : \Num(Y) \to \Num(X)$ is well-defined injective map. 
\end{prop}

\begin{proof}
Let $\L$ be a line bundle on $Y$ such that $f^* \L$ is numerically trivial we need to show that $\L$ is numerically trivial. Given $C \subset X$ then $\deg_C f^* \L = \deg_{f(C)} \L = 0$ but also every curve in $Y$ is of the form $f(C)$ because we can pull back and take an irreducible component. 
\end{proof}

\subsection{\etale Cohomology}

Recall that for smooth $X$ there is a Chern class $c_1 : \Pic{X} \to H^2_{\et}(X, \Q_\ell(1))$. This preserves algebraic equivalence. 

\begin{lemma}
$c_1$ is well-defined modulo numerical equivalence.
\end{lemma}

\begin{proof}
Indeed, if $\L \sim_{\text{num}} 0$ then $\L^n \sim_{\text{alg}} 0$ for some $n$ so $n c_1(\L) = 0$ but $H^2_{\et}(X, \Q_{\ell}(1))$ is a vector space over $\Q_\ell$ which has characteristic zero so $c_1(\L) = 0$.
\end{proof}

Hence we have a map $c_1 : \Num(X) \to H^2_{\et}(X, \Q_\ell(1))$ and if $c_1(\L) = 0$ then $\deg{\L|_C} = c_1(\L) \frown [C] = 0$ so $\L$ is numerically trivial so $c_1$ is injective. 

\begin{theorem}
$\Num(X)$ is finitely generated.
\end{theorem}

\begin{proof}
We know that $c_1 : \Num(X) \to H^2_{\et}(X, \Q_\ell(1))$ is injective and $H^2_{\et}(X, \Q_\ell(1))$ is a finite $\Q_\ell$-vectorspace. Let $[C_1], \dots, [C_r]$ be the largest independent set of fundamental classes of curves in $H^2_{\et}(X, \Q_\ell(1))$. Consider $\lambda : \Num(X) \to \Z^n$ given by intersection against these curves. Then $\lambda$ is injective becase if $\lambda(\L) = 0$ then for all $\deg_C \L = 0$ because in cohomology we can write
\[ [C] = \sum \alpha_i [C_i] \]
and $\deg_{C_i} \L = 0$ so 
\[ \deg_C \L = \sum_i \alpha_i c_1(\L) \frown [C_i] = 0 \]
Therefore we win. 
\end{proof}

\subsection{Families}

\renewcommand{\NS}{\mathrm{NS}}

\begin{prop}
Let $X$ be a flat proper scheme over a locally noetherian $S$ such that $\fPic_{X/S}$ exists as a scheme. Then the subfunctor $\fPic_{X/S}^{\tau}$ consisting of points which are algebraically torsion in their fiber is open group subscheme. 
\end{prop}

\begin{proof}
Given $T \to \fPic_{X/S}$ corresponding to a line bundle $\L$ we need to show that the locus of points $t \in T$ such that $\L_t \in \fPic_{X_t}^{\tau}$ is open. Note that $t \mapsto \chi(X, \L^p_t(n))$ is constant on each connected component of $T$ by flatness. Therefore, we apply the last part of the theorem to say that the locus is a union of connected components of $T$. 
\end{proof}

\begin{prop}
Let $X$ be a projective scheme with geometrically integral fibers over a noetherian scheme $S$ then $\# \NS(X_{\bar{s}})_{\tors}$ is bounded. 
\end{prop}

\begin{proof}
By noetherian induction I may assume $X \to S$ is flat. Then $\fPic_{X/S}$ exists as a scheme and $\fPic_{X/S}^{\tau}$ is finite type and therefore the number of geometric connected components of its fibers is bounded by Hilbert scheme arguments.
\end{proof}

\begin{example}
Rank of $\rank{\NS(X_s)}$ is not constructible. Let $E_t$ be a family of elliptic curves then consider $E_t \times E_t$. Then,
\[ \rank{(E_t \times E_t)} = 
\begin{cases}
3 & E_t \text{ not CM}
\\
4 & E_t \text{ CM} 
\end{cases} \]
Indeed, if $C \subset E_t \times E_t$ is a cycle then it induces a correspondence and hence a map $\Pic{E_t} \to \Pic{E_t}$. If $E_t$ does not have CM then this is just $[n]$. Then
\end{example}

\begin{theorem}
Let $X$ be a projective scheme over a noetherian scheme with geometrically integral fibers. Then $\rank{\NS(X_{\bar{s}})}$ is bounded. Over a field $\rank{\NS(X)} \le \dim H^2_{\et}(X, \Q_\ell(1))$ and therefore
\[ \rank{\NS(X_{\bar{S}})} \le \dim H^2_{\et}(X_{\bar{s}}, \Q_\ell(1)) = \rank (R^2 p_* \Q_{\ell})_{\bar{s}} \]
\end{theorem}

\begin{proof}
This is true because $R^2 p_* \Q_\ell$ is an \etale local system for $X$ smooth. We may assume $S$ is integral with generic point $\eta$. After finite inseperable extension $X_\eta$ has an alteration $X'_{\eta} \to X_\eta$ we can spread out $X'_\eta$ to $X' / U'$ for $U' \to U \subset S$ pure inseparable map. For every $s \in S$ the map $\Num(X_{\bar{s}}) \to \Num(X'_{\bar{s}})$ injective so we are done. 
\end{proof}

\begin{theorem}[Generic Representability]
Let $X/S$ is a proper scheme over $S$ locally noetherian then $\exists U \subset S$ dense open such that $\fPic_{X_U/U}$ exists. 
\end{theorem}

\begin{cor}

\end{cor}

\end{document}

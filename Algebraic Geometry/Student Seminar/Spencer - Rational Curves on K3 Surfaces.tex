\documentclass[12pt]{article}
\usepackage{hyperref}
\hypersetup{
    colorlinks=true,
    linkcolor=blue,
    filecolor=magenta,      
    urlcolor=blue,
}

\usepackage{import}
\import{../}{AlgGeoCommands}

\begin{document}

\section{Rational Curves on K3 Surfaces}

\begin{defn}
A K3 surface $X / k$ is a smooth projective surface such that $K_X = 0$ and $H^1(X, \struct{X}) = 0$.
\end{defn}

\begin{rmk}
The condition $H^1(X, \struct{X}) = 0$ is used to rule out abelian surfaces. Equivalently we could require $\pi_1(X) = 0$.
\end{rmk}

\subsection{Basics over $\CC$}

The Hodge diamond has $h^{2,0} = 1$ and $h^{1,1} = 20$ and all other (not obviously nonzero by symmetry) are zero.
\bigskip\\
From the exponential sequence,
\begin{center}
\begin{tikzcd}
0 \arrow[r] & \Z \arrow[r] & \struct{X} \arrow[r, "\exp"] & \struct{X}^\times \arrow[r] & 0
\end{tikzcd}
\end{center}
by the long exact sequence,
\begin{center}
\begin{tikzcd}
H^1(X, \struct{X}) \arrow[r] & H^1(X, \struct{X}^\times) \arrow[r] & H^2(X, \Z) 
\end{tikzcd}
\end{center}
but $H^1(X, \struct{X}) = 0$ and $H^2(X, \Z) = \Z^{22}$ thus $\Pic{X}$ is free of rank $\rho \le 22$. In fact since $\Pic{X} \to H^{1,1}(X)$ is injective we have,
\[ \rho := \rank{\Pic{X}} \le 20 \]

\begin{theorem}[Mori-Mukai '83]
Every K3 surface over $\CC$ has a rational curve. Furthermore, a very general K3 has infinitely many. 
\end{theorem}

\begin{defn}
A \textit{polarization} $H$ on a K3 $X$ is an ample line bundle $H$ which is primitive (not an integral multiple of another class). 
\end{defn}

Let $\K_g$ be the moduli stack of polarized K3 surfaces with $H^2 = g$. Fact: $H^2$ is always even. Indeed, by adjunction,
\[ 2 g(H) - 2 = H \cdot (H + K_X) = H^2 \]

\begin{conj}[Bogomolov]
For any K3 surface $X$ over $\CC$ has infinitely many rational curves. Or if $X / K$ with $K$ a number field then any $K$-point has a rational curve defined over $\ol{\Q}$ passing through it.
\end{conj}

\begin{theorem}[Bogomolov-Hasset-Tschinkel]
If $\rho = 1$ and $g = 2$ then there exist infinitely many rational curves. 
\end{theorem}

\begin{theorem}[Li-Liedtke]
The same is true for any odd $\rho$ and any degree $g$.
\end{theorem}

\subsection{Proof Strategy}

Reduction mod $p$.

\begin{enumerate}
\item Deformation theory reduces to $X/F$ for some number field $F$
\item Find many good primes $p$ to reduce at
\item Compare Picard groups -> show there exists a rational curve not lifting
\item Exhibit a sum of rational curves which does lift 
\end{enumerate}

\subsection{K3 surfaces over finite fields}

\begin{theorem}
If $X$ is a K3 surface over a field $k$ of characteristic $p$ and $k = \bar{k}$ then there exists $T / W(k)$ finite and some $\X \to \Spec{T}$ smooth projective lifting $X$ and generically a K3. Let $S$ be the generic fiber. Then by smooth base change,
\[ H^2_{\et}(X, \mu_{\ell^n}) \cong H^2(S, \Z / \ell^n) \]
\end{theorem}

Pitfall, $X$ in characteristic $p$ can be supersingular. Recall that we have \etale cohomology groups $H^{2i}_{\et}(X, \Z_\ell(i))$ and there is a cycle class map,
\[ \Pic{X} \to H^2(X, \Z_\ell(1)) \]
given by the limit of the connecting maps in the sequence,
\[ 0 \to \mu_{\ell^n} \to \Gm \to \Gm \to 0 \]
The tate conjecture predicts that,
\[ \Pic{X}_{\Q} \to H^2_{\et}(X, \Z_\ell(1)) \]
surjects onto the $\Frob_q$-fixed part. 

\begin{theorem}[Charles]
Over $k$ finite with characteristic $\ge 5$ and $X/k$ a K3 surface then the Tate conjecture holds. 
\end{theorem}

\begin{defn}
A K3 is \textit{supersingular} if $\Frob \acts H^2(X, \Z_\ell(1))$ is trivial. 
\end{defn}

Under the tate conjecture, supersingularity is equvalent to $\rho = 22$. Another pathology: K3 can be unirational but then they are supersingular. 

\begin{prop}
If $X$ is a K3 and the Tate conjecture holds, then $\rho$ is even over $\bar{k}$.
\end{prop}

\begin{proof}
Let $\alpha_1, \dots, \alpha_{22}$ be the eigenvalues of $\Frob \acts H^2_{\et}(X, \Z_{\ell}(1))$. This representation is semisimple. Poincare duality identifies $\alpha_i \mapsto \alpha^{-1}$. Consider classes of eigenvalues,
\begin{enumerate}
\item not roots of unity: even cardinality by Poincare pairing
\item roots of unity: also even since must add up to $22$.
\end{enumerate}
The second class must be algebraic cycles over $\bar{k}$ by Tate. 
\end{proof}

\subsection{Reduction to Number Fields}

Let $S / k$ be a K3 with $k$ a field of characteristic zero. WLOG $k = \Frac{B}$ where $B/F$ smooth variety over $F$ with $F$ a number field. Then $S$ spreads out to $\mathcal{S} \to B$ smooth projective with fibers K3 surfaces. We want to spread out the property of having infinitely many rational curves. 

\subsection{Comparison of Picard Groups}

\begin{theorem}
Let $S / F$ be a number field and $\p$ is some prime. There is a specialization map $\Pic{S_{\bar{F}}} \to \Pic{S_{\bar{\kappa}}}$ which is injective away from characteristic of $\p$.
\end{theorem}

\begin{prop}
If $p \ge 5$ then there exists $\L_p$ in $\Pic{S_{\bar{\p}}}$ not lifting to $S_{\bar{\Q}}$. 
\end{prop}

\begin{theorem}[Bogomolov-Tschinkel]
If $X$ is a K3 over $k = \bar{k}$ then any effective divisor has a representative by an effective sum of rational curves.
\end{theorem}

\begin{cor}
There exists a rational curve $C_p$ not lifting to $S_{\bar{K}}$. 
\end{cor}

Because $H$ is ample, there is some large $N_p$ such that $N_p H - C_p$ is effective so applying the theorem again we get,
\[ C_p + \sum_i n_i R_{p,i} \in | N_p H | \]
where the $R_{p,i}$ are also rational curves. But $H$ lifts by definition.  

\begin{prop}
Assume that $S_p$ is smooth, not supersingular, and assume $C_1 + \dots + C_r$ all distinct rational curves lifts as a divisor class but no subset does. Then there exists a rational curve $C \subset S_{\bar{K}}$ whose divisor class specializes to the divisor class $C_1 + \dots + C_r$. 
\end{prop}

\begin{proof}
Consider the moduli space of stable maps $\ol{\M}_0 \to \Def_{S_p / W(\bar{k})}$. Since $S_p$ is not uniruled, the fibers have dimension $0$. Fact: this map has relative dimension $-1$ and the image is the deformation space compatible with the polarization. Therefore, there is a generic lift to a stable map $T \to S$ and minimality means that $\im{T}$ is irreducible. 
\end{proof}

Varying the prime we can ensure the minimality for different values of $N_p$ 

\end{document}



\documentclass[12pt]{article}
\usepackage{hyperref}
\hypersetup{
    colorlinks=true,
    linkcolor=blue,
    filecolor=magenta,      
    urlcolor=blue,
}

\usepackage{diagbox}
\usepackage{tablefootnote}
\usepackage{import}
\import{../}{AlgGeoCommands}

\DeclareMathOperator{\torsion}{\mathrm{Tor}}

\begin{document}

\section{Introduction}

\begin{defn}
A smooth projective variety is \textit{Fano} if $\omega_X^\vee$ is ample.
\end{defn}

Recall the main rationality properties:

\begin{center}
\begin{tikzcd}
& \text{rational} \arrow[d] \arrow[rrdd]
\\
& \text{stably-rational} \arrow[d]
\\
\text{decomposition of } \Delta \arrow[dd] & \arrow[l] \text{retract-rational}\arrow[d] & & \text{ruled} \arrow[ddd]  
\\
& \text{unirational} \arrow[d] 
\\
\text{universally } \mathrm{CH}_0\text{-trivial}  \arrow[d] & \text{rationally connected} \arrow[d] & \text{Fano} \arrow[l]
\\
\mathrm{CH}_0\text{-trivial} & \arrow[l] \text{rationally chain connected} \arrow[rr] & & \text{uniruled}
\end{tikzcd}
\end{center}

No arrow is expected to be reversible (except rationally connected to chain connected for smooth varities). No methods for obstructing unirationality for rationally connected varities.
\bigskip\\
What we are actually going to show is that general Fano hypersurfaces are not universally $\mathrm{CH}_0$-trivial which in particular shows that they are not rational.

\subsection{Some Examples}

Hypersurfaces $X_d \subset \P^{N+1}$ are Fano when $d \le N+2$. We write the best property known to hold for the very general $X_d$. We put a star where Schrader's results rule out stable rationality. Note that unirationality has not been ruled out in any Fano case.
\bigskip\\

\begin{table}[h]
\begin{center}
\begin{tabular}{|l|c|c|c|c|c|}
\hline
\diagbox{dim}{deg} & 2 & 3 & 4 & 5 & 6 \\
\hline
2 & \text{rational} & \text{rational} & \text{K3} & \text{gen-type} & \text{gen-type} \\
\hline
3 & \text{rational} & \text{unirational}\tablefootnote{known to be irrational by Clemens and Griffiths} & \text{RC}* & \text{CY} & \text{gen-type} \\
\hline
4 & \text{rational} & \text{??}\tablefootnote{general quartic $3$-folds are not even known to be irrational} & \text{RC}* & \text{RC}* & \text{CY} \\
\hline
5 & \text{rational} & \text{??} & \text{??} & \text{RC}* & \text{RC}* \\
\hline
\end{tabular}
\end{center}
\end{table}

The cases marked $??$ are known to be RC and are probably expected to be not rational or unirational but little is known about them. 

\subsection{The Main Theorem I}

Let $k$ be an uncountable algebraically closed field of characteristic not $2$.

\begin{theorem}[Schreieder I]
A very general hypersurface $X \subset \P^{N+1}_k$ of degree $d \ge \log_{2}(N) + 2$ and dimension $N \ge 3$ is not stably rational. 
\end{theorem}

\subsection{Decompositions of the Diagonal}

There are a number of ways to detect / obstruct rationality for curves. One way is to look at Chow groups. We could ask if $\mathrm{CH}_0(C)$ is finitely generated but this is usually very hard to check and doesn't generalize well (as we will see shortly for Fanos). Another thing we might notice is that $X = \P^1$ is the only curve for which for some point $z \in X$,
\[ [\Delta_X] = [X \times z] + [z \times X] \in \CH_1(X \times X) \]
This is the sort of thing we call a decomposition of the diagonal and it generalizes well to higher dimensional varities. The following examples show how we should generalize this.

\begin{example}
Let $X = \P^2$ then we can check that,
\[ [\Delta_X] = [X \times z] + [z \times X] + [\ell \times \ell] \in \CH_2(X \times X) \]
where $[\ell]$ is the class of a line. The last term we need to deal with.
\end{example}

\begin{example}
Let $X = \Bl_{x}(\P^2)$. Then we can check,
\[ [\Delta_X] = [X \times z] + [z \times X] + [\ell \times \ell] - [E \times E] \in \CH_2(X \times X) \]
where $[\ell]$ is the class of a line and $[E]$ is the class of the exceptional. This notion will only be birationally invariant if we kill off these extra terms somehow. This leads to the following definition.
\end{example}

\begin{defn}
A scheme $X$ finite type over a field $k$ of oure dimension $n$ admits a \textit{decomposition of the diagonal} if,
\[ [\Delta_X] = [X \times z] + Z \in \CH_{n}(X \times_k X) \]
where $\Delta_X \subset X \times_k X$ is the diagonal and $Z_X$ is a cycle on $X \times_k X$ which does not dominate any component of the first factor meaning it is supported on $D \times_k X$ for a divisor $D \subsetneq X$.
\end{defn}

\subsection{Torsion-Orders}

It will be more convenient to use the following trick to kill off the additional cycles $K$.

\begin{lemma}
A variety over a field $k$ admits a decomposition of the diagonal if and only if there is a $0$-cycle $z \in Z_0(X)$ on $X$ such that,
\[ [\delta_X] = [z_K] \in \CH_0(X_K) \]
where $K = k(X)$ and,
\[ \delta_X : \Spec{K} \to \Spec{K} \times_k X \to X \times_k X \]
is induced by the diagonal.
\end{lemma}

\begin{proof}
Let $n = \dim{X}$. There is a natural isomorphism,
\[ \dlim_{\substack{U \subset X \\ U \neq \empty \text{ open}}} \CH_n(U \times_k X) \iso \CH_0(X_K) \]
Therefore, $[\delta_X] = [z_K]$ is exactly saying that,
\[ [\Delta_X] = [X \times z] + Z \]
for $Z$ supported on some $D \times_k X$.
\end{proof}

This perspective will alow us to more cleanly reason about decompositions without lugging around the cycle $Z$. But now, something amazing happens.

\begin{lemma}
If $X$ is a rationally chain connected variety over an algebraically closed field $\bar{k}$ then,
\[ \deg : \CH_0(X) \iso \Z \]
is an isomorphism.
\end{lemma} 

\begin{proof}
For any two closed points $x, y \in X$ there exists a chain of rational curves connecting $x, y$. Since $k = \bar{k}$ the intersection points of these rational curves are defined over $k$. Thus $[x] = [y] \in \CH_0(X)$. Furthermore, since $k = \bar{k}$, $X$ admits a point of degree $1$.
\end{proof}

\begin{cor}
Let $X$ be a rationally chain connected variety over a field $k$. Then for any extension of fields $K / k$ the group,
\[ A_0(X_K) = \ker{(\CH_0(X_K) \to \Z)} \]
is torsion.
\end{cor}

\begin{proof}
For any cycle $\alpha \in A_0(X_K)$ then $\alpha_{\bar{K}} = 0$ since $A_0(X_{\bar{K}}) = 0$ and $\CH_0(X) \to \Z$ is natural. Therefore, there exists a finite extension $L / K$ such that $\alpha_{L} = 0$. Since the natural composition,
\[ \CH_0(X_K) \to \CH_0(X_{L}) \to \CH_0(X_K) \]
is $\deg{(L/K)}$ we find that $\deg{(L/K)} \cdot \alpha = 0$.
\end{proof}

\begin{cor}
In particular, taking $K = k(X)$ and $z \in X(k)$ a degree $n$ point, we see that $n \cdot [\delta_X] - [z_K]$ is torsion so there is some $N$ such that $N \cdot [\delta_X] = [z'_K] \in \CH_0(X_K)$.
\end{cor}

\begin{defn}
Let $X$ be a proper variety over a field $k$ and $[\delta_X] \in \CH_0(X_K)$ induced by the diagonal. Then the \textit{torsion order} $\torsion(X) \in \Z_{+} \cup \{ \infty \}$ of $X$ is the smallest positive integer such that as $0$-cycles,
\[ \torsion(X) \cdot [\delta_X] = [z_K] \in \CH_0(X_K) \]
for some $0$-cycle $z \in \CH_0(X)$. 
\end{defn}

\begin{cor}
If $X$ is rationally connected then $\torsion(X) < \infty$.
\end{cor}

\begin{lemma}
A variety $X$ admits a decomposition of the diagonal (over $k$) if and only if $\torsion(X) = 1$. Furthermore, the cycle $e \cdot [\Delta_X]$ admits a decomposition if and only if $\torsion(X) \divides e$.
\end{lemma}


This is very interesting and it is true that $\torsion(X)$ is a birational invariant, but why should we actually care about this number $\torsion(X)$ and not just if it is $1$ or not.

\begin{lemma}
Let $f : X \to Y$ be a proper dominant morphism between proper $k$-varities. Then,
\[ \torsion(Y) \divides \deg{(f)} \cdot \torsion(X) \] 
\end{lemma}

\begin{proof}
Consider the proper morphism,
\[ f \times f : X \times X \to Y \times Y \]
Then $(f \times f)|_{\Delta_X} : \Delta_X \to \Delta_Y$ equals $f$ so it has degree $\deg{f}$. Therefore, using the proper map $f' : X_{k(X)} \to Y_{k(Y)}$ we see that for some $z \in \CH_0(X)$,
\[ 0 = f'_* (\torsion(X) \cdot \delta_X - z_{k(X)}) = \deg{(f)} \cdot \torsion(X) \cdot \delta_Y - \deg{(f)} \cdot (f_* z)_{k(Y)} \in \CH_0(Y_{k(Y)}) \]
and therefore,
\[ \torsion(Y) \divides \deg{(f)} \cdot \torsion(X) \] 
\end{proof}

\begin{cor}
Let $Y$ be a proper $k$-variety of dimension $n$ and $f : \P^n_k \rat Y$ a dominant rational map. Then $\torsion(Y) \divides \deg{f}$.
\end{cor}

\begin{proof}
Resolving the map via the graph $X \subset \P^n_k \times Y$ yields a rational variety $X$ and a dominant morphism $f : X \to Y$. Thus by the previous lemmas $\torsion(X) = 1$ and so $\torsion(Y) \divides \deg{f}$.
\end{proof}

\begin{rmk}
This is one of the only know ways to obstruct maps $\P^n_k \rat Y$ of a given degree. 
\end{rmk}

\subsection{Unramified Cohomology}

But how do we actually obstruct a decomposition of the diagonal? One way is to use pairing on \etale cohomology classes:
\[ \CH_{\dim{X}}(X \times X) \times H_{\et}^i(X, \mu) \to H_{\et}^i(X, \mu) \]
defined by,
\[ \inner{\Gamma}{\alpha} = \pi_{1*}(\gamma(\Gamma) \smile \pi_2^* \alpha) \]

(HOW PAIRING WORKS)


\section{The Main Theorem II}

\begin{theorem}[Schreieder II]
A very general Fano hypersurface $X_d \subset \P^{N+1}_k$ has torsion order $\torsion(X) \sim d!$. 
\end{theorem}

\begin{rmk}
When we prove the theorem we will understand the exact numerics and see that II $\implies$ I.
\end{rmk}

\begin{proof}
\begin{enumerate}
\item find a cleverly chosen degeneration $X \spto X_0$ such that $X_0$ admits an unramified cohomology class $\alpha$ of order $m$

\item if $N \cdot [\Delta] = 0$ in $\CH_0(X_K)$ this specializes to $\torsion(X_0) \divides N$ and therefore $m \divides N$
\end{enumerate}
\end{proof}

\section{Unramified Cohomology I}

Give primer on etale cohomology of function fields. Define unramified cohomology and prove basic properties (PAGES). How do we build unramified classes.

\section{Unramified Cohomology II}

Prove the pairing between Chow groups and unramified cohomology (PAGES). 

\section{Torsion-Orders of Fanos}

Prove basic properties of torsion-orders (PAGES)
\begin{enumerate}
\item prove that retract-rational varities have decompositions of the diagonal (Schreieder: Slopes, Lemma 2.4) (Schreieder: Unramified Cohomology, Lemma 7.5).

\item torsion of the diagonal = universal exponent of torsion in $\CH_0(X)$ (Voisin, Prop 2.9).

\item 
\end{enumerate}
Prove that for $X_d \subset \P^{N+1}_k$ we have the upper bound $\torsion(X) \divides d!$. The lower bound will occupy the rest of our time.

\section{Constructing Unramified Classes I}

(PAGES)

\section{Constructing Unramified Classes II}

(PAGES)

\section{Degeneration Methods}

(PAGES)

\end{document}

\documentclass[12pt]{article}
\usepackage{hyperref}
\hypersetup{
    colorlinks=true,
    linkcolor=blue,
    filecolor=magenta,      
    urlcolor=blue,
}

\usepackage{import}
\import{../}{AlgGeoCommands}

\begin{document}

\section{Ax-Grothendieck}

\begin{rmk}
The classical result is that if $f : \C^n \to \C^n$ is injective then it is surjective. Grothendieck generalized this as follows.
\end{rmk}

\begin{theorem}
Let $S$ a scheme and $X$ a finite type $S$-scheme and $f : X \to X$ is an $S$-endomorphism. Then,
\begin{enumerate}
\item if $f$ is radicial then it is surjective
\item if $X$ is finitely presented and $f$ is monic then it is an isomorphism.
\end{enumerate}
\end{theorem}

\begin{defn}
$f : X \to Y$ is \textit{radicial} if for every field $K$ the map $X(K) \to Y(K)$ is injective.
\end{defn}

\begin{rmk}
In analogy with above, we can rephrase the property of being monic as $X(R) \to Y(R)$ is injective for all rings $R$.
\end{rmk}

\begin{example}
Frobenius $F_X : X \to X$ is radicial but not monic. The normalization of a cusp is radicial but not monic. $\Spec{k[\epsilon]} \to \Spec{k}$ is radicial but not monic.
\end{example}

\begin{rmk}
universal homeomorphism iff radicial, surjective, integral. 
\end{rmk}

\begin{rmk}
radicial iff universally injective iff $f^{-1}(y)$ is a single topological point with purely inseparable residue field extension.
\end{rmk}

\begin{proof}[Proof of (a)]
To show surjectivity we only need to show this on fibers so we may assume that $S = \Spec{A}$ for $A$ noetherian (not that helpful to assume it is a field) then spread out so that $S$ is finite type over $\Z$. Then $f : X \to X$ is finite type so by Chevalley's theorem $f(X) \subset X$ is constructible but $X$ is Jacobson, so to check that $f(X) = X$ is suffices to check that $f(X)$ contains all closed points. Now the residue field of every closed point is finite and $X$ is finite type so $\# X(\FF_q) < \infty$ because $X$ is quasi-compact and finite-type so covered by finitely many affine opens $U = \Spec{A}$ with $A$ a finite type $\Z$-algebra so $\Hom{}{A}{\FF_q}$ is finite. Then $X(\FF_q) \to X(\FF_q)$ is injective and hence surjective proving the claim. 
\end{proof}

\begin{thm}
\etale radicial maps are open embeddings.
\end{thm}

\begin{proof}
Suppose that $f : X \to Y$ is a \etale and radicial. Then $f(X)$ is open, pass from $Y$ to $f(X)$ to assume that $f$ is surjective. We need to show $f$ is an isomorphism. Now $f$ is radicial, surjective, and open. Therefore $f$ is a universal homeomorphism and thus proper ($\Delta$ is surjective by radicial and an immersion so it is a closed immersion plus universal homeomorphism implies universally closed). Therefore $f$ is quasi-finite and porper so $f$ is finite and also flat and finitely presented and hence finite locally free. Then we reduce to $X = \Spec{B}$ and $Y = \Spec{A}$ and $B$ is finite over $A$ and we need to show the rank is $1$. But the fiber $\Spec{B / \m B} \to \Spec{A / \m}$ is \etale and radicial and therefore an isomorphism so the degree is $1$ and thus we are done. 
\end{proof}

\begin{defn}
A morphism $f : X \to Y$ is fpqc if $f$ is faithfully flat and quasi-compact. 
\end{defn}

\begin{rmk}
The problem with this is that it does not refine Zariski opens because not all open sets are retocompact. We can say the fpqc topology is generated by fpqc morphisms and (Zariski) open embeddings.
\end{rmk}

\begin{rmk}
Alternatively, we can replace quasi-compactness in the definition with: for $V \subset Y$ open and quasi-compact then there exists $U \subset X$ open and quasi-compact such that $f(U) = V$. 
\end{rmk}

\begin{defn}
A property $\cP$ of $f : X \to Y$ \textit{descends} if,
\[ f : X \to Y \text{ has } \cP \iff f_{Y'} : X \times_Y Y' \to Y' \text{ has } \cP \text{ for all fpqc covers } Y' \to Y \]
\end{defn}

\begin{prop}
The following properties descend: affine, closed embedding, finite.
\end{prop}

\begin{proof}
Can assume that $Y = \Spec{A}$ and $Y' = \Spec{A'}$ because if $Y'$ is quasi-compact there is a finite cover of affines and the finite disjoint union is still finite. Let $\F$ be quasi-coherent on $X$. We need to show that $H^1(X, \F) = 0$. However, by flat base change,
\[ H^1(X_{A'}, \F_{A'}) = H^1(X, \F) \ot_A A' = 0 \]
so by faithfullness we see that $H^1(X, \F) = 0$.
\bigskip\\
Suppose that $f' : X' \to Y'$ is a closed embedding then $f$ is affine. Given $A \to B$ such that $A' \to B \ot_A A'$ is surjective then $A \to B$ is surjective (take the cokernel and use faithful flatness) so $f$ is a closed embedding.
\bigskip\\
Suppose that $f'$ is finite. Then we can assume that $f$ is affine and $A' \to B \ot_A A'$ is finite. Then writing,
\[ x_i = \sum_j b_{ij} \ot a'_{ij} \]
gives $A^{\{ b_{ij} \}} \onto B$ so $B$ is finite.
\end{proof}

\begin{thm}
Two facts we need,
\begin{enumerate}
\item proper + quasi-finite $\implies$ finite
\item proper + monic $\implies$ closed embedding.
\end{enumerate}
\end{thm}

\begin{proof}
Spreading out, we can assume $Y$ is noetherian local $Y = \Spec{A}$. By noetherian, $A \to \hat{A}$ is faithfully flat and thus by descent can assume $A = \hat{A}$. 
\end{proof}

\begin{lemma}
If $f : X \to \Spec{\hat{A}}$ is quasi-finite then $X = X' \sqcup X''$ where $X'$ is finite and $X''$ has empty special fiber. 
\end{lemma}


\begin{proof}
There are finitely many points in the special fiber. By the lemma $\stalk{X}{x_i}$ are $\hat{A}$-finite then $\Spec{\stalk{X}{x_i}} \to X$ is a closed embedding because it is finite over $\hat{A}$. However it is closed under generalization so it is also an open embedding. 
\end{proof}

\begin{example}
Given a two dimensional local ring $R$ consider $\Spec{R} \setminus \{ \m \} \to \Spec{R}$. 
\end{example}

\begin{lemma}
If $M$ is an $\hat{A}$-module then $M$ is $\hat{A}$-finite if and only if $M / \m M$ is $\hat{A}$-finite and $M$ is $\m$-adically separated meaning,
\[ \bigcap_{n \ge 1} \m^n M = 0 \]
\end{lemma}

\begin{proof}
Choose $m_1, \dots, m_n \in M$ which residually generate. Let $m \in M$. Then $\bar{m} = a_1 m_1 + \cdots + a_n m_n$ in $M / \m M$. Then consider,
\[ m_1 = m - (a_1 m_1 + \cdots + a_n m_n) \in \m M \]
Then we can write,
\[ m_1 = x_1 m_{11} + \cdots + x_k m_{1k} \]
for $x_i \in \m$. We can keep doing this and the coefficients coverge in $\hat{A}$. Therefore $M$ is actually generated by $m_1, \dots, m_n$.
\end{proof}

\begin{proof}[quasi-finite + proper]
By the first lemma we see that if $f$ is quasi-finite and proper over $\Spec{\hat{A}}$ then it is actually finite (properness kills the second component). Now we do proper and monic.
If $\hat{A} \to B$ is finite and monic then $B / \m B \ot_{\hat{A} / \m} B / \m B \to B / \m B$ is an isomorphism and therefore $A / \m \onto B / \m B$ is surjective so Nakayama gives $A \onto B$ is surjective. 
\end{proof}

\begin{proof}[Proof of (b)]
We know $f$ is surjective. Spreading out alows us to assume $S$ is finite type over $\Z$. It suffices to prove that $f$ is \etale because \etale radicial maps are open embeddings.
\bigskip\\
Because $f$ is monic, for $x \in X$, we have $\kappa(f(x)) \iso \kappa(x)$. Therefore, $\widehat{\stalk{X}{f(x)}} \to \widehat{\stalk{X}{x}}$ is an isomorphism on residue fields. It suffices to prove this is an isomorphism because this proves that $f$ is \etale (in fact we only need that $f$ is flat or equivalently that $\widehat{\stalk{X}{f(x)}} \to \widehat{\stalk{X}{x}}$ is flat). 
\bigskip\\
Becaue the \etale locus is open and $X$ is Jacobson (spread out over a f.t. $\Z$-algebra) so it suffices to show this condition at closed points $x \in X$. We want to show that,
\[ \stalk{X}{f(x)} / \m_{f(x)}^{n+1} \iso \stalk{X}{x} / \m_x^{n+1} \]
for all $n$. Consider,
\[ T_{p,d} = \{ x \in X \mid \# \kappa(x) \divides p^d \} \]
since every closed point has a finite residue field. Let $\I$ be the ideal sheaf,
\[ \I = \prod_{x \in T_{p,d}} \m_x^{n+1} \]
Then $Z = V(\I)$ gives,
\begin{center}
\begin{tikzcd}
Z \arrow[r, "g"] \arrow[d] & Z \arrow[d] 
\\
X \arrow[r, "f"] & X
\end{tikzcd}
\end{center}
because $f$ induces maps on the $n+1$-fold quotient. Then $g$ is monic because $f$ is monic. Now $Z$ is affine and the spectrium of a finite ring and thus $f$ is finite but also monic so $g$ is a closed immersion but $Z = \Spec{B}$ so $g^\# : B \onto B$ is surjective and hence an isomorphism and therefore $g^\#$ is bijective. Therefore, $g$ is an isomorphism completing the proof. 
\end{proof}

\end{document}


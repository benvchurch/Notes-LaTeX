\documentclass[12pt]{article}
\usepackage{hyperref}
\hypersetup{
    colorlinks=true,
    linkcolor=blue,
    filecolor=magenta,      
    urlcolor=blue,
}

\usepackage{import}
\import{../}{AlgGeoCommands}

\newcommand{\LL}{\mathbb{L}}

\begin{document}

\section{Sep. 30}

\subsection{Introduction}

\begin{thm}[Deligne-Illusie]
Let $X / k$ be a smooth proper scheme with $k$ a field of characteristic zero and $\Omega^\bullet_{X/k}$ is deRham complex. Then, the Hodge-to-deRham spectral sequence,
\[ E_1^{p,q} = H^{q}(X, \Omega^p_{X/k}) \implies H_{\dR}^{p+q}(X) \]
degenerates at the $E_1$-page.
\end{thm}

\begin{cor}
Then,
\[ \dim{H^n_{\dR}(X)} = \sum_{p + q = n} \dim{H^q(X, \Omega^p_{X/k})} \]
\end{cor}

\begin{rmk}
For $k = \CC$, we can prove the above equality using analytic techniques (i.e. Hodge theory). 
\end{rmk}

\begin{rmk}
D-I give an purely algebraic proof. The idea is use degeneration in positive characteristic to get degeneration in characteristic zero. 
\end{rmk}

\subsection{de Rham Complex}

Let $f : X \to Y$ be a morphism of schemes. 

\begin{defn}
Then $\Omega^1_{X/Y}$ is the sheaf of relative differentials on $X/Y$. Then,
\[ \Omega^1_{X/Y} = \Delta^* \C_{X \times_Y X/X} \]
is the conormal bundle for the diagonal $\Delta_{X/Y} : X \to X \times_Y X$. Then,
\[ \Omega^i_{X/Y} = \bigwedge^i \Omega^1_{X/Y} \]
and let $\Omega^0_{X/Y} = \struct{X}$. Furthermore, there exists a unique family of maps $\d^i : \Omega^i_{X/Y} \to \Omega^{i+1}_{X/Y}$ such that,
\begin{enumerate}
\item $\d^i$ is a $Y$-antiderivation of the total complex,
\[ \Omega_{X/Y} = \bigoplus_{i = 0}^\infty \Omega^i_{X/Y} \]
meaning that $\d$ is $f^{-1} \struct{Y}$-linear and on local sections it satisfies the graded Leibniz law,
\[ \d{(a \wedge b)} = \d{a} \wedge b + (-1)^i a \wedge \d{b} \]
\item $\d^2 = 0$
\item $\d{a} = \d_{X/Y}{a}$ for $\deg{a} = 0$.
\end{enumerate}
Then $(\Omega^\bullet_{X/Y}, \d)$ is the deRham complex of $X/Y$,
\begin{center}
\begin{tikzcd}
0 \arrow[r] & \struct{X} \arrow[r, "\d"] & \Omega^1_{X/Y} \arrow[r] & \Omega^2_{X/Y} \arrow[r] & \Omega^3_{X/Y} \arrow[r] & \cdots
\end{tikzcd}
\end{center}
\end{defn}

\begin{rmk}
Working over $k = \CC$, there is also an analytic deRham complex $(\Omega^\bullet_{X/Y})^\an$. Then GAGA tells you that you get the same cohomology in the algebraic and analytic cases. Furthermore, the analytic deRham complex is a (not acyclic!!) resolution of the constant sheaf $\CC$.
\end{rmk}

\begin{defn}
$H^n_{\dR}(X) = \mathbb{H}^n(X, \Omega^\bullet_{X/Y})$
\end{defn}

\begin{rmk}
$\mathbb{H}^n(X, \Omega_{X/Y}^\bullet) = R^n \Gamma(\Omega_{X/Y}^\bullet)$.
\end{rmk}

\begin{rmk}
There exists a hypercohomology spectral sequence,
\[ E_1^{p,q} = R^q \Gamma(X, C^p) \implies \mathbb{H}^{p+q}(C^\bullet) \]
Applying this to the deRham complex gives the Hodge-to-deRham spectral sequence,
\[ H^q(X, \Omega^p_{X/Y}) \implies H^{p+q}_{\dR}(X) \]
\end{rmk}

\subsection{Frobenius and Cartier Isomorphisms}

\newcommand{\Fr}{\mathrm{Fr}}

\begin{defn}
Let $X$ be a scheme of characteristic $p$ (meaning $p \struct{X} = 0$). Then there is a natural map $\Fr : X \to X$ via $\id$ on topological spaces and $\struct{X} \to \struct{X}$ via $x \mapsto x^p$. This is natural, in the sense that for any map $f : X \to Y$ there is a commutative diagram,
\begin{center}
\begin{tikzcd}
X \arrow[r, "\Fr_X"] \arrow[d, "f"] & X \arrow[d, "f"]
\\
Y \arrow[r, "\Fr_Y"] & Y
\end{tikzcd}
\end{center}
Therefore, we can define via pullbacks,
\begin{center}
\begin{tikzcd}
X \arrow[rd, "f"'] \arrow[r, "F_{X/Y}", dashed] & X^{(p)} \arrow[d] \arrow[r] & X \arrow[d, "f"]
\\
& Y \arrow[r, "\Fr_Y"] & Y
\end{tikzcd}
\end{center}
giving the relative Frobenius $F_{X/Y} : X \to X^{(p)}$. 
\end{defn}

\begin{prop}
If $Y$ has characteristic $p$ and $f : X \to Y$ is smooth of relative dimension $n$ then $F_{X/Y} : X \to X^{(p)}$ is finite and flat of degree $n$. Therefore, $F_* \struct{X}$ is locally free of rank $n$ as a $\struct{X^{(p)}}$-module.
\end{prop}

\begin{proof}
When $f$ is \etale then $F_{X/Y}$ is actually an isomorphism. Indeed, $F_{X/Y}$ composed with $X^{(p)} \to Y$ is \etale and $X^{(p)} \to Y$ is \etale by base change so $F_{X/Y}$ is \etale but it is also radicial since $\Fr_X$ is. Thus $F_{X/Y}$ is a surjective open immerison. In general, this is a local question so we reduce to a standard smooth which factors as the composition of an \etale map and a projection from affine space which can be done directly.
\end{proof}

\begin{prop}
Let $\d = \d_{X/Y}$. Let $s$ be a local section of $\struct{X}$. Then,
\[ \d{(s^p)} = p s^{p-1} \d{s} = 0 \]
since $\d{(s^p)} = F_{X/Y}^*(\d{s}) = F^*_{X/Y}(1 \otimes \d{s})$. Thus,
\begin{enumerate}
\item $\Fr^* \Omega^i_{X/Y} \to \Omega_{X/Y}^i$ is zero
\item $F^*_{X/Y} \Omega^i_{X^{(p)}/Y} \to \Omega^i_{X/Y}$ is zero
\item $\d$ on the complex $(F_{X/Y})_* \Omega^\bullet_{X/Y}$ is $\struct{X^p}$-linear.
\end{enumerate}
\end{prop}

\renewcommand{\H}{\mathcal{H}}

\begin{theorem}[Cartier]
There exists a unique morphism of graded $\struct{X^{(p)}}$-algebras,
\[ \gamma : \bigoplus_{i} \Omega^i_{X^{(p)}/Y} \to \bigoplus_i \H^i((F_{X/Y})_* \Omega^\bullet_{X/Y}) \]
such that
\begin{enumerate}
\item for $i = 0$, we have $\gamma$ is the map $\struct{X^{(p)}} \to (F_{X/Y})_* \struct{X}$
\item for $i = 1$, we have $\gamma(1 \otimes \d{s}) = s^{p-1} \d{s}$ in $\H^i(F_{X/Y*} \Omega^\bullet_{X/Y})$
\end{enumerate}
Furthermore, if $f$ is smooth then $\gamma$ is an isomorphism and we call $c = \gamma^{-1}$.
\end{theorem}

\begin{rmk}
If $Y = \Spec{k}$ and $X$ is smooth then $\gamma$ is called the absolute Cartier isomorphism. 
\end{rmk}

\begin{rmk}
The theorem tells us that $\gamma$ is determined by how it acts in degree $0$ and degree $1$ because it is a morphism of graded algebras and the deRham complex is generated in degrees $0$ and $1$. Explicitly,
\[ \gamma(\tau \wedge \sigma) = \gamma(\tau) \wedge \gamma(\sigma) \]
\end{rmk}

\subsection{Relationship to the HdDSS}

Now let $Y = \Spec{k}$ with $k$ a perfect field.
D-I realized that the Cartier isomorphism is related to degeneration of the HdDSS,
\[ E_1^{p,q} = H^q(X, \Omega_{X/k}^p) \implies H^{p+q}_{\dR}(X/k) \]
Consider the complex,
\[ C = \bigoplus_i \Omega^i_{X^{(p)}/Y} [-i] \]
Then $\H^i(C)$ is the graded parts of the domain of the Cartier isomorphism. Furthermore, the codomain is $\H^i(F_* \Omega^\bullet_{X/k})$. Then we might ask if there is a map of complexes,
\[ \phi : C \to F_* \Omega^\bullet_{X/k} \]
which induces the Cartier map. 

\begin{prop}
If there is such a quasi-isomorphism $\phi$, then the HdRSS degenerates at $E_1$.
\end{prop}

\begin{proof}
This follows from the chain of isomorphisms,
\[ \mathbb{H}^n(X, \Omega^\bullet_X) \cong \mathbb{H}^n(X^{(p)}, F_* \Omega^\bullet_X) \cong \bigoplus_{i} H^{n-i}(X^{(p)}, \Omega^i_{X^{(p)}}) \cong \bigoplus_{i} H^{n-i}(X, \Omega^i_X) \]
The first isomorphism comes from the fact that $F$ is finite and thus affine. The second isomorphism is the inverse of the map induced by $\phi$ on cohomology. Finally,
\[ H^{j}(X^{p}, \Omega^i_{X^{(p)}}) = H^{j}(X, \Omega^i_X) \]
becuase $F : X \to X^{(p)}$ is an isomorphism of schemes (not of $k$-schemes). 
Therefore the dimensions match whcih implies that the spectral sequence must have degenerated since the dimensions of the terms matches those of the filtered pieces already.
\end{proof}

\section{Oct. 14}

\subsection{Degeneration in Characteristic $p$}

First we state the main theorem for today.

\begin{thm}
Let $S \to \Z / p \Z$ be a scheme of characteristic $p$ and a flat lift to $\Z / p^2 \Z$,
\begin{center}
\begin{tikzcd}
S \arrow[r, hook] \arrow[d] &  \wt{S} \arrow[d]
\\
\Spec{\Z/p\Z} \arrow[r, hook] & \Spec{\Z/p^2\Z}
\end{tikzcd}
\end{center}
If $X/S$ is smooth and proper and $X^{(p)}$ admits a smooth lift over $\wt{S}$ then,
\[ \tau^{< p} (F_{X/S})_* \Omega^\bullet_{X/S} \]
is decomposable in $D(X^{(p)})$ meaning it is isomorphic to a complex whose differentials are all zero (i.e. it is isomorphic to its cohomology).
\end{thm}

\begin{rmk}
The de Rham complex is not an element of the derived category of $\struct{X}$-modules because the transition maps are not $\struct{X}$-linear. However, the useful fact about $(F_{X/S})_* \Omega_{X/S}^\bullet$ is that the transition maps are $\struct{X^{(p)}}$-linear because for any $f \in \struct{X^{(p)}}(U)$ and $\omega \in \Omega_{X/S}(F_{X/S}^{-1}(U))$ we have,
\[ \d{(f \cdot \omega)} = \d{(F_{X/S}^\#(f) \omega)} = \d{(F_{X/S}^\#(f))} \wedge \omega + F_{X/S}^\#(f) \d{\omega} = f \cdot \d{\omega} \]
because $\d{(F_{X/S}^\#(f))} = 0$ since this is $\d$ relative to $S$ and $F_{X/S}$ acts via $x \mapsto x^p$ ``relative to $S$''.
\end{rmk}

\begin{cor}
If $k$ is a perfect field and $X/k$ is smooth, proper, and $\dim{X} < p$ and $X$ lifts over $W_2(k)$ then the Hodge-to-de Rham spectral sequence degenerates at $E_1$. 
\end{cor}

\begin{proof}
We apply this to the case $S = \Spec{k}$ and $\wt{S} = \Spec{W_2(k)}$. By above, we have that $(F_{X/S})_* \Omega^\bullet_{X/S}$ is decomposable and the hyperderived spectral sequence of any decomposable complex degenerates at $E_1$ just because the differentials of the spectral sequence are formed from the transition maps on the complex which are zero up to quasi-isomorphism. Therefore,
\[ \dim \mathbb{H}^n(X, \Omega_{X/k}^\bullet) = \dim \mathbb{H}^n(X^{(p)}, (F_{X/k})_* \Omega_{X/k}^\bullet) = \sum_{p + q = n} h^q(X^{(p)}, (F_{X/k})_* \Omega^p_{X/k}) = \sum_{p + q = n} h^q(X, \Omega^p_{X/k}) \]
because the Frobenius is affine and therefore the dimensions add up for the Hodge-to-de Rham spectral sequence already at the $E_1$ page proving that the differentials must already be zero.
\end{proof}

\subsection{Recall the Cartier Isomorphism}

Let $X/S$ be a smooth scheme with $S$ characteristic $p$. Then there is a graded isomorphism,
\[ C^{-1} : \bigoplus_{i} \Omega^i_{X^{(p)}/S} \iso \bigoplus_i \mathcal{H}^i((F_{X/S})_* \Omega^\bullet_{X/S}) \]
such that,
\begin{enumerate}
\item in $i = 0$ the map $\struct{X^{(p)}} \to (F_{X/S})_* \struct{X}$ is $F_{X/S}^{\#}$
\item in $i = 1$,
\[ C^{-1}(1 \otimes \d{s}) = s^{p-1} \d{s} \in \mathcal{H}^1((F_{X/S})_* \Omega^\bullet_{X/S}) \]
think of this as like ``$\frac{F^*(\d{s})}{p}$''.  
\end{enumerate}

To prove the main theorem for today, we will exhibit a quasi-isomorphism
\[ \varphi : \bigoplus_{i < p} \Omega^i_{X^{(p)}/S}[-i] \to \tau^{< p} (F_{X/S})_* \Omega^\bullet_{X/S} \]
that induces $C^{-1}$ on cohomology for $i < p$ (and thus is a quasi-isomorphism).

\begin{rmk}
Note that when $S$ is perfect (meaning $\Fr_S$ is an isomorphism) we also get an ``absolute'' version of the theorem since $(\Fr_S)_{X*} \Omega_{X^{(p)}/S}^i = \Omega_{X/S}^i$ because $(\Fr_S)_{X} : X^{(p)} \to X$ is also an isomorphism. Therefore, pushing forward $\varphi$ gives a quasi-isomorphism
\[ (\Fr_S)_{X*} \varphi : \bigoplus_{i < p} \Omega^i_{X^{(p)}/S} \iso (\Fr_X)_* \Omega_{X/S}^{\bullet} \]
\end{rmk}

We want to reduce to constructing $\varphi^1$ where $\varphi^i$ are the components of the map from the direct sum. For $\varphi^0$ we just define,
\[ \varphi^0 : \struct{X^{(p)}} \xrightarrow{C^{-1}} \mathcal{H}^0((F_{X/S})_* \Omega^\bullet_{X/S}) \embed (F_{X/S})_* \Omega^\bullet_{X/S} \]
Now assume we have constructed,
\[ \varphi^1 : \Omega^1_{X^{(p)}/S} [-1] \to (F_{X/S})_* \Omega^\bullet_{X/S} \]
inducing $C^{-1}$ on $\mathcal{H}^1$.
Then there exists,
\[ \left( \Omega^1_{X^{(p)}/S} \right)^{\otimes i} \to \Omega^i_{X^{(p)}/S} \]
by sending,
\[ w_1 \otimes \cdots \otimes w_i \mapsto w_1 \wedge \cdots \wedge w_i \]
If $i < p$ (or in characteristic zero) then there exists a section to this map,
\[ a(w_1 \wedge \cdots \wedge w_i)  = \frac{1}{i!} \sum_{\sigma \in S_i} \mathrm{sign}(i) w_{\sigma(1)} \otimes \cdots \otimes w_{\sigma(i)} \]
Therefore we get,
\begin{center}
\begin{tikzcd}
(\Omega^1_{X^{(p)}/S})^{\otimes i} \arrow[r, "\varphi_1^{\otimes i}"] & \left( (F_{X/S})_* \Omega_{X/S}^\bullet \right)^{\otimes^{\LL} i} \arrow[d]
\\
\Omega^i_{X^{(p)}/S} \arrow[u] \arrow[r, dashed, "\varphi^i"] & (F_{X/S})_* \Omega^\bullet_{X/S}
\end{tikzcd}
\end{center}
Because this construction agrees with the product structure and the Cartier isomorphism is determined (using the product structure) by its values in degree $1$ this means that $\varphi^i$ must induce $C^{-1}$ in degree $i$.

\subsection{Construction of $\varphi^1$}

First we consider the case when $F_{X/S}$ admits a global lift. Given,
\begin{center}
\begin{tikzcd}
S \arrow[d] \arrow[r] & \tilde{S} \arrow[d]
\\
\Spec{\Z/p\Z} \arrow[r, hook] & \Spec{\Z/p^2\Z}
\end{tikzcd}
\end{center}
and $X/S$ is smooth and proper. We want there to be a digram,
\begin{center}
\begin{tikzcd}
X \arrow[d, "F_{X/S}"'] \pullback \arrow[r] & \wt{X} \arrow[d, "\wt{F_{X/S}}"] 
\\
X^{(p)} \arrow[r] & \wt{X}^{(p)}
\end{tikzcd}
\end{center}
where $\wt{X} \to \wt{S}$ and $\wt{X}^{(p)} \to S$ are smooth (flat implies this) lifts of $X \to S$ and $X^{(p)} \to S$. Note that we assumed the existence of the smooth lift $\wt{X} \to \tilde{S}$ in the hypothesis of the thorem but we did not assume the existence of a lift of $F_{X/S}$. However, a lift of $F_{X/S}$ exists locally so we will assume a lift exists and then use uniqueness to patch together the results obtained for each local lift. 

\begin{rmk}
We will only apply this for $S = \Spec{k}$ with $k$ a perfect field and $\wt{S} = \Spec{W_2(k)}$. Note that $W_2(k)$ is the \textit{unique} flat lift of $k$ along
\[ \Spec{\Z/p\Z} \embed \Spec{\Z/p^2 \Z} \]
This is what people mean when they say $W(k)$ lifting is an ``unramified'' lift, it is unramified over $\Z_p$. Indeed, another characteristic zero lift say over a ring $R$ will have $\Spec{R} \to \Spec{\Z_p}$ ramified (the fiber over $(p)$ is a nonreduced structure on $\Spec{k}$) so it does not induce a flat deformation of $\Spec{k}$ over $\Z/p^2 \Z$ only of the nonreduced scheme.
\end{rmk}

\begin{rmk}
Note that if $S = \Spec{k}$ for $k$ a perfect field, if $X$ lifts to $W_2(k)$ then so does $X^{(p)}$. Indeed, absolute frobenius of $k$ lifts to $W_2(k)$ so we can pullback a lift of $X$ along this. Also $\Fr_k$ is an automorphism so it is directly clear that $X$ lifts if and only if $X^{(p)}$ lifts.
\end{rmk}

\begin{rmk}
Because of flatness, multiplication by $p$ induces an isomorphism $p : \struct{S} \iso p \struct{\wt{S}}$. 
Inded, from the exact sequence,
\begin{center}
\begin{tikzcd}
0 \arrow[r] & \Z/p\Z \arrow[r, "p"] & \Z/p^2\Z \arrow[r] & \Z/p \Z \arrow[r] & 0
\end{tikzcd}
\end{center}
we see that $\Z/p\Z \iso p \Z/p^2\Z$ meaning that this is an extension  by the module $\Z/p\Z$. Then by the flatness of $\wt{S} \to \Spec{\Z/p^2\Z}$ the exact sequence,
\begin{center}
\begin{tikzcd}
0 \arrow[r] & \struct{S} \arrow[r, "p"] & \struct{\wt{S}} \arrow[r] & \struct{S} \arrow[r] & 0
\end{tikzcd}
\end{center}
so the extension is by the ideal $p \cdot \struct{\wt{S}}$ which is isomorphic to $\struct{S}$. The exact same argument for $X \embed \wt{X}$ which is also a flat lift over $\Spec{\Z/p\Z} \to \Spec{\Z/p^2\Z}$ shows that $\wt{X}$ is an extension of $X$ by $\struct{X} \iso p \struct{\wt{X}}$. Therefore, by local freeness, we get a similar isomorphism,
\[ p : \Omega^1_{X/S} \iso p \cdot \Omega^1_{\tilde{X}/\tilde{S}} \]
\end{rmk}

Now to perform the construction notice that,
\[ \im{(\wt{F_{X/S}}^* : \Omega^1_{\wt{X^{(p)}}/\wt{S}} \to (\wt{F_{X/S}})_* \Omega^1_{\wt{X}/\wt{S}})} \subset p \cdot (\wt{F_{X/S}})_* \Omega^1_{\wt{X}/\wt{S}} \]
because pulling back differentials by Frobenius introduces a factor of $p$. Therefore, we get a diagram,
\begin{center}
\begin{tikzcd}
\Omega^1_{\wt{X^{(p)}}/\wt{S}} \arrow[d, two heads] \arrow[r, "\wt{F_{X/S}}"] & p \cdot (\wt{F_{X/S}})_* \Omega^1_{\wt{X}/\wt{S}} 
\\
\Omega^1_{X^{(p)}/S} \arrow[r, dashed, "\varphi^1"] & (F_{X/S})_* \Omega^1_{X/S} \arrow[u, "p \cdot (-)"]
\end{tikzcd}
\end{center}
which exists because the right upward map is an isomorphism and the kernel of the left downward map is the multiples of $p$ which are sent to zero. 
I claim that
\[ \im{\varphi^1} \subset Z^1((F_{X/S})_* \Omega^\bullet_{X/S}) \]
and $\varphi^1$ induces $C^{-1}$ in degree $1$. For local section $a'\in \Gamma(U^{(p)}, \struct{\wt{X^{(p)}}})$ pulled back from $a \in \Gamma(U, \struct{X})$, the differential $\d{a}$ is acted on via
\[ \wt{F_{X/S}}^*(\d{a'}) = \d{\,\wt{F_{X/S}}^\# a'} = p a^{p-1} \d{a} + p \, \d{b} \]
where $\wt{F_{X/S}}^\# a' = a^p + p \, b$ where $p \, b$ is the error term. Hence
\[ \varphi^1(\d{a'}) = a^{p-1} \d{a} + \d{b} \]
which is clearly an exact form (lies in $Z^1$). But notice that the second term is exact and therefore dies in the quotient
\[ Z^1((F_{X/S})_* \Omega^\bullet_{X/S}) \to \cH^1((F_{X/S})_* \Omega^\bullet_{X/S}) \]
so the induced map is exactly given by the Cartier isomorphism in degree $1$.

\subsection{What about if $F$ doesn't lift?}

From smoothness, we know that lifts exist locally. We need to compare the outputs of different lifts. 

\begin{lemma}
Given flat lifts $\wt{X}_i$ of $X$ and $G_i : \wt{X} \to \wt{X^{(p)}}$ of $F_{X/S}$ over $\wt{S}$ there is a canonical element,
\[ h(G_1, G_2) : \Omega^1_{X^{(p)}/S} \to (F_{X/S})_* \struct{X} \]
such that,
\[ \varphi^1_{G_2} - \varphi^1_{G_1} = \d{h(G_1, G_2)} \] 
and if $G_3 : \wt{X}_3 \to \wt{X^{(p)}}$ is a third lifting then
\[ h(G_1, G_2) + h(G_2, G_3) = h(G_1, G_3) \]
\end{lemma}



\begin{proof}
Choose an isomorphism $u : \wt{X}_1 \iso \wt{X}_2$ of lifts (which may only exist locally) then 
\[ u^* G_2 - G_1 : \struct{X^{(p)}} \to (F_{X/S})_* \struct{X} \]
is a derivation which does not depend on the choice of isomorphism $u$. Indeed, given $u' : \wt{X}_1 \iso \wt{X}_2$ the difference is a derivation or equivalently a map
\[ \delta : \Omega_{X/S}^1 \to \struct{X} \]
Then $u^* G_2$ and $u'^* G_2$ differ by the composition of $\delta$ with the pullback $F^*_{X/S} \Omega^1_{X^{(p)}/S} \to \Omega^1_{X/S}$ which is zero. Hence $u^* G_2 = u'^* G_2$. Therefore, working locally on $X$ so that an isomorphism $u$ exists, we get a well-defined derivation 
\[ h(G_1, G_2) : \Omega^1_{X^{(p)}/S} \to (\wt{F_{X/S}})_* \struct{X} \]
via the difference above. Then 
\[ \varphi_{G_2}^1 - \varphi_{G_1}^1 = \d{h(G_1, G_2)} \]
from the formula for $\varphi^1$ since $G_2^{\#}(a') - G_1^{\#}(a') = b_2 - b_1$ in $(F_{X/S})_* \struct{X} = p \cdot (\wt{F_{X/S}})_* \struct{\wt{X}}$ then
\[ \varphi^1_{G_2}(a') - \varphi^1_{G_1}(a') = \d{(b_2 - b_1)} \]
\end{proof}

\subsection{Proof of the Theorem}

Now fix the lifting $\wt{X^{(p)}}$ of $X^{(p)}$ over $\wt{S}$. Choose an open covering $\U = (U_i)_{i \in I}$ of $X$ so that for each $i$ there is a lifting $\wt{U}_i$ of $U_i$ over $\wt{S}$ and a lifting $G_i : \wt{U}_i \to \wt{X^{(p)}}$ of $F|_{U_i}$. We have built, for each $i$ a map of complexes
\[ f_i = \varphi^1_{G_i} : \Omega^1_{X^{(p)}/S}[-1] \to F_* \Omega_{X/S}^\bullet|_{U_i} \]
and for each pair $(i,j)$, a homomorphism
\[ h_{ij} = h(G_i|_{U_{ij}}, G_j|_{U_{ij}}) : \Omega^1_{X^{(p)}/S}|_{U_{ij}} \to F_* \Omega^\bullet_{X/S}|_{U_{ij}} \]
where $U_{ij} = U_i \cap U_j$. These datum are related via
\[ f_j - f_i = \d{h_{ij}} \text{ over } U_{ij} \]
\[ h_{ij} + h_{jk} = h_{ik} \text{ over } U_{ijk} = U_i \cap U_j \cap U_k \]
These make it possible to define a homomorphism of complexes of $\struct{X^{(p)}}$-modules
\[ \varphi^1_{\wt{X^{(p)}}, (U_i, G_i)} : \Omega^1_{X^{(p)}/S}[-1] \to \check{C}(\U, F_* \Omega^\bullet_{X/S}) \]
where the target is the total complex associated to the $\check{C}$ech bicomplex of the cover $\U$ with values in the complex $F_* \Omega^\bullet_{X/S}$. Explicitly, this the complex
\[ \check{C}(\U, F_* \Omega_{X/S}^\bullet)^n = \bigoplus_{i+j = n} \check{C}^j(\U, F_* \Omega_{X/S}^i) \]
with differential $\d = \d_1 + \d_2$ where $\d_1$ is the de Rham differential and $\d_2$ is, in bidegree $(i,j)$, equal to $(-1)^i \sum (-1)^i \partial^i$ for the $\check{C}$ech differential. In particular,
\[ \check{C}(\U, F_* \Omega^\bullet_{X/S})^1 = \check{C}(\U, F_* \struct{X}) \oplus \check{C}^0(\U, F_* \Omega_{X/S}^1) \]
The morphism $\varphi^1_{\wt{X^{(p)}}, (U_i, G_i)}$ is defined as having for components $(\varphi_1, \varphi_2)$ in degree $1$, with
\[ (\varphi_1 \omega)(i,j) = h_{ij}(\omega)|_{U_{ij}} \quad (\varphi_2 \omega)(i) = f_i(\omega)|_{U_i} \]
Using the fact that the $f_i$ are morphisms of complexes, together with the above forumlas relating the $f_i$ and $h_{ij}$, it follows that $\varphi^1_{X^{(p)}, (U_i, G_i)}$ is thus a well-defined morphism of complexes. We also has at our disposal the natural augmentation
\[ \epsilon : F_* \Omega_{X/S}^\bullet \to \check{C}(\U, F_* \Omega_{X/S}^\bullet) \]
which is a quasi-isomorphism. Because for each $i$, the complex $\check{C}(\U, F_* \Omega_{X/S}^i)$ is a resolution of $F_* \Omega_{X/S}^i$. We then define $\varphi^1_Z$ by inverting $\epsilon$. Comparing two coverings we can show that $\varphi^1$ does not depend on the choices. 

\section{Passage to Characteristic Zero}

\begin{rmk}
Today again all schemes are noetherian.
\end{rmk}

\begin{prop}[Nullstellensatz]
If $K$ is a finite type $k$-algebra and $K$ is a field then $K/k$ is finite.
\end{prop}

\begin{proof}
Suppose not. Then there is an injection $k[t] \embed K$ because $K$ cannot be algebraic. Then $\Spec{K} \to \A^1_k$ so by Chevalley the image is constructible. But the image the generic point which is not constructible giving a contradiction. 
\end{proof}

\begin{cor}
Every nonempty constructible subset of a finite type $k$-scheme has a closed point.
\end{cor}

\begin{proof}
Let $C \subset X$ be locally closed and affine let $C = \Spec{A}$. Then $A / \m$ is a field finite type over $k$ so it is finite. Then consider $\overline{\{ \m \}} \subset X$ is closed. However, the generic point of $\overline{ \{ \m \} }$ has transcendence degree zero.  
\end{proof}

\begin{defn}
$X$ is \textit{Jacobson} if every nonempty constructible subset has a closed (in $X$) point.
\end{defn}

\begin{rmk}
This is equivalent to every closed set is the closure of its closed points.
\end{rmk}

\begin{example}
Some (non) examples of Jacobson schemes,
\begin{enumerate}
\item finte type $k$-schemes are Jacobson
\item $\Spec{\Z}$ is Jacobson
\item if $R$ is a local ring of $\dim{R} \ge 1$ then not Jacobson
\item $X = \Spec{R} \setminus \{ \m_R \}$ is Jacobson.
\end{enumerate}
\end{example}

\begin{prop}
Let $S$ be Jacobson and $f : X \to S$ is finite type. 
\begin{enumerate}
\item If $x \in X$ is a closed point then $f(x)$ is closed.
\item $X$ is Jacobson.
\end{enumerate}
\end{prop}

\begin{proof}
For (a) let $x \in X$ be a closed point then Chevalley's theorem implies that $\{ f(x) \}$ is constructible so $\{ f(x) \}$ is closed because $S$ is Jacobson. For (b) let $C \subset X$ be constructible. Then Chevalley's theorem implies that $f(C) \subset S$ is constructible so there is a closed point $s \in f(C)$. Then $X_s \to \kappa(s)$ is finite type so $X_s$ is Jacobson. Then $X_s \cap C \subset X_s$ is nonempty constructible so it has a closed point $x \in C \cap X_s$ and $X_s$ is closed (because $s \in S$ is closed) so $x$ is a closed point.
\end{proof}

\begin{cor}
Finite type $\Z$-schemes are Jacobson and have finite residue fields at closed points.
\end{cor}

\begin{proof}
The first part is immediate. Then if $x \in X$ is a closed point then it lies over some $p \in \Spec{\Z}$ nonzero (because $x$ is closed) so $x \in X_p$ and $X_p$ is finite type over $\kappa(p) = \FF_p$. Then it follows from the Nullstellensatz.
\end{proof}

\begin{prop}
If $X \to \Spec{\Z}$ is finite tpye and $X$ is reduced then there is a dense open such that $U \to \Spec{\Z}$ is smooth.
\end{prop}

\begin{proof}
This follows from two facts:
\begin{enumerate}
\item if $k$ is a perfect field and $X$ is a finite type reduced $k$-scheme then it is generically smooth.
\item if $f : X \to S$ is finite type then the smooth locus is open. 
\end{enumerate}
We can assume that $X$ is integral then $K(x) / k$ is finitely generated. Since $k$ is perfect there is a separating transcendence basis $t_1, \dots, t_n \in K(X)$ such that $K(X) / k(t_1, \dots, t_n)$ is finite separable. Then $K(X) = k(t_1, \dots, t_n)[T]/(G(T))$ by the primitive element theorem. By localizing on $X$ we get an open affine $U \subset X$ with $U \embed \A^{n+1}_k$ defined by $G$. Then $U \setminus V(G)$ is smooth and $V(G)$ does not contain the generic point so this is a dense open.
\bigskip\\
To see the second part, locally embedd $X \embed \A^N_S$ by $f_1, \dots, f_m$ then smoothness is characterized by the nonvanihsing og some minors of the jacobian of $f_1, \dots, f_m$ which is a closed condition. 
\end{proof}

\begin{thm}
If $\pi : X \to S$ is proper, $\F$ is coherent over $X$ then $R^i \pi_* \F$ is also coherent.
\end{thm}

\begin{proof}
The proof is long but,
\begin{enumerate}
\item first deal with the projective case by showing $H^i(\P^n_A, \struct{}(m))$ is finite over $A$ for all $i, m, n$.
\item if $\F$ is coherent on $\P^n_A$ then there exists a surjection $\struct{}(-N)^M \onto \F$ then we use descending induction to show that $H^i(\P^n_A, \F)$ is finite for all $i$.
\item $X$ is projective then use $\iota : X \embed \P^n_A$ and exactness of affine pushforward to reduce to the case of projective space.
\item In general, Chow's lemma gives $f : \tilde{X} \to X$ over $S$ such that $\tilde{X}$ is projective over $S$ and $f$ is projective and surjective. Use Leray spectral sequence argument [EGA III, 3.1-2].
\end{enumerate}
\end{proof}

\begin{rmk}
The same coherence statement also holds if $\F$ is a bounded complex of coherent sheaves. This follows from the spectral sequence,
\[ E_1^{i,j} = R^j f_* K^i \implies R^{i+j} f_* K^\bullet \]
which is just the first spectral sequence for hypercohomology.
\end{rmk}

\begin{thm}[flat base change]
Consider a Cartesian diagram,
\begin{center}
\begin{tikzcd}
X' \arrow[r, "g'"] \arrow[d, "f'"'] & X \arrow[d, "f"]
\\
S' \arrow[r, "g"] & S
\end{tikzcd}
\end{center}
where $g$ is flat and $f$ is finite type and separated.
Let $\F$ be quasi-coherent on $X$ then the natural base change map,
\[ g^* R f_* \F \to R f'_* g'^* \F \]
is an isomorphism. By adjunction this is the same as a map,
\[ R f_* \F \to R g_* Rf'_* g'^* \F = R f_* R g'_* g'^* \F \]
which we have by applying $R f_*$ to $\F \to R g'_* g'^* \F$.
\end{thm}

\begin{theorem}[Cohomology and Base Change]
Let $f : X \to S$ be proper and $\F$ is coherent on $X$ and flat over $S$. Suppose that $R^i f_* \F$ is finite locally free for all $i$. Then given any diagram,
\[ g^* R^i f_* \F \to R^i f'_* g'^* \F \]
is an isomorphism for all $n$ for all maps $g$.
\end{theorem}

\begin{rmk}
The same holds if $\F$ is replaced with a bounded complex of coherent sheaves with flat cohomology sheaves over $S$ such that $R^i f_* K^\bullet$ is finite locally free for all $n$.
\end{rmk}

\begin{thm}
If $f : X \to S$ is finite type, the function,
\[ x \mapsto \dim_x X_{f(x)} \]
is upper semi-continuous. If $f $ is closed then the function,
\[ s \mapsto \dim X_s \]
is also semi-continuous. 
\end{thm}

\begin{proof}
The second follows from the first because,
\[ f(\{ x \in X \mid \dim_x X_{f(x)} \ge n \}) \]
is closed. 
\end{proof}

\subsection{Completing the Proof}

\begin{rmk}
Previously, we proved the following.
\end{rmk}

\begin{thm}
Let $k$ be perfect of characteristic $p > 0$ and $X$ is smooth and proper over $k$ and $\dim{X} < p$ and $X$ admits a lift to $W_2(k)$ then,
\[ E^{p,q}_1 = H^q(X, \Omega^p_X) \implies H_{\dR}^{p+q}(X) \]
degenerates at $E_1$.
\end{thm}

\begin{rmk}
We now use this to deduce the main theorem.
\end{rmk}

\begin{thm}
Let $K$ be a field of char zero and $X$ is smooth and proper over $K$. Then,
\[ E_1^{p,q} = H^q(X, \Omega^p_X) \implies H^{p+q}_{\dR}(X) \]
degenerates at $E_1$.
\end{thm}

\renewcommand{\X}{\mathfrak{X}}

\begin{proof}
Spread out $X$ to some smooth and proper $\X \to \Spec{A}$ for $A \subset K$ finite type over $\Z$. This is because $K = \varinjlim A$ for finite type $\Z$-subalgebras of $K$ then we spread out to schemes over each $A$ and smooth and proper spreads out. Thus we get a Cartesian diagram,
\begin{center}
\begin{tikzcd}
X \pullback \arrow[r] \arrow[d] & \X \arrow[d]
\\
\Spec{K} \arrow[r] & \Spec{A}
\end{tikzcd}
\end{center}
Now by base change we can assume that $K = \bar{K}$ and $X$ is connected of dimension $d$. By upper-semi continuity we can assume that all fibers of $\X \to S = \Spec{A}$ are of dimension $d$ by shrinking $A$. Furthermore, we can shirnk $A$ such that $\Spec{A} \to \Spec{\Z}$ is smooth. This is because $A_\Q$ is reduced and thus $\Spec{A_\Q} \to \Spec{\Q}$ is smooth on a dense open and therefore the smooth locus of $\Spec{A} \to \Spec{\Z}$ contains the generic point and thus is a nonempty open so we can shrink to that open.
\bigskip\\
Now $R^n f_* \Omega^i_{\X / S}$ and $R^n f_* \Omega^\bullet_{\X/S}$ are coherent. Therefore, by shrinking $S$ we can assume that all of them are finite locally free (this works because there are finitely many since it vanishes when $i > d$ and $n > d$) because they are generically free. Let $h^{i,j} = \dim H^j(X, \Omega^i_{X/K})$ and $h^n = \dim_K H^n_{\dR}(X)$. It suffices to show that,
\[ h^n = \sum_{i + j = n} h^{i,j} \]
Because all pushforwards in sight are finite locally free and therefore these pushforwards commute with arbitrary base change. In particular if $s \in S$ is any point then,
\[ h^{i,j} = \dim_{\kappa(s)} H^j(\X_s, \Omega^i_{\X_s / \kappa(s)}) \quad \text{ and } \quad h^n = \dim_{\kappa(s)} H^n_{\dR}(\X_s) \]
We want to find an $s$ such that $\X_s \to \Spec{\kappa(s)}$ satisfies our previous conditions for degeneration of Hodge-to-deRham. Thus we want,
\begin{enumerate}
\item $\dim{\X_s} < \mathrm{char}(\kappa(s))$
\item $\X_s$ lifts to $W_2(\kappa(s))$
\end{enumerate}
If we can do this then,
\[ E^{i,j}_1 = H^j(\X_s, \Omega^i_{\X_s/\kappa(s)}) \implies H^{i+j}_{\dR}(\X_s) \]
degenerates at $E_1$ and therefore,
\[ h^n = \dim_{\kappa(s)} H^n_{\dR}(\X_s) = \sum_{i+j = n} \dim_{\kappa(s)} H^j(\X_s, \Omega^i_{\X_s / \kappa(s)}) = \sum_{i + j = n} h^{i,j} \]
which is what we wanted to show.
\bigskip\\
Set,
\[ N = \prod_{\substack{p \le d \\ p \text{ prime}}} p \]
Replace $A$ by $A[1/N]$ so no residue field of $A$ can have characteristic $\le d$. Then $A$ is finite over $\Z$ so it has a closed point $s \in \Spec{A}$ and thus $\mathrm{char}(\kappa(s)) > d$ and $d = \dim{\X_s}$. Choose this point $s \in \Spec{A}$.
\bigskip\\
Now, we have a diagram,
\begin{center}
\begin{tikzcd}
\Spec{\kappa(s)} \arrow[d, "\text{nilpotent thickening}"'] \arrow[r] & S \arrow[d, "\text{smooth}"]
\\
\Spec{W_2(\kappa(s))} \arrow[d] \arrow[ru, dashed] \arrow[r] & \Spec{\Z}
\\
\Spec{\Z_p} \arrow[ru] 
\end{tikzcd}
\end{center}
there exists a lift because $S \to \Spec{\Z}$ is smooth. Therefore, by pulling back along this lift gives a lift of $\X_s$,
\begin{center}
\begin{tikzcd}
\wt{\X_s} \arrow[d] \arrow[r] & \X_s \arrow[d]
\\
\Spec{W_2(\kappa(s))} \arrow[r] & S
\end{tikzcd}
\end{center}
therefore $\X_s$ lifts over $W_2(\kappa(s))$ so we are done.
\end{proof}

\end{document}

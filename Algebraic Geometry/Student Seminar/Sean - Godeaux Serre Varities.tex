\documentclass[12pt]{article}
\usepackage{hyperref}
\hypersetup{
    colorlinks=true,
    linkcolor=blue,
    filecolor=magenta,      
    urlcolor=blue,
}

\usepackage{import}
\import{../}{AlgGeoCommands}

\begin{document}

\newcommand{\fPic}{\mathrm{Pic}}

\section{Godeaux-Serre Varities}

The main theorem:

\begin{thm}
$k$ field and $G$ is a finite commutative $k$-group scheme then there exists a smooth projective geometrically connected $3$-dimensional $k$-scheme $X$ such that,
\[ \mathrm{Pic}^{\tau}_{X/k} \cong G^\vee \]
\end{thm}

\subsection{Other examples}

\begin{enumerate}
\item For any finite abstract group $G$ there exists $X$ with $\pi_1^{\et}(X_{\bar{k}}) \cong G$ 
\item failure of Hodge symmetry in characteristic $p$
\item failure of lifting of surfaces in char $p$
\item if done in families then jumping of Hodge and de Rham numbers in mixed or equal char $p$
\end{enumerate}

\subsection{Finite Commutative Group Schemes}

If $H$ is a finite abstract group then then there is a finite constant $k$-group $H$. 
\bigskip\\
If $H$ is a finte abstract group with a $G_k = \Gal{k^\sep/k}$-action $\alpha$ then we get a finite \etale group scheme $H_\alpha$ representing the functor,
\[ H_\alpha(k') = H^{\Gal{k^\sep / k'}} \]
for $k'$ finite seperable over $k$. 

\begin{prop}
This is an equivalence of categories between finite \etale $k$-groups and finite groups with a Galois action.
\end{prop}

\begin{rmk}
In characteristic $0$, all finite group schemes are \etale and thus all arise from the above constructions. 
\end{rmk}

\begin{example}
$\mu_n$ over $\Q$ corresponds to $\mu_n(\overline{\Q})$ as a $\Gal{\overline{\Q}/\Q}$-module.
\end{example}

\begin{example}
$\mu_p$ and $\alpha_p$ in characteristic $p$ are (nontrivial) connected finite group schemes
\end{example}

\begin{rmk}
A sequence of $k$-groups,
\begin{center}
\begin{tikzcd}
1 \arrow[r] & G' \arrow[r] & G \arrow[r, "\pi"] & G''  \arrow[r] & 1
\end{tikzcd}
\end{center}
is exact if it is an exact sequence of fppf sheaves. This is equivalent to $\pi : G \to G''$ a faithfully flat surjection and $G'$ is its kernel. 
\end{rmk}

\begin{thm}[Connected-\etale sequence]
There is a canonical short exact sequence,
\begin{center}
\begin{tikzcd}
1 \arrow[r] & G^0 \arrow[r] & G \arrow[r, "\pi"] & G / G^0 \arrow[r] & 1
\end{tikzcd}
\end{center}
where $G^0$ is connected and $G / G^0$ is \etale. Furthermore, if $k$ is perfect then this sequence splits.
\end{thm}

\begin{rmk}
What is $G / G^0$? At first it is just the fppf quotient sheaf. However, we want this to be representable.
\end{rmk}

\begin{thm}[SGA3, Exp V, Thm. 4.1]
Let $X$ be a quasi-projective scheme over $k$ and $G$ is a finite $k$-group scheme acting on $X$. Then the ringed space quoteint $X / G$ is a quasi-projective scheme and the map $\pi : X \to X / G$ is finite. If $G$ acts freely then $\pi$ is a $G$-torsor (meaning $\pi : G \to X / G$ if fppf and $G \times X \to X \times_{X / G} X$ is an isomorphism). 
\end{thm}

\begin{rmk}
In general, if $G$ acts freely, then ``all good properties'' of $X$ descend to $X/G$ e.g.
\begin{enumerate}
\item smoothness
\item normality
\item cohen-macaullayness
\end{enumerate}
by faithfully flat descent. 
\end{rmk}

\begin{rmk}
Even if $G \acts X$ is not free $X/G$ exists (in the generality of the theorem) and $X/G$ is the coarse space of the stack $[X/G]$. 
\end{rmk}

\subsection{Cartier Duality}

For a finite commutative $k$-group scheme $G$ we can construct the dual,
\[ G^\vee = \shHom{}{G}{\Gm} \]
where this is the sheaf,
\[ G^\vee(S) = \Hom{\text{gp}}{G_S}{\Gm} \]
It turns out that $G^\vee$ is representable by $\Spec{k[G]^*}$ (see Mumford's book on abelian varities). 

\begin{thm}
The functor $G \mapsto G^\vee$ is an anti-equivalence sending short exact sequences to short exact sequences and $G \to (G^\vee)^\vee$ is an isomorphism.
\end{thm}

\begin{example}
$(\Z / n \Z)^\vee = \shHom{}{\Z / n \Z}{\mu_n} = \mu_n$. Furthermore, $\alpha_p^\vee = \alpha_p$. By double duality $\mu_n^\vee = \Z / n \Z$.
\end{example}

\begin{rmk}
Therefore we can classify finite group schemes into four types: \etale - \etale, \etale - infinitessimal, infinitessimal - \etale, infinitessimal - infinitessimal. Examples of these three types are $\Z / n \Z$ for $p \ndivides n$ and $\Z / p \Z$ and $\mu_p$ and $\alpha_p$.
\end{rmk}

\begin{rmk}
Cartier duality works for finite commutative group schemes over any base.
\end{rmk}

\subsection{Picard Schemes}

For $X$ a $k$-scheme, define the functor,
\[ \mathrm{Pic}^{\text{pre}}_{X/k}(S) = \Pic{X \times_k S} \]
and define $\mathrm{Pic}_{X/k}$ to be the fppf sheafification. 

\begin{thm}
If $X$ is projective and geometrically intergral then $\mathrm{Pic}_{X/k}$ is represented by a locally finite type $k$-scheme and $\mathrm{Pic}_{X/k}(\bar{k}) / \mathrm{Pic}^0_{X/k}(\bar{k})$ is a finitely generated abelian group. 
\end{thm}

\begin{rmk}
We denote $\mathrm{Pic}^0_{X/k}$ the identity component. And $\mathrm{Pic}^{\tau}_{X/k}$ the union of components which are torsion in $\mathrm{Pic}_{X/k} / \mathrm{Pic}^0_{X/k}$ which is finite type.
\end{rmk}

\begin{example}
If $X$ is a curve then $\mathrm{Pic}_{X/k}$ is smooth. To see this we can verify the infinitessimal criterion. If $A$ is Artin local and $I \subset A$ is square zero then there is an exact sequence of sheaves,
\begin{center}
\begin{tikzcd}
1 \arrow[r] & \struct{X} \ot I \arrow[r] & \struct{X_A}^\times \arrow[r] & \struct{X_{A/I}}^\times \arrow[r] & 0 
\end{tikzcd}
\end{center}
and from the long exact sequence of cohomology,
\begin{center}
\begin{tikzcd}
H^1(X_A, \struct{X_A}^\times) \arrow[r] & H^1(X_{A/I}, \struct{X_{A/I}}^\times) \arrow[r] & H^2(X, \struct{X} \ot I) 
\end{tikzcd}
\end{center}
but $\dim{X} = 1$ so $H^2(X, \struct{X} \ot I) = 0$ and therefore the map $\Pic{X_A} \to \Pic{X_{A/I}}$ is surjective.
\end{example}

\begin{example}
If $X$ is smooth, then $\mathrm{Pic}^{\tau}_{X/k}$ is proper. Verify this using the valuative criterion.
\end{example}

\begin{example}
If $X$ is a smooth curve of genus $g$ then $\mathrm{Pic}^0_{X/k}$ is a smooth proper commutative group scheme of dimension $g$. We have,
\[ T_0 \mathrm{Pic}_{X/k} \cong H^1(X, \struct{X}) \]
Use the previous construction with $A = k[\epsilon]$ and,
\[ T_0 \mathrm{Pic}_{X/k} = \ker{(\mathrm{Pic}_{X/k}(k[\epsilon]) \to \mathrm{Pic}_{X/k}(k))} \]
\end{example}

\begin{rmk}
If $X(k) \neq \empty$ then $\mathrm{Pic}_{X/k}(S) = \mathrm{Pic}(X_S) / \Pic{S}$. 
\end{rmk}

\begin{rmk}
When $X$ is not proper, why is $\mathrm{Pic}_{X/k}$ not nec. representable. The functorial criterion for loc. fin. pres: if $Y$ is a scheme then $Y$ is lpf iff for any direct limit $A = \varinjlim A_i$ we have $Y(A) = \varinjlim Y(A_i)$.
\bigskip\\
Using spreading out, $\mathrm{Pic}_{X/k}$ always satisfies this but $T_0 \mathrm{Pic}_{X/k} = H^1(X, \struct{X})$ is infinite dimensional so it can't be the tangent space of a finite type scheme.
\end{rmk}

\begin{example}
For $X$ a nodal curve over $k$ and $Y$ a cuspidal curve over $k$ we have $\mathrm{Pic}^0_{X/k} = \Gm$ and $\mathrm{Pic}^0_{Y/k} = \Ga$. ``Pinching two points adds a $\Gm$'' and ``collapsing a tangent vector introduces a $\Ga$''. 
\end{example}

\begin{lemma}
If $X \subset \P^n$ is a complete intersection of dimension $\ge 3$, then $\fPic_{X/k} \cong \Z$. 
\end{lemma}

\begin{proof}
Ihe claim that $\Pic{X} \cong \Z$ is a \textit{Leftschetz-Theorem} (SGA2, Exp. XII, Cor. 3.7). The only other point is to show that it is \etale which follows from $H^1(X, \struct{X}) = 0$. 
\end{proof}

\begin{prop}
For $X$ a complete intersection $\dim{X} = d \ge 1$ and $N \ge 0$ then,
\[ H^i(X, \struct{X}(-N)) = 0 \]
for $1 \le i \le d - 1$ and,
\[ H^0(X, \struct{X}(-N)) = 
\begin{cases}
k & N = 0 
\\
0
\end{cases} \]
\end{prop}

\begin{proof}
Induction.
\end{proof}

\begin{lemma}
If $\pi : Y \to X$ is a $G$-torsor, then $\ker{(\fPic_{X/k} \to \fPic_{Y/k})} \cong G^\vee$.
\end{lemma}

\begin{proof}
By fppf descent for line bundles: a line bundle on $X$ is the same as a $G$-linearized line bundle on $Y$ and the map forgets the linearization. 
\end{proof}

\subsection{The Main Construction}

We want to construct $X$ smooth projective $\dim{X} = 3$ such that $\fPic^\tau_{X/k} \cong G$. In light of the previous lemmas, it would be enough to find a complete intersection $X \subset \P^n$ with a free action of $G$ such that $X / G$ is smooth because then because this is a finite group scheme,
\[ \fPic^\tau_{X/G} = \ker{(\fPic^\tau_{X/G} \to \fPic^\tau_{X})} \cong G^\vee \]
because $\fPic^\tau_X = 0$. 

\begin{enumerate}
\item Find a projective space $P = \P^n$ with an action of $G$ which is free away from codim $\ge 3$. 

\item Let $Z = P / G$ be a projective $k$-schemem and a finite map $\pi : P \to P / G$. If $U \subset P$ is the free locus for the action of $G$, then $U / G \embed Z$ is smooth and open.

\item Bertini's theorem (Poonen's if $k$ is finite) shows that after slicing by finitely many hypersurfaces $H_1, \dots, H_m$ so that if $Y = Z \cap H_1 \cap \cdots \cap H_m$ then $Y$ is smooth and geometrically integral of dimension $3$. Then we get a Cartesian diagram,
\begin{center}
\begin{tikzcd}
X \pullback \arrow[d] \arrow[r] & P \arrow[d]
\\
Y \arrow[r] & Z
\end{tikzcd}
\end{center}
Given an ample line bundle $\L$ on $Z$ then $\pi^* \L$ is ample on $P$ (because $\pi$ is finite) so $\pi^{-1}(H_i)$ is also a hypersurface in $P$ so $X$ is a complete intersection and $X \subset U$ so $X \to Y$ is a $G$-torsor. 
\end{enumerate}

To do (1) let $G$ act on $\P((k[G]^{\ot n})^*)$ for some integer $n$ ($n = 3$ will work). Then the free locus of this action is complementary codimension $\ge 3$ if $n \ge 3$. We will prove this when $G$ has no nontrivial proper subgroups (if $k = \bar{k}$ this is equivalent to $G = \mu_\ell$ or $\Z / p \Z$ or $\alpha_p$) but it's more difficult in general. If $K$ is a field over $k$ then any point $x \in P(K)$ can be lifted to $\varphi : G^n \to \A^1$ which is nonzero. This is because,
\[ \P(K) = (K[G^n] \setminus \{ 0 \}) / K^* \]
and an element of $K[G^n]$ defines a map $G^n \to \A^1$ which has to be nonzero. The assumption shows that if $G_x \neq 0$ then $G_x = G$, so the $G$-invariance means,
\[ \varphi(g + g_1, g + g_2, g + g_3) = \eta(g) \varphi(g_1, g_2, g_3) \]
for some $\eta : G \to \Gm$ because we lifted. Therefore we can define,
\[ \psi(g_1, g_2, g_3) = \eta(-g_1) \varphi(g_1, g_2, g_3) \]
This is $G$-invariant if and only if $\psi$ factors through $G^3 \to G^2$ but now $k[G^2]$ is codim $(\dim{G})^3 - (\dim{G})^2 \ge 4$ so the condition that $G_x \neq 0$ gives large enough codimension.

\begin{rmk}
There is an issue with extending this proof that is for a finite commutative group scheme there might be $\infty$-many distinct subgroups. E.g. $\alpha_p^2$ contas $\ker{F_L}$ for any line $L$ passing though $0 \in \A^2$.
\end{rmk}

\end{document}


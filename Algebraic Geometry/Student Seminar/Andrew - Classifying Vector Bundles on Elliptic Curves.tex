\documentclass[12pt]{article}
\usepackage{hyperref}
\hypersetup{
    colorlinks=true,
    linkcolor=blue,
    filecolor=magenta,      
    urlcolor=blue,
}

\usepackage{import}
\import{../}{AlgGeoCommands}

\begin{document}

\begin{thm}
all vector bundles on $\P^1$ split into line bundles.
\end{thm}

\begin{defn}
$E(r,d)$ is the set of isomorphism classes of rank $r$ vector bundles of degree $d$.
\end{defn}

\begin{defn}
For $\F$ a coherent sheaf on $X$,
\[ \chi(X, \F) = \sum_{i = 0}^\infty (-1)^i \dim{H^i(X, \F)} \]
\end{defn}

\begin{defn}
For $X$ proper $E$ coherent then,
\[ \deg{E} = \chi(X, E) - \rank{(E)} \cdot \chi(X, \struct{X}) \]
If $E$ is a vector bundle then,
\[ \deg{E} = \deg{\det{E}} \]
\end{defn}

\begin{rmk}
If $X$ is an elliptic curve $g = 1$ and,
\[ \chi(X, \struct{X}) = 0 \]
therefore,
\[ \deg{E} = \chi(X, E) \]
\end{rmk}

\begin{prop}
Degree is additive on short exact sequences and,
\[ \chi(X, \F \ot \G) = \rank{(\F)} \cdot \chi(X, \G) + \rank{(\G)} \cdot \chi(X, \F) \]
\end{prop}

\begin{thm}[Devissage]
If $P$ is a property of coherent sheaves on a locally noetherian scheme $X$ such that,
\begin{enumerate}
\item If there is a short exact sequence of coherent sheaves,
\begin{center}
\begin{tikzcd}
0 \arrow[r] & \F_1 \arrow[r] & \F_2 \arrow[r] & \F_3 \arrow[r] & 0
\end{tikzcd}
\end{center}
if $\F_1$ and $\F_2$ have $P$ then $\F_3$ has $P$.
\item for all $Z \subset X$ closed integral with generic point $\xi \in Z$ there is some coherent sheaf $\G$ on $X$ such that $\Supp{}{\G} = Z$ and $\dim_{\kappa(\xi)} \G_\xi = 1$ and $\G$ has $P$
\end{enumerate}
Then $P$ holds for all coherent sheaves on $X$.
\end{thm}

\begin{rmk}
We can use this to prove the previous tensor formula for curves since the closed irreducible sets are just a point. 
\end{rmk}


First operation in ``Euclidean Algorithm'',
\[ E(r,d) \to E(r, d + rk) \]
for any $k \in \Z$ fix a line bundle $\L$ of degree $1$ and send,
\[ \E \mapsto \E \ot \L^k \]
so by the formula,
\[ \deg{(\E \ot \L^k)} = d + r k \]
Therefore, we can always reduce to $0 \le d < r$ via choosing $k$. Now we want to reduce $r$,
\[ E(r+d,d) \iso E(r,d) \]
Idea: given $\E \in E(r, d)$ find a trivial subbundle of rank $d$ with indecomposable quotient. 
\bigskip\\
Main tool,
\[ \{ \text{extensions of } \E' \text{ by } \struct{X}^s \} \cong \Ext{1}{\struct{X}}{\E'}{\struct{X}^s} = H^1(X, \E'^\vee \ot \struct{X}^s) = \Hom{}{\Gamma(\struct{X}^s)^\vee}{H^1(X, \E'^\vee)} \]
Fact: $\E$ is indecomposable iff the class $\delta$ in Hom is injective. For now, assume $\deg{\E} = d > 0$ then $s = \dim{\Gamma(\E)} > 0$ because,
\[ \deg{\E} = \chi(\E) = h^0(\E) - h^1(\E) \]
since $\chi(X, \struct{X}) = 0$. 

\begin{lemma}
If $r > d$ then any maximal degree sublinebundle in $\E$ must have degree $0$. 
\end{lemma}

\begin{proof}
Find a maximal degree line bundle $\L \subset \E$ then we proceed to $\E / \L$ to get a filtration,
\[ 0 = \E_0 \subset \E_1 \subset \cdots \subset \E_r = \E \]
where $\E_{i+1}/\E_i \cong \L_i$ with $\L_0 = \L$. Then we can show that,
\[ \deg{\L_{i+1}} \ge \deg{\L_i} \]
and that,
\[ \deg{\E} = \sum_{i = 0}^r \deg{\L_i} \]
Therefore, if $\deg{\L_0} > 0$ then the above sum is at least $r$ proving that $\deg{\E} \ge r$ contradicting the assumption.
\end{proof}

By Lemma, any $\varphi \in \Gamma(\E)$ generates a line bundle of degree $0$ (it cannot have negative degree because its generated by the section $\varphi$).

\begin{example}
If $\L$ is a line bundle on an elliptic curve, $\deg{\L} = 0$ and $h^0(\L) > 0$ then $\L$ is trivial. (DONT NEED ELLIPTIC CURVE HERE)
\end{example}

If all $\varphi$ generate trivial line bundles then $\Gamma(\E) \to \E_x$ is injective. Therefore $\Gamma(\E)$ generate $\struct{X}^s \subset \E$.
\bigskip\\
Now we want to show that $d = s$. By indcution on $r$: if $r = 1$, $d > 0$, then $h^0(\E) = d$ by Riemann-Roch. Assuming $\E' = \E / \struct{X}^s$ is indecomposable, then $\dim{\Gamma(\E')} = d$ by induction. Assuming that $\E, \E'$ are indecomposable, claim if $\E' = \E / \struct{X}^s$ then,
\[ \dim{\Gamma(\E')} = \dim{\Gamma(\E)} \]
then from,
\begin{center}
\begin{tikzcd}
0 \arrow[r] & \E'^\vee \arrow[r] & \E^\vee \arrow[r] & \struct{X}^s \arrow[r] & 0
\end{tikzcd}
\end{center}
we see that $\deg{\E} = \deg{\E'}$ and also this gives a long exact sequence,
\begin{center}
\begin{tikzcd}
0 \arrow[r] & \Gamma(X, \E'^\vee) \arrow[r] & \Gamma(X, \E^\vee) \arrow[r, "\delta"] & H^1(X, \E'^\vee) 
\end{tikzcd}
\end{center}
given that $\E$ is indecomposble iff $\delta $ is injective this implies that $\Gamma(\E'^\vee) = \Gamma(\E^\vee)$. Then by Serre duality (recalling that $\omega_X = \struct{X}$),
\[ h^1(X, \E') = h^1(X, \E) \]
and also $\deg{\E} = \deg{\E'}$ so $h^0(X, \E) = h^0(X, \E')$. 
\bigskip\\
Given $\E' \in E(r,d)$ and $d > 0$ there is a unique $\E \in E(r+d, d)$ which is an extension of $\E'$ by $\struct{X}^d$. As for degree $0$ case, for $\E \in E(r,0)$ there exists a unique $\F_r \in E(r, 0)$ with $\Gamma(\F_r) \neq 0$, and for all $\E \in E(r, 0)$ there exists a unique $\L \in E(1, 0)$ such that,
\[ \E = \F_r \ot \L \]
\end{document}


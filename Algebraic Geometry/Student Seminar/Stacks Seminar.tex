\documentclass[12pt]{article}
\usepackage{hyperref}
\hypersetup{
    colorlinks=true,
    linkcolor=blue,
    filecolor=magenta,      
    urlcolor=blue,
}

\usepackage{import}
\import{../}{AlgGeoCommands}

\begin{document}

Most taken for the section in Jarod Alper's notes on Quasi-Coherent Sheaves.

\section{Quasi-Coherent Sheaves}

Recall that for a DM-stack we defined the small \etale site:

\begin{defn}
Let $\X$ be a DM-stack. Then the \textit{small \etale site} $\X_{\et}$ of $\X$ is the category of schemes equipped with an \etale map $U \to \X$. A covering is $\{ U_i \to U \}$ over $\X$ such that $\sqcup_i U_i \to U$ is surjective.
\end{defn}

Then for a sheaf $\F$ on $\X_{\et}$ we defined its global sections,
\[ \Gamma(\X, \F) := \Hom{\Sh(\X_{\et})}{1}{\F} \]
where $1$ is the terminal sheaf (the sheafification of $U \mapsto *$). 

\begin{rmk}
This definition works nicely for $\X$ DM and naturally generalizes the \etale site $X_{\et}$ of a scheme. However, there is a glaring flaw if we attempt to extend this definition to Artin stacks there is a catastrophic failure: $\X_{\et}$ could be empty! For example, $(B \Gm)_{\et}$ is empty. Indeed, DM-stacks are exactly those stacks with schemes as \etale neighborhoods. To remedy this we could take the smooth site of $\X$. To stay in the world of \etale cohomology we consider a hybrid site where the schemes are smooth over $\X$ but the covers are all \etale.
\end{rmk}

\newcommand{\lisset}{\ell-\et}
\newcommand{\Y}{\mathcal{Y}}
\newcommand{\Zar}{\mathrm{Zar}}
\newcommand{\B}{\mathbf{B}}
\newcommand{\fU}{\mathfrak{U}}
\renewcommand{\QCoh}{\mathrm{QCoh}}
\newcommand{\DD}{\mathrm{DD}}
\newcommand{\cU}{\mathcal{U}}


\begin{defn}
Let $\X$ be an algebraic stack. Then the \textit{lisse-\etale site} $\X_{\lisset}$ is the category of schemes smooth over $\X$ with \textit{arbitrary} maps of schemes over $\X$. A covering $\{ U_i \to U \}$ is a collection of morphisms such that $\sqcup_i U_i \to U$ is surjective or \etale.
\end{defn}

\begin{defn}
Let $\F$ be a sheaf on $\X_{\lisset}$ then,
\[ \Gamma(\cU, \F) = \Hom{\Sh(\cU_{\lisset})}{1_{\fU}}{\F|_{\cU_{\lisset}}} \]
where $1_{\cU}$ is the \textit{indicator sheaf} of the smooth $\X$-stack $\cU \to \X$ the sheafification of the constant sheaf $\underline{*}$. This is the terminal object of $\cU_{\lisset}$.
This can be computed by choosing a smooth presentation,
\[ R \rightrightarrows U \to \cU \]
and setting,
\[ \Gamma(\fU, \F) = \eq \left( \F(U) \rightrightarrows \F(R) \right) \] 
\end{defn}

\begin{defn}
The structure sheaf $\struct{\X}$ is defined via,
\[ \struct{\X}(U) = \Gamma(U, \struct{U}) \]
is a ring object in the abelian category $\Ab(\X_{\lisset})$. We therefore define the abelian category $\Mod{\struct{\X}}$. Given a morphism $f : \X \to \Y$ of algebraic stacks there are adjoint pairs of functors\footnote{I think these do NOT arise from morphisms of topoi on the topoi of sheaves of sets, in the sense that $f^{-1}$ is not left exact in general. This is why people say that the lisse-\etale site is NOT functorial. However, $f^*$ is left exact on abelian sheaves see \chref{https://stacks.math.columbia.edu/tag/00XS}{Tag 00XS} {\color{red} HOW TO CONSTRUCT IT}},
\begin{center}
\begin{tikzcd}
\Ab(\X_{\lisset}) \arrow[r, bend left, "f_*"] & \Ab(\Y_{\lisset}) \arrow[l, bend left, "f^*"] & & \Mod{\struct{\X}} \arrow[r, bend left, "f_*"] & \Mod{\struct{\Y}} \arrow[l, bend left, "f^*"] 
\end{tikzcd}
\end{center}
Given two $\struct{\X}$-modules $\F$ and $\G$ we define the tensor product $\F \ot_{\struct{\X}} \G$ as the sheafification of,
\[ U \mapsto \F(U) \ot_{\struct{\X}(U)} \G(U) \]
and the Hom sheaf $\shHom{\struct{\X}}{\F}{\G}$ as the sheaf,
\[ U \mapsto \Hom{\struct{U}}{\F|_U}{\G|_U} \]
where $\F|_U$ means the restriction to the site $U_{\lisset}$ (note this is much more data than the restriction to $U_{\Zar}$). 
\end{defn}

\subsection{Quasi-Coherent Sheaves}

As above we denote by $\F|_U$ the restriction of $\F$ to $U_{\lisset}$ and $\F|_{U_{\Zar}}$ the restriction to $U_{\Zar}$. Then we define,

\begin{defn}
Let $\X$ be an algebraic stack. A $\struct{\X}$-module $\F$ is \textit{quasi-coherent} if:
\begin{enumerate}
\item for every smooth $U \to \X$ the restriction $\F|_{U_{\Zar}}$ is a quasi-coherent $\struct{U_{\Zar}}$-module
\item for every morphism $f : V \to U$ of smooth $\X$-schemes, the induced morphism,
\[ f^* (\F|_{U_{\Zar}}) \to \F_{V_{\Zar}} \]
is an isomorphism.
\end{enumerate}
\end{defn} 

\begin{rmk}
The above definition can be made in any site which refines the Zariski topology on each of its opens. However, in this generality such an object is usually called a \textit{crystal in quasi-coherent sheaves} and the term \textit{quasi-coherent} in an arbitrary site is reserved for the notion developed below. However, in most sites the two notions agree.
\end{rmk}


\begin{defn}
In an arbitrary ringed site $(\C, \struct{})$ (or even an arbitrary ringed topos) a $\struct{}$-module $\F$ is \textit{quasi-coherent} if for each object $U \in \C$ there exists a cover $\{ U_i \to U \}$ such that $\F|_{\C / U_i}$ is a \textit{presentable} $\struct{}$-module meaning there exists a presentation,
\begin{center}
\begin{tikzcd}
\bigoplus_{J} \struct{}|_{\C / U_i} \arrow[r] & \bigoplus_{I} \struct{}|_{\C / U_i} \arrow[r] & \F|_{\C / U_i} \arrow[r] & 0 
\end{tikzcd}
\end{center}
We call the abelian subcategory of such sheaves $\QCoh(\C) \subset \Mod{\struct{\C}}$.
\end{defn}

\begin{defn}
Let $S$ be a scheme and $\C \subset \Sch_S$ a subcategory. Consider the presheaf of rings,
\begin{align*}
\struct{} : \C^\op & \to \mathrm{Ring}
\\
(T \to S)  & \mapsto \Gamma(T, \struct{T}) 
\end{align*}
This is a sheaf for the fpqc topology. Furthermore, for any sheaf $\F$ on $S_{\Zar}$ there is a presheaf,
\begin{align*}
\struct{} : \C^\op & \to \mathrm{Ab}
\\
(f : T \to S) & \mapsto \Gamma(T, f^* \F)
\end{align*} 
which is a $\struct{}$-module. Furthermore, if $\F$ is quasi-coherent then $\F^a$ is a fpqc sheaf by descent.
\end{defn}

\begin{theorem}[\chref{https://stacks.math.columbia.edu/tag/03OJ}{Tag 03OJ}]
Let $S$ be a scheme. Let $\C$ be a site such that,
\begin{enumerate}
\item $\C$ is a full subcategory of $\Sch_S$
\item any Zariski covering of $T \in \C$ can be refined by a covering of $\C$
\item $\id : S \to S$ is an object of $\C$ (so it particular $\C$ has a terminal object)
\item every covering of $\C$ is an fpqc covering of schemes
\end{enumerate}
Then the presheaf $\struct{}$ is a sheaf on $\C$ and there is an equivalence of categories,
\begin{align*}
\QCoh(S) & \iso \QCoh(\C)
\\
\F & \mapsto \F^a
\end{align*}
\end{theorem}

\begin{proof}
This is basically a rephrasing of fpqc descent.
\end{proof}

\begin{prop}
Let $\F$ be a $\struct{\X_{\lisset}}$-module. Then the following are equivalent,
\begin{enumerate}
\item $\F$ is quasi-coherent in the general sense
\item $\F$ is quasi-coherent in the crystal sense.
\end{enumerate}
\end{prop}

\begin{proof}
C.f. \chref{https://stacks.math.columbia.edu/tag/06WK}{06WK}. Let $\C = \X_{\lisset}$. Suppose that $\F$ satisfies (a). Then the restriction of $\F$ is quasi-coherent on $\C_{/U}$ and thus by the previous lemma $\F|_{\C} = (\F|_{U_{\Zar}})^a$ and therefore satisfies (b). Given (b) take any $U \to \X$ smooth. Then we know $\F|_{U_{\Zar}}$ is quasi-coherent so there is an affine Zariski open cover $\{ U_i \to U \}$ such that $\F|_{(U_i)_{\Zar}}$ is presented. Then the claim is that $\F|_{\C/U_i}$ is also presented. Indeed, the comparison map induced by $f : V \to U$ is an isomorphism the presentation pulls back to give a presentation of $\F|_{\C/U_i}$. 
\end{proof}

\subsection{Descent Data}

\begin{defn}
Let $(U,R,s,t,c, e)$ be a groupoid scheme over $S$ where $s,t : R \rightrightarrows U$ are the source and target maps and $c : R \times_{s,U,t} R \to R$ is the composition and $e : U \to R$ is the identity. Then the category of \textit{descent data} consists of the category of pairs $(\F, \varphi)$ where $\F$ is a sheaf on $U$ and $\varphi$ is an isomorphism,
\[ \varphi : t^* \F \iso s^* \F \]
such that $e^* \varphi = \id$ and satisfying the cocycle condition,
\[ c^* \varphi = \pi_2^* \varphi \circ \pi_1^* \varphi \]
as morphisms of sheaves on $R \times_{s,U,t} R$. 
\end{defn}

\begin{example}
For any cover $U \to X$ we can form the ``Cech groupoid'' $U \times_X U \rightrightarrows U$ whose composition is given by projection,
\[ (U \times_X U) \times_{\pi_1, U, \pi_2} (U \times_X U) = U \times_X U \times_X U \to U \times_X U \quad \quad ((a,b), (c,a)) \mapsto (c,a,b) \mapsto (c, b) \] 
For this we recover the ordinary notion of a descent datum. 
\end{example}

\begin{example}
Let $G \acts X$ be an action of an algebraic group on a scheme. Then there a groupoid $G \times X \rightrightarrows X$ whose composition $G \times G \times X \to G \times X$ is given by multiplication in the group. 

For this we will recover the notion of $G$-equivariance. 
\end{example}


\begin{prop}
Let $R \rightrightarrows U$ be a smooth presentation of an algebraic stack $\X$ by schemes. There is an equivalence of categories,
\[ \QCoh(\X) \to \DD_{\mathrm{QCoh}}(U/R) \quad \F \mapsto (\F|_{U_{\Zar}}, \varphi) \]
where $\mathrm{DD}_{\mathrm{QCoh}}(U/R)$ is the category of descent data for quasi-coherent sheaves along the groupoid $R \rightrightarrows U$.
\end{prop}

\begin{proof}
For any smooth map $V \to \X$ there is a further smooth refinement $V' \to V$ such that $V' \to \X$ factors through $U \to \X$. Hence, applying descent to $V' \to V$, any quasi-coherent sheaf $\F$ on $\X_{\lisset}$ is determined by its descent data over $R \rightrightarrows U$.
\end{proof}

\begin{defn}
Let $G \acts X$ be an action of a group scheme on a scheme (or algebraic space). The category of $G$-equivariant sheaves is defined as the category of descent data for the groupoid $G \times X \rightrightarrows X$.
\end{defn}

\begin{rmk}
Some standard diagram chasing shows that this is formally the same as the usual definition of a $G$-equivariant sheaf in [\chref{https://stacks.math.columbia.edu/tag/03LE}{Stacks}]. In the case that $G$ is a finite constant group it is easy to check that this agrees with the naive notion in terms of compatible isomorphisms between the pullbacks along the action by elements $g \in G$. 
\end{rmk}

\begin{prop}
There is an equivalence of categories,
\[ \QCoh([X/G]) \to \QCoh_{G}(X) \]
\end{prop}

\begin{proof}
This is a special case of the previous proposition.
\end{proof}

\subsection{Examples}

\begin{example}
Let $\X \to S$ be a DM-stack. Then the sheaf,
\[ \Omega_{\X/S} : (U \to \X) \mapsto \Gamma(U, \Omega_{U/S}) \]
is quasi-coherent since any morphism $f : V \to U$ in $\X_{\et}$ is \etale so the map,
\[ f^* \Omega_{U/S} \iso \Omega_{V/S} \]
is an isomorphism. However, if $\X \to S$ is not DM we don't have access to $\X_{\et}$ nor can we define $(\Omega_{X/S})^a$ on $X_{\fppf}$ as we can for a scheme since there is no Zariski or \etale site to define this sheaf over for a bootstrap. There is still a sheaf of $\struct{\X_{\lisset}}$-modules,
\[ \Omega_{\X/S} : (U \to \X) \mapsto \Gamma(U, \Omega_{U/S}) \]
but it is not quasi-coherent. This is the sort of sheaf the stacks project calls \textit{locally quasi-coherent} meaning that it is quasi-coherent when restricted to $U_{\et}$ for any $U \to \X$. 
\end{example}

\newcommand{\LL}{\mathbb{L}}

\begin{rmk}
Indeed, it is not clear that an Artin stack $\X \to S$ should have any good notion of a cotangent bundle $\Omega_{\X/S}$. For example, consider $\X = \B \Gm$ which is smooth of relative dimension $-1$ so what should $\Omega_{\X/S}$ even be? It can't be a vector bundle of rank $-1$ can it! To fix this conundrum, we either work with $\Omega_{\X/S}$ as defined above which is not quasi-coherent and hence does not have a well-defined rank or we define the cotangent complex $\LL_{\X/S} \in D_{\QCoh}^{\le 1}(\X)$ (technically it's an ind-object in this generality) [Champs Algebriques, Chapter 17] which encodes the deformation theory of $\X$. Note that unlike for a scheme, $\LL_{\X/S}$ can be supported in degree $1$. In fact, the following are equivalent,
\begin{enumerate}
\item $\X \to S$ is DM
\item $\cH^1(\LL_{\X/S}) = 0$
\end{enumerate}
Proof: [Champs Algebriques, Cor. 17.9.2]. 
\end{rmk}


\subsection{Picard Groups}

\newcommand{\Mbar}{\overline{\M}}

Let $\X$ be an algebraic stack. Then $\Pic{\X}$ denotes the set of isomorphism classes of line bundles on $\X$ which becomes an abelian group under $\ot$.

\begin{example}
If $G$ is an affine algebraic $k$-group then $\Pic{\B G} = \Hom{\text{gp}}{G}{\Gm}$ is the group of characters. For example, 
\begin{enumerate}
\item $\Pic{\B \Gm} = \Z$
\item $\Pic{\B \GL_n} = \Z$
\item $\Pic{\B \PGL_n} = \{ 0 \}$.
\end{enumerate}
This is because line bundles on $\B G$ are the same as line bundles on $\Spec{k}$ along with descent data i.e. a $G$-action. This is the same as a $1$-dimensional $G$-representation. 
\end{example}

\begin{example}
Consider the action, $\Gm \acts \A^n$ with weights $d_1, \dots, d_n$. Let the \textit{weighted projective stack} be the DM-stack (at least if $p \ndivides d_i$),
\[ \cP(d_1, \dots, d_n) = [(\A^n \sm \{ 0 \}) / \Gm] \] 
Here let $k$ be a field of characteristic not dividing any $d_i$ or $2$ or $3$.
\begin{enumerate}
\item The map $\Pic{\B \Gm} \to \Pic{\cP(d_1, \dots, d_n)}$ induced by the canonical $\Gm$-bundle is an isomorphism. Indeed, this reduces to classifying $\Gm$-actions on $\struct{\A^n \sm \{ 0 \}}$. By Hartogs' these correspond to $\Gm$-actions on $\struct{\A^n}$ and thus to different grading of $A = k[x_1, \dots, x_n]$ as an $A$-module with $x_i$ given weight $d_i$. These are just overall shifts $A(d)$ i.e. putting $1$ in degree $d$. This is the same as the pullback of the bundle over $\B \Gm$ corresponding to $\Gm \xrightarrow{z^d} \Gm$.

\item Using Weierstrass models we get an isomorphism,
\[ \Mbar_{1,1} \cong \cP(4,6) \]
Therefore, $\Pic{\Mbar_{1,1}} = \Z \omega_{\C/\M}$ 

\item Then it turns out that,
\[ \Pic{\M_{1,1}} = \Z / 12 \Z \omega_{\C/\M} \]
This is because the discriminant $\Delta$ is a section of $\struct{}(12)$ which is nowhere vanishing for smooth families.
\end{enumerate}
\end{example}

\subsection{Global Quotients and the Resolution Property}

\begin{defn}
An algebraic stack $\X$ is a \textit{global quotient stack} if there is an isomorphism $\X \cong [U / \GL_n]$ where $U$ is an algebraic space. This is equivalent to asking for the existence of a $\GL_n$-bundle $U \to \X$ where $U$ is an algebraic space. By definition this is the same as a representable morphism $\X \to \B \GL_n$.
\end{defn}

\begin{prop}
Let $\X \to \Y$ be a surjective, flat, and projective morphism of noetherian algebraic stacks. If $\X$ is a quotient stack then $\Y$ is a quotient stack.
\end{prop}

\begin{defn}
A noetherian algebraic stack has the \textit{resolution property} if every coherent sheaf if a quotient of a vector bundle.
\end{defn}

A smooth or quasi-projective scheme over a field has the resolution property. Any noetherian normal $\Q$-factorial scheme with affine diagonal also has the resolution property.

\begin{prop}
Let $G$ be an affine algebraic $k$-group with an action $G \acts U$ on a quasi-projective $k$-scheme $U$. Assume that there is an ample line bundle $\L$ with a $G$-action. Then $[\Spec{A}/G]$ has the resolution property.
\end{prop}

\begin{rmk}
While not every line bundle $\L$ on a normal $k$-scheme admits a $G$-action, it turns out there is always some positive power such that $\L^{\ot n}$ has a $G$-action.
\end{rmk}

\begin{proof}
The $G$-line bundle $\L$ corresponds to a line bundle on $[U/G]$ which is ample which respect to the morphism $p : [U/G] \to \B G$ since relative ampleness can be checked after smooth covers (it can be reduced to a fiberwise condition). For a coherent sheaf $\F$ on $[U/G]$ the natural map,
\[ \L^{-\ot N} \ot p^* p_* (\L^{\ot N} \ot \F) \onto \F \]
is surjective for $N \gg 0$ since relative ampleness implies global generation of $\L^{\ot N} \ot \F$. The pushforward $p_* (\L^{\ot N} \ot \F)$ is quasi-coherent on $\B G$ hence a $G$-representation. We can hence write it as an increasing union of finite-dimensional $G$-representations $V_i$ and obtain,
\[ \colim_i (\L^{-\ot N} \ot p^* V_i) \onto \F \]
since $\F$ is coherent, this stabilizes at some stage meaning,
\[ \L^{-\ot N} \ot p^* V_i \onto \F \]
at some finite stage $i$.
\end{proof}

\begin{theorem}[Totaro-Gross]
Let $\X$ be a quasi-separated normal algebraic stack of finite type over $k$. Assume that the stabilizer group at every closed point is smooth and affine. Then the following are equivalent:
\begin{enumerate}
\item $\X$ has the resolution property
\item $\X \cong [U / \GL_n]$ with $U$ quasi-affine
\item $\X \cong [\Spec{A} / G]$ with $G$ an affine algebraic group.
\end{enumerate}
In particular, $\X$ has affine diagonal.
\end{theorem}

\begin{rmk}
The normality hypothesis on $\X$ and smoothness hypothesis on the stabilizers are unnecessary. However, the affineness hypothesis on the stabilizers is necessary. For example, $\B E$ the classifying stack of an elliptic curve has the resolution property. 
\end{rmk}

\subsection{Sheaf Cohomology}

\renewcommand{\Cech}{\check{C}}

\begin{lemma}
If $\X$ is an algebraic stack, the categories $\Ab(\X_{\lisset})$ and $\Mod{\X}$ have enough injective. If $\X$ is quasi-separated then $\QCoh{(\X)}$ has enough injectives.
\end{lemma}

\begin{defn}
Let $\X$ be an algebraic stack and $\F$ a sheaf on $\X_{\lisset}$. The \textit{cohomology groups} $H^i(\X_{\lisset}, \F)$ are the derived functors of,
\[ \Gamma(\X, -) : \Ab(\X_{\lisset}) \to \Ab \]
applied to $\F$,
\[ H^i(\X_{\lisset}, \F) = R^i \Gamma(\X, \F) \]
\end{defn}

\begin{defn}
Given a smooth covering $\fU = \{ \cU_i \to \X \}_{i \in I}$ of algebraic stacks and an abelian presheaf $\F$ on $\X_{\lisset}$ the \textit{Cech complex} $\Cech^\bullet(\fU, \F)$ of $\fU$ with respect to $\fU$ is,
\[ \Cech^n(\fU, \F) = \prod_{(i_0, \dots, i_n) \in I^{n+1}} \F(\cU_{i_0} \times_{\X} \cdots \times_{\X} \cU_{i_n}) \]
with differential,
\[ \d^n : \Cech^n(\fU, \F) \to \Cech^{n+1}(\fU, \F) \quad (s_{i_0, \dots, i_n}) \mapsto \left( \sum_{k = 0}^{n+1} (-1)^k p_{\hat{k}}^* s_{i_0, \dots, \hat{i}_k, \dots, i_n} \right)_{i_0, \dots, i_{n+1}} \]
where the projection $p_{\hat{k}}$ forgets the $t^{\text{th}}$ coordinate. The \textit{\v{C}ech cohomology} of $\F$ with respect to $\fU$ is,
\[ \check{H}^i(\fU, \F) := H^i(\Cech^\bullet(\fU, \F)) \] 
\end{defn}

\begin{theorem}
Let $\X$ be an algebraic stack and $\F$ a quasi-coherent sheaf on $\X_{\lisset}$. Then for any cover $\fU = \{ \cU_i \to \X \}_{i \in I}$ there exists a spectral sequence,
\[ E^{p,q}_2 = \check{H}^p(\fU, H^q(-,\F)) \implies H^{p+q}(\X_{\lisset}, \F) \]
where $H^q(-,\F)$ is the presheaf $U \mapsto H^q(U_{\lisset}, \F)$. 
\end{theorem}

\begin{proof}
Consider the commutative diagram of functors,
\begin{center}
\begin{tikzcd}
\mathrm{Sh}(\X_{\lisset}) \arrow[rd, "\Gamma"'] \arrow[r, "a", hook] & \mathrm{PSh}(\X_{\lisset}) \arrow[d, "\check{H}^0"]
\\
& \Ab
\end{tikzcd}
\end{center}
Notice that $\Cech^\bullet(\fU, -)$ is exact in the category of presheaves which shows that $\check{H}^\bullet(\fU, -)$ forms a $\delta$-functor. In fact, since $\check{H}^i(\fU, \I) = 0$ for $i > 0$ and any injective sheaf (this is a very general fact, see \chref{https://stacks.math.columbia.edu/tag/03OR}{Tag 03OR}) it is a universal $\delta$-functor. Now the inclusion $a$ takes injectives to injectives because sheaves form a reflexive subcategory (maps to a sheaf factors through the sheafification). Therefore, we apply the Grothendieck spectral sequence so it suffices to compute $R^q a(\F)$ of a sheaf $\F$. Since the functor $(-) \mapsto \Gamma(U, -)$ is exact on presheaves we see that,
\[ R^q a(\F)(U) = R^q \Gamma(U, \F) = H^q(U, \F) \]
so we conclude.
\end{proof}

\begin{theorem}
If $X$ is an affine scheme and $\F$ is a quasi-coherent $\struct{\X_{\lisset}}$-module then,
\[ H^i(X_{\lisset}, \F) = 
\begin{cases}
\Gamma(X, \F) & i = 0
\\
0 & i > 0
\end{cases} \]
\end{theorem}

\begin{proof}
We refine to affine coverings $\{ \Spec{B} \to \Spec{A} \}$ then $\F$ is quasi-coherent (in all equivalent notions) and hence arises from some $A$-module $M$. To show that $\check{H}^{>0} = 0$ for this covering we show that the Amistur complex,
\begin{center}
\begin{tikzcd}
0 \arrow[r] & M \arrow[r] & M \ot_A B \arrow[r] & M \ot_A B \ot_A B \arrow[r] & M \ot_A B \ot_A B \ot_A B \arrow[r] & \cdots
\end{tikzcd}
\end{center}
is exact. Indeed, after applying $B \ot_A -$ which is faithfully flat this complex obtains a nullhomotopy. Now to conclude, we can either apply Cartan's criterion (\chref{https://stacks.math.columbia.edu/tag/03F9}{Tag 03F9}) or use hypercoverings and the fact that hypercover Cech cohomology computes derived functor cohomology.
\end{proof}

\begin{prop}
Let $\X$ be an algebraic stack with affine diagonal and $\F$ be a quasi-coherent sheaf. If $\fU = \{U_i \to \X \}$ is an \etale covering with each $U_i$ affine, then $H^i(\X_{\lisset}, \F) = \check{H}^i(\fU, \F)$. 
\end{prop}

\begin{proof}
Follows immediately from the Cech-to-derived spectral sequence and the above. 
\end{proof}

\begin{rmk}
To remove the ``affine diagonal'' condition we need to use hypercovers. Indeed, if $U_\bullet \to \X$ is a simplicial hypercover in $\X_{\lisset}$ where each $U_\bullet$ is an affine scheme and $\F$ is quasi-coherent then,
\[ H^i(\X, \F) = \check{H}^i(U_\bullet, \F) \]
\end{rmk}

\begin{prop}
Let $X$ be a scheme (or a DM-stack with a sheaf on $\X_{\et}$) \ul{with affine diagonal}\footnote{If we use hypercovers (see the discussion in the proof then we can remove this condition.} and $\F$ a quasi-coherent sheaf. Then,
\[ H^i(X, \F) = H^i(X_{\lisset}, \F_{\lisset}) \]
for all $i$ where $\F_{\lisset}$ is the $\struct{X_{\lisset}}$-module defined by,
\[ \F_{\lisset}(U) = \Gamma(U, f^* \F) \]
for a smooth map $f : U \to X$. (In the stack case it is pullback under $f : \X_{\lisset} \to \X_{\et}$).
\end{prop}

\begin{proof}
Choose an affine Zariski cover $\U$ of $X$ (affine \etale cover $\U$ of $\X$) by the assumption on the diagonal we see that,
\[ H^i(X_{\lisset}, \F) = \check{H}^i(\U, \F) = H^i(X, \F) \]
(and similarly for $\X$). The affine diagonal condition is to ensure that projects in the Cech complex are affine and hence have vanishing $H^{>0}$. However, this condition is not necessary. We can always choose a Zariski hypercover $U_\bullet \to X$ by affines and similar arguments show that,
\[ H^i(X_{\lisset}, \F) = \check{H}^i(U_\bullet, \F) = H^i(X, \F) \]
\end{proof}

\begin{prop}
Let $\X$ be an algebraic stack.
\begin{enumerate}
\item $\F$ is an $\struct{\X}$-module then $H^i(\X_{\lisset}, \F)$ agrees with $R^i \Gamma : \Mod{\struct{\X}} \to \Ab$ computed in the category of $\struct{\X}$-modules.
\item If $\X$ has affine diagonal and $\F$ is a quasi-coherent sheaf on $\X$, then the cohomology $H^i(\X_{\lisset}, \F)$ agrees with $R^i \Gamma : \QCoh{(\X)} \to \Ab$ computed in the category of quasi-coherent modules.
\end{enumerate}
For a morphism $f : \X \to \Y$ of algebraic stacks (resp. quasi-compact morphism of algebraic stacks with affine diagonals) then (a) (resp. (b)) holds also for $R^i f_* \F$ of an $\struct{\X}$-module (resp. quasi-coherent sheaf).
\end{prop}

\begin{rmk}
If $\X$ does not have affine diagonal, then the sheaf cohomology $H^i(\X_{\lisset}, \F)$ of a quasi-coherent sheaf may differ from the derived functor $R^i \Gamma(\X, -) : \QCoh{(\X)} \to \Ab$.
\end{rmk}

\begin{prop}
If $\X$ is an algebraic stack and $\F_i$ is a directed system of abelian sheaves in $\X_{\lisset}$ then $\colim_i H^i(\X, \F_i) \to H^i(\X, \colim_i \F_i)$ is an isomorphism. 
\end{prop}

\section{July 8 Affine GIT and Good moduli spaces}

\subsection{Good Moduli Spaces}

\begin{defn}
A quasi-compact quasi-separated morphism $\pi : \X \to X$ from an algebraic stack to an algebraic space is a \textit{good moduli space} if,
\begin{enumerate}
\item $\struct{X} \iso \pi_* \struct{\X}$
\item $\pi_* : \QCoh(\X) \to \QCoh(X)$ is exact.
\end{enumerate}
\end{defn}

\begin{example}
Let $G$ be linearly $k$-reductive (meaning taking invairants of representations is exact) then $G \acts \Spec{A}$ gives a diagram,
\begin{center}
\begin{tikzcd}
\Spec{A} \arrow[rd] \arrow[d]
\\
\left[\Spec{A}/G\right] \arrow[r, "\pi"] & \Spec{A^G}
\end{tikzcd}
\end{center}
Then $\pi$ satisfies the properties of a good moduli space morhism. Indeed,
\begin{enumerate}
\item $\Gamma([\Spec{A}/G], \struct{[\Spec{A}/G]}) = A^G$ 

\item and $\pi_* : \QCoh([\Spec{A}/G]) \to \QCoh(\Spec{A^G})$ is exact since taking invariants of a $G$-representation is exact.
\end{enumerate}
\end{example}

\begin{example}
$[\A^1 / \Gm] \to \Spec{k}$ is a good moduli space
\end{example}

\begin{example}
$[\P^1 / \Gm] \to \Spec{k}$ is NOT a good moduli space. Indeed (2) fails because $\pi_* = H^0$ is not exact for $\Gm$-equivariant sheaves on $\P^1$. Here two closed points specializes to 1 closed point downstairs. We will see this is a problem. 
\end{example}

\begin{theorem}
Let $\pi : \X \to X$ be a good moduli space and $\X$ be q-sep over a scheme $S$.
\begin{enumerate}
\item $\pi$ is surjective and universally closed
\item if $\Z_1, \Z_2 \subset \X$ are closed substacks then $\pi(\Z_1 \cap \Z_2) = \pi(\Z_1) \cap \pi(\Z_2)$ where this denotes scheme-theoretic (or rather stack-theoretic) images and intersections. In particular, for geometric points $x_1, x_2 \in \X(k)$ then $\pi(x_1) = \pi(x_2)$ iff $\overline{ \{ x_1 \} } \cap \overline{ \{ x_2 \} } \neq \empty$. 
\item If $\X$ is noetherian then $X$ is also noetherian.
\item If $\X$ is of finite type over $S$ and $S$ is noetherian then $X$ is finite type over $S$ and $\pi_*$ preserves coherent sheaves.

\item If $\X$ is noetherian then $\pi$ is initial for maps to algebraic spaces.
\end{enumerate}
\end{theorem}

\begin{cor}
Let $G$ be linearly reductive group over $k = \bar{k}$ and $\tilde{\pi} : U = \Spec{A} \to U // G = \Spec{A^G}$ then,
\begin{enumerate}
\item $\tilde{\pi}$ is surjective and for any $G$-invariant closed $Z \subset U$ (meaning corresponding to a closed substack of $[U / G]$) then $\pi(Z) \subset U // G$ is closed. This property remains true after base change
\item given $G$-invariant closed $Z_1, Z_2 \subset U$ then $\pi(Z_1 \cap Z_2) = \pi(Z_1) \cap \pi(Z_2)$ so if $x_1, x_2 \in U(k)$ then $\tilde{\pi}(x_1) = \tilde{\pi}(x_2)$ iff $\overline{G \cdot x_1} \cap \overline{G \cdot x_2} \neq \empty$.

\item If $A$ is Noetherian so is $A^G$. If $A$ is f.g. over $k$ then $AG^G$ is f.g. over $k$ and for any f.g. $A$-module $M$ with a $G$-action, $M^G$ is a f.g. $A^G$-module

\item if $A$ is noetherian then $\tilde{\pi}$ is initial for $G$-invariant maps to algebraic spaces.
\end{enumerate}
\end{cor}

\subsection{Cohomologically Affine Morphisms}

\begin{defn}
A quasi-compact quasi-separated morphism $f : \X \to \Y$ is \textit{cohomologically affine} if $f_* : \QCoh(\X) \to \QCoh(\Y)$ is exact. We say $\X$ is \textit{cohomologically affine} if $\X \to \Spec{\Z}$ is.
\end{defn}

\begin{example}
\begin{enumerate}
\item an affine morphism is cohomologically affine
\item an affine algebraic group $G / k$ is linearly reductive iff $\B G \to \Spec{k}$ is cohomologically affine
\end{enumerate}
\end{example}

\begin{lemma}
Consider a diagram,
\begin{center}
\begin{tikzcd}
\X' \arrow[d, "\pi"'] \arrow[r, "f"] & \X \arrow[d, "\pi"]
\\
\Y' \arrow[r, "g"] & \Y
\end{tikzcd}
\end{center}

\begin{enumerate}
\item if $f$ is faithfully flat and $\pi'$ is a GMS then $\pi$ is a GMS
\item if $\Y$ has quasi-affine diagonal and $\pi$ is a GMS then $\pi'$ is a GMS
\end{enumerate}
\end{lemma}

\begin{proof}
(a) by flat base change $\pi_*' f^* \cong g^* \pi_*$ and $f^*$ is exact and $g^*$ is faithfully exact so we conclude. For (b), first consider the case that $g$ is quasi-affine. Then factor as,
\[ \Y' \embed \rSpec{\Y}{f_* \struct{\Y'}} \to \Y \]
where the second map is affine by definition. If $g$ is affine then $g_*$ is faithfully exact. If $g$ is an open immersion then for $F' \onto G'$ in $\QCoh(\X')$ we conclude that $G := \im{(f_* F' \to f_* G')}$ since $f^* f_* = \id$ we get $f^* G = G'$ and $\pi_*$ is exact so $\pi_* f_* F' \to \pi_* F$ so pullback via $g$ then,
\[ g^* \pi_* f_* F' = \pi_*' f^* f_* F' = \pi'_* F' \onto g^* \pi_* G = \pi'_* f^* G = \pi_*' G \]
the first by flat basechange. Thus we conclude. To do the general case, assume $\Y$ and $\Y'$ are quasi-compact and choose a smooth presentation $Y = \Spec{A} \to \Y$ which is quasi-affine since $\Y$ is. Then $\Y'_Y \to \Y$ is faithfully flat so by (a) suffices to show that $\X'_Y \to \Y'_Y$ is cohomologically affine. By (a) again we can pass to a smooth cover $Y' \to \Y'_{Y}$ and reduce to a morphism of affine schemes which is hence affine so we win. 
\end{proof}

\begin{cor}
If $f : \X \to \Y$ is representable and $\Y$ has quasi-affine diagonal. If $f$ is cohomologically affine then $f$ is affine.
\end{cor}

\begin{proof}
Suffices to prove this for a map $f : X \to Y$ of schemes with $Y$ affine. However, this is just Serre's criterion for affineness. 
\end{proof}

\subsection{First Properties of GMS}

\begin{lemma}
Consider a diagram,
\begin{center}
\begin{tikzcd}
\X' \arrow[r, "f"] \arrow[d, "\pi'"] & \X \arrow[d, "\pi"]
\\
X' \arrow[r, "g"] & X 
\end{tikzcd}
\end{center}
with $X', X$ quasi-separated algebraic spaces and $X$ has quasi-affine diagonal.
\begin{enumerate}
\item if $g$ is faithfully flat and $\pi'$ is a GMS then $\pi$ is a GMS
\item if $\pi$ is a GMS then $\pi'$ is a GMS.
\end{enumerate}
Now assume that $\pi$ is a GMS then,
\begin{enumerate}
\item there is a projection formula for $F \in \QCoh(\X)$ and $G \in \QCoh(X)$ then,
\[ \pi_* \F \ot G \iso \pi_* (F \ot \pi^* G) \]
In particular,
\[ G \iso \pi_* \pi^* G \]
\item $F \in \QCoh(\X)$ then,
\[ g^* \pi_* F \iso \pi'_* f^* F \]
\item for any $\I \subset \struct{X}$ a quasi-coherent ideal sheaf then,
\[ \I \iso \pi_* (\pi^{-1} \I \cdot \struct{\X}) \]
\end{enumerate}
\end{lemma}

\begin{proof}
For (a) and $g$ flat then $g^* (\struct{X} \to \pi_* \struct{\X})$ is just $\struct{X'} \to \pi_*' \struct{\X'}$ by flat base change. Thus if $g$ is faithfully flat then we get,
\[ \struct{X} \iso \pi_* \struct{\X} \iff \struct{X'} \iso \pi_*' \struct{\X'} \]
Furthermore, the lemma says that cohomological flatness descends under $g$ proving (a). We also showed that (b) holds for flat base change. To prove it for arbitrary base change we first prove the projection formula. Indeed, let $U \to X$ be an \etale presentation with $U$ disjoint union of affine schemes. Then $\pi_U : \X_U \to U$ is a GMS by flat case. Then the pullback of $\id \to \pi_* \pi^*$ is $\id \to \pi_{U*} \pi_U^*$. We can assume that $X = \Spec{A}$ and consider,
\[ G_2 \to G_1 \to G \to 0 \]
is a free presentation. Then consider,
\begin{center}
\begin{tikzcd}
\pi_* F \ot G_2 \arrow[d, equals] \arrow[r] & \pi_* F \ot G \arrow[d, equals] \arrow[r] & \pi_ F \ot G \arrow[d] \arrow[r] & 0
\\
\pi_* (F \ot \pi^* G_2) \arrow[r] & \pi_*(F \ot \pi^* G_1) \arrow[r] & \pi_* (F \ot \pi^* G) \arrow[r] & 0  
\end{tikzcd}
\end{center}
using that $\pi_*$ is exact and the locally free form of the projection formula. Then we conclude the projection formula by the 5 lemma. 
(DO THE REST)
\end{proof}

\begin{lemma}
Let $\pi : \X \to X$ be a GMS and $X$ is quasi-separated. Then,
\begin{enumerate}
\item for $\cA \in \QCoh(\X)$ then $\rSpec{\X}{\cA} \to \rSpec{X}{\pi_* \cA}$ is a GMS
\item $\Z \subset \X$ is a closed substack defined by $\I$ then $Z = \pi(\Z) \subset X$ satisfies that $\Z \to Z$ is a GMS. 
\end{enumerate}
\end{lemma}

\begin{proof}
Indeed (b) is the special case of (a) where $\cA = \struct{\X} / \I$. For (a) we see that,
\[ \rSpec{\X}{\cA} \to \X \times_{X} \rSpec{X}{\pi_* \cA} \to \rSpec{X}{\pi_* \cA} \]
the first is affine and the second is cohomologically affine by base change. 
\end{proof}

I SHOULD HAVE BEEN CAREFUL ABOUT $\pi$ vs $\im{}$ since I MEAN SCHEME THEORETIC IMAGE

\begin{proof}[Proof of Theorem]
(a) if $\X$ is quasi-sep then so is $X$. Then for all $x \in X(k)$ use Lemma 6.3.20 (b) $\X \times_X \Spec{k} \to \Spec{k}$ is a GMS. Then $\Gamma(\X_x, \struct{\X_x}) = k$ so $\X_x \neq \empty$ and hence $\pi$ is surjective. To prove universal closedness consider $\Z \subset \X$ a closed substack use Lemma~6.3.22 (b) then $\Z \to \pi(\Z)$ is a GMS and hence surjective hence $\pi(\Z)$ the scheme theoretic image is just equal to the image and hence the image is closed. Then use preservation under base change to get universally closed. (b) if $\Z_i \subset \X$ are defined by $\I_i$ then apply $\pi_*$ to the sequence,
\begin{center}
\begin{tikzcd}
0 \arrow[r] & \I_1 \arrow[r] & \I_1 + \I_2 \arrow[r] & \I_2 / (\I_1 \cap \I_2) \arrow[r] & 0
\end{tikzcd}
\end{center}
so we get,
\[ \pi_* \I_1 + \pi_* \I_2 \cong \pi^* (\I_1 + \I_2) \]
and hence,
\[ \im(\Z_1) \cap \im(\Z_2) = \im(\Z_1 \cap \Z_2) \]
\end{proof}

\subsection{Finite Typeness of GMS}

\begin{defn}
A morphism $f : X \to Y$ of schemes is \textit{universally submersive} if it is surjective and $Y$ has the quotient topology ($U \subset Y$ open iff $f^{-1}(U)$ is open) and this is true after any base change. 
\end{defn}

\begin{lemma}
Valuative criterion:
\begin{center}
\begin{tikzcd}
\Spec{R'} \arrow[d] \arrow[r, dashed] & X \arrow[d]
\\
\Spec{R} \arrow[r] & Y
\end{tikzcd}
\end{center}
if $X \to Y$ is universally submersive then there exists $R' / R$ extensions of DVRs lifitng. 
\end{lemma}

\begin{example}
The following maps are universally submersive,
\begin{enumerate}
\item fppf covering
\item good moduli space morphism
\end{enumerate}

\begin{proof}

\end{proof}
\end{example}

\section{Determinantal Line Bundles}

\newcommand{\R}{\mathbf{R}}
\newcommand{\univ}{\mathrm{univ}}
\newcommand{\cN}{\mathcal{N}}
\newcommand{\detstar}{\det\nolimits^{*}}

Recall that $k$ is an algebraically closed field of characteristic zero.
\bigskip\\
Let $X$ be a smooth, projective and connected curve over $k$. Let $\S$ be an algebraic stack over $k$. Then consider the diagram,
\begin{center}
\begin{tikzcd}
& X \times \S \arrow[rd, "p"] \arrow[ld, "q"']
\\
X & & \S
\end{tikzcd}
\end{center}
Since $X \times \S \to \S$ is representable by schemes we can use cohomology and base change theorem. If $\E$ is a vector bundle on $X \times \S$ then $\R p_* \E$ is a refect complex on $\S$ with amplitude in $[0,1]$ since this is true for the projection map $X \times T \to T$ from a test scheme. {\color{red} what does amplitude mean}

\begin{prop}
Let $k$ be a field, $S$ a $k$-scheme of finite type and $f : X \to S$ a smooth projective morphism of relative dimension $n$. If $F$ is a flat coherent sheaf on $X$ then there is a locally free resolution
\[ 0 \to F_n \to F_{n-1} \to \cdots \to F_0 \to F \to 0 \]
such that $R^n f_* F_\nu$ is locally free for $\nu = 0, \dots, n$ and $R^i f_* F_\nu = 0$ for $i \neq n$ and $\nu = 0, \dots, n$. Moreover, there is a natural quasi-isomorphism
\[ [R^n f_* F_n \to R^n f_* F_{n-1} \to \cdots \to R^n f_* F_0] \to (\R f_* F)[n] \]
\end{prop}

\begin{proof}
Let $\struct{X}(1)$ be an $f$-very ample line bundle on $X$. Since $f$ is a Cohen-Macaulay morphism (it is even smooth) we have a Serre duality pairing,
\[ R^i f_* \F(-m) \times R^{n-i} f_* (\F^\vee(m) \ot \omega_{X/Y}) \to R^n f_* \omega_{X/Y} = \struct{X} \]
which is perfect. Using the mumford complex, there is some $m_0$ such that for $m \ge m_0$ we have $R^{n-i} f_* (\F^\vee(m) \ot \omega_{X/Y}) = 0$ for all $i < n$ and hence $R^i f_* \F(-m) = 0$. Define $S$-flat sheaves $K_\nu$ and $G_\nu$ inductively as follows. Let $K_0 := F$. Assume that $K_\nu$ has been constructed. Since this is a bounded family, for sufficiently large $m \gg m_0$ all fibers $(K_\nu)_s$ are $m$-regular. Hence, by cohomology and base change $f_* K_\nu(m)$ is locally free (since its higher cohomology vanishes) and there is a natural surjection $G_\nu := f^* (f_* K_\nu(m))(-m) \to K_\nu$ (surjective on fibers by $m$-regularity). Then $G_\nu$ is locally free and
\[ R^i f_* G_\nu = f_* K_\nu(m) \ot R^i f_* \struct{X}(-m) \]
by the projection formula. In particular, $R^n f_* G_\nu$ is locally free and the other direct image sheaves vanish. Finally, define
\[ K_{\nu + 1} := \ker{(G_\nu \to K_\nu)} \]
Therefore we get an infinite locally free resolution $G_\bullet \to F$. Since for each $s \in S$
\[ (K_n)_s = \ker{((G_{n-1})_s \to (G_{n-2})_s)} \]
and the fiber $X_s$ is regular of dimension $n$ it has global dimension $n$ and therefore $(K_n)_s$ is locally free since it is the $n^{\text{th}}$ syzygy module in the locally free resolution of the locally free sheaf $F_s$. Hence $K_n$ is itself locally free by Nakayama\footnote{Suppose that $f : X \to S$ is a morphism and $F$ is a coherent $\struct{X}$-module. If $F_x$ is flat over $\stalk{S}{s}$ then $F$ is locally free at $x$ if and only if $F_s$ is locally free at $x \in X_s$. Recall that $F$ is locally free at $x$ if and only if $F_x$ is free over $\stalk{X}{x}$ since $F$ is coherent. Then if $(F_s)_x = F_x / \m_s F_x$ is free we lift the basis to
\[ 0 \to K \to \struct{X}{x}^{\oplus r} \to F_x \to 0 \]
but $F_x$ is flat over $\stalk{S}{s}$ so this stays exact when applying $- \ot \stalk{S}{s}/\m_s$ so we see that $K / \m_s K = 0$ and hence $K / \m_x K = 0$ so by Nakayama's lemma $K = 0$.}. Therefore, we can truncate to get the locally free resolution $F_\bullet$. The last statement follows from the first. Indeed, the quasi-isomorphism,
\[ [F_n \to \cdots \to F_0] \to F[0] \]
gives the desired result after applying $\R f_*$ and showing that the natural map {\color{red} WHERE DOES IT COME FROM?}
\[ \R f_* [F_n \to \cdots \to F_0] \to [R^n f_* F_n \to \cdots \to R^n f_* F_0][-n] \]
is a quasi-isomorphism. Indeed, the derived functor spectral sequence shows that,
\[ E^{p,q}_2 = \cH^p(R^q f_* F_\bullet) \implies \cH^{p+q}(\R f_* F_\bullet) \]
but the only nonzero part of the $E_2$ page is the column $(p, n)$ and thus
\[ E^{p,n} = \cH^p(R^n f_* F_\bullet) = \cH^{p+n}(\R f_* F_\bullet) \]
showing that the natural map is a quasi-isomorphism.
\end{proof}

{\color{red} WHY CAN WE APPLY THIS TO A REPRESENTABLE MAP OF STACKS}

Let $\E$ be a vector bundle on $X \times \S$ then there exists a short exact sequence,
\begin{center}
\begin{tikzcd}
0 \arrow[r] & \E^{-1} \arrow[r] & \E^0 \arrow[r] & \E \arrow[r] & 0
\end{tikzcd}
\end{center}
such that $R^0 p_* \E^{j-1} = 0$ and $K^j = \R^1 \E^{j-1}$ is locally free for $j = 0, 1$ {\color{red} (Note: I think there is a typo in the indices in the notes)} therefore there is a quasi-isomorphism,
\[ [ K^0 \to K^1] \to \R p_* \E \]

\begin{defn}
If $\E$ is a vector bundle on $X \times \S$ and $[K^0 \to K^1]$ is a two-term complex of locally free sheaves quasi-isomorphic to $\R p_* \E$ then we define the line bundle,
\[ \det{\R p_* \E} := \det{K^0} \ot (\det{K^1})^\vee \] 
\end{defn}

We could also make this definition for any perfect complex on $\S$. The rank of a perfect complex is defined as the alternating sum of the ranks of a representative. This is independent of the choice of representative because localizing at the generic point reduces to the case of finite dimensional vectorspaces for whcih Euler characteristic of a bounded complex coincides with the alternating sum of dimensions. If $\rank{(\R p_* \E)} = 0$ then by definition $\rank{K^0} = \rank{K^1}$ so the dual $(\det{\R p_* \E})^\vee$ is then equipped with a section given by the determinant of the map $K^0 \to K^1$. 

\begin{lemma}
The definition of $\det{\R p_* \E}$ is independent of the choice of representative perfect complex.
\end{lemma}

\begin{lemma}
Given an exact sequence,
\begin{center}
\begin{tikzcd}
0 \arrow[r] & \E' \arrow[r] & \E \arrow[r] & \E'' \arrow[r] & 0
\end{tikzcd}
\end{center}
then,
\[ \det{\R p_* \E} = (\det{\R p_* \E'}) \ot (\det{\R p_* \E''}) \]
\end{lemma}

\begin{proof}
We find a complex $\E^\bullet = [\E^{-1} \to \E^0]$ and $\E^\bullet \to \E$ a quasi-isomorphism as above. Choose $\E'^\bullet$ and $\E''^\bullet$ similarly. In fat, it follows from the construction that these resolutions may be choosen comptabily so that there is a short exact sequence of complexes,
\begin{center}
\begin{tikzcd}
0 \arrow[r] & \E'^\bullet \arrow[r] & \E^\bullet \arrow[r] & \E''^\bullet \arrow[r] & 0 
\end{tikzcd}
\end{center}
compatible with the quasi-isomorphisms to the previous sequence. Taking cohomology, we find a short exact sequence
\begin{center}
\begin{tikzcd}
0 \arrow[r] & K'^\bullet \arrow[r] & K^\bullet \arrow[r] & K''^\bullet \arrow[r] & 0
\end{tikzcd}
\end{center}
of complex of locally free sheaves on $\S$. The result follows from the multiplicativity of determinants in short exact sequences of locally free sheaves,
\[ \det{K^0} = \det{K'^0} \ot \det{K''^0} \quad \det{K^1} = \det{K'^1} \ot \det{K''^1} \]
and therefore,
\[ \det{K^\bullet} := \det{K^0} \to (\det{K^1})^\vee = \det{K'^0} \to \det{K''^0} \ot (\det{K'^1} \ot \det{K''^1})^\vee = \det{K'^\bullet} \ot (\det{K''^\bullet})^\vee \]
\end{proof}


We will apply this construction to the case $\S = \M_X(r,d)$ and $\E = \E_{\univ} \ot q^* V$ where $V$ is a vector bundle on $X$. 

\begin{defn}
For a vector bundle $V$ on $X$ we define the \textit{determinantal line bundle}
\[ \L_V := (\det{\R p_* (\E_{\univ} \ot q^* V)}^\vee \]
on $\M_X(r,d)$ associated to $V$. If $\chi(X, E \ot V) = 0$ for all $[E] \in \M_X(R,d)$
then the rank of $\R p_* (\E_{\univ} \ot q^* V)$ is zero and we define the section
\[ s_V \in \Gamma(\M_X(r,d), \L_V) \]
\end{defn}

\begin{rmk}
Since $\R p_* (\E_{\univ} \ot q^* V)$ is perfect, its construction commutes with base change. In particular, its restrciction to a $k$-point $[E] \in \M_X(r,d)$ is identified with the two-term complex $\R \Gamma(X, \E \ot V)$. If moreover
\[ \chi(X, E \ot V) = h^0(X, E \ot V) - h^1(X, E \ot V) = 0 \]
we see that indeed $\rank{\R p_* (\E_{\univ} \ot q^* V)} = 0$.
\end{rmk}

\begin{rmk}
Note that,
\[ \deg{E \ot V} = (\deg{E}) (\rank{V}) + (\rank{E}) (\deg{V}) \]
and therefore by Riemann-Roch $\chi(X, E \ot V) = 0$ if and only if,
\[ d \rank{V} + r \deg{V} + (1-g) r \rank{V} = 0 \]
or equivalently,
\[ (\ast) \quad \mu(E) \rank{V} + \deg{V} + (1-g) \rank{V} = 0 \]
Notice that this is actually just a numerical condition on $V$ that only depends on the slope of $E$. Therefore, if $(\ast)$ holds then $\chi(X, E \ot V) = 0$ for all vector bundles $E$ with fixed slope $\mu(E)$.
\end{rmk}

The construnction of the determinantal line bundle defines a morphism,
\[ \det : \M_X(r, d) \to \cPic^d_X \]
such that,
\[ (\id_X \times \det)^* \cP = p^* \det{\E_{\univ}} \]
where $\cP$ is the Poincare bundle on $\cPic^d_X \times X$. 

\begin{prop}
The following hold:
\begin{enumerate}
\item the assigment $V \mapsto \L_V$ induces a group homomorphism,
\[ K_0(X) \to \Pic{\M_X(r,d)} \]
meaning the isomorphism class of $\L_V$ depends only on $\rank{V}$ and $\det{V}$
\item if $V$ and $W$ are vector bundles of the same rank and degree then there exists a line bundle $\cN$ on $\cPic_X^d$ such that,
\[ \L_W \cong \L_V \ot \detstar \cN \]
where $\detstar$ is the pullback along the map $\det : \M_X(r,d) \to \cPic^d_X$.
\end{enumerate}
\end{prop}

\begin{proof}
For an exact sequence of vector bundles on $X$,
\begin{center}
\begin{tikzcd}
0 \arrow[r] & V_1 \arrow[r] & V_2 \arrow[r] & V_3 \arrow[r] & 0 
\end{tikzcd}
\end{center}
we get an exact sequence using flatness of $q$ and the fact that $\E_{\univ}$ is a vector bundle,
\begin{center}
\begin{tikzcd}
0 \arrow[r] & q^* V_1 \ot \E_{\univ} \arrow[r] & q^* V_2 \ot \E_{\univ} \arrow[r] & q^* V_3 \ot \E_{\univ} \arrow[r] & 0
\end{tikzcd}
\end{center}
which thus induces an isomorphism,
\[ \L_{V_2} \cong \L_{V_1} \ot \L_{V_3} \]
proving the factorization of the map of monoids through $K_0(X)$,
\begin{center}
\begin{tikzcd}
\mathrm{Vect}_X \arrow[rr] \arrow[rd] & & \Pic{\M_X(r,d)} 
\\
& K_0(X) \arrow[ru, dashed]
\end{tikzcd}
\end{center}
Then the consequence follows from the isomorphism $K_0(X) = \Z \oplus \Pic{X}$ given by $V \mapsto (\rank{V}, \det{V})$. 
\par
To prove the second statment, using the first we may assume that $V = \struct{X}^{\oplus r_V - 1} \oplus \struct{X}(D)$ for some divisor $D$ on $X$ and similarly for $W$. Moreover, writing $D = D_1 - D_2$ as a difference of effective divisors we see that,
\[ [\struct{X}(D)] = [\struct{X}] + [\struct{D_1}] - [\struct{D_2}] \]
in $K_0(X)$. Therefore, the classes of $V$ and $W$ in $K_0(X)$ differ only by the class of a divisor of degree $0$ {\color{red} DOESNT THE PROOF WORK FOR DIVISORS OF ANY DEGREE?}. Thus by the additivity of the derminantal construction, it suffices to prove that,
\[ \det{\R p_* (\E_{\univ} \ot q^* \struct{x})} \cong \detstar \cN' \]
for some line bundle $\cN'$ on $\cPic^d_X$ where $x \in X$ is a closed point. Viewing $\E_x$ and $\cP_x$ as sheaves on $\M_X(r,d)$ and $\cPic^d_X$ respectively as pullback along the sections $\S \to X \times \S$ defined by $x \in X$ then,
\[ \det{\R p_* (\E_{\univ} \ot q^* \struct{x})} = \det{\E_x} = \detstar \cP_x \]
which proves the claim. 
\end{proof}

\begin{defn}
Let $E$ be a vector bundle on $X$. We say that $E$ is \textit{cohomology-free} if $h^0(X, E) = h^1(X, E) = 0$.
\end{defn}

\begin{prop}
If $(\ast)$ holds for $V$ then the following are equivalent,
\begin{enumerate}
\item the section $s_V \in \Gamma(\M_X(r,d), \L_V)$ is nonzero at $[E]$
\item $E \ot V$ is cohomology-free
\end{enumerate}
\end{prop}

\begin{proof}
The morphism $\det{j} : \det{K_0} \to \det{K_1}$ of line bundles is nonzero at the point $[E] \in \M_X(r,d)$ if and only if the morphism of vector bundles $j : K-0 \to K_1$ is an isomorphism at $[E]$. Since $\R p_* (\E_{\univ} \ot q^* V)$ is quasi-isomorphic to $[K_0 \to K_1]$ we see that $j$ is an isomorphism if and only if $(\R p_* (\E_{\univ} \ot q^* V))_{[E]} = 0$ if and only if $h^0(X, E \ot V) = h^1(X, E \ot V) = 0$ by cohomology and base change. 
\end{proof}

\begin{rmk}
While $\L_V$ only depends on $\rank{V}$ and $\det{V}$ the section $s_V$ \textit{does} depend on $V$ itself since it can detect cohomology-freeness as above. We will leverage this fact to produce enough sections of $\L_V$ to establish ampleness. Notice also that, under the assumption that $\chi(X, E \ot V) = 0$ the vanishing of $H^0(X, E \ot V)$ is equivalent to the vanishing of $H^1(X, E \ot V)$.
\end{rmk}

Now we specialize this discussion to the open substack $\M_X^{ss}(r,d) \subset \M_X(r,d)$. The goal is to prove that the good mouli space $M_X^{ss}(r,d)$ is projective. Our candidate ample line bundle is the descent of some $\L_V$.

\begin{prop}
The determinantal line bundle $\L_V$ associated to a vector bundle $V$ satisfying $(\ast)$ descends to $M_X^{ss}(r,d)$ uniquely meaning there exists a unique line bundle $L_V \in \Pic{M_X^{ss}(r,d)}$ such that $\L_V \cong \phi^* L_V$.
\end{prop}

\begin{proof}
By Thorem 3.5(iv), we must show that stabilizers of $\M_X^{ss}(r,d)$ act trivially on the fibers of $\L_V$. By Theorem 3.12(ii), the closed points of $\M_X^{ss}(r,d)$ correspond to polystable bundles,
\[  E = \bigoplus_{j = 1}^n E_j^{\oplus m_j} \]
where the $E_i$ are pairwise nonisomorphic stable bundles. Since $E$ is also semistable the slopes of the $E_i$ must all be equal so $\mu(E_i) = \mu(E) = \frac{d}{r}$. Since these are nonisomorphic stable bundles with the same slope there are no nonzero morphisms between them (we have $\Hom{}{E_i}{E_j} = k \cdot \delta_{ij}$). Since $\End{E_j} = k$ we see that,
\[ \Aut{E} = \GL_{m_1} \times \cdots \times \GL_{m_n} \]
The fiber of $\L_V|_{[E]}$ is identified with
\[ \det{\R \Gamma(X, E \ot V)} = \prod_{i = 1}^n (\det{H^i(X, E \ot V)})^{\ot (-1)^i} \]
An element $(g_1, \dots, g_n) \in \Aut{E}$ act on,
\[ \det{H^i(X, E \ot V)} \cong \bigotimes_{j = 1}^n (\det{H^i(X, E_j \ot V)})^{\ot m_j} \]
by multiplication with,
\[ \prod_{j = 1}^n \det{(g_i)}^{\dim{H^i(X, E_j \ot V)}} \]
and thus on $\L_V|_{[E]}$ by multiplication with,
\[ \prod_{j = 1}^n \det{(g_j)}^{\chi(X, E_j \ot V)} \]
but each $E_j$ has slope $\mu(E_j) = \frac{d}{r}$ so by $(\ast)$ we have $\chi(X, E_j \ot V) = 0$ since it implies vanishing for all $E$ with $\mu(E) = \frac{d}{r}$ and hence the action is trivial.
\end{proof}

\subsubsection{Moduli of Vector Bundles with Fixed Determinant}

Let $L$ be a line bundle on $X$ of degree $d$ corresponding to a closed point of $\cPic^d_X$ represented by a morphism $[L] : \Spec{k} \to \cPic_X^d$ {\color{red} IS EVERY $k$-point closed? in Pic? I Think so since $X/k$ is SMOOTH so we should have separatedness}. Then consider the diagram,
\begin{center}
\begin{tikzcd}
\M_X(r,L) \arrow[d] \arrow[r] \pullback & \M_X(r,d) \arrow[d, "\detstar"] 
\\
\Spec{k} \arrow[r, "L"] & \cPic^d_X
\end{tikzcd}
\end{center}

Explicitly, $\M_X(r,L)$ is the stack of pairs $(\E, \varphi)$ where $\E$ is a vector bundle on $X \times S$ of rank $r$ and degree $d$ fiberwise equipped with an isomorphism $\varphi : \det{\E} \iso L|_{X \times S}$. The condition of constant degree $d$ on fibers is implied by determinant isomorphism which is why $d$ is left out of the notation.

\begin{cor}
For a line bundle $L$ of degree $d$ on $X$, the restriction of the determinantal line bundle $L_V$ to $M_X^{ss}(r, L)$ only depends on the rank and degree of $V$.
\end{cor}

\begin{proof}
We have commuting diagrams,
\begin{center}
\begin{tikzcd}
\M_X^{ss}(r, L) \pullback \arrow[d] \arrow[r] & \M_X^{ss}(r,d) \arrow[d]
\\
\Spec{k} \arrow[r, "L"] & \cPic^d_X
\end{tikzcd}
\quad \text{ and } \quad
\begin{tikzcd}
\M_X^{ss}(r, L) \arrow[d] \arrow[r, hook] & \M_X^{ss}(r,d) \arrow[d]
\\
M_X^{ss}(r, L) \arrow[r, hook] & M_X^{ss}(r, d)
\end{tikzcd}
\end{center}
where in the second square, both vertical arrows are hood moduli space morphisms. If $V$ and $W$ are vector bundles of equal rank and degree on $X$, both satisfying condition (10) then there exists a line bundle $\cN$ on $\cPic^d_X$ such that $\L_W \cong \L_V \ot \detstar \cN$. The left diagram shows that, restriction to $\M_X^{ss}(r,L)$, this isomrophism becomes $\L_W \cong \L_V$. The right diagram now shows that the restriction of $L_V$ and $L_W$ to $M_X^{ss}(r, L)$ become isomrophic after pulling back to $\M_X^{ss}(r, L)$, so the restrictions must be isomorphic by the uniqueness of the descent along the good moduli space morphism $\M_X^{ss}(r, L) \to M_X^{ss}(r, L)$.
\end{proof}

\begin{theorem}[Dr\'{e}zet-Narasimhan]
There exist isomorphisms,
\begin{align*}
\Pic{M_X^{ss}(r,d)} & \cong \Pic{\fPic_X^d} \oplus \Z 
\\
\Pic{M_X^{ss}(r,L)} & \cong \Z 
\end{align*}
In the first line, $\Z$ is {\color{red} WHAT IS IT GENERATED BY?}
In the second line, $\Z$ is generated by the determinantal line bunle $L_V$ where $V$ is chosen to be of minimal rank,
\end{theorem}
\end{document}



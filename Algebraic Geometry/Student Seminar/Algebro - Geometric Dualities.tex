\documentclass[12pt]{article}
\usepackage{hyperref}
\hypersetup{
    colorlinks=true,
    linkcolor=blue,
    filecolor=magenta,      
    urlcolor=blue,
}

\usepackage{import}
\import{../}{AlgGeoCommands}

\begin{document}
\subsection{Examples}

\begin{enumerate}
\item $\{ \text{stone spaces} \} \iff \{ \text{boolean algebras} \}$
\item $\{ \text{affine schemes} \} \iff \{ \text{comm rings} \}^\op$.
\end{enumerate}

Have the notion of a dualizing object in both cases. For example $S = \Spec{A}$ then,
\[ \Hom{\mathrm{Aff}}{\Spec{A}}{\A^1_{\Z}} \cong \Hom{\mathrm{CRing}}{\Z[x]}{A} \cong A \]
The dualizing object for Stone spaces and Boolean algebras is $\mathbb{B} = \{ 0, 1 \}_{\text{disc}}$ because maps to these give clopen subsets. 
\bigskip\\
There are adjunctions, $\mathbf{Top} \to \mathbf{Locales}$ and $\mathbf{Top} \to \mathbf{Topoi}$ but these are not equvalence of categories. 
\bigskip\\
Let $X$ be a topological space. Then $\struct{}(X)$ is the lattice of open subsets with union and intersection. 


\begin{defn}
A poset with all colimits and finite limits is called a frame with distributative property (c.f. distributive lattice).
\end{defn}

\begin{defn}
$\mathrm{Locales} = \mathrm{Frame}^\op$. Then there is a functor $\wt{F} : \mathrm{Top} \to \mathrm{Loc}$ given by $X \mapsto \struct{}(X)$ and $f : X \to Y$ is sent to $f^* : \struct{}(X) \to \struct{}(Y)$ (goes the opposite way in $\mathrm{Loc}$. 
\end{defn}

\begin{rmk}
This is NOT faithful since $X = \{ 0, 1 \}_{\text{indiscrete}}$ then $\wt{F}=U$ treats $X$ and $*$ the same.
\end{rmk}

\begin{rmk}
To to recover $\struct{}(X)$ from a top space $X$. Let $S = \{ 0, 1 \}$ with $\{ 1 \}$ open but $\{ 0 \}$ not this is the Serpinski space. Then $\Hom{}{X}{S} = \struct{}(X)$.  
\end{rmk}

\begin{rmk}
The category $\mathrm{Frame} = \mathrm{Loc}^\op$ is naturally ``algebraic'' meaning the forgetful functor $\mathrm{Frame} \to \mathrm{Set}$ has a left adjoint.
\bigskip\\
However, $\mathrm{Top}^\op$ is not algebraic in any real sense.
\bigskip\\
This functor $u : \mathrm{Top} \to \mathrm{loc}$ is fully faithful in many cases for example on sober spaces.
\end{rmk}

\begin{rmk}
$\mathrm{Locales}$ are ``pointless'' topology 
\end{rmk}

\begin{rmk}
Let $X \in \mathrm{Top}$ then we get a category,
\[ \mathrm{Sh}(X) = \{ \text{sheaves on } X \} \]
We can define what a sheaf is on any local and $\mathrm{Sh}$ factors through $\mathrm{Top} \to \mathrm{Locales}$.  
\end{rmk}

\begin{defn}
A \textit{topos} is a category equivalent to $\mathrm{Sh}(\C)$ where $\C$ is not necessarily localic (it is just some category with a grothendieck topology). 
\end{defn}

\begin{rmk}
An abelian group $A$ is a surjection $\Z^N \onto A$.
\end{rmk}

\begin{defn}
A \textit{logos} $\xi$ is a category that can be presented as a left-exact locaization of a presheaf category:
\[ f : \mathrm{Pr}(\C) \to \xi \]
meaning $f$ admits a fully-faitful right-adjoint and $f$ preserves finite limits where $\C$ is small. 
\end{defn}

\begin{rmk}
The category of abelian groups is locally presentable but not a logos.
\end{rmk}

\subsection{Map of Logoi}

Impose conditions so that they ''look like'' continuous map of topological spaces. Let $f : X \to Y$ be a map of topological spaces then I get included maps between their categories of sheaves $f^* : \mathrm{Sh}(Y) \to \mathrm{Sh}(X)$ and $f_* : \mathrm{Sh}(X) \to \mathrm{Sh}(Y)$ where $f^*$ is left-adjoint to $f_*$ and $f^*$ preserves finite limits. 

\begin{defn}
A map of logoi $f^* : \xi \to \eta$ is cocontinuous and preserves finite limits. 
\end{defn}

\begin{rmk}
Then there is a right adjoint by the adjoint functor theorem for locally presentable categories. 
\end{rmk}

\begin{example}
If $X$ is a topological space. Then $\mathrm{Sh}(X)$ is a logos with  \[ F : \mathrm{Pr}(\struct{}(X)) \to \mathrm{Sh}(X) \]
sheafification. 
\bigskip\\
The same thing also works for locales. Then covers are $\{ U_i \to U \}$ such that $\coprod U_i = U$.
\bigskip\\
Then the category $\mathrm{Set}[X] = \mathrm{Nat}(\mathrm{Fin}, \mathrm{Set})$ is a logos. This is $\mathrm{Pr}(\mathrm{Fin}^\op)$ is the left exact cocompletion of the trivial category. That is, if $\xi$ is a logis, then an object of $\xi$ is exacty a map,
\[ \mathrm{Set}[X] \to \xi \]
of logoi. This is because an object of a category is a map $\{ * \} \to \C$ then we take the completion so that $\{ * \}$ becomes a logos. This is completely analogous to how we get elements of a ring $A$ via,
\[ \Hom{}{\Z[X]}{A} \]
\end{example}

\begin{rmk}
Any logos is a quotient of a free logos. 
\end{rmk}

\begin{defn}
We can define $\mathrm{Topoi} = \mathrm{Logoi}^\op$. Then there is a map $\mathrm{Top} \to \mathrm{Topoi}$ given by $X \mapsto \mathrm{Sh}(X)$ however we usually think of $\mathrm{Sh}(X)$ as the corresponds topos. 
\end{defn}

\begin{rmk}
Topoi have a dualizing object, the topos corresponding to $\mathrm{Set}[X]$. 
\end{rmk}

\end{document}


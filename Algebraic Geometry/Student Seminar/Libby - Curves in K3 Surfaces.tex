\documentclass[12pt]{article}
\usepackage{hyperref}
\hypersetup{
    colorlinks=true,
    linkcolor=blue,
    filecolor=magenta,      
    urlcolor=blue,
}

\usepackage{import}
\import{../}{AlgGeoCommands}

\begin{document}

\newcommand{\Hilb}{\mathrm{Hilb}}

\section{Curves in K3 Surfaces}

\subsection{Cast of Characters}

\begin{enumerate}
\item $S$ is a K3 surface of degree (we will define this) $2g-2$ and $\Pic{S}= \Z$
\item $\mathrm{Hilb}^g(S)$ the Hilbert scheme of $g$-points in $S$
\item $\M^d(S)$ the moduli space of certain sheaves on $S$.
\end{enumerate}

\begin{defn}
A K3 surface $S$ is a smooth proper surface over $\CC$ such that the canonical bundle $\omega_S$ is trivial and $H^1(S, \struct{S}) = 0$.
\end{defn}

\begin{rmk}
The Picard group is discrete (because $H^1(S, \struct{S}) = 0$) but it can have large rank. In characteristic zero it can be up to $20$ in positive characteristic, it can have Picard rank $22$ for supersingular K3 suraces.
\end{rmk}

\begin{defn}
Let $(S, \L)$ be a polarized K3 (given a very ample line bundle) then $\deg{S} = \L \cdot \L$ the intersection pairing. 
\end{defn}

\begin{example}
\begin{enumerate}
\item Any degree $2$ K3 is a double cover of $\P^2$ branched over a sectic
\item Any degree 4 K3 is a quartic surface in $\P^3$
\item Any degree 6 K3 is the intersection of a conic and a cubic in $\P^4$
\item Any degree 8 K3 is the intersection of three quadrics in $\P^5$
\item A degree $2g-2$ K3 embeds into $\P^g$ via $\L$.
\end{enumerate}
\end{example}

\begin{thm}
Every (smooth proper complex) curve of $2 \le g \le 8$ lies in a K3 surface.
\end{thm}

\begin{thm}
Let $C \subset S$ be a smooth irreducible curve of genus $g \ge 2$. If $C' \in |C|$ then,
\[ \mathrm{Cliff}(C') = \mathrm{Cliff}(C) \]
\end{thm}

\begin{defn}
Where,
\[ \mathrm{Cliff}(C) = \min \{ \deg{\L}  - 2 r(\L) \mid h^0(\L) \ge 2 \text{ and } h^1(\L) \ge 2 \]
\end{defn}

\newcommand{\g}{\mathfrak{g}}

\begin{rmk}
If $C$ has an ``exceptional $\g^r_f$ then each $C' \in |C|$ has an ``equally exceptional $\g^r_d$''. 
\end{rmk}

\begin{rmk}
The rank $r$ of a line bundle is $h^0(\L) - 1$.
\end{rmk}

\begin{defn}
A $\g^r_d$ on a curve $C$ is the space of all line bundles $\L$ on $C$ with degree $\alpha$ and rank $r$.
\end{defn}

\begin{thm}[Brill-Noether]
Let $C$ be a general curve in $\M_g$ then,
\[ \dim \g^r_d(C) = \rho(r, g, d) \]
where $\rho(r, g, d)$ is the Brill-Noether number,
\[ \rho(r, g, d) := g - (r+1)(g - d + r) \]
\end{thm}

\begin{thm}[Lazarsfeld '86]
Let $S$ be a K3 surface of Picard rank $1$ then the general curve in $S$ is Brill-Noether general.
\end{thm}

\begin{prop}
Every genus $g$ curve in a $\deg = 2g - 2$ K3 and $\Pic{S} = \Z$ is the intersection of $S$ with an hyperplane $H \subset \P^g$,
\[ H \in (\P^g)^\vee \]
\end{prop}

\begin{defn}
Let $\M^g(S)$ be the moduli space of degree $g$ ideal sheaves on genus $g$ curves in $S$. It is equal to the moduli space of semistable sheaves on $S$ with $r = 0$ and $c_1 = \L$ and $c_2 = 0$.
\end{defn}

\begin{rmk}
If the curve is smooth, these ideal sheaves are just line bundles. However, when the curve has singularities then ideal sheaves are line bundle tensored with the ideal sheaves of the singular points.
\end{rmk}

\begin{prop}
$\M^g(S)$ is proper of dimension $2g$.
\end{prop}

\begin{prop}
$\M^g(S) = \overline{\mathrm{Pic}}^g(U_g)$ where $U_g$ is the universal curve over $\M^g(S)$ (pulled back from the moduli space of curves on $S$).
\end{prop}

\begin{defn}
$\Hilb^g(S)$ is the moduli space of $g$ points in $S$.
\end{defn}

\begin{thm}[Yoshioka '01]
Every component of the moduli space of $S$ stable sheaves on $S$ is deformation equivalent to some $\Hilb^n(S)$.
\end{thm}

\begin{thm}[T]
There exists a birational map $\varphi_d : \M^g(S) \rat \Hilb^g(S)$ for all $g$ such that the codimension of the locus of definition of both $\varphi_g$ and its inverse is exactly $2$. Futhermore, there exists a common smooth resolution such that $\varphi_g$ becomes a morphism. 
\end{thm}

\begin{proof}
When $g = 3$ the degree of $S$ is $4$ so $S \subset \P^3$ is a quartic surface. We define $\varphi_g$ on pairs $(C, \L)$ where $C$ is a smooth curve and $\L$ is a line bundle on $\C$ via,
\[ \varphi_g : (C, \L) \mapsto V(s) \text{ for } s \in H^0(C, \L) \]
this is well-defined when $\L$ has exactly one section which happens by Brill-Noether theory in exactly codimension $2$. For $\varphi_g^{-1}$ take $g = 3$ points in $S$. Assume that $(p,q,r)$ span a plane $H \subset \P^3$ then $H \cap S = C$ is a curve of genus $g = 3$ and let $\L = \struct{C}(p+q+r)$ on $C$. Then let,
\[ \varphi_g^{-1} : (p,q,r) \mapsto (C, \struct{C}(p + q + r)) \]
This is defined when $(p,q,r)$ are in general position. 
\end{proof}

\begin{defn}
We have explict descriptions of the resolutions,
\begin{enumerate}
\item $\wt{\M^g(S)}$ is the moduli space of tuples $(C, \L, s)$ for $(C, \L) \in \M^g(S)$ and $s \in H^0(C, \L)$. 

\item  $\wt{\Hilb^g(S)}$ is rhe moduli space of $(p_1, \dots, p_g, H)$ for $g$ points $p_i \in S$ and a hyperplane $H \subset \P^g$ containing the $p_i$.
\end{enumerate}
\end{defn}

\begin{prop}
$\varphi_g$ extens to an isomorphism $\varphi_g : \wt{\mathcal{M}^g(S)} \iso \wt{\Hilb^g(S)}$.
\end{prop}

\begin{thm}[$\star$]
$D(\M^g(S)) \cong D(\Hilb^g(S))$ because they are birational.
\end{thm}

\subsection{Semi-stability}

\begin{defn}
For a coherent sheaf $E$ let,
\[ \mu(E) = \frac{\rank{E}}{\deg{E}} \]
We say that $E$ is (semi)stable if for every $F \subsetneq E$ then,
\[ \mu(F) < \mu(E) \quad \text{resp.} \quad (\mu(F) \le \mu(E)) \]
\end{defn}

\end{document}

\documentclass[12pt]{article}
\usepackage{hyperref}
\hypersetup{
    colorlinks=true,
    linkcolor=blue,
    filecolor=magenta,      
    urlcolor=blue,
}

\usepackage{import}
\import{../}{AlgGeoCommands}

\begin{document}

\newcommand{\Ad}{\mathrm{Ad}}
\newcommand{\gl}{\mathfrak{gl}}
\newcommand{\Hod}{\mathrm{Hod}}

\section{Introduction}
Our goal is to understand which groups arise as the fundamental group of a smooth complex variety $\pi_1(X(\CC))$. This is a daunting task. To limit our scope very slightly, we study the complex representation theory of $\pi_1(X(\CC))$. 
Let us consider a representation,
\[ \rho : \pi_1(X(\CC)) \to \GL_r(\CC) \]
Orginary Hodge theory provides a description the deformation space of $1$-dimensional representations. Indeed, for $r = 1$ we are asking about characters,
\[ \pi_1(X(\CC)) \to \CC^\times \]
A deformation of this character is given by,
\[ \Hom{}{\pi_1(X(\CC))}{\CC} = H^1(X, \CC) = H^1(X, \struct{X}) \oplus H^0(X, \Omega) \]
Therefore we get a decomposition of the tangent space into two directions: the first represents the deformations of a holomorphic line bundle and the second the deformations of a holomorphic $1$-form. Furthermore, we can say that given a character $\pi_1(X(\CC)) \to \CC^\times$ this defines a local system and hence a line bundle $\L$ and the above decomposition relates to the deformations of $\L$ and of a $1$-form. 
\bigskip\\
Non-abelian Hodge theory generalizes this decomposition for the spaces,
\[ H^1(\pi_1(X(\CC)), \GL_r(\CC)) := \Hom{}{\pi_1(X(\CC))}{\GL_r(\CC)} / \GL_r(\CC) \]
which we interpret as the set $\CC$-points of a moduli space of representations. At a given representation $\rho : \pi_1(X(\CC)) \to \GL_r(\CC)$ defining a local system this defines a vector bundle $E$ and non-abelian Hodge theory gives an analogous decomposition of the tangent space at $\rho$ in terms of $E$,
\[ T_{[\rho]} M_B :=  H^1(X, \End{E}) \oplus H^0(X, \End{E} \ot \Omega^1) \]
The first term again parametrizes the deformations of $E$ while the second term involves $1$-forms.
Notice that unlike the case $r = 1$, this decomposition is not globally a direct sum since the factors depend on $E$ in a nontrivial way. We should think of the right-hand-side as the tangent space to a moduli space of new objects called Higgs bundles which are vector bundles along with an extra $1$-form structure. Then we get a correspondence between representations and Higgs bundles. Often people write schematically to show the connection to Hodge theory,
\[ ``H^1(\pi_1(X(\CC)), \GL_r) =  H^1(X, \GL_r) \oplus H^0(X, \GL_r \ot \Omega^1)'' \]
I would not take this statement or notation very seriously. On tangent spaces this becomes precise as an isomorphism,
\[ H^1(\pi_1(X(\CC)), \Ad{\rho}) = H^1(X, \End{E}) \oplus H^0(X, \End{E} \ot \Omega^1) \]

\begin{rmk}
The identification of $H^1(\pi_1, \Ad{\rho})$ with the tangent space at $[\rho]$ of the character variety is as follows. We deform the map $\rho : \pi \to \GL_r$ via,
\[ \gamma \mapsto (1 + \epsilon \eta(\gamma)) \cdot \rho(\gamma) \]
where $\eta(\gamma) \in \gl_r$ and the second term forms a right-invariant vector field. Then we require,
\[ \gamma_1 \gamma_2 \mapsto [(1 + \epsilon \eta(\gamma_1)) \cdot \rho(\gamma_1)][(1 + \epsilon \eta(\gamma_2)) \cdot \rho(\gamma_2)] = (1 + \epsilon \eta(\gamma_1 \gamma_2)) \cdot \rho(\gamma_1 \gamma_2) \]
and hence,
\[ \eta(\gamma_1 \gamma_2) = \eta(\gamma_1) + \rho(\gamma_1) \eta(\gamma_2) \rho(\gamma_1^{-1}) \]
meaning that $\eta$ is a crossed-hom for the adjoint action. Furthermore, a deformation is induced by the conjugation action if there exists $B \in \gl_r$ such that,
\[ \gamma \mapsto (1 + \epsilon B) \rho(\gamma) (1 + \epsilon B)^{-1} = \rho(\gamma) + \epsilon [B \rho(\gamma) - \rho(\gamma) B] = (1 + \epsilon [B - \rho(\gamma) B \rho(\gamma)^{-1}]) \cdot \rho(\gamma) \]
therefore meaning exactly that $\eta$ is a principal crossed homomorphism for the adjoint action.
\end{rmk}

\section{Higgs Bundles}

In this section we work on a smooth variety $X$.

\begin{defn}
A \textit{Higgs bundle} is a pair $(\E, \phi)$ where $\E$ is a vector bundle and $\phi$ is a $\struct{X}$-linear map,
\[ \phi : \E \to \E \ot_{\struct{X}} \Omega_X^1 \]
such that $\phi \wedge \phi = 0$.
\end{defn}

\begin{rmk}
We should define the notation $\phi \wedge \phi$. Such a linear map is equivalent to a section of,
\[ \phi \in \Gamma(X, \End{\E} \ot_{\struct{X}} \Omega^1_X) \]
We endow $\End{\E} \ot_{\struct{X}} \Omega^\bullet_X$ with the structure of a sheaf of graded $\struct{X}$-algebras under the following operation,
\[ (\varphi_1 \ot \omega_1, \varphi_2 \ot \omega_2) \mapsto (\varphi_1 \circ \varphi_2) \ot (\omega_1 \wedge \omega_2) \]
then extended $\struct{X}$-linearly. This operation is denoted $\wedge$. However, do not let this mislead you into thinking that $\wedge$ is antisymmetric since if $\rank{\E} > 1$ then the composition in $\End{\E}$ is noncommutative. Hence $\phi \wedge \phi = 0$ is a nontrivial condition when $\E$ has rank at least $2$.
\end{rmk}

\newcommand{\inner}[2]{\left< #1, #2 \right>}

\begin{rmk}
We refer to $\phi \wedge \phi = 0$ as the \textit{integrability} condition. This is because we call a flat connection integrable and the condition $\phi \wedge \phi = 0$ is in analogy with the condition for a connection $\nabla$ to be flat, namely $\nabla \circ \nabla = 0$. We will now spell out the relationship of Higgs bundles to flat connections. Recall that a connection $\nabla$ is flat if and only if the associated map, $\T_X \to \End{\E}$
is a map of sheaves of Lie algebras. Similarly, the map $\phi$ satisfies $\phi \wedge \phi = 0$ if and only if the associated holomorphic map $\T_X \to \End{\E}$ is a map of Lie algebras where $\T_X$ is given the trivial Lie bracket i.e. for any local vector fields $\xi_1, \xi_2$ the endomorphism $\phi_{\xi_1}, \phi_{\xi_2}$ commute. Indeed, in local coordinates,
\[ \phi = \sum_i \phi_i \ot \d{z_i} \]
where $\phi_i : \E \to \E$ are endomorphisms. Write $\inner{\xi}{\omega}$ for the canonical pairing between $\T_X$ and $\Omega_X$ for clarity. Then consider,
\begin{align*}
[\phi_{\xi_1}, \phi_{\xi_2}] &= \sum_{ij} [\phi_i, \phi_j] \inner{\xi_1}{\d{z_i}} \inner{\xi_2}{\d{z_j}} = \sum_{ij} \phi_i \circ \phi_j \left( \inner{\xi_1}{\d{z_i}} \inner{\xi_2}{\d{z_j}} - \inner{\xi_1}{\d{z_j}} \inner{\xi_2}{\d{z_i}} \right) 
\\
& = \sum_{ij} \phi_i \circ \phi_j \inner{\xi_1 \wedge \xi_2}{\d{z_i} \wedge \d{z_j}} = \inner{\xi_1 \wedge \xi_2}{\phi \wedge \phi}
\end{align*}
and hence the bracket vanishes if and only if $\phi \wedge \phi = 0$.
\end{rmk}

There are a number of ways to motivate the definition of a Higgs bundle. My favorite is to think of them as degenerations of a flat connection where we send the nonlinear part to zero. In order to make this precise we introduce the notion of a $t$-connection.

\begin{defn}
Let $X$ be an $S$-scheme. Let $\E$ be a coherent sheaf on $X$. A $t$-\textit{connection} on $\E$ over $X/S$ is a triple $(t, \E, \nabla)$ where $t : X \to \A^1_S$ is a global function and $\nabla$ is a $S$-linear map,
\[ \nabla : \E \to \E \ot \Omega^1_X \]
satisfying the $t$-scaled Leibniz law,
\[ \nabla(f s) = t \d{f} \ot s + f \nabla s \] 
\end{defn}

\begin{rmk}
Notice that if $t = 0$ then $\nabla$ is $\struct{X}$-linear.
\end{rmk}

\begin{defn}
There is a natural extension of $\nabla$ to,
\[ \nabla_p : \E \ot_{\struct{X}} \Omega^p_X \to \E \ot_{\struct{X}} \Omega^{p+1}_X \]
defined on pure tensors as follows
\[ \nabla_p(s \ot \omega) = t s \ot \d{\omega} + (-1)^p \nabla s \wedge \omega \]
Then we define the curvature of $\nabla$,
\[ \omega_{\nabla} = \nabla_1 \circ \nabla \]
A straightforward calculation shows that,
\[ \omega_{\nabla} : \E \to \E \ot \Omega^2_X \]
is $\struct{X}$-linear. We say that $\nabla$ is \textit{flat} or \textit{integrable} if $\omega_{\nabla} = 0$. In this case $\nabla$ is a differential meaning the de Rham complex,
\[ 0 \to \E \xrightarrow{\nabla} \E \ot \Omega^1_X \xrightarrow{\nabla_1} \E \ot \Omega_X^2 \xrightarrow{\nabla_2} \E \ot \Omega_X^3 \to \cdots \]
is actually a complex.
\end{defn}

\begin{rmk}
Notice in the case that $t = 0$ we saw $\nabla$ is $\struct{X}$-linear. Call $\phi := \nabla$. Then notice,
\[ \omega_{\nabla}(s) = \nabla_1 \circ \nabla(s) = \nabla_1 \left( \sum_i s_i \ot \omega_i \right) = - \phi(s_i) \wedge \omega_i = - (\phi \wedge \phi)(s) \] 
where,
\[ \phi(s) = \sum_i s_i \ot \omega_i \]
Therefore, $\phi \wedge \phi = 0$ if and only if the $t$-connection $\nabla$ is flat. 
\end{rmk}

The previous calculation shows that a $t$-connection is a gadget that interpolates between a flat connection on $X_1$ over $t = 1$ and a Higgs bundle on $X_0$ over $t = 0$. If we take the constant $t$-scheme $X \times \A^1 \to \A^1$ and a constant coherent sheaf $\pi_1^* \E$ then a $t$-connection is literally just linearly interpolating between a connection on $\E$ and a Higgs bundle structure on $\E$. This picture is completely functorial so we get a universal interpretation,
\begin{center}
\begin{tikzcd}
\M_{\text{Dol}}(X) \arrow[d] \arrow[r, hook] & \M_{\text{Hod}}(X) \arrow[d] & \M_{\text{dR}}(X) \arrow[l, hook] \arrow[d]
\\
\{ t = 0 \} \arrow[r, hook] & \A^1 & \{ t = 1 \} \arrow[l, hook] 
\end{tikzcd}
\end{center}
so we get a moduli space $\M_{\Hod}(X)$ of flat $t$-connections $(t, \E, \nabla)$ with a $\Gm$-equivariant map,
\[ \M_{\text{Hod}}(X) \to \A^1 \quad \quad (t, \E, \nabla) \mapsto t \]
where the $\Gm$-acts via,
\[ \lambda \cdot (t, \E, \nabla) = (\lambda t, \E, \lambda \nabla) \]

\section{Variation of Hodge Structures}

We have seen that a Higgs bundle is morally a ``linearized connection''. These arise from the somewhat artificial construction of a $t$-connection. However, they also arise from a much more natural object: a polarized $\CC$-variation of Hodge structures. 

\begin{defn}
A \textit{complex variation of Hodge structures of weight} $k$ over a algebraic (or analytic) variety $S$ is the data $(V, V^{p,q})$ of,
\begin{enumerate}
\item a $\CC$-local system $V$
\item analytic sub-bundles $V^{p,q}$ of $V \ot \struct{S}$ with $p + q = k$
\end{enumerate}
such that,
\begin{enumerate}
\item for each $s \in S$,
\[ V_s = \bigoplus_{p+q=k} V^{p,q}_s \]
\item the sub-bundles,
\[ F^p := \bigoplus_{i \ge p} V^{i, k - i} \]
are holomorphic and,
\[ \ol{F}^q := \bigoplus_{i \le k-q} V^{i, k-i} \]
is anti-holomorphic
\item let $\nabla$ be the flat connection induced on $V \ot \struct{S}$. Then,
\[ \nabla_X F^p \subset F^{p-1} \quad \nabla_X \ol{F}^q \subset \ol{F}^{q-1} \]
for any local vector-field $X$. 
\end{enumerate}
\end{defn}

\begin{rmk}
In the case of a $\Z$ or $\Q$ or $\RR$-variation of Hodge structures where we have a real structure on the vectorspace the data of $F^p$ alone is sufficient to determine the Hodge structure since $\ol{F}^p$ is the complex conjugate and $V^{p,q} = F^p \cap \ol{F}^q$. However, without the $\RR$ structure we cannot recover $\ol{F}^q$ from $\ol{F}^p$.
\end{rmk}

We abuse notation by letting $V$ refer to both the underlying local system and the holomorphic vector bundle $V \ot \struct{S}$ and the underlying $C^{\infty}$-vector bundle.

\begin{defn}
A \textit{polarization} of a complex variation of Hodge structure $(V, V^{p,q})$ is a Hermitian form $\psi : V \ot V \to \CC$ which is parallel for the flat connection $\nabla$ satisfying the so-called Riemann bilinear relations:
\begin{enumerate}
\item the decomposition,
\[ V = \bigoplus_{p+q = k} V^{p,q} \]
is orthogonal for $\psi$
\item $(-1)^{p-k} \psi$ is positive definite on $V^{p,q}$
\end{enumerate}
\end{defn}

\begin{rmk}
Note that usually a polarization is given in terms of a bilinear form $Q$ such that $\psi(u,v) = Q(u,h(i) \bar{v})$ a definite Hermitian form on each $V^{p,q}$ for which the Hodge decomposition is orthogonal.
\end{rmk}

\begin{prop}
Let $(V, V^{p,q}, \nabla, \psi)$ be a complex polarized variation of Hodge structures. Then, the maps,
\[ \nabla : F^p / F^{p+1} \to (F^{p-1} / F^{p}) \ot \Omega^1_S \] 
are linear because if $f$ is a local section of $F^p$ then $\nabla(sf) = s \nabla f + f \ot \d{s}$ but $f \ot \d{s}$ is a section of $F^p \ot \Omega_S^1$ and hence zero in the quotient. Furthermore, since $F^p / F^{p+1} = V^{p,k-p}$ we get linear maps,
\[ \theta^p : V^{p,k-p} \to V^{p-1,k-p+1} \ot \Omega^1_S \]
which define a Higgs field $\theta$ on $V$. This defines a \textit{graded Higgs bundle} 
\end{prop}

\begin{rmk}
The vector bundles $E^{p,q} = F^p / F^{p+q}$ are holomorphic. However, the identification $V^{p,q} = F^p / F^{p+q}$ is not holomorphic in the sense that $F^p / F^{p+q} \embed V$ is not necessarily a map of holomorphic vector bundles.
\end{rmk}


\begin{defn}
A graded Higgs bundle $\big( \bigoplus\limits_{i+j = k} E^{i,j}, \theta \big)$ is a Higgs bundle on a graded holomorphic vector bundle,
\[ E = \bigoplus_{i + j = k} E^{i,j} \]
where the Higgs field $\theta$ is the sum of linear maps,
\[ \theta^i : E^{i,j} \to E^{i-1,j+1} \ot \Omega^1_S \]
\end{defn}


One of the main theorems of Hitchin and Simpson shows that this construction can be reversed. The graded Higgs bundle contains all of the information in the polarized variation of Hodge structures. 

\begin{theorem}[Simpson, Hitchin]
The functor,
\[ \left\{ \text{complex polarized VHS } (V, V^{p,q}, \nabla, \psi) \right\} \xrightarrow{\mathrm{Gr}_{F^\bullet}} \left\{ \text{graded Higgs bundles } \big( \bigoplus_{i+j = k} E^{i,j}, \theta \big) \right\} \]
is an equivalence of categories onto the subcategory of polystable graded Higgs bundles with trivial chern classes.
\end{theorem}

\section{Stability Conditions}

We fix an ample class $H$ on $X$ (or as Simpson does, fix a Kahler class on $X$).

\begin{defn}
The slope of a torsion-free coherent sheaf $\F$ is,
\[ \mu(\F) := \frac{c_1(\F) \cdot H^{n-1}}{\rank{\F}} \]
\end{defn}

\begin{defn}
A subsheaf $\F \subset E$ of a Higgs bundle $(E, \phi$) is $\phi$-\textit{stable} if $\phi(\F) \subset \F \ot \Omega^1_X$.
\end{defn}

\begin{defn}
We say that a Higgs bundle $(E, \psi)$ is,
\begin{enumerate}
\item (semi)-stable if for all $\phi$-stable coherent subsheaf $0 \subsetneq \F \subsetneq E$ then $\mu(\F) < \mu(E)$ (resp. $\le$)
\item polystable if it is a direct sum of stable Higgs bundles of the same slope.
\end{enumerate}
\end{defn}

\section{Moduli Spaces}

\newcommand{\Dol}{\mathrm{Dol}}

There are three Moduli problems we are concerned with:
\begin{enumerate}
\item  $\M_B(X)$ the moduli of local systems (representations of $\pi_1(X(\CC))$)
\item $\M_{\dR}(X)$ the moduli of vector bundles with flat connection
\item $\M_{\Dol}(X)$ the moduli of Higgs bundles 
\end{enumerate}

The first two are related by the Riemann-Hilbert correspondence. We saw that there was a degeneration $\M_{\dR} \spto \M_{\Dol}$ so we should expect that their underlying topological spaces might be related. It will be more convenient to work in terms of the GIT coarse space (the corresponding good moduli spaces of these moduli stacks). 

\begin{theorem}
There exist homeomorphisms (diffeomorphisms over the smooth loci) of the coarse moduli spaces,
\[ M_{\Dol}^{ss} \cong_h M_B^{ss} \cong M_{\dR} \]
where $M_B^{ss}$ coarse space parametrizing semi-simple representations and $M_{\Dol}^{ss}$ is the coarse space of semi-stable Higgs bundles. Moreover, this restricts to homeomorphisms,
\[ M_{\Dol}^{s} \cong_h M_B^{+} \cong M_{\dR}^{+} \]
where $M_B^{+}$ is the fine character variety of irreducible representations and $M_{\Dol}^{s}$ is the fine moduli space of stable Higgs bundles\footnote{These moduli spaces are fine since the stable/irreducible objects have non-scalar automorphisms} and $M_{\dR}^{+}$ is the fine moduli space of irreducible vector bundles with flat connections.
\end{theorem}

\begin{rmk}
The second map $M_B \cong M_{\dR}$ is an isomorphism of complex analytic spaces (it is not algebraic) via the Riemann-Hilbert correspondence. However, the maps $M_{\Dol}^{ss} \to M_B^{ss}$ are not. Transporting the complex structure on $M_B^{ss}$ to $M_{\Dol}^{ss}$ gives an additional complex structure $J$ in addition to the natural complex structure $I$ on $M_{\Dol}^{ss}$. These are not equal. In fact, we will show that $IJ = -JI$. Therefore if $K = IJ$ then $I,J,K$ form a representation of the quaternions and hence $M_{\Dol}^{ss}$ becomes a hyperkahler manifold.
\end{rmk}

\subsection{The Correspondence}


\newcommand{\dbar}{\bar{\partial}}
\newcommand{\hol}{\mathrm{hol}}

We want to describe a map $M_{\dR} \to M_{\Dol}$. 
\bigskip\\
Let $(E, \nabla)$ be a flat vector bundle. We choose a hermitian metric $h$ on $E$. Then there is a unique decomposition,
\[ \nabla = \nabla_h + \Psi \]
where $\nabla_h$ satisfies $\nabla_h h = 0$ and $\Psi$ is self-adjoint for $h$. Then we get a Higgs bundle,
\[ (E, \dbar_E = \nabla_h^{0,1}, \Phi = \Psi^{1,0}) \]
as long as $(\nabla_h^{0,1})^2 = 0$  and $\dbar_E \Phi = 0$. These both follow from the flatness $\nabla^2 = 0$. Indeed,
\[ F_{\nabla_h} + \nabla \Psi + \Psi \wedge \Psi = 0 \] 
\bigskip\\
Going backward, if we start with $(E, \dbar_E, \phi)$ then we need to recover $\nabla = \nabla_h + \Psi$. Let's assume we can recover the metric $h$ on $E$ that we used in the other direction. Then $\nabla_h$ is just the Chern connection and $\Psi^{0,1} = \psi$ but $\Psi$ is self-adjoint so $\Psi = \psi + \psi^{\dagger}_h$. Therefore,
\[ \nabla = \nabla_h + \psi + \psi^{\dagger}_h \]
Then the integrability condition becomes,
\[ F_{\nabla} = F_{\nabla_h} + [\psi, \psi^{\dagger}_h] = 0 \]
This is called the \textit{Hitchin equation}. We will need to show that there exists a metric that solves this equation. 

\section{$G$-Higgs Bundles}

What we have defined so far is a $\GL_r$-Higgs bundle since its underlying structure is a vector bundle i.e. a principle $\GL_r$-vector bundle. We can generalize this to any connected reductive group $G$. 


\newcommand{\ad}{\mathrm{ad}}
\newcommand{\g}{\mathfrak{g}}

\begin{defn}
Let $P$ be a principal $G$-bundle and $\ad(P) = P \times_{\ad} \g$ the adjoint bundle. Define an operation $[-,-]$ endowing $\ad(P) \ot \Omega^{\bullet}_X$ with the structure of a sheaf of graded $\struct{X}$-algebras by,
\[ (\xi_1 \ot \omega_1, \xi_2 \ot \omega_2) = [\xi_1, \xi_2] \ot (\omega_1 \wedge \omega_2)  \]
\end{defn}

\begin{defn}
A $G$-Higgs bundle is a pair $(P, \Phi)$ of a principal $G$-bundle and a section,
\[ \Phi \in H^0(X, \ad(P) \ot \Omega_X^1) \quad \text{ such that } \quad [\Phi, \Phi] = 0 \]
\end{defn}

\section{Harmonic Bundles}

We should first disambiguate ''vector bundle with flat connection''. On a holomorphic vector bundle $E$ we could have two notions of a flat connection: a holomorphic connection which is a map,
\[ \E_{\hol} \to \E_{\hol} \ot (\Omega^1_X)_{\hol} \]
satisfying the flatness condition,
or a connection on the underlying smooth vector bundle,
\[ \E_{C^{\infty}} \to \E_{C^{\infty}} \ot (\Omega^1_X)_{C^{\infty}} \]
which satisfies an identical flatness condition. It is clear that the former induces the later. Much more surprisingly, this is a bijection between these two types of connections (using flatness in an essential way here!). 
\bigskip\\
Indeed if $(E, \nabla)$ is a vector bundle with a flat connection we first claim that $E$ has a canonical holomorphic structure for which $\nabla$ is compatible. Indeed, a holomorphic structure on $E$ is uniquely determined by an operator $\dbar_E : \cA^0(E) \to \cA^{0,1}(E)$ satisfying the Leibniz law and $(\dbar_E)^2 = 0$ so we set $\dbar_E = \nabla^{0,1}$. Since the curvature of $\nabla$ vanishes,
\[ (\nabla^{1,0} + \nabla^{0,1})^2 = 0 \implies (\nabla^{1,0})^2 = (\nabla^{0,1})^2 = \nabla^{0,1} \circ \nabla^{1,0} + \nabla^{1,0} \circ \nabla^{0,1} = 0 \] 
Therefore setting $\dbar_E = \nabla^{0,1}$ defines a holomorphic structure. Then I claim that $\nabla^{1,0}$ is a holomorphic connection. Indeed, we just need to check that if $s$ is a holomorphic section of $E$ then $\nabla^{1,0} s$ is a holomorphic section of $\Omega_X^1 \ot E$. However, $\nabla^{1,0} \circ \nabla^{0,1} + \nabla^{0,1} \circ \nabla^{1,0} = 0$ so applying this operator to $s$ and recalling that $\nabla^{0,1} s = 0$ by definition of a holomorphic section we get,
\[ \nabla^{0,1} \circ \nabla^{1,0} s = 0 \]
proving that $\nabla^{1,0} s$ is holomorphic. Thus $(E, \nabla)$ is equivalent to the data of a holomorphic vector bundle with a flat holomorphic connection.

Note that we can package the data (holomorphic structure and Higgs field) of a Higgs bundle into a triple $(E, \dbar_E, \phi)$ such that,
\[ (\dbar_E + \phi)^2 = 0 \]

\begin{defn}
A \textit{hermitian bundle} is a pair $(E, h)$ of a complex (smooth) vector-bundle $E$ and a bilinear map $h : E \oplus E \to \CC$ which is fiberwise a hermitian metric.
\end{defn}

\begin{defn}
Let $(E, h)$ be a hermitian bundle. We say that a connection $\nabla$ on $E$ is \textit{hermitian} or \textit{unitary} if $\nabla h = 0$. Explicitly, this means for any two local sections $s_1, s_2$ of $E$,
\[ h(\nabla s_1, s_2) + h(s_1, \nabla s_2) = 0 \] 
\end{defn}

\begin{theorem}
If $(E, h)$ is a holomorphic hermitian bundle there is a unique hermitian connection $\nabla$ compatible with the complex structure in the sense that $\nabla^{0,1} = \bar{\partial}_E$. We call $\nabla$ the \textit{Chern connection}.
\end{theorem}



\begin{theorem}
\begin{enumerate}
\item A flat bundle $V$ has a harmonic metric if and only if it is semi-simple
\item A Higgs bundle $E$ has a Hermitian-Yang-Mills metric if and only if it is polystable. 
\item the Hermitian-Yang-Mills metric is harmonic if and only if $\ch_1(E) \cdot [\omega]^{\dim{X}-1} = 0$ and $\ch_2(E) \cdot [\omega]^{\dim{X}-2} = 0$.
\end{enumerate}
\end{theorem}

Let $(E, \phi)$ be a Higgs bundle. We first choose a hermitian metric $h$ on $E$ which induces a Chern connection $\nabla_h$. The Chern connection will likely not be flat so we modify it by the Higgs field, $\nabla' := \nabla_h + \phi$. Computing the curvature gives,
\[ F_{\nabla'} = F_{\nabla_h} + \nabla \phi + \phi \wedge \phi \]

\section{Moduli Spaces of Vector Bundles and Higgs Bundles}

\newcommand{\Isom}{\mathrm{Isom}}

From now on let $S$ be a base scheme (the only important case this quarter will be $S = \Spec{\CC}$ but we may talk about $S = \Spec{\Q_p}$ or $\Spec{\FF_p}$ next quarter).

\begin{defn}
Let $X$ be an $S$-scheme and $r$ an integer. Then let $\M(X, r)$ be the stack fibered in groupoids over $\Sch_S$ with,
\begin{enumerate}
\item objects are pairs $(T, \E)$ where $T$ is an $S$-scheme and $\E$ is a vector bundle over $X_T$ of rank $r$ flat over $T$ (flatness is automatic if $X \to S$ is flat)
\item morphisms are pairs $(f, \alpha) : (T, \E) \to (T', \E')$ of a morphism $f : T \to T'$ over $S$ and an isomorphism $\alpha : (\id \times f)^* \E' \iso \E$.
\end{enumerate} 
\end{defn}

\begin{theorem}
If $X \to S$ is proper and of finite presentation then $\M(X, r)$ is an algebraic stack locally of finite presentation over $S$ with affine diagonal of finite presentation.
\end{theorem}

\renewcommand{\Coh}{\mathbf{Coh}}

\begin{proof}
In the stacks project the stack $\Coh_{X/S}$ of all coherent sheaves is constructed for $X \to S$ finitely presented. This is the stack fibered in groupoids over $\Sch_S$ with,
\begin{enumerate}
\item objects are pairs $(T, \F)$ where $T$ is an $S$-scheme and $\F$ is a quasi-coherent $\struct{X_T}$-module of finite presentation, flat over $T$, with support proper over $T$
\item morphism are pairs $(f, \alpha) : (T, \F) \to (T', \F')$ with $f : T \to T'$ and $\alpha : (\id \times f)^* \F' \iso \F$.
\end{enumerate}
A few remarks: we just require $\F$ to be of finite presentation rather than coherent (these are equivalent over noetherian test schemes) because it is preserved under base change. The properness condition is necessary for the automorphisms to be finite-dimensional\footnote{For example the stack $\X$ of all vector bundles of rank $r$ on $\A^1$ does not form an algebraic stack because it is limit-preserving since the a vector bundle on $(\ilim T_i) \times \A^1$ in an inverse system of affine schemes $\{ T_i \}$ is defined over some $T_i \times \A^1$ and the morphisms are given by the colimits by spreading out. Therefore, by Tag~\chref{https://stacks.math.columbia.edu/tag/0CXI}{0CXI} if $\X$ were algebraic then $\Delta_{\X}$ would be representable by locally finite type algebraic spaces hence the isomorphism schemes would be finite dimensional. However, the infinitesimal automorphisms of $\struct{X}$ on $\A^1$ is infinite dimensional.}
The stacks project (e.g. Tag~\chref{https://stacks.math.columbia.edu/tag/09DS}{09DS}) checks Artin's axioms to prove that $\Coh_{X/S}$ is an algebraic stack over $S$ if $X \to S$ is separated and of finite presentation.
\bigskip\\
Now $\M(X, r) \subset \Coh_{X/S}$ is an open sub-stack when $X \to S$ is additionally universally closed. Indeed, since $\F$ is flat over $T$ local freeness can be checked on fibers hence we just need to show that the locus on $T$ where $\F$ is locally free on the fiber is open and indeed since the locus where $\F$ on $X_T \to T$ is a vector bundle of rank $r$ is open on $X_T$ the locus on $T$ is open since the $X_T \to T$ is closed.
\bigskip\\
Moreover, $\Coh_{X/S} \to S$ is locally of finite presentation because it is limit preserving (one of Artin's axioms) which one shows via spreading out vector bundles. The diagonal is affine and finitely presented because the Isom space,
\[ \Isom(\F, \G) : T/S \mapsto \Isom_{T}(\F_T, \G_T) \]
between finitely presented $\struct{X}$-modules is representable by a scheme affine and of finite presentation over $T$ (see Tag~\chref{https://stacks.math.columbia.edu/tag/08K9}{08K9}).
\end{proof}

Now we want to discuss the moduli of vector bundles with connections. We need the notion of a relative connection.

\begin{defn}
Let $f : X \to S$ be a morphism of schemes. Then an $S$-connection on a vector bundle $\E$ over $X$ is an $f^{-1} \struct{S}$-linear map,
\[ \nabla : \E \to \Omega^1_{X/S} \ot \E \]
satisfying the Leibniz rule. The $f^{-1} \struct{S}$-linearity is automatic from the form of the Leibniz law and the definition of $\Omega_{X/S}$.
\end{defn}

\begin{defn}
Let $\M_{\dR}(X, r)$ be the stack fibered in groupoids over $\Sch_S$ with,
\begin{enumerate}
\item objects are $(T, \E, \nabla)$ where $T$ is an $S$-scheme and $\E$ is a vector bundle over $X_T$ of rank $r$ flat over $T$ and $\nabla$ is a flat $T$-connection on $\E$
\item morphisms $(f, \alpha) : (T, \E, \nabla) \to (T', \E', \nabla')$ are pairs $f : T \to T'$ and $\alpha : (\id \times f)^* \E' \iso \E$ an isomorphism such that,
\begin{center}
\begin{tikzcd}
(\id \times f)^* \E' \arrow[d, "\alpha"] \arrow[r, "\nabla'"] &  (\id \times f)^* \E' \ot \Omega_{X_T/T}^1 \arrow[d, "\alpha \ot \id"] 
\\
\E \arrow[r, "\nabla"] & \E \ot \Omega^1_{X_T/T}
\end{tikzcd}
\end{center}
commutes\footnote{Given a morphism $f : X \to Y$ and a vector bundle $(\E, \nabla)$ with connection on $Y$ there is a pullback connection $(f^* \E, f^* \nabla)$ defined as the unique connection such that $(f^* \nabla) (f^* s) = f^* (\nabla s)$ }
\end{enumerate}
\end{defn}

\renewcommand{\X}{\mathcal{X}}
\newcommand{\Y}{\mathcal{Y}}

\begin{lemma}
Let $f : \X \to \Y$ be a morphism of algebraic stacks over $S$. Suppose that $\cP$ is a property of morphisms of algebraic stacks such that,
\begin{enumerate}
\item $\cP$ is preserved under base change and composition
\item $\cP$ holds for $f$
\item $\cP$ holds for $\Delta_{\Y/S}$
\end{enumerate}
then $\cP$ holds for $\Delta_{\X/S}$.
\end{lemma}

\begin{proof}
Consider the diagram,
\begin{center}
\begin{tikzcd}
\X \arrow[d, "f"] \arrow[r, "\Delta_{\X/S}"] & \X \times_S \X \arrow[d, "f \times f"]
\\
\Y \arrow[r, "\Delta_{\Y/S}"] & \Y \times_S \Y
\end{tikzcd}
\end{center}
Then factoring $\Delta_{\X/S}$ via the graph we see that $\Delta_{\X/S}$ is the composition of the base change of the diagonal of $f \times f$ and the base change of $(\Delta_{\Y/S} \circ f)$ all of which satisfy $\cP$.
\end{proof}

\begin{prop}
Let $X \to S$ be proper and of finite presentation with $\Omega^1_{X/S}$ flat over $S$ (e.g. if $X \to S$ is smooth). Then $\M_{\dR}(X, r)$ is an algebraic stack lfp over $S$ with affine diagonal.
\end{prop}

\begin{proof}
We will show that the forgetful map $\M_{\dR}(X, r) \to \M(X, r)$ is affine and finite presentation (in particular representable). Then any smooth presentation of $\M(X,r)$ gives a smooth presentation of $\M_{\dR}(X, r)$ by base change. Then we apply the lemma with $\cP$ being ``affine and locally of finite presentation'' to the map $\M_{\dR}(X, r) \to \M(X, r)$. 
\bigskip\\
We need to show that for any test scheme $T \to \M(X, r)$ that $\M_{\dR}(X, r) \times_{\M(X, r)} T \to T$ is affine and of finite presentation. Indeed the test map fixes a vector bundle $\E$ on $X_T$ of rank $r$ and we need to consider the functor,
\[ T'/T \mapsto \{ \nabla \text{ integrable } T'\text{-connection on } \E_{T'} \} \]
To understand this, recall that $T'$-connections on $\E_{T'}$ correspond to sections of the sequence,
\begin{center}
\begin{tikzcd}
0 \arrow[r] & \E_{T'} \ot \Omega^1_{X_T'/T'} \arrow[r] & \cP^1_{X/S}(\E)_{T'} \arrow[r] & \E_{T'} \arrow[r] & 0
\end{tikzcd}
\end{center}
where $\nabla = j - \sigma$. Therefore, the integrable connections correspond to the pullback along,
\[ \Hom{T'}{\E_{T'}}{\cP^1_{X/S}(\E)_{T'}} \to \Hom{T'}{\E_{T'}}{\E_{T'} \ot \Omega^2_{X'_T/T'}} \times \Hom{T'}{\E_{T'}}{\E_{T'}} \]
of the section $(0, \id)$ (note the curvature map is nonlinear). Furthermore, these sheaves are all pulled back along $X_{T'} \to X_T$. Since the pullback of a section induces the pullback connection and $f^* F_{\nabla} = F_{f^* \nabla}$ so this is a map of sheaves. Then the functor is the pullback along,
\[ \Hom{}{\E}{\cP^1_{X/S}(\E)_{T}} \to \Hom{}{\E}{\E \ot \Omega^2_{X_T/T}} \times \Hom{}{\E}{\E} \]
but since these sheaves are finitely presented (since $X \to S$ is finitely presented), flat over $T$, and with proper support over $T$ these functors are representable by schemes (Tag~\chref{https://stacks.math.columbia.edu/tag/08K6}{08K6}) affine and finitely-presented over $T$ hence so is the pullback of $T$ along this map since the map is also affine and finitely presented.
\end{proof}

\begin{rmk}
Taking moduli of relative connections is necessary if we want $\M_{\dR}$ to even be a stack! Indeed, the functor of all connections on $\E$ does not satisfy the fppf sheaf condition. Indeed this is for the same reason that $T \mapsto H^0(T, \Omega_T)$ is not a sheaf in the fppf topology in positive characteristic (it is a sheaf for the \etale and hence smooth topology). Interestingly in characteristic zero it is a sheaf in the $h$-topology restricted to smooth schemes but not in general. However, more problematically $T \mapsto H^0(T, \Omega_T)$ is never representable. Indeed, the functor is limit preserving so its representing object would have to be finite type but it has exactly one closed point (lets work with the case $S = \Spec{\CC}$) but it has nontrivial maps from $\A^1$ so it cannot be represented by an Artin ring. 
\end{rmk}

Now we consider the moduli problem of $t$-connections. There are two ways one might approach this that boil down to the question: should the $k$-points of this moduli space allow for $t$ to be a function or should it take on a constant value e.g. if $X$ is a $t$-scheme over $k$ we could imagine a $k$-point corresponds to a $t$-connection over $X$. This is not what we want because it does equip our moduli space with a fibration over $\A^1$. Instead, we will let the $t$-scheme structure arise from the test scheme so the stack will be fibered over the category of $t$-schemes i.e. $\Sch_{\A^1_S}$. Categorically, this amounts to nothing more than changing our base scheme from $S$ to $\A^1_S$. Then we can define the moduli problem.


\begin{defn}
Let $\M_{\Hod}(X, r)$ be the stack fibered in groupoids over $\Sch_{\A^1_S}$ with,
\begin{enumerate}
\item objects are $(T, \E, \nabla)$ where $T$ is an $\A^1_{S}$-scheme and $\E$ is a vector bundle over $X_T$ of rank $r$ flat over $T$ and $\nabla$ is a flat $t$-connection on $\E$ relative to $T$
\item morphisms $(f, \alpha) : (T, \E, \nabla) \to (T', \E', \nabla')$ are pairs $f : T \to T'$ and $\alpha : (\id \times f)^* \E' \iso \E$ an isomorphism such that,
\begin{center}
\begin{tikzcd}
(\id \times f)^* \E' \arrow[d, "\alpha"] \arrow[r, "\nabla'"] &  (\id \times f)^* \E' \ot \Omega_{X_T/T}^1 \arrow[d, "\alpha \ot \id"] 
\\
\E \arrow[r, "\nabla"] & \E \ot \Omega^1_{X_T/T}
\end{tikzcd}
\end{center}
commutes\footnote{Given a morphism $f : X \to Y$ and a vector bundle $(\E, \nabla)$ with connection on $Y$ there is a pullback connection $(f^* \E, f^* \nabla)$ defined as the unique connection such that $(f^* \nabla) (f^* s) = f^* (\nabla s)$ }
\end{enumerate}
\end{defn}

\begin{rmk}
Because the $t$-parameter arises from a map $T \to \A^1$ and the $t$-connection is relative to $T$ we treat $t$ like a constant when applying $\nabla$ there is no $\d{t}$ term that appears.
\end{rmk}

\begin{prop}
Let $X \to S$ be proper and of finite presentation with $\Omega^1_{X/S}$ flat over $S$ (e.g. if $X \to S$ is smooth). Then $\M_{\Hod}(X, r)$ is an algebraic stack lfp over $S$ with affine diagonal.
\end{prop}

\begin{proof}
As before, we will show that the forgetful map $\M_{\Hod}(X, r) \to \M(X, r) \times \A^1_S$ is affine and finite presentation (in particular representable). 
We need to show that for any test $t$-scheme $T \to \M(X, r)$ that $\M_{\dR}(X, r) \times_{\M(X, r)} T \to T$ is affine and of finite presentation. Indeed this fixes the data $(T, t, \E)$ where $\E$ is a vector bundle on $X_T$ of rank $r$ and $t : T \to \A^1_S$. We need to show that the functor,
\[ T'/T \mapsto \{ \nabla \text{ integrable } t\text{-connection on } \E_{T'} \text{ over } T' \} \]
is representable. To understand this, check that $t$-connections on $\E_{T'}$ over $T'$ correspond not to sections of,
\begin{center}
\begin{tikzcd}
0 \arrow[r] & \E_{T'} \ot \Omega^1_{X_T'/T'} \arrow[r] & \cP^1_{X/S}(\E)_{T'} \arrow[r] & \E_{T'} \arrow[r] & 0
\end{tikzcd}
\end{center}
but to maps $\sigma$ such that $\pi \circ \sigma = t$ is the multiplication by $t$ map 
where we set $\nabla = t \cdot j - \sigma$.  Therefore, the integrable connections correspond to the pullback along,
\[ \Hom{T'}{\E_{T'}}{\cP^1_{X/S}(\E)_{T'}} \to \Hom{T'}{\E_{T'}}{\E_{T'} \ot \Omega^2_{X'_T/T'}} \times \Hom{T'}{\E_{T'}}{\E_{T'}} \]
of the section $(0, t)$. Furthermore, these sheaves are all pulled back along $X_{T'} \to X_T$. Since the pullback of a section induces the pullback connection and $f^* F_{\nabla} = F_{f^* \nabla}$ so this is a map of sheaves. Therefore the functor is the pullback along,
\[ \Hom{}{\E}{\cP^1_{X/S}(\E)_{T}} \to \Hom{}{\E}{\E \ot \Omega^2_{X_T/T}} \times \Hom{}{\E}{\E} \]
but since these sheaves are finitely presented (since $X \to S$ is finitely presented), flat over $T$, and with proper support over $T$ these functors are representable by schemes (Tag~\chref{https://stacks.math.columbia.edu/tag/08K6}{08K6}) affine and finitely-presented over $T$ hence so is the pullback of $T$ along this map since the map is also affine and finitely presented.
\end{proof}

\begin{rmk}
Alternatively, we could describe $t$-connections  as sections of the sequence obtained by pulling back along $\E_{T'} \xrightarrow{t} \E_{T'}$,
\begin{center}
\begin{tikzcd}
0 \arrow[r] & \E_{T'} \ot \Omega^1_{X_T'/T'} \arrow[r] & \cP^{t}_{X/S}(\E)_{T'} \arrow[r] & \E_{T'} \arrow[r] & 0
\end{tikzcd}
\end{center}
where explicitly $\cP^{t}_{X/S}(\E)_{T'}$ is the kernel of $\cP^1_{X/S}(\E)_{T'} \oplus \E_{T'} \to \E_{T'}$ via $\pi - t$ so that a section of this new sequence is a map $\sigma : \E_{T'} \to \cP^1_{X/S}(\E)_{T'}$ and a map $\lambda : \E_{T'} \to \E_{T'}$ so that $(\sigma, \lambda)$ is a section meaning that $\lambda = \id$ and that lands in $\cP^{t}_{X/S}(\E)_{T'}$ meaning $\pi \circ \sigma = t$. Under this interpretation we we need to consider the pullback of $(0, \id)$ along,
\[ \Hom{}{\E}{\cP^t_{X/S}(\E)_{T}} \to \Hom{}{\E}{\E \ot g^* \Omega^2_{X/S}} \times \Hom{}{\E}{\E} \]
but why is the first representable? We need to show that $\cP^t_{X/S}(\E)$ is finitely presented and flat over $S$ (we already know its support is proper over $S$. It is not generally true that the kernel of a surjection between finitely-presented modules is finitely-presented. However, this is true if the quotient is flat since then the sequence is locally split (using that finitely presented flat modules are projective).
\end{rmk}

\subsection{Framed Moduli}

For representations $\pi \to \GL_r(\CC)$ we often instead consider the moduli space of \textit{framed representations} meaning without modding out by conjugation (not taking maps up to isomorphism but rather just the set of group homomorphisms). This has the advantage of have a moduli problem representable by a scheme rather than a stack. Note, when $\pi = \pi_1(X(\CC))$ we need to pick a base point $x \in X$ since otherwise this group is only defined up to conjugation so the set of maps is not well-defined.
\bigskip\\
This framing for representations of $\pi_1$ corresponds to \textit{framed local systems} meaning local systems $\L$ equipped with an isomorphism $\varphi : \L_x \iso \CC^r$. As long as $X$ is connected, this rigidifies the problem since $\Hom{}{\L}{\L'} \embed \Hom{}{\L_x}{\L_x'}$ is injective because the sheaves are locally constant. 

\begin{defn}
$M_B^{\square}(X, x, r)$ is the moduli space of framed local systems of rank $r$ or equivalently framed representations of $\pi_1(X(\CC), x)$.
\end{defn}

\begin{prop}
$M_B^{\square}(X, x, r)$ is an affine scheme finite type over $\CC$.
\end{prop}

\begin{proof}
Since $X(\CC)$ is a finite CW complex we know that $\pi_1(X(\CC))$ is a finitely presented group. For any finitely presented group $G$, let $\{ g_i \}_{1 \le i \le n}$ be a set of generators. Then the framed character variety is exactly the closed subscheme of $(\GL_r)^n$ cut out by the relations. 
\end{proof}

Now we can likewise define the moduli spaces of framed vector bundles Higgs bundles. However, there is a subtlety which is that, in general, a framing does not adequately rigidity the problem in order to obtain a fine moduli space representable by a scheme.

\begin{example}
For a simple example, let $X = \P^1$ and $\E = \struct{}(-1) \oplus \struct{}(1)$ and let $\phi = 0$ be a trivial Higgs field. Then,
\[ \End{\E, \phi} = \left \{ \begin{pmatrix}
\CC & \Hom{}{\struct{}(-1)}{\struct{}(1)}
\\
0 & \CC
\end{pmatrix} \right\} \]
Thus, for example the automorphism,
\[ \varphi = \begin{pmatrix}
1 & X^2_0
\\
0 & 1
\end{pmatrix} \]
is the identity at $\infty = [0 : 1]$ and hence is compatible with a framing. We see the problem is $\E$ is not polystable.
\end{example}

However, the same behavior does not happen for vector bundles with connection.

\begin{prop}
Let $X$ be a smooth connected variety over $\CC$. Let $(\E, \nabla)$ be a vector bundle with connection and $\varphi : \E \to \E$ a parallel automorphism. If $\varphi_x = \id$ then $\varphi = \id$.
\end{prop}

\begin{proof}
Indeed, $\nabla$ induces a connection on $\Hom{}{\E}{\E}$ defined by the rule that,
\[ (\nabla \varphi)(s) = \nabla \varphi(s) - \varphi(\nabla s) \]
and by definition we have $\nabla \varphi = 0$. Therefore, $\varphi$ and $\id$ are both flat sections of $\Hom{}{\E}{\E}$. We will show that the locus where $\varphi$ and $\id$ are equal is both open and closed. Closedness is clear. To prove openness we use the uniqueness of solutions to first-order ODEs. We need to show that given a vector bundle with connection $(\E, \nabla)$ that the locus where two flat sections $s_1, s_2$ agree is open. Shrink so that we are on a disk $D^n \subset \CC^n$ and $\E \cong \struct{}^r$ then the connection takes the form,
\[ \nabla = \d + A \]
where $A$ is an $r \times r$ matrix of $1$-forms. Now $s_1$ and $s_2$ two solutions to the system of first-order ODEs so if they are equal at a point then they are equal on the disk.  
\end{proof}

(CAN THIS HAPPEN FOR SINGULAR VARITIES?)

\subsection{Stability Conditions}

\begin{lemma}
Let $\E_1$ and $\E_2$ be semi-stable vector bundles (or Higgs bundles)
\begin{enumerate}
\item if $\mu(\E_1) > \mu(\E_2)$ then $\Hom{X}{\E_1}{\E_2} = 0$
\item if $\mu(\E_1) \ge \mu(\E_2)$ and $\E_2$ is stable then any nonzero $\varphi \in \Hom{X}{\E_1}{\E_2}$ is surjective
\end{enumerate}
\end{lemma}

\begin{proof}
Let $\varphi : \E_1 \to \E_2$ and let $\K$ and $\F$ be the kernel and image respectively. In the case of Higgs bundles, these are $\phi$-stable sub-sheaves. Since $\E_1$ is semi-stable $\mu(\K) \le \mu(\E_1)$ hence,
\begin{align*}
\mu(\F) \rank{\F} &= \mu(\E_1) \rank{\E_1} - \mu(\K) \rank{\K} 
\\
& \ge \mu(\E_1) \rank{\E_1} - \mu(\E_1) \rank{\K}
\\
& = \mu(\E_1) \rank{\F}
\end{align*}
hence either $\F$ is zero or has positive rank in which case $\mu(\F) \ge \mu(\E_1)$. For (a) we would have $\mu(\F) \ge \mu(\E_1) > \mu(\E_2)$ contradicting the semi-stability of $\E_2$. For (b) we would have $\mu(\F) \ge \mu(\E_1) \ge \mu(\E_2)$ contradicting the stability of $\E_1$ unless $\F = \E_1$ meaning $\varphi$ is surjective.
\end{proof}

\begin{cor}

\end{cor}

\subsection{Coarse Spaces}

\subsubsection{GIT}

\newcommand{\Set}{\mathbf{Set}}

\begin{defn}
Let $X^{\sharp} : \Sch_S^\op \to \Set$ be a functor. A pair $(X, \varphi)$ of a scheme $X \in \Sch_S$ and a map of functors $\varphi : X^{\sharp} \to X$ \textit{coarse represents} (or corepresents) the functor $X^\sharp$ if it is initial for all such pairs. Explicitly, given any other $f : X^{\sharp} \to Y$ it factors uniquely as $f = f' \circ \varphi$,
\begin{center}
\begin{tikzcd}
X^{\sharp} \arrow[r, "\varphi"] \arrow[rd, "f"] & X \arrow[d, dashed, "f'"]
\\
& Y
\end{tikzcd}
\end{center}
If it exists, this uniquely determines the pair $(X, \varphi)$ up to unique isomorphism.
\bigskip\\
Moreover, we say that $(X, \varphi)$ universally coarse represents $X^\sharp$ if for each map of $S$-schemes $X' \to X$ then $\varphi' : X^{\sharp} \times_X X' \to X'$ is a initial (it coarse represents $X^{\sharp} \times_X X'$).
\end{defn}


\begin{example}
Let $X$ be a scheme with a $G$-action where $G$ is an $S$-group. Let $Y^\sharp(T) = X(T)/G(T)$. Then categorical quotients $X \to Y$ correspond exactly to coarse representing objects $Y^{\sharp} \to Y$. Indeed, this claim will immediately follow from the identification of morphisms $\varphi : Y^\sharp \to Y$ with $G$-invariant morphisms $X \to Y$. But morphisms $\varphi : Y^\sharp \to Y$ are maps natural transformations $X(T) \to Y(T)$ which factor through $X(T) \to X(T)/G(T)$ meaning that $X(T) \to Y(T)$ is $G(T)$-invariant. This is equivalent to $G$-invariance of $X \to Y$.
\end{example}

\begin{rmk}
Note that $f : X \to Y$ is $G$-invariant if the diagram,
\begin{center}
\begin{tikzcd}
G \times X \arrow[d, "\rho"]\arrow[r, "\pi_2"] & X \arrow[d, "f"]
\\
X \arrow[r, "f"] & Y
\end{tikzcd}
\end{center}
commutes. This is equivalent to $X(T) \to Y(T)$ is a $G(T)$-invariant map of sets. Indeed, the diagram shows that for any $g \in G(T)$ and $x \in X(T)$ we see that $f(g \cdot x) = f(x)$ from commutativity. Conversely, to from this diagram we consider $X(G \times X) \to Y(G \times X)$ which must be $G(G \times X)$-invariant. This means that $f(\pi_1 \cdot \pi_2) = f(\pi_2)$ but $\pi_1 \cdot \pi_2 = \rho$ as a map $G \times X \to X$ and hence we get $f \circ \rho = f \circ \pi_2$ as desired.
\end{rmk}

Let $S = \Spec{k}$ with $k$ algebraically closed of characteristic zero and $G$ be a reductive group.

\begin{defn}
The morphism $\varphi : X \to Y$ is a \textit{good quotient} if it is a universal categorical quotient (the associated $Y^\sharp \to Y$ universally coarse represents $Y^\sharp$), affine, and $Y$ is quasi-projective.
\end{defn}

\begin{theorem}[Mumford]
Suppose $\varphi : X \to Y$ is a good quotient. If $V_1, V_2$ are two distinct $G$-invariant closed subsets of $X$ then $\varphi(V_1)$ and $\varphi(V_2)$ are disjoint. The closed points $x \in Y$ correspond to closed orbits of $X$. If $x \in X$ is a closed point, its image is the point $y$ corresponding to the (unique) closed orbit in the closure of the orbit of $z$. 
\end{theorem}

Suppose that $G \acts X$ and $\L$ is an invertible sheaf on $X$ with $G$-action. 

\begin{defn}
A point $x \in X$ is \textit{semi-stable} if there exists a $G$-invariant section $f \in H^0(X, \L^{\ot n})^G$ such that $f(x) \neq 0$ and $D(f)$ is affine. A point $x \in X$ is \textit{stable} if furthermore we can find $f$ such that the orbits of $G \acts D(f)$ are all closed in $D(f)$ and the stabilizer of $x$ is finite. We call these loci,
\[ X^s \subset X^{ss} \subset X \]
\end{defn}

\begin{prop}
With the above notation. There exists a good quotient $\varphi : X^{ss} \to Y$ and an open set $Y^s \subset Y$ such that $\varphi^{-1}(Y^s) = X^s$ and the quotient $Y^s = Z^s / G$ is a universal geometric quotient. There is an ample invertible sheaf $\L_Y$ on $Y$ (it is always quasi-projective by definition) with $\varphi^* \L_Y = \L$. If $X$ is projective and $\L$ is ample, then $Y$ is projective. 
\end{prop}

Conversely, if $\varphi : X \to Y$ is a good quotient, then choose an ample $\L_Y$ on $Y$ and set $\L = \varphi^* \L_Y$ which comes equipped with a $G$-action. Running the procedure to $(X, \L)$ we obtain $X = X^{ss}$ and recover $\varphi : X \to Y$.

\subsubsection{The Betti Moduli Space}

Let $\Gamma$ be a finitely generated group and fix $r$.
\bigskip\\
We say that two $\Gamma$-representations are Jordan equivalent if their \textit{semi-simplifications} are isomorphic. The semi-simplification is the associated graded representation corresponding to any composition series. This is independent of the representation by the Jordan-Holder theorem.


\begin{thm}[Mumford, Seshadri]
There exists a universal categorical quotient $\mathbf{R}(\Gamma, r) \to \mathbf{M}(\Gamma, r)$ by the conjugation action of $\GL_r$. Then $M(\Gamma, r)$ is an affine scheme of finite type over $k$ whose closed points represent Jordan equivalence classes of representations.
\end{thm}

Hence we apply this to produce,
\[ M^\square_B(X, x, r) \to M_B(X, r) \]
the coarse space of representations. In this case, it is the coarse space of the stack quotient,
\[ \M_B = [M^\square_B(X, x, r) / \GL_r] \]

\subsubsection{The Moduli Space of Higgs Bundles}



\section{Clarifications and Examples}

\begin{rmk}
[Thank Matt] I described $\M_{\Hod}(X, r)$ as parametrizing pairs $(\E, \nabla)$ of a vector bundle of rank $r$ on $X \times \A^1$ and a $t$-connections $\nabla$ on $\E$ for $\pi_2 : X \times \A^1 \to \A^1$. This was misleading, I was actually describing sections of $\M_{\Hod} \to \A^1$ since those should be the objects which interpolate between the $t = 0$ and $t = 1$ fibers. Since we want a map $\M_{\Hod}(X, r) \to \A^1$ we can see that $\M_{\Hod}(X, r)$ should be valued on $\A^1$-schemes. Thus, I didn't mean to describe the stack sending $T$ to vector bundles on $X \times \A^1 \times T$ with $t$-connection for the map to $\A^1_T$. This stack does not have an obvious map to $\A^1$. Indeed, this fibered category is $\M_{\Hod}(X)(\A^1_T)$ rather than $\M_{\Hod}(X)(T)$ meaning this stack is the Weil restriction of $\M_{\Hod}(X) \to \A^1_S$ along $\A^1_S \to S$.
\bigskip\\
Likewise, I said that a $t$-connection $\nabla$ on $\pi_1^* \E$ for $t : X \times \A^1 \to \A^1$ the $t$-coordinate where $\E$ a vector bundle on $X$ is the correct notion of a degeneration of a connection to a Higgs bundle on a \textit{fixed} underlying vector bundle $\E$. In our formalism the space of such degenerations is nicely described as the sections of $M_{\Hod}(X, \E) \to \A^1_S$ where $M_{\Hod}(X, \E)$ is the \textit{scheme},
\begin{center}
\begin{tikzcd}
M_{\Hod}(X, \E) \arrow[r] \arrow[d] \pullback &  \A^1_S \arrow[d, "{[\E]}"]
\\
\M_{\Hod}(X, r) \arrow[r] & \M(X,r) \times_S \A^1_S
\end{tikzcd}
\end{center}
\end{rmk}

\subsection{Endomorphisms of Vector Bundles}

About the question of endomorphisms of semi-stable vector bundles. We want to show the following:

\begin{prop}
Let $\E, \E'$ be semi-stable Higgs (resp. vector) bundles with $\mu(\E) \ge \mu(\E')$ and let $x \in X$ then the map,
\[ \Hom{X}{\E}{\E'} \to \Hom{}{\E(x)}{\E'(x)} \]
is injective.
\end{prop}

\begin{proof}
We first prove this in the case that $\E'$ is stable. We showed that any nonzero $\varphi \in \Hom{X}{\E}{\E'}$ is surjective. Thus if $\varphi_x = 0$ then $\varphi = 0$. Now we prove the general case by induction on $\rank{\E'}$. If $\E'$ is not stable then we can filter it by a composition series (the Jordan-H\"{o}lder filtration). Really all we need is a sequence of vector bundles,
\[ 0 \to \E'_1 \to \E' \to \E'_2 \to 0 \] 
with $\E'_1$ stable and $\E'_2$ semi-stable all of slope $\mu(\E')$. Then we get,
\begin{center}
\begin{tikzcd}
0 \arrow[r] & \Hom{}{\E}{\E'_1} \arrow[d, hook] \arrow[r] & \Hom{}{\E}{\E'} \arrow[d] \arrow[r] & \Hom{}{\E}{\E_2'} \arrow[d, hook]
\\
0 \arrow[r] & \Hom{}{\E(x)}{\E'_1(x)} \arrow[r] & \Hom{}{\E(x)}{\E'(x)} \arrow[r] & \Hom{}{\E(x)}{\E_2'(x)} 
\end{tikzcd}
\end{center}
the outside maps are injective by the induction hypothesis so we conclude by a diagram chase.
\end{proof}

\newcommand{\sat}{\mathrm{sat}}
\newcommand{\LF}{\mathrm{LF}}

However, we swept something under the rug. It it true by Zorn's lemma that if $\E'$ is not stable then there exists a minimal subsheaf $\F$ of slope $\mu(\F) = \mu(\E')$ so $\F$ is stable. However, there is no reason this is a sub-bundle. We can replace $\F$ by $\F^{\sat}$ (the the kernel of $\E' \to (\E'/\F)_{\text{tors-free}}$) which has $\mu(\F^{\sat}) \ge \mu(\F)$. If $X$ is a curve then any torsion-free sheaf is locally free and hence $\F^{\sat}$ is a sub-bundle. Therefore, if $X$ is a curve we only need to consider sub-bundles to check (semi)-stability and there is a Jordan-H\"{o}lder filtration by sub-bundles.
\bigskip\\
For $\dim{X} > 1$ the Jordan-H\"{o}lder and Harder-Narashimhan filtration are only by torsion-free sub-sheaves. In order to get a rigidified fine moduli space, Simpson makes the following definition which we state in the relative setting: for $X \to S$ smooth projective with a section $\xi$ we say $\E$ satisfies $\LF(\xi)$ if its Jordan-H\"{o}lder subquotients are locally free along $\xi$. 

\begin{prop}
Let $\E, \E'$ be semi-stable Higgs (resp. vector) bundles satisfying $\LF(\xi)$ and $\mu(\E) \ge \mu(\E')$ then the map,
\[ \Hom{X}{\E}{\E'} \to \Hom{S}{\xi^* \E}{\xi^* \E'} \]
is injective.
\end{prop}

\subsection{A striking example}

We have mentioned over and over that the moduli spaces $M_B$ and $M_{\dR}$ are only biholomorphic 
\textit{not algebraically isomorphic} and these are only homeomorphic to $M_{\Dol}$. I want to give a striking example.
\bigskip\\
Let $E$ be an elliptic curve. Let $r = 1$ so we are looking for line bundles with connection and characters of $\pi_1(E) = \Z^2$. Clearly $M_B = \Gm^2$ and since $\pi_1(E)$ does not know about the complex structure of $E$ we see that $M_B$ does not know ``which elliptic curve'' $E$ is.
\bigskip\\
However, any $(\L, \nabla)$ must have $c_1(\L) = 0$ since the Atiyah class (obstruction to having an algebraic connection) for rank $1$ vector bundles is just $c_1(\L) \in H^1(X, \Omega^1)$ giving $M_{\dR} \to \fPic^0_E = E$. Since the space of connections on $\L$ is affine over $H^0(E, \Omega_E^1) = \CC$, the map is an $\A^1$-torsor trivial in the \etale topology (its smooth so has \etale sections Therefore, by the ``homotopy invariance'' for line-bundles $\fPic_{M_{\dR}}^0 = E$ so $M_{\dR}$ knows everything about the elliptic curve! However, $M_{\dR}^\an \cong M_B^\an = (\CC^\times)^2$ loses all this information!
\bigskip\\
This is actually a really cool example. Since we can send $(\L, \nabla), (\L', \nabla') \mapsto (\L \ot \L', \nabla \ot \nabla')$ this makes $M_{\dR} \to E$ a map of algebraic groups and, in accordance with Chevallay's theorem, there is an exact sequence,
\[ 0 \to \Ga \to M_{\dR} \to E \to 0 \]
However, if $M_{\dR}$ were affine then it would be an affine algebraic group which we know means it would be linear algebraic. Chevallay proved every quotient of a linear algebraic group is linear algebraic giving a contradiction. Hence $M_{\dR}$ is a non-affine scheme such that $M_{\dR}^\an$ is Stein. In fact, $M_{\dR}$ has no non-constant algebraic functions even though it can be holomophically embedded in affine space.
\bigskip\\
Even worse, $M_{\Dol} = \fPic^0_{E} \times H^0(E, \Omega^1_E) = E \times \A^1$ which is not analytically isomorphic to $(\CC^\times)^2$ since the latter is Stein but not the former (since $H^1(X, \struct{X}) \neq 0$). Alternatively, there is an embedding $E \embed E \times \CC$ but every holomorphic map $E \to (\CC^\times)^2$ is constant by the maximum principle. In this case the Simpson correspondence is given as follows, write $\CC^\times = S^1 \times \RR^+$ and then we have homeomorphisms,
\[ E \times \CC \cong (S^1)^2 \times (\RR^+)^2 \cong (\CC^\times)^2 \]
If we write out what happens in detail, fixing a homology basis $\gamma_1, \gamma_2$ we will find that $E \times \{ 0 \}$ corresponds to unitary representations so we send $x \in E$ to the monodromy of the unique harmonic connection on $\struct{}(x - 0)$ around $\gamma_1$ and $\gamma_2$ and the Higgs field is mapped to $(\RR^+)^2$ via,
\[ \theta \mapsto \left( \exp\left( - \int_{\gamma_1} \theta + \bar{\theta} \right), \exp\left( - \int_{\gamma_2} \theta + \bar{\theta} \right) \right) \] 
which are positive real numbers since the interior integrals are over real forms.

\section{$p$-Stability and $\mu$-Stability}

From now on, let $X$ be a smooth projective scheme of dimension $d$ over $S = \Spec{\CC}$ with very ample line bundle $\struct{X}(1)$. We write the ample class $H$ on $X$. Then recall the slope is,

\begin{defn}
Recall that the slope of a torsion-free sheaf is,
\[ \mu(\E) := \frac{c_1(\E) \cdot H^{n-1}}{\rank{\E}} \]
\end{defn}

\newcommand{\Td}{\mathrm{Td}}
\renewcommand{\ch}{\mathrm{ch}}

\begin{rmk}
Recall that HRR allows us to compute,
\[ \chi(X, \E(\ell)) = \int \ch(\E(\ell)) \Td(X) = \int \ch(\E) (e^{\ell H}) \Td(X) \]
This is a polynomial in $\ell$. Let's compute the terms of order $\ell^d$ and $\ell^{d-1}$. We have,
\[ \ch(\E) = r + c_1(\E) + \cdots \quad \quad \Td(X) = 1 + \tfrac{1}{2} c_1(X) + \cdots \]
Also,
\[ e^{\ell H} = 1 + \ell H + \cdots + \tfrac{1}{d!} \ell^n H^d \]
The term $\ell^d H^d$ can only pair against $0$-forms and the term $\ell^{d-1} H^{d-1}$ can only pair against $1$-forms. There are two such terms: $c_1(\E) \cdot 1$ and $r \cdot \tfrac{1}{2} c_1(X)$ and thus,
\[ \chi(X, \E(\ell)) = r \ell^d/d! + [c_1(\E)\cdot H^{d-1} + \tfrac{1}{2} r c_1(X) \cdot H^{d-1}] \ell^{d-1}/(d-1)! + \cdots \]
Hence if we write the Hilbert polynomial of $\E$ as,
\[ p(\E) = r \ell^d / d! + a \ell^{d-1} / (d-1)! + \cdots \]
we identify $a = [c_1(\E) + \tfrac{1}{2} c_1(X)] \cdot H^{d-1}$. Now Simpson defines the slope as,
\[ \mu_{\text{Simpson}}(\E) = \frac{a}{r} = \frac{c_1(\E) \cdot H^n}{\rank{\E}} + \frac{1}{2} c_1(X) \cdot H^{d-1} \]
which is exactly our notion of slope up to an additive constant hence the two notions of $\mu$-stability are identical. However, in the case $\mu(\E) = 0$ it is convenient to have a slightly finer notion of stability which takes into account higher Chern classes. The easiest way to do this is Simpson's notion of $p$-stability which makes his proofs of boundedness easier.
\end{rmk}

\begin{defn}
A coherent sheaf $\E$ is \textit{pure dimension} $d = d(\E)$ if for any nonzero subsheaf $\F \subset \E$ we have $\dim{\Supp{}{\F}} = d$. A coherent sheaf $\E$ is $p$-\textit{semi-stable} (resp. $p$-\textit{stable}) if it is of pure dimension and for any subsheaf $\F \subset \E$ we have
\[ \frac{p(\F, n)}{\rank{\F}} \le \frac{p(\E, n)}{\rank{\F}} \]
for all $n \gg 0$.  
\end{defn}

These notions are equivalent when $X$ is a curve.
\bigskip\\
Note that an asymptotic inequality of polynomials gives an inequality on their leading term that differs hence $p$-semi-stability implies $\mu$-semi-stability whereas a \textit{strict} inequality of a leading non-equal coefficient implies \textit{strict} asymptotic inequality of the polynomials hence $\mu$-stability implies $p$-stability. 

\section{Daniel}

\newcommand{\prim}{\mathrm{prim}}

\begin{theorem}
Let $X/\CC$ be a smooth projective variety. There are homeomorphisms,
\[ \M^{ss}_{\Dol} \cong \M^{ss}_{\dR} \cong \M_B \]
\end{theorem}

Here points of $\M^{ss}_{\Dol}$ correspond to direct sums of stable Higgs bundles with $c_1 [\omega]^{d-1} = c_2 [\omega]^{d-2} = 0$. The points of $\M_B$ correspond to semisimple representations.

\begin{conj}
Isolated points in $\M_B$ are motivic meaning they appear as summands in the monodromy representation of a family of varities.
\end{conj}

In particular, the corresponding representations of $\pi_1(X)$ are defined over the ring of integers of a number field. This will give an interesting restriction of what $\pi_1(X)$ can be, but the conjecture is very open as far as I can tell. On the other hand, the following consequence of the above conjecture is a theorem of Simpson.

\begin{theorem}[Simpson]
Let $(V, D) \in \M^{ss}_{\dR}$ be an isolated point. Then $(V, D)$ admits a variation of Hodge structure.
\end{theorem}

\subsection{Variations of Hodge Structure}

We first need to define what a variation of Hodge structure on a complex vector bundle is. This is supposed to imitate what the de Rham cohomology $H^i_{\dR}(X/\CC)_{\prim}$ of a family of projective varities would look like.

\begin{defn}
A complex \textit{variation of Hodge structures} is a $C^{\infty}$-complex vector bundle $V$ together with a decomposition,
\[ V = \bigoplus_{p + q = k} V^{p,q} \]
and a flat connection on $V$ satisfying:
\begin{enumerate}
\item Griffiths transversality
\[ D : V^{p,q} \to \cA^{0,1}(V^{p+1,q-1}) \oplus \cA^{1,0}(V^{p,q}) \oplus \cA^{0,1}(V^{p,q}) \oplus \cA^{1,0}(V^{p-1, q+1}) \]
\item there exists a $D$-flat hermitian form $H$ with the property that $(-1)^p H$ is a positive-definite Hermitian metric on $V^{p,q}$.
\end{enumerate}
\end{defn}

\begin{rmk}
Note that we do not require $V^{p,q} \cong V^{q,p}$ in our definition because there is no complex conjugation action assumed on $V$.
\end{rmk}

Let's quickly say where these structures arise from in the case of a geometric VHS. The decomposition is the usual Hodge decomposition, and the flat connection is the Gauss-Manin connecion. The hermitian gorm $H$ comes from a global hyperplane class $[\omega]$, adn the $H$ is flat because $[\omega]$ is flat. The positive-definiteness is just the Hodge index theorem. 
\bigskip\\
For Griffiths transversality, usually we consider the relative Hodge filtrations,
\[ F^p = \bigoplus_{p' \ge p} V^{p', k-p'} \]
is a holomorphic vector bundle and,
\[ \ol{F}^p = \bigoplus_{p' \le p} V^{p',k-p'} \]
is an antiholomorphic vector bundle and
\[ \nabla : F^p \to F^{p-1} \ot \Omega^1 \]
is a flat holomorphic connection. To actually promote $\nabla$ to a smooth connected, we write $D = \tilde{\nabla} + \dbar$. It follows that
\[ D : F^{p} \to A^{1,0}(F^{p-1}) \oplus A^{0,1}(F^p) \]
since $F^p$ is holomorphic (so $\dbar$ is always a well-defined connection on $F^p$). Similarly,
\[ D : \ol{F}^p \to A^{1,0}(\ol{F}^p) \oplus A^{0,1}(\ol{F}^{p+1}) \]
Therfore, $D$ sends $V^{p,q} = F^p \cap \ol{F}^q$ into the sum of four tems above.
\bigskip\\
The corresponding representations are automatically semisimple because the monodromy is contained in a unitary group (because of the polarization) and all representations of a unitary group are semisimple. By changing signs of $H$ appropriately, we get a hermitian metric $K$. This turns out to be harmonic. If we decompose
\[ D = \ol{\theta} + \partial + \dbar + \theta \]
according to Griffiths transversality and set $D'' = \dbar + \theta$ and $D' = \partial + \ol{\theta}$ then we can check that $D''$ is the corresponding connection to the Harmonic metric $K$. Therefore, this is the Higgs bundle. 

\begin{defn}
A \textit{system of Hodge bundles} is a Higgs bundle $(E, \theta)$ with a decomposition,
\[ E = \bigoplus_{p, q} E^{p,q} \]
such that $\theta$ decomposes as,
\[ \theta : E^{p,q} \to E^{p-1, q+1} \ot \Omega_X^1 \]
\end{defn}

Then there is a correspondence between a variation of Hodge structures on a given local system and a system of Hosge bundles on the corresponding Higgs bundle.
\bigskip\\
Recal that there is a natural $\Gm$-action on the space of Higgs bundles given by $(E, \theta) \mapsto (E, t \theta)$. Using the Simpson correspondence, it will correspond to a mysterious action on the space of semisimple representations of $\pi_1(X)$. This action will preserve the monodromy group.

\begin{lemma}
If $(E, \theta) \cong (E, t \theta)$ for some $t \in \CC^\times$ not a root of unity then $E$ has the structure of a sysem of Hodge bundles. If $E$ is stable then this decomposition is unique up to reindexing.
\end{lemma}

\begin{proof}
Let $f : E \to E$ be the isomorphism satisfying $f \circ \theta = t \theta \circ f$. There is always a decomposition of $E$ into generalized eigensheaves,
\[ E = \bigoplus_{\lambda \in \CC} E_{\lambda} \]
where $(f|_{E_\lambda} - \lambda \id)^r = 0$ where $r$ is the rank of $E$.  
\end{proof}

\begin{lemma}
Let $f : E \to E$ be an automorphism of vector bundles over a complex projective variety. Then there is a decompositon,
\[ E = \bigoplus_{\lambda \in \CC} E_\lambda \]
into generalized eigensheaves of $f$. Then $\theta$ maps $E_\lambda \to E_{t \lambda}$. {\color{red} WHERE IS $t$ NOT ROOT OF UNITY USED?}
\end{lemma}

\begin{proof}
Consider,
\[ \bigoplus_{\lambda \in \CC} \ker{(f - \lambda \id)^r} \to E \]
where $r = \rank{E}$. If we knew each were a fiber bundle then it would be immediate to check that this map is an isomorphism. One way to do this is to note that the coefficients of the characteristic polynomial of $f$ are global sections of $X$ and hence constant. {\color{red} FINISH}
\end{proof}

\begin{rmk}
This is clearly false if $X$ is not projective. For example, consider $f : \struct{\Gm} \to \struct{\Gm}$ given by multiplication by $z$. Then the eigenvalue over $z$ is $z$. 
\end{rmk}

\begin{cor}
The representations of $\pi_1(X)$ which come from a complex variation of Hodge structures are exactly the semisimple representations fixed by the $\Gm$-action {\color{red} HOW TO GET THE OTHER DIRECTION?}
\end{cor}

In particular, rigid representations must be equipped with a variation of Hodge structures. 

\subsection{Monodromy Groups}

\begin{defn}
Let $V$ be a local system on $X$ and consider $\rho : \pi_1(X, x) \to \GL(V_x)$ the associated monodromy representation. The \textit{monodromy group} or \textit{Mumford-Tate group} $M(V,x) \subset \GL(V_x)$ is the Zariski closure of the image of $\rho$.
\end{defn}

Question: can we recover this grom the Higgs bundle side of the equation? To do this, we review some Tannakian principles.

\begin{prop}
Let $G$ be a complex linear algebraic group and $G \embed \GL(V)$ a faithful representation. Then $G$ is the subgroup of elements $g \in \GL(V)$ with the following property: for every subspace $W \subset V^{\ot a} \ot (V^\vee)^{\ot b}$ stable under $G$, the element $g$ stabilizes $W$ as well. 
\end{prop}

This motivates us to make the following definition.

\begin{defn}
Let $(E, \theta)$ be a Higgs bundle with vanishing Chern classes. We define $M(E, x) \subset \GL(E_x)$ as the subgroup of $g \in \GL(E_x)$ with the following property: for every Higgs subbundle $W \subset E^{\ot a} \ot (E^{\vee})^{\ot b}$ of deggree zero, the element preserves $W_x$. 
\end{defn}

\begin{prop}
If $E$ is a harmonic bundle corresponding to a flat bundle $V$ and a Higgs bundle $E$ then $M(E, x) = M(V, x)$ and moreover this group is reductive.
\end{prop}

\begin{proof}
We basically just showed this. Any Higgs bundle with degree zero is a direct summand {\color{red} WHY?}. Reductivity follows from the fact that $V$ is a semisimple representation.
\end{proof}

We can now strengthen the result on rigid representations:

\begin{prop}
Let $\rho : \pi_1(X, x) \to \GL(V_x)$ be a semisimple representation. Let $G = M(V, x)$ be its monodromy group and assume that it defines an isolated point in $\M_B(G) = \Hom{}{\pi_1(X, x)}{G}^{ss} // G$. Then $V$ admits a variaton of Hodge structures.
\end{prop}

Here is a further refinement.

\begin{theorem}
Let $G$ be a reductive complex algebraic group. Then any (not necessarily semisimple) representation $\pi_1(X, x) \to \GL(V_x)$ can be deformed to a representation that arises from a variation of Hodge structures.
\end{theorem}

\begin{proof}
We can first deform the representation so that its monodromy group $M$ is reductive, the idea is that if $M$ is not reductive, there exists a cocharacter $\Gm \to G$ such that $\lim_{t \to 0} \ad(t)(m)$ exists for all $m \in M$ and moreover the dimension decreases in the limit. Doing this inductively reduces to the case when $M$ is reductive.
\bigskip\\
Recall that we had this proper map $\M_{\Dol} \to A$ where $A$ is the Hitchin base. At this point, we use the $\Gm$-action on $\M_{\Dol}$ which preserves the monodromy group. If we look at the $t \to 0$ limit, we see that the limiit point must be something that is in $\M_{\Dol}(M)$ and also arises from a variation of Hodge structure. {\color{red} FIX THIS PROOF}
\end{proof}

\subsection{Groups of Hodge Type}

There are interesting restrictions on what the monodromy groups of a variation of Hodge structures can look like.

\begin{defn}
Let $W$ be a real algebraic group and $G = W_{\CC}$. A \textit{Cartan involution} $C : W \to W$ is an automorphism such that $C^2 = 1$ and the fixed points of $\sigma C = C \sigma$ on $G$ is compact where $\sigma : G \to G$ is complex conjugation.
\end{defn}

\begin{defn}
A real linear algebraic group $W$ is said to be of \textit{Hodge type} if there exists a $(\Gm)_{\CC}$-action on $G = W_{\CC}$ such that $U(1)$ preserves $W$ and $-1$ acts by a Cartan involution.
\end{defn}

\begin{example}
A complex group, regarded as a real group, is not of Hodge type. The groups $\SL_n(\RR)$ for $n \ge 3$ and $\SU(2n)$ for $n \ge 3$, and $\SO(p,q)$ for $p,q$ odd, are not of Hodge type. However, groups like $\Sp(2n, \RR)$ are of Hodge type.
\end{example}

\begin{prop}
Let $\rho$ be a representation that arises from a variation of Hodge structures, and let $G$ be the associated monodromy subgroup. Let $W \subset \Res{\CC}{\RR}{G} \subset \Res{\CC}{\RR}{\GL(V_x)}$ be the real Zariski closure of the image of $\rho$. Then $W$ is a real form of $G$ and moreover $W$ is of Hodge type.
\end{prop}

\begin{proof}
We will use the Higgs interpretation of the monodromy group $G = M(E, x)$. Consider a polarization of the flat vector bundle and the corresponding harmonic metric $K$. Let $\tau$ be the complex conjugate $\GL(E_x) \cong {}^c \GL(E_x)$ induced by $K$, and then its fixed points by $U(E_x, K_x)$. 
\bigskip\\
The first claim is that $\tau$ stabilizes $G$ and therefore $U(E_x, K_x) \cap G$ is a real gorm of $G$ {\color{red} TODO}
\bigskip\\
Now we consider the action of $\Gm$ on $E$ given by scaling $E^{p,q}$ by $t^p$. This induces a natural action on $\GL(E_x)$ and it preserves $G$ because the tensors $S$ cutting out $G$ are preserved {\color{red} CHECK}. The elements of $U(1)$ preserves the polarization and the metric, hence preserve $U(E_x, K_x) \cap G$. The element $C$, which is the image of $-1$, gives a new complex conjugation $\tau C$ which corresponds to the hermtiian form $H$. This shows that $\pi_1(X)$ lands in this real form $W$ of $G$.
\end{proof}

\begin{cor}
Let $W$ be a real reductive algebraic group, with complex form $G$. Suppose $X$ is a smooth projective variety and let $\rho : \pi_1(X) \to W$ is a representation. Assume there is a subgroup $\Gamma \subset \pi_1(X)$ wuth $\rho|_\Gamma$ rigid as a rpresentation for $G$ and also have Zariski dense image in $G$. Then $W$ is a group of Hodge type.
\end{cor}

\begin{proof}
First deform $\rho : \pi_1(X) \to G$ into a representation $\rho'$ admitting a variation of Hodge structures. Since $\rho|_\Gamma$ is rigid, we see that $\rho'|_\Gamma \cong \rho|_\Gamma$ as representations into $G$. Modifying $\rho'$ by a conjugation, we may assume that $\rho'|_\Gamma$ has image contained in $W$ (because $\rho(\Gamma) \subset W \subset G$ and $\rho'|_\Gamma$ is isomorphic to $\rho|_\Gamma$). Since its Zariski closure must be a real form of $G$, it has to be equal to $W$. Therefore, $W$ is of Hodge type. 
\end{proof}

\begin{example}
The group $\SL_n(\ZZ)$ for $n \ge 3$ cannot be the fundamental group of a smooth projective variety. Indeed, $\SL_n(\RR)$ is not of Hodge type but $\SL_n(\ZZ) \to \SL_n(\RR)$ is rigid. 
\end{example}
\section{Final Talk}

Let $M$ be a complex manifold (or $M = X^\an$ for a smooth proper variety).

\begin{defn}
A \textit{harmonic bundle} is a tuple $(E, \dbar, \theta, h)$ where,
\begin{enumerate}
\item $E$ is a smooth $\CC$-vector bundle
\item $(E, \dbar)$ is a holomorphic vector bundle
\item $h$ is a Hermitian metric on $E$
\item $\theta : \E \to \E \ot \Omega^1_X$ is linear where $\E = \ker{\dbar}$ is the sheaf of holomorphic sections (i.e. $\theta$ is holomorphic)
\end{enumerate}
such that,
\[ \nabla := \partial_h + \dbar + \theta + \theta^\dagger_h \]
is a flat connection. Here we have used the following notation: $\partial_h$ is the Chern connection wrt to the data $(E, \dbar, h)$ and $\theta_h^\dagger : \cA^0(E) \to \cA^{0,1}(E)$ is the adjoint of $\theta$ wrt to $h$ meaning that for all $u,v \in \cA^0(E)$,
\[ h(\theta(u), v) = h(u, \theta^\dagger_h(v)) \]
explicitly, let $e_i$ be a local orthonormal frame of $(E, h)$ and $\d{z_i}$ a local frame of $\Omega_M$ then if we write
\[ \theta(s_j) = C_{ijk} \, s_i \ot \d{z_k} \]
then
\[ \theta^\dagger_h(s_j) = C_{jik}^* \, s_i \ot \d{\bar{z}_k} \]
\end{defn}

\begin{rmk}
Notational remarks:
\begin{enumerate}
\item Given a locally free sheaf $\E$ on $M$ with associated holomorphic bundle $(E, \dbar)$ we say that $\E$ \textit{admits a harmonic bundle structure} if there exists a harmonic bundle of the form $(E, \dbar, \theta, h)$

\item Given a Higgs bundle $(\E, \theta)$ on $M$ with associated holomorphic bundle $(E, \dbar)$, we say that $(\E, \theta)$ \textit{admits a harmonic bundle structue} if there exists a harmonic bundle of the form $(E, \dbar, \theta, h)$ meaning there exists an $h$ fitting into the tuple

\item We say that a flat bundle $(E, \nabla)$ \textit{admits a harmonic bundle structure} if there exists a harmonic bundle $(E, \dbar, \theta, h)$ such that $\nabla = \partial_h + \dbar + \theta + \theta_h^\dagger$.
\end{enumerate}
\end{rmk}

\begin{rmk}
Notice that if $(E, \nabla)$ is a flat bundle then $(E, \nabla^{0,1})$ is a holomorphic bundle since flatness of $\nabla$ implies that $\nabla^{0,1}$ is an integrable type $(0,1)$-connection. However, we should be careful that when we say that $(E, \nabla)$ admits a harmonic bundle structure $(E, \dbar, \theta, h)$ then the holomorphic structure $(E, \dbar)$ does \textit{not} agree with the holomorphic structure $(E, \nabla^{0,1}) = (E, \dbar + \theta_h^\dagger)$ undless $\theta = 0$. This is related to the fact that the Simpson correspondence is not an analytic isomorphism of the moduli spaces.
\end{rmk}

\begin{rmk}
Note that given a harmonic bundle $(E, \dbar, \theta, h)$ we can recover the entire datum from $(E, \nabla, h)$ indeed $h$ determines $\partial_h$ so $\nabla^{1,0} = \partial + \theta$ determines $\theta$ and $\nabla^{0,1} = \dbar + \theta_h^\dagger$ then determines $\dbar$.
\end{rmk}

\begin{theorem}[Simpson]
A flat bundle $(E, \nabla)$ is semi-simple if and only if it admits a harmonic bundle structure $(E, \dbar, \theta, h)$. Then the metric $h$ is unique up to flat automorphism of $(E, \nabla)$.
\end{theorem}

\begin{cor}
If a flat bundle $(E, \nabla)$ admits a harmonic bundle structure $(E, \dbar, \theta, h)$ then the Higgs bundle $(E, \dbar, \theta)$ is uniquely determined by $(E, \nabla)$ up to isomorphism.
\end{cor}

\begin{proof}
Suppose that $(E, \dbar', \theta', h')$ is a second harmonic bundle structure. By the previous result there is a flat automorphism $\varphi : E \to E$ such that $\varphi^* h = h'$ but then $\varphi^* \partial_h = \partial_{h'}$ and by assumption $\varphi^* \nabla = \nabla$ but $(E, \nabla, h)$ and $(E, \nabla, h')$ determine $(E, \dbar, \theta, h)$ and $(E, \dbar', \theta', h')$ respectively. Therefore, $\varphi^* \dbar = \dbar'$ and $\varphi^* \theta = \theta$ meaning that $\varphi$ is holomorphic and respects the Higgs field. Therefore, $\varphi : (E, \dbar', \theta') \to (E, \dbar, \theta)$ is an isomorphism of Higgs bundles. 
\end{proof}

\begin{theorem}
Let $(E, \dbar, \theta)$ be a Higgs bundle with $\ch_1(E) \cdot [\omega]^{n-1} = \ch_2(E) \cdot [\omega]^{n-2} = 0$. Then $(E, \dbar, \theta)$ admits a harmonic metric $h$ (meaning $(E, \dbar, \theta, h)$ is a harmonic bundle) if and only if it is polystable. Furthermore, the harmonic metric $h$ is unique up to automorphism of $(E, \dbar, \theta)$.
\end{theorem}

\begin{cor}
If a Higgs bundle $(E, \dbar, \theta)$ admits a harmonic bundle structure $(E, \theta, \theta, h)$ then the flat bundle $(E, \nabla)$ is unqiuely determined by $(E, \dbar, \theta)$ up to isomorphism.
\end{cor}

\begin{proof}
Suppose $\varphi : (E, \dbar, \theta) \to (E, \dbar, \theta)$ is an automorphism such that $\varphi^* h = h'$ then $\varphi^* \nabla = \nabla'$ so $\varphi : (E, \nabla') \to (E, \nabla)$ is a flat isomorphism.
\end{proof}


Daniel made a good observation. We showed that if $(E, \dbar, \theta)$ is a semistable Higgs bundle with vanishing first two Chern numbers then $E$ admits a flat bundle structure. But by Chern-Weil theory, the Chern classes $c_i(E)$ are computed in terms of the curvature which vanishes. Hence all $c_i(E) = 0$.
\bigskip\\
Indeed: Simpson is confusing about this point but I think it is implied by Theorem 2 in Higgs bundles and local systems. This is also proved in Theorem 3.1 of Numerically Flat Higgs Vector Bundles by Bruzo and Otero.  

\subsection{The Simpson Correspondence in Terms of $t$-Connections}

\begin{prop}
Let $(E, \dbar, \theta, h)$ be a harmonic bundle. Then there exists a holomorphic vector bundle with $t$-connection $(\E, D_t)$ on $X \times \A^1$ such that $(\E_0, D_0) = (E, \dbar, \theta)$ is the associated Higgs bundle and $(\E_1, D_1) = (\ker{\nabla^{0,1}}, \nabla^{1,0})$ is the associated flat bundle.
\end{prop}

\begin{proof}
Recall that $\nabla = \partial_h + \dbar + \theta + \theta^\dagger_h$ is flat. In previous weeks we actually showed that the condition for flatness is equivalent to the condition (which is stronger but easier to analyize) that $\nabla_t = t \partial + \dbar + \theta + t \theta_h^\dagger$ satisfies $(\nabla_t)^2 = 0$. Therefore, the $(0,1)$ and $(1,0)$ parts commute and are individually flat. This means that $D_t = \dbar + t \theta^\dagger_h$ is a flat $t$-connection on the holomorphic bundle $\E_t = (E, \dbar + t \theta^\dagger_h)$. For $t = 0$ we get $\E_0 = (E, \dbar)$ and $D_0 = \theta$ so $(\E_0, D_0)$ is the associated Higgs bundle. For $t = 1$ we get $\E_1 = (E, \dbar + \theta^\dagger_h) = (E, \nabla^{0,1})$ and $D_1 = \nabla^{1,0}$ so we recover the flat bundle $(\E_1, D_1) = (E, \nabla)$.
\end{proof}

\subsection{Cohomology of Higgs Bundles}

\newcommand{\HH}{\mathbb{H}}

\begin{defn}
Let $(\E, \nabla)$ be a vector bundle with a flat algebraic/holomorphic connection. Then the algebraic/holomorphic de Rham complex $(\E \ot \Omega^\bullet, \nabla)$ is the complex,
\[ 0 \to \E \xrightarrow{\nabla} \E \ot \Omega^1_X \xrightarrow{\nabla^2} \E \ot \Omega^2_X \to \cdots \]
Note that $(\E \ot \Omega^\bullet, \nabla)$ is a complex if and only if $\nabla$ is flat. Then the de Rham cohomology of $(\E, \nabla)$ is defined as the hypercohomology of this complex,
\[ H_{\dR}^\bullet(X, (E, \nabla)) = \HH^\bullet(X, (\E \ot \Omega_X^\bullet, \nabla)) \]
\end{defn}

\begin{rmk}
If $(E, \nabla)$ is a flat bundle then $\E = \ker{\nabla^{0,1}}$ is a holomorphic bundle and $(\E, \nabla^{1,0})$ is a flat holomorphic connection. We can likewise consider the de Rham complex $\cA^\bullet(E)$ of $(E, \nabla)$,
\[ 0 \to E \xrightarrow{\nabla} \cA^1(E) \xrightarrow{\nabla^2} \cA^1(E) \to \cdots \]
and so we likewise define,
\[ \HH^i_{\dR}(X, (E, \nabla)) := \HH^i(X, \cA^\bullet(E)) = H^\bullet( \Gamma(X, \cA^\bullet(E))) \]
since $\cA^\bullet(E)$ are flasque sheaves. 
\end{rmk}

\begin{lemma}[Poincar\'{e}]
The complexes $\cA^\bullet(E)$ and $(\E \ot \Omega_X^\bullet)^\an$ are exact in positive degrees.
\end{lemma}

\begin{proof}
Since $\nabla$ is flat, locally there is a flat frame $e_1, \dots, e_n$ so forms,
\[ \nabla \left( \sum_i e_i \ot \omega_i \right) = \sum e_i \ot \d{\omega_i} \]
and therefore we immediately reduce to $n$ copies of the (holomorphic) Poincar\'{e} lemma. 
\end{proof}

\begin{cor}
Let $(E, \nabla)$ be a flat bundle and $(\E, \nabla^{1,0})$ with $\E = \ker{\nabla^{0,1}}$ the associated holomorphic bundle. Let $\L$ be the associated local system $\L = \ker{\nabla} = \ker{(\nabla^{1,0} | \E)}$. Then the maps,
\[ \L[0] \to (\E \ot \Omega_X^\bullet) \to \cA^\bullet(E) \]
are quasi-isomorphism. Hence we obtain natural isomorphisms,
\[ H^i(X, \L) = H^i_{\dR}(X, (\E, \nabla^{1,0})) = H^i_{\dR}(X, (E, \nabla)) \]
\end{cor}

\begin{defn}
Let $(\E, \theta)$ be a Higgs bundle. Then we define the \textit{Dolbeault complex} $(\E \ot \Omega^\bullet, \theta)$,
\[ 0 \to \E \xrightarrow{\theta} \E \ot \Omega^1_X \xrightarrow{\theta \wedge} \E \ot \Omega^2_X \to \cdots \]
Note that this is a complex if and only if $\theta$ satisfies the integrabiliy condition $\theta \wedge \theta = 0$. Then we define the Dolbeault cohomology of $(\E, \theta)$,
\[ H_{\Dol}^i(X, (\E, \theta)) := \HH^i(X, (\E \ot \Omega_X^\bullet, \theta)) \]
\end{defn}

\begin{theorem}
Let $(\E, \theta)$ be a Higgs bundle and $(E, \nabla)$ a flat bundle related by the Simpson correspondence. Then there is a canonical isomorphism,
\[ H^i_{\Dol}(X, (\E, \theta)) = H^i_{\dR}(X, (E, \nabla)) \]
\end{theorem}

\begin{proof}
Show that classes on both sides have unique harmonic representatives which agree. 
\end{proof}

\begin{rmk}
A nice way to remember this is that the de Rham cohomology of flat $t$-connection $(\E_t, D_t)$ we constructed earlier,
\[ \HH^i(X, (\E_t \ot \Omega_X^\bullet, D_t)) \]
is independent of $t \in \CC$.
\end{rmk}

\begin{rmk}
The above description is exactly the proof of the Hodge decomposition which is not a coincidence. Let's give the simplest example of the Simpson correspondence. Consider the harmonic bundle $(\struct{C^\infty}, \dbar, 0, h)$ where $\dbar$ is the canonical differential and $h$ is given by $h(s_1, s_2) = s_1 \bar{s_2}$. Then $(\struct{X}, 0)$ is the associated Higgs bundle and $(\struct{\C^\infty}, \d)$ is the associated flat bundle with associated holomorphic flat bundle $(\struct{X}, \partial)$. The flat bundle has de Rham complex $(\cA^\bullet, \d)$ or the holomorphic version $(\Omega_X^\bullet, \partial)$. Likewise, the Dolbeault complex of $(\struct{X}, 0)$ is $(\Omega_X^\bullet, 0)$ hence the correspondence gives us the Hodge decomposition,
\[ H^n(X, \CC) = H_{\dR}^n(X, (\struct{X}, \partial)) = H_{\Dol}^n(X, (\struct{X}, 0)) = \HH^n\big(X, \bigoplus_j \Omega_X^j[j]\big) = \bigoplus_{p+q=n} H^p(X, \Omega_X^q) \]
\end{rmk}


Our next goal is to understand the cohomology jumping loci $\Sigma^i_k(M) \subset M$ consisting of objects $V \in M$ such that $h^i(X, V) \ge k$. Here we have not specified if $M$ is $M_{B}$ or $M_{\dR}^{ss}$ or $M_{\Dol}^{ss}$ because of the homeomorphisms $M_B \cong M_{\dR} \cong M_{\Dol}$ which we have seen respect the cohomology,
\[ H^i(X, \L) = H^i_{\dR}(X, (E, \nabla)) = H^i_{\Dol}(X, (\E, \theta)) \]
so there is a well-defined topological jumping locus. Thinking of $\Sigma^i_k$ as living in one of the particular moduli spaces corresponds to a choice of complex structure on $\Sigma^i_k$ turning it into a complex submanifold in distinct ways (actually because it is definined by coherent cohomological conditions $\Sigma^i_k$ are automatically subvarities of $M_{\dR}$ and $M_{\Dol}$ and $M_B$ with the corresponding algebraic structures). Let $M$ be the underlying topological space of these moduli spaces. Then $\Sigma^i_k \subset M$ is compatible with the complex structures $I$ arising from the identification $M = M_{\dR}^{ss}$ and $J$ from $M = M_{\Dol}^{ss}$. We first need to understand more about these complex structures.


\subsection{Hyperkahler structures}

Set $K = IJ$.

\begin{theorem}[Hitchin, Simpson, Fujiki]
On the smooth locus $M^{\sm}$ the complex structures $(I, J, K)$ endows $M^{\sm}$ with a Hyperk\"{a}hler structure. This means $(I, J, K)$ form a representation of the quaternions and there exists a metric $g$ such that $I,J,K$ are simultaneously K\"{a}hler with respect to $g$.
\end{theorem}

There are some very amusing corollaries of having quaternionic structure that arise from the following fact: every ``quaternionically holomorphic'' function is linear! Indeed, suppose $f : \HH \to \HH$ is a function such that,
\[ \lim_{h \to 0} \frac{f(q + h) - f(q)}{h} \]
exists. In particular, $f : \CC^2 \to \CC^2$ is holomorphic so we can write $f$ as a power series in $z_1, z_2 \in \CC = \left< 1, i \right>$ where $q = z_1 + z_2 j$,
\[ f(q) = \sum_{n_1, n_2 \ge 0} c_{n_1, n_2} z_1^{n_1} z_2^{n_2} \]
Taking the derivative at $q = 0$ we see that the linear term must be $c_1 q$. Then nearby the derivative is governed by quadratic terms but this is impossible. 
\bigskip\\
This is basically equivalent to the fact that there are no quaternionically bilinear forms $Q : \HH \times \HH \to \HH$ (indeed multiplication is only bilinear when considered correctly as a bimodule) for the dumb reason that,
\[ ij Q(x,y) = i Q(x, j y) = Q(ix, jy) = j Q(ix, y) = ji Q(x,y) = -ij Q(x,y) \]
However, suppose that $Y \subset X$ is a submanifold of a hyperkahler manifold whose tangent bundle is invariant under the entire quaternionic structure (we say that $Y$ is a quaternionic submanifold) then $Y$ is totally geodesic (meaning every geodesic on $Y$ is a geodesic of $X$ or equivalently it says that the restriction of the Levi-Civita connection is the Levi-Civita connection of the submanifold). Indeed, let $\nabla$ be the Levi-Civita connection on $X$ and $\pi : T X|_Y \to N_Y$ the projection to the normal bundle. Then the second fundamental form $\alpha(u,v) = \pi(\nabla_u v)$ is a quadratic form $T Y \times T Y \to N_Y$ measuring the failure of $\nabla|_Y$ to be the Levi-Civita connection on $Y$.


\begin{rmk}
We have, 
\[ \nabla_u v - \nabla_v u = [u,v] \]
and hence if $u, v \in TY$ then $[u,v] \in TY$ (it is integrable hence an involutative distribution) so $\pi(\nabla_u v) = \pi(\nabla_v u)$ proving symmetry. Now $\alpha$ is clearly linear in its first argument but we have shown it is symmetric and hence linear also in the second argument. 
\end{rmk}

However, if $Y$ is quaternionic then $\alpha$ is invariant under $I,J,K$ because $I,J,K$ are K\"{a}hler for $g$ so flat for $\nabla$ and hence $\nabla Iv = I \nabla v$ and $\nabla_{I u} = I \nabla_u$ and likewise for $J,K$. Therefore, if $\pi$ is a map of quaternion representations (i.e. if $Y$ is quaternionic) then $\alpha$ is a quaternion bilinear form hence $\alpha = 0$.

\subsection{Applications}

Applying this to the cohomology jump loci $\Sigma^i_k \subset M$ we see that $\Sigma^i_k$ are totally geodesic submanifolds of even complex dimension. 

\begin{cor}
Consider the case $r = 1$ i.e. for the group $\Gm$ then $M_B = \Gm^{b_1}$. The cohomology jump loci $\Sigma^i_k \subset M_B$ are unions of translates of subtori.
\end{cor}

\begin{proof}
Canonically $M_B$ has universal cover $H^1(X, \CC)$ and we can explicitly check that the hyperkahler structure is the standard linear one on a vector space of real dimension $4n$. But therefore the preimage of $\Sigma^i_k$ is totally geodesic and hence a union of affine space of dimension $4 m$. The image in $M_B$ is then a union of translates of a subtori.
\end{proof}

Even better, remembering that $M(\Gm)$ is a group, these loci are union of translates of what Simpson calls ``triple tori'' because they are tori in ''three different way''. Explcicity, a triple torus is a closed real analytic subgroup $N \subset M$ such that $N_{\Dol}, N_{\dR}, N_B$ with the three induced algebraic structures induced as closed sets in $M$ with its three algebraic structures are all algebraic subgroups. Since $M_{\Dol} = \A^g \times \fPic^0_X$ we see that triple tori must be a product of a linear space and an abelian subvariety of $\fPic^0_X$. However, $M_B = \Gm^{b_1}$ and the subtori must be some number of copies of $\Gm$ so we see that the dimension of the linear part and the compact part are equal! This is how we get real dimension $4m$.
\bigskip\\
What we have just said actually imposes a restriction on the topological type of the smooth projective variety $X$ we started with. To see this, consider how we compute the cohomology jump loci in $M_B(X) = \Hom{}{\pi_1(X)}{\GL_n}//\GL_n$ for any CW complex $X$. Let $\pi := \pi_1(X)$. For a representation $\rho : \pi \to \GL(V)$ the cohomology with twisted coefficients,
\[ H^i(X, V_\rho) = H^i(X, \L) \]
which is the cohomology of the associated local system $\L$ is computed as follows. Let $\wt{X} \to X$ be the universal cover and choose a CW structure such that $\pi \acts \wt{X}$ by cellular maps. Let $C_\bullet(\wt{X})$ be the cellular chain complex of $\wt{X}$. Now, 
\[ H^i(X, V_\rho) = H^i(\Hom{\pi}{C_\bullet(\wt{X})}{V_\rho}) \]
these dimensions are computed by knowing ranks of certain maps which are built, as block matrices, in terms of integer matrices computed via the cell attacking maps and blocks that are $\rho(g_i)$ for generators $g_i \in M$. Therefore, the minors are polynomials in the coordinates of the $\rho(g_i)$ and hence the $\Sigma^i_k \subset M_B(X)$ are subvarieties. By modifying the attaching map, we can write down virtually any subvariety defined over $\ZZ$. Therefore, it is easy to have $X$ for which the $\Sigma^i_k$ are not unions of translates of subtori and may not even have even complex dimension! Simpson writes down an example showing that one can attach cells to a torus such that the cohomology jump loci have the wrong dimension and hence this cannot be the homotopy type of a smooth projective variety.

\end{document}


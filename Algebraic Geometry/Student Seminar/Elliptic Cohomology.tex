\documentclass[12pt]{article}
\usepackage{hyperref}
\hypersetup{
    colorlinks=true,
    linkcolor=blue,
    filecolor=magenta,      
    urlcolor=blue,
}

\usepackage{import}
\import{../}{AlgGeoCommands}

\begin{document}

\section{Motivation}

\begin{example}
Complex $K$-theory. $K^0(X)$ is Stable isomorphism classes of complex vector bundles on $X$. With ring structure. $K^{-n}(S^n X)$ and $K^n(\Omega^n X)$. Then $K$-theory is exceptional because,
\[ K(*) = \Z[\beta, \beta^{-1}] \]
We see that,
\[ K^n(*) = 
\begin{cases}
\Z & n \text{ is even}
\\
0 & \text{else}
\end{cases} \]
\end{example}

$K$-theory is even, periodic, and multiplicative (we say ``nice''). We can make singular cohomology have these properties by definitin,
\[ H^n(X) := \prod_{k \in \Z} H^{n + 2k}_{\text{sing}}(X) \]

If $A$ is a cohomology theory that is ''nice'' we should look at,
\[ A(\CP^\infty) \cong A(*) [[t]] \]
this is a noncanonical isomorphism. This comes from Chern classes.

\begin{example}
$H^*(\CP^\infty) = \Z[[t]]$
because $\CP^\infty = K(\Z, 2)$ and thus,
\[ H^2(X, \Z) = [X, \CP^\infty] \]
so $\CP^\infty$ is also the universal object for line bundles. Given a line bundle $L$ on $X$ there is a map $f : X \to \CP^\infty$ unique up to homotopy and therefore by the above identification we get $f^* t$ the first chern class.
\end{example}

Analogously, if we have come cohomology theory $A$, we define the first Chern class $c_1^A(\struct{}(1)) = t \in A(\CP^\infty)$. Then we get, $c_1^A(L) = f^* c_1^A(\struct{}(1)) = t \in A^2(X)$.  
\bigskip\\
What happens if I consider,
\[ c_1(L_1 \ot L_2) = c_1(L_1) + c_1(L_2) \]
This does not hold for generalized cohomology theories. For example, in Complex $K$-theory,
\[ c_1^K(L_1 \ot L_2) = c_1^K(L_1) + c_1^K(L_2) + c_1(L_1) \cdot c_1(L_2) \]
The way to see this is to consider, 
\[ A(\CP^\infty \times \CP^\infty) = A(*)[[t_1, t_2]] \]
Then we can compute,
\[ c_1^A(\pi_1^* \struct{}(1) \ot \pi_2^* \struct{}(1)) = f(t_1, t_2) \in A(*)[[t_1, t_2]] \]
By naturality this gives the formula for tensor products. We have the following properties,
\begin{enumerate}
\item trivial bundles have trivial chern classes so $f(t,0) = f(0,t) = t$.
\item tensor product is symmetric $f(u,v) = f(v,u)$
\item tensor product is associative $f(f(u,v),w) = f(u, f(v,w))$
\end{enumerate}
This is a formal group law over $A(*)$. 
\bigskip\\
Given a formal group law over $A(*)$ I get a formal group,
\[ \mathrm{Spf}(A(*)[[t]]) \]
which does not depend on any choices of isomorphisms.

\begin{example}
The formal group laws,
\begin{enumerate}
\item $f(u,v) = u + v$ for singular cohomology
\item $f(u,v ) = u + v + u v$ for complex $K$-theory
\end{enumerate}
\end{example}

Quillen showed that there exists a cohomology theory called MP such that,
\[ MP(\CP^\infty) = MP(*)[[t]] \]
canonically such that,
\[ \Hom{}{MP(*)}{R} = \{ \text{formal group laws over } R \} \]
In many cases, given some $\mathbb{G} = \mathrm{Spf}(A(*)[[t]])$ and this should have some formal group law and this gives a map $f : MP(*) \to R$ and then define,
\[ A_{\mathbb{G}}^n(X) = MP^n(X) \ot_{MP(*)} R \]
but this only works if $R$ is flat. 
\bigskip\\
Let $\M_{\text{FLG}}$ be the moduli stack of formual group laws. By a theorem of Quillen $\M_{\text{FGL}} = \Spec{MP(*)}$. Let $G$ be the group scheme of automorphisms of the formal affine line $\mathrm{Spf}(\Z[[x]])$. Then we consider the moduli stack of formal groups,
\[ \M_{\text{FG}} = [\M_{\text{FGL}} / G ] \]
Using this stack, we can give a correspondence between $MP(X)$ and quasi-coherent sheaves on $\M_{\text{FG}}$. This gives better criteria for $A_{\mathbb{G}}^n$ to be a cohomology theory.
\bigskip\\
The formal groups that we get from singular cohomology and $K$-theory are the formal completions of $\Ga$ and $\Gm$ respectively. 
\bigskip\\
Maybe to find more cohomology theories we should look at formal completions of $1$-dimensional commutative algebraic groups. However, over an algebraically closed field there are not very many,
\begin{enumerate}
\item $\Ga$
\item $\Gm$
\item elliptic curves
\end{enumerate} 

\subsection{Elliptic Cohomology}

We now want to consider elliptic curves. Let $\M_{1,1}$ be the moduli stack of elliptic curves. 

\begin{thm}
There exists a unique up to homotopy lift,
\begin{center}
\begin{tikzcd}
& \{ \text{spectra} \} \arrow[d]
\\
\{ \text{elliptic curves} \} \arrow[ru, dashed] \arrow[r] & \{ \text{cohomology theories} \}
\end{tikzcd}
\end{center}
\end{thm}

\section{Lurie's A Survey of Elliptic Cohomology}

Hopkins' and Miller's insight is to lift to $E_{\infty}$-rings and give a proper sheaf of $E_{\infty}$-rings on $\M_{1,1}$.

\subsection{$E_\infty$-rings}

Let $A$ be a space equiped with a multiplication map $a : A \times A \to A$. There are some axioms you might want to hold of $A$ to make it a ring object. However, we only consider these diagrams to commute ``up to homotopy''. However, just up to homotopy is not enough. You really want to include the homotopies with higher coherence. 
\bigskip\\
Consequences: $\pi_*(A)$ is a graded ring. $\pi_n(A)$ has a natural addition structure and also Eckman-Hilton gives that addition on $A$ gives the same operation on $\pi_n(A)$ then multiplication on $A$ gives multiplication on $\pi_*(A)$. 
\bigskip\\
A map $A \to B$ of $E_{\infty}$-rings is an equivalence if it induces an isomorphism on the homotopy groups preserving the addition and multiplications thus gives a map of graded rings $\pi_*(A) \to \pi_*(B)$.
\bigskip\\
This tells me that $\pi_0(A)$ is an ordinary ring. We want to think of higher homotopy groups $\pi_n(A)$ as measuring the failure of $A$ to be a ring. 

\subsection{Spectral AG}

\begin{defn}
If $A$ is an $E_{\infty}$-ring then $\Spec{A} = (\Spec{\pi_0(A)}, \struct{})$ with a sheaf of $E_{\infty}$-rings.
\end{defn}
\end{document}


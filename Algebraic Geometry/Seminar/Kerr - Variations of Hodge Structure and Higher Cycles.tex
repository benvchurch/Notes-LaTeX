\documentclass[12pt]{article}
\usepackage{import}
\import{../}{AlgGeoCommands}

\begin{document}

\section{A}

\newcommand{\cV}{\mathcal{V}}

\subsection{$\Z$-HS}

Pure of weight $n$ is a lattice $V_\Z$ with a decreasing filtration,
\[ V_{\C} = F_0 \supset F^1 \supset \cdots \supset F^n = \{ 0 \} \]
where $V_{\C} = F^p \oplus \ol{F{n-p+1}}$. Comparison between the lattice and the filtration gives the period matrix. 
\bigskip\\
Polarization: $Q : V_{\Z} \times V_{\Z} \to \Z$ nondegenerate $(-1)^n$-symmetric such that
\begin{enumerate}
\item $Q(F^p, F^{n-p+1}) = 0$ or equivalently the decomposition $V = \bigoplus V^{p,q}$ is orthogonal
\item $i^{2 p - n} Q(\xi, \xi) > 0$ for nonzero $\xi \in V^{p,n-p} := F^p \cap \ol{F^{n-p}}$. 
\end{enumerate}

\subsection{Variations of PHS} 

Of weight $n$ over a complex manifold $S$. A VHS $\cV$ consists of
\begin{enumerate}
\item a $\Z$-local system $V_{\Z}$ over $S$
\item $Q : V_{\Z} \times V_{\Z} \to \Z$ a pairing 
\item a filtration $F^\bullet \subset V_{\cO} := V_{\Z} \ot \struct{S}$ by holomorphic vector bundles
\item a connection $\nabla : V_{\cO} \to V_{\cO} \ot \nabla$ with $V_{\Z} = \ker{\nabla}$
\end{enumerate}
such that 
\begin{enumerate}
\item on each fiber the filtration and $Q$ gives a PHS 
\item $\nabla F^p \subset F^{p-1} \ot \Omega^1_S$ for all $p$
\end{enumerate}
The first two give the data of a representation,
\[ \rho : \pi_1(S, s_0) \to \Aut((V_\Z)_{s_0}, Q_{s_0}) \]
and we define the monodromy group
\[ M = ( \ol{\rho(\pi_1)}^{\text{Zar}} )^\circ \]

\begin{rmk}
If $\pi : X \to S$ is smooth projective then it defines a VHS on cohomology with $V_{\Z} = (R^n \pi_* \ul{\Z})/\text{tors}$ and $\F^p = \R^n \pi_* \Omega^{\bullet \ge p}_{X/S}$. The connection is induced by
\[ \pi^* \Omega_S^1 \ot \Omega^{\bullet \ge p-1}_X[1] \to \Omega^{\bullet \ge p}_X \to \Omega^{\bullet \ge p}_{X/S} \]
\end{rmk}

\begin{rmk}
Let $S = \P^1 \sm \Sigma$ where $\Sigma$ is a finite set of points and $V_{\C}$ is irreducible and $h^{n,0} \neq 0$ then there is a section $\mu \in \Gamma(\P^1, \F^n_e)$ (where $\F^n_e$ is the extension to $\P^1$) such that $(V_{\cO}, \nabla) \cong \D / \D L$ for some $L \in \CC[D, t]$ is a $\D$-module. This $L$ is called the Picard-Fuchs operator. The periods:
\[ \inner{\mu}{\gamma} = \pi_\gamma(t) \]
for $\gamma \in \Gamma(S^\an, V_{\Z}^\vee) \]
satisfy $L \pi_\gamma = 0$. 
\end{rmk}

\begin{rmk}
Let $S = \Delta^*$ and let $T$ be the monodromy operator around the look. Then $T$ is quasi-unipotent meaning $T = T_{ss} Y_u$ such that $T_{ss}^m = I$ and $(T_u - I)^k = 0$ and $[T_{ss}, T_u] = 0$.  Write $N = \log{T_u}$.
\end{rmk}

($\Q-$)LMHS $\psi_t \V$ basechange such that $T$ is unipotent then $\F^\bullet \subset V_{\cO}$ extends to 
\[ \F^\bullet_e \subset V_e := e^{- \frac{\log(t)}{2 \pi i} N} V_{\Q} \ot \struct{S} \]  
which is well-defined over $\Delta$. Then $N$ is part of an $\sl_2$-tripple $(N, Y, N^+)$. Then $V = \cV_e |_{t = 0}$ has two filtrations
\begin{enumerate}
\item $F_e^\bullet |_{t = 0}$
\item $W_\bullet = W(N)[-n]$
\end{enumerate}
this defines a mixed hodge structure. 

\subsection{Example}

Conifold point: $A_1$ singularity on a CY 3-fold.

\subsection{hypergeometric variations}

Let $\alpha = (\alpha_1, \dots, \alpha_r)$ and $\beta = (\beta_1, \dots, \beta_r)$ be numbers in $(0, 1] \cap \Q$ and $\alpha_i \neq \beta_j$ for all $i, j$. We suppose that the $\ul{\alpha}$ and $\ul{\beta}$ satisfy, 
\[ q_\infty(\lambda) := \prod_j (\lambda - e^{2 \pi i \alpha_j})  \in \Q[\lambda] \]
and 
\[ q_0(\lambda) := \prod_j (\lambda - e^{2 \pi i \beta_j}) \in \Q[\lambda] \]

\begin{theorem}
There exists a geometrric $\Q$-PVHS $\mathcal{V}_{\alpha,\beta}$ over $\P^1_z \sm \{ 0, 1, \infty \}$ with $L = \prod (D + \beta_j - 1) - z \prod( D _ \alpha_j)$ such that $q_0, q_\infty$ are the char polys o $T_0$ and $T_\infty$ and period
\[ \prod = \sum_{k \ge 0} \frac{\prod [\alpha_j]_k}{\prod [\beta_j]_k} z^k \]
where $[\alpha]_k$ is the rising factorial $\alpha (\alpha + 1) \cdots (\alpha + k)$.
There is an interesting formula for the hodge numbers in terms of a zig-zag diagram. Furthermore, the monodromy group is
\begin{enumerate}
\item $\{ 1 \}$ weight zero (i.e. if $\alpha, \beta$ are intertwined: they alternate in order)
\item  $\Sp_r$ for odd weight
\item $\SO(h_{\text{even}}, h_{\text{odd}})$ for even weight,
\end{enumerate}  
\end{theorem}


\subsection{Higher Cycles}

Let $Z^p(X, r)$ be $p$-cycles on $X \times \A^r$ we define a boundary map 

$K_r(X) \cong \bigoplus_p \CH^p(X, r)$.

\end{document}
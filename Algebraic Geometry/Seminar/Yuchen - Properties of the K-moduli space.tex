\documentclass[12pt]{article}
\usepackage{import}
\import{../}{AlgGeoCommands}

\begin{document}

\section{Obstruction to rationality of conic bundles threefolds}

Let $k$ be an arbitrary field of characteristic not $2$.

\subsection{Conics}

Let $X$ be a smooth plane conic. Then $X \cong \P^1_k$ iff $X(k) \neq \empty$. (This lets you probe $2$-torsion in $\Br{k}$ since these are Brauer-Severi of index = period = 2). 

\begin{defn}
Let $X$ be an $n$-dimensional variety over $k$. Then $X$ is \textit{rational} if there exists $X \birat \P^n_k$ \textit{over} $k$.
\end{defn}

\subsection{The rationality problem for conic bundles}

Fibration $\pi : X \to B$ of conics (general fiber is a smooth conic) and all varieties are smooth projective and geometrically integral. Then there is a divisor $\Delta \subset B$ (purity) which is a branch locus where the conic is singular. 
\bigskip\\
In general $\Delta^{\text{smooth}}$ is the locus of a disjoint union of two lines, and $\Delta^{\text{sing}}$ is the locus of double lines. We restrict to the case that $\Delta$ is smooth (gemerically ordinary).

\begin{defn}
$\pi : X \to B$ is \textit{geometrically standard} if the $\varpi : \wt{\Delta} \to \Delta$ (Stein factorization of the Fano scheme of lines) is \etale and geometrically irreducible. 
\end{defn}

\begin{enumerate}
\item $\dim B = 0$ then $X$ is rational iff $X(k) \neq \empty$

\item $\dim{B} = 1$ and $X$ is minimal (no $-1$-curves in the fibers defined over the group field) then $X$ is rational iff $X(k) \neq \empty$ and $\deg{\Delta} \le 3$. If $X$ is rational then $B = \P^1$ (it can't be a conic since then it would have a rational point since $X$ does). However, $X_{\bar{k}}$ is always rational because you can contract the $(-1)$-curves over a larger field.

\item $\dim{B} = 2$ if $X$ is rational then $B$ is unirational (over $k = \bar{k}$ this implies that $B$ is rational) so let's first assume that $B = \P^2_k$. Then $X_{\bar{k}}$ is rationall iff $\deg{\Delta} \le 4$ or $\deg{\Delta} = 5$ and $\wt{\Delta} / \Delta$ corresponds to an even $\theta$-characteristic. 
\end{enumerate}

\begin{lemma}[Prakhorov]
If $B = \P^2_k$ and $X(k) \neq \empty$ and $\deg{\Delta} \le 3$ then $X$ is rational over $k$.
\end{lemma}

\begin{question}
What is happening in $\deg{\Delta} = 4$ case? Do there exist examples which are both rational and irrational over $k$?
\end{question}

For ``yes'' if $\wt{X}(k) \neq \empty$ then $X$ is rational. There are also irrational examples for $k = \RR$. 

\subsection{Family 2-18 in the Fanography}

Here $\Delta \subset \P^2$ is a degree $4$ plane curve so $g(\Delta) = 3$ and then $\wt{\Delta}$ is genus $5$ with a degree $2$ cover $\pi : \wt{\Delta} \to \Delta$. We can write,
\[ \Delta = V(Q_1 Q_3 - Q_2^2) \]
and $\wt{\Delta}$ has canonical model,
\[ V(Q_1 - r^2, Q_3 - s^2, Q_2 - rs) \subset \P^4 \]
where $Q_1, Q_2, Q_3$ are some quadrics which depend on $\pi$. 
\bigskip\\
These are $Y_{\wt{\Delta}/\Delta} \to \P^1 \times \P^2$ a degree $2$ cover branched over a $(2, 2)$-divisor with equation,
\[ z^2 = t_0^2 Q_1 + 2 t_0 t_1 Q_2 + t_1^2 Q_3 \]
where $t_0, t_1$ are the coordinates on $\P^1$. Then the projection $Y_{\wt{\Delta} / \Delta} \to \P^2$ gives a conic bundle (its a quadratic equation in $z$ which is branched over $\Delta$). However, the other projection gives a quadric surface bundle $Y_{\wt{\Delta} / \Delta} \to \P^1$. 
\bigskip\\
The minimality condition corresponds to $\pi : \wt{\Delta} \to \Delta$ being geometrically connected. Taking explicit $Q_1, Q_2, Q_3$ gives an example with $Y(\RR)$ disconnected and therefore it cannot be birational to $\P^3$ over $\RR$. How to see that $Y(\RR)$ is disconnected. Look at the fibration $Y \to \P^1$ then over the circle $\P^1(\RR)$ the 4 singular points of the quadrics the topolog of the fiber can change. In two regions it is empty (the $\RR$-points) and in two regions it is a sphere so the points are disconnected. 

\subsection{Algebraic Cycles on Conic Bundles}

Intermediate jacobian of smooth projective rationally connected $3$-folds. The hodge structure on $H^3$ is just,
\[ H^{2,1} \oplus H^{1,2} \]
since rationally connected implies $h^{3,0} = 0$. Therefore, we get a weight $1$ hodge structure which correponds to an abelian variety,
\[ IJ(X) = H^{1,2} / H^3(X, \ZZ) \]
then,
\[ IJ(\Bl_c X) \cong IJ(X) \times J(C) \]
therefore the part of $IJ(X)$ not arising from Jacobians is a birational invariant. This is a classical obstruction to rationality due to Clemens and Griffiths.

\begin{theorem}[Clemens-Griffiths]
If $X / \CC$ is rational then as ppav,
\[ IJ(X) \cong \prod J(C_i) \]
\end{theorem}

\begin{cor}
Cubic $3$-folds are not rational. 
\end{cor}

\begin{rmk}
The case they consider is the $\deg{\Delta} = 5$ and $\wt{\Delta} / \Delta$ corresponds to an odd $\theta$ characteristic.
\end{rmk}

\begin{rmk}
How do we work with the intermediate Jacobian over any field? 
\end{rmk}

There is $\Pic_{C/k}$ which is the Picard scheme. Correspondingly over a field $k$, Benoist-Wittenberg define $\CH^2_{X/k}$ codim $2$ Chow scheme (only for geometrically rational $3$-folds) which represents the Chow group functor. The identity functor gives a ppav agreeing with the intermediate jacobian. 

\begin{theorem}[Benoist-Wittenberg] 
If $X/k$ is rational then $(\CH^2_{X/k})^\circ \cong \prod \Pic^\circ_{C_i/k}$. 
\end{theorem}

However, we could also condier the other components of this groups scheme as torsors over the identity component. We have,
\[ (\CH^2_{X/k}) / (\CH^2_{X/k})^\circ \cong \NS^2(X_{\bar{k}}) \]
with the Galois action. So given $\gamma \in \NS^2(X_{\bar{k}})^{G_k}$ then get a torsor $(\CH^2_{X/k})^\gamma$ which is a torsor.

\begin{theorem}
If $X/k$ is rational and $(\CH^2_{X/k})^\circ \cong \Pic^\circ_{C/k}$ then for all $\gamma \in \NS^2(X_{\bar{k}})^{G_k}$ there exists $d \in \ZZ$ such that,
\[ (\CH^2_{X/k})^\gamma \cong \Pic^d_{C/d} \]
\end{theorem}

\begin{theorem}[Beaville]
Let $\pi : X \to \P^2$ be a conic bundle and $\Delta$ is a smooth plane quartic. Then $IJ(X) \cong \mathrm{Prym}(\wt{\Delta} / \Delta) = \ker{(\varpi_* : \Pic^\circ_{\wt{\Delta}} \to \Pic^\circ_{\Delta})}^\circ$. 
\end{theorem}

\begin{defn}
The $\struct{}(1)$-polarized Prym of $\wt{\Delta} \to \Delta$ is,
\[ \PPrym^{\struct{}(1)} = \{ [L] \in \Pic_{\wt{\Delta}} \mid [\varpi_*
\end{defn}

\begin{defn}

\end{defn}

\end{document}

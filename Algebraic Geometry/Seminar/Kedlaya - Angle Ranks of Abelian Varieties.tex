\documentclass[12pt]{article}
\usepackage{import}
\import{../}{AlgGeoCommands}

\begin{document}

\section{Angle Ranks of Abelian Varieties}

\subsection{Abelian Varieties over Finite Fields}

\newcommand{\Frob}{\mathrm{Frob}}
\newcommand{\ol}[1]{\overline{#1}}

Let $A$ be an abelian variety over $\FF_q$ then it has a Weil polynomial which is the char poly of $\Frob_q \acts H^1_{\et}(A, \Q_\ell)$. It is monic of degree $2g$,
\[ p = T^{2g} + a_1 T^{2g-1} + \cdots + a_g T^g + q a_{g-1} T^{g-1} + \cdots + q^g \]
and its roots in $\CC$ are $\alpha_1, \dots, \alpha_{2g}$ where $\alpha_{g+1} = \ol{\alpha_i}$ and $| \alpha_i | = q^{\frac{1}{2}}$. 

\begin{thm}[Honda-Tate]
There is basically a 1-to-1 correspondence between isogeny classes of abelian varieties over $\FF_q$ and $q$-Weil polynomials with integral coeffients.
\end{thm}

\begin{rmk}
``Basically'' because sometimes a $q$-Weil polynomial only gives an isogeny class over some field extension. 
\end{rmk}

The newton polygon is for $p$-adic valuation normalized so that $v(q) = 1$.


\begin{rmk}
Degenerate case straight line is supersingular. Supersingular abelian variety is always product of supersingular elliptic curves (DOESNT THIS MEAN THERE IS ONLY ONE? DOES HE MEAN ISOGENOUS TO?)
\end{rmk}

\subsection{Angle Rank}

Consider $\alpha_1, \dots, \alpha_{2g} \in \CC$ Frobenius eigenvalues. We are looking for polynomial relations. For example $\alpha_i \alpha_{g+i} = q$. Then angle rank is,
\[ \rank_{\Z} \frac{\alpha_1^\Z \alpha_2^\Z \dots \alpha_{2g}^\Z}{q^\Z} = \rank_{\Z} (\Z \arg(\alpha_1) + \Z \arg(\alpha_2) + \cdots + \Z \arg(\alpha_{2g})) / \Z 2\pi  \]

\subsection{Angle Rank and the Tate Conjecture}

The Tate conjecture: eigenvalue $q^i$ on $H^{2i}(A)$ is entirely explained by cycle classes menaing everything in the eigenspace is spanned by cycle classes (equivalent to twisting by $i$ and considering invariants). This is true for $i = 1$ by Tate (similar to Lefschetz $(1,1)$-theorem for case $i = 1$ of Hodge conjecture). Also true for any $A$ for which all $q^i$-eigenvalues are generated in codimension $1$ which happends iff angle rank = $g$ (generic).

\begin{example}
$A$ supersingular iff angle rank $ = 0$ (boils down to alg integers with conjugates on unit circle).
\end{example}

\subsection{A Theorem of Tankeev}

Let $g = \dim{A}$ and consider $A$ absolutely irreducible (meaning not isogenous to a product)

\begin{thm}[Tankeev, 1984]
If $g$ is prime then angle rank of $A$ is in $\{1, g-1, g \}$ and all occur. 
\end{thm}

WHY DO ABELIAN VAR OVER FIN FIELDS CORRESPOND TO CM ABELIAN VARIETIES??

\begin{defn}
An abelain variety $A$ is almost ordinary if its newton polygon is $(0,0) \to (g-1,0) \to (g+1, 1) \to (2g,g)$. This is codimension $1$ in moduli. 
\end{defn}

\begin{thm}[LEnstra-Zarhin]
If $A$ is almost ordinary then,
\begin{enumerate}
\item if $g$ is even then angle rank $= g$
\item if $g$ is odd then angle rank $\ge g - 1$. 
\end{enumerate}
\end{thm}

\begin{rmk}
This is also true if the newton slopes look like this $2$-adically e.g. $1/3, 1/3, 1/3, 1/2,1/2, 2/3, 2/3,2/3$ only 2 slopes.
\end{rmk}

\subsection{Slope Vectors and the angle rank.}

Let $V \subset \Q^g$ be the subspace spanned by slope vectors. Let $\beta_i = \frac{\alpha_i}{\ol{\alpha}_i}$ then $(v(\beta_1), \dots, v(\beta_g))$ for each valuation of $\Q(\beta_1, \dots, \beta_g)$ above $p$. Then $\dim{V} = $ angle rank which is a $\Q$-representation of some finite group. Let $G = \Gal{\Q(\alpha_1, \dots, \alpha_{2g})/\Q}$ then get a sequence,
\begin{center}
\begin{tikzcd}
1 \arrow[r] & C \arrow[d] \arrow[r] & G \arrow[d, hook] \arrow[r] & \bar{G} \arrow[r] & 1
\\
1 \arrow[r] & \Z_2^g \arrow[r] & \Z_2^g \rtimes S_g \arrow[r] & S_g \arrow[r] & 1
\end{tikzcd}
\end{center}
$C$ is the code of $A$ in the sense of ``binary linear code''. Then $G \acts V$ and the constracts on dimension of $G$-reps give constaints on the angle ranks (e.g. Tankeev). 

\subsection{Effects of the Code on the angle rank}

\begin{thm}
Suppose $\bar{G}$ acts primitively on $\left< 1, \dots, g \right>$ (meaning no nontrivial partition which is acted upon by the group). Notice that $(1, \dots, 1) \in C$ corresponding to complex conjugation is $C$ is not generated by this element then $A$ has maximal angle rank (meaning $=g$). 
\end{thm}

\begin{thm}[Effective Zarhin]
Let $A$ be abs. simple AB over $\FF_q$ and $\dim{A} = g$. Let $\alpha_1, \dots, \alpha_{2g}$ be the Frob eigenvalues and $G = \Gal{\Q(\alpha_1, \dots, \alpha_{2g})/\Q}$ and $\delta$ the angle rank. Then the vectors $(e_1, \dots, e_{2g}) \in \Z^{2g}$ for which $\alpha_1^{e_1} \cdots \alpha_{2g}^{e_{2g}} \in q^{\Z}$ is generated by vectors of weight at most,
\[ |G|(|G|-\delta)^3(g \delta)^\delta \]
and we know $|G| \le 2^g g!$. 
\end{thm}

$G = \Gal{\Q(\alpha_1, \dots, \alpha_{2g})} \acts V$ by signed permutations because the second half of the $\alpha_j$ are conjugate to the first half up to multplies of $q$ so we get only have the $\beta$ are interesting and elements of Galois group exchaning $\alpha_i$ and $\alpha_j$ might change $\beta$ to $\beta^{-1}$. 

\begin{thm}
Hodge conj for all CM AB ove $\CC$ implies Tate for all AB over finite fields. 
\end{thm}

\end{document}

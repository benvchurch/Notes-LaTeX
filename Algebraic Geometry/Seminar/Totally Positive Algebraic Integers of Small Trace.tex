\documentclass[12pt]{article}
\usepackage{import}
\import{../}{AlgGeoCommands}

\newcommand{\tors}{\mathrm{tors}}
\renewcommand{\C}{\mathbb{C}}
\usepackage{mathdots}
\newcommand{\atr}{\mathrm{atr}}

\begin{document}

\begin{defn}
Take $\alpha \in \CC$ to be an algebraic integer with minmal polynomial,
\[ p(z) = z^n + a_{n-1} z^{n-1} + \cdots + a_n \]
Then,
\[ \tr{\alpha} = \sum_{\text{conjugates}} \sigma(\alpha) = - a_{n-1} \]
Then the absolute trace is,
\[ \atr{\alpha} = \frac{1}{\deg{\alpha}}\sum_{\text{conjugates}} \sigma(\alpha) = - \frac{a_{n-1}}{n} \]
\end{defn}

Call $\alpha$ totally positive (TPAI = totally positive algebraic integer) if all roots of $p$ are positive reals (if $\Q(\alpha)$ is totally real and $\alpha$ always embeds as a positive number). 
\bigskip\\
There are TPAIs with $\atr = 1, \frac{3}{2}, \frac{5}{3}, \dots$ approaching $2$. If $\zeta^p = 1$ and $\zeta \neq 1$ and $p$ is an odd prime then $\zeta + \zeta^{-1} + 2$ as $\atr < 2$.

\subsection{Schur-Siegel-Smyth Trace Problem}

Choose an enummeration $\alpha_1, \alpha_2, \dots$ of TPAI define,
\[ \lambda_{SSS} = \liminf_{k \to \infty} \atr(\alpha_k) \le k \]
We want to show that $\lambda_{SSS} = 2$.

\subsection{Serre}

There are lots of results like $\lambda_{SSS} > 1.79$. Serre showed that their method cannot prove $\lambda_{SSS} > \lambda_{\text{Serre}}$ where $\lambda_{\text{Serre}} \simeq 1.899$.

\subsection{What Alex Showed}

Proved that $1.809 > \lambda_{SSS} > 1.802$ so the conjecture is false.

\subsection{Methods}

We two methods.

\subsubsection{Integrality of Resultants}

Suppose $\alpha$ is an AI with conjugates $\alpha_1, \dots, \alpha_n$. Take $\Q$ be an integer polynomial and suppose that $Q(\alpha) \neq 0$ then,
\[ \prod_{i \le n} |Q(\alpha_i)| \ge 1 \]

\subsubsection{Integrality of Discriminants}

\[ \prod_{i < j} |\alpha_i - \alpha_j|^2 \ge 1 \]

\subsubsection{Analysis}

\begin{defn}
Given $\alpha \in \CC$ take $\delta_\alpha$ to be the Borel measure on $\C$ defined by,
\[ \delta_\alpha(Y) = 
\begin{cases}
1 & \alpha \in Y 
\\
0 & \alpha \notin Y 
\end{cases} \]
Given $P(z) = a_n(z - \alpha_1) \cdots (z - \alpha_n)$ then we define,
\[ \mu_P = \tfrac{1}{n}(\delta_{\alpha_1} + \cdots + \delta_{\alpha_n}) \]

\begin{prop}
If $\alpha$ is an AI with min poly $P$ and given an int poly $Q$ with $P + Q$ then,
\[ \int \log{|Q(z)|} \d{\mu_O} \ge 0 \]
\end{prop}

\subsubsection{Smyth Approach}

Find integer polynomials $Q_1, \dots, A_N$ and positive numbers $a_1, \dots, a_N, \lambda$ so that ,
\[ t \ge \lambda + \sum_{i \le N} a_i \log{(Q_i(t)} \]
for $t \ge 0$. Suppose $\alpha$ is TPAI with min poly $P$ and $p \ndivides Q_1, \dots, Q_N$ then,
\[ \atr(\alpha) = \int t \d{\mu_P}(t) \ge \int \lambda \d{\mu_P} + \sum_{i \le N} a_i \int \log{|Q_i|} \d{\mu_P} \ge \lambda \]
The exception list is exactly the roots of the $Q_1, \dots, Q_N$. 
\end{defn}

\begin{prop}
Suppose every probability measure $\mu$ on $\RR^{\ge 0}$ satisfying,
\begin{enumerate}
\item $\int \log{|Q_i|} \d{\mu} \ge 0$
\item $I(\mu) = \int \int \log{|z-w|} \d{\mu(z)} \d{\mu(w)} \ge 0$ 
\end{enumerate}
also satisfies,
\[ \int t \d{\mu} \ge \lambda \]
Then $\lambda_{SSS} \ge \lambda$.

\subsubsection{Schur}

Uses no polynomials but incorporates the discriminant.

\subsubsection{Siegel}

Uses only one polynomial $Q_1(x) = x$. 

\subsubsection{Alexander Smith}


Uses about $10$ polynomials. Key ingredients in the proof, given a sequence of prob. measures $\mu_1, \mu_2, \dots$ on $\P^1_\C$ there is a measure $\mu$ and subsequence, $n_1, n_2, \dots,$ so $\mu_{n_k} \to \mu$ converge in the weak-$*$ topology (this is because $\mu_i(\P^1_\C) = 1$ for all $i$ so this is in the unit ball). Also for every $f : \CC^\infty \to \RR$ continuous then,
\[ \lim_{k \to \infty} \int f \d{\mu_{n_k}} = \int f \d{\mu} \]
\end{prop}

\begin{rmk}
This is just because $C(X)^*$ is the space of finite sined measures on $X$ and the probability measures are the unit sphere which is inside the unit ball which is weak-$*$ compact. For what it's worth $(L^1)^* = L^\infty$ and $(L^\infty)^*$ are finitely-additive signed measures absolutely continuous with respect to the measure defining $L^\infty$. 
\end{rmk}

\begin{thm}
Let $\Sigma$ be a finite union of closed bounded real intervals. 
\end{thm}

\end{document}

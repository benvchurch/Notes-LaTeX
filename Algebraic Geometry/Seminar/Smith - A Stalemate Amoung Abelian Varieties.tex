\documentclass[12pt]{article}
\usepackage{import}
\import{../}{AlgGeoCommands}

\begin{document}

\section{A Stalemate among Abelian Varieties over $\FF_q$}

\subsection{A new construction for algebraic integers}

\begin{defn}
A complex number $\alpha \in \CC$ is an algebraic integer if there exists some monic integer polynomial $p \in \Z[x]$ such that $p(\alpha) = 0$.
\end{defn}

\begin{defn}
The minimal polynomial of $\alpha$ is the unique polynomial of the above form with minimal degree. The roots of the minimal polynomial are called the conjugates of $\alpha$.
\end{defn}

\begin{defn}
Take an algebraic integer $\alpha$ with conjugates $\alpha_1, \dots, \alpha_n$. Then $\deg{\alpha} = n$ and $\tr{\alpha} = \alpha_1 + \cdots + \alpha_n$. Call $\alpha$ totally positive if $\alpha_1, \dots, \alpha_n > 0$.
\end{defn}

\begin{example}
Consider $\zeta_7$ a primitive root of $x^7 - 1$. Then $\deg{\zeta_7} = 6$ and $\tr{\zeta_7} = -1$. 
\end{example}

\begin{example}
Consider $\beta = \zeta_7 + \bar{\zeta_7}$ then $\tr{\zeta_7 + \bar{\zeta_7}} = -1$. Then we can take $\gamma = \zeta_7 + \bar{\zeta_7} + 2$ then $\gamma$ is totally positive with $\deg{\gamma} = 3$ and $\tr{\gamma} = 5$.
\end{example}

\begin{example}
If you take $\beta_n = \zeta_n + \bar{\zeta_n} + 2$ these are totally positive. The distribution of these conjugates follows $(1 - (x/2-1)^2)^{-\frac{1}{2}}$ in the interval $[0,4]$.
\end{example}

\begin{rmk}
Shur - Siegal - Smith conjectured for any $\epsilon > 0$ there are only finitely many TPAIs  $\alpha$ such that,
\[ \tr{\alpha} < (2 - \epsilon) \deg{\alpha} \]
\end{rmk}

\begin{thm}[Smith]
There are infinitely many TPIA $\alpha$ so,
\[ \tr{\alpha} < 1.809 \deg{\alpha} \]
\end{thm}

\begin{defn}
Let $\Sigma$ be a real interval of length greater than $4$. Call an algebraic integer $\alpha$ a $\Sigma$-algebraic integer if all its conjugates lie in $\Sigma$. Given such an $\alpha$ we define a measure on Borel sets,
\[ \mu_\alpha(Y) = \frac{\# \{ \beta \in Y \mid m_\alpha(\beta) = 0 \}}{\deg{\alpha}} \] 
\end{defn}

\begin{defn}
We say that a sequence of measures $\mu_i$ weak-$*$ converges $\mu_i \to \mu$,
\[ \lim_{i \to \infty} \int f \d{\mu_i} = \int f \d{\mu} \]
for all continuous $f : \Sigma \to \RR$.
\end{defn}

\begin{rmk}
This is weak-$*$ convergence in the dual space of $C(\Sigma)$. (CHECK THIS)
\end{rmk}

\begin{defn}
Given $\mu$ on $\Sigma$ consider the conditions,
\begin{enumerate}
\item[(A)] there is a sequence $\alpha_i$ of distinct $\Sigma$-AIs such that $\mu_{\alpha_i} \to \mu$
\item[(B)] for every nonzero integer polynomial $Q$,
\[ \int_{\Sigma} \log{|Q|} \d{\mu} \ge 0 \]
\end{enumerate}
\end{defn}

\begin{prop}
Condition (A) implies (B).
\end{prop}

\begin{proof}
Given $Q$ and $\alpha$ with conjugates $\alpha_1, \dots, \alpha_n$. Consider,
\[ \beta = \prod_{i \le n} Q(\alpha_i) \]
which is invariant under every Galois automorphism and thus $\beta \in \Z$. Furthermore,
\[ \left| \prod_{i \le n} Q(\alpha_i) \right| \ge 1 \]
unless $Q(\alpha) = 0$. Therefore,
\[ \int \log{|Q|} \d{\mu_\alpha} \ge 0 \]
Then by dominated convergence,
\[ \int \log{|Q|} \d{\mu} \ge 0 \]
\end{proof}

\begin{example}
Let $Q$ be the minimal polynomial for some other algebraic integer $\beta$. Then,
\[ \int \log{|z - w|} \d{\mu_\alpha} \d{\mu_\beta} \ge 0 \]
showing that the conjugates of $\alpha$ and $\beta$ are spaced apart.
\end{example}

\begin{thm}[Smith (2021)]
Condition (B) implies condition (A).
\end{thm}

\begin{rmk}
Previously this required an energy condition,
\[ I(\mu) = \int \int \log{|z - w|} \d{\mu(z)} \d{\mu(w)} \ge 0 \]
\end{rmk}

\subsection{Abelian Varieties}

We require the Weil conjectures but actually only the parts that were known to Weil.

\begin{theorem}[Weil]
Given a ``nice'' curve $C / \FF_q$ of genus $g$, there are AIs,
\[ \lambda_1, \dots, \lambda_g \]
suh that $|\lambda| = q^{\frac{1}{2}}$ for $i \le g$ such that,
\[ \# C(\FF_{q^n}) = q^n + 1 - (\lambda_1^n + \bar{\lambda}_1^n + \cdots + \lambda_g^n + \bar{\lambda}_g^n) \]
\end{theorem}

\begin{thm}[Weil]
Given $A / \FF_q$ of dimension $g$, there exist,
\[ \lambda_1, \dots, \lambda_g \]
with $|\lambda| = q^{\frac{1}{2}}$ and,
\[ \# A(\FF_{q^n}) = \prod_{i \le g} (\lambda_i^n - 1)(\bar{\lambda}^n_i - 1) \]
If $A$ is the Jacobian of $C$, these $\lambda_i$ agree with the previous ones for $C$.
\end{thm}

\subsection{Honda - Tate Theory}

Consider the map $A \mapsto \lambda_1 + \bar{\lambda}_1$. This gives,
\[ \{ \text{simple } A / \FF_q \text{ up to isogeny} \} \to \{ \text{AIs } \alpha \text{ with conjugates in } [-2\sqrt{q}, 2\sqrt{q}] \text{ up to conjugacy} \} \]
(Note: you can write any such $\lambda \in [-2\sqrt{q}, 2\sqrt{q}]$ as a sum of a Weil $q$ number and its conjugate under $z \mapsto z + \frac{q}{z}$
Tate showed that this map is injective. Honda showed that this map is bijective.
\bigskip\\
Notice that,
\[ \# A(\FF_{q^n}) = \prod_{i \le g} (\lambda_i^n - 1)(\bar{\lambda}^n_i - 1)  = \prod_{i \le n} (q + 1 - \lambda_i - \bar{\lambda}_i) \]
Therefore, 

\end{document}

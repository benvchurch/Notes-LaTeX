\documentclass[12pt]{extarticle}
\usepackage{import}
\import{./}{KCommands}
\newcommand{\R}{\mathbb{R}}

\begin{document}

\section{Stable Equivalence of Vector Bundles}

\begin{definition}
Let $E_1, E_2$ be vector bundles on $X$ and $\E^n$ be the rank $n$ trivial vector bundle. Then we say that $E_1$ and $E_2$ are stably equivalent $E_1 \sim_s E_2$ if there is some $n \in \Z_+$ such that,
\[ E_1 \oplus \E^n \cong E_2 \oplus \E^n \]
Furthermore we say that $E_1$ and $E_2$ are weakly equivalent $E_1 \sim E_2$ if there exist $n,m \in \Z_+$ such that,
\[ E_1 \oplus \E^n \cong E_2 \oplus \E^m \]
\end{definition}

\begin{example}
Consider $X = S^1$ and the M\"{o}bius bundle $\mu$. Then $\mu \not\cong \E^1$ since $\mu$ has no nonvanishing global sections. However, I claim that,
\[ \mu \oplus \E^1 \cong \E^2 \]
so $\mu \cong_s \E^1$. To see this I must find two everywhere linearly independent sections. 
\bigskip\\
We describe $\mu$ explicitly as the tautological bundle of $S^1 = \RP^1$,
\[ \mu = \{ (x, v) \mid x \in \RP^1, v \in \mathrm{Span}(x) \subset \R^2 \} = \{ (e^{2 \pi i t}, v) \mid t \in [0, 1] \: \: v \in \mathrm{Span}(e^{\pi i t}) \} \]
Then take, $s_1, s_2 \in \Gamma(X, \mu \oplus \E^2)$ to be,
\[ s_1(t) = (e^{2 \pi i}, e^{\pi i t} \cos{(\pi t)}) \oplus (e^{2 \pi i}, \sin{(2 \pi t)}) \quad \quad s_2(t) = (e^{2 \pi i t}, e^{\pi i t} \sin{(\pi t)}) \oplus (e^{2 \pi i t}, \cos{(2 \pi t)}  \]
\end{example}

\section{Definition of Topological K-Theory}


\section{Questions for March 3}

\begin{enumerate}
\item Split exact structure on a cateogry is a type of exact structure (right?)
\item So if I have a split exact category $C$ then $K_0(C) = K_0^\oplus(C)$ (correct?)
\item However, if $A$ is an abelian vategory then $K_0(A)$ is equivalent to the grothendieck group of $A$ as an exact category while $K_0^\oplus(A)$ is the grothendieck group of $A$ as a split exact category. These are not the same.
\item Why is the category of vectorbundles on a scheme not split exact? Vector bundles are locally free and thus flat but I know not projective in $\shMod{\struct{X}}$. However, are they projective in $\QCoh{\struct{X}}$? 
\item Okay so it seems that if we let $X = \P^1_R$ then,
\begin{center}
\begin{tikzcd}
0 \arrow[r] & \struct{\P^1}(-2) \arrow[r] & \struct{\P^1}(-1) \oplus \struct{\P^1}(-1) \arrow[r] & \struct{\P^1} \arrow[r] & 0
\end{tikzcd}
\end{center}
cannot split because there are no maps $\struct{\P^1} \to \struct{\P^1}(-1)$ since $\struct{\P^1}(-1)$ has no global sections.
\item In the noetherian case. What is the notion of pseudo-coherent? We say $M$ is pseudo-coherent if it has a resolution by f.g. projectives. So then $M$ is finitely presented. Then $G_0(R) = K_0(\text{pseudo-coherent})$ as an exact category. Why is this the correct notion rather than taking finitely presented. 
\end{enumerate}

\end{document}

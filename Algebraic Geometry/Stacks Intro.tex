\documentclass[12pt]{article}
\usepackage{import}
\import{./}{AlgGeoCommands}
\renewcommand{\U}{\mathfrak{U}}

\begin{document}

\section{Definitions}

\newcommand{\fppf}{\mathrm{fppf}}
\newcommand{\Set}{\mathrm{Set}}
\newcommand{\Def}{\mathrm{Def}}
\newcommand{\Inf}{\mathrm{Inf}}
\renewcommand{\X}{\mathcal{X}}
\newcommand{\Y}{\mathcal{Y}}
\newcommand{\Spf}[1]{\mathrm{Spf} \! \left(#1 \right)}
\renewcommand{\F}{\mathcal{F}}
\renewcommand{\G}{\mathcal{G}}


\begin{defn}
An \textit{algebraic space} is a functor $X : (\Sch_S)_{\fppf}^{\op} \to \Set$ such that,
\begin{enumerate}
\item $F$ is a sheaf in the $\fppf$ topology
\item the diagonal $\Delta_{X/S} : X \to X \times_S X$ is representable by schemes
\item there is a scheme $U$ and an \etale surjection $U \onto X$.
\end{enumerate}
\end{defn}

\begin{defn}
An \textit{algebraic stack} is a category fibered in groupoids $\X \to (\Sch_S)_{\fppf}$ such that,
\begin{enumerate}
\item $\X$ is a stack in the $\fppf$ topology
\item $\Delta_{\X / S} : \X \to \X \times_S \X$ is representable by algebraic spaces 
\item there is an algebraic space $U$ and an \etale surjection $U \onto \X$.
\end{enumerate}
\end{defn}

\begin{rmk}
The map $U \to \X$ is only necessarily representable by algebraic spaces so to express the property of being an \etale surjection consider any map from a scheme $T \to \X$ and an \etale cover from a scheme $V \to U \times_{\X} T$ in the diagram,
\begin{center}
\begin{tikzcd}
V \arrow[r] \arrow[rr, bend left, "\et \text{ surj}"] & U \times_{\X} T \arrow[d] \arrow[r] & T \arrow[d]
\\
& U \arrow[r] & \X  
\end{tikzcd}
\end{center}
This property is independent of the choice of \etale cover $V \to U \times_{\X} T$ by \etale descent for \etale surjective morphisms.
\end{rmk}

\begin{rmk}
Why do we only require that $\X$ be smooth locally an algebraic space and its diagonal be representable by only algebraic spaces? The diagonal is closely related to the automorphism groups of objects $\X$ parametrizes. 
\end{rmk}

(PRODUCTS OF STACKS)

(INERTIA)

(STABILIZERS)

\section{Presentations}

\begin{prop}
Let $X$ be an algebraic space over $S$ and $f : U \onto X$ an \etale surjection from a scheme $U$. Set $R = U \times_X U$ in the pullback diagram,
\begin{center}
\begin{tikzcd}
R \arrow[d] \arrow[r] & U \arrow[d]
\\
U \arrow[r] & X 
\end{tikzcd}
\end{center}
then we have,
\begin{enumerate}
\item $j : R \to U \times_S U$ is a monomorphism and $R(T) \subset U(T) \times U(T)$ is an equivalence relation for all $T \to S$
\item the projections $s,t : R \to U$ are \etale
\item the diagram,
\begin{center}
\begin{tikzcd}
R \arrow[r, shift left, "s"] \arrow[r, shift right, "t"'] & U \arrow[r] & X
\end{tikzcd}
\end{center}
is a coequalizer in $\mathrm{Sh}((\Sch_S)_{\fppf})$. 
\end{enumerate}
\end{prop}

\begin{proof}
The first two are immediate 

The last follows in any category of sheaves given that $U \to X$ is surjective. 
\end{proof}

(PRESENTATIONS OF STACKS)

\section{Examples}

(BG)
(MG)
(AG)


\section{Infinitesimal Deformation Theory}

\begin{rmk}
First we recall how to apply infinitessimal deformation theory in the relative setting. In the basic case, we want to probe properties of a morphisms of schemes $f : X \to S$ near a finite type point $x : \Spec{k} \to S$. There is some affine open $\Spec{\Lambda} \subset X$ containing $x$. Then we need to consider Artinian local rings $A$ and diagrams,
\begin{center}
\begin{tikzcd}
& & & X \arrow[d, "f"]
\\
\Spec{k} \arrow[r] & \Spec{A} \arrow[r] \arrow[rru, dashed] & \Spec{\Lambda} \arrow[r, hook] & S 
\end{tikzcd}
\end{center}
and consider the set of dashed arrows. This means our base category should be the category of local Artinian $\Lambda$-algebras with residue field $k$.
\end{rmk}

\begin{defn}
Let $\Lambda$ be a Noetherian ring and $\Lambda \to k$ a finite ring map with $k$ a field. Let $\C_\Lambda$ be the category of,
\begin{enumerate}
\item $(A, \varphi)$ where $A$ is an Artinian local $\Lambda$-algebra and $\varphi : A / \m_A \to k$ a $\Lambda$-algebra isomorphism
\item morphisms $f : (B, \psi) \to (A, \varphi)$ are local $\Lambda$-algebra maps such that $\varphi \circ (f \text{ mod } \m_A) = \psi$
\end{enumerate}
\end{defn}

\begin{rmk}
As in the absolute case (which corresponds to $\Lambda = k$) we can factor any extension $B \onto A$ into \textit{small} extensions $\varphi : B' \onto A$ where $\ker{\varphi}$ is principal and annihilated by $\m_B$.
\end{rmk}

\begin{defn}
Let $\Lambda$ be a Noetherian ring and let $\Lambda \to k$ be a finite ring map where $k$ is a field. Define the category $\widehat{\C}_\Lambda$ of,
\begin{enumerate}
\item pairs $(R, \varphi)$ where $R$ is a Noetherian complete local $\Lambda$-algebra and $\varphi : R / \m_R \to k$ is a $\Lambda$-algebra isomorphism,
\item morphisms $f : (S, \psi) \to (R, \varphi)$ are local $\Lambda$-algebra map such that $\varphi \circ (f \text{ mod } \m_S) = \psi$.
\end{enumerate}
\end{defn}

\begin{rmk}
Then $\C_{\Lambda} \subset \widehat{\C}_{\Lambda}$ is naturally a full subcategory. 
\end{rmk}

\subsection{Deformation Functors}

\begin{defn}
A \textit{predeformation functor} is a functor $F : \C_{\Lambda} \to \Set$ such that $F(k) = \{ * \}$.
\end{defn}

\begin{rmk}
The condition $F(k) = \{ * \}$ corresponds to choosing a fixed base object for the deformations.
\end{rmk}

\begin{defn}
Given a predeformation functor $F : \C_{\Lambda} \to \Set$ we extend it to $\wh{F} : \wh{\C}_{\Lambda} \to \Set$ via,
\[ \wh{F}(R) = \varprojlim_n F(R/\m_R^n) \]
A functor $F$ is \textit{pro-representable} if $\wh{F}$ is representable.
\end{defn}

\begin{defn}
Let $F : \C_\Lambda \to \Set$ be a predeformation functor. A \textit{hull}\footnote{Some authors use the terminology \textit{miniversal} formal object. However, in the deformation category setting, a minimal versal object may not induce an isomorphism of the tangent space so we reserve the term \textit{miniversal} for a minimal versal object see \href{https://stacks.math.columbia.edu/tag/06T0}{Tag 06T0}.} for $F$ is a pair $(R, \eta)$ where $R \in \wh{\C}_{\Lambda}$ and $\eta \in \wh{F}(R)$ such that $h_R \to F$ is formally smooth and bijective on tangent spaces.
\end{defn}

\begin{rmk}
Let $k[\epsilon]$ be the ring $k[\epsilon]/(\epsilon^2)$ with the trivial $\Lambda$-algebra structure. 
\end{rmk}

\begin{defn}
Let $F : \C_\Lambda \to \Set$ be a predeformation functor. If $A' \to A$ and $A'' \to A$ are morphisms in $\C_\Lambda$ there is a natural map,
\[ (*) \quad F(A' \times_A A'') \to F(A') \times_{F(A)} F(A'') \]
Then Schlessinger's conditions are as follows,
\begin{enumerate}
\item[(H1)] if $A'' \onto A$ is a small thickening then $(*)$ is surjective
\item[(H2)] if $A = k$ and $A'' = k[\epsilon]$ then $(*)$ is bijective
\item[(H3)] $T_F = F(k[\epsilon])$ is finite-dimensional
\item[(H4)] if $A'' = A'$ and $A' \to A$ is a small thickening, then $(*)$ is bijective.
\end{enumerate}
\end{defn}

\begin{defn}
A predeformation functor $F : \C_{\Lambda} \to \Set$ is a \textit{deformation functor} if it satisfies (H1) and (H2).
\end{defn}

\begin{theorem}[Schlessinger]
Let $F : \C_{\Lambda} \to \Set$ be a deformation functor. Then,
\begin{enumerate}
\item $F$ admits a hull if and only if it satisfies (H3)
\item $F$ is prorepresentable if and only if it satisfies (H3) and (H4).
\end{enumerate}
\end{theorem}

\begin{example}
Let $X$ be a scheme, the functor,
\[ \Def_X : A \mapsto \{ (X', \varphi) \mid X' \text{ flat } A\text{-scheme with } \varphi : X' \ot_A k' \iso X \} / \cong \]
is a deformation functor. 
\end{example}

\begin{example}
Let $X = \Spec{k[x,y]/(xy)}$ and $F = \Def_X$. If $A$ is a finite type $k$-algebra and $P \onto A$ is a presentation from a polynomial ring with kernel $K$ then [H, Ex. 9.8] shows that,
\begin{center}
\begin{tikzcd}
\Hom{A}{\Omega_{P/k} \ot_k A}{A} \arrow[r] & \Hom{A}{J/J^2}{A} \arrow[r] & T_{\Def_A} \arrow[r] & 0
\end{tikzcd}
\end{center}
arising from the conormal exact sequence,
\begin{center}
\begin{tikzcd}
J/J^2 \arrow[r] & \Omega_{P/k} \ot_P A \arrow[r] & \Omega_{A/k} \arrow[r] & 0
\end{tikzcd}
\end{center}
In our case, let $P = k[x,y]$ and $J = (xy)$. Then we have,
\begin{center}
\begin{tikzcd}
A \partial_x \oplus A \partial_y \arrow[r] & A \arrow[r] & T_{\Def_A} \arrow[r] & 0
\end{tikzcd}
\end{center}
and therefore $T_{\Def_A} = A/(x,y) = k$. Thus $\Def_X$ satisfies (H1) - (H3) so it should have a hull. Indeed,
\[ (k[[t]], \, \Spf{k[[t]][x,y]/(xy - t)}) \]
is a hull (note the formal object is effective). Let's first understand why this hull is not a pro-representing object. For any map, $\varphi : k[[t]] \to A$ the induced object,
\[ \varphi_*(\Spf{k[[t]][x,y]/(xy - t)}) = \Spec{A[x,y]/(xy - \varphi(t))} \]
is unchanged (in isomorphism class) if we replace $\varphi$ buy $\varphi' = u \varphi$ for any unit $u \in A$ since then we can scale $x$ or $y$ to remove $u$. However, recall that a deformation $X'$ is equipped with a distinguished isomorphism $\varphi : X' \ot_A k \iso X$ with which isomorphisms of deformations must be compatible. Therefore, $\varphi' = u \varphi$ and $\varphi$ define the same deformation if $u \in A^\times$ is a unit and $u \equiv 1 \mod \m_A$. Therefore, the map, $h_R \to \Def_X$ is not injective for general $A$ but is injective for $A = k[\epsilon]$ (since $(1 +  a \epsilon) \cdot \epsilon = \epsilon$ so multiplication by such $a$ does nothing) as must be true for a hull.  
\bigskip\\
However $\Def_X$ is not pro-representable since it does not satisfy (H4). Indeed, consider $A = k[\epsilon]/(\epsilon^3)$ and consider,
\[ \Def_X(A \times_k A) \to \Def_X(A) \times \Def_X(A) \]
I claim this is not injective. Indeed, $t = \epsilon_1 + \epsilon_2$ and $t = \epsilon_1 + \epsilon_2 + \epsilon_1 \epsilon_2$ map to the same pair of deformations but I claim they are not related by such a unit. Write,
\[ u = 1 + a \epsilon_1 + b \epsilon_2 + O(\epsilon^2) \]
then,
\[ u (\epsilon_1 + \epsilon_2) = \epsilon_1 + \epsilon_2 + a \epsilon_1^2 + (a + b) \epsilon_1 \epsilon_2 + b \epsilon_2^2 + O(\epsilon^3) \] 
and we cannot have $a = b = 0$ but $a + b = 1$.
\end{example}

\begin{rmk}
The above illustrates why it is necessary to define deformations of a scheme as equipped with a distinguished isomorphism $\varphi : X' \ot_A l \iso X$ otherwise $\Def_X$ will not be a deformation functor.
\end{rmk}


\subsection{Deformation Categories}

\begin{defn}
A \textit{predeformation category} is a category cofibered in groupoids $\F \to \C_{\Lambda}$ such that $\F(k)$ is equivalent to the trivial category. 
\end{defn}

\begin{rmk}
Let $\F$ be a predeformation category and $x_0 \in \F(k)$. Then for any $x \in \F$ over $A$ let $q : A \to k$ then there is a pushforward $x \to q_* x$ and $q_* x \in \F(k)$ so there is a unique isomorphism $q_* x \iso x_0$ and hence there is a canonical morphism $x \to x_0$ in $\F$. 
\end{rmk}

\begin{rmk}
If $F : \C_\Lambda \to \Set$ is a predeformation functor then the associated cofibered set $\F_F \to \C_{\Lambda}$ is a predeformation category. Likewise, if $\F \to \C_\Lambda$ is a predeformation category then the functor of isomorphism classes $\overline{\F} : \C_{\Lambda} \to \Set$ is a predeformation functor. 
\end{rmk}

\begin{defn}
Let $\F \to \C_\Lambda$ be a category cofibered in groupoids. The \textit{category of formal objects of} $\wh{\F}$ is the category of,
\begin{enumerate}
\item formal objects $(R, \xi_n, f_n)$ consists of an object $R \in \wh{\C}_\Lambda$, and objects $\xi_n \in \F(R/\m_R^n)$ and morphisms $f_n : \xi_{n+1} \to \xi_n$ over the projection $R / \m_R^{n+1} \to R / \m_R^n$

\item morphisms $a : (R, \xi_n, f_n) \to (S, \eta_n, g_n)$ consists of a map $a_0 : R \to S$ in $\wh{\C}_\Lambda$ and a collection $a_n : \xi_n \to \eta_n$ of morphisms in $\F$ lying over $R / \m_R^n \to S / \m_S^n$ such that the diagrams,
\begin{center}
\begin{tikzcd}
\xi_{n+1} \arrow[d, "a_{n+1}"] \arrow[r, "f_n"] & \xi_n \arrow[d, "a_n"]
\\
\eta_{n+1} \arrow[r, "g_n"] & \eta_n
\end{tikzcd}
\end{center} 
commute for each $n \in \N$.
\end{enumerate}
\end{defn}

\begin{prop}
The formal objects forms a category cofibered in groupoids $\hat{p} : \wh{\F} \to \wh{\C}_\Lambda$.
\end{prop}

\begin{defn}
Let $p : \F \to \C_\Lambda$ be a category cofibered in groupoids. We say that $\F$ satisfies the \textit{Rim-Schlessinger (RS) condition} if for all $A_1 \to A$ and $A_2 \to A$ in $\C_\Lambda$ with $A_2 \onto A$ surjective,
\[ \F(A_1 \times_A A_2) \to \F(A_1) \times_{\F(A)} \F(A_2) \]
is an equivalence. A \textit{deformation category} is a predeformation category $\F$ satisfying (RS).
\end{defn}

\begin{lemma}
The RS condition is equivalent to: for every diagram in $\F$,
\begin{center}
\begin{tikzcd}
& x_2 \arrow[d]
\\
x_1 \arrow[r] & x
\end{tikzcd}
\quad \text{lying over} \quad
\begin{tikzcd}
& A_2 \arrow[d]
\\
A_1 \arrow[r] & A
\end{tikzcd}
\end{center}
in $\C_\Lambda$ with $A_2 \to A$ surjective, there exists a fiber product $x_1 \times_x x_2$ in $\F$ such that the diagram,
\begin{center}
\begin{tikzcd}
x_1 \times_x x_2 \arrow[r] \arrow[d] & x_2 \arrow[d]
\\
x_1 \arrow[r] & x
\end{tikzcd}
\quad \text{lying over} \quad
\begin{tikzcd}
A_1 \times_A A_2 \arrow[r] \arrow[d] & A_2 \arrow[d]
\\
A_1 \arrow[r] & A
\end{tikzcd}
\end{center}
\end{lemma}

\begin{lemma}
Let $F : \C_\Lambda \to \Set$ be a predeformation functor then if the associated predeformation category $\F_F$ satsifies (RS) then $F$ satisfies (H1), (H2), and (H4).
\end{lemma}

\begin{rmk}
IS IT ACTUALLY EQUIVALENT e.g. https://stacks.math.columbia.edu/tag/06J1
\end{rmk}


\begin{lemma}
If $\X \to S$ is an algebraic stack then for any $\Spec{k} \to S$ and $x_0 \in \X(k)$ the deformation category $\F_{\X, k, x_0}$ satisfies (RS). 
\end{lemma}

\begin{rmk}
By Schlessinger's theorem, this is telling us that a deformation functor $F = D_{X,x_0}$ represented by some pointed algebraic space $x_0 \in X$ is pro-representable. So even though $X$ does not have a canonical local ring it does have a formal local ring $\wh{\stalk{X}{x_0}}$. We can calculate it from the formal local ring of any \etale cover $U \to X$. This is well-defined because for two \etale covers $U_1 \to X$ and $U_2 \to X$ we have $U_1 \times_X U_2$ is an \etale cover of both and these maps identify the formal local rings.
\end{rmk}

\subsection{Versality}

\begin{rmk}
A versal object is a universal object without the ``uni'' i.e. without the uniqueness. 
\end{rmk}

\begin{defn}
A morphism $\varphi : \F \to \G$ of categories cofibered in groupoids over $\C_\Lambda$ is \textit{smooth} if for every extension $B \onto A$ in $\C_{\Lambda}$ the map,
\[ \F(B) \to \F(A) \times_{\G(A)} \F(B) \]
is essentially surjective.
\end{defn}

\begin{rmk}
This is basically the formal lifting criterion for formal smoothness. Indeed, if these deformation categories are induced by the representable functors for a morphism of schemes $f : X \to Y$ then we get that,
\[ X(B) \to X(A) \times_{Y(A)} Y(B) \]
is surjective which is equivalent to there existing a dashed arrow in each lifting diagram,
\begin{center}
\begin{tikzcd}
\Spec{A} \arrow[d, hook] \arrow[r] & X \arrow[d, "f"]
\\
\Spec{B} \arrow[r]  \arrow[ru, dashed] & Y
\end{tikzcd}
\end{center} 
\end{rmk}

\begin{lemma}
Smoothness of $\varphi : \F \to \G$ is equivalent to the following explicit condition. For every surjection $B \onto A$ in $\C_{\Lambda}$ and $y \in \G(B)$ and $x \in \F(A)$ equipped with a map $y \to \varphi(x)$ over $B \onto A$ there is $x' \in \F(B)$ and a morphism $x' \to x$ over $B \onto A$ and a morphism $\varphi(x') \to y$ over $\id : B \to V$ such that,
\begin{center}
\begin{tikzcd}
\varphi(x') \arrow[r] \arrow[rd] & y \arrow[d]
\\
& \varphi(x)
\end{tikzcd}
\end{center}
\end{lemma}

\begin{defn}
Let $R \in \wh{\C}_{\Lambda}$. We say $\xi \in \wh{\F}(R)$ is \textit{versal} if the morphism $\xi : \underline{R}|_{\C_\Lambda} \to \F$ defined by $\xi$ is smooth.
\end{defn}

\begin{rmk}
The morphism is defined as follows. For any $A \in \C_{\Lambda}$ and map $\varphi : R \to A$ it will factor as $\varphi_n : R / \m^n \to A$ we send $(A, \varphi) \mapsto (\varphi_n)_* \xi_n$. The compatibility isomorphisms of the formal object $\xi$ make this well-defined. 
\end{rmk}

\begin{rmk}
Let $\xi$ be a formal object of $\F$. Versality of $\xi$ is equivalent to: the existence of a dashed arrow for any diagram,
\begin{center}
\begin{tikzcd}
& y \arrow[d]
\\
\xi \arrow[ru, dashed] \arrow[r] & x
\end{tikzcd}
\end{center}
in $\wh{\F}$ such that $y \to x$ lies over a surjective map $B \onto A$ of Artinian rings. 
\end{rmk}

\begin{theorem}[Rim-Schlessinger]
A deformation category $\F$ with $T \F = \overline{\F}(k[\epsilon])$ is finite dimensional admits a versal formal object.
\end{theorem}

(DO SOME EXAMPLES!!!)

\begin{defn}
Given a category fibered in groupoids,
\[ p : \X \to (\Sch_S)_{\fppf} \]
and 

https://stacks.math.columbia.edu/tag/07T2

(DEF DEF CAT AND VERSAL)
\end{defn}


\begin{defn}
Let $S$ be a locally noetherian scheme and $p : \X \to (\Sch_S)_{\fppf}$ a category fibered in groupoits. We say $\X$ satisfies \textit{opennes of versality} if given a scheme $U$ locally of finite type over $S$, an open $x \in \X(U)$, and a finite type point $u_0 \in Y$ such that $x$ is versal at $u_0$ then there is exists an open neiborhood $u_0 \in U' \subset U$ such that $x$ is versal at every finite type point of $U'$.
\end{defn}

(EXAMPLES)

\subsection{Effectivity}

\begin{defn}
We say a formal object $\xi = (R, \xi_n, f_n) \in \wh{\F}_{\X, k, x_0}$ is \textit{effective} if it arises from some $\tilde{\xi} \in \X(R)$.
\end{defn}

\begin{lemma}
If $\X$ is an algebraic stack then every formal object is effective.
\end{lemma}

\begin{proof}
First, if $X$ is a scheme then for all local rings $R$ factoring $\Spec{k} \to X$ the map corresponds to $\Spec{R} \to \Spec{\stalk{X}{x}} \to X$ so if $R$ is complete,
\[ X(R) = \Hom{\text{loc}}{\stalk{X}{x}}{R} = \varprojlim_{n} \Hom{\text{loc}}{\stalk{X}{x}}{R/\m_R^n} = \varprojlim_n X(R/\m_R^n) \]
Now in general, choose a smooth cover $\pi : U \to \X$ from a scheme. 

https://stacks.math.columbia.edu/tag/07X3
\end{proof}

\section{Artin's Axioms}

\begin{theorem}[Artin]
Let $S$ be a locally noetherian scheme and $p : \X \to (\Sch_S)_{\fppf}$ a category fibered in groupoids. Let $R$ be a Noetherian complete local ring with redicue field $k$ with $\Spec{R} \to S$ finite type and $x \in \X(R)$. Let $s \in S$ be the image of $\Spec{k} \to \Spec{R} \to S$. Assume that,
\begin{enumerate}
\item $\stalk{S}{s}$ is a $G$-ring
\item $p$ is limit-preserving on objects.
\end{enumerate}
Then for every $N \ge 1$ there exist,
\begin{enumerate}
\item a finite type $S$-algebra $A$
\item a maximal ideal $\m_A \subset A$
\item an object $x_A \in \X(A)$
\item an $S$-isomorphism $R / \m_R^N \iso A / \m_A^N$
\item an isomorphism $x|_{R/\m_R^N} \iso x_A |_{A / \m_A^N}$ over the previous map
\item an isomorphism $\gr{\m_R}{R} \iso \gr{\m_A}{A}$ of graded $k$-algebras.
\end{enumerate}
\end{theorem}

\begin{lemma}
Let $S$ be a locally noetherian scheme and $p : \X \to (\Sch_S)_{\fppf}$ a category fibered in groupoids. Let $\xi$ be a formal object of $\X$ with $x_0 = \xi_1$ lying over $\Spec{k} \to S$ with image $s \in S$ such that,
\begin{enumerate}
\item $\xi$ is versal
\item $\xi$ is effective
\item $\stalk{S}{s}$ is a $G$-ring 
\item $p : \X \to (\Sch_S)_{\fppf}$ is limit-preserving
\end{enumerate}
then there exists a finite type morphism $U \to S$, a finite type point $u_0 \in U$ with residue field $k$ and $x \in \X(U)$ such that $x : U \to \X$ is versal at $u_0$ and $x|_{\Spec{\stalk{U}{u_0}}}$ induces $\xi$. 
\end{lemma}

\begin{proof}
Choose an object $x_R \in \X(R)$ whose completion is $\xi$. Apply Artin approximation with $N = 2$ to obtain $A, \m_A, x_A \in \X(A)$ apprxoimating $\xi$. Let $\eta$ be the formal object completing $x_A |_{\Spec{\hat{A}}}$ (the completion of $A$ at $\m_A$). Then a lift for the diagram in $\wh{\F}_{\X, k, x_0}$,
\begin{center}
\begin{tikzcd}
& \eta \arrow[d]
\\
\xi \arrow[r] \arrow[ru, dashed] & \xi_2 = \eta_2 
\end{tikzcd}
\quad \text{lying over} \quad
\begin{tikzcd}
& \hat{A} \arrow[d]
\\
R \arrow[r] \arrow[ru, dashed] & R / \m_R^2 = A / \m_A^2
\end{tikzcd}
\end{center}
exists because $\xi$ is versal. Since the map $R \to \hat{A}$ induces an isomorphism on tangent spaces and by construction $\dim_k \m_R^n / \m_R^{n+1} = \dim_k \m_A^n / \m_A^{n+1}$ we conclude that $R \to \hat{A}$ is an isomorphism. Hence $\eta \cong \xi$ is versal so the map $x_A : \Spec{A} \to \X$ is versal at $\wh{x_A|_{\Spec{\hat{A}}}}  = \eta$. 
\end{proof}

\begin{theorem}
Let $S$ be a locally Noetherian base scheme and consider a category cofibered in groupoids,
\[ p : \X \to (\Sch_S)_{\fppf} \]
For each finite type morphism $\Spec{k} \to S$ with $k$ a field and $x_0 \in \X(\Spec{k})$  assume that,
\begin{enumerate}
\item $\X$ is a stack for the \etale topology
\item $\Delta_{\X / S} : \X \to \X \times_S \X$ is representable by algebraic spaces
\item $\X$ is limit preserving (preserves filtered colimits)
\item $\X$ satisfies the Rim-Schlessinger condition (RS)
\item $T \F_{\X, k, x_0}$ and $\Inf{\F_{\X, k, x_0}}$ are finite dimensional for all $k$ and all $x_0 \in \F(k)$
\item every formal object of $\X$ is effective
\item $\X$ satisfies opennes of versality
\item $\stalk{S}{s}$ is a $G$-ring for all finite type points $s \in S$ 
\item a set theoretic condition
\end{enumerate}
then $\X$ is an algebraic stack. 
\end{theorem}

\begin{proof}
It suffices to show that for each finite type $\Spec{k} \to S$ and $x_0 \in \X(k)$ there is a finite type morphism $U \to S$ and a smooth map $U \to \X$ such that there is a finite type point $u_0 : \Spec{k} \to U$ such that $x|_{u_0} \cong x_0$.
\bigskip\\
By Rim-Schelssinger $\F_{\X, k, x_0}$ admits a versal formal object $\xi$ which is then effective. Artin approximation allows us to approximate an effective formal object by a finite type object $U \to \X$ which is versal at $u_0 \in U$. By openness of versality, we can shrink $U$ such that $U \to \X$ is versal at every finite type point.  
\bigskip\\
Finally, prove that a representable morphism $f : \X \to \Y$ of limit preserving categories fibered in groupoids which is smooth on deformation categories is smooth (Tag \href{https://stacks.math.columbia.edu/tag/07XX}{07XX}. Indeed, for $T \to \Y$ the condition says that $f : \X_T \to T$ is a formally smooth map of algebraic spaces\footnote{There is a subtily here with changing fields that requires the full strength of (RS) where as proving that a versal object exists only requires (S1) and (S2) and finite-dimensionality of tangent spaces} and the limit-preserving condition gives finitely presented. 
\end{proof}

(WHERE NEED INF(X) bounded?)

(EXAMPLES)

\end{document}